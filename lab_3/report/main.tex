\documentclass[11pt,a4paper]{article}
\usepackage[top=1.2cm, bottom=1.8cm, left=1.8cm, right=1.8cm]{geometry}

\usepackage{float}
\usepackage{subfig}
\usepackage{graphicx}
\usepackage{xcolor}
\usepackage[utf8]{inputenc}
\usepackage{enumitem}
\usepackage{siunitx}
\usepackage[newfloat]{minted}
\usepackage{caption}
\usepackage{amsmath}
\usepackage{amsfonts,amssymb}
\usepackage[makeroom]{cancel}

% Declarations for tikz drawings
\usepackage{tikz}
\usepackage{pgfplots}
\usetikzlibrary{calc}
\definecolor{lightgreen}{HTML}{90EE90}
\newcommand*{\boxcolor}{lightgreen}
\makeatletter
\renewcommand{\boxed}[1]{\textcolor{\boxcolor}{%
\tikz[baseline={([yshift=-1ex]current bounding box.center)}] \node [rectangle, minimum width=5ex,rounded corners,draw,line width=0.25mm] {\normalcolor\m@th$\displaystyle#1$};}}
 \makeatother

 % Fix for symbol errors in code listings (see https://tex.stackexchange.com/a/343506)
 \usepackage{etoolbox,xpatch}
 \makeatletter
 \AtBeginEnvironment{minted}{\dontdofcolorbox}
 \def\dontdofcolorbox{\renewcommand\fcolorbox[4][]{##4}}
 \xpatchcmd{\inputminted}{\minted@fvset}{\minted@fvset\dontdofcolorbox}{}{}
 \xpatchcmd{\mintinline}{\minted@fvset}{\minted@fvset\dontdofcolorbox}{}{}
 \makeatother
 % Fix for distance of captions from listings
 \captionsetup[listing]{skip=-10pt}

% \usepackage[style=authoryear, backend=biber]{biblatex}
% \addbibresource{main.bib}
\DeclareMathOperator*{\argmin}{arg\,min}
\DeclareMathOperator*{\argmax}{arg\,max}

\title{COMP6248: Lab Exercise 3}
\author{
David Jones (dsj1n15@soton.ac.uk)}
\date{}
\setlength{\intextsep}{1mm}

\definecolor{mintedbackground}{rgb}{0.95,0.95,0.95}
\newmintedfile[pythoncode]{python}{
    bgcolor=mintedbackground,
    style=friendly,
    % fontfamily=fi4,
    fontsize=\small,
    linenos=true,
    numberblanklines=true,
    numbersep=5pt,
    gobble=0,
    frame=leftline,
    framerule=0.4pt,
    framesep=2mm,
    funcnamehighlighting=true,
    tabsize=4,
    obeytabs=false,
    mathescape=false
    samepage=false,
    showspaces=false,
    showtabs =false,
    texcl=false,
}

\newcommand{\norm}[1]{\left\lVert#1\right\rVert}

\begin{document}

\maketitle
\textbf{Task:} Optimisation
\vspace{-0.5em}
\section{Exercise 1}
\textbf{Exercise 1.1:} Optimising the Rastrigin function
\begin{figure}[H]
    \centering
    \begin{tabular}{cc}
    \subfloat[Loss Plot]{%% Creator: Matplotlib, PGF backend
%%
%% To include the figure in your LaTeX document, write
%%   \input{<filename>.pgf}
%%
%% Make sure the required packages are loaded in your preamble
%%   \usepackage{pgf}
%%
%% Figures using additional raster images can only be included by \input if
%% they are in the same directory as the main LaTeX file. For loading figures
%% from other directories you can use the `import` package
%%   \usepackage{import}
%% and then include the figures with
%%   \import{<path to file>}{<filename>.pgf}
%%
%% Matplotlib used the following preamble
%%
\begingroup%
\makeatletter%
\begin{pgfpicture}%
\pgfpathrectangle{\pgfpointorigin}{\pgfqpoint{3.750000in}{2.500000in}}%
\pgfusepath{use as bounding box, clip}%
\begin{pgfscope}%
\pgfsetbuttcap%
\pgfsetmiterjoin%
\definecolor{currentfill}{rgb}{1.000000,1.000000,1.000000}%
\pgfsetfillcolor{currentfill}%
\pgfsetlinewidth{0.000000pt}%
\definecolor{currentstroke}{rgb}{1.000000,1.000000,1.000000}%
\pgfsetstrokecolor{currentstroke}%
\pgfsetdash{}{0pt}%
\pgfpathmoveto{\pgfqpoint{0.000000in}{0.000000in}}%
\pgfpathlineto{\pgfqpoint{3.750000in}{0.000000in}}%
\pgfpathlineto{\pgfqpoint{3.750000in}{2.500000in}}%
\pgfpathlineto{\pgfqpoint{0.000000in}{2.500000in}}%
\pgfpathclose%
\pgfusepath{fill}%
\end{pgfscope}%
\begin{pgfscope}%
\pgfsetbuttcap%
\pgfsetmiterjoin%
\definecolor{currentfill}{rgb}{1.000000,1.000000,1.000000}%
\pgfsetfillcolor{currentfill}%
\pgfsetlinewidth{0.000000pt}%
\definecolor{currentstroke}{rgb}{0.000000,0.000000,0.000000}%
\pgfsetstrokecolor{currentstroke}%
\pgfsetstrokeopacity{0.000000}%
\pgfsetdash{}{0pt}%
\pgfpathmoveto{\pgfqpoint{0.580556in}{0.549691in}}%
\pgfpathlineto{\pgfqpoint{3.600000in}{0.549691in}}%
\pgfpathlineto{\pgfqpoint{3.600000in}{2.350000in}}%
\pgfpathlineto{\pgfqpoint{0.580556in}{2.350000in}}%
\pgfpathclose%
\pgfusepath{fill}%
\end{pgfscope}%
\begin{pgfscope}%
\pgfsetbuttcap%
\pgfsetroundjoin%
\definecolor{currentfill}{rgb}{0.000000,0.000000,0.000000}%
\pgfsetfillcolor{currentfill}%
\pgfsetlinewidth{0.803000pt}%
\definecolor{currentstroke}{rgb}{0.000000,0.000000,0.000000}%
\pgfsetstrokecolor{currentstroke}%
\pgfsetdash{}{0pt}%
\pgfsys@defobject{currentmarker}{\pgfqpoint{0.000000in}{-0.048611in}}{\pgfqpoint{0.000000in}{0.000000in}}{%
\pgfpathmoveto{\pgfqpoint{0.000000in}{0.000000in}}%
\pgfpathlineto{\pgfqpoint{0.000000in}{-0.048611in}}%
\pgfusepath{stroke,fill}%
}%
\begin{pgfscope}%
\pgfsys@transformshift{0.717803in}{0.549691in}%
\pgfsys@useobject{currentmarker}{}%
\end{pgfscope}%
\end{pgfscope}%
\begin{pgfscope}%
\definecolor{textcolor}{rgb}{0.000000,0.000000,0.000000}%
\pgfsetstrokecolor{textcolor}%
\pgfsetfillcolor{textcolor}%
\pgftext[x=0.717803in,y=0.452469in,,top]{\color{textcolor}\rmfamily\fontsize{10.000000}{12.000000}\selectfont \(\displaystyle 0\)}%
\end{pgfscope}%
\begin{pgfscope}%
\pgfsetbuttcap%
\pgfsetroundjoin%
\definecolor{currentfill}{rgb}{0.000000,0.000000,0.000000}%
\pgfsetfillcolor{currentfill}%
\pgfsetlinewidth{0.803000pt}%
\definecolor{currentstroke}{rgb}{0.000000,0.000000,0.000000}%
\pgfsetstrokecolor{currentstroke}%
\pgfsetdash{}{0pt}%
\pgfsys@defobject{currentmarker}{\pgfqpoint{0.000000in}{-0.048611in}}{\pgfqpoint{0.000000in}{0.000000in}}{%
\pgfpathmoveto{\pgfqpoint{0.000000in}{0.000000in}}%
\pgfpathlineto{\pgfqpoint{0.000000in}{-0.048611in}}%
\pgfusepath{stroke,fill}%
}%
\begin{pgfscope}%
\pgfsys@transformshift{1.266793in}{0.549691in}%
\pgfsys@useobject{currentmarker}{}%
\end{pgfscope}%
\end{pgfscope}%
\begin{pgfscope}%
\definecolor{textcolor}{rgb}{0.000000,0.000000,0.000000}%
\pgfsetstrokecolor{textcolor}%
\pgfsetfillcolor{textcolor}%
\pgftext[x=1.266793in,y=0.452469in,,top]{\color{textcolor}\rmfamily\fontsize{10.000000}{12.000000}\selectfont \(\displaystyle 20\)}%
\end{pgfscope}%
\begin{pgfscope}%
\pgfsetbuttcap%
\pgfsetroundjoin%
\definecolor{currentfill}{rgb}{0.000000,0.000000,0.000000}%
\pgfsetfillcolor{currentfill}%
\pgfsetlinewidth{0.803000pt}%
\definecolor{currentstroke}{rgb}{0.000000,0.000000,0.000000}%
\pgfsetstrokecolor{currentstroke}%
\pgfsetdash{}{0pt}%
\pgfsys@defobject{currentmarker}{\pgfqpoint{0.000000in}{-0.048611in}}{\pgfqpoint{0.000000in}{0.000000in}}{%
\pgfpathmoveto{\pgfqpoint{0.000000in}{0.000000in}}%
\pgfpathlineto{\pgfqpoint{0.000000in}{-0.048611in}}%
\pgfusepath{stroke,fill}%
}%
\begin{pgfscope}%
\pgfsys@transformshift{1.815783in}{0.549691in}%
\pgfsys@useobject{currentmarker}{}%
\end{pgfscope}%
\end{pgfscope}%
\begin{pgfscope}%
\definecolor{textcolor}{rgb}{0.000000,0.000000,0.000000}%
\pgfsetstrokecolor{textcolor}%
\pgfsetfillcolor{textcolor}%
\pgftext[x=1.815783in,y=0.452469in,,top]{\color{textcolor}\rmfamily\fontsize{10.000000}{12.000000}\selectfont \(\displaystyle 40\)}%
\end{pgfscope}%
\begin{pgfscope}%
\pgfsetbuttcap%
\pgfsetroundjoin%
\definecolor{currentfill}{rgb}{0.000000,0.000000,0.000000}%
\pgfsetfillcolor{currentfill}%
\pgfsetlinewidth{0.803000pt}%
\definecolor{currentstroke}{rgb}{0.000000,0.000000,0.000000}%
\pgfsetstrokecolor{currentstroke}%
\pgfsetdash{}{0pt}%
\pgfsys@defobject{currentmarker}{\pgfqpoint{0.000000in}{-0.048611in}}{\pgfqpoint{0.000000in}{0.000000in}}{%
\pgfpathmoveto{\pgfqpoint{0.000000in}{0.000000in}}%
\pgfpathlineto{\pgfqpoint{0.000000in}{-0.048611in}}%
\pgfusepath{stroke,fill}%
}%
\begin{pgfscope}%
\pgfsys@transformshift{2.364773in}{0.549691in}%
\pgfsys@useobject{currentmarker}{}%
\end{pgfscope}%
\end{pgfscope}%
\begin{pgfscope}%
\definecolor{textcolor}{rgb}{0.000000,0.000000,0.000000}%
\pgfsetstrokecolor{textcolor}%
\pgfsetfillcolor{textcolor}%
\pgftext[x=2.364773in,y=0.452469in,,top]{\color{textcolor}\rmfamily\fontsize{10.000000}{12.000000}\selectfont \(\displaystyle 60\)}%
\end{pgfscope}%
\begin{pgfscope}%
\pgfsetbuttcap%
\pgfsetroundjoin%
\definecolor{currentfill}{rgb}{0.000000,0.000000,0.000000}%
\pgfsetfillcolor{currentfill}%
\pgfsetlinewidth{0.803000pt}%
\definecolor{currentstroke}{rgb}{0.000000,0.000000,0.000000}%
\pgfsetstrokecolor{currentstroke}%
\pgfsetdash{}{0pt}%
\pgfsys@defobject{currentmarker}{\pgfqpoint{0.000000in}{-0.048611in}}{\pgfqpoint{0.000000in}{0.000000in}}{%
\pgfpathmoveto{\pgfqpoint{0.000000in}{0.000000in}}%
\pgfpathlineto{\pgfqpoint{0.000000in}{-0.048611in}}%
\pgfusepath{stroke,fill}%
}%
\begin{pgfscope}%
\pgfsys@transformshift{2.913763in}{0.549691in}%
\pgfsys@useobject{currentmarker}{}%
\end{pgfscope}%
\end{pgfscope}%
\begin{pgfscope}%
\definecolor{textcolor}{rgb}{0.000000,0.000000,0.000000}%
\pgfsetstrokecolor{textcolor}%
\pgfsetfillcolor{textcolor}%
\pgftext[x=2.913763in,y=0.452469in,,top]{\color{textcolor}\rmfamily\fontsize{10.000000}{12.000000}\selectfont \(\displaystyle 80\)}%
\end{pgfscope}%
\begin{pgfscope}%
\pgfsetbuttcap%
\pgfsetroundjoin%
\definecolor{currentfill}{rgb}{0.000000,0.000000,0.000000}%
\pgfsetfillcolor{currentfill}%
\pgfsetlinewidth{0.803000pt}%
\definecolor{currentstroke}{rgb}{0.000000,0.000000,0.000000}%
\pgfsetstrokecolor{currentstroke}%
\pgfsetdash{}{0pt}%
\pgfsys@defobject{currentmarker}{\pgfqpoint{0.000000in}{-0.048611in}}{\pgfqpoint{0.000000in}{0.000000in}}{%
\pgfpathmoveto{\pgfqpoint{0.000000in}{0.000000in}}%
\pgfpathlineto{\pgfqpoint{0.000000in}{-0.048611in}}%
\pgfusepath{stroke,fill}%
}%
\begin{pgfscope}%
\pgfsys@transformshift{3.462753in}{0.549691in}%
\pgfsys@useobject{currentmarker}{}%
\end{pgfscope}%
\end{pgfscope}%
\begin{pgfscope}%
\definecolor{textcolor}{rgb}{0.000000,0.000000,0.000000}%
\pgfsetstrokecolor{textcolor}%
\pgfsetfillcolor{textcolor}%
\pgftext[x=3.462753in,y=0.452469in,,top]{\color{textcolor}\rmfamily\fontsize{10.000000}{12.000000}\selectfont \(\displaystyle 100\)}%
\end{pgfscope}%
\begin{pgfscope}%
\definecolor{textcolor}{rgb}{0.000000,0.000000,0.000000}%
\pgfsetstrokecolor{textcolor}%
\pgfsetfillcolor{textcolor}%
\pgftext[x=2.090278in,y=0.273457in,,top]{\color{textcolor}\rmfamily\fontsize{10.000000}{12.000000}\selectfont epoch}%
\end{pgfscope}%
\begin{pgfscope}%
\pgfsetbuttcap%
\pgfsetroundjoin%
\definecolor{currentfill}{rgb}{0.000000,0.000000,0.000000}%
\pgfsetfillcolor{currentfill}%
\pgfsetlinewidth{0.803000pt}%
\definecolor{currentstroke}{rgb}{0.000000,0.000000,0.000000}%
\pgfsetstrokecolor{currentstroke}%
\pgfsetdash{}{0pt}%
\pgfsys@defobject{currentmarker}{\pgfqpoint{-0.048611in}{0.000000in}}{\pgfqpoint{0.000000in}{0.000000in}}{%
\pgfpathmoveto{\pgfqpoint{0.000000in}{0.000000in}}%
\pgfpathlineto{\pgfqpoint{-0.048611in}{0.000000in}}%
\pgfusepath{stroke,fill}%
}%
\begin{pgfscope}%
\pgfsys@transformshift{0.580556in}{0.628900in}%
\pgfsys@useobject{currentmarker}{}%
\end{pgfscope}%
\end{pgfscope}%
\begin{pgfscope}%
\definecolor{textcolor}{rgb}{0.000000,0.000000,0.000000}%
\pgfsetstrokecolor{textcolor}%
\pgfsetfillcolor{textcolor}%
\pgftext[x=0.413889in,y=0.580675in,left,base]{\color{textcolor}\rmfamily\fontsize{10.000000}{12.000000}\selectfont \(\displaystyle 0\)}%
\end{pgfscope}%
\begin{pgfscope}%
\pgfsetbuttcap%
\pgfsetroundjoin%
\definecolor{currentfill}{rgb}{0.000000,0.000000,0.000000}%
\pgfsetfillcolor{currentfill}%
\pgfsetlinewidth{0.803000pt}%
\definecolor{currentstroke}{rgb}{0.000000,0.000000,0.000000}%
\pgfsetstrokecolor{currentstroke}%
\pgfsetdash{}{0pt}%
\pgfsys@defobject{currentmarker}{\pgfqpoint{-0.048611in}{0.000000in}}{\pgfqpoint{0.000000in}{0.000000in}}{%
\pgfpathmoveto{\pgfqpoint{0.000000in}{0.000000in}}%
\pgfpathlineto{\pgfqpoint{-0.048611in}{0.000000in}}%
\pgfusepath{stroke,fill}%
}%
\begin{pgfscope}%
\pgfsys@transformshift{0.580556in}{0.956753in}%
\pgfsys@useobject{currentmarker}{}%
\end{pgfscope}%
\end{pgfscope}%
\begin{pgfscope}%
\definecolor{textcolor}{rgb}{0.000000,0.000000,0.000000}%
\pgfsetstrokecolor{textcolor}%
\pgfsetfillcolor{textcolor}%
\pgftext[x=0.344444in,y=0.908528in,left,base]{\color{textcolor}\rmfamily\fontsize{10.000000}{12.000000}\selectfont \(\displaystyle 10\)}%
\end{pgfscope}%
\begin{pgfscope}%
\pgfsetbuttcap%
\pgfsetroundjoin%
\definecolor{currentfill}{rgb}{0.000000,0.000000,0.000000}%
\pgfsetfillcolor{currentfill}%
\pgfsetlinewidth{0.803000pt}%
\definecolor{currentstroke}{rgb}{0.000000,0.000000,0.000000}%
\pgfsetstrokecolor{currentstroke}%
\pgfsetdash{}{0pt}%
\pgfsys@defobject{currentmarker}{\pgfqpoint{-0.048611in}{0.000000in}}{\pgfqpoint{0.000000in}{0.000000in}}{%
\pgfpathmoveto{\pgfqpoint{0.000000in}{0.000000in}}%
\pgfpathlineto{\pgfqpoint{-0.048611in}{0.000000in}}%
\pgfusepath{stroke,fill}%
}%
\begin{pgfscope}%
\pgfsys@transformshift{0.580556in}{1.284607in}%
\pgfsys@useobject{currentmarker}{}%
\end{pgfscope}%
\end{pgfscope}%
\begin{pgfscope}%
\definecolor{textcolor}{rgb}{0.000000,0.000000,0.000000}%
\pgfsetstrokecolor{textcolor}%
\pgfsetfillcolor{textcolor}%
\pgftext[x=0.344444in,y=1.236382in,left,base]{\color{textcolor}\rmfamily\fontsize{10.000000}{12.000000}\selectfont \(\displaystyle 20\)}%
\end{pgfscope}%
\begin{pgfscope}%
\pgfsetbuttcap%
\pgfsetroundjoin%
\definecolor{currentfill}{rgb}{0.000000,0.000000,0.000000}%
\pgfsetfillcolor{currentfill}%
\pgfsetlinewidth{0.803000pt}%
\definecolor{currentstroke}{rgb}{0.000000,0.000000,0.000000}%
\pgfsetstrokecolor{currentstroke}%
\pgfsetdash{}{0pt}%
\pgfsys@defobject{currentmarker}{\pgfqpoint{-0.048611in}{0.000000in}}{\pgfqpoint{0.000000in}{0.000000in}}{%
\pgfpathmoveto{\pgfqpoint{0.000000in}{0.000000in}}%
\pgfpathlineto{\pgfqpoint{-0.048611in}{0.000000in}}%
\pgfusepath{stroke,fill}%
}%
\begin{pgfscope}%
\pgfsys@transformshift{0.580556in}{1.612461in}%
\pgfsys@useobject{currentmarker}{}%
\end{pgfscope}%
\end{pgfscope}%
\begin{pgfscope}%
\definecolor{textcolor}{rgb}{0.000000,0.000000,0.000000}%
\pgfsetstrokecolor{textcolor}%
\pgfsetfillcolor{textcolor}%
\pgftext[x=0.344444in,y=1.564235in,left,base]{\color{textcolor}\rmfamily\fontsize{10.000000}{12.000000}\selectfont \(\displaystyle 30\)}%
\end{pgfscope}%
\begin{pgfscope}%
\pgfsetbuttcap%
\pgfsetroundjoin%
\definecolor{currentfill}{rgb}{0.000000,0.000000,0.000000}%
\pgfsetfillcolor{currentfill}%
\pgfsetlinewidth{0.803000pt}%
\definecolor{currentstroke}{rgb}{0.000000,0.000000,0.000000}%
\pgfsetstrokecolor{currentstroke}%
\pgfsetdash{}{0pt}%
\pgfsys@defobject{currentmarker}{\pgfqpoint{-0.048611in}{0.000000in}}{\pgfqpoint{0.000000in}{0.000000in}}{%
\pgfpathmoveto{\pgfqpoint{0.000000in}{0.000000in}}%
\pgfpathlineto{\pgfqpoint{-0.048611in}{0.000000in}}%
\pgfusepath{stroke,fill}%
}%
\begin{pgfscope}%
\pgfsys@transformshift{0.580556in}{1.940314in}%
\pgfsys@useobject{currentmarker}{}%
\end{pgfscope}%
\end{pgfscope}%
\begin{pgfscope}%
\definecolor{textcolor}{rgb}{0.000000,0.000000,0.000000}%
\pgfsetstrokecolor{textcolor}%
\pgfsetfillcolor{textcolor}%
\pgftext[x=0.344444in,y=1.892089in,left,base]{\color{textcolor}\rmfamily\fontsize{10.000000}{12.000000}\selectfont \(\displaystyle 40\)}%
\end{pgfscope}%
\begin{pgfscope}%
\pgfsetbuttcap%
\pgfsetroundjoin%
\definecolor{currentfill}{rgb}{0.000000,0.000000,0.000000}%
\pgfsetfillcolor{currentfill}%
\pgfsetlinewidth{0.803000pt}%
\definecolor{currentstroke}{rgb}{0.000000,0.000000,0.000000}%
\pgfsetstrokecolor{currentstroke}%
\pgfsetdash{}{0pt}%
\pgfsys@defobject{currentmarker}{\pgfqpoint{-0.048611in}{0.000000in}}{\pgfqpoint{0.000000in}{0.000000in}}{%
\pgfpathmoveto{\pgfqpoint{0.000000in}{0.000000in}}%
\pgfpathlineto{\pgfqpoint{-0.048611in}{0.000000in}}%
\pgfusepath{stroke,fill}%
}%
\begin{pgfscope}%
\pgfsys@transformshift{0.580556in}{2.268168in}%
\pgfsys@useobject{currentmarker}{}%
\end{pgfscope}%
\end{pgfscope}%
\begin{pgfscope}%
\definecolor{textcolor}{rgb}{0.000000,0.000000,0.000000}%
\pgfsetstrokecolor{textcolor}%
\pgfsetfillcolor{textcolor}%
\pgftext[x=0.344444in,y=2.219943in,left,base]{\color{textcolor}\rmfamily\fontsize{10.000000}{12.000000}\selectfont \(\displaystyle 50\)}%
\end{pgfscope}%
\begin{pgfscope}%
\definecolor{textcolor}{rgb}{0.000000,0.000000,0.000000}%
\pgfsetstrokecolor{textcolor}%
\pgfsetfillcolor{textcolor}%
\pgftext[x=0.288889in,y=1.449846in,,bottom,rotate=90.000000]{\color{textcolor}\rmfamily\fontsize{10.000000}{12.000000}\selectfont \(\displaystyle f(x, y)\)}%
\end{pgfscope}%
\begin{pgfscope}%
\pgfpathrectangle{\pgfqpoint{0.580556in}{0.549691in}}{\pgfqpoint{3.019444in}{1.800309in}}%
\pgfusepath{clip}%
\pgfsetrectcap%
\pgfsetroundjoin%
\pgfsetlinewidth{1.505625pt}%
\definecolor{currentstroke}{rgb}{1.000000,0.498039,0.054902}%
\pgfsetstrokecolor{currentstroke}%
\pgfsetdash{}{0pt}%
\pgfpathmoveto{\pgfqpoint{0.717803in}{2.268168in}}%
\pgfpathlineto{\pgfqpoint{0.745253in}{2.215776in}}%
\pgfpathlineto{\pgfqpoint{0.772702in}{2.173684in}}%
\pgfpathlineto{\pgfqpoint{0.800152in}{2.127698in}}%
\pgfpathlineto{\pgfqpoint{0.827601in}{1.990207in}}%
\pgfpathlineto{\pgfqpoint{0.855051in}{1.703567in}}%
\pgfpathlineto{\pgfqpoint{0.882500in}{1.607089in}}%
\pgfpathlineto{\pgfqpoint{0.909950in}{1.413031in}}%
\pgfpathlineto{\pgfqpoint{0.937399in}{1.188003in}}%
\pgfpathlineto{\pgfqpoint{0.964849in}{1.126783in}}%
\pgfpathlineto{\pgfqpoint{0.992298in}{0.880803in}}%
\pgfpathlineto{\pgfqpoint{1.019748in}{0.910242in}}%
\pgfpathlineto{\pgfqpoint{1.047197in}{0.722216in}}%
\pgfpathlineto{\pgfqpoint{1.074647in}{0.754551in}}%
\pgfpathlineto{\pgfqpoint{1.102096in}{0.732552in}}%
\pgfpathlineto{\pgfqpoint{1.129546in}{0.631523in}}%
\pgfpathlineto{\pgfqpoint{1.156995in}{0.745207in}}%
\pgfpathlineto{\pgfqpoint{1.184445in}{0.772467in}}%
\pgfpathlineto{\pgfqpoint{1.211894in}{0.710453in}}%
\pgfpathlineto{\pgfqpoint{1.239344in}{0.709422in}}%
\pgfpathlineto{\pgfqpoint{1.266793in}{0.792532in}}%
\pgfpathlineto{\pgfqpoint{1.294243in}{0.832330in}}%
\pgfpathlineto{\pgfqpoint{1.321692in}{0.820251in}}%
\pgfpathlineto{\pgfqpoint{1.349142in}{0.760316in}}%
\pgfpathlineto{\pgfqpoint{1.376591in}{0.695600in}}%
\pgfpathlineto{\pgfqpoint{1.404041in}{0.709166in}}%
\pgfpathlineto{\pgfqpoint{1.431490in}{0.746589in}}%
\pgfpathlineto{\pgfqpoint{1.458940in}{0.764385in}}%
\pgfpathlineto{\pgfqpoint{1.486389in}{0.769566in}}%
\pgfpathlineto{\pgfqpoint{1.513839in}{0.768702in}}%
\pgfpathlineto{\pgfqpoint{1.541288in}{0.761502in}}%
\pgfpathlineto{\pgfqpoint{1.568738in}{0.742687in}}%
\pgfpathlineto{\pgfqpoint{1.596187in}{0.710404in}}%
\pgfpathlineto{\pgfqpoint{1.623637in}{0.691342in}}%
\pgfpathlineto{\pgfqpoint{1.651086in}{0.712839in}}%
\pgfpathlineto{\pgfqpoint{1.678536in}{0.735715in}}%
\pgfpathlineto{\pgfqpoint{1.705985in}{0.727602in}}%
\pgfpathlineto{\pgfqpoint{1.733435in}{0.700920in}}%
\pgfpathlineto{\pgfqpoint{1.760884in}{0.691952in}}%
\pgfpathlineto{\pgfqpoint{1.788334in}{0.705573in}}%
\pgfpathlineto{\pgfqpoint{1.815783in}{0.717016in}}%
\pgfpathlineto{\pgfqpoint{1.843233in}{0.715462in}}%
\pgfpathlineto{\pgfqpoint{1.870682in}{0.703136in}}%
\pgfpathlineto{\pgfqpoint{1.898132in}{0.691974in}}%
\pgfpathlineto{\pgfqpoint{1.925581in}{0.695333in}}%
\pgfpathlineto{\pgfqpoint{1.953031in}{0.705515in}}%
\pgfpathlineto{\pgfqpoint{1.980480in}{0.705822in}}%
\pgfpathlineto{\pgfqpoint{2.007930in}{0.696531in}}%
\pgfpathlineto{\pgfqpoint{2.035379in}{0.691285in}}%
\pgfpathlineto{\pgfqpoint{2.062828in}{0.695302in}}%
\pgfpathlineto{\pgfqpoint{2.090278in}{0.700348in}}%
\pgfpathlineto{\pgfqpoint{2.117727in}{0.699754in}}%
\pgfpathlineto{\pgfqpoint{2.145177in}{0.694646in}}%
\pgfpathlineto{\pgfqpoint{2.172626in}{0.691305in}}%
\pgfpathlineto{\pgfqpoint{2.200076in}{0.693495in}}%
\pgfpathlineto{\pgfqpoint{2.227525in}{0.696834in}}%
\pgfpathlineto{\pgfqpoint{2.254975in}{0.695978in}}%
\pgfpathlineto{\pgfqpoint{2.282424in}{0.692505in}}%
\pgfpathlineto{\pgfqpoint{2.309874in}{0.691367in}}%
\pgfpathlineto{\pgfqpoint{2.337323in}{0.693238in}}%
\pgfpathlineto{\pgfqpoint{2.364773in}{0.694748in}}%
\pgfpathlineto{\pgfqpoint{2.392222in}{0.693857in}}%
\pgfpathlineto{\pgfqpoint{2.419672in}{0.691884in}}%
\pgfpathlineto{\pgfqpoint{2.447121in}{0.691370in}}%
\pgfpathlineto{\pgfqpoint{2.474571in}{0.692598in}}%
\pgfpathlineto{\pgfqpoint{2.502020in}{0.693393in}}%
\pgfpathlineto{\pgfqpoint{2.529470in}{0.692567in}}%
\pgfpathlineto{\pgfqpoint{2.556919in}{0.691434in}}%
\pgfpathlineto{\pgfqpoint{2.584369in}{0.691472in}}%
\pgfpathlineto{\pgfqpoint{2.611818in}{0.692266in}}%
\pgfpathlineto{\pgfqpoint{2.639268in}{0.692525in}}%
\pgfpathlineto{\pgfqpoint{2.666717in}{0.691930in}}%
\pgfpathlineto{\pgfqpoint{2.694167in}{0.691330in}}%
\pgfpathlineto{\pgfqpoint{2.721616in}{0.691455in}}%
\pgfpathlineto{\pgfqpoint{2.749066in}{0.691943in}}%
\pgfpathlineto{\pgfqpoint{2.776515in}{0.691994in}}%
\pgfpathlineto{\pgfqpoint{2.803965in}{0.691565in}}%
\pgfpathlineto{\pgfqpoint{2.831414in}{0.691287in}}%
\pgfpathlineto{\pgfqpoint{2.858864in}{0.691460in}}%
\pgfpathlineto{\pgfqpoint{2.886313in}{0.691722in}}%
\pgfpathlineto{\pgfqpoint{2.913763in}{0.691675in}}%
\pgfpathlineto{\pgfqpoint{2.941212in}{0.691407in}}%
\pgfpathlineto{\pgfqpoint{2.968662in}{0.691287in}}%
\pgfpathlineto{\pgfqpoint{2.996111in}{0.691424in}}%
\pgfpathlineto{\pgfqpoint{3.023561in}{0.691561in}}%
\pgfpathlineto{\pgfqpoint{3.051010in}{0.691489in}}%
\pgfpathlineto{\pgfqpoint{3.078460in}{0.691327in}}%
\pgfpathlineto{\pgfqpoint{3.105909in}{0.691295in}}%
\pgfpathlineto{\pgfqpoint{3.133359in}{0.691395in}}%
\pgfpathlineto{\pgfqpoint{3.160808in}{0.691454in}}%
\pgfpathlineto{\pgfqpoint{3.188258in}{0.691388in}}%
\pgfpathlineto{\pgfqpoint{3.215707in}{0.691298in}}%
\pgfpathlineto{\pgfqpoint{3.243157in}{0.691299in}}%
\pgfpathlineto{\pgfqpoint{3.270606in}{0.691363in}}%
\pgfpathlineto{\pgfqpoint{3.298056in}{0.691384in}}%
\pgfpathlineto{\pgfqpoint{3.325505in}{0.691334in}}%
\pgfpathlineto{\pgfqpoint{3.352955in}{0.691287in}}%
\pgfpathlineto{\pgfqpoint{3.380404in}{0.691301in}}%
\pgfpathlineto{\pgfqpoint{3.407854in}{0.691339in}}%
\pgfpathlineto{\pgfqpoint{3.435303in}{0.691341in}}%
\pgfpathlineto{\pgfqpoint{3.462753in}{0.691307in}}%
\pgfusepath{stroke}%
\end{pgfscope}%
\begin{pgfscope}%
\pgfpathrectangle{\pgfqpoint{0.580556in}{0.549691in}}{\pgfqpoint{3.019444in}{1.800309in}}%
\pgfusepath{clip}%
\pgfsetrectcap%
\pgfsetroundjoin%
\pgfsetlinewidth{1.505625pt}%
\definecolor{currentstroke}{rgb}{0.839216,0.152941,0.156863}%
\pgfsetstrokecolor{currentstroke}%
\pgfsetdash{}{0pt}%
\pgfpathmoveto{\pgfqpoint{0.717803in}{2.268168in}}%
\pgfpathlineto{\pgfqpoint{0.745253in}{2.215776in}}%
\pgfpathlineto{\pgfqpoint{0.772702in}{2.195064in}}%
\pgfpathlineto{\pgfqpoint{0.800152in}{2.183880in}}%
\pgfpathlineto{\pgfqpoint{0.827601in}{2.175988in}}%
\pgfpathlineto{\pgfqpoint{0.855051in}{2.169129in}}%
\pgfpathlineto{\pgfqpoint{0.882500in}{2.162053in}}%
\pgfpathlineto{\pgfqpoint{0.909950in}{2.153562in}}%
\pgfpathlineto{\pgfqpoint{0.937399in}{2.141862in}}%
\pgfpathlineto{\pgfqpoint{0.964849in}{2.123538in}}%
\pgfpathlineto{\pgfqpoint{0.992298in}{2.091516in}}%
\pgfpathlineto{\pgfqpoint{1.019748in}{2.031901in}}%
\pgfpathlineto{\pgfqpoint{1.047197in}{1.926709in}}%
\pgfpathlineto{\pgfqpoint{1.074647in}{1.788524in}}%
\pgfpathlineto{\pgfqpoint{1.102096in}{1.688482in}}%
\pgfpathlineto{\pgfqpoint{1.129546in}{1.648259in}}%
\pgfpathlineto{\pgfqpoint{1.156995in}{1.633541in}}%
\pgfpathlineto{\pgfqpoint{1.184445in}{1.626951in}}%
\pgfpathlineto{\pgfqpoint{1.211894in}{1.623358in}}%
\pgfpathlineto{\pgfqpoint{1.239344in}{1.621078in}}%
\pgfpathlineto{\pgfqpoint{1.266793in}{1.619453in}}%
\pgfpathlineto{\pgfqpoint{1.294243in}{1.618178in}}%
\pgfpathlineto{\pgfqpoint{1.321692in}{1.617095in}}%
\pgfpathlineto{\pgfqpoint{1.349142in}{1.616107in}}%
\pgfpathlineto{\pgfqpoint{1.376591in}{1.615146in}}%
\pgfpathlineto{\pgfqpoint{1.404041in}{1.614152in}}%
\pgfpathlineto{\pgfqpoint{1.431490in}{1.613064in}}%
\pgfpathlineto{\pgfqpoint{1.458940in}{1.611801in}}%
\pgfpathlineto{\pgfqpoint{1.486389in}{1.610247in}}%
\pgfpathlineto{\pgfqpoint{1.513839in}{1.608217in}}%
\pgfpathlineto{\pgfqpoint{1.541288in}{1.605393in}}%
\pgfpathlineto{\pgfqpoint{1.568738in}{1.601195in}}%
\pgfpathlineto{\pgfqpoint{1.596187in}{1.594512in}}%
\pgfpathlineto{\pgfqpoint{1.623637in}{1.583105in}}%
\pgfpathlineto{\pgfqpoint{1.651086in}{1.562401in}}%
\pgfpathlineto{\pgfqpoint{1.678536in}{1.523584in}}%
\pgfpathlineto{\pgfqpoint{1.705985in}{1.454074in}}%
\pgfpathlineto{\pgfqpoint{1.733435in}{1.353544in}}%
\pgfpathlineto{\pgfqpoint{1.760884in}{1.261870in}}%
\pgfpathlineto{\pgfqpoint{1.788334in}{1.215641in}}%
\pgfpathlineto{\pgfqpoint{1.815783in}{1.199086in}}%
\pgfpathlineto{\pgfqpoint{1.843233in}{1.193029in}}%
\pgfpathlineto{\pgfqpoint{1.870682in}{1.190510in}}%
\pgfpathlineto{\pgfqpoint{1.898132in}{1.189339in}}%
\pgfpathlineto{\pgfqpoint{1.925581in}{1.188748in}}%
\pgfpathlineto{\pgfqpoint{1.953031in}{1.188432in}}%
\pgfpathlineto{\pgfqpoint{1.980480in}{1.188255in}}%
\pgfpathlineto{\pgfqpoint{2.007930in}{1.188152in}}%
\pgfpathlineto{\pgfqpoint{2.035379in}{1.188091in}}%
\pgfpathlineto{\pgfqpoint{2.062828in}{1.188054in}}%
\pgfpathlineto{\pgfqpoint{2.090278in}{1.188031in}}%
\pgfpathlineto{\pgfqpoint{2.117727in}{1.188017in}}%
\pgfpathlineto{\pgfqpoint{2.145177in}{1.188008in}}%
\pgfpathlineto{\pgfqpoint{2.172626in}{1.188003in}}%
\pgfpathlineto{\pgfqpoint{2.200076in}{1.187999in}}%
\pgfpathlineto{\pgfqpoint{2.227525in}{1.187997in}}%
\pgfpathlineto{\pgfqpoint{2.254975in}{1.187995in}}%
\pgfpathlineto{\pgfqpoint{2.282424in}{1.187995in}}%
\pgfpathlineto{\pgfqpoint{2.309874in}{1.187994in}}%
\pgfpathlineto{\pgfqpoint{2.337323in}{1.187994in}}%
\pgfpathlineto{\pgfqpoint{2.364773in}{1.187993in}}%
\pgfpathlineto{\pgfqpoint{2.392222in}{1.187993in}}%
\pgfpathlineto{\pgfqpoint{2.419672in}{1.187993in}}%
\pgfpathlineto{\pgfqpoint{2.447121in}{1.187993in}}%
\pgfpathlineto{\pgfqpoint{2.474571in}{1.187993in}}%
\pgfpathlineto{\pgfqpoint{2.502020in}{1.187993in}}%
\pgfpathlineto{\pgfqpoint{2.529470in}{1.187993in}}%
\pgfpathlineto{\pgfqpoint{2.556919in}{1.187993in}}%
\pgfpathlineto{\pgfqpoint{2.584369in}{1.187993in}}%
\pgfpathlineto{\pgfqpoint{2.611818in}{1.187993in}}%
\pgfpathlineto{\pgfqpoint{2.639268in}{1.187993in}}%
\pgfpathlineto{\pgfqpoint{2.666717in}{1.187993in}}%
\pgfpathlineto{\pgfqpoint{2.694167in}{1.187993in}}%
\pgfpathlineto{\pgfqpoint{2.721616in}{1.187993in}}%
\pgfpathlineto{\pgfqpoint{2.749066in}{1.187993in}}%
\pgfpathlineto{\pgfqpoint{2.776515in}{1.187993in}}%
\pgfpathlineto{\pgfqpoint{2.803965in}{1.187993in}}%
\pgfpathlineto{\pgfqpoint{2.831414in}{1.187993in}}%
\pgfpathlineto{\pgfqpoint{2.858864in}{1.187993in}}%
\pgfpathlineto{\pgfqpoint{2.886313in}{1.187993in}}%
\pgfpathlineto{\pgfqpoint{2.913763in}{1.187993in}}%
\pgfpathlineto{\pgfqpoint{2.941212in}{1.187993in}}%
\pgfpathlineto{\pgfqpoint{2.968662in}{1.187993in}}%
\pgfpathlineto{\pgfqpoint{2.996111in}{1.187993in}}%
\pgfpathlineto{\pgfqpoint{3.023561in}{1.187993in}}%
\pgfpathlineto{\pgfqpoint{3.051010in}{1.187993in}}%
\pgfpathlineto{\pgfqpoint{3.078460in}{1.187993in}}%
\pgfpathlineto{\pgfqpoint{3.105909in}{1.187993in}}%
\pgfpathlineto{\pgfqpoint{3.133359in}{1.187993in}}%
\pgfpathlineto{\pgfqpoint{3.160808in}{1.187993in}}%
\pgfpathlineto{\pgfqpoint{3.188258in}{1.187993in}}%
\pgfpathlineto{\pgfqpoint{3.215707in}{1.187993in}}%
\pgfpathlineto{\pgfqpoint{3.243157in}{1.187993in}}%
\pgfpathlineto{\pgfqpoint{3.270606in}{1.187993in}}%
\pgfpathlineto{\pgfqpoint{3.298056in}{1.187993in}}%
\pgfpathlineto{\pgfqpoint{3.325505in}{1.187993in}}%
\pgfpathlineto{\pgfqpoint{3.352955in}{1.187993in}}%
\pgfpathlineto{\pgfqpoint{3.380404in}{1.187993in}}%
\pgfpathlineto{\pgfqpoint{3.407854in}{1.187993in}}%
\pgfpathlineto{\pgfqpoint{3.435303in}{1.187993in}}%
\pgfpathlineto{\pgfqpoint{3.462753in}{1.187993in}}%
\pgfusepath{stroke}%
\end{pgfscope}%
\begin{pgfscope}%
\pgfpathrectangle{\pgfqpoint{0.580556in}{0.549691in}}{\pgfqpoint{3.019444in}{1.800309in}}%
\pgfusepath{clip}%
\pgfsetrectcap%
\pgfsetroundjoin%
\pgfsetlinewidth{1.505625pt}%
\definecolor{currentstroke}{rgb}{0.172549,0.627451,0.172549}%
\pgfsetstrokecolor{currentstroke}%
\pgfsetdash{}{0pt}%
\pgfpathmoveto{\pgfqpoint{0.717803in}{2.268168in}}%
\pgfpathlineto{\pgfqpoint{0.745253in}{2.261747in}}%
\pgfpathlineto{\pgfqpoint{0.772702in}{2.255620in}}%
\pgfpathlineto{\pgfqpoint{0.800152in}{2.249800in}}%
\pgfpathlineto{\pgfqpoint{0.827601in}{2.244294in}}%
\pgfpathlineto{\pgfqpoint{0.855051in}{2.239102in}}%
\pgfpathlineto{\pgfqpoint{0.882500in}{2.234225in}}%
\pgfpathlineto{\pgfqpoint{0.909950in}{2.229656in}}%
\pgfpathlineto{\pgfqpoint{0.937399in}{2.225387in}}%
\pgfpathlineto{\pgfqpoint{0.964849in}{2.221405in}}%
\pgfpathlineto{\pgfqpoint{0.992298in}{2.217698in}}%
\pgfpathlineto{\pgfqpoint{1.019748in}{2.214249in}}%
\pgfpathlineto{\pgfqpoint{1.047197in}{2.211042in}}%
\pgfpathlineto{\pgfqpoint{1.074647in}{2.208058in}}%
\pgfpathlineto{\pgfqpoint{1.102096in}{2.205281in}}%
\pgfpathlineto{\pgfqpoint{1.129546in}{2.202695in}}%
\pgfpathlineto{\pgfqpoint{1.156995in}{2.200281in}}%
\pgfpathlineto{\pgfqpoint{1.184445in}{2.198025in}}%
\pgfpathlineto{\pgfqpoint{1.211894in}{2.195911in}}%
\pgfpathlineto{\pgfqpoint{1.239344in}{2.193926in}}%
\pgfpathlineto{\pgfqpoint{1.266793in}{2.192055in}}%
\pgfpathlineto{\pgfqpoint{1.294243in}{2.190288in}}%
\pgfpathlineto{\pgfqpoint{1.321692in}{2.188613in}}%
\pgfpathlineto{\pgfqpoint{1.349142in}{2.187020in}}%
\pgfpathlineto{\pgfqpoint{1.376591in}{2.185498in}}%
\pgfpathlineto{\pgfqpoint{1.404041in}{2.184040in}}%
\pgfpathlineto{\pgfqpoint{1.431490in}{2.182636in}}%
\pgfpathlineto{\pgfqpoint{1.458940in}{2.181280in}}%
\pgfpathlineto{\pgfqpoint{1.486389in}{2.179964in}}%
\pgfpathlineto{\pgfqpoint{1.513839in}{2.178681in}}%
\pgfpathlineto{\pgfqpoint{1.541288in}{2.177425in}}%
\pgfpathlineto{\pgfqpoint{1.568738in}{2.176191in}}%
\pgfpathlineto{\pgfqpoint{1.596187in}{2.174972in}}%
\pgfpathlineto{\pgfqpoint{1.623637in}{2.173763in}}%
\pgfpathlineto{\pgfqpoint{1.651086in}{2.172557in}}%
\pgfpathlineto{\pgfqpoint{1.678536in}{2.171350in}}%
\pgfpathlineto{\pgfqpoint{1.705985in}{2.170135in}}%
\pgfpathlineto{\pgfqpoint{1.733435in}{2.168907in}}%
\pgfpathlineto{\pgfqpoint{1.760884in}{2.167660in}}%
\pgfpathlineto{\pgfqpoint{1.788334in}{2.166388in}}%
\pgfpathlineto{\pgfqpoint{1.815783in}{2.165083in}}%
\pgfpathlineto{\pgfqpoint{1.843233in}{2.163738in}}%
\pgfpathlineto{\pgfqpoint{1.870682in}{2.162345in}}%
\pgfpathlineto{\pgfqpoint{1.898132in}{2.160897in}}%
\pgfpathlineto{\pgfqpoint{1.925581in}{2.159382in}}%
\pgfpathlineto{\pgfqpoint{1.953031in}{2.157790in}}%
\pgfpathlineto{\pgfqpoint{1.980480in}{2.156110in}}%
\pgfpathlineto{\pgfqpoint{2.007930in}{2.154328in}}%
\pgfpathlineto{\pgfqpoint{2.035379in}{2.152428in}}%
\pgfpathlineto{\pgfqpoint{2.062828in}{2.150394in}}%
\pgfpathlineto{\pgfqpoint{2.090278in}{2.148206in}}%
\pgfpathlineto{\pgfqpoint{2.117727in}{2.145842in}}%
\pgfpathlineto{\pgfqpoint{2.145177in}{2.143276in}}%
\pgfpathlineto{\pgfqpoint{2.172626in}{2.140479in}}%
\pgfpathlineto{\pgfqpoint{2.200076in}{2.137419in}}%
\pgfpathlineto{\pgfqpoint{2.227525in}{2.134059in}}%
\pgfpathlineto{\pgfqpoint{2.254975in}{2.130357in}}%
\pgfpathlineto{\pgfqpoint{2.282424in}{2.126265in}}%
\pgfpathlineto{\pgfqpoint{2.309874in}{2.121732in}}%
\pgfpathlineto{\pgfqpoint{2.337323in}{2.116699in}}%
\pgfpathlineto{\pgfqpoint{2.364773in}{2.111104in}}%
\pgfpathlineto{\pgfqpoint{2.392222in}{2.104879in}}%
\pgfpathlineto{\pgfqpoint{2.419672in}{2.097953in}}%
\pgfpathlineto{\pgfqpoint{2.447121in}{2.090253in}}%
\pgfpathlineto{\pgfqpoint{2.474571in}{2.081708in}}%
\pgfpathlineto{\pgfqpoint{2.502020in}{2.072249in}}%
\pgfpathlineto{\pgfqpoint{2.529470in}{2.061817in}}%
\pgfpathlineto{\pgfqpoint{2.556919in}{2.050365in}}%
\pgfpathlineto{\pgfqpoint{2.584369in}{2.037863in}}%
\pgfpathlineto{\pgfqpoint{2.611818in}{2.024305in}}%
\pgfpathlineto{\pgfqpoint{2.639268in}{2.009708in}}%
\pgfpathlineto{\pgfqpoint{2.666717in}{1.994123in}}%
\pgfpathlineto{\pgfqpoint{2.694167in}{1.977631in}}%
\pgfpathlineto{\pgfqpoint{2.721616in}{1.960345in}}%
\pgfpathlineto{\pgfqpoint{2.749066in}{1.942407in}}%
\pgfpathlineto{\pgfqpoint{2.776515in}{1.923985in}}%
\pgfpathlineto{\pgfqpoint{2.803965in}{1.905264in}}%
\pgfpathlineto{\pgfqpoint{2.831414in}{1.886441in}}%
\pgfpathlineto{\pgfqpoint{2.858864in}{1.867718in}}%
\pgfpathlineto{\pgfqpoint{2.886313in}{1.849288in}}%
\pgfpathlineto{\pgfqpoint{2.913763in}{1.831334in}}%
\pgfpathlineto{\pgfqpoint{2.941212in}{1.814016in}}%
\pgfpathlineto{\pgfqpoint{2.968662in}{1.797472in}}%
\pgfpathlineto{\pgfqpoint{2.996111in}{1.781811in}}%
\pgfpathlineto{\pgfqpoint{3.023561in}{1.767111in}}%
\pgfpathlineto{\pgfqpoint{3.051010in}{1.753423in}}%
\pgfpathlineto{\pgfqpoint{3.078460in}{1.740769in}}%
\pgfpathlineto{\pgfqpoint{3.105909in}{1.729146in}}%
\pgfpathlineto{\pgfqpoint{3.133359in}{1.718535in}}%
\pgfpathlineto{\pgfqpoint{3.160808in}{1.708895in}}%
\pgfpathlineto{\pgfqpoint{3.188258in}{1.700176in}}%
\pgfpathlineto{\pgfqpoint{3.215707in}{1.692318in}}%
\pgfpathlineto{\pgfqpoint{3.243157in}{1.685257in}}%
\pgfpathlineto{\pgfqpoint{3.270606in}{1.678926in}}%
\pgfpathlineto{\pgfqpoint{3.298056in}{1.673259in}}%
\pgfpathlineto{\pgfqpoint{3.325505in}{1.668192in}}%
\pgfpathlineto{\pgfqpoint{3.352955in}{1.663663in}}%
\pgfpathlineto{\pgfqpoint{3.380404in}{1.659615in}}%
\pgfpathlineto{\pgfqpoint{3.407854in}{1.655995in}}%
\pgfpathlineto{\pgfqpoint{3.435303in}{1.652755in}}%
\pgfpathlineto{\pgfqpoint{3.462753in}{1.649852in}}%
\pgfusepath{stroke}%
\end{pgfscope}%
\begin{pgfscope}%
\pgfpathrectangle{\pgfqpoint{0.580556in}{0.549691in}}{\pgfqpoint{3.019444in}{1.800309in}}%
\pgfusepath{clip}%
\pgfsetrectcap%
\pgfsetroundjoin%
\pgfsetlinewidth{1.505625pt}%
\definecolor{currentstroke}{rgb}{0.121569,0.466667,0.705882}%
\pgfsetstrokecolor{currentstroke}%
\pgfsetdash{}{0pt}%
\pgfpathmoveto{\pgfqpoint{0.717803in}{2.268168in}}%
\pgfpathlineto{\pgfqpoint{0.745253in}{2.261747in}}%
\pgfpathlineto{\pgfqpoint{0.772702in}{2.257462in}}%
\pgfpathlineto{\pgfqpoint{0.800152in}{2.254105in}}%
\pgfpathlineto{\pgfqpoint{0.827601in}{2.251296in}}%
\pgfpathlineto{\pgfqpoint{0.855051in}{2.248856in}}%
\pgfpathlineto{\pgfqpoint{0.882500in}{2.246689in}}%
\pgfpathlineto{\pgfqpoint{0.909950in}{2.244730in}}%
\pgfpathlineto{\pgfqpoint{0.937399in}{2.242941in}}%
\pgfpathlineto{\pgfqpoint{0.964849in}{2.241289in}}%
\pgfpathlineto{\pgfqpoint{0.992298in}{2.239754in}}%
\pgfpathlineto{\pgfqpoint{1.019748in}{2.238319in}}%
\pgfpathlineto{\pgfqpoint{1.047197in}{2.236971in}}%
\pgfpathlineto{\pgfqpoint{1.074647in}{2.235698in}}%
\pgfpathlineto{\pgfqpoint{1.102096in}{2.234494in}}%
\pgfpathlineto{\pgfqpoint{1.129546in}{2.233349in}}%
\pgfpathlineto{\pgfqpoint{1.156995in}{2.232259in}}%
\pgfpathlineto{\pgfqpoint{1.184445in}{2.231217in}}%
\pgfpathlineto{\pgfqpoint{1.211894in}{2.230221in}}%
\pgfpathlineto{\pgfqpoint{1.239344in}{2.229265in}}%
\pgfpathlineto{\pgfqpoint{1.266793in}{2.228347in}}%
\pgfpathlineto{\pgfqpoint{1.294243in}{2.227464in}}%
\pgfpathlineto{\pgfqpoint{1.321692in}{2.226614in}}%
\pgfpathlineto{\pgfqpoint{1.349142in}{2.225792in}}%
\pgfpathlineto{\pgfqpoint{1.376591in}{2.224999in}}%
\pgfpathlineto{\pgfqpoint{1.404041in}{2.224232in}}%
\pgfpathlineto{\pgfqpoint{1.431490in}{2.223490in}}%
\pgfpathlineto{\pgfqpoint{1.458940in}{2.222770in}}%
\pgfpathlineto{\pgfqpoint{1.486389in}{2.222072in}}%
\pgfpathlineto{\pgfqpoint{1.513839in}{2.221395in}}%
\pgfpathlineto{\pgfqpoint{1.541288in}{2.220736in}}%
\pgfpathlineto{\pgfqpoint{1.568738in}{2.220096in}}%
\pgfpathlineto{\pgfqpoint{1.596187in}{2.219473in}}%
\pgfpathlineto{\pgfqpoint{1.623637in}{2.218866in}}%
\pgfpathlineto{\pgfqpoint{1.651086in}{2.218275in}}%
\pgfpathlineto{\pgfqpoint{1.678536in}{2.217699in}}%
\pgfpathlineto{\pgfqpoint{1.705985in}{2.217137in}}%
\pgfpathlineto{\pgfqpoint{1.733435in}{2.216589in}}%
\pgfpathlineto{\pgfqpoint{1.760884in}{2.216053in}}%
\pgfpathlineto{\pgfqpoint{1.788334in}{2.215530in}}%
\pgfpathlineto{\pgfqpoint{1.815783in}{2.215019in}}%
\pgfpathlineto{\pgfqpoint{1.843233in}{2.214518in}}%
\pgfpathlineto{\pgfqpoint{1.870682in}{2.214029in}}%
\pgfpathlineto{\pgfqpoint{1.898132in}{2.213550in}}%
\pgfpathlineto{\pgfqpoint{1.925581in}{2.213081in}}%
\pgfpathlineto{\pgfqpoint{1.953031in}{2.212622in}}%
\pgfpathlineto{\pgfqpoint{1.980480in}{2.212172in}}%
\pgfpathlineto{\pgfqpoint{2.007930in}{2.211731in}}%
\pgfpathlineto{\pgfqpoint{2.035379in}{2.211299in}}%
\pgfpathlineto{\pgfqpoint{2.062828in}{2.210874in}}%
\pgfpathlineto{\pgfqpoint{2.090278in}{2.210458in}}%
\pgfpathlineto{\pgfqpoint{2.117727in}{2.210050in}}%
\pgfpathlineto{\pgfqpoint{2.145177in}{2.209648in}}%
\pgfpathlineto{\pgfqpoint{2.172626in}{2.209254in}}%
\pgfpathlineto{\pgfqpoint{2.200076in}{2.208867in}}%
\pgfpathlineto{\pgfqpoint{2.227525in}{2.208487in}}%
\pgfpathlineto{\pgfqpoint{2.254975in}{2.208114in}}%
\pgfpathlineto{\pgfqpoint{2.282424in}{2.207746in}}%
\pgfpathlineto{\pgfqpoint{2.309874in}{2.207385in}}%
\pgfpathlineto{\pgfqpoint{2.337323in}{2.207030in}}%
\pgfpathlineto{\pgfqpoint{2.364773in}{2.206680in}}%
\pgfpathlineto{\pgfqpoint{2.392222in}{2.206336in}}%
\pgfpathlineto{\pgfqpoint{2.419672in}{2.205998in}}%
\pgfpathlineto{\pgfqpoint{2.447121in}{2.205665in}}%
\pgfpathlineto{\pgfqpoint{2.474571in}{2.205337in}}%
\pgfpathlineto{\pgfqpoint{2.502020in}{2.205013in}}%
\pgfpathlineto{\pgfqpoint{2.529470in}{2.204695in}}%
\pgfpathlineto{\pgfqpoint{2.556919in}{2.204381in}}%
\pgfpathlineto{\pgfqpoint{2.584369in}{2.204072in}}%
\pgfpathlineto{\pgfqpoint{2.611818in}{2.203768in}}%
\pgfpathlineto{\pgfqpoint{2.639268in}{2.203468in}}%
\pgfpathlineto{\pgfqpoint{2.666717in}{2.203171in}}%
\pgfpathlineto{\pgfqpoint{2.694167in}{2.202880in}}%
\pgfpathlineto{\pgfqpoint{2.721616in}{2.202592in}}%
\pgfpathlineto{\pgfqpoint{2.749066in}{2.202308in}}%
\pgfpathlineto{\pgfqpoint{2.776515in}{2.202027in}}%
\pgfpathlineto{\pgfqpoint{2.803965in}{2.201751in}}%
\pgfpathlineto{\pgfqpoint{2.831414in}{2.201478in}}%
\pgfpathlineto{\pgfqpoint{2.858864in}{2.201209in}}%
\pgfpathlineto{\pgfqpoint{2.886313in}{2.200943in}}%
\pgfpathlineto{\pgfqpoint{2.913763in}{2.200680in}}%
\pgfpathlineto{\pgfqpoint{2.941212in}{2.200421in}}%
\pgfpathlineto{\pgfqpoint{2.968662in}{2.200165in}}%
\pgfpathlineto{\pgfqpoint{2.996111in}{2.199912in}}%
\pgfpathlineto{\pgfqpoint{3.023561in}{2.199662in}}%
\pgfpathlineto{\pgfqpoint{3.051010in}{2.199415in}}%
\pgfpathlineto{\pgfqpoint{3.078460in}{2.199171in}}%
\pgfpathlineto{\pgfqpoint{3.105909in}{2.198930in}}%
\pgfpathlineto{\pgfqpoint{3.133359in}{2.198692in}}%
\pgfpathlineto{\pgfqpoint{3.160808in}{2.198456in}}%
\pgfpathlineto{\pgfqpoint{3.188258in}{2.198224in}}%
\pgfpathlineto{\pgfqpoint{3.215707in}{2.197993in}}%
\pgfpathlineto{\pgfqpoint{3.243157in}{2.197766in}}%
\pgfpathlineto{\pgfqpoint{3.270606in}{2.197540in}}%
\pgfpathlineto{\pgfqpoint{3.298056in}{2.197318in}}%
\pgfpathlineto{\pgfqpoint{3.325505in}{2.197097in}}%
\pgfpathlineto{\pgfqpoint{3.352955in}{2.196879in}}%
\pgfpathlineto{\pgfqpoint{3.380404in}{2.196664in}}%
\pgfpathlineto{\pgfqpoint{3.407854in}{2.196450in}}%
\pgfpathlineto{\pgfqpoint{3.435303in}{2.196239in}}%
\pgfpathlineto{\pgfqpoint{3.462753in}{2.196030in}}%
\pgfusepath{stroke}%
\end{pgfscope}%
\begin{pgfscope}%
\pgfsetrectcap%
\pgfsetmiterjoin%
\pgfsetlinewidth{0.803000pt}%
\definecolor{currentstroke}{rgb}{0.000000,0.000000,0.000000}%
\pgfsetstrokecolor{currentstroke}%
\pgfsetdash{}{0pt}%
\pgfpathmoveto{\pgfqpoint{0.580556in}{0.549691in}}%
\pgfpathlineto{\pgfqpoint{0.580556in}{2.350000in}}%
\pgfusepath{stroke}%
\end{pgfscope}%
\begin{pgfscope}%
\pgfsetrectcap%
\pgfsetmiterjoin%
\pgfsetlinewidth{0.803000pt}%
\definecolor{currentstroke}{rgb}{0.000000,0.000000,0.000000}%
\pgfsetstrokecolor{currentstroke}%
\pgfsetdash{}{0pt}%
\pgfpathmoveto{\pgfqpoint{0.580556in}{0.549691in}}%
\pgfpathlineto{\pgfqpoint{3.600000in}{0.549691in}}%
\pgfusepath{stroke}%
\end{pgfscope}%
\end{pgfpicture}%
\makeatother%
\endgroup%
}
    \hspace{1.5mm}
    \subfloat[XY Plot]{%% Creator: Matplotlib, PGF backend
%%
%% To include the figure in your LaTeX document, write
%%   \input{<filename>.pgf}
%%
%% Make sure the required packages are loaded in your preamble
%%   \usepackage{pgf}
%%
%% Figures using additional raster images can only be included by \input if
%% they are in the same directory as the main LaTeX file. For loading figures
%% from other directories you can use the `import` package
%%   \usepackage{import}
%% and then include the figures with
%%   \import{<path to file>}{<filename>.pgf}
%%
%% Matplotlib used the following preamble
%%
\begingroup%
\makeatletter%
\begin{pgfpicture}%
\pgfpathrectangle{\pgfpointorigin}{\pgfqpoint{2.700000in}{2.500000in}}%
\pgfusepath{use as bounding box, clip}%
\begin{pgfscope}%
\pgfsetbuttcap%
\pgfsetmiterjoin%
\definecolor{currentfill}{rgb}{1.000000,1.000000,1.000000}%
\pgfsetfillcolor{currentfill}%
\pgfsetlinewidth{0.000000pt}%
\definecolor{currentstroke}{rgb}{1.000000,1.000000,1.000000}%
\pgfsetstrokecolor{currentstroke}%
\pgfsetdash{}{0pt}%
\pgfpathmoveto{\pgfqpoint{0.000000in}{0.000000in}}%
\pgfpathlineto{\pgfqpoint{2.700000in}{0.000000in}}%
\pgfpathlineto{\pgfqpoint{2.700000in}{2.500000in}}%
\pgfpathlineto{\pgfqpoint{0.000000in}{2.500000in}}%
\pgfpathclose%
\pgfusepath{fill}%
\end{pgfscope}%
\begin{pgfscope}%
\pgfsetbuttcap%
\pgfsetmiterjoin%
\definecolor{currentfill}{rgb}{1.000000,1.000000,1.000000}%
\pgfsetfillcolor{currentfill}%
\pgfsetlinewidth{0.000000pt}%
\definecolor{currentstroke}{rgb}{0.000000,0.000000,0.000000}%
\pgfsetstrokecolor{currentstroke}%
\pgfsetstrokeopacity{0.000000}%
\pgfsetdash{}{0pt}%
\pgfpathmoveto{\pgfqpoint{0.150000in}{0.549691in}}%
\pgfpathlineto{\pgfqpoint{2.096296in}{0.549691in}}%
\pgfpathlineto{\pgfqpoint{2.096296in}{2.350000in}}%
\pgfpathlineto{\pgfqpoint{0.150000in}{2.350000in}}%
\pgfpathclose%
\pgfusepath{fill}%
\end{pgfscope}%
\begin{pgfscope}%
\pgfpathrectangle{\pgfqpoint{0.150000in}{0.549691in}}{\pgfqpoint{1.946296in}{1.800309in}}%
\pgfusepath{clip}%
\pgfsetbuttcap%
\pgfsetroundjoin%
\definecolor{currentfill}{rgb}{0.011765,0.011765,0.011765}%
\pgfsetfillcolor{currentfill}%
\pgfsetlinewidth{0.000000pt}%
\definecolor{currentstroke}{rgb}{0.000000,0.000000,0.000000}%
\pgfsetstrokecolor{currentstroke}%
\pgfsetdash{}{0pt}%
\pgfpathmoveto{\pgfqpoint{1.103685in}{1.431803in}}%
\pgfpathlineto{\pgfqpoint{1.113416in}{1.426043in}}%
\pgfpathlineto{\pgfqpoint{1.123148in}{1.424062in}}%
\pgfpathlineto{\pgfqpoint{1.132879in}{1.426043in}}%
\pgfpathlineto{\pgfqpoint{1.142611in}{1.431803in}}%
\pgfpathlineto{\pgfqpoint{1.142654in}{1.431842in}}%
\pgfpathlineto{\pgfqpoint{1.148880in}{1.440844in}}%
\pgfpathlineto{\pgfqpoint{1.151022in}{1.449846in}}%
\pgfpathlineto{\pgfqpoint{1.148880in}{1.458847in}}%
\pgfpathlineto{\pgfqpoint{1.142654in}{1.467849in}}%
\pgfpathlineto{\pgfqpoint{1.142611in}{1.467889in}}%
\pgfpathlineto{\pgfqpoint{1.132879in}{1.473648in}}%
\pgfpathlineto{\pgfqpoint{1.123148in}{1.475629in}}%
\pgfpathlineto{\pgfqpoint{1.113416in}{1.473648in}}%
\pgfpathlineto{\pgfqpoint{1.103685in}{1.467889in}}%
\pgfpathlineto{\pgfqpoint{1.103642in}{1.467849in}}%
\pgfpathlineto{\pgfqpoint{1.097416in}{1.458847in}}%
\pgfpathlineto{\pgfqpoint{1.095274in}{1.449846in}}%
\pgfpathlineto{\pgfqpoint{1.097416in}{1.440844in}}%
\pgfpathlineto{\pgfqpoint{1.103642in}{1.431842in}}%
\pgfpathclose%
\pgfusepath{fill}%
\end{pgfscope}%
\begin{pgfscope}%
\pgfpathrectangle{\pgfqpoint{0.150000in}{0.549691in}}{\pgfqpoint{1.946296in}{1.800309in}}%
\pgfusepath{clip}%
\pgfsetbuttcap%
\pgfsetroundjoin%
\definecolor{currentfill}{rgb}{0.043137,0.043137,0.043137}%
\pgfsetfillcolor{currentfill}%
\pgfsetlinewidth{0.000000pt}%
\definecolor{currentstroke}{rgb}{0.000000,0.000000,0.000000}%
\pgfsetstrokecolor{currentstroke}%
\pgfsetdash{}{0pt}%
\pgfpathmoveto{\pgfqpoint{1.123148in}{1.275669in}}%
\pgfpathlineto{\pgfqpoint{1.126359in}{1.278816in}}%
\pgfpathlineto{\pgfqpoint{1.123148in}{1.281901in}}%
\pgfpathlineto{\pgfqpoint{1.119937in}{1.278816in}}%
\pgfpathclose%
\pgfusepath{fill}%
\end{pgfscope}%
\begin{pgfscope}%
\pgfpathrectangle{\pgfqpoint{0.150000in}{0.549691in}}{\pgfqpoint{1.946296in}{1.800309in}}%
\pgfusepath{clip}%
\pgfsetbuttcap%
\pgfsetroundjoin%
\definecolor{currentfill}{rgb}{0.043137,0.043137,0.043137}%
\pgfsetfillcolor{currentfill}%
\pgfsetlinewidth{0.000000pt}%
\definecolor{currentstroke}{rgb}{0.000000,0.000000,0.000000}%
\pgfsetstrokecolor{currentstroke}%
\pgfsetdash{}{0pt}%
\pgfpathmoveto{\pgfqpoint{1.103685in}{1.412850in}}%
\pgfpathlineto{\pgfqpoint{1.113416in}{1.408789in}}%
\pgfpathlineto{\pgfqpoint{1.123148in}{1.407392in}}%
\pgfpathlineto{\pgfqpoint{1.132879in}{1.408789in}}%
\pgfpathlineto{\pgfqpoint{1.142611in}{1.412850in}}%
\pgfpathlineto{\pgfqpoint{1.144129in}{1.413839in}}%
\pgfpathlineto{\pgfqpoint{1.152342in}{1.419833in}}%
\pgfpathlineto{\pgfqpoint{1.155594in}{1.422841in}}%
\pgfpathlineto{\pgfqpoint{1.162074in}{1.430439in}}%
\pgfpathlineto{\pgfqpoint{1.163144in}{1.431842in}}%
\pgfpathlineto{\pgfqpoint{1.167534in}{1.440844in}}%
\pgfpathlineto{\pgfqpoint{1.169044in}{1.449846in}}%
\pgfpathlineto{\pgfqpoint{1.167534in}{1.458847in}}%
\pgfpathlineto{\pgfqpoint{1.163144in}{1.467849in}}%
\pgfpathlineto{\pgfqpoint{1.162074in}{1.469253in}}%
\pgfpathlineto{\pgfqpoint{1.155594in}{1.476850in}}%
\pgfpathlineto{\pgfqpoint{1.152342in}{1.479858in}}%
\pgfpathlineto{\pgfqpoint{1.144129in}{1.485852in}}%
\pgfpathlineto{\pgfqpoint{1.142611in}{1.486842in}}%
\pgfpathlineto{\pgfqpoint{1.132879in}{1.490902in}}%
\pgfpathlineto{\pgfqpoint{1.123148in}{1.492299in}}%
\pgfpathlineto{\pgfqpoint{1.113416in}{1.490902in}}%
\pgfpathlineto{\pgfqpoint{1.103685in}{1.486842in}}%
\pgfpathlineto{\pgfqpoint{1.102167in}{1.485852in}}%
\pgfpathlineto{\pgfqpoint{1.093953in}{1.479858in}}%
\pgfpathlineto{\pgfqpoint{1.090702in}{1.476850in}}%
\pgfpathlineto{\pgfqpoint{1.084222in}{1.469253in}}%
\pgfpathlineto{\pgfqpoint{1.083152in}{1.467849in}}%
\pgfpathlineto{\pgfqpoint{1.078762in}{1.458847in}}%
\pgfpathlineto{\pgfqpoint{1.077252in}{1.449846in}}%
\pgfpathlineto{\pgfqpoint{1.078762in}{1.440844in}}%
\pgfpathlineto{\pgfqpoint{1.083152in}{1.431842in}}%
\pgfpathlineto{\pgfqpoint{1.084222in}{1.430439in}}%
\pgfpathlineto{\pgfqpoint{1.090702in}{1.422841in}}%
\pgfpathlineto{\pgfqpoint{1.093953in}{1.419833in}}%
\pgfpathlineto{\pgfqpoint{1.102167in}{1.413839in}}%
\pgfpathclose%
\pgfpathmoveto{\pgfqpoint{1.103642in}{1.431842in}}%
\pgfpathlineto{\pgfqpoint{1.097416in}{1.440844in}}%
\pgfpathlineto{\pgfqpoint{1.095274in}{1.449846in}}%
\pgfpathlineto{\pgfqpoint{1.097416in}{1.458847in}}%
\pgfpathlineto{\pgfqpoint{1.103642in}{1.467849in}}%
\pgfpathlineto{\pgfqpoint{1.103685in}{1.467889in}}%
\pgfpathlineto{\pgfqpoint{1.113416in}{1.473648in}}%
\pgfpathlineto{\pgfqpoint{1.123148in}{1.475629in}}%
\pgfpathlineto{\pgfqpoint{1.132879in}{1.473648in}}%
\pgfpathlineto{\pgfqpoint{1.142611in}{1.467889in}}%
\pgfpathlineto{\pgfqpoint{1.142654in}{1.467849in}}%
\pgfpathlineto{\pgfqpoint{1.148880in}{1.458847in}}%
\pgfpathlineto{\pgfqpoint{1.151022in}{1.449846in}}%
\pgfpathlineto{\pgfqpoint{1.148880in}{1.440844in}}%
\pgfpathlineto{\pgfqpoint{1.142654in}{1.431842in}}%
\pgfpathlineto{\pgfqpoint{1.142611in}{1.431803in}}%
\pgfpathlineto{\pgfqpoint{1.132879in}{1.426043in}}%
\pgfpathlineto{\pgfqpoint{1.123148in}{1.424062in}}%
\pgfpathlineto{\pgfqpoint{1.113416in}{1.426043in}}%
\pgfpathlineto{\pgfqpoint{1.103685in}{1.431803in}}%
\pgfpathclose%
\pgfusepath{fill}%
\end{pgfscope}%
\begin{pgfscope}%
\pgfpathrectangle{\pgfqpoint{0.150000in}{0.549691in}}{\pgfqpoint{1.946296in}{1.800309in}}%
\pgfusepath{clip}%
\pgfsetbuttcap%
\pgfsetroundjoin%
\definecolor{currentfill}{rgb}{0.043137,0.043137,0.043137}%
\pgfsetfillcolor{currentfill}%
\pgfsetlinewidth{0.000000pt}%
\definecolor{currentstroke}{rgb}{0.000000,0.000000,0.000000}%
\pgfsetstrokecolor{currentstroke}%
\pgfsetdash{}{0pt}%
\pgfpathmoveto{\pgfqpoint{0.938250in}{1.446875in}}%
\pgfpathlineto{\pgfqpoint{0.941585in}{1.449846in}}%
\pgfpathlineto{\pgfqpoint{0.938250in}{1.452816in}}%
\pgfpathlineto{\pgfqpoint{0.934848in}{1.449846in}}%
\pgfpathclose%
\pgfusepath{fill}%
\end{pgfscope}%
\begin{pgfscope}%
\pgfpathrectangle{\pgfqpoint{0.150000in}{0.549691in}}{\pgfqpoint{1.946296in}{1.800309in}}%
\pgfusepath{clip}%
\pgfsetbuttcap%
\pgfsetroundjoin%
\definecolor{currentfill}{rgb}{0.043137,0.043137,0.043137}%
\pgfsetfillcolor{currentfill}%
\pgfsetlinewidth{0.000000pt}%
\definecolor{currentstroke}{rgb}{0.000000,0.000000,0.000000}%
\pgfsetstrokecolor{currentstroke}%
\pgfsetdash{}{0pt}%
\pgfpathmoveto{\pgfqpoint{1.308046in}{1.446875in}}%
\pgfpathlineto{\pgfqpoint{1.311448in}{1.449846in}}%
\pgfpathlineto{\pgfqpoint{1.308046in}{1.452816in}}%
\pgfpathlineto{\pgfqpoint{1.304711in}{1.449846in}}%
\pgfpathclose%
\pgfusepath{fill}%
\end{pgfscope}%
\begin{pgfscope}%
\pgfpathrectangle{\pgfqpoint{0.150000in}{0.549691in}}{\pgfqpoint{1.946296in}{1.800309in}}%
\pgfusepath{clip}%
\pgfsetbuttcap%
\pgfsetroundjoin%
\definecolor{currentfill}{rgb}{0.043137,0.043137,0.043137}%
\pgfsetfillcolor{currentfill}%
\pgfsetlinewidth{0.000000pt}%
\definecolor{currentstroke}{rgb}{0.000000,0.000000,0.000000}%
\pgfsetstrokecolor{currentstroke}%
\pgfsetdash{}{0pt}%
\pgfpathmoveto{\pgfqpoint{1.123148in}{1.617790in}}%
\pgfpathlineto{\pgfqpoint{1.126359in}{1.620875in}}%
\pgfpathlineto{\pgfqpoint{1.123148in}{1.624022in}}%
\pgfpathlineto{\pgfqpoint{1.119937in}{1.620875in}}%
\pgfpathclose%
\pgfusepath{fill}%
\end{pgfscope}%
\begin{pgfscope}%
\pgfpathrectangle{\pgfqpoint{0.150000in}{0.549691in}}{\pgfqpoint{1.946296in}{1.800309in}}%
\pgfusepath{clip}%
\pgfsetbuttcap%
\pgfsetroundjoin%
\definecolor{currentfill}{rgb}{0.070588,0.070588,0.070588}%
\pgfsetfillcolor{currentfill}%
\pgfsetlinewidth{0.000000pt}%
\definecolor{currentstroke}{rgb}{0.000000,0.000000,0.000000}%
\pgfsetstrokecolor{currentstroke}%
\pgfsetdash{}{0pt}%
\pgfpathmoveto{\pgfqpoint{1.123148in}{1.242197in}}%
\pgfpathlineto{\pgfqpoint{1.128253in}{1.242810in}}%
\pgfpathlineto{\pgfqpoint{1.132879in}{1.243470in}}%
\pgfpathlineto{\pgfqpoint{1.142611in}{1.247503in}}%
\pgfpathlineto{\pgfqpoint{1.149261in}{1.251812in}}%
\pgfpathlineto{\pgfqpoint{1.152342in}{1.254481in}}%
\pgfpathlineto{\pgfqpoint{1.158108in}{1.260813in}}%
\pgfpathlineto{\pgfqpoint{1.162074in}{1.267989in}}%
\pgfpathlineto{\pgfqpoint{1.162976in}{1.269815in}}%
\pgfpathlineto{\pgfqpoint{1.164401in}{1.278816in}}%
\pgfpathlineto{\pgfqpoint{1.162947in}{1.287818in}}%
\pgfpathlineto{\pgfqpoint{1.162074in}{1.289819in}}%
\pgfpathlineto{\pgfqpoint{1.158658in}{1.296819in}}%
\pgfpathlineto{\pgfqpoint{1.152342in}{1.305637in}}%
\pgfpathlineto{\pgfqpoint{1.152176in}{1.305821in}}%
\pgfpathlineto{\pgfqpoint{1.142611in}{1.314753in}}%
\pgfpathlineto{\pgfqpoint{1.142495in}{1.314822in}}%
\pgfpathlineto{\pgfqpoint{1.132879in}{1.320446in}}%
\pgfpathlineto{\pgfqpoint{1.123148in}{1.322404in}}%
\pgfpathlineto{\pgfqpoint{1.113416in}{1.320446in}}%
\pgfpathlineto{\pgfqpoint{1.103801in}{1.314822in}}%
\pgfpathlineto{\pgfqpoint{1.103685in}{1.314753in}}%
\pgfpathlineto{\pgfqpoint{1.094120in}{1.305821in}}%
\pgfpathlineto{\pgfqpoint{1.093953in}{1.305637in}}%
\pgfpathlineto{\pgfqpoint{1.087638in}{1.296819in}}%
\pgfpathlineto{\pgfqpoint{1.084222in}{1.289819in}}%
\pgfpathlineto{\pgfqpoint{1.083349in}{1.287818in}}%
\pgfpathlineto{\pgfqpoint{1.081895in}{1.278816in}}%
\pgfpathlineto{\pgfqpoint{1.083320in}{1.269815in}}%
\pgfpathlineto{\pgfqpoint{1.084222in}{1.267989in}}%
\pgfpathlineto{\pgfqpoint{1.088187in}{1.260813in}}%
\pgfpathlineto{\pgfqpoint{1.093953in}{1.254481in}}%
\pgfpathlineto{\pgfqpoint{1.097034in}{1.251812in}}%
\pgfpathlineto{\pgfqpoint{1.103685in}{1.247503in}}%
\pgfpathlineto{\pgfqpoint{1.113416in}{1.243470in}}%
\pgfpathlineto{\pgfqpoint{1.118043in}{1.242810in}}%
\pgfpathclose%
\pgfpathmoveto{\pgfqpoint{1.119937in}{1.278816in}}%
\pgfpathlineto{\pgfqpoint{1.123148in}{1.281901in}}%
\pgfpathlineto{\pgfqpoint{1.126359in}{1.278816in}}%
\pgfpathlineto{\pgfqpoint{1.123148in}{1.275669in}}%
\pgfpathclose%
\pgfusepath{fill}%
\end{pgfscope}%
\begin{pgfscope}%
\pgfpathrectangle{\pgfqpoint{0.150000in}{0.549691in}}{\pgfqpoint{1.946296in}{1.800309in}}%
\pgfusepath{clip}%
\pgfsetbuttcap%
\pgfsetroundjoin%
\definecolor{currentfill}{rgb}{0.070588,0.070588,0.070588}%
\pgfsetfillcolor{currentfill}%
\pgfsetlinewidth{0.000000pt}%
\definecolor{currentstroke}{rgb}{0.000000,0.000000,0.000000}%
\pgfsetstrokecolor{currentstroke}%
\pgfsetdash{}{0pt}%
\pgfpathmoveto{\pgfqpoint{1.123148in}{1.385046in}}%
\pgfpathlineto{\pgfqpoint{1.132874in}{1.386835in}}%
\pgfpathlineto{\pgfqpoint{1.132879in}{1.386836in}}%
\pgfpathlineto{\pgfqpoint{1.142611in}{1.391160in}}%
\pgfpathlineto{\pgfqpoint{1.149343in}{1.395836in}}%
\pgfpathlineto{\pgfqpoint{1.152342in}{1.397763in}}%
\pgfpathlineto{\pgfqpoint{1.161030in}{1.404838in}}%
\pgfpathlineto{\pgfqpoint{1.162074in}{1.405701in}}%
\pgfpathlineto{\pgfqpoint{1.170872in}{1.413839in}}%
\pgfpathlineto{\pgfqpoint{1.171805in}{1.414805in}}%
\pgfpathlineto{\pgfqpoint{1.179454in}{1.422841in}}%
\pgfpathlineto{\pgfqpoint{1.181537in}{1.425615in}}%
\pgfpathlineto{\pgfqpoint{1.186592in}{1.431842in}}%
\pgfpathlineto{\pgfqpoint{1.191267in}{1.440844in}}%
\pgfpathlineto{\pgfqpoint{1.191268in}{1.440849in}}%
\pgfpathlineto{\pgfqpoint{1.193202in}{1.449846in}}%
\pgfpathlineto{\pgfqpoint{1.191268in}{1.458842in}}%
\pgfpathlineto{\pgfqpoint{1.191267in}{1.458847in}}%
\pgfpathlineto{\pgfqpoint{1.186592in}{1.467849in}}%
\pgfpathlineto{\pgfqpoint{1.181537in}{1.474076in}}%
\pgfpathlineto{\pgfqpoint{1.179454in}{1.476850in}}%
\pgfpathlineto{\pgfqpoint{1.171805in}{1.484886in}}%
\pgfpathlineto{\pgfqpoint{1.170872in}{1.485852in}}%
\pgfpathlineto{\pgfqpoint{1.162074in}{1.493990in}}%
\pgfpathlineto{\pgfqpoint{1.161030in}{1.494853in}}%
\pgfpathlineto{\pgfqpoint{1.152342in}{1.501928in}}%
\pgfpathlineto{\pgfqpoint{1.149343in}{1.503855in}}%
\pgfpathlineto{\pgfqpoint{1.142611in}{1.508531in}}%
\pgfpathlineto{\pgfqpoint{1.132879in}{1.512856in}}%
\pgfpathlineto{\pgfqpoint{1.132874in}{1.512856in}}%
\pgfpathlineto{\pgfqpoint{1.123148in}{1.514645in}}%
\pgfpathlineto{\pgfqpoint{1.113422in}{1.512856in}}%
\pgfpathlineto{\pgfqpoint{1.113416in}{1.512856in}}%
\pgfpathlineto{\pgfqpoint{1.103685in}{1.508531in}}%
\pgfpathlineto{\pgfqpoint{1.096953in}{1.503855in}}%
\pgfpathlineto{\pgfqpoint{1.093953in}{1.501928in}}%
\pgfpathlineto{\pgfqpoint{1.085266in}{1.494853in}}%
\pgfpathlineto{\pgfqpoint{1.084222in}{1.493990in}}%
\pgfpathlineto{\pgfqpoint{1.075423in}{1.485852in}}%
\pgfpathlineto{\pgfqpoint{1.074491in}{1.484886in}}%
\pgfpathlineto{\pgfqpoint{1.066842in}{1.476850in}}%
\pgfpathlineto{\pgfqpoint{1.064759in}{1.474076in}}%
\pgfpathlineto{\pgfqpoint{1.059704in}{1.467849in}}%
\pgfpathlineto{\pgfqpoint{1.055028in}{1.458847in}}%
\pgfpathlineto{\pgfqpoint{1.055028in}{1.458842in}}%
\pgfpathlineto{\pgfqpoint{1.053094in}{1.449846in}}%
\pgfpathlineto{\pgfqpoint{1.055028in}{1.440849in}}%
\pgfpathlineto{\pgfqpoint{1.055028in}{1.440844in}}%
\pgfpathlineto{\pgfqpoint{1.059704in}{1.431842in}}%
\pgfpathlineto{\pgfqpoint{1.064759in}{1.425615in}}%
\pgfpathlineto{\pgfqpoint{1.066842in}{1.422841in}}%
\pgfpathlineto{\pgfqpoint{1.074491in}{1.414805in}}%
\pgfpathlineto{\pgfqpoint{1.075423in}{1.413839in}}%
\pgfpathlineto{\pgfqpoint{1.084222in}{1.405701in}}%
\pgfpathlineto{\pgfqpoint{1.085266in}{1.404838in}}%
\pgfpathlineto{\pgfqpoint{1.093953in}{1.397763in}}%
\pgfpathlineto{\pgfqpoint{1.096953in}{1.395836in}}%
\pgfpathlineto{\pgfqpoint{1.103685in}{1.391160in}}%
\pgfpathlineto{\pgfqpoint{1.113416in}{1.386836in}}%
\pgfpathlineto{\pgfqpoint{1.113422in}{1.386835in}}%
\pgfpathclose%
\pgfpathmoveto{\pgfqpoint{1.102167in}{1.413839in}}%
\pgfpathlineto{\pgfqpoint{1.093953in}{1.419833in}}%
\pgfpathlineto{\pgfqpoint{1.090702in}{1.422841in}}%
\pgfpathlineto{\pgfqpoint{1.084222in}{1.430439in}}%
\pgfpathlineto{\pgfqpoint{1.083152in}{1.431842in}}%
\pgfpathlineto{\pgfqpoint{1.078762in}{1.440844in}}%
\pgfpathlineto{\pgfqpoint{1.077252in}{1.449846in}}%
\pgfpathlineto{\pgfqpoint{1.078762in}{1.458847in}}%
\pgfpathlineto{\pgfqpoint{1.083152in}{1.467849in}}%
\pgfpathlineto{\pgfqpoint{1.084222in}{1.469253in}}%
\pgfpathlineto{\pgfqpoint{1.090702in}{1.476850in}}%
\pgfpathlineto{\pgfqpoint{1.093953in}{1.479858in}}%
\pgfpathlineto{\pgfqpoint{1.102167in}{1.485852in}}%
\pgfpathlineto{\pgfqpoint{1.103685in}{1.486842in}}%
\pgfpathlineto{\pgfqpoint{1.113416in}{1.490902in}}%
\pgfpathlineto{\pgfqpoint{1.123148in}{1.492299in}}%
\pgfpathlineto{\pgfqpoint{1.132879in}{1.490902in}}%
\pgfpathlineto{\pgfqpoint{1.142611in}{1.486842in}}%
\pgfpathlineto{\pgfqpoint{1.144129in}{1.485852in}}%
\pgfpathlineto{\pgfqpoint{1.152342in}{1.479858in}}%
\pgfpathlineto{\pgfqpoint{1.155594in}{1.476850in}}%
\pgfpathlineto{\pgfqpoint{1.162074in}{1.469253in}}%
\pgfpathlineto{\pgfqpoint{1.163144in}{1.467849in}}%
\pgfpathlineto{\pgfqpoint{1.167534in}{1.458847in}}%
\pgfpathlineto{\pgfqpoint{1.169044in}{1.449846in}}%
\pgfpathlineto{\pgfqpoint{1.167534in}{1.440844in}}%
\pgfpathlineto{\pgfqpoint{1.163144in}{1.431842in}}%
\pgfpathlineto{\pgfqpoint{1.162074in}{1.430439in}}%
\pgfpathlineto{\pgfqpoint{1.155594in}{1.422841in}}%
\pgfpathlineto{\pgfqpoint{1.152342in}{1.419833in}}%
\pgfpathlineto{\pgfqpoint{1.144129in}{1.413839in}}%
\pgfpathlineto{\pgfqpoint{1.142611in}{1.412850in}}%
\pgfpathlineto{\pgfqpoint{1.132879in}{1.408789in}}%
\pgfpathlineto{\pgfqpoint{1.123148in}{1.407392in}}%
\pgfpathlineto{\pgfqpoint{1.113416in}{1.408789in}}%
\pgfpathlineto{\pgfqpoint{1.103685in}{1.412850in}}%
\pgfpathclose%
\pgfusepath{fill}%
\end{pgfscope}%
\begin{pgfscope}%
\pgfpathrectangle{\pgfqpoint{0.150000in}{0.549691in}}{\pgfqpoint{1.946296in}{1.800309in}}%
\pgfusepath{clip}%
\pgfsetbuttcap%
\pgfsetroundjoin%
\definecolor{currentfill}{rgb}{0.070588,0.070588,0.070588}%
\pgfsetfillcolor{currentfill}%
\pgfsetlinewidth{0.000000pt}%
\definecolor{currentstroke}{rgb}{0.000000,0.000000,0.000000}%
\pgfsetstrokecolor{currentstroke}%
\pgfsetdash{}{0pt}%
\pgfpathmoveto{\pgfqpoint{0.928518in}{1.413005in}}%
\pgfpathlineto{\pgfqpoint{0.938250in}{1.411687in}}%
\pgfpathlineto{\pgfqpoint{0.947981in}{1.413032in}}%
\pgfpathlineto{\pgfqpoint{0.950145in}{1.413839in}}%
\pgfpathlineto{\pgfqpoint{0.957713in}{1.416999in}}%
\pgfpathlineto{\pgfqpoint{0.967246in}{1.422841in}}%
\pgfpathlineto{\pgfqpoint{0.967444in}{1.422995in}}%
\pgfpathlineto{\pgfqpoint{0.977101in}{1.431842in}}%
\pgfpathlineto{\pgfqpoint{0.977176in}{1.431950in}}%
\pgfpathlineto{\pgfqpoint{0.983255in}{1.440844in}}%
\pgfpathlineto{\pgfqpoint{0.985372in}{1.449846in}}%
\pgfpathlineto{\pgfqpoint{0.983255in}{1.458847in}}%
\pgfpathlineto{\pgfqpoint{0.977176in}{1.467741in}}%
\pgfpathlineto{\pgfqpoint{0.977101in}{1.467849in}}%
\pgfpathlineto{\pgfqpoint{0.967444in}{1.476696in}}%
\pgfpathlineto{\pgfqpoint{0.967246in}{1.476850in}}%
\pgfpathlineto{\pgfqpoint{0.957713in}{1.482692in}}%
\pgfpathlineto{\pgfqpoint{0.950145in}{1.485852in}}%
\pgfpathlineto{\pgfqpoint{0.947981in}{1.486659in}}%
\pgfpathlineto{\pgfqpoint{0.938250in}{1.488004in}}%
\pgfpathlineto{\pgfqpoint{0.928518in}{1.486686in}}%
\pgfpathlineto{\pgfqpoint{0.926545in}{1.485852in}}%
\pgfpathlineto{\pgfqpoint{0.918787in}{1.482184in}}%
\pgfpathlineto{\pgfqpoint{0.911941in}{1.476850in}}%
\pgfpathlineto{\pgfqpoint{0.909055in}{1.474000in}}%
\pgfpathlineto{\pgfqpoint{0.904398in}{1.467849in}}%
\pgfpathlineto{\pgfqpoint{0.900037in}{1.458847in}}%
\pgfpathlineto{\pgfqpoint{0.899324in}{1.454567in}}%
\pgfpathlineto{\pgfqpoint{0.898661in}{1.449846in}}%
\pgfpathlineto{\pgfqpoint{0.899324in}{1.445124in}}%
\pgfpathlineto{\pgfqpoint{0.900037in}{1.440844in}}%
\pgfpathlineto{\pgfqpoint{0.904398in}{1.431842in}}%
\pgfpathlineto{\pgfqpoint{0.909055in}{1.425691in}}%
\pgfpathlineto{\pgfqpoint{0.911941in}{1.422841in}}%
\pgfpathlineto{\pgfqpoint{0.918787in}{1.417507in}}%
\pgfpathlineto{\pgfqpoint{0.926545in}{1.413839in}}%
\pgfpathclose%
\pgfpathmoveto{\pgfqpoint{0.934848in}{1.449846in}}%
\pgfpathlineto{\pgfqpoint{0.938250in}{1.452816in}}%
\pgfpathlineto{\pgfqpoint{0.941585in}{1.449846in}}%
\pgfpathlineto{\pgfqpoint{0.938250in}{1.446875in}}%
\pgfpathclose%
\pgfusepath{fill}%
\end{pgfscope}%
\begin{pgfscope}%
\pgfpathrectangle{\pgfqpoint{0.150000in}{0.549691in}}{\pgfqpoint{1.946296in}{1.800309in}}%
\pgfusepath{clip}%
\pgfsetbuttcap%
\pgfsetroundjoin%
\definecolor{currentfill}{rgb}{0.070588,0.070588,0.070588}%
\pgfsetfillcolor{currentfill}%
\pgfsetlinewidth{0.000000pt}%
\definecolor{currentstroke}{rgb}{0.000000,0.000000,0.000000}%
\pgfsetstrokecolor{currentstroke}%
\pgfsetdash{}{0pt}%
\pgfpathmoveto{\pgfqpoint{1.298315in}{1.413032in}}%
\pgfpathlineto{\pgfqpoint{1.308046in}{1.411687in}}%
\pgfpathlineto{\pgfqpoint{1.317778in}{1.413005in}}%
\pgfpathlineto{\pgfqpoint{1.319751in}{1.413839in}}%
\pgfpathlineto{\pgfqpoint{1.327509in}{1.417507in}}%
\pgfpathlineto{\pgfqpoint{1.334355in}{1.422841in}}%
\pgfpathlineto{\pgfqpoint{1.337240in}{1.425691in}}%
\pgfpathlineto{\pgfqpoint{1.341898in}{1.431842in}}%
\pgfpathlineto{\pgfqpoint{1.346259in}{1.440844in}}%
\pgfpathlineto{\pgfqpoint{1.346972in}{1.445124in}}%
\pgfpathlineto{\pgfqpoint{1.347635in}{1.449846in}}%
\pgfpathlineto{\pgfqpoint{1.346972in}{1.454567in}}%
\pgfpathlineto{\pgfqpoint{1.346259in}{1.458847in}}%
\pgfpathlineto{\pgfqpoint{1.341898in}{1.467849in}}%
\pgfpathlineto{\pgfqpoint{1.337240in}{1.474000in}}%
\pgfpathlineto{\pgfqpoint{1.334355in}{1.476850in}}%
\pgfpathlineto{\pgfqpoint{1.327509in}{1.482184in}}%
\pgfpathlineto{\pgfqpoint{1.319751in}{1.485852in}}%
\pgfpathlineto{\pgfqpoint{1.317778in}{1.486686in}}%
\pgfpathlineto{\pgfqpoint{1.308046in}{1.488004in}}%
\pgfpathlineto{\pgfqpoint{1.298315in}{1.486659in}}%
\pgfpathlineto{\pgfqpoint{1.296151in}{1.485852in}}%
\pgfpathlineto{\pgfqpoint{1.288583in}{1.482692in}}%
\pgfpathlineto{\pgfqpoint{1.279050in}{1.476850in}}%
\pgfpathlineto{\pgfqpoint{1.278852in}{1.476696in}}%
\pgfpathlineto{\pgfqpoint{1.269195in}{1.467849in}}%
\pgfpathlineto{\pgfqpoint{1.269120in}{1.467741in}}%
\pgfpathlineto{\pgfqpoint{1.263041in}{1.458847in}}%
\pgfpathlineto{\pgfqpoint{1.260924in}{1.449846in}}%
\pgfpathlineto{\pgfqpoint{1.263041in}{1.440844in}}%
\pgfpathlineto{\pgfqpoint{1.269120in}{1.431950in}}%
\pgfpathlineto{\pgfqpoint{1.269195in}{1.431842in}}%
\pgfpathlineto{\pgfqpoint{1.278852in}{1.422995in}}%
\pgfpathlineto{\pgfqpoint{1.279050in}{1.422841in}}%
\pgfpathlineto{\pgfqpoint{1.288583in}{1.416999in}}%
\pgfpathlineto{\pgfqpoint{1.296151in}{1.413839in}}%
\pgfpathclose%
\pgfpathmoveto{\pgfqpoint{1.304711in}{1.449846in}}%
\pgfpathlineto{\pgfqpoint{1.308046in}{1.452816in}}%
\pgfpathlineto{\pgfqpoint{1.311448in}{1.449846in}}%
\pgfpathlineto{\pgfqpoint{1.308046in}{1.446875in}}%
\pgfpathclose%
\pgfusepath{fill}%
\end{pgfscope}%
\begin{pgfscope}%
\pgfpathrectangle{\pgfqpoint{0.150000in}{0.549691in}}{\pgfqpoint{1.946296in}{1.800309in}}%
\pgfusepath{clip}%
\pgfsetbuttcap%
\pgfsetroundjoin%
\definecolor{currentfill}{rgb}{0.070588,0.070588,0.070588}%
\pgfsetfillcolor{currentfill}%
\pgfsetlinewidth{0.000000pt}%
\definecolor{currentstroke}{rgb}{0.000000,0.000000,0.000000}%
\pgfsetstrokecolor{currentstroke}%
\pgfsetdash{}{0pt}%
\pgfpathmoveto{\pgfqpoint{1.113416in}{1.579245in}}%
\pgfpathlineto{\pgfqpoint{1.123148in}{1.577287in}}%
\pgfpathlineto{\pgfqpoint{1.132879in}{1.579245in}}%
\pgfpathlineto{\pgfqpoint{1.142495in}{1.584869in}}%
\pgfpathlineto{\pgfqpoint{1.142611in}{1.584938in}}%
\pgfpathlineto{\pgfqpoint{1.152176in}{1.593870in}}%
\pgfpathlineto{\pgfqpoint{1.152342in}{1.594054in}}%
\pgfpathlineto{\pgfqpoint{1.158658in}{1.602872in}}%
\pgfpathlineto{\pgfqpoint{1.162074in}{1.609872in}}%
\pgfpathlineto{\pgfqpoint{1.162947in}{1.611873in}}%
\pgfpathlineto{\pgfqpoint{1.164401in}{1.620875in}}%
\pgfpathlineto{\pgfqpoint{1.162976in}{1.629876in}}%
\pgfpathlineto{\pgfqpoint{1.162074in}{1.631702in}}%
\pgfpathlineto{\pgfqpoint{1.158108in}{1.638878in}}%
\pgfpathlineto{\pgfqpoint{1.152342in}{1.645210in}}%
\pgfpathlineto{\pgfqpoint{1.149261in}{1.647880in}}%
\pgfpathlineto{\pgfqpoint{1.142611in}{1.652188in}}%
\pgfpathlineto{\pgfqpoint{1.132879in}{1.656221in}}%
\pgfpathlineto{\pgfqpoint{1.128253in}{1.656881in}}%
\pgfpathlineto{\pgfqpoint{1.123148in}{1.657494in}}%
\pgfpathlineto{\pgfqpoint{1.118043in}{1.656881in}}%
\pgfpathlineto{\pgfqpoint{1.113416in}{1.656221in}}%
\pgfpathlineto{\pgfqpoint{1.103685in}{1.652188in}}%
\pgfpathlineto{\pgfqpoint{1.097034in}{1.647880in}}%
\pgfpathlineto{\pgfqpoint{1.093953in}{1.645210in}}%
\pgfpathlineto{\pgfqpoint{1.088187in}{1.638878in}}%
\pgfpathlineto{\pgfqpoint{1.084222in}{1.631702in}}%
\pgfpathlineto{\pgfqpoint{1.083320in}{1.629876in}}%
\pgfpathlineto{\pgfqpoint{1.081895in}{1.620875in}}%
\pgfpathlineto{\pgfqpoint{1.083349in}{1.611873in}}%
\pgfpathlineto{\pgfqpoint{1.084222in}{1.609872in}}%
\pgfpathlineto{\pgfqpoint{1.087638in}{1.602872in}}%
\pgfpathlineto{\pgfqpoint{1.093953in}{1.594054in}}%
\pgfpathlineto{\pgfqpoint{1.094120in}{1.593870in}}%
\pgfpathlineto{\pgfqpoint{1.103685in}{1.584938in}}%
\pgfpathlineto{\pgfqpoint{1.103801in}{1.584869in}}%
\pgfpathclose%
\pgfpathmoveto{\pgfqpoint{1.119937in}{1.620875in}}%
\pgfpathlineto{\pgfqpoint{1.123148in}{1.624022in}}%
\pgfpathlineto{\pgfqpoint{1.126359in}{1.620875in}}%
\pgfpathlineto{\pgfqpoint{1.123148in}{1.617790in}}%
\pgfpathclose%
\pgfusepath{fill}%
\end{pgfscope}%
\begin{pgfscope}%
\pgfpathrectangle{\pgfqpoint{0.150000in}{0.549691in}}{\pgfqpoint{1.946296in}{1.800309in}}%
\pgfusepath{clip}%
\pgfsetbuttcap%
\pgfsetroundjoin%
\definecolor{currentfill}{rgb}{0.101961,0.101961,0.101961}%
\pgfsetfillcolor{currentfill}%
\pgfsetlinewidth{0.000000pt}%
\definecolor{currentstroke}{rgb}{0.000000,0.000000,0.000000}%
\pgfsetstrokecolor{currentstroke}%
\pgfsetdash{}{0pt}%
\pgfpathmoveto{\pgfqpoint{1.103685in}{1.222514in}}%
\pgfpathlineto{\pgfqpoint{1.113416in}{1.219430in}}%
\pgfpathlineto{\pgfqpoint{1.123148in}{1.218369in}}%
\pgfpathlineto{\pgfqpoint{1.132879in}{1.219430in}}%
\pgfpathlineto{\pgfqpoint{1.142611in}{1.222514in}}%
\pgfpathlineto{\pgfqpoint{1.147240in}{1.224807in}}%
\pgfpathlineto{\pgfqpoint{1.152342in}{1.227363in}}%
\pgfpathlineto{\pgfqpoint{1.162074in}{1.233543in}}%
\pgfpathlineto{\pgfqpoint{1.162447in}{1.233808in}}%
\pgfpathlineto{\pgfqpoint{1.171805in}{1.241050in}}%
\pgfpathlineto{\pgfqpoint{1.174045in}{1.242810in}}%
\pgfpathlineto{\pgfqpoint{1.181537in}{1.249797in}}%
\pgfpathlineto{\pgfqpoint{1.183871in}{1.251812in}}%
\pgfpathlineto{\pgfqpoint{1.191268in}{1.260346in}}%
\pgfpathlineto{\pgfqpoint{1.191755in}{1.260813in}}%
\pgfpathlineto{\pgfqpoint{1.197451in}{1.269815in}}%
\pgfpathlineto{\pgfqpoint{1.199278in}{1.278816in}}%
\pgfpathlineto{\pgfqpoint{1.197414in}{1.287818in}}%
\pgfpathlineto{\pgfqpoint{1.192384in}{1.296819in}}%
\pgfpathlineto{\pgfqpoint{1.191268in}{1.298180in}}%
\pgfpathlineto{\pgfqpoint{1.186060in}{1.305821in}}%
\pgfpathlineto{\pgfqpoint{1.181537in}{1.311446in}}%
\pgfpathlineto{\pgfqpoint{1.179025in}{1.314822in}}%
\pgfpathlineto{\pgfqpoint{1.172186in}{1.323824in}}%
\pgfpathlineto{\pgfqpoint{1.171805in}{1.324385in}}%
\pgfpathlineto{\pgfqpoint{1.165990in}{1.332825in}}%
\pgfpathlineto{\pgfqpoint{1.162074in}{1.340391in}}%
\pgfpathlineto{\pgfqpoint{1.161242in}{1.341827in}}%
\pgfpathlineto{\pgfqpoint{1.158465in}{1.350829in}}%
\pgfpathlineto{\pgfqpoint{1.158582in}{1.359830in}}%
\pgfpathlineto{\pgfqpoint{1.161750in}{1.368832in}}%
\pgfpathlineto{\pgfqpoint{1.162074in}{1.369313in}}%
\pgfpathlineto{\pgfqpoint{1.167201in}{1.377833in}}%
\pgfpathlineto{\pgfqpoint{1.171805in}{1.383290in}}%
\pgfpathlineto{\pgfqpoint{1.174752in}{1.386835in}}%
\pgfpathlineto{\pgfqpoint{1.181537in}{1.393620in}}%
\pgfpathlineto{\pgfqpoint{1.183933in}{1.395836in}}%
\pgfpathlineto{\pgfqpoint{1.191268in}{1.402112in}}%
\pgfpathlineto{\pgfqpoint{1.195101in}{1.404838in}}%
\pgfpathlineto{\pgfqpoint{1.201000in}{1.409096in}}%
\pgfpathlineto{\pgfqpoint{1.210211in}{1.413839in}}%
\pgfpathlineto{\pgfqpoint{1.210731in}{1.414139in}}%
\pgfpathlineto{\pgfqpoint{1.220463in}{1.417069in}}%
\pgfpathlineto{\pgfqpoint{1.230194in}{1.417177in}}%
\pgfpathlineto{\pgfqpoint{1.239926in}{1.414609in}}%
\pgfpathlineto{\pgfqpoint{1.241478in}{1.413839in}}%
\pgfpathlineto{\pgfqpoint{1.249657in}{1.410217in}}%
\pgfpathlineto{\pgfqpoint{1.258783in}{1.404838in}}%
\pgfpathlineto{\pgfqpoint{1.259389in}{1.404486in}}%
\pgfpathlineto{\pgfqpoint{1.269120in}{1.398159in}}%
\pgfpathlineto{\pgfqpoint{1.272770in}{1.395836in}}%
\pgfpathlineto{\pgfqpoint{1.278852in}{1.391652in}}%
\pgfpathlineto{\pgfqpoint{1.287112in}{1.386835in}}%
\pgfpathlineto{\pgfqpoint{1.288583in}{1.385802in}}%
\pgfpathlineto{\pgfqpoint{1.298315in}{1.381150in}}%
\pgfpathlineto{\pgfqpoint{1.308046in}{1.379426in}}%
\pgfpathlineto{\pgfqpoint{1.317778in}{1.381115in}}%
\pgfpathlineto{\pgfqpoint{1.327509in}{1.386384in}}%
\pgfpathlineto{\pgfqpoint{1.328014in}{1.386835in}}%
\pgfpathlineto{\pgfqpoint{1.337240in}{1.393677in}}%
\pgfpathlineto{\pgfqpoint{1.339418in}{1.395836in}}%
\pgfpathlineto{\pgfqpoint{1.346972in}{1.402766in}}%
\pgfpathlineto{\pgfqpoint{1.348874in}{1.404838in}}%
\pgfpathlineto{\pgfqpoint{1.356703in}{1.413494in}}%
\pgfpathlineto{\pgfqpoint{1.356990in}{1.413839in}}%
\pgfpathlineto{\pgfqpoint{1.363671in}{1.422841in}}%
\pgfpathlineto{\pgfqpoint{1.366435in}{1.427560in}}%
\pgfpathlineto{\pgfqpoint{1.368914in}{1.431842in}}%
\pgfpathlineto{\pgfqpoint{1.372248in}{1.440844in}}%
\pgfpathlineto{\pgfqpoint{1.373394in}{1.449846in}}%
\pgfpathlineto{\pgfqpoint{1.372248in}{1.458847in}}%
\pgfpathlineto{\pgfqpoint{1.368914in}{1.467849in}}%
\pgfpathlineto{\pgfqpoint{1.366435in}{1.472131in}}%
\pgfpathlineto{\pgfqpoint{1.363671in}{1.476850in}}%
\pgfpathlineto{\pgfqpoint{1.356990in}{1.485852in}}%
\pgfpathlineto{\pgfqpoint{1.356703in}{1.486197in}}%
\pgfpathlineto{\pgfqpoint{1.348874in}{1.494853in}}%
\pgfpathlineto{\pgfqpoint{1.346972in}{1.496925in}}%
\pgfpathlineto{\pgfqpoint{1.339418in}{1.503855in}}%
\pgfpathlineto{\pgfqpoint{1.337240in}{1.506014in}}%
\pgfpathlineto{\pgfqpoint{1.328014in}{1.512856in}}%
\pgfpathlineto{\pgfqpoint{1.327509in}{1.513307in}}%
\pgfpathlineto{\pgfqpoint{1.317778in}{1.518576in}}%
\pgfpathlineto{\pgfqpoint{1.308046in}{1.520265in}}%
\pgfpathlineto{\pgfqpoint{1.298315in}{1.518541in}}%
\pgfpathlineto{\pgfqpoint{1.288583in}{1.513889in}}%
\pgfpathlineto{\pgfqpoint{1.287112in}{1.512856in}}%
\pgfpathlineto{\pgfqpoint{1.278852in}{1.508039in}}%
\pgfpathlineto{\pgfqpoint{1.272770in}{1.503855in}}%
\pgfpathlineto{\pgfqpoint{1.269120in}{1.501532in}}%
\pgfpathlineto{\pgfqpoint{1.259389in}{1.495205in}}%
\pgfpathlineto{\pgfqpoint{1.258783in}{1.494853in}}%
\pgfpathlineto{\pgfqpoint{1.249657in}{1.489474in}}%
\pgfpathlineto{\pgfqpoint{1.241478in}{1.485852in}}%
\pgfpathlineto{\pgfqpoint{1.239926in}{1.485082in}}%
\pgfpathlineto{\pgfqpoint{1.230194in}{1.482514in}}%
\pgfpathlineto{\pgfqpoint{1.220463in}{1.482622in}}%
\pgfpathlineto{\pgfqpoint{1.210731in}{1.485552in}}%
\pgfpathlineto{\pgfqpoint{1.210211in}{1.485852in}}%
\pgfpathlineto{\pgfqpoint{1.201000in}{1.490595in}}%
\pgfpathlineto{\pgfqpoint{1.195101in}{1.494853in}}%
\pgfpathlineto{\pgfqpoint{1.191268in}{1.497579in}}%
\pgfpathlineto{\pgfqpoint{1.183933in}{1.503855in}}%
\pgfpathlineto{\pgfqpoint{1.181537in}{1.506071in}}%
\pgfpathlineto{\pgfqpoint{1.174752in}{1.512856in}}%
\pgfpathlineto{\pgfqpoint{1.171805in}{1.516401in}}%
\pgfpathlineto{\pgfqpoint{1.167201in}{1.521858in}}%
\pgfpathlineto{\pgfqpoint{1.162074in}{1.530378in}}%
\pgfpathlineto{\pgfqpoint{1.161750in}{1.530859in}}%
\pgfpathlineto{\pgfqpoint{1.158582in}{1.539861in}}%
\pgfpathlineto{\pgfqpoint{1.158465in}{1.548863in}}%
\pgfpathlineto{\pgfqpoint{1.161242in}{1.557864in}}%
\pgfpathlineto{\pgfqpoint{1.162074in}{1.559300in}}%
\pgfpathlineto{\pgfqpoint{1.165990in}{1.566866in}}%
\pgfpathlineto{\pgfqpoint{1.171805in}{1.575307in}}%
\pgfpathlineto{\pgfqpoint{1.172186in}{1.575867in}}%
\pgfpathlineto{\pgfqpoint{1.179025in}{1.584869in}}%
\pgfpathlineto{\pgfqpoint{1.181537in}{1.588245in}}%
\pgfpathlineto{\pgfqpoint{1.186060in}{1.593870in}}%
\pgfpathlineto{\pgfqpoint{1.191268in}{1.601511in}}%
\pgfpathlineto{\pgfqpoint{1.192384in}{1.602872in}}%
\pgfpathlineto{\pgfqpoint{1.197414in}{1.611873in}}%
\pgfpathlineto{\pgfqpoint{1.199278in}{1.620875in}}%
\pgfpathlineto{\pgfqpoint{1.197451in}{1.629876in}}%
\pgfpathlineto{\pgfqpoint{1.191755in}{1.638878in}}%
\pgfpathlineto{\pgfqpoint{1.191268in}{1.639345in}}%
\pgfpathlineto{\pgfqpoint{1.183871in}{1.647880in}}%
\pgfpathlineto{\pgfqpoint{1.181537in}{1.649894in}}%
\pgfpathlineto{\pgfqpoint{1.174045in}{1.656881in}}%
\pgfpathlineto{\pgfqpoint{1.171805in}{1.658641in}}%
\pgfpathlineto{\pgfqpoint{1.162447in}{1.665883in}}%
\pgfpathlineto{\pgfqpoint{1.162074in}{1.666148in}}%
\pgfpathlineto{\pgfqpoint{1.152342in}{1.672328in}}%
\pgfpathlineto{\pgfqpoint{1.147240in}{1.674884in}}%
\pgfpathlineto{\pgfqpoint{1.142611in}{1.677177in}}%
\pgfpathlineto{\pgfqpoint{1.132879in}{1.680261in}}%
\pgfpathlineto{\pgfqpoint{1.123148in}{1.681322in}}%
\pgfpathlineto{\pgfqpoint{1.113416in}{1.680261in}}%
\pgfpathlineto{\pgfqpoint{1.103685in}{1.677177in}}%
\pgfpathlineto{\pgfqpoint{1.099055in}{1.674884in}}%
\pgfpathlineto{\pgfqpoint{1.093953in}{1.672328in}}%
\pgfpathlineto{\pgfqpoint{1.084222in}{1.666148in}}%
\pgfpathlineto{\pgfqpoint{1.083849in}{1.665883in}}%
\pgfpathlineto{\pgfqpoint{1.074491in}{1.658641in}}%
\pgfpathlineto{\pgfqpoint{1.072251in}{1.656881in}}%
\pgfpathlineto{\pgfqpoint{1.064759in}{1.649894in}}%
\pgfpathlineto{\pgfqpoint{1.062424in}{1.647880in}}%
\pgfpathlineto{\pgfqpoint{1.055028in}{1.639345in}}%
\pgfpathlineto{\pgfqpoint{1.054541in}{1.638878in}}%
\pgfpathlineto{\pgfqpoint{1.048844in}{1.629876in}}%
\pgfpathlineto{\pgfqpoint{1.047018in}{1.620875in}}%
\pgfpathlineto{\pgfqpoint{1.048881in}{1.611873in}}%
\pgfpathlineto{\pgfqpoint{1.053911in}{1.602872in}}%
\pgfpathlineto{\pgfqpoint{1.055028in}{1.601511in}}%
\pgfpathlineto{\pgfqpoint{1.060236in}{1.593870in}}%
\pgfpathlineto{\pgfqpoint{1.064759in}{1.588245in}}%
\pgfpathlineto{\pgfqpoint{1.067270in}{1.584869in}}%
\pgfpathlineto{\pgfqpoint{1.074110in}{1.575867in}}%
\pgfpathlineto{\pgfqpoint{1.074491in}{1.575307in}}%
\pgfpathlineto{\pgfqpoint{1.080306in}{1.566866in}}%
\pgfpathlineto{\pgfqpoint{1.084222in}{1.559300in}}%
\pgfpathlineto{\pgfqpoint{1.085054in}{1.557864in}}%
\pgfpathlineto{\pgfqpoint{1.087831in}{1.548863in}}%
\pgfpathlineto{\pgfqpoint{1.087714in}{1.539861in}}%
\pgfpathlineto{\pgfqpoint{1.084546in}{1.530859in}}%
\pgfpathlineto{\pgfqpoint{1.084222in}{1.530378in}}%
\pgfpathlineto{\pgfqpoint{1.079095in}{1.521858in}}%
\pgfpathlineto{\pgfqpoint{1.074491in}{1.516401in}}%
\pgfpathlineto{\pgfqpoint{1.071544in}{1.512856in}}%
\pgfpathlineto{\pgfqpoint{1.064759in}{1.506071in}}%
\pgfpathlineto{\pgfqpoint{1.062363in}{1.503855in}}%
\pgfpathlineto{\pgfqpoint{1.055028in}{1.497579in}}%
\pgfpathlineto{\pgfqpoint{1.051195in}{1.494853in}}%
\pgfpathlineto{\pgfqpoint{1.045296in}{1.490595in}}%
\pgfpathlineto{\pgfqpoint{1.036085in}{1.485852in}}%
\pgfpathlineto{\pgfqpoint{1.035565in}{1.485552in}}%
\pgfpathlineto{\pgfqpoint{1.025833in}{1.482622in}}%
\pgfpathlineto{\pgfqpoint{1.016102in}{1.482514in}}%
\pgfpathlineto{\pgfqpoint{1.006370in}{1.485082in}}%
\pgfpathlineto{\pgfqpoint{1.004818in}{1.485852in}}%
\pgfpathlineto{\pgfqpoint{0.996639in}{1.489474in}}%
\pgfpathlineto{\pgfqpoint{0.987513in}{1.494853in}}%
\pgfpathlineto{\pgfqpoint{0.986907in}{1.495205in}}%
\pgfpathlineto{\pgfqpoint{0.977176in}{1.501532in}}%
\pgfpathlineto{\pgfqpoint{0.973525in}{1.503855in}}%
\pgfpathlineto{\pgfqpoint{0.967444in}{1.508039in}}%
\pgfpathlineto{\pgfqpoint{0.959184in}{1.512856in}}%
\pgfpathlineto{\pgfqpoint{0.957713in}{1.513889in}}%
\pgfpathlineto{\pgfqpoint{0.947981in}{1.518541in}}%
\pgfpathlineto{\pgfqpoint{0.938250in}{1.520265in}}%
\pgfpathlineto{\pgfqpoint{0.928518in}{1.518576in}}%
\pgfpathlineto{\pgfqpoint{0.918787in}{1.513307in}}%
\pgfpathlineto{\pgfqpoint{0.918282in}{1.512856in}}%
\pgfpathlineto{\pgfqpoint{0.909055in}{1.506014in}}%
\pgfpathlineto{\pgfqpoint{0.906878in}{1.503855in}}%
\pgfpathlineto{\pgfqpoint{0.899324in}{1.496925in}}%
\pgfpathlineto{\pgfqpoint{0.897422in}{1.494853in}}%
\pgfpathlineto{\pgfqpoint{0.889592in}{1.486197in}}%
\pgfpathlineto{\pgfqpoint{0.889306in}{1.485852in}}%
\pgfpathlineto{\pgfqpoint{0.882624in}{1.476850in}}%
\pgfpathlineto{\pgfqpoint{0.879861in}{1.472131in}}%
\pgfpathlineto{\pgfqpoint{0.877382in}{1.467849in}}%
\pgfpathlineto{\pgfqpoint{0.874048in}{1.458847in}}%
\pgfpathlineto{\pgfqpoint{0.872901in}{1.449846in}}%
\pgfpathlineto{\pgfqpoint{0.874048in}{1.440844in}}%
\pgfpathlineto{\pgfqpoint{0.877382in}{1.431842in}}%
\pgfpathlineto{\pgfqpoint{0.879861in}{1.427560in}}%
\pgfpathlineto{\pgfqpoint{0.882624in}{1.422841in}}%
\pgfpathlineto{\pgfqpoint{0.889306in}{1.413839in}}%
\pgfpathlineto{\pgfqpoint{0.889592in}{1.413494in}}%
\pgfpathlineto{\pgfqpoint{0.897422in}{1.404838in}}%
\pgfpathlineto{\pgfqpoint{0.899324in}{1.402766in}}%
\pgfpathlineto{\pgfqpoint{0.906878in}{1.395836in}}%
\pgfpathlineto{\pgfqpoint{0.909055in}{1.393677in}}%
\pgfpathlineto{\pgfqpoint{0.918282in}{1.386835in}}%
\pgfpathlineto{\pgfqpoint{0.918787in}{1.386384in}}%
\pgfpathlineto{\pgfqpoint{0.928518in}{1.381115in}}%
\pgfpathlineto{\pgfqpoint{0.938250in}{1.379426in}}%
\pgfpathlineto{\pgfqpoint{0.947981in}{1.381150in}}%
\pgfpathlineto{\pgfqpoint{0.957713in}{1.385802in}}%
\pgfpathlineto{\pgfqpoint{0.959184in}{1.386835in}}%
\pgfpathlineto{\pgfqpoint{0.967444in}{1.391652in}}%
\pgfpathlineto{\pgfqpoint{0.973525in}{1.395836in}}%
\pgfpathlineto{\pgfqpoint{0.977176in}{1.398159in}}%
\pgfpathlineto{\pgfqpoint{0.986907in}{1.404486in}}%
\pgfpathlineto{\pgfqpoint{0.987513in}{1.404838in}}%
\pgfpathlineto{\pgfqpoint{0.996639in}{1.410217in}}%
\pgfpathlineto{\pgfqpoint{1.004818in}{1.413839in}}%
\pgfpathlineto{\pgfqpoint{1.006370in}{1.414609in}}%
\pgfpathlineto{\pgfqpoint{1.016102in}{1.417177in}}%
\pgfpathlineto{\pgfqpoint{1.025833in}{1.417069in}}%
\pgfpathlineto{\pgfqpoint{1.035565in}{1.414139in}}%
\pgfpathlineto{\pgfqpoint{1.036085in}{1.413839in}}%
\pgfpathlineto{\pgfqpoint{1.045296in}{1.409096in}}%
\pgfpathlineto{\pgfqpoint{1.051195in}{1.404838in}}%
\pgfpathlineto{\pgfqpoint{1.055028in}{1.402112in}}%
\pgfpathlineto{\pgfqpoint{1.062363in}{1.395836in}}%
\pgfpathlineto{\pgfqpoint{1.064759in}{1.393620in}}%
\pgfpathlineto{\pgfqpoint{1.071544in}{1.386835in}}%
\pgfpathlineto{\pgfqpoint{1.074491in}{1.383290in}}%
\pgfpathlineto{\pgfqpoint{1.079095in}{1.377833in}}%
\pgfpathlineto{\pgfqpoint{1.084222in}{1.369313in}}%
\pgfpathlineto{\pgfqpoint{1.084546in}{1.368832in}}%
\pgfpathlineto{\pgfqpoint{1.087714in}{1.359830in}}%
\pgfpathlineto{\pgfqpoint{1.087831in}{1.350829in}}%
\pgfpathlineto{\pgfqpoint{1.085054in}{1.341827in}}%
\pgfpathlineto{\pgfqpoint{1.084222in}{1.340391in}}%
\pgfpathlineto{\pgfqpoint{1.080306in}{1.332825in}}%
\pgfpathlineto{\pgfqpoint{1.074491in}{1.324385in}}%
\pgfpathlineto{\pgfqpoint{1.074110in}{1.323824in}}%
\pgfpathlineto{\pgfqpoint{1.067270in}{1.314822in}}%
\pgfpathlineto{\pgfqpoint{1.064759in}{1.311446in}}%
\pgfpathlineto{\pgfqpoint{1.060236in}{1.305821in}}%
\pgfpathlineto{\pgfqpoint{1.055028in}{1.298180in}}%
\pgfpathlineto{\pgfqpoint{1.053911in}{1.296819in}}%
\pgfpathlineto{\pgfqpoint{1.048881in}{1.287818in}}%
\pgfpathlineto{\pgfqpoint{1.047018in}{1.278816in}}%
\pgfpathlineto{\pgfqpoint{1.048844in}{1.269815in}}%
\pgfpathlineto{\pgfqpoint{1.054541in}{1.260813in}}%
\pgfpathlineto{\pgfqpoint{1.055028in}{1.260346in}}%
\pgfpathlineto{\pgfqpoint{1.062424in}{1.251812in}}%
\pgfpathlineto{\pgfqpoint{1.064759in}{1.249797in}}%
\pgfpathlineto{\pgfqpoint{1.072251in}{1.242810in}}%
\pgfpathlineto{\pgfqpoint{1.074491in}{1.241050in}}%
\pgfpathlineto{\pgfqpoint{1.083849in}{1.233808in}}%
\pgfpathlineto{\pgfqpoint{1.084222in}{1.233543in}}%
\pgfpathlineto{\pgfqpoint{1.093953in}{1.227363in}}%
\pgfpathlineto{\pgfqpoint{1.099055in}{1.224807in}}%
\pgfpathclose%
\pgfpathmoveto{\pgfqpoint{1.118043in}{1.242810in}}%
\pgfpathlineto{\pgfqpoint{1.113416in}{1.243470in}}%
\pgfpathlineto{\pgfqpoint{1.103685in}{1.247503in}}%
\pgfpathlineto{\pgfqpoint{1.097034in}{1.251812in}}%
\pgfpathlineto{\pgfqpoint{1.093953in}{1.254481in}}%
\pgfpathlineto{\pgfqpoint{1.088187in}{1.260813in}}%
\pgfpathlineto{\pgfqpoint{1.084222in}{1.267989in}}%
\pgfpathlineto{\pgfqpoint{1.083320in}{1.269815in}}%
\pgfpathlineto{\pgfqpoint{1.081895in}{1.278816in}}%
\pgfpathlineto{\pgfqpoint{1.083349in}{1.287818in}}%
\pgfpathlineto{\pgfqpoint{1.084222in}{1.289819in}}%
\pgfpathlineto{\pgfqpoint{1.087638in}{1.296819in}}%
\pgfpathlineto{\pgfqpoint{1.093953in}{1.305637in}}%
\pgfpathlineto{\pgfqpoint{1.094120in}{1.305821in}}%
\pgfpathlineto{\pgfqpoint{1.103685in}{1.314753in}}%
\pgfpathlineto{\pgfqpoint{1.103801in}{1.314822in}}%
\pgfpathlineto{\pgfqpoint{1.113416in}{1.320446in}}%
\pgfpathlineto{\pgfqpoint{1.123148in}{1.322404in}}%
\pgfpathlineto{\pgfqpoint{1.132879in}{1.320446in}}%
\pgfpathlineto{\pgfqpoint{1.142495in}{1.314822in}}%
\pgfpathlineto{\pgfqpoint{1.142611in}{1.314753in}}%
\pgfpathlineto{\pgfqpoint{1.152176in}{1.305821in}}%
\pgfpathlineto{\pgfqpoint{1.152342in}{1.305637in}}%
\pgfpathlineto{\pgfqpoint{1.158658in}{1.296819in}}%
\pgfpathlineto{\pgfqpoint{1.162074in}{1.289819in}}%
\pgfpathlineto{\pgfqpoint{1.162947in}{1.287818in}}%
\pgfpathlineto{\pgfqpoint{1.164401in}{1.278816in}}%
\pgfpathlineto{\pgfqpoint{1.162976in}{1.269815in}}%
\pgfpathlineto{\pgfqpoint{1.162074in}{1.267989in}}%
\pgfpathlineto{\pgfqpoint{1.158108in}{1.260813in}}%
\pgfpathlineto{\pgfqpoint{1.152342in}{1.254481in}}%
\pgfpathlineto{\pgfqpoint{1.149261in}{1.251812in}}%
\pgfpathlineto{\pgfqpoint{1.142611in}{1.247503in}}%
\pgfpathlineto{\pgfqpoint{1.132879in}{1.243470in}}%
\pgfpathlineto{\pgfqpoint{1.128253in}{1.242810in}}%
\pgfpathlineto{\pgfqpoint{1.123148in}{1.242197in}}%
\pgfpathclose%
\pgfpathmoveto{\pgfqpoint{1.113422in}{1.386835in}}%
\pgfpathlineto{\pgfqpoint{1.113416in}{1.386836in}}%
\pgfpathlineto{\pgfqpoint{1.103685in}{1.391160in}}%
\pgfpathlineto{\pgfqpoint{1.096953in}{1.395836in}}%
\pgfpathlineto{\pgfqpoint{1.093953in}{1.397763in}}%
\pgfpathlineto{\pgfqpoint{1.085266in}{1.404838in}}%
\pgfpathlineto{\pgfqpoint{1.084222in}{1.405701in}}%
\pgfpathlineto{\pgfqpoint{1.075423in}{1.413839in}}%
\pgfpathlineto{\pgfqpoint{1.074491in}{1.414805in}}%
\pgfpathlineto{\pgfqpoint{1.066842in}{1.422841in}}%
\pgfpathlineto{\pgfqpoint{1.064759in}{1.425615in}}%
\pgfpathlineto{\pgfqpoint{1.059704in}{1.431842in}}%
\pgfpathlineto{\pgfqpoint{1.055028in}{1.440844in}}%
\pgfpathlineto{\pgfqpoint{1.055028in}{1.440849in}}%
\pgfpathlineto{\pgfqpoint{1.053094in}{1.449846in}}%
\pgfpathlineto{\pgfqpoint{1.055028in}{1.458842in}}%
\pgfpathlineto{\pgfqpoint{1.055028in}{1.458847in}}%
\pgfpathlineto{\pgfqpoint{1.059704in}{1.467849in}}%
\pgfpathlineto{\pgfqpoint{1.064759in}{1.474076in}}%
\pgfpathlineto{\pgfqpoint{1.066842in}{1.476850in}}%
\pgfpathlineto{\pgfqpoint{1.074491in}{1.484886in}}%
\pgfpathlineto{\pgfqpoint{1.075423in}{1.485852in}}%
\pgfpathlineto{\pgfqpoint{1.084222in}{1.493990in}}%
\pgfpathlineto{\pgfqpoint{1.085266in}{1.494853in}}%
\pgfpathlineto{\pgfqpoint{1.093953in}{1.501928in}}%
\pgfpathlineto{\pgfqpoint{1.096953in}{1.503855in}}%
\pgfpathlineto{\pgfqpoint{1.103685in}{1.508531in}}%
\pgfpathlineto{\pgfqpoint{1.113416in}{1.512856in}}%
\pgfpathlineto{\pgfqpoint{1.113422in}{1.512856in}}%
\pgfpathlineto{\pgfqpoint{1.123148in}{1.514645in}}%
\pgfpathlineto{\pgfqpoint{1.132874in}{1.512856in}}%
\pgfpathlineto{\pgfqpoint{1.132879in}{1.512856in}}%
\pgfpathlineto{\pgfqpoint{1.142611in}{1.508531in}}%
\pgfpathlineto{\pgfqpoint{1.149343in}{1.503855in}}%
\pgfpathlineto{\pgfqpoint{1.152342in}{1.501928in}}%
\pgfpathlineto{\pgfqpoint{1.161030in}{1.494853in}}%
\pgfpathlineto{\pgfqpoint{1.162074in}{1.493990in}}%
\pgfpathlineto{\pgfqpoint{1.170872in}{1.485852in}}%
\pgfpathlineto{\pgfqpoint{1.171805in}{1.484886in}}%
\pgfpathlineto{\pgfqpoint{1.179454in}{1.476850in}}%
\pgfpathlineto{\pgfqpoint{1.181537in}{1.474076in}}%
\pgfpathlineto{\pgfqpoint{1.186592in}{1.467849in}}%
\pgfpathlineto{\pgfqpoint{1.191267in}{1.458847in}}%
\pgfpathlineto{\pgfqpoint{1.191268in}{1.458842in}}%
\pgfpathlineto{\pgfqpoint{1.193202in}{1.449846in}}%
\pgfpathlineto{\pgfqpoint{1.191268in}{1.440849in}}%
\pgfpathlineto{\pgfqpoint{1.191267in}{1.440844in}}%
\pgfpathlineto{\pgfqpoint{1.186592in}{1.431842in}}%
\pgfpathlineto{\pgfqpoint{1.181537in}{1.425615in}}%
\pgfpathlineto{\pgfqpoint{1.179454in}{1.422841in}}%
\pgfpathlineto{\pgfqpoint{1.171805in}{1.414805in}}%
\pgfpathlineto{\pgfqpoint{1.170872in}{1.413839in}}%
\pgfpathlineto{\pgfqpoint{1.162074in}{1.405701in}}%
\pgfpathlineto{\pgfqpoint{1.161030in}{1.404838in}}%
\pgfpathlineto{\pgfqpoint{1.152342in}{1.397763in}}%
\pgfpathlineto{\pgfqpoint{1.149343in}{1.395836in}}%
\pgfpathlineto{\pgfqpoint{1.142611in}{1.391160in}}%
\pgfpathlineto{\pgfqpoint{1.132879in}{1.386836in}}%
\pgfpathlineto{\pgfqpoint{1.132874in}{1.386835in}}%
\pgfpathlineto{\pgfqpoint{1.123148in}{1.385046in}}%
\pgfpathclose%
\pgfpathmoveto{\pgfqpoint{0.926545in}{1.413839in}}%
\pgfpathlineto{\pgfqpoint{0.918787in}{1.417507in}}%
\pgfpathlineto{\pgfqpoint{0.911941in}{1.422841in}}%
\pgfpathlineto{\pgfqpoint{0.909055in}{1.425691in}}%
\pgfpathlineto{\pgfqpoint{0.904398in}{1.431842in}}%
\pgfpathlineto{\pgfqpoint{0.900037in}{1.440844in}}%
\pgfpathlineto{\pgfqpoint{0.899324in}{1.445124in}}%
\pgfpathlineto{\pgfqpoint{0.898661in}{1.449846in}}%
\pgfpathlineto{\pgfqpoint{0.899324in}{1.454567in}}%
\pgfpathlineto{\pgfqpoint{0.900037in}{1.458847in}}%
\pgfpathlineto{\pgfqpoint{0.904398in}{1.467849in}}%
\pgfpathlineto{\pgfqpoint{0.909055in}{1.474000in}}%
\pgfpathlineto{\pgfqpoint{0.911941in}{1.476850in}}%
\pgfpathlineto{\pgfqpoint{0.918787in}{1.482184in}}%
\pgfpathlineto{\pgfqpoint{0.926545in}{1.485852in}}%
\pgfpathlineto{\pgfqpoint{0.928518in}{1.486686in}}%
\pgfpathlineto{\pgfqpoint{0.938250in}{1.488004in}}%
\pgfpathlineto{\pgfqpoint{0.947981in}{1.486659in}}%
\pgfpathlineto{\pgfqpoint{0.950145in}{1.485852in}}%
\pgfpathlineto{\pgfqpoint{0.957713in}{1.482692in}}%
\pgfpathlineto{\pgfqpoint{0.967246in}{1.476850in}}%
\pgfpathlineto{\pgfqpoint{0.967444in}{1.476696in}}%
\pgfpathlineto{\pgfqpoint{0.977101in}{1.467849in}}%
\pgfpathlineto{\pgfqpoint{0.977176in}{1.467741in}}%
\pgfpathlineto{\pgfqpoint{0.983255in}{1.458847in}}%
\pgfpathlineto{\pgfqpoint{0.985372in}{1.449846in}}%
\pgfpathlineto{\pgfqpoint{0.983255in}{1.440844in}}%
\pgfpathlineto{\pgfqpoint{0.977176in}{1.431950in}}%
\pgfpathlineto{\pgfqpoint{0.977101in}{1.431842in}}%
\pgfpathlineto{\pgfqpoint{0.967444in}{1.422995in}}%
\pgfpathlineto{\pgfqpoint{0.967246in}{1.422841in}}%
\pgfpathlineto{\pgfqpoint{0.957713in}{1.416999in}}%
\pgfpathlineto{\pgfqpoint{0.950145in}{1.413839in}}%
\pgfpathlineto{\pgfqpoint{0.947981in}{1.413032in}}%
\pgfpathlineto{\pgfqpoint{0.938250in}{1.411687in}}%
\pgfpathlineto{\pgfqpoint{0.928518in}{1.413005in}}%
\pgfpathclose%
\pgfpathmoveto{\pgfqpoint{1.296151in}{1.413839in}}%
\pgfpathlineto{\pgfqpoint{1.288583in}{1.416999in}}%
\pgfpathlineto{\pgfqpoint{1.279050in}{1.422841in}}%
\pgfpathlineto{\pgfqpoint{1.278852in}{1.422995in}}%
\pgfpathlineto{\pgfqpoint{1.269195in}{1.431842in}}%
\pgfpathlineto{\pgfqpoint{1.269120in}{1.431950in}}%
\pgfpathlineto{\pgfqpoint{1.263041in}{1.440844in}}%
\pgfpathlineto{\pgfqpoint{1.260924in}{1.449846in}}%
\pgfpathlineto{\pgfqpoint{1.263041in}{1.458847in}}%
\pgfpathlineto{\pgfqpoint{1.269120in}{1.467741in}}%
\pgfpathlineto{\pgfqpoint{1.269195in}{1.467849in}}%
\pgfpathlineto{\pgfqpoint{1.278852in}{1.476696in}}%
\pgfpathlineto{\pgfqpoint{1.279050in}{1.476850in}}%
\pgfpathlineto{\pgfqpoint{1.288583in}{1.482692in}}%
\pgfpathlineto{\pgfqpoint{1.296151in}{1.485852in}}%
\pgfpathlineto{\pgfqpoint{1.298315in}{1.486659in}}%
\pgfpathlineto{\pgfqpoint{1.308046in}{1.488004in}}%
\pgfpathlineto{\pgfqpoint{1.317778in}{1.486686in}}%
\pgfpathlineto{\pgfqpoint{1.319751in}{1.485852in}}%
\pgfpathlineto{\pgfqpoint{1.327509in}{1.482184in}}%
\pgfpathlineto{\pgfqpoint{1.334355in}{1.476850in}}%
\pgfpathlineto{\pgfqpoint{1.337240in}{1.474000in}}%
\pgfpathlineto{\pgfqpoint{1.341898in}{1.467849in}}%
\pgfpathlineto{\pgfqpoint{1.346259in}{1.458847in}}%
\pgfpathlineto{\pgfqpoint{1.346972in}{1.454567in}}%
\pgfpathlineto{\pgfqpoint{1.347635in}{1.449846in}}%
\pgfpathlineto{\pgfqpoint{1.346972in}{1.445124in}}%
\pgfpathlineto{\pgfqpoint{1.346259in}{1.440844in}}%
\pgfpathlineto{\pgfqpoint{1.341898in}{1.431842in}}%
\pgfpathlineto{\pgfqpoint{1.337240in}{1.425691in}}%
\pgfpathlineto{\pgfqpoint{1.334355in}{1.422841in}}%
\pgfpathlineto{\pgfqpoint{1.327509in}{1.417507in}}%
\pgfpathlineto{\pgfqpoint{1.319751in}{1.413839in}}%
\pgfpathlineto{\pgfqpoint{1.317778in}{1.413005in}}%
\pgfpathlineto{\pgfqpoint{1.308046in}{1.411687in}}%
\pgfpathlineto{\pgfqpoint{1.298315in}{1.413032in}}%
\pgfpathclose%
\pgfpathmoveto{\pgfqpoint{1.103801in}{1.584869in}}%
\pgfpathlineto{\pgfqpoint{1.103685in}{1.584938in}}%
\pgfpathlineto{\pgfqpoint{1.094120in}{1.593870in}}%
\pgfpathlineto{\pgfqpoint{1.093953in}{1.594054in}}%
\pgfpathlineto{\pgfqpoint{1.087638in}{1.602872in}}%
\pgfpathlineto{\pgfqpoint{1.084222in}{1.609872in}}%
\pgfpathlineto{\pgfqpoint{1.083349in}{1.611873in}}%
\pgfpathlineto{\pgfqpoint{1.081895in}{1.620875in}}%
\pgfpathlineto{\pgfqpoint{1.083320in}{1.629876in}}%
\pgfpathlineto{\pgfqpoint{1.084222in}{1.631702in}}%
\pgfpathlineto{\pgfqpoint{1.088187in}{1.638878in}}%
\pgfpathlineto{\pgfqpoint{1.093953in}{1.645210in}}%
\pgfpathlineto{\pgfqpoint{1.097034in}{1.647880in}}%
\pgfpathlineto{\pgfqpoint{1.103685in}{1.652188in}}%
\pgfpathlineto{\pgfqpoint{1.113416in}{1.656221in}}%
\pgfpathlineto{\pgfqpoint{1.118043in}{1.656881in}}%
\pgfpathlineto{\pgfqpoint{1.123148in}{1.657494in}}%
\pgfpathlineto{\pgfqpoint{1.128253in}{1.656881in}}%
\pgfpathlineto{\pgfqpoint{1.132879in}{1.656221in}}%
\pgfpathlineto{\pgfqpoint{1.142611in}{1.652188in}}%
\pgfpathlineto{\pgfqpoint{1.149261in}{1.647880in}}%
\pgfpathlineto{\pgfqpoint{1.152342in}{1.645210in}}%
\pgfpathlineto{\pgfqpoint{1.158108in}{1.638878in}}%
\pgfpathlineto{\pgfqpoint{1.162074in}{1.631702in}}%
\pgfpathlineto{\pgfqpoint{1.162976in}{1.629876in}}%
\pgfpathlineto{\pgfqpoint{1.164401in}{1.620875in}}%
\pgfpathlineto{\pgfqpoint{1.162947in}{1.611873in}}%
\pgfpathlineto{\pgfqpoint{1.162074in}{1.609872in}}%
\pgfpathlineto{\pgfqpoint{1.158658in}{1.602872in}}%
\pgfpathlineto{\pgfqpoint{1.152342in}{1.594054in}}%
\pgfpathlineto{\pgfqpoint{1.152176in}{1.593870in}}%
\pgfpathlineto{\pgfqpoint{1.142611in}{1.584938in}}%
\pgfpathlineto{\pgfqpoint{1.142495in}{1.584869in}}%
\pgfpathlineto{\pgfqpoint{1.132879in}{1.579245in}}%
\pgfpathlineto{\pgfqpoint{1.123148in}{1.577287in}}%
\pgfpathlineto{\pgfqpoint{1.113416in}{1.579245in}}%
\pgfpathclose%
\pgfusepath{fill}%
\end{pgfscope}%
\begin{pgfscope}%
\pgfpathrectangle{\pgfqpoint{0.150000in}{0.549691in}}{\pgfqpoint{1.946296in}{1.800309in}}%
\pgfusepath{clip}%
\pgfsetbuttcap%
\pgfsetroundjoin%
\definecolor{currentfill}{rgb}{0.101961,0.101961,0.101961}%
\pgfsetfillcolor{currentfill}%
\pgfsetlinewidth{0.000000pt}%
\definecolor{currentstroke}{rgb}{0.000000,0.000000,0.000000}%
\pgfsetstrokecolor{currentstroke}%
\pgfsetdash{}{0pt}%
\pgfpathmoveto{\pgfqpoint{0.928518in}{1.239631in}}%
\pgfpathlineto{\pgfqpoint{0.938250in}{1.238528in}}%
\pgfpathlineto{\pgfqpoint{0.947981in}{1.239653in}}%
\pgfpathlineto{\pgfqpoint{0.957713in}{1.242691in}}%
\pgfpathlineto{\pgfqpoint{0.957973in}{1.242810in}}%
\pgfpathlineto{\pgfqpoint{0.967444in}{1.247962in}}%
\pgfpathlineto{\pgfqpoint{0.973443in}{1.251812in}}%
\pgfpathlineto{\pgfqpoint{0.977176in}{1.255015in}}%
\pgfpathlineto{\pgfqpoint{0.983789in}{1.260813in}}%
\pgfpathlineto{\pgfqpoint{0.986907in}{1.265318in}}%
\pgfpathlineto{\pgfqpoint{0.990392in}{1.269815in}}%
\pgfpathlineto{\pgfqpoint{0.992629in}{1.278816in}}%
\pgfpathlineto{\pgfqpoint{0.990347in}{1.287818in}}%
\pgfpathlineto{\pgfqpoint{0.986907in}{1.292844in}}%
\pgfpathlineto{\pgfqpoint{0.984477in}{1.296819in}}%
\pgfpathlineto{\pgfqpoint{0.977176in}{1.304958in}}%
\pgfpathlineto{\pgfqpoint{0.976385in}{1.305821in}}%
\pgfpathlineto{\pgfqpoint{0.967444in}{1.314091in}}%
\pgfpathlineto{\pgfqpoint{0.966512in}{1.314822in}}%
\pgfpathlineto{\pgfqpoint{0.957713in}{1.321576in}}%
\pgfpathlineto{\pgfqpoint{0.953415in}{1.323824in}}%
\pgfpathlineto{\pgfqpoint{0.947981in}{1.327006in}}%
\pgfpathlineto{\pgfqpoint{0.938250in}{1.329116in}}%
\pgfpathlineto{\pgfqpoint{0.928518in}{1.327048in}}%
\pgfpathlineto{\pgfqpoint{0.923657in}{1.323824in}}%
\pgfpathlineto{\pgfqpoint{0.918787in}{1.320940in}}%
\pgfpathlineto{\pgfqpoint{0.912519in}{1.314822in}}%
\pgfpathlineto{\pgfqpoint{0.909055in}{1.311369in}}%
\pgfpathlineto{\pgfqpoint{0.904894in}{1.305821in}}%
\pgfpathlineto{\pgfqpoint{0.899324in}{1.297060in}}%
\pgfpathlineto{\pgfqpoint{0.899195in}{1.296819in}}%
\pgfpathlineto{\pgfqpoint{0.895911in}{1.287818in}}%
\pgfpathlineto{\pgfqpoint{0.894694in}{1.278816in}}%
\pgfpathlineto{\pgfqpoint{0.895887in}{1.269815in}}%
\pgfpathlineto{\pgfqpoint{0.899324in}{1.261496in}}%
\pgfpathlineto{\pgfqpoint{0.899659in}{1.260813in}}%
\pgfpathlineto{\pgfqpoint{0.906935in}{1.251812in}}%
\pgfpathlineto{\pgfqpoint{0.909055in}{1.249850in}}%
\pgfpathlineto{\pgfqpoint{0.918787in}{1.243120in}}%
\pgfpathlineto{\pgfqpoint{0.919525in}{1.242810in}}%
\pgfpathclose%
\pgfusepath{fill}%
\end{pgfscope}%
\begin{pgfscope}%
\pgfpathrectangle{\pgfqpoint{0.150000in}{0.549691in}}{\pgfqpoint{1.946296in}{1.800309in}}%
\pgfusepath{clip}%
\pgfsetbuttcap%
\pgfsetroundjoin%
\definecolor{currentfill}{rgb}{0.101961,0.101961,0.101961}%
\pgfsetfillcolor{currentfill}%
\pgfsetlinewidth{0.000000pt}%
\definecolor{currentstroke}{rgb}{0.000000,0.000000,0.000000}%
\pgfsetstrokecolor{currentstroke}%
\pgfsetdash{}{0pt}%
\pgfpathmoveto{\pgfqpoint{1.288583in}{1.242691in}}%
\pgfpathlineto{\pgfqpoint{1.298315in}{1.239653in}}%
\pgfpathlineto{\pgfqpoint{1.308046in}{1.238528in}}%
\pgfpathlineto{\pgfqpoint{1.317778in}{1.239631in}}%
\pgfpathlineto{\pgfqpoint{1.326771in}{1.242810in}}%
\pgfpathlineto{\pgfqpoint{1.327509in}{1.243120in}}%
\pgfpathlineto{\pgfqpoint{1.337240in}{1.249850in}}%
\pgfpathlineto{\pgfqpoint{1.339361in}{1.251812in}}%
\pgfpathlineto{\pgfqpoint{1.346637in}{1.260813in}}%
\pgfpathlineto{\pgfqpoint{1.346972in}{1.261496in}}%
\pgfpathlineto{\pgfqpoint{1.350409in}{1.269815in}}%
\pgfpathlineto{\pgfqpoint{1.351601in}{1.278816in}}%
\pgfpathlineto{\pgfqpoint{1.350385in}{1.287818in}}%
\pgfpathlineto{\pgfqpoint{1.347101in}{1.296819in}}%
\pgfpathlineto{\pgfqpoint{1.346972in}{1.297060in}}%
\pgfpathlineto{\pgfqpoint{1.341402in}{1.305821in}}%
\pgfpathlineto{\pgfqpoint{1.337240in}{1.311369in}}%
\pgfpathlineto{\pgfqpoint{1.333777in}{1.314822in}}%
\pgfpathlineto{\pgfqpoint{1.327509in}{1.320940in}}%
\pgfpathlineto{\pgfqpoint{1.322639in}{1.323824in}}%
\pgfpathlineto{\pgfqpoint{1.317778in}{1.327048in}}%
\pgfpathlineto{\pgfqpoint{1.308046in}{1.329116in}}%
\pgfpathlineto{\pgfqpoint{1.298315in}{1.327006in}}%
\pgfpathlineto{\pgfqpoint{1.292881in}{1.323824in}}%
\pgfpathlineto{\pgfqpoint{1.288583in}{1.321576in}}%
\pgfpathlineto{\pgfqpoint{1.279784in}{1.314822in}}%
\pgfpathlineto{\pgfqpoint{1.278852in}{1.314091in}}%
\pgfpathlineto{\pgfqpoint{1.269911in}{1.305821in}}%
\pgfpathlineto{\pgfqpoint{1.269120in}{1.304958in}}%
\pgfpathlineto{\pgfqpoint{1.261819in}{1.296819in}}%
\pgfpathlineto{\pgfqpoint{1.259389in}{1.292844in}}%
\pgfpathlineto{\pgfqpoint{1.255949in}{1.287818in}}%
\pgfpathlineto{\pgfqpoint{1.253667in}{1.278816in}}%
\pgfpathlineto{\pgfqpoint{1.255904in}{1.269815in}}%
\pgfpathlineto{\pgfqpoint{1.259389in}{1.265318in}}%
\pgfpathlineto{\pgfqpoint{1.262507in}{1.260813in}}%
\pgfpathlineto{\pgfqpoint{1.269120in}{1.255015in}}%
\pgfpathlineto{\pgfqpoint{1.272853in}{1.251812in}}%
\pgfpathlineto{\pgfqpoint{1.278852in}{1.247962in}}%
\pgfpathlineto{\pgfqpoint{1.288323in}{1.242810in}}%
\pgfpathclose%
\pgfusepath{fill}%
\end{pgfscope}%
\begin{pgfscope}%
\pgfpathrectangle{\pgfqpoint{0.150000in}{0.549691in}}{\pgfqpoint{1.946296in}{1.800309in}}%
\pgfusepath{clip}%
\pgfsetbuttcap%
\pgfsetroundjoin%
\definecolor{currentfill}{rgb}{0.101961,0.101961,0.101961}%
\pgfsetfillcolor{currentfill}%
\pgfsetlinewidth{0.000000pt}%
\definecolor{currentstroke}{rgb}{0.000000,0.000000,0.000000}%
\pgfsetstrokecolor{currentstroke}%
\pgfsetdash{}{0pt}%
\pgfpathmoveto{\pgfqpoint{0.928518in}{1.572644in}}%
\pgfpathlineto{\pgfqpoint{0.938250in}{1.570575in}}%
\pgfpathlineto{\pgfqpoint{0.947981in}{1.572685in}}%
\pgfpathlineto{\pgfqpoint{0.953415in}{1.575867in}}%
\pgfpathlineto{\pgfqpoint{0.957713in}{1.578115in}}%
\pgfpathlineto{\pgfqpoint{0.966512in}{1.584869in}}%
\pgfpathlineto{\pgfqpoint{0.967444in}{1.585600in}}%
\pgfpathlineto{\pgfqpoint{0.976385in}{1.593870in}}%
\pgfpathlineto{\pgfqpoint{0.977176in}{1.594733in}}%
\pgfpathlineto{\pgfqpoint{0.984477in}{1.602872in}}%
\pgfpathlineto{\pgfqpoint{0.986907in}{1.606847in}}%
\pgfpathlineto{\pgfqpoint{0.990347in}{1.611873in}}%
\pgfpathlineto{\pgfqpoint{0.992629in}{1.620875in}}%
\pgfpathlineto{\pgfqpoint{0.990392in}{1.629876in}}%
\pgfpathlineto{\pgfqpoint{0.986907in}{1.634373in}}%
\pgfpathlineto{\pgfqpoint{0.983789in}{1.638878in}}%
\pgfpathlineto{\pgfqpoint{0.977176in}{1.644676in}}%
\pgfpathlineto{\pgfqpoint{0.973443in}{1.647880in}}%
\pgfpathlineto{\pgfqpoint{0.967444in}{1.651729in}}%
\pgfpathlineto{\pgfqpoint{0.957973in}{1.656881in}}%
\pgfpathlineto{\pgfqpoint{0.957713in}{1.657000in}}%
\pgfpathlineto{\pgfqpoint{0.947981in}{1.660038in}}%
\pgfpathlineto{\pgfqpoint{0.938250in}{1.661163in}}%
\pgfpathlineto{\pgfqpoint{0.928518in}{1.660060in}}%
\pgfpathlineto{\pgfqpoint{0.919525in}{1.656881in}}%
\pgfpathlineto{\pgfqpoint{0.918787in}{1.656571in}}%
\pgfpathlineto{\pgfqpoint{0.909055in}{1.649841in}}%
\pgfpathlineto{\pgfqpoint{0.906935in}{1.647880in}}%
\pgfpathlineto{\pgfqpoint{0.899659in}{1.638878in}}%
\pgfpathlineto{\pgfqpoint{0.899324in}{1.638195in}}%
\pgfpathlineto{\pgfqpoint{0.895887in}{1.629876in}}%
\pgfpathlineto{\pgfqpoint{0.894694in}{1.620875in}}%
\pgfpathlineto{\pgfqpoint{0.895911in}{1.611873in}}%
\pgfpathlineto{\pgfqpoint{0.899195in}{1.602872in}}%
\pgfpathlineto{\pgfqpoint{0.899324in}{1.602631in}}%
\pgfpathlineto{\pgfqpoint{0.904894in}{1.593870in}}%
\pgfpathlineto{\pgfqpoint{0.909055in}{1.588322in}}%
\pgfpathlineto{\pgfqpoint{0.912519in}{1.584869in}}%
\pgfpathlineto{\pgfqpoint{0.918787in}{1.578751in}}%
\pgfpathlineto{\pgfqpoint{0.923657in}{1.575867in}}%
\pgfpathclose%
\pgfusepath{fill}%
\end{pgfscope}%
\begin{pgfscope}%
\pgfpathrectangle{\pgfqpoint{0.150000in}{0.549691in}}{\pgfqpoint{1.946296in}{1.800309in}}%
\pgfusepath{clip}%
\pgfsetbuttcap%
\pgfsetroundjoin%
\definecolor{currentfill}{rgb}{0.101961,0.101961,0.101961}%
\pgfsetfillcolor{currentfill}%
\pgfsetlinewidth{0.000000pt}%
\definecolor{currentstroke}{rgb}{0.000000,0.000000,0.000000}%
\pgfsetstrokecolor{currentstroke}%
\pgfsetdash{}{0pt}%
\pgfpathmoveto{\pgfqpoint{1.298315in}{1.572685in}}%
\pgfpathlineto{\pgfqpoint{1.308046in}{1.570575in}}%
\pgfpathlineto{\pgfqpoint{1.317778in}{1.572644in}}%
\pgfpathlineto{\pgfqpoint{1.322639in}{1.575867in}}%
\pgfpathlineto{\pgfqpoint{1.327509in}{1.578751in}}%
\pgfpathlineto{\pgfqpoint{1.333777in}{1.584869in}}%
\pgfpathlineto{\pgfqpoint{1.337240in}{1.588322in}}%
\pgfpathlineto{\pgfqpoint{1.341402in}{1.593870in}}%
\pgfpathlineto{\pgfqpoint{1.346972in}{1.602631in}}%
\pgfpathlineto{\pgfqpoint{1.347101in}{1.602872in}}%
\pgfpathlineto{\pgfqpoint{1.350385in}{1.611873in}}%
\pgfpathlineto{\pgfqpoint{1.351601in}{1.620875in}}%
\pgfpathlineto{\pgfqpoint{1.350409in}{1.629876in}}%
\pgfpathlineto{\pgfqpoint{1.346972in}{1.638195in}}%
\pgfpathlineto{\pgfqpoint{1.346637in}{1.638878in}}%
\pgfpathlineto{\pgfqpoint{1.339361in}{1.647880in}}%
\pgfpathlineto{\pgfqpoint{1.337240in}{1.649841in}}%
\pgfpathlineto{\pgfqpoint{1.327509in}{1.656571in}}%
\pgfpathlineto{\pgfqpoint{1.326771in}{1.656881in}}%
\pgfpathlineto{\pgfqpoint{1.317778in}{1.660060in}}%
\pgfpathlineto{\pgfqpoint{1.308046in}{1.661163in}}%
\pgfpathlineto{\pgfqpoint{1.298315in}{1.660038in}}%
\pgfpathlineto{\pgfqpoint{1.288583in}{1.657000in}}%
\pgfpathlineto{\pgfqpoint{1.288323in}{1.656881in}}%
\pgfpathlineto{\pgfqpoint{1.278852in}{1.651729in}}%
\pgfpathlineto{\pgfqpoint{1.272853in}{1.647880in}}%
\pgfpathlineto{\pgfqpoint{1.269120in}{1.644676in}}%
\pgfpathlineto{\pgfqpoint{1.262507in}{1.638878in}}%
\pgfpathlineto{\pgfqpoint{1.259389in}{1.634373in}}%
\pgfpathlineto{\pgfqpoint{1.255904in}{1.629876in}}%
\pgfpathlineto{\pgfqpoint{1.253667in}{1.620875in}}%
\pgfpathlineto{\pgfqpoint{1.255949in}{1.611873in}}%
\pgfpathlineto{\pgfqpoint{1.259389in}{1.606847in}}%
\pgfpathlineto{\pgfqpoint{1.261819in}{1.602872in}}%
\pgfpathlineto{\pgfqpoint{1.269120in}{1.594733in}}%
\pgfpathlineto{\pgfqpoint{1.269911in}{1.593870in}}%
\pgfpathlineto{\pgfqpoint{1.278852in}{1.585600in}}%
\pgfpathlineto{\pgfqpoint{1.279784in}{1.584869in}}%
\pgfpathlineto{\pgfqpoint{1.288583in}{1.578115in}}%
\pgfpathlineto{\pgfqpoint{1.292881in}{1.575867in}}%
\pgfpathclose%
\pgfusepath{fill}%
\end{pgfscope}%
\begin{pgfscope}%
\pgfpathrectangle{\pgfqpoint{0.150000in}{0.549691in}}{\pgfqpoint{1.946296in}{1.800309in}}%
\pgfusepath{clip}%
\pgfsetbuttcap%
\pgfsetroundjoin%
\definecolor{currentfill}{rgb}{0.129412,0.129412,0.129412}%
\pgfsetfillcolor{currentfill}%
\pgfsetlinewidth{0.000000pt}%
\definecolor{currentstroke}{rgb}{0.000000,0.000000,0.000000}%
\pgfsetstrokecolor{currentstroke}%
\pgfsetdash{}{0pt}%
\pgfpathmoveto{\pgfqpoint{1.113416in}{1.077350in}}%
\pgfpathlineto{\pgfqpoint{1.123148in}{1.076025in}}%
\pgfpathlineto{\pgfqpoint{1.132879in}{1.077350in}}%
\pgfpathlineto{\pgfqpoint{1.141554in}{1.080782in}}%
\pgfpathlineto{\pgfqpoint{1.142611in}{1.081364in}}%
\pgfpathlineto{\pgfqpoint{1.152342in}{1.089731in}}%
\pgfpathlineto{\pgfqpoint{1.152391in}{1.089784in}}%
\pgfpathlineto{\pgfqpoint{1.157270in}{1.098785in}}%
\pgfpathlineto{\pgfqpoint{1.158928in}{1.107787in}}%
\pgfpathlineto{\pgfqpoint{1.157820in}{1.116788in}}%
\pgfpathlineto{\pgfqpoint{1.154658in}{1.125790in}}%
\pgfpathlineto{\pgfqpoint{1.152342in}{1.130614in}}%
\pgfpathlineto{\pgfqpoint{1.149801in}{1.134792in}}%
\pgfpathlineto{\pgfqpoint{1.144117in}{1.143793in}}%
\pgfpathlineto{\pgfqpoint{1.142611in}{1.146720in}}%
\pgfpathlineto{\pgfqpoint{1.137725in}{1.152795in}}%
\pgfpathlineto{\pgfqpoint{1.133903in}{1.161796in}}%
\pgfpathlineto{\pgfqpoint{1.134909in}{1.170798in}}%
\pgfpathlineto{\pgfqpoint{1.141648in}{1.179799in}}%
\pgfpathlineto{\pgfqpoint{1.142611in}{1.180477in}}%
\pgfpathlineto{\pgfqpoint{1.150174in}{1.188801in}}%
\pgfpathlineto{\pgfqpoint{1.152342in}{1.190449in}}%
\pgfpathlineto{\pgfqpoint{1.159978in}{1.197802in}}%
\pgfpathlineto{\pgfqpoint{1.162074in}{1.199404in}}%
\pgfpathlineto{\pgfqpoint{1.170731in}{1.206804in}}%
\pgfpathlineto{\pgfqpoint{1.171805in}{1.207605in}}%
\pgfpathlineto{\pgfqpoint{1.181537in}{1.214970in}}%
\pgfpathlineto{\pgfqpoint{1.182730in}{1.215805in}}%
\pgfpathlineto{\pgfqpoint{1.191268in}{1.221437in}}%
\pgfpathlineto{\pgfqpoint{1.197414in}{1.224807in}}%
\pgfpathlineto{\pgfqpoint{1.201000in}{1.226796in}}%
\pgfpathlineto{\pgfqpoint{1.210731in}{1.230645in}}%
\pgfpathlineto{\pgfqpoint{1.220463in}{1.232657in}}%
\pgfpathlineto{\pgfqpoint{1.230194in}{1.232731in}}%
\pgfpathlineto{\pgfqpoint{1.239926in}{1.230968in}}%
\pgfpathlineto{\pgfqpoint{1.249657in}{1.227656in}}%
\pgfpathlineto{\pgfqpoint{1.255949in}{1.224807in}}%
\pgfpathlineto{\pgfqpoint{1.259389in}{1.223267in}}%
\pgfpathlineto{\pgfqpoint{1.269120in}{1.218390in}}%
\pgfpathlineto{\pgfqpoint{1.274388in}{1.215805in}}%
\pgfpathlineto{\pgfqpoint{1.278852in}{1.213481in}}%
\pgfpathlineto{\pgfqpoint{1.288583in}{1.209186in}}%
\pgfpathlineto{\pgfqpoint{1.296501in}{1.206804in}}%
\pgfpathlineto{\pgfqpoint{1.298315in}{1.206179in}}%
\pgfpathlineto{\pgfqpoint{1.308046in}{1.204936in}}%
\pgfpathlineto{\pgfqpoint{1.317778in}{1.206154in}}%
\pgfpathlineto{\pgfqpoint{1.319442in}{1.206804in}}%
\pgfpathlineto{\pgfqpoint{1.327509in}{1.209552in}}%
\pgfpathlineto{\pgfqpoint{1.337240in}{1.215013in}}%
\pgfpathlineto{\pgfqpoint{1.338296in}{1.215805in}}%
\pgfpathlineto{\pgfqpoint{1.346972in}{1.221941in}}%
\pgfpathlineto{\pgfqpoint{1.350385in}{1.224807in}}%
\pgfpathlineto{\pgfqpoint{1.356703in}{1.230174in}}%
\pgfpathlineto{\pgfqpoint{1.360632in}{1.233808in}}%
\pgfpathlineto{\pgfqpoint{1.366435in}{1.239653in}}%
\pgfpathlineto{\pgfqpoint{1.369533in}{1.242810in}}%
\pgfpathlineto{\pgfqpoint{1.376166in}{1.250836in}}%
\pgfpathlineto{\pgfqpoint{1.377023in}{1.251812in}}%
\pgfpathlineto{\pgfqpoint{1.382927in}{1.260813in}}%
\pgfpathlineto{\pgfqpoint{1.385898in}{1.268275in}}%
\pgfpathlineto{\pgfqpoint{1.386600in}{1.269815in}}%
\pgfpathlineto{\pgfqpoint{1.387918in}{1.278816in}}%
\pgfpathlineto{\pgfqpoint{1.386574in}{1.287818in}}%
\pgfpathlineto{\pgfqpoint{1.385898in}{1.289495in}}%
\pgfpathlineto{\pgfqpoint{1.383323in}{1.296819in}}%
\pgfpathlineto{\pgfqpoint{1.378679in}{1.305821in}}%
\pgfpathlineto{\pgfqpoint{1.376166in}{1.309949in}}%
\pgfpathlineto{\pgfqpoint{1.373372in}{1.314822in}}%
\pgfpathlineto{\pgfqpoint{1.368099in}{1.323824in}}%
\pgfpathlineto{\pgfqpoint{1.366435in}{1.327005in}}%
\pgfpathlineto{\pgfqpoint{1.363354in}{1.332825in}}%
\pgfpathlineto{\pgfqpoint{1.359775in}{1.341827in}}%
\pgfpathlineto{\pgfqpoint{1.357868in}{1.350829in}}%
\pgfpathlineto{\pgfqpoint{1.357948in}{1.359830in}}%
\pgfpathlineto{\pgfqpoint{1.360123in}{1.368832in}}%
\pgfpathlineto{\pgfqpoint{1.364285in}{1.377833in}}%
\pgfpathlineto{\pgfqpoint{1.366435in}{1.381150in}}%
\pgfpathlineto{\pgfqpoint{1.370078in}{1.386835in}}%
\pgfpathlineto{\pgfqpoint{1.376166in}{1.394733in}}%
\pgfpathlineto{\pgfqpoint{1.377069in}{1.395836in}}%
\pgfpathlineto{\pgfqpoint{1.385032in}{1.404838in}}%
\pgfpathlineto{\pgfqpoint{1.385898in}{1.405832in}}%
\pgfpathlineto{\pgfqpoint{1.393898in}{1.413839in}}%
\pgfpathlineto{\pgfqpoint{1.395629in}{1.415778in}}%
\pgfpathlineto{\pgfqpoint{1.403579in}{1.422841in}}%
\pgfpathlineto{\pgfqpoint{1.405361in}{1.424847in}}%
\pgfpathlineto{\pgfqpoint{1.414359in}{1.431842in}}%
\pgfpathlineto{\pgfqpoint{1.415092in}{1.432733in}}%
\pgfpathlineto{\pgfqpoint{1.424824in}{1.438967in}}%
\pgfpathlineto{\pgfqpoint{1.434555in}{1.439898in}}%
\pgfpathlineto{\pgfqpoint{1.444287in}{1.436362in}}%
\pgfpathlineto{\pgfqpoint{1.450854in}{1.431842in}}%
\pgfpathlineto{\pgfqpoint{1.454018in}{1.430449in}}%
\pgfpathlineto{\pgfqpoint{1.463750in}{1.425191in}}%
\pgfpathlineto{\pgfqpoint{1.468266in}{1.422841in}}%
\pgfpathlineto{\pgfqpoint{1.473481in}{1.420699in}}%
\pgfpathlineto{\pgfqpoint{1.483213in}{1.417774in}}%
\pgfpathlineto{\pgfqpoint{1.492944in}{1.416750in}}%
\pgfpathlineto{\pgfqpoint{1.502676in}{1.418283in}}%
\pgfpathlineto{\pgfqpoint{1.512407in}{1.422796in}}%
\pgfpathlineto{\pgfqpoint{1.512464in}{1.422841in}}%
\pgfpathlineto{\pgfqpoint{1.521510in}{1.431842in}}%
\pgfpathlineto{\pgfqpoint{1.522139in}{1.432820in}}%
\pgfpathlineto{\pgfqpoint{1.525850in}{1.440844in}}%
\pgfpathlineto{\pgfqpoint{1.527282in}{1.449846in}}%
\pgfpathlineto{\pgfqpoint{1.525850in}{1.458847in}}%
\pgfpathlineto{\pgfqpoint{1.522139in}{1.466871in}}%
\pgfpathlineto{\pgfqpoint{1.521510in}{1.467849in}}%
\pgfpathlineto{\pgfqpoint{1.512464in}{1.476850in}}%
\pgfpathlineto{\pgfqpoint{1.512407in}{1.476895in}}%
\pgfpathlineto{\pgfqpoint{1.502676in}{1.481408in}}%
\pgfpathlineto{\pgfqpoint{1.492944in}{1.482942in}}%
\pgfpathlineto{\pgfqpoint{1.483213in}{1.481917in}}%
\pgfpathlineto{\pgfqpoint{1.473481in}{1.478992in}}%
\pgfpathlineto{\pgfqpoint{1.468266in}{1.476850in}}%
\pgfpathlineto{\pgfqpoint{1.463750in}{1.474500in}}%
\pgfpathlineto{\pgfqpoint{1.454018in}{1.469242in}}%
\pgfpathlineto{\pgfqpoint{1.450854in}{1.467849in}}%
\pgfpathlineto{\pgfqpoint{1.444287in}{1.463329in}}%
\pgfpathlineto{\pgfqpoint{1.434555in}{1.459794in}}%
\pgfpathlineto{\pgfqpoint{1.424824in}{1.460724in}}%
\pgfpathlineto{\pgfqpoint{1.415092in}{1.466958in}}%
\pgfpathlineto{\pgfqpoint{1.414359in}{1.467849in}}%
\pgfpathlineto{\pgfqpoint{1.405361in}{1.474844in}}%
\pgfpathlineto{\pgfqpoint{1.403579in}{1.476850in}}%
\pgfpathlineto{\pgfqpoint{1.395629in}{1.483913in}}%
\pgfpathlineto{\pgfqpoint{1.393898in}{1.485852in}}%
\pgfpathlineto{\pgfqpoint{1.385898in}{1.493859in}}%
\pgfpathlineto{\pgfqpoint{1.385032in}{1.494853in}}%
\pgfpathlineto{\pgfqpoint{1.377069in}{1.503855in}}%
\pgfpathlineto{\pgfqpoint{1.376166in}{1.504958in}}%
\pgfpathlineto{\pgfqpoint{1.370078in}{1.512856in}}%
\pgfpathlineto{\pgfqpoint{1.366435in}{1.518541in}}%
\pgfpathlineto{\pgfqpoint{1.364285in}{1.521858in}}%
\pgfpathlineto{\pgfqpoint{1.360123in}{1.530859in}}%
\pgfpathlineto{\pgfqpoint{1.357948in}{1.539861in}}%
\pgfpathlineto{\pgfqpoint{1.357868in}{1.548863in}}%
\pgfpathlineto{\pgfqpoint{1.359775in}{1.557864in}}%
\pgfpathlineto{\pgfqpoint{1.363354in}{1.566866in}}%
\pgfpathlineto{\pgfqpoint{1.366435in}{1.572686in}}%
\pgfpathlineto{\pgfqpoint{1.368099in}{1.575867in}}%
\pgfpathlineto{\pgfqpoint{1.373372in}{1.584869in}}%
\pgfpathlineto{\pgfqpoint{1.376166in}{1.589742in}}%
\pgfpathlineto{\pgfqpoint{1.378679in}{1.593870in}}%
\pgfpathlineto{\pgfqpoint{1.383323in}{1.602872in}}%
\pgfpathlineto{\pgfqpoint{1.385898in}{1.610196in}}%
\pgfpathlineto{\pgfqpoint{1.386574in}{1.611873in}}%
\pgfpathlineto{\pgfqpoint{1.387918in}{1.620875in}}%
\pgfpathlineto{\pgfqpoint{1.386600in}{1.629876in}}%
\pgfpathlineto{\pgfqpoint{1.385898in}{1.631416in}}%
\pgfpathlineto{\pgfqpoint{1.382927in}{1.638878in}}%
\pgfpathlineto{\pgfqpoint{1.377023in}{1.647880in}}%
\pgfpathlineto{\pgfqpoint{1.376166in}{1.648855in}}%
\pgfpathlineto{\pgfqpoint{1.369533in}{1.656881in}}%
\pgfpathlineto{\pgfqpoint{1.366435in}{1.660038in}}%
\pgfpathlineto{\pgfqpoint{1.360632in}{1.665883in}}%
\pgfpathlineto{\pgfqpoint{1.356703in}{1.669517in}}%
\pgfpathlineto{\pgfqpoint{1.350385in}{1.674884in}}%
\pgfpathlineto{\pgfqpoint{1.346972in}{1.677750in}}%
\pgfpathlineto{\pgfqpoint{1.338296in}{1.683886in}}%
\pgfpathlineto{\pgfqpoint{1.337240in}{1.684678in}}%
\pgfpathlineto{\pgfqpoint{1.327509in}{1.690139in}}%
\pgfpathlineto{\pgfqpoint{1.319442in}{1.692887in}}%
\pgfpathlineto{\pgfqpoint{1.317778in}{1.693537in}}%
\pgfpathlineto{\pgfqpoint{1.308046in}{1.694756in}}%
\pgfpathlineto{\pgfqpoint{1.298315in}{1.693512in}}%
\pgfpathlineto{\pgfqpoint{1.296501in}{1.692887in}}%
\pgfpathlineto{\pgfqpoint{1.288583in}{1.690506in}}%
\pgfpathlineto{\pgfqpoint{1.278852in}{1.686210in}}%
\pgfpathlineto{\pgfqpoint{1.274388in}{1.683886in}}%
\pgfpathlineto{\pgfqpoint{1.269120in}{1.681301in}}%
\pgfpathlineto{\pgfqpoint{1.259389in}{1.676424in}}%
\pgfpathlineto{\pgfqpoint{1.255949in}{1.674884in}}%
\pgfpathlineto{\pgfqpoint{1.249657in}{1.672035in}}%
\pgfpathlineto{\pgfqpoint{1.239926in}{1.668723in}}%
\pgfpathlineto{\pgfqpoint{1.230194in}{1.666960in}}%
\pgfpathlineto{\pgfqpoint{1.220463in}{1.667034in}}%
\pgfpathlineto{\pgfqpoint{1.210731in}{1.669046in}}%
\pgfpathlineto{\pgfqpoint{1.201000in}{1.672895in}}%
\pgfpathlineto{\pgfqpoint{1.197414in}{1.674884in}}%
\pgfpathlineto{\pgfqpoint{1.191268in}{1.678254in}}%
\pgfpathlineto{\pgfqpoint{1.182730in}{1.683886in}}%
\pgfpathlineto{\pgfqpoint{1.181537in}{1.684721in}}%
\pgfpathlineto{\pgfqpoint{1.171805in}{1.692086in}}%
\pgfpathlineto{\pgfqpoint{1.170731in}{1.692887in}}%
\pgfpathlineto{\pgfqpoint{1.162074in}{1.700287in}}%
\pgfpathlineto{\pgfqpoint{1.159978in}{1.701889in}}%
\pgfpathlineto{\pgfqpoint{1.152342in}{1.709243in}}%
\pgfpathlineto{\pgfqpoint{1.150174in}{1.710890in}}%
\pgfpathlineto{\pgfqpoint{1.142611in}{1.719214in}}%
\pgfpathlineto{\pgfqpoint{1.141648in}{1.719892in}}%
\pgfpathlineto{\pgfqpoint{1.134909in}{1.728893in}}%
\pgfpathlineto{\pgfqpoint{1.133903in}{1.737895in}}%
\pgfpathlineto{\pgfqpoint{1.137725in}{1.746897in}}%
\pgfpathlineto{\pgfqpoint{1.142611in}{1.752971in}}%
\pgfpathlineto{\pgfqpoint{1.144117in}{1.755898in}}%
\pgfpathlineto{\pgfqpoint{1.149801in}{1.764900in}}%
\pgfpathlineto{\pgfqpoint{1.152342in}{1.769077in}}%
\pgfpathlineto{\pgfqpoint{1.154658in}{1.773901in}}%
\pgfpathlineto{\pgfqpoint{1.157820in}{1.782903in}}%
\pgfpathlineto{\pgfqpoint{1.158928in}{1.791904in}}%
\pgfpathlineto{\pgfqpoint{1.157270in}{1.800906in}}%
\pgfpathlineto{\pgfqpoint{1.152391in}{1.809907in}}%
\pgfpathlineto{\pgfqpoint{1.152342in}{1.809960in}}%
\pgfpathlineto{\pgfqpoint{1.142611in}{1.818327in}}%
\pgfpathlineto{\pgfqpoint{1.141554in}{1.818909in}}%
\pgfpathlineto{\pgfqpoint{1.132879in}{1.822342in}}%
\pgfpathlineto{\pgfqpoint{1.123148in}{1.823666in}}%
\pgfpathlineto{\pgfqpoint{1.113416in}{1.822342in}}%
\pgfpathlineto{\pgfqpoint{1.104742in}{1.818909in}}%
\pgfpathlineto{\pgfqpoint{1.103685in}{1.818327in}}%
\pgfpathlineto{\pgfqpoint{1.093953in}{1.809960in}}%
\pgfpathlineto{\pgfqpoint{1.093905in}{1.809907in}}%
\pgfpathlineto{\pgfqpoint{1.089026in}{1.800906in}}%
\pgfpathlineto{\pgfqpoint{1.087368in}{1.791904in}}%
\pgfpathlineto{\pgfqpoint{1.088476in}{1.782903in}}%
\pgfpathlineto{\pgfqpoint{1.091638in}{1.773901in}}%
\pgfpathlineto{\pgfqpoint{1.093953in}{1.769077in}}%
\pgfpathlineto{\pgfqpoint{1.096495in}{1.764900in}}%
\pgfpathlineto{\pgfqpoint{1.102178in}{1.755898in}}%
\pgfpathlineto{\pgfqpoint{1.103685in}{1.752971in}}%
\pgfpathlineto{\pgfqpoint{1.108571in}{1.746897in}}%
\pgfpathlineto{\pgfqpoint{1.112393in}{1.737895in}}%
\pgfpathlineto{\pgfqpoint{1.111387in}{1.728893in}}%
\pgfpathlineto{\pgfqpoint{1.104648in}{1.719892in}}%
\pgfpathlineto{\pgfqpoint{1.103685in}{1.719214in}}%
\pgfpathlineto{\pgfqpoint{1.096122in}{1.710890in}}%
\pgfpathlineto{\pgfqpoint{1.093953in}{1.709243in}}%
\pgfpathlineto{\pgfqpoint{1.086318in}{1.701889in}}%
\pgfpathlineto{\pgfqpoint{1.084222in}{1.700287in}}%
\pgfpathlineto{\pgfqpoint{1.075565in}{1.692887in}}%
\pgfpathlineto{\pgfqpoint{1.074491in}{1.692086in}}%
\pgfpathlineto{\pgfqpoint{1.064759in}{1.684721in}}%
\pgfpathlineto{\pgfqpoint{1.063566in}{1.683886in}}%
\pgfpathlineto{\pgfqpoint{1.055028in}{1.678254in}}%
\pgfpathlineto{\pgfqpoint{1.048882in}{1.674884in}}%
\pgfpathlineto{\pgfqpoint{1.045296in}{1.672895in}}%
\pgfpathlineto{\pgfqpoint{1.035565in}{1.669046in}}%
\pgfpathlineto{\pgfqpoint{1.025833in}{1.667034in}}%
\pgfpathlineto{\pgfqpoint{1.016102in}{1.666960in}}%
\pgfpathlineto{\pgfqpoint{1.006370in}{1.668723in}}%
\pgfpathlineto{\pgfqpoint{0.996639in}{1.672035in}}%
\pgfpathlineto{\pgfqpoint{0.990347in}{1.674884in}}%
\pgfpathlineto{\pgfqpoint{0.986907in}{1.676424in}}%
\pgfpathlineto{\pgfqpoint{0.977176in}{1.681301in}}%
\pgfpathlineto{\pgfqpoint{0.971908in}{1.683886in}}%
\pgfpathlineto{\pgfqpoint{0.967444in}{1.686210in}}%
\pgfpathlineto{\pgfqpoint{0.957713in}{1.690506in}}%
\pgfpathlineto{\pgfqpoint{0.949794in}{1.692887in}}%
\pgfpathlineto{\pgfqpoint{0.947981in}{1.693512in}}%
\pgfpathlineto{\pgfqpoint{0.938250in}{1.694756in}}%
\pgfpathlineto{\pgfqpoint{0.928518in}{1.693537in}}%
\pgfpathlineto{\pgfqpoint{0.926854in}{1.692887in}}%
\pgfpathlineto{\pgfqpoint{0.918787in}{1.690139in}}%
\pgfpathlineto{\pgfqpoint{0.909055in}{1.684678in}}%
\pgfpathlineto{\pgfqpoint{0.908000in}{1.683886in}}%
\pgfpathlineto{\pgfqpoint{0.899324in}{1.677750in}}%
\pgfpathlineto{\pgfqpoint{0.895911in}{1.674884in}}%
\pgfpathlineto{\pgfqpoint{0.889592in}{1.669517in}}%
\pgfpathlineto{\pgfqpoint{0.885663in}{1.665883in}}%
\pgfpathlineto{\pgfqpoint{0.879861in}{1.660038in}}%
\pgfpathlineto{\pgfqpoint{0.876763in}{1.656881in}}%
\pgfpathlineto{\pgfqpoint{0.870129in}{1.648855in}}%
\pgfpathlineto{\pgfqpoint{0.869273in}{1.647880in}}%
\pgfpathlineto{\pgfqpoint{0.863369in}{1.638878in}}%
\pgfpathlineto{\pgfqpoint{0.860398in}{1.631416in}}%
\pgfpathlineto{\pgfqpoint{0.859695in}{1.629876in}}%
\pgfpathlineto{\pgfqpoint{0.858378in}{1.620875in}}%
\pgfpathlineto{\pgfqpoint{0.859722in}{1.611873in}}%
\pgfpathlineto{\pgfqpoint{0.860398in}{1.610196in}}%
\pgfpathlineto{\pgfqpoint{0.862973in}{1.602872in}}%
\pgfpathlineto{\pgfqpoint{0.867617in}{1.593870in}}%
\pgfpathlineto{\pgfqpoint{0.870129in}{1.589742in}}%
\pgfpathlineto{\pgfqpoint{0.872924in}{1.584869in}}%
\pgfpathlineto{\pgfqpoint{0.878196in}{1.575867in}}%
\pgfpathlineto{\pgfqpoint{0.879861in}{1.572686in}}%
\pgfpathlineto{\pgfqpoint{0.882942in}{1.566866in}}%
\pgfpathlineto{\pgfqpoint{0.886521in}{1.557864in}}%
\pgfpathlineto{\pgfqpoint{0.888428in}{1.548863in}}%
\pgfpathlineto{\pgfqpoint{0.888348in}{1.539861in}}%
\pgfpathlineto{\pgfqpoint{0.886173in}{1.530859in}}%
\pgfpathlineto{\pgfqpoint{0.882011in}{1.521858in}}%
\pgfpathlineto{\pgfqpoint{0.879861in}{1.518541in}}%
\pgfpathlineto{\pgfqpoint{0.876218in}{1.512856in}}%
\pgfpathlineto{\pgfqpoint{0.870129in}{1.504958in}}%
\pgfpathlineto{\pgfqpoint{0.869227in}{1.503855in}}%
\pgfpathlineto{\pgfqpoint{0.861264in}{1.494853in}}%
\pgfpathlineto{\pgfqpoint{0.860398in}{1.493859in}}%
\pgfpathlineto{\pgfqpoint{0.852398in}{1.485852in}}%
\pgfpathlineto{\pgfqpoint{0.850667in}{1.483913in}}%
\pgfpathlineto{\pgfqpoint{0.842716in}{1.476850in}}%
\pgfpathlineto{\pgfqpoint{0.840935in}{1.474844in}}%
\pgfpathlineto{\pgfqpoint{0.831937in}{1.467849in}}%
\pgfpathlineto{\pgfqpoint{0.831204in}{1.466958in}}%
\pgfpathlineto{\pgfqpoint{0.821472in}{1.460724in}}%
\pgfpathlineto{\pgfqpoint{0.811741in}{1.459794in}}%
\pgfpathlineto{\pgfqpoint{0.802009in}{1.463329in}}%
\pgfpathlineto{\pgfqpoint{0.795442in}{1.467849in}}%
\pgfpathlineto{\pgfqpoint{0.792278in}{1.469242in}}%
\pgfpathlineto{\pgfqpoint{0.782546in}{1.474500in}}%
\pgfpathlineto{\pgfqpoint{0.778030in}{1.476850in}}%
\pgfpathlineto{\pgfqpoint{0.772815in}{1.478992in}}%
\pgfpathlineto{\pgfqpoint{0.763083in}{1.481917in}}%
\pgfpathlineto{\pgfqpoint{0.753352in}{1.482942in}}%
\pgfpathlineto{\pgfqpoint{0.743620in}{1.481408in}}%
\pgfpathlineto{\pgfqpoint{0.733889in}{1.476895in}}%
\pgfpathlineto{\pgfqpoint{0.733832in}{1.476850in}}%
\pgfpathlineto{\pgfqpoint{0.724786in}{1.467849in}}%
\pgfpathlineto{\pgfqpoint{0.724157in}{1.466871in}}%
\pgfpathlineto{\pgfqpoint{0.720446in}{1.458847in}}%
\pgfpathlineto{\pgfqpoint{0.719014in}{1.449846in}}%
\pgfpathlineto{\pgfqpoint{0.720446in}{1.440844in}}%
\pgfpathlineto{\pgfqpoint{0.724157in}{1.432820in}}%
\pgfpathlineto{\pgfqpoint{0.724786in}{1.431842in}}%
\pgfpathlineto{\pgfqpoint{0.733832in}{1.422841in}}%
\pgfpathlineto{\pgfqpoint{0.733889in}{1.422796in}}%
\pgfpathlineto{\pgfqpoint{0.743620in}{1.418283in}}%
\pgfpathlineto{\pgfqpoint{0.753352in}{1.416750in}}%
\pgfpathlineto{\pgfqpoint{0.763083in}{1.417774in}}%
\pgfpathlineto{\pgfqpoint{0.772815in}{1.420699in}}%
\pgfpathlineto{\pgfqpoint{0.778030in}{1.422841in}}%
\pgfpathlineto{\pgfqpoint{0.782546in}{1.425191in}}%
\pgfpathlineto{\pgfqpoint{0.792278in}{1.430449in}}%
\pgfpathlineto{\pgfqpoint{0.795442in}{1.431842in}}%
\pgfpathlineto{\pgfqpoint{0.802009in}{1.436362in}}%
\pgfpathlineto{\pgfqpoint{0.811741in}{1.439898in}}%
\pgfpathlineto{\pgfqpoint{0.821472in}{1.438967in}}%
\pgfpathlineto{\pgfqpoint{0.831204in}{1.432733in}}%
\pgfpathlineto{\pgfqpoint{0.831937in}{1.431842in}}%
\pgfpathlineto{\pgfqpoint{0.840935in}{1.424847in}}%
\pgfpathlineto{\pgfqpoint{0.842716in}{1.422841in}}%
\pgfpathlineto{\pgfqpoint{0.850667in}{1.415778in}}%
\pgfpathlineto{\pgfqpoint{0.852398in}{1.413839in}}%
\pgfpathlineto{\pgfqpoint{0.860398in}{1.405832in}}%
\pgfpathlineto{\pgfqpoint{0.861264in}{1.404838in}}%
\pgfpathlineto{\pgfqpoint{0.869227in}{1.395836in}}%
\pgfpathlineto{\pgfqpoint{0.870129in}{1.394733in}}%
\pgfpathlineto{\pgfqpoint{0.876218in}{1.386835in}}%
\pgfpathlineto{\pgfqpoint{0.879861in}{1.381150in}}%
\pgfpathlineto{\pgfqpoint{0.882011in}{1.377833in}}%
\pgfpathlineto{\pgfqpoint{0.886173in}{1.368832in}}%
\pgfpathlineto{\pgfqpoint{0.888348in}{1.359830in}}%
\pgfpathlineto{\pgfqpoint{0.888428in}{1.350829in}}%
\pgfpathlineto{\pgfqpoint{0.886521in}{1.341827in}}%
\pgfpathlineto{\pgfqpoint{0.882942in}{1.332825in}}%
\pgfpathlineto{\pgfqpoint{0.879861in}{1.327005in}}%
\pgfpathlineto{\pgfqpoint{0.878196in}{1.323824in}}%
\pgfpathlineto{\pgfqpoint{0.872924in}{1.314822in}}%
\pgfpathlineto{\pgfqpoint{0.870129in}{1.309949in}}%
\pgfpathlineto{\pgfqpoint{0.867617in}{1.305821in}}%
\pgfpathlineto{\pgfqpoint{0.862973in}{1.296819in}}%
\pgfpathlineto{\pgfqpoint{0.860398in}{1.289495in}}%
\pgfpathlineto{\pgfqpoint{0.859722in}{1.287818in}}%
\pgfpathlineto{\pgfqpoint{0.858378in}{1.278816in}}%
\pgfpathlineto{\pgfqpoint{0.859695in}{1.269815in}}%
\pgfpathlineto{\pgfqpoint{0.860398in}{1.268275in}}%
\pgfpathlineto{\pgfqpoint{0.863369in}{1.260813in}}%
\pgfpathlineto{\pgfqpoint{0.869273in}{1.251812in}}%
\pgfpathlineto{\pgfqpoint{0.870129in}{1.250836in}}%
\pgfpathlineto{\pgfqpoint{0.876763in}{1.242810in}}%
\pgfpathlineto{\pgfqpoint{0.879861in}{1.239653in}}%
\pgfpathlineto{\pgfqpoint{0.885663in}{1.233808in}}%
\pgfpathlineto{\pgfqpoint{0.889592in}{1.230174in}}%
\pgfpathlineto{\pgfqpoint{0.895911in}{1.224807in}}%
\pgfpathlineto{\pgfqpoint{0.899324in}{1.221941in}}%
\pgfpathlineto{\pgfqpoint{0.908000in}{1.215805in}}%
\pgfpathlineto{\pgfqpoint{0.909055in}{1.215013in}}%
\pgfpathlineto{\pgfqpoint{0.918787in}{1.209552in}}%
\pgfpathlineto{\pgfqpoint{0.926854in}{1.206804in}}%
\pgfpathlineto{\pgfqpoint{0.928518in}{1.206154in}}%
\pgfpathlineto{\pgfqpoint{0.938250in}{1.204936in}}%
\pgfpathlineto{\pgfqpoint{0.947981in}{1.206179in}}%
\pgfpathlineto{\pgfqpoint{0.949794in}{1.206804in}}%
\pgfpathlineto{\pgfqpoint{0.957713in}{1.209186in}}%
\pgfpathlineto{\pgfqpoint{0.967444in}{1.213481in}}%
\pgfpathlineto{\pgfqpoint{0.971908in}{1.215805in}}%
\pgfpathlineto{\pgfqpoint{0.977176in}{1.218390in}}%
\pgfpathlineto{\pgfqpoint{0.986907in}{1.223267in}}%
\pgfpathlineto{\pgfqpoint{0.990347in}{1.224807in}}%
\pgfpathlineto{\pgfqpoint{0.996639in}{1.227656in}}%
\pgfpathlineto{\pgfqpoint{1.006370in}{1.230968in}}%
\pgfpathlineto{\pgfqpoint{1.016102in}{1.232731in}}%
\pgfpathlineto{\pgfqpoint{1.025833in}{1.232657in}}%
\pgfpathlineto{\pgfqpoint{1.035565in}{1.230645in}}%
\pgfpathlineto{\pgfqpoint{1.045296in}{1.226796in}}%
\pgfpathlineto{\pgfqpoint{1.048882in}{1.224807in}}%
\pgfpathlineto{\pgfqpoint{1.055028in}{1.221437in}}%
\pgfpathlineto{\pgfqpoint{1.063566in}{1.215805in}}%
\pgfpathlineto{\pgfqpoint{1.064759in}{1.214970in}}%
\pgfpathlineto{\pgfqpoint{1.074491in}{1.207605in}}%
\pgfpathlineto{\pgfqpoint{1.075565in}{1.206804in}}%
\pgfpathlineto{\pgfqpoint{1.084222in}{1.199404in}}%
\pgfpathlineto{\pgfqpoint{1.086318in}{1.197802in}}%
\pgfpathlineto{\pgfqpoint{1.093953in}{1.190449in}}%
\pgfpathlineto{\pgfqpoint{1.096122in}{1.188801in}}%
\pgfpathlineto{\pgfqpoint{1.103685in}{1.180477in}}%
\pgfpathlineto{\pgfqpoint{1.104648in}{1.179799in}}%
\pgfpathlineto{\pgfqpoint{1.111387in}{1.170798in}}%
\pgfpathlineto{\pgfqpoint{1.112393in}{1.161796in}}%
\pgfpathlineto{\pgfqpoint{1.108571in}{1.152795in}}%
\pgfpathlineto{\pgfqpoint{1.103685in}{1.146720in}}%
\pgfpathlineto{\pgfqpoint{1.102178in}{1.143793in}}%
\pgfpathlineto{\pgfqpoint{1.096495in}{1.134792in}}%
\pgfpathlineto{\pgfqpoint{1.093953in}{1.130614in}}%
\pgfpathlineto{\pgfqpoint{1.091638in}{1.125790in}}%
\pgfpathlineto{\pgfqpoint{1.088476in}{1.116788in}}%
\pgfpathlineto{\pgfqpoint{1.087368in}{1.107787in}}%
\pgfpathlineto{\pgfqpoint{1.089026in}{1.098785in}}%
\pgfpathlineto{\pgfqpoint{1.093905in}{1.089784in}}%
\pgfpathlineto{\pgfqpoint{1.093953in}{1.089731in}}%
\pgfpathlineto{\pgfqpoint{1.103685in}{1.081364in}}%
\pgfpathlineto{\pgfqpoint{1.104742in}{1.080782in}}%
\pgfpathclose%
\pgfpathmoveto{\pgfqpoint{1.099055in}{1.224807in}}%
\pgfpathlineto{\pgfqpoint{1.093953in}{1.227363in}}%
\pgfpathlineto{\pgfqpoint{1.084222in}{1.233543in}}%
\pgfpathlineto{\pgfqpoint{1.083849in}{1.233808in}}%
\pgfpathlineto{\pgfqpoint{1.074491in}{1.241050in}}%
\pgfpathlineto{\pgfqpoint{1.072251in}{1.242810in}}%
\pgfpathlineto{\pgfqpoint{1.064759in}{1.249797in}}%
\pgfpathlineto{\pgfqpoint{1.062424in}{1.251812in}}%
\pgfpathlineto{\pgfqpoint{1.055028in}{1.260346in}}%
\pgfpathlineto{\pgfqpoint{1.054541in}{1.260813in}}%
\pgfpathlineto{\pgfqpoint{1.048844in}{1.269815in}}%
\pgfpathlineto{\pgfqpoint{1.047018in}{1.278816in}}%
\pgfpathlineto{\pgfqpoint{1.048881in}{1.287818in}}%
\pgfpathlineto{\pgfqpoint{1.053911in}{1.296819in}}%
\pgfpathlineto{\pgfqpoint{1.055028in}{1.298180in}}%
\pgfpathlineto{\pgfqpoint{1.060236in}{1.305821in}}%
\pgfpathlineto{\pgfqpoint{1.064759in}{1.311446in}}%
\pgfpathlineto{\pgfqpoint{1.067270in}{1.314822in}}%
\pgfpathlineto{\pgfqpoint{1.074110in}{1.323824in}}%
\pgfpathlineto{\pgfqpoint{1.074491in}{1.324385in}}%
\pgfpathlineto{\pgfqpoint{1.080306in}{1.332825in}}%
\pgfpathlineto{\pgfqpoint{1.084222in}{1.340391in}}%
\pgfpathlineto{\pgfqpoint{1.085054in}{1.341827in}}%
\pgfpathlineto{\pgfqpoint{1.087831in}{1.350829in}}%
\pgfpathlineto{\pgfqpoint{1.087714in}{1.359830in}}%
\pgfpathlineto{\pgfqpoint{1.084546in}{1.368832in}}%
\pgfpathlineto{\pgfqpoint{1.084222in}{1.369313in}}%
\pgfpathlineto{\pgfqpoint{1.079095in}{1.377833in}}%
\pgfpathlineto{\pgfqpoint{1.074491in}{1.383290in}}%
\pgfpathlineto{\pgfqpoint{1.071544in}{1.386835in}}%
\pgfpathlineto{\pgfqpoint{1.064759in}{1.393620in}}%
\pgfpathlineto{\pgfqpoint{1.062363in}{1.395836in}}%
\pgfpathlineto{\pgfqpoint{1.055028in}{1.402112in}}%
\pgfpathlineto{\pgfqpoint{1.051195in}{1.404838in}}%
\pgfpathlineto{\pgfqpoint{1.045296in}{1.409096in}}%
\pgfpathlineto{\pgfqpoint{1.036085in}{1.413839in}}%
\pgfpathlineto{\pgfqpoint{1.035565in}{1.414139in}}%
\pgfpathlineto{\pgfqpoint{1.025833in}{1.417069in}}%
\pgfpathlineto{\pgfqpoint{1.016102in}{1.417177in}}%
\pgfpathlineto{\pgfqpoint{1.006370in}{1.414609in}}%
\pgfpathlineto{\pgfqpoint{1.004818in}{1.413839in}}%
\pgfpathlineto{\pgfqpoint{0.996639in}{1.410217in}}%
\pgfpathlineto{\pgfqpoint{0.987513in}{1.404838in}}%
\pgfpathlineto{\pgfqpoint{0.986907in}{1.404486in}}%
\pgfpathlineto{\pgfqpoint{0.977176in}{1.398159in}}%
\pgfpathlineto{\pgfqpoint{0.973525in}{1.395836in}}%
\pgfpathlineto{\pgfqpoint{0.967444in}{1.391652in}}%
\pgfpathlineto{\pgfqpoint{0.959184in}{1.386835in}}%
\pgfpathlineto{\pgfqpoint{0.957713in}{1.385802in}}%
\pgfpathlineto{\pgfqpoint{0.947981in}{1.381150in}}%
\pgfpathlineto{\pgfqpoint{0.938250in}{1.379426in}}%
\pgfpathlineto{\pgfqpoint{0.928518in}{1.381115in}}%
\pgfpathlineto{\pgfqpoint{0.918787in}{1.386384in}}%
\pgfpathlineto{\pgfqpoint{0.918282in}{1.386835in}}%
\pgfpathlineto{\pgfqpoint{0.909055in}{1.393677in}}%
\pgfpathlineto{\pgfqpoint{0.906878in}{1.395836in}}%
\pgfpathlineto{\pgfqpoint{0.899324in}{1.402766in}}%
\pgfpathlineto{\pgfqpoint{0.897422in}{1.404838in}}%
\pgfpathlineto{\pgfqpoint{0.889592in}{1.413494in}}%
\pgfpathlineto{\pgfqpoint{0.889306in}{1.413839in}}%
\pgfpathlineto{\pgfqpoint{0.882624in}{1.422841in}}%
\pgfpathlineto{\pgfqpoint{0.879861in}{1.427560in}}%
\pgfpathlineto{\pgfqpoint{0.877382in}{1.431842in}}%
\pgfpathlineto{\pgfqpoint{0.874048in}{1.440844in}}%
\pgfpathlineto{\pgfqpoint{0.872901in}{1.449846in}}%
\pgfpathlineto{\pgfqpoint{0.874048in}{1.458847in}}%
\pgfpathlineto{\pgfqpoint{0.877382in}{1.467849in}}%
\pgfpathlineto{\pgfqpoint{0.879861in}{1.472131in}}%
\pgfpathlineto{\pgfqpoint{0.882624in}{1.476850in}}%
\pgfpathlineto{\pgfqpoint{0.889306in}{1.485852in}}%
\pgfpathlineto{\pgfqpoint{0.889592in}{1.486197in}}%
\pgfpathlineto{\pgfqpoint{0.897422in}{1.494853in}}%
\pgfpathlineto{\pgfqpoint{0.899324in}{1.496925in}}%
\pgfpathlineto{\pgfqpoint{0.906878in}{1.503855in}}%
\pgfpathlineto{\pgfqpoint{0.909055in}{1.506014in}}%
\pgfpathlineto{\pgfqpoint{0.918282in}{1.512856in}}%
\pgfpathlineto{\pgfqpoint{0.918787in}{1.513307in}}%
\pgfpathlineto{\pgfqpoint{0.928518in}{1.518576in}}%
\pgfpathlineto{\pgfqpoint{0.938250in}{1.520265in}}%
\pgfpathlineto{\pgfqpoint{0.947981in}{1.518541in}}%
\pgfpathlineto{\pgfqpoint{0.957713in}{1.513889in}}%
\pgfpathlineto{\pgfqpoint{0.959184in}{1.512856in}}%
\pgfpathlineto{\pgfqpoint{0.967444in}{1.508039in}}%
\pgfpathlineto{\pgfqpoint{0.973525in}{1.503855in}}%
\pgfpathlineto{\pgfqpoint{0.977176in}{1.501532in}}%
\pgfpathlineto{\pgfqpoint{0.986907in}{1.495205in}}%
\pgfpathlineto{\pgfqpoint{0.987513in}{1.494853in}}%
\pgfpathlineto{\pgfqpoint{0.996639in}{1.489474in}}%
\pgfpathlineto{\pgfqpoint{1.004818in}{1.485852in}}%
\pgfpathlineto{\pgfqpoint{1.006370in}{1.485082in}}%
\pgfpathlineto{\pgfqpoint{1.016102in}{1.482514in}}%
\pgfpathlineto{\pgfqpoint{1.025833in}{1.482622in}}%
\pgfpathlineto{\pgfqpoint{1.035565in}{1.485552in}}%
\pgfpathlineto{\pgfqpoint{1.036085in}{1.485852in}}%
\pgfpathlineto{\pgfqpoint{1.045296in}{1.490595in}}%
\pgfpathlineto{\pgfqpoint{1.051195in}{1.494853in}}%
\pgfpathlineto{\pgfqpoint{1.055028in}{1.497579in}}%
\pgfpathlineto{\pgfqpoint{1.062363in}{1.503855in}}%
\pgfpathlineto{\pgfqpoint{1.064759in}{1.506071in}}%
\pgfpathlineto{\pgfqpoint{1.071544in}{1.512856in}}%
\pgfpathlineto{\pgfqpoint{1.074491in}{1.516401in}}%
\pgfpathlineto{\pgfqpoint{1.079095in}{1.521858in}}%
\pgfpathlineto{\pgfqpoint{1.084222in}{1.530378in}}%
\pgfpathlineto{\pgfqpoint{1.084546in}{1.530859in}}%
\pgfpathlineto{\pgfqpoint{1.087714in}{1.539861in}}%
\pgfpathlineto{\pgfqpoint{1.087831in}{1.548863in}}%
\pgfpathlineto{\pgfqpoint{1.085054in}{1.557864in}}%
\pgfpathlineto{\pgfqpoint{1.084222in}{1.559300in}}%
\pgfpathlineto{\pgfqpoint{1.080306in}{1.566866in}}%
\pgfpathlineto{\pgfqpoint{1.074491in}{1.575307in}}%
\pgfpathlineto{\pgfqpoint{1.074110in}{1.575867in}}%
\pgfpathlineto{\pgfqpoint{1.067270in}{1.584869in}}%
\pgfpathlineto{\pgfqpoint{1.064759in}{1.588245in}}%
\pgfpathlineto{\pgfqpoint{1.060236in}{1.593870in}}%
\pgfpathlineto{\pgfqpoint{1.055028in}{1.601511in}}%
\pgfpathlineto{\pgfqpoint{1.053911in}{1.602872in}}%
\pgfpathlineto{\pgfqpoint{1.048881in}{1.611873in}}%
\pgfpathlineto{\pgfqpoint{1.047018in}{1.620875in}}%
\pgfpathlineto{\pgfqpoint{1.048844in}{1.629876in}}%
\pgfpathlineto{\pgfqpoint{1.054541in}{1.638878in}}%
\pgfpathlineto{\pgfqpoint{1.055028in}{1.639345in}}%
\pgfpathlineto{\pgfqpoint{1.062424in}{1.647880in}}%
\pgfpathlineto{\pgfqpoint{1.064759in}{1.649894in}}%
\pgfpathlineto{\pgfqpoint{1.072251in}{1.656881in}}%
\pgfpathlineto{\pgfqpoint{1.074491in}{1.658641in}}%
\pgfpathlineto{\pgfqpoint{1.083849in}{1.665883in}}%
\pgfpathlineto{\pgfqpoint{1.084222in}{1.666148in}}%
\pgfpathlineto{\pgfqpoint{1.093953in}{1.672328in}}%
\pgfpathlineto{\pgfqpoint{1.099055in}{1.674884in}}%
\pgfpathlineto{\pgfqpoint{1.103685in}{1.677177in}}%
\pgfpathlineto{\pgfqpoint{1.113416in}{1.680261in}}%
\pgfpathlineto{\pgfqpoint{1.123148in}{1.681322in}}%
\pgfpathlineto{\pgfqpoint{1.132879in}{1.680261in}}%
\pgfpathlineto{\pgfqpoint{1.142611in}{1.677177in}}%
\pgfpathlineto{\pgfqpoint{1.147240in}{1.674884in}}%
\pgfpathlineto{\pgfqpoint{1.152342in}{1.672328in}}%
\pgfpathlineto{\pgfqpoint{1.162074in}{1.666148in}}%
\pgfpathlineto{\pgfqpoint{1.162447in}{1.665883in}}%
\pgfpathlineto{\pgfqpoint{1.171805in}{1.658641in}}%
\pgfpathlineto{\pgfqpoint{1.174045in}{1.656881in}}%
\pgfpathlineto{\pgfqpoint{1.181537in}{1.649894in}}%
\pgfpathlineto{\pgfqpoint{1.183871in}{1.647880in}}%
\pgfpathlineto{\pgfqpoint{1.191268in}{1.639345in}}%
\pgfpathlineto{\pgfqpoint{1.191755in}{1.638878in}}%
\pgfpathlineto{\pgfqpoint{1.197451in}{1.629876in}}%
\pgfpathlineto{\pgfqpoint{1.199278in}{1.620875in}}%
\pgfpathlineto{\pgfqpoint{1.197414in}{1.611873in}}%
\pgfpathlineto{\pgfqpoint{1.192384in}{1.602872in}}%
\pgfpathlineto{\pgfqpoint{1.191268in}{1.601511in}}%
\pgfpathlineto{\pgfqpoint{1.186060in}{1.593870in}}%
\pgfpathlineto{\pgfqpoint{1.181537in}{1.588245in}}%
\pgfpathlineto{\pgfqpoint{1.179025in}{1.584869in}}%
\pgfpathlineto{\pgfqpoint{1.172186in}{1.575867in}}%
\pgfpathlineto{\pgfqpoint{1.171805in}{1.575307in}}%
\pgfpathlineto{\pgfqpoint{1.165990in}{1.566866in}}%
\pgfpathlineto{\pgfqpoint{1.162074in}{1.559300in}}%
\pgfpathlineto{\pgfqpoint{1.161242in}{1.557864in}}%
\pgfpathlineto{\pgfqpoint{1.158465in}{1.548863in}}%
\pgfpathlineto{\pgfqpoint{1.158582in}{1.539861in}}%
\pgfpathlineto{\pgfqpoint{1.161750in}{1.530859in}}%
\pgfpathlineto{\pgfqpoint{1.162074in}{1.530378in}}%
\pgfpathlineto{\pgfqpoint{1.167201in}{1.521858in}}%
\pgfpathlineto{\pgfqpoint{1.171805in}{1.516401in}}%
\pgfpathlineto{\pgfqpoint{1.174752in}{1.512856in}}%
\pgfpathlineto{\pgfqpoint{1.181537in}{1.506071in}}%
\pgfpathlineto{\pgfqpoint{1.183933in}{1.503855in}}%
\pgfpathlineto{\pgfqpoint{1.191268in}{1.497579in}}%
\pgfpathlineto{\pgfqpoint{1.195101in}{1.494853in}}%
\pgfpathlineto{\pgfqpoint{1.201000in}{1.490595in}}%
\pgfpathlineto{\pgfqpoint{1.210211in}{1.485852in}}%
\pgfpathlineto{\pgfqpoint{1.210731in}{1.485552in}}%
\pgfpathlineto{\pgfqpoint{1.220463in}{1.482622in}}%
\pgfpathlineto{\pgfqpoint{1.230194in}{1.482514in}}%
\pgfpathlineto{\pgfqpoint{1.239926in}{1.485082in}}%
\pgfpathlineto{\pgfqpoint{1.241478in}{1.485852in}}%
\pgfpathlineto{\pgfqpoint{1.249657in}{1.489474in}}%
\pgfpathlineto{\pgfqpoint{1.258783in}{1.494853in}}%
\pgfpathlineto{\pgfqpoint{1.259389in}{1.495205in}}%
\pgfpathlineto{\pgfqpoint{1.269120in}{1.501532in}}%
\pgfpathlineto{\pgfqpoint{1.272770in}{1.503855in}}%
\pgfpathlineto{\pgfqpoint{1.278852in}{1.508039in}}%
\pgfpathlineto{\pgfqpoint{1.287112in}{1.512856in}}%
\pgfpathlineto{\pgfqpoint{1.288583in}{1.513889in}}%
\pgfpathlineto{\pgfqpoint{1.298315in}{1.518541in}}%
\pgfpathlineto{\pgfqpoint{1.308046in}{1.520265in}}%
\pgfpathlineto{\pgfqpoint{1.317778in}{1.518576in}}%
\pgfpathlineto{\pgfqpoint{1.327509in}{1.513307in}}%
\pgfpathlineto{\pgfqpoint{1.328014in}{1.512856in}}%
\pgfpathlineto{\pgfqpoint{1.337240in}{1.506014in}}%
\pgfpathlineto{\pgfqpoint{1.339418in}{1.503855in}}%
\pgfpathlineto{\pgfqpoint{1.346972in}{1.496925in}}%
\pgfpathlineto{\pgfqpoint{1.348874in}{1.494853in}}%
\pgfpathlineto{\pgfqpoint{1.356703in}{1.486197in}}%
\pgfpathlineto{\pgfqpoint{1.356990in}{1.485852in}}%
\pgfpathlineto{\pgfqpoint{1.363671in}{1.476850in}}%
\pgfpathlineto{\pgfqpoint{1.366435in}{1.472131in}}%
\pgfpathlineto{\pgfqpoint{1.368914in}{1.467849in}}%
\pgfpathlineto{\pgfqpoint{1.372248in}{1.458847in}}%
\pgfpathlineto{\pgfqpoint{1.373394in}{1.449846in}}%
\pgfpathlineto{\pgfqpoint{1.372248in}{1.440844in}}%
\pgfpathlineto{\pgfqpoint{1.368914in}{1.431842in}}%
\pgfpathlineto{\pgfqpoint{1.366435in}{1.427560in}}%
\pgfpathlineto{\pgfqpoint{1.363671in}{1.422841in}}%
\pgfpathlineto{\pgfqpoint{1.356990in}{1.413839in}}%
\pgfpathlineto{\pgfqpoint{1.356703in}{1.413494in}}%
\pgfpathlineto{\pgfqpoint{1.348874in}{1.404838in}}%
\pgfpathlineto{\pgfqpoint{1.346972in}{1.402766in}}%
\pgfpathlineto{\pgfqpoint{1.339418in}{1.395836in}}%
\pgfpathlineto{\pgfqpoint{1.337240in}{1.393677in}}%
\pgfpathlineto{\pgfqpoint{1.328014in}{1.386835in}}%
\pgfpathlineto{\pgfqpoint{1.327509in}{1.386384in}}%
\pgfpathlineto{\pgfqpoint{1.317778in}{1.381115in}}%
\pgfpathlineto{\pgfqpoint{1.308046in}{1.379426in}}%
\pgfpathlineto{\pgfqpoint{1.298315in}{1.381150in}}%
\pgfpathlineto{\pgfqpoint{1.288583in}{1.385802in}}%
\pgfpathlineto{\pgfqpoint{1.287112in}{1.386835in}}%
\pgfpathlineto{\pgfqpoint{1.278852in}{1.391652in}}%
\pgfpathlineto{\pgfqpoint{1.272770in}{1.395836in}}%
\pgfpathlineto{\pgfqpoint{1.269120in}{1.398159in}}%
\pgfpathlineto{\pgfqpoint{1.259389in}{1.404486in}}%
\pgfpathlineto{\pgfqpoint{1.258783in}{1.404838in}}%
\pgfpathlineto{\pgfqpoint{1.249657in}{1.410217in}}%
\pgfpathlineto{\pgfqpoint{1.241478in}{1.413839in}}%
\pgfpathlineto{\pgfqpoint{1.239926in}{1.414609in}}%
\pgfpathlineto{\pgfqpoint{1.230194in}{1.417177in}}%
\pgfpathlineto{\pgfqpoint{1.220463in}{1.417069in}}%
\pgfpathlineto{\pgfqpoint{1.210731in}{1.414139in}}%
\pgfpathlineto{\pgfqpoint{1.210211in}{1.413839in}}%
\pgfpathlineto{\pgfqpoint{1.201000in}{1.409096in}}%
\pgfpathlineto{\pgfqpoint{1.195101in}{1.404838in}}%
\pgfpathlineto{\pgfqpoint{1.191268in}{1.402112in}}%
\pgfpathlineto{\pgfqpoint{1.183933in}{1.395836in}}%
\pgfpathlineto{\pgfqpoint{1.181537in}{1.393620in}}%
\pgfpathlineto{\pgfqpoint{1.174752in}{1.386835in}}%
\pgfpathlineto{\pgfqpoint{1.171805in}{1.383290in}}%
\pgfpathlineto{\pgfqpoint{1.167201in}{1.377833in}}%
\pgfpathlineto{\pgfqpoint{1.162074in}{1.369313in}}%
\pgfpathlineto{\pgfqpoint{1.161750in}{1.368832in}}%
\pgfpathlineto{\pgfqpoint{1.158582in}{1.359830in}}%
\pgfpathlineto{\pgfqpoint{1.158465in}{1.350829in}}%
\pgfpathlineto{\pgfqpoint{1.161242in}{1.341827in}}%
\pgfpathlineto{\pgfqpoint{1.162074in}{1.340391in}}%
\pgfpathlineto{\pgfqpoint{1.165990in}{1.332825in}}%
\pgfpathlineto{\pgfqpoint{1.171805in}{1.324385in}}%
\pgfpathlineto{\pgfqpoint{1.172186in}{1.323824in}}%
\pgfpathlineto{\pgfqpoint{1.179025in}{1.314822in}}%
\pgfpathlineto{\pgfqpoint{1.181537in}{1.311446in}}%
\pgfpathlineto{\pgfqpoint{1.186060in}{1.305821in}}%
\pgfpathlineto{\pgfqpoint{1.191268in}{1.298180in}}%
\pgfpathlineto{\pgfqpoint{1.192384in}{1.296819in}}%
\pgfpathlineto{\pgfqpoint{1.197414in}{1.287818in}}%
\pgfpathlineto{\pgfqpoint{1.199278in}{1.278816in}}%
\pgfpathlineto{\pgfqpoint{1.197451in}{1.269815in}}%
\pgfpathlineto{\pgfqpoint{1.191755in}{1.260813in}}%
\pgfpathlineto{\pgfqpoint{1.191268in}{1.260346in}}%
\pgfpathlineto{\pgfqpoint{1.183871in}{1.251812in}}%
\pgfpathlineto{\pgfqpoint{1.181537in}{1.249797in}}%
\pgfpathlineto{\pgfqpoint{1.174045in}{1.242810in}}%
\pgfpathlineto{\pgfqpoint{1.171805in}{1.241050in}}%
\pgfpathlineto{\pgfqpoint{1.162447in}{1.233808in}}%
\pgfpathlineto{\pgfqpoint{1.162074in}{1.233543in}}%
\pgfpathlineto{\pgfqpoint{1.152342in}{1.227363in}}%
\pgfpathlineto{\pgfqpoint{1.147240in}{1.224807in}}%
\pgfpathlineto{\pgfqpoint{1.142611in}{1.222514in}}%
\pgfpathlineto{\pgfqpoint{1.132879in}{1.219430in}}%
\pgfpathlineto{\pgfqpoint{1.123148in}{1.218369in}}%
\pgfpathlineto{\pgfqpoint{1.113416in}{1.219430in}}%
\pgfpathlineto{\pgfqpoint{1.103685in}{1.222514in}}%
\pgfpathclose%
\pgfpathmoveto{\pgfqpoint{0.919525in}{1.242810in}}%
\pgfpathlineto{\pgfqpoint{0.918787in}{1.243120in}}%
\pgfpathlineto{\pgfqpoint{0.909055in}{1.249850in}}%
\pgfpathlineto{\pgfqpoint{0.906935in}{1.251812in}}%
\pgfpathlineto{\pgfqpoint{0.899659in}{1.260813in}}%
\pgfpathlineto{\pgfqpoint{0.899324in}{1.261496in}}%
\pgfpathlineto{\pgfqpoint{0.895887in}{1.269815in}}%
\pgfpathlineto{\pgfqpoint{0.894694in}{1.278816in}}%
\pgfpathlineto{\pgfqpoint{0.895911in}{1.287818in}}%
\pgfpathlineto{\pgfqpoint{0.899195in}{1.296819in}}%
\pgfpathlineto{\pgfqpoint{0.899324in}{1.297060in}}%
\pgfpathlineto{\pgfqpoint{0.904894in}{1.305821in}}%
\pgfpathlineto{\pgfqpoint{0.909055in}{1.311369in}}%
\pgfpathlineto{\pgfqpoint{0.912519in}{1.314822in}}%
\pgfpathlineto{\pgfqpoint{0.918787in}{1.320940in}}%
\pgfpathlineto{\pgfqpoint{0.923657in}{1.323824in}}%
\pgfpathlineto{\pgfqpoint{0.928518in}{1.327048in}}%
\pgfpathlineto{\pgfqpoint{0.938250in}{1.329116in}}%
\pgfpathlineto{\pgfqpoint{0.947981in}{1.327006in}}%
\pgfpathlineto{\pgfqpoint{0.953415in}{1.323824in}}%
\pgfpathlineto{\pgfqpoint{0.957713in}{1.321576in}}%
\pgfpathlineto{\pgfqpoint{0.966512in}{1.314822in}}%
\pgfpathlineto{\pgfqpoint{0.967444in}{1.314091in}}%
\pgfpathlineto{\pgfqpoint{0.976385in}{1.305821in}}%
\pgfpathlineto{\pgfqpoint{0.977176in}{1.304958in}}%
\pgfpathlineto{\pgfqpoint{0.984477in}{1.296819in}}%
\pgfpathlineto{\pgfqpoint{0.986907in}{1.292844in}}%
\pgfpathlineto{\pgfqpoint{0.990347in}{1.287818in}}%
\pgfpathlineto{\pgfqpoint{0.992629in}{1.278816in}}%
\pgfpathlineto{\pgfqpoint{0.990392in}{1.269815in}}%
\pgfpathlineto{\pgfqpoint{0.986907in}{1.265318in}}%
\pgfpathlineto{\pgfqpoint{0.983789in}{1.260813in}}%
\pgfpathlineto{\pgfqpoint{0.977176in}{1.255015in}}%
\pgfpathlineto{\pgfqpoint{0.973443in}{1.251812in}}%
\pgfpathlineto{\pgfqpoint{0.967444in}{1.247962in}}%
\pgfpathlineto{\pgfqpoint{0.957973in}{1.242810in}}%
\pgfpathlineto{\pgfqpoint{0.957713in}{1.242691in}}%
\pgfpathlineto{\pgfqpoint{0.947981in}{1.239653in}}%
\pgfpathlineto{\pgfqpoint{0.938250in}{1.238528in}}%
\pgfpathlineto{\pgfqpoint{0.928518in}{1.239631in}}%
\pgfpathclose%
\pgfpathmoveto{\pgfqpoint{1.288323in}{1.242810in}}%
\pgfpathlineto{\pgfqpoint{1.278852in}{1.247962in}}%
\pgfpathlineto{\pgfqpoint{1.272853in}{1.251812in}}%
\pgfpathlineto{\pgfqpoint{1.269120in}{1.255015in}}%
\pgfpathlineto{\pgfqpoint{1.262507in}{1.260813in}}%
\pgfpathlineto{\pgfqpoint{1.259389in}{1.265318in}}%
\pgfpathlineto{\pgfqpoint{1.255904in}{1.269815in}}%
\pgfpathlineto{\pgfqpoint{1.253667in}{1.278816in}}%
\pgfpathlineto{\pgfqpoint{1.255949in}{1.287818in}}%
\pgfpathlineto{\pgfqpoint{1.259389in}{1.292844in}}%
\pgfpathlineto{\pgfqpoint{1.261819in}{1.296819in}}%
\pgfpathlineto{\pgfqpoint{1.269120in}{1.304958in}}%
\pgfpathlineto{\pgfqpoint{1.269911in}{1.305821in}}%
\pgfpathlineto{\pgfqpoint{1.278852in}{1.314091in}}%
\pgfpathlineto{\pgfqpoint{1.279784in}{1.314822in}}%
\pgfpathlineto{\pgfqpoint{1.288583in}{1.321576in}}%
\pgfpathlineto{\pgfqpoint{1.292881in}{1.323824in}}%
\pgfpathlineto{\pgfqpoint{1.298315in}{1.327006in}}%
\pgfpathlineto{\pgfqpoint{1.308046in}{1.329116in}}%
\pgfpathlineto{\pgfqpoint{1.317778in}{1.327048in}}%
\pgfpathlineto{\pgfqpoint{1.322639in}{1.323824in}}%
\pgfpathlineto{\pgfqpoint{1.327509in}{1.320940in}}%
\pgfpathlineto{\pgfqpoint{1.333777in}{1.314822in}}%
\pgfpathlineto{\pgfqpoint{1.337240in}{1.311369in}}%
\pgfpathlineto{\pgfqpoint{1.341402in}{1.305821in}}%
\pgfpathlineto{\pgfqpoint{1.346972in}{1.297060in}}%
\pgfpathlineto{\pgfqpoint{1.347101in}{1.296819in}}%
\pgfpathlineto{\pgfqpoint{1.350385in}{1.287818in}}%
\pgfpathlineto{\pgfqpoint{1.351601in}{1.278816in}}%
\pgfpathlineto{\pgfqpoint{1.350409in}{1.269815in}}%
\pgfpathlineto{\pgfqpoint{1.346972in}{1.261496in}}%
\pgfpathlineto{\pgfqpoint{1.346637in}{1.260813in}}%
\pgfpathlineto{\pgfqpoint{1.339361in}{1.251812in}}%
\pgfpathlineto{\pgfqpoint{1.337240in}{1.249850in}}%
\pgfpathlineto{\pgfqpoint{1.327509in}{1.243120in}}%
\pgfpathlineto{\pgfqpoint{1.326771in}{1.242810in}}%
\pgfpathlineto{\pgfqpoint{1.317778in}{1.239631in}}%
\pgfpathlineto{\pgfqpoint{1.308046in}{1.238528in}}%
\pgfpathlineto{\pgfqpoint{1.298315in}{1.239653in}}%
\pgfpathlineto{\pgfqpoint{1.288583in}{1.242691in}}%
\pgfpathclose%
\pgfpathmoveto{\pgfqpoint{1.007937in}{1.350829in}}%
\pgfpathlineto{\pgfqpoint{1.008346in}{1.359830in}}%
\pgfpathlineto{\pgfqpoint{1.016102in}{1.366118in}}%
\pgfpathlineto{\pgfqpoint{1.025833in}{1.365786in}}%
\pgfpathlineto{\pgfqpoint{1.032272in}{1.359830in}}%
\pgfpathlineto{\pgfqpoint{1.032631in}{1.350829in}}%
\pgfpathlineto{\pgfqpoint{1.025833in}{1.343655in}}%
\pgfpathlineto{\pgfqpoint{1.016102in}{1.343276in}}%
\pgfpathclose%
\pgfpathmoveto{\pgfqpoint{1.213665in}{1.350829in}}%
\pgfpathlineto{\pgfqpoint{1.214023in}{1.359830in}}%
\pgfpathlineto{\pgfqpoint{1.220463in}{1.365786in}}%
\pgfpathlineto{\pgfqpoint{1.230194in}{1.366118in}}%
\pgfpathlineto{\pgfqpoint{1.237950in}{1.359830in}}%
\pgfpathlineto{\pgfqpoint{1.238359in}{1.350829in}}%
\pgfpathlineto{\pgfqpoint{1.230194in}{1.343276in}}%
\pgfpathlineto{\pgfqpoint{1.220463in}{1.343655in}}%
\pgfpathclose%
\pgfpathmoveto{\pgfqpoint{1.008346in}{1.539861in}}%
\pgfpathlineto{\pgfqpoint{1.007937in}{1.548863in}}%
\pgfpathlineto{\pgfqpoint{1.016102in}{1.556415in}}%
\pgfpathlineto{\pgfqpoint{1.025833in}{1.556036in}}%
\pgfpathlineto{\pgfqpoint{1.032631in}{1.548863in}}%
\pgfpathlineto{\pgfqpoint{1.032272in}{1.539861in}}%
\pgfpathlineto{\pgfqpoint{1.025833in}{1.533905in}}%
\pgfpathlineto{\pgfqpoint{1.016102in}{1.533573in}}%
\pgfpathclose%
\pgfpathmoveto{\pgfqpoint{1.214023in}{1.539861in}}%
\pgfpathlineto{\pgfqpoint{1.213665in}{1.548863in}}%
\pgfpathlineto{\pgfqpoint{1.220463in}{1.556036in}}%
\pgfpathlineto{\pgfqpoint{1.230194in}{1.556415in}}%
\pgfpathlineto{\pgfqpoint{1.238359in}{1.548863in}}%
\pgfpathlineto{\pgfqpoint{1.237950in}{1.539861in}}%
\pgfpathlineto{\pgfqpoint{1.230194in}{1.533573in}}%
\pgfpathlineto{\pgfqpoint{1.220463in}{1.533905in}}%
\pgfpathclose%
\pgfpathmoveto{\pgfqpoint{0.923657in}{1.575867in}}%
\pgfpathlineto{\pgfqpoint{0.918787in}{1.578751in}}%
\pgfpathlineto{\pgfqpoint{0.912519in}{1.584869in}}%
\pgfpathlineto{\pgfqpoint{0.909055in}{1.588322in}}%
\pgfpathlineto{\pgfqpoint{0.904894in}{1.593870in}}%
\pgfpathlineto{\pgfqpoint{0.899324in}{1.602631in}}%
\pgfpathlineto{\pgfqpoint{0.899195in}{1.602872in}}%
\pgfpathlineto{\pgfqpoint{0.895911in}{1.611873in}}%
\pgfpathlineto{\pgfqpoint{0.894694in}{1.620875in}}%
\pgfpathlineto{\pgfqpoint{0.895887in}{1.629876in}}%
\pgfpathlineto{\pgfqpoint{0.899324in}{1.638195in}}%
\pgfpathlineto{\pgfqpoint{0.899659in}{1.638878in}}%
\pgfpathlineto{\pgfqpoint{0.906935in}{1.647880in}}%
\pgfpathlineto{\pgfqpoint{0.909055in}{1.649841in}}%
\pgfpathlineto{\pgfqpoint{0.918787in}{1.656571in}}%
\pgfpathlineto{\pgfqpoint{0.919525in}{1.656881in}}%
\pgfpathlineto{\pgfqpoint{0.928518in}{1.660060in}}%
\pgfpathlineto{\pgfqpoint{0.938250in}{1.661163in}}%
\pgfpathlineto{\pgfqpoint{0.947981in}{1.660038in}}%
\pgfpathlineto{\pgfqpoint{0.957713in}{1.657000in}}%
\pgfpathlineto{\pgfqpoint{0.957973in}{1.656881in}}%
\pgfpathlineto{\pgfqpoint{0.967444in}{1.651729in}}%
\pgfpathlineto{\pgfqpoint{0.973443in}{1.647880in}}%
\pgfpathlineto{\pgfqpoint{0.977176in}{1.644676in}}%
\pgfpathlineto{\pgfqpoint{0.983789in}{1.638878in}}%
\pgfpathlineto{\pgfqpoint{0.986907in}{1.634373in}}%
\pgfpathlineto{\pgfqpoint{0.990392in}{1.629876in}}%
\pgfpathlineto{\pgfqpoint{0.992629in}{1.620875in}}%
\pgfpathlineto{\pgfqpoint{0.990347in}{1.611873in}}%
\pgfpathlineto{\pgfqpoint{0.986907in}{1.606847in}}%
\pgfpathlineto{\pgfqpoint{0.984477in}{1.602872in}}%
\pgfpathlineto{\pgfqpoint{0.977176in}{1.594733in}}%
\pgfpathlineto{\pgfqpoint{0.976385in}{1.593870in}}%
\pgfpathlineto{\pgfqpoint{0.967444in}{1.585600in}}%
\pgfpathlineto{\pgfqpoint{0.966512in}{1.584869in}}%
\pgfpathlineto{\pgfqpoint{0.957713in}{1.578115in}}%
\pgfpathlineto{\pgfqpoint{0.953415in}{1.575867in}}%
\pgfpathlineto{\pgfqpoint{0.947981in}{1.572685in}}%
\pgfpathlineto{\pgfqpoint{0.938250in}{1.570575in}}%
\pgfpathlineto{\pgfqpoint{0.928518in}{1.572644in}}%
\pgfpathclose%
\pgfpathmoveto{\pgfqpoint{1.292881in}{1.575867in}}%
\pgfpathlineto{\pgfqpoint{1.288583in}{1.578115in}}%
\pgfpathlineto{\pgfqpoint{1.279784in}{1.584869in}}%
\pgfpathlineto{\pgfqpoint{1.278852in}{1.585600in}}%
\pgfpathlineto{\pgfqpoint{1.269911in}{1.593870in}}%
\pgfpathlineto{\pgfqpoint{1.269120in}{1.594733in}}%
\pgfpathlineto{\pgfqpoint{1.261819in}{1.602872in}}%
\pgfpathlineto{\pgfqpoint{1.259389in}{1.606847in}}%
\pgfpathlineto{\pgfqpoint{1.255949in}{1.611873in}}%
\pgfpathlineto{\pgfqpoint{1.253667in}{1.620875in}}%
\pgfpathlineto{\pgfqpoint{1.255904in}{1.629876in}}%
\pgfpathlineto{\pgfqpoint{1.259389in}{1.634373in}}%
\pgfpathlineto{\pgfqpoint{1.262507in}{1.638878in}}%
\pgfpathlineto{\pgfqpoint{1.269120in}{1.644676in}}%
\pgfpathlineto{\pgfqpoint{1.272853in}{1.647880in}}%
\pgfpathlineto{\pgfqpoint{1.278852in}{1.651729in}}%
\pgfpathlineto{\pgfqpoint{1.288323in}{1.656881in}}%
\pgfpathlineto{\pgfqpoint{1.288583in}{1.657000in}}%
\pgfpathlineto{\pgfqpoint{1.298315in}{1.660038in}}%
\pgfpathlineto{\pgfqpoint{1.308046in}{1.661163in}}%
\pgfpathlineto{\pgfqpoint{1.317778in}{1.660060in}}%
\pgfpathlineto{\pgfqpoint{1.326771in}{1.656881in}}%
\pgfpathlineto{\pgfqpoint{1.327509in}{1.656571in}}%
\pgfpathlineto{\pgfqpoint{1.337240in}{1.649841in}}%
\pgfpathlineto{\pgfqpoint{1.339361in}{1.647880in}}%
\pgfpathlineto{\pgfqpoint{1.346637in}{1.638878in}}%
\pgfpathlineto{\pgfqpoint{1.346972in}{1.638195in}}%
\pgfpathlineto{\pgfqpoint{1.350409in}{1.629876in}}%
\pgfpathlineto{\pgfqpoint{1.351601in}{1.620875in}}%
\pgfpathlineto{\pgfqpoint{1.350385in}{1.611873in}}%
\pgfpathlineto{\pgfqpoint{1.347101in}{1.602872in}}%
\pgfpathlineto{\pgfqpoint{1.346972in}{1.602631in}}%
\pgfpathlineto{\pgfqpoint{1.341402in}{1.593870in}}%
\pgfpathlineto{\pgfqpoint{1.337240in}{1.588322in}}%
\pgfpathlineto{\pgfqpoint{1.333777in}{1.584869in}}%
\pgfpathlineto{\pgfqpoint{1.327509in}{1.578751in}}%
\pgfpathlineto{\pgfqpoint{1.322639in}{1.575867in}}%
\pgfpathlineto{\pgfqpoint{1.317778in}{1.572644in}}%
\pgfpathlineto{\pgfqpoint{1.308046in}{1.570575in}}%
\pgfpathlineto{\pgfqpoint{1.298315in}{1.572685in}}%
\pgfpathclose%
\pgfusepath{fill}%
\end{pgfscope}%
\begin{pgfscope}%
\pgfpathrectangle{\pgfqpoint{0.150000in}{0.549691in}}{\pgfqpoint{1.946296in}{1.800309in}}%
\pgfusepath{clip}%
\pgfsetbuttcap%
\pgfsetroundjoin%
\definecolor{currentfill}{rgb}{0.160784,0.160784,0.160784}%
\pgfsetfillcolor{currentfill}%
\pgfsetlinewidth{0.000000pt}%
\definecolor{currentstroke}{rgb}{0.000000,0.000000,0.000000}%
\pgfsetstrokecolor{currentstroke}%
\pgfsetdash{}{0pt}%
\pgfpathmoveto{\pgfqpoint{1.123148in}{1.035282in}}%
\pgfpathlineto{\pgfqpoint{1.128439in}{1.035775in}}%
\pgfpathlineto{\pgfqpoint{1.132879in}{1.036168in}}%
\pgfpathlineto{\pgfqpoint{1.142611in}{1.038677in}}%
\pgfpathlineto{\pgfqpoint{1.152342in}{1.042599in}}%
\pgfpathlineto{\pgfqpoint{1.156605in}{1.044776in}}%
\pgfpathlineto{\pgfqpoint{1.162074in}{1.047596in}}%
\pgfpathlineto{\pgfqpoint{1.171805in}{1.053210in}}%
\pgfpathlineto{\pgfqpoint{1.172774in}{1.053778in}}%
\pgfpathlineto{\pgfqpoint{1.181537in}{1.059274in}}%
\pgfpathlineto{\pgfqpoint{1.187579in}{1.062779in}}%
\pgfpathlineto{\pgfqpoint{1.191268in}{1.065228in}}%
\pgfpathlineto{\pgfqpoint{1.201000in}{1.070598in}}%
\pgfpathlineto{\pgfqpoint{1.204005in}{1.071781in}}%
\pgfpathlineto{\pgfqpoint{1.210731in}{1.075066in}}%
\pgfpathlineto{\pgfqpoint{1.220463in}{1.077549in}}%
\pgfpathlineto{\pgfqpoint{1.230194in}{1.077641in}}%
\pgfpathlineto{\pgfqpoint{1.239926in}{1.075464in}}%
\pgfpathlineto{\pgfqpoint{1.248694in}{1.071781in}}%
\pgfpathlineto{\pgfqpoint{1.249657in}{1.071454in}}%
\pgfpathlineto{\pgfqpoint{1.259389in}{1.067070in}}%
\pgfpathlineto{\pgfqpoint{1.267895in}{1.062779in}}%
\pgfpathlineto{\pgfqpoint{1.269120in}{1.062239in}}%
\pgfpathlineto{\pgfqpoint{1.278852in}{1.058040in}}%
\pgfpathlineto{\pgfqpoint{1.288583in}{1.054480in}}%
\pgfpathlineto{\pgfqpoint{1.291400in}{1.053778in}}%
\pgfpathlineto{\pgfqpoint{1.298315in}{1.052168in}}%
\pgfpathlineto{\pgfqpoint{1.308046in}{1.051329in}}%
\pgfpathlineto{\pgfqpoint{1.317778in}{1.052152in}}%
\pgfpathlineto{\pgfqpoint{1.323946in}{1.053778in}}%
\pgfpathlineto{\pgfqpoint{1.327509in}{1.054783in}}%
\pgfpathlineto{\pgfqpoint{1.337240in}{1.059309in}}%
\pgfpathlineto{\pgfqpoint{1.342818in}{1.062779in}}%
\pgfpathlineto{\pgfqpoint{1.346972in}{1.065735in}}%
\pgfpathlineto{\pgfqpoint{1.354125in}{1.071781in}}%
\pgfpathlineto{\pgfqpoint{1.356703in}{1.074484in}}%
\pgfpathlineto{\pgfqpoint{1.362219in}{1.080782in}}%
\pgfpathlineto{\pgfqpoint{1.366435in}{1.087475in}}%
\pgfpathlineto{\pgfqpoint{1.367873in}{1.089784in}}%
\pgfpathlineto{\pgfqpoint{1.371184in}{1.098785in}}%
\pgfpathlineto{\pgfqpoint{1.372309in}{1.107787in}}%
\pgfpathlineto{\pgfqpoint{1.371557in}{1.116788in}}%
\pgfpathlineto{\pgfqpoint{1.369411in}{1.125790in}}%
\pgfpathlineto{\pgfqpoint{1.366479in}{1.134792in}}%
\pgfpathlineto{\pgfqpoint{1.366435in}{1.134922in}}%
\pgfpathlineto{\pgfqpoint{1.363401in}{1.143793in}}%
\pgfpathlineto{\pgfqpoint{1.360892in}{1.152795in}}%
\pgfpathlineto{\pgfqpoint{1.359567in}{1.161796in}}%
\pgfpathlineto{\pgfqpoint{1.359916in}{1.170798in}}%
\pgfpathlineto{\pgfqpoint{1.362251in}{1.179799in}}%
\pgfpathlineto{\pgfqpoint{1.366435in}{1.188300in}}%
\pgfpathlineto{\pgfqpoint{1.366679in}{1.188801in}}%
\pgfpathlineto{\pgfqpoint{1.373022in}{1.197802in}}%
\pgfpathlineto{\pgfqpoint{1.376166in}{1.201342in}}%
\pgfpathlineto{\pgfqpoint{1.381317in}{1.206804in}}%
\pgfpathlineto{\pgfqpoint{1.385898in}{1.211041in}}%
\pgfpathlineto{\pgfqpoint{1.391803in}{1.215805in}}%
\pgfpathlineto{\pgfqpoint{1.395629in}{1.218714in}}%
\pgfpathlineto{\pgfqpoint{1.405361in}{1.224582in}}%
\pgfpathlineto{\pgfqpoint{1.405902in}{1.224807in}}%
\pgfpathlineto{\pgfqpoint{1.415092in}{1.228677in}}%
\pgfpathlineto{\pgfqpoint{1.424824in}{1.230837in}}%
\pgfpathlineto{\pgfqpoint{1.434555in}{1.231160in}}%
\pgfpathlineto{\pgfqpoint{1.444287in}{1.229934in}}%
\pgfpathlineto{\pgfqpoint{1.454018in}{1.227613in}}%
\pgfpathlineto{\pgfqpoint{1.463608in}{1.224807in}}%
\pgfpathlineto{\pgfqpoint{1.463750in}{1.224766in}}%
\pgfpathlineto{\pgfqpoint{1.473481in}{1.222054in}}%
\pgfpathlineto{\pgfqpoint{1.483213in}{1.220069in}}%
\pgfpathlineto{\pgfqpoint{1.492944in}{1.219373in}}%
\pgfpathlineto{\pgfqpoint{1.502676in}{1.220414in}}%
\pgfpathlineto{\pgfqpoint{1.512407in}{1.223477in}}%
\pgfpathlineto{\pgfqpoint{1.514903in}{1.224807in}}%
\pgfpathlineto{\pgfqpoint{1.522139in}{1.228707in}}%
\pgfpathlineto{\pgfqpoint{1.528947in}{1.233808in}}%
\pgfpathlineto{\pgfqpoint{1.531870in}{1.236193in}}%
\pgfpathlineto{\pgfqpoint{1.538406in}{1.242810in}}%
\pgfpathlineto{\pgfqpoint{1.541602in}{1.246652in}}%
\pgfpathlineto{\pgfqpoint{1.545353in}{1.251812in}}%
\pgfpathlineto{\pgfqpoint{1.550246in}{1.260813in}}%
\pgfpathlineto{\pgfqpoint{1.551333in}{1.264108in}}%
\pgfpathlineto{\pgfqpoint{1.553091in}{1.269815in}}%
\pgfpathlineto{\pgfqpoint{1.553980in}{1.278816in}}%
\pgfpathlineto{\pgfqpoint{1.553073in}{1.287818in}}%
\pgfpathlineto{\pgfqpoint{1.551333in}{1.294214in}}%
\pgfpathlineto{\pgfqpoint{1.550574in}{1.296819in}}%
\pgfpathlineto{\pgfqpoint{1.546725in}{1.305821in}}%
\pgfpathlineto{\pgfqpoint{1.542185in}{1.314822in}}%
\pgfpathlineto{\pgfqpoint{1.541602in}{1.315955in}}%
\pgfpathlineto{\pgfqpoint{1.536963in}{1.323824in}}%
\pgfpathlineto{\pgfqpoint{1.532223in}{1.332825in}}%
\pgfpathlineto{\pgfqpoint{1.531870in}{1.333717in}}%
\pgfpathlineto{\pgfqpoint{1.527888in}{1.341827in}}%
\pgfpathlineto{\pgfqpoint{1.525535in}{1.350829in}}%
\pgfpathlineto{\pgfqpoint{1.525634in}{1.359830in}}%
\pgfpathlineto{\pgfqpoint{1.528319in}{1.368832in}}%
\pgfpathlineto{\pgfqpoint{1.531870in}{1.375054in}}%
\pgfpathlineto{\pgfqpoint{1.533149in}{1.377833in}}%
\pgfpathlineto{\pgfqpoint{1.538954in}{1.386835in}}%
\pgfpathlineto{\pgfqpoint{1.541602in}{1.390248in}}%
\pgfpathlineto{\pgfqpoint{1.545391in}{1.395836in}}%
\pgfpathlineto{\pgfqpoint{1.551333in}{1.403942in}}%
\pgfpathlineto{\pgfqpoint{1.551946in}{1.404838in}}%
\pgfpathlineto{\pgfqpoint{1.558016in}{1.413839in}}%
\pgfpathlineto{\pgfqpoint{1.561064in}{1.418898in}}%
\pgfpathlineto{\pgfqpoint{1.563418in}{1.422841in}}%
\pgfpathlineto{\pgfqpoint{1.567658in}{1.431842in}}%
\pgfpathlineto{\pgfqpoint{1.570370in}{1.440844in}}%
\pgfpathlineto{\pgfqpoint{1.570796in}{1.444952in}}%
\pgfpathlineto{\pgfqpoint{1.571328in}{1.449846in}}%
\pgfpathlineto{\pgfqpoint{1.570796in}{1.454740in}}%
\pgfpathlineto{\pgfqpoint{1.570370in}{1.458847in}}%
\pgfpathlineto{\pgfqpoint{1.567658in}{1.467849in}}%
\pgfpathlineto{\pgfqpoint{1.563418in}{1.476850in}}%
\pgfpathlineto{\pgfqpoint{1.561064in}{1.480793in}}%
\pgfpathlineto{\pgfqpoint{1.558016in}{1.485852in}}%
\pgfpathlineto{\pgfqpoint{1.551946in}{1.494853in}}%
\pgfpathlineto{\pgfqpoint{1.551333in}{1.495749in}}%
\pgfpathlineto{\pgfqpoint{1.545391in}{1.503855in}}%
\pgfpathlineto{\pgfqpoint{1.541602in}{1.509444in}}%
\pgfpathlineto{\pgfqpoint{1.538954in}{1.512856in}}%
\pgfpathlineto{\pgfqpoint{1.533149in}{1.521858in}}%
\pgfpathlineto{\pgfqpoint{1.531870in}{1.524638in}}%
\pgfpathlineto{\pgfqpoint{1.528319in}{1.530859in}}%
\pgfpathlineto{\pgfqpoint{1.525634in}{1.539861in}}%
\pgfpathlineto{\pgfqpoint{1.525535in}{1.548863in}}%
\pgfpathlineto{\pgfqpoint{1.527888in}{1.557864in}}%
\pgfpathlineto{\pgfqpoint{1.531870in}{1.565974in}}%
\pgfpathlineto{\pgfqpoint{1.532223in}{1.566866in}}%
\pgfpathlineto{\pgfqpoint{1.536963in}{1.575867in}}%
\pgfpathlineto{\pgfqpoint{1.541602in}{1.583736in}}%
\pgfpathlineto{\pgfqpoint{1.542185in}{1.584869in}}%
\pgfpathlineto{\pgfqpoint{1.546725in}{1.593870in}}%
\pgfpathlineto{\pgfqpoint{1.550574in}{1.602872in}}%
\pgfpathlineto{\pgfqpoint{1.551333in}{1.605478in}}%
\pgfpathlineto{\pgfqpoint{1.553073in}{1.611873in}}%
\pgfpathlineto{\pgfqpoint{1.553980in}{1.620875in}}%
\pgfpathlineto{\pgfqpoint{1.553091in}{1.629876in}}%
\pgfpathlineto{\pgfqpoint{1.551333in}{1.635583in}}%
\pgfpathlineto{\pgfqpoint{1.550246in}{1.638878in}}%
\pgfpathlineto{\pgfqpoint{1.545353in}{1.647880in}}%
\pgfpathlineto{\pgfqpoint{1.541602in}{1.653039in}}%
\pgfpathlineto{\pgfqpoint{1.538406in}{1.656881in}}%
\pgfpathlineto{\pgfqpoint{1.531870in}{1.663498in}}%
\pgfpathlineto{\pgfqpoint{1.528947in}{1.665883in}}%
\pgfpathlineto{\pgfqpoint{1.522139in}{1.670984in}}%
\pgfpathlineto{\pgfqpoint{1.514903in}{1.674884in}}%
\pgfpathlineto{\pgfqpoint{1.512407in}{1.676214in}}%
\pgfpathlineto{\pgfqpoint{1.502676in}{1.679277in}}%
\pgfpathlineto{\pgfqpoint{1.492944in}{1.680318in}}%
\pgfpathlineto{\pgfqpoint{1.483213in}{1.679622in}}%
\pgfpathlineto{\pgfqpoint{1.473481in}{1.677637in}}%
\pgfpathlineto{\pgfqpoint{1.463750in}{1.674925in}}%
\pgfpathlineto{\pgfqpoint{1.463608in}{1.674884in}}%
\pgfpathlineto{\pgfqpoint{1.454018in}{1.672078in}}%
\pgfpathlineto{\pgfqpoint{1.444287in}{1.669757in}}%
\pgfpathlineto{\pgfqpoint{1.434555in}{1.668532in}}%
\pgfpathlineto{\pgfqpoint{1.424824in}{1.668854in}}%
\pgfpathlineto{\pgfqpoint{1.415092in}{1.671014in}}%
\pgfpathlineto{\pgfqpoint{1.405902in}{1.674884in}}%
\pgfpathlineto{\pgfqpoint{1.405361in}{1.675110in}}%
\pgfpathlineto{\pgfqpoint{1.395629in}{1.680977in}}%
\pgfpathlineto{\pgfqpoint{1.391803in}{1.683886in}}%
\pgfpathlineto{\pgfqpoint{1.385898in}{1.688650in}}%
\pgfpathlineto{\pgfqpoint{1.381317in}{1.692887in}}%
\pgfpathlineto{\pgfqpoint{1.376166in}{1.698350in}}%
\pgfpathlineto{\pgfqpoint{1.373022in}{1.701889in}}%
\pgfpathlineto{\pgfqpoint{1.366679in}{1.710890in}}%
\pgfpathlineto{\pgfqpoint{1.366435in}{1.711391in}}%
\pgfpathlineto{\pgfqpoint{1.362251in}{1.719892in}}%
\pgfpathlineto{\pgfqpoint{1.359916in}{1.728893in}}%
\pgfpathlineto{\pgfqpoint{1.359567in}{1.737895in}}%
\pgfpathlineto{\pgfqpoint{1.360892in}{1.746897in}}%
\pgfpathlineto{\pgfqpoint{1.363401in}{1.755898in}}%
\pgfpathlineto{\pgfqpoint{1.366435in}{1.764769in}}%
\pgfpathlineto{\pgfqpoint{1.366479in}{1.764900in}}%
\pgfpathlineto{\pgfqpoint{1.369411in}{1.773901in}}%
\pgfpathlineto{\pgfqpoint{1.371557in}{1.782903in}}%
\pgfpathlineto{\pgfqpoint{1.372309in}{1.791904in}}%
\pgfpathlineto{\pgfqpoint{1.371184in}{1.800906in}}%
\pgfpathlineto{\pgfqpoint{1.367873in}{1.809907in}}%
\pgfpathlineto{\pgfqpoint{1.366435in}{1.812216in}}%
\pgfpathlineto{\pgfqpoint{1.362219in}{1.818909in}}%
\pgfpathlineto{\pgfqpoint{1.356703in}{1.825207in}}%
\pgfpathlineto{\pgfqpoint{1.354125in}{1.827910in}}%
\pgfpathlineto{\pgfqpoint{1.346972in}{1.833956in}}%
\pgfpathlineto{\pgfqpoint{1.342818in}{1.836912in}}%
\pgfpathlineto{\pgfqpoint{1.337240in}{1.840382in}}%
\pgfpathlineto{\pgfqpoint{1.327509in}{1.844908in}}%
\pgfpathlineto{\pgfqpoint{1.323946in}{1.845914in}}%
\pgfpathlineto{\pgfqpoint{1.317778in}{1.847539in}}%
\pgfpathlineto{\pgfqpoint{1.308046in}{1.848362in}}%
\pgfpathlineto{\pgfqpoint{1.298315in}{1.847523in}}%
\pgfpathlineto{\pgfqpoint{1.291400in}{1.845914in}}%
\pgfpathlineto{\pgfqpoint{1.288583in}{1.845211in}}%
\pgfpathlineto{\pgfqpoint{1.278852in}{1.841651in}}%
\pgfpathlineto{\pgfqpoint{1.269120in}{1.837452in}}%
\pgfpathlineto{\pgfqpoint{1.267895in}{1.836912in}}%
\pgfpathlineto{\pgfqpoint{1.259389in}{1.832621in}}%
\pgfpathlineto{\pgfqpoint{1.249657in}{1.828237in}}%
\pgfpathlineto{\pgfqpoint{1.248694in}{1.827910in}}%
\pgfpathlineto{\pgfqpoint{1.239926in}{1.824227in}}%
\pgfpathlineto{\pgfqpoint{1.230194in}{1.822050in}}%
\pgfpathlineto{\pgfqpoint{1.220463in}{1.822142in}}%
\pgfpathlineto{\pgfqpoint{1.210731in}{1.824626in}}%
\pgfpathlineto{\pgfqpoint{1.204005in}{1.827910in}}%
\pgfpathlineto{\pgfqpoint{1.201000in}{1.829093in}}%
\pgfpathlineto{\pgfqpoint{1.191268in}{1.834463in}}%
\pgfpathlineto{\pgfqpoint{1.187579in}{1.836912in}}%
\pgfpathlineto{\pgfqpoint{1.181537in}{1.840417in}}%
\pgfpathlineto{\pgfqpoint{1.172774in}{1.845914in}}%
\pgfpathlineto{\pgfqpoint{1.171805in}{1.846481in}}%
\pgfpathlineto{\pgfqpoint{1.162074in}{1.852095in}}%
\pgfpathlineto{\pgfqpoint{1.156605in}{1.854915in}}%
\pgfpathlineto{\pgfqpoint{1.152342in}{1.857092in}}%
\pgfpathlineto{\pgfqpoint{1.142611in}{1.861014in}}%
\pgfpathlineto{\pgfqpoint{1.132879in}{1.863523in}}%
\pgfpathlineto{\pgfqpoint{1.128439in}{1.863917in}}%
\pgfpathlineto{\pgfqpoint{1.123148in}{1.864409in}}%
\pgfpathlineto{\pgfqpoint{1.117857in}{1.863917in}}%
\pgfpathlineto{\pgfqpoint{1.113416in}{1.863523in}}%
\pgfpathlineto{\pgfqpoint{1.103685in}{1.861014in}}%
\pgfpathlineto{\pgfqpoint{1.093953in}{1.857092in}}%
\pgfpathlineto{\pgfqpoint{1.089691in}{1.854915in}}%
\pgfpathlineto{\pgfqpoint{1.084222in}{1.852095in}}%
\pgfpathlineto{\pgfqpoint{1.074491in}{1.846481in}}%
\pgfpathlineto{\pgfqpoint{1.073522in}{1.845914in}}%
\pgfpathlineto{\pgfqpoint{1.064759in}{1.840417in}}%
\pgfpathlineto{\pgfqpoint{1.058717in}{1.836912in}}%
\pgfpathlineto{\pgfqpoint{1.055028in}{1.834463in}}%
\pgfpathlineto{\pgfqpoint{1.045296in}{1.829093in}}%
\pgfpathlineto{\pgfqpoint{1.042291in}{1.827910in}}%
\pgfpathlineto{\pgfqpoint{1.035565in}{1.824626in}}%
\pgfpathlineto{\pgfqpoint{1.025833in}{1.822142in}}%
\pgfpathlineto{\pgfqpoint{1.016102in}{1.822050in}}%
\pgfpathlineto{\pgfqpoint{1.006370in}{1.824227in}}%
\pgfpathlineto{\pgfqpoint{0.997602in}{1.827910in}}%
\pgfpathlineto{\pgfqpoint{0.996639in}{1.828237in}}%
\pgfpathlineto{\pgfqpoint{0.986907in}{1.832621in}}%
\pgfpathlineto{\pgfqpoint{0.978401in}{1.836912in}}%
\pgfpathlineto{\pgfqpoint{0.977176in}{1.837452in}}%
\pgfpathlineto{\pgfqpoint{0.967444in}{1.841651in}}%
\pgfpathlineto{\pgfqpoint{0.957713in}{1.845211in}}%
\pgfpathlineto{\pgfqpoint{0.954896in}{1.845914in}}%
\pgfpathlineto{\pgfqpoint{0.947981in}{1.847523in}}%
\pgfpathlineto{\pgfqpoint{0.938250in}{1.848362in}}%
\pgfpathlineto{\pgfqpoint{0.928518in}{1.847539in}}%
\pgfpathlineto{\pgfqpoint{0.922349in}{1.845914in}}%
\pgfpathlineto{\pgfqpoint{0.918787in}{1.844908in}}%
\pgfpathlineto{\pgfqpoint{0.909055in}{1.840382in}}%
\pgfpathlineto{\pgfqpoint{0.903477in}{1.836912in}}%
\pgfpathlineto{\pgfqpoint{0.899324in}{1.833956in}}%
\pgfpathlineto{\pgfqpoint{0.892170in}{1.827910in}}%
\pgfpathlineto{\pgfqpoint{0.889592in}{1.825207in}}%
\pgfpathlineto{\pgfqpoint{0.884077in}{1.818909in}}%
\pgfpathlineto{\pgfqpoint{0.879861in}{1.812216in}}%
\pgfpathlineto{\pgfqpoint{0.878423in}{1.809907in}}%
\pgfpathlineto{\pgfqpoint{0.875111in}{1.800906in}}%
\pgfpathlineto{\pgfqpoint{0.873987in}{1.791904in}}%
\pgfpathlineto{\pgfqpoint{0.874739in}{1.782903in}}%
\pgfpathlineto{\pgfqpoint{0.876885in}{1.773901in}}%
\pgfpathlineto{\pgfqpoint{0.879817in}{1.764900in}}%
\pgfpathlineto{\pgfqpoint{0.879861in}{1.764769in}}%
\pgfpathlineto{\pgfqpoint{0.882895in}{1.755898in}}%
\pgfpathlineto{\pgfqpoint{0.885404in}{1.746897in}}%
\pgfpathlineto{\pgfqpoint{0.886729in}{1.737895in}}%
\pgfpathlineto{\pgfqpoint{0.886380in}{1.728893in}}%
\pgfpathlineto{\pgfqpoint{0.884045in}{1.719892in}}%
\pgfpathlineto{\pgfqpoint{0.879861in}{1.711391in}}%
\pgfpathlineto{\pgfqpoint{0.879617in}{1.710890in}}%
\pgfpathlineto{\pgfqpoint{0.873274in}{1.701889in}}%
\pgfpathlineto{\pgfqpoint{0.870129in}{1.698350in}}%
\pgfpathlineto{\pgfqpoint{0.864979in}{1.692887in}}%
\pgfpathlineto{\pgfqpoint{0.860398in}{1.688650in}}%
\pgfpathlineto{\pgfqpoint{0.854493in}{1.683886in}}%
\pgfpathlineto{\pgfqpoint{0.850667in}{1.680977in}}%
\pgfpathlineto{\pgfqpoint{0.840935in}{1.675110in}}%
\pgfpathlineto{\pgfqpoint{0.840394in}{1.674884in}}%
\pgfpathlineto{\pgfqpoint{0.831204in}{1.671014in}}%
\pgfpathlineto{\pgfqpoint{0.821472in}{1.668854in}}%
\pgfpathlineto{\pgfqpoint{0.811741in}{1.668532in}}%
\pgfpathlineto{\pgfqpoint{0.802009in}{1.669757in}}%
\pgfpathlineto{\pgfqpoint{0.792278in}{1.672078in}}%
\pgfpathlineto{\pgfqpoint{0.782688in}{1.674884in}}%
\pgfpathlineto{\pgfqpoint{0.782546in}{1.674925in}}%
\pgfpathlineto{\pgfqpoint{0.772815in}{1.677637in}}%
\pgfpathlineto{\pgfqpoint{0.763083in}{1.679622in}}%
\pgfpathlineto{\pgfqpoint{0.753352in}{1.680318in}}%
\pgfpathlineto{\pgfqpoint{0.743620in}{1.679277in}}%
\pgfpathlineto{\pgfqpoint{0.733889in}{1.676214in}}%
\pgfpathlineto{\pgfqpoint{0.731393in}{1.674884in}}%
\pgfpathlineto{\pgfqpoint{0.724157in}{1.670984in}}%
\pgfpathlineto{\pgfqpoint{0.717349in}{1.665883in}}%
\pgfpathlineto{\pgfqpoint{0.714426in}{1.663498in}}%
\pgfpathlineto{\pgfqpoint{0.707890in}{1.656881in}}%
\pgfpathlineto{\pgfqpoint{0.704694in}{1.653039in}}%
\pgfpathlineto{\pgfqpoint{0.700943in}{1.647880in}}%
\pgfpathlineto{\pgfqpoint{0.696050in}{1.638878in}}%
\pgfpathlineto{\pgfqpoint{0.694963in}{1.635583in}}%
\pgfpathlineto{\pgfqpoint{0.693205in}{1.629876in}}%
\pgfpathlineto{\pgfqpoint{0.692316in}{1.620875in}}%
\pgfpathlineto{\pgfqpoint{0.693223in}{1.611873in}}%
\pgfpathlineto{\pgfqpoint{0.694963in}{1.605478in}}%
\pgfpathlineto{\pgfqpoint{0.695722in}{1.602872in}}%
\pgfpathlineto{\pgfqpoint{0.699571in}{1.593870in}}%
\pgfpathlineto{\pgfqpoint{0.704111in}{1.584869in}}%
\pgfpathlineto{\pgfqpoint{0.704694in}{1.583736in}}%
\pgfpathlineto{\pgfqpoint{0.709333in}{1.575867in}}%
\pgfpathlineto{\pgfqpoint{0.714073in}{1.566866in}}%
\pgfpathlineto{\pgfqpoint{0.714426in}{1.565974in}}%
\pgfpathlineto{\pgfqpoint{0.718407in}{1.557864in}}%
\pgfpathlineto{\pgfqpoint{0.720761in}{1.548863in}}%
\pgfpathlineto{\pgfqpoint{0.720662in}{1.539861in}}%
\pgfpathlineto{\pgfqpoint{0.717977in}{1.530859in}}%
\pgfpathlineto{\pgfqpoint{0.714426in}{1.524638in}}%
\pgfpathlineto{\pgfqpoint{0.713147in}{1.521858in}}%
\pgfpathlineto{\pgfqpoint{0.707342in}{1.512856in}}%
\pgfpathlineto{\pgfqpoint{0.704694in}{1.509444in}}%
\pgfpathlineto{\pgfqpoint{0.700905in}{1.503855in}}%
\pgfpathlineto{\pgfqpoint{0.694963in}{1.495749in}}%
\pgfpathlineto{\pgfqpoint{0.694349in}{1.494853in}}%
\pgfpathlineto{\pgfqpoint{0.688280in}{1.485852in}}%
\pgfpathlineto{\pgfqpoint{0.685231in}{1.480793in}}%
\pgfpathlineto{\pgfqpoint{0.682878in}{1.476850in}}%
\pgfpathlineto{\pgfqpoint{0.678638in}{1.467849in}}%
\pgfpathlineto{\pgfqpoint{0.675926in}{1.458847in}}%
\pgfpathlineto{\pgfqpoint{0.675500in}{1.454740in}}%
\pgfpathlineto{\pgfqpoint{0.674968in}{1.449846in}}%
\pgfpathlineto{\pgfqpoint{0.675500in}{1.444952in}}%
\pgfpathlineto{\pgfqpoint{0.675926in}{1.440844in}}%
\pgfpathlineto{\pgfqpoint{0.678638in}{1.431842in}}%
\pgfpathlineto{\pgfqpoint{0.682878in}{1.422841in}}%
\pgfpathlineto{\pgfqpoint{0.685231in}{1.418898in}}%
\pgfpathlineto{\pgfqpoint{0.688280in}{1.413839in}}%
\pgfpathlineto{\pgfqpoint{0.694349in}{1.404838in}}%
\pgfpathlineto{\pgfqpoint{0.694963in}{1.403942in}}%
\pgfpathlineto{\pgfqpoint{0.700905in}{1.395836in}}%
\pgfpathlineto{\pgfqpoint{0.704694in}{1.390248in}}%
\pgfpathlineto{\pgfqpoint{0.707342in}{1.386835in}}%
\pgfpathlineto{\pgfqpoint{0.713147in}{1.377833in}}%
\pgfpathlineto{\pgfqpoint{0.714426in}{1.375054in}}%
\pgfpathlineto{\pgfqpoint{0.717977in}{1.368832in}}%
\pgfpathlineto{\pgfqpoint{0.720662in}{1.359830in}}%
\pgfpathlineto{\pgfqpoint{0.720761in}{1.350829in}}%
\pgfpathlineto{\pgfqpoint{0.718407in}{1.341827in}}%
\pgfpathlineto{\pgfqpoint{0.714426in}{1.333717in}}%
\pgfpathlineto{\pgfqpoint{0.714073in}{1.332825in}}%
\pgfpathlineto{\pgfqpoint{0.709333in}{1.323824in}}%
\pgfpathlineto{\pgfqpoint{0.704694in}{1.315955in}}%
\pgfpathlineto{\pgfqpoint{0.704111in}{1.314822in}}%
\pgfpathlineto{\pgfqpoint{0.699571in}{1.305821in}}%
\pgfpathlineto{\pgfqpoint{0.695722in}{1.296819in}}%
\pgfpathlineto{\pgfqpoint{0.694963in}{1.294214in}}%
\pgfpathlineto{\pgfqpoint{0.693223in}{1.287818in}}%
\pgfpathlineto{\pgfqpoint{0.692316in}{1.278816in}}%
\pgfpathlineto{\pgfqpoint{0.693205in}{1.269815in}}%
\pgfpathlineto{\pgfqpoint{0.694963in}{1.264108in}}%
\pgfpathlineto{\pgfqpoint{0.696050in}{1.260813in}}%
\pgfpathlineto{\pgfqpoint{0.700943in}{1.251812in}}%
\pgfpathlineto{\pgfqpoint{0.704694in}{1.246652in}}%
\pgfpathlineto{\pgfqpoint{0.707890in}{1.242810in}}%
\pgfpathlineto{\pgfqpoint{0.714426in}{1.236193in}}%
\pgfpathlineto{\pgfqpoint{0.717349in}{1.233808in}}%
\pgfpathlineto{\pgfqpoint{0.724157in}{1.228707in}}%
\pgfpathlineto{\pgfqpoint{0.731393in}{1.224807in}}%
\pgfpathlineto{\pgfqpoint{0.733889in}{1.223477in}}%
\pgfpathlineto{\pgfqpoint{0.743620in}{1.220414in}}%
\pgfpathlineto{\pgfqpoint{0.753352in}{1.219373in}}%
\pgfpathlineto{\pgfqpoint{0.763083in}{1.220069in}}%
\pgfpathlineto{\pgfqpoint{0.772815in}{1.222054in}}%
\pgfpathlineto{\pgfqpoint{0.782546in}{1.224766in}}%
\pgfpathlineto{\pgfqpoint{0.782688in}{1.224807in}}%
\pgfpathlineto{\pgfqpoint{0.792278in}{1.227613in}}%
\pgfpathlineto{\pgfqpoint{0.802009in}{1.229934in}}%
\pgfpathlineto{\pgfqpoint{0.811741in}{1.231160in}}%
\pgfpathlineto{\pgfqpoint{0.821472in}{1.230837in}}%
\pgfpathlineto{\pgfqpoint{0.831204in}{1.228677in}}%
\pgfpathlineto{\pgfqpoint{0.840394in}{1.224807in}}%
\pgfpathlineto{\pgfqpoint{0.840935in}{1.224582in}}%
\pgfpathlineto{\pgfqpoint{0.850667in}{1.218714in}}%
\pgfpathlineto{\pgfqpoint{0.854493in}{1.215805in}}%
\pgfpathlineto{\pgfqpoint{0.860398in}{1.211041in}}%
\pgfpathlineto{\pgfqpoint{0.864979in}{1.206804in}}%
\pgfpathlineto{\pgfqpoint{0.870129in}{1.201342in}}%
\pgfpathlineto{\pgfqpoint{0.873274in}{1.197802in}}%
\pgfpathlineto{\pgfqpoint{0.879617in}{1.188801in}}%
\pgfpathlineto{\pgfqpoint{0.879861in}{1.188300in}}%
\pgfpathlineto{\pgfqpoint{0.884045in}{1.179799in}}%
\pgfpathlineto{\pgfqpoint{0.886380in}{1.170798in}}%
\pgfpathlineto{\pgfqpoint{0.886729in}{1.161796in}}%
\pgfpathlineto{\pgfqpoint{0.885404in}{1.152795in}}%
\pgfpathlineto{\pgfqpoint{0.882895in}{1.143793in}}%
\pgfpathlineto{\pgfqpoint{0.879861in}{1.134922in}}%
\pgfpathlineto{\pgfqpoint{0.879817in}{1.134792in}}%
\pgfpathlineto{\pgfqpoint{0.876885in}{1.125790in}}%
\pgfpathlineto{\pgfqpoint{0.874739in}{1.116788in}}%
\pgfpathlineto{\pgfqpoint{0.873987in}{1.107787in}}%
\pgfpathlineto{\pgfqpoint{0.875111in}{1.098785in}}%
\pgfpathlineto{\pgfqpoint{0.878423in}{1.089784in}}%
\pgfpathlineto{\pgfqpoint{0.879861in}{1.087475in}}%
\pgfpathlineto{\pgfqpoint{0.884077in}{1.080782in}}%
\pgfpathlineto{\pgfqpoint{0.889592in}{1.074484in}}%
\pgfpathlineto{\pgfqpoint{0.892170in}{1.071781in}}%
\pgfpathlineto{\pgfqpoint{0.899324in}{1.065735in}}%
\pgfpathlineto{\pgfqpoint{0.903477in}{1.062779in}}%
\pgfpathlineto{\pgfqpoint{0.909055in}{1.059309in}}%
\pgfpathlineto{\pgfqpoint{0.918787in}{1.054783in}}%
\pgfpathlineto{\pgfqpoint{0.922349in}{1.053778in}}%
\pgfpathlineto{\pgfqpoint{0.928518in}{1.052152in}}%
\pgfpathlineto{\pgfqpoint{0.938250in}{1.051329in}}%
\pgfpathlineto{\pgfqpoint{0.947981in}{1.052168in}}%
\pgfpathlineto{\pgfqpoint{0.954896in}{1.053778in}}%
\pgfpathlineto{\pgfqpoint{0.957713in}{1.054480in}}%
\pgfpathlineto{\pgfqpoint{0.967444in}{1.058040in}}%
\pgfpathlineto{\pgfqpoint{0.977176in}{1.062239in}}%
\pgfpathlineto{\pgfqpoint{0.978401in}{1.062779in}}%
\pgfpathlineto{\pgfqpoint{0.986907in}{1.067070in}}%
\pgfpathlineto{\pgfqpoint{0.996639in}{1.071454in}}%
\pgfpathlineto{\pgfqpoint{0.997602in}{1.071781in}}%
\pgfpathlineto{\pgfqpoint{1.006370in}{1.075464in}}%
\pgfpathlineto{\pgfqpoint{1.016102in}{1.077641in}}%
\pgfpathlineto{\pgfqpoint{1.025833in}{1.077549in}}%
\pgfpathlineto{\pgfqpoint{1.035565in}{1.075066in}}%
\pgfpathlineto{\pgfqpoint{1.042291in}{1.071781in}}%
\pgfpathlineto{\pgfqpoint{1.045296in}{1.070598in}}%
\pgfpathlineto{\pgfqpoint{1.055028in}{1.065228in}}%
\pgfpathlineto{\pgfqpoint{1.058717in}{1.062779in}}%
\pgfpathlineto{\pgfqpoint{1.064759in}{1.059274in}}%
\pgfpathlineto{\pgfqpoint{1.073522in}{1.053778in}}%
\pgfpathlineto{\pgfqpoint{1.074491in}{1.053210in}}%
\pgfpathlineto{\pgfqpoint{1.084222in}{1.047596in}}%
\pgfpathlineto{\pgfqpoint{1.089691in}{1.044776in}}%
\pgfpathlineto{\pgfqpoint{1.093953in}{1.042599in}}%
\pgfpathlineto{\pgfqpoint{1.103685in}{1.038677in}}%
\pgfpathlineto{\pgfqpoint{1.113416in}{1.036168in}}%
\pgfpathlineto{\pgfqpoint{1.117857in}{1.035775in}}%
\pgfpathclose%
\pgfpathmoveto{\pgfqpoint{1.104742in}{1.080782in}}%
\pgfpathlineto{\pgfqpoint{1.103685in}{1.081364in}}%
\pgfpathlineto{\pgfqpoint{1.093953in}{1.089731in}}%
\pgfpathlineto{\pgfqpoint{1.093905in}{1.089784in}}%
\pgfpathlineto{\pgfqpoint{1.089026in}{1.098785in}}%
\pgfpathlineto{\pgfqpoint{1.087368in}{1.107787in}}%
\pgfpathlineto{\pgfqpoint{1.088476in}{1.116788in}}%
\pgfpathlineto{\pgfqpoint{1.091638in}{1.125790in}}%
\pgfpathlineto{\pgfqpoint{1.093953in}{1.130614in}}%
\pgfpathlineto{\pgfqpoint{1.096495in}{1.134792in}}%
\pgfpathlineto{\pgfqpoint{1.102178in}{1.143793in}}%
\pgfpathlineto{\pgfqpoint{1.103685in}{1.146720in}}%
\pgfpathlineto{\pgfqpoint{1.108571in}{1.152795in}}%
\pgfpathlineto{\pgfqpoint{1.112393in}{1.161796in}}%
\pgfpathlineto{\pgfqpoint{1.111387in}{1.170798in}}%
\pgfpathlineto{\pgfqpoint{1.104648in}{1.179799in}}%
\pgfpathlineto{\pgfqpoint{1.103685in}{1.180477in}}%
\pgfpathlineto{\pgfqpoint{1.096122in}{1.188801in}}%
\pgfpathlineto{\pgfqpoint{1.093953in}{1.190449in}}%
\pgfpathlineto{\pgfqpoint{1.086318in}{1.197802in}}%
\pgfpathlineto{\pgfqpoint{1.084222in}{1.199404in}}%
\pgfpathlineto{\pgfqpoint{1.075565in}{1.206804in}}%
\pgfpathlineto{\pgfqpoint{1.074491in}{1.207605in}}%
\pgfpathlineto{\pgfqpoint{1.064759in}{1.214970in}}%
\pgfpathlineto{\pgfqpoint{1.063566in}{1.215805in}}%
\pgfpathlineto{\pgfqpoint{1.055028in}{1.221437in}}%
\pgfpathlineto{\pgfqpoint{1.048882in}{1.224807in}}%
\pgfpathlineto{\pgfqpoint{1.045296in}{1.226796in}}%
\pgfpathlineto{\pgfqpoint{1.035565in}{1.230645in}}%
\pgfpathlineto{\pgfqpoint{1.025833in}{1.232657in}}%
\pgfpathlineto{\pgfqpoint{1.016102in}{1.232731in}}%
\pgfpathlineto{\pgfqpoint{1.006370in}{1.230968in}}%
\pgfpathlineto{\pgfqpoint{0.996639in}{1.227656in}}%
\pgfpathlineto{\pgfqpoint{0.990347in}{1.224807in}}%
\pgfpathlineto{\pgfqpoint{0.986907in}{1.223267in}}%
\pgfpathlineto{\pgfqpoint{0.977176in}{1.218390in}}%
\pgfpathlineto{\pgfqpoint{0.971908in}{1.215805in}}%
\pgfpathlineto{\pgfqpoint{0.967444in}{1.213481in}}%
\pgfpathlineto{\pgfqpoint{0.957713in}{1.209186in}}%
\pgfpathlineto{\pgfqpoint{0.949794in}{1.206804in}}%
\pgfpathlineto{\pgfqpoint{0.947981in}{1.206179in}}%
\pgfpathlineto{\pgfqpoint{0.938250in}{1.204936in}}%
\pgfpathlineto{\pgfqpoint{0.928518in}{1.206154in}}%
\pgfpathlineto{\pgfqpoint{0.926854in}{1.206804in}}%
\pgfpathlineto{\pgfqpoint{0.918787in}{1.209552in}}%
\pgfpathlineto{\pgfqpoint{0.909055in}{1.215013in}}%
\pgfpathlineto{\pgfqpoint{0.908000in}{1.215805in}}%
\pgfpathlineto{\pgfqpoint{0.899324in}{1.221941in}}%
\pgfpathlineto{\pgfqpoint{0.895911in}{1.224807in}}%
\pgfpathlineto{\pgfqpoint{0.889592in}{1.230174in}}%
\pgfpathlineto{\pgfqpoint{0.885663in}{1.233808in}}%
\pgfpathlineto{\pgfqpoint{0.879861in}{1.239653in}}%
\pgfpathlineto{\pgfqpoint{0.876763in}{1.242810in}}%
\pgfpathlineto{\pgfqpoint{0.870129in}{1.250836in}}%
\pgfpathlineto{\pgfqpoint{0.869273in}{1.251812in}}%
\pgfpathlineto{\pgfqpoint{0.863369in}{1.260813in}}%
\pgfpathlineto{\pgfqpoint{0.860398in}{1.268275in}}%
\pgfpathlineto{\pgfqpoint{0.859695in}{1.269815in}}%
\pgfpathlineto{\pgfqpoint{0.858378in}{1.278816in}}%
\pgfpathlineto{\pgfqpoint{0.859722in}{1.287818in}}%
\pgfpathlineto{\pgfqpoint{0.860398in}{1.289495in}}%
\pgfpathlineto{\pgfqpoint{0.862973in}{1.296819in}}%
\pgfpathlineto{\pgfqpoint{0.867617in}{1.305821in}}%
\pgfpathlineto{\pgfqpoint{0.870129in}{1.309949in}}%
\pgfpathlineto{\pgfqpoint{0.872924in}{1.314822in}}%
\pgfpathlineto{\pgfqpoint{0.878196in}{1.323824in}}%
\pgfpathlineto{\pgfqpoint{0.879861in}{1.327005in}}%
\pgfpathlineto{\pgfqpoint{0.882942in}{1.332825in}}%
\pgfpathlineto{\pgfqpoint{0.886521in}{1.341827in}}%
\pgfpathlineto{\pgfqpoint{0.888428in}{1.350829in}}%
\pgfpathlineto{\pgfqpoint{0.888348in}{1.359830in}}%
\pgfpathlineto{\pgfqpoint{0.886173in}{1.368832in}}%
\pgfpathlineto{\pgfqpoint{0.882011in}{1.377833in}}%
\pgfpathlineto{\pgfqpoint{0.879861in}{1.381150in}}%
\pgfpathlineto{\pgfqpoint{0.876218in}{1.386835in}}%
\pgfpathlineto{\pgfqpoint{0.870129in}{1.394733in}}%
\pgfpathlineto{\pgfqpoint{0.869227in}{1.395836in}}%
\pgfpathlineto{\pgfqpoint{0.861264in}{1.404838in}}%
\pgfpathlineto{\pgfqpoint{0.860398in}{1.405832in}}%
\pgfpathlineto{\pgfqpoint{0.852398in}{1.413839in}}%
\pgfpathlineto{\pgfqpoint{0.850667in}{1.415778in}}%
\pgfpathlineto{\pgfqpoint{0.842716in}{1.422841in}}%
\pgfpathlineto{\pgfqpoint{0.840935in}{1.424847in}}%
\pgfpathlineto{\pgfqpoint{0.831937in}{1.431842in}}%
\pgfpathlineto{\pgfqpoint{0.831204in}{1.432733in}}%
\pgfpathlineto{\pgfqpoint{0.821472in}{1.438967in}}%
\pgfpathlineto{\pgfqpoint{0.811741in}{1.439898in}}%
\pgfpathlineto{\pgfqpoint{0.802009in}{1.436362in}}%
\pgfpathlineto{\pgfqpoint{0.795442in}{1.431842in}}%
\pgfpathlineto{\pgfqpoint{0.792278in}{1.430449in}}%
\pgfpathlineto{\pgfqpoint{0.782546in}{1.425191in}}%
\pgfpathlineto{\pgfqpoint{0.778030in}{1.422841in}}%
\pgfpathlineto{\pgfqpoint{0.772815in}{1.420699in}}%
\pgfpathlineto{\pgfqpoint{0.763083in}{1.417774in}}%
\pgfpathlineto{\pgfqpoint{0.753352in}{1.416750in}}%
\pgfpathlineto{\pgfqpoint{0.743620in}{1.418283in}}%
\pgfpathlineto{\pgfqpoint{0.733889in}{1.422796in}}%
\pgfpathlineto{\pgfqpoint{0.733832in}{1.422841in}}%
\pgfpathlineto{\pgfqpoint{0.724786in}{1.431842in}}%
\pgfpathlineto{\pgfqpoint{0.724157in}{1.432820in}}%
\pgfpathlineto{\pgfqpoint{0.720446in}{1.440844in}}%
\pgfpathlineto{\pgfqpoint{0.719014in}{1.449846in}}%
\pgfpathlineto{\pgfqpoint{0.720446in}{1.458847in}}%
\pgfpathlineto{\pgfqpoint{0.724157in}{1.466871in}}%
\pgfpathlineto{\pgfqpoint{0.724786in}{1.467849in}}%
\pgfpathlineto{\pgfqpoint{0.733832in}{1.476850in}}%
\pgfpathlineto{\pgfqpoint{0.733889in}{1.476895in}}%
\pgfpathlineto{\pgfqpoint{0.743620in}{1.481408in}}%
\pgfpathlineto{\pgfqpoint{0.753352in}{1.482942in}}%
\pgfpathlineto{\pgfqpoint{0.763083in}{1.481917in}}%
\pgfpathlineto{\pgfqpoint{0.772815in}{1.478992in}}%
\pgfpathlineto{\pgfqpoint{0.778030in}{1.476850in}}%
\pgfpathlineto{\pgfqpoint{0.782546in}{1.474500in}}%
\pgfpathlineto{\pgfqpoint{0.792278in}{1.469242in}}%
\pgfpathlineto{\pgfqpoint{0.795442in}{1.467849in}}%
\pgfpathlineto{\pgfqpoint{0.802009in}{1.463329in}}%
\pgfpathlineto{\pgfqpoint{0.811741in}{1.459794in}}%
\pgfpathlineto{\pgfqpoint{0.821472in}{1.460724in}}%
\pgfpathlineto{\pgfqpoint{0.831204in}{1.466958in}}%
\pgfpathlineto{\pgfqpoint{0.831937in}{1.467849in}}%
\pgfpathlineto{\pgfqpoint{0.840935in}{1.474844in}}%
\pgfpathlineto{\pgfqpoint{0.842716in}{1.476850in}}%
\pgfpathlineto{\pgfqpoint{0.850667in}{1.483913in}}%
\pgfpathlineto{\pgfqpoint{0.852398in}{1.485852in}}%
\pgfpathlineto{\pgfqpoint{0.860398in}{1.493859in}}%
\pgfpathlineto{\pgfqpoint{0.861264in}{1.494853in}}%
\pgfpathlineto{\pgfqpoint{0.869227in}{1.503855in}}%
\pgfpathlineto{\pgfqpoint{0.870129in}{1.504958in}}%
\pgfpathlineto{\pgfqpoint{0.876218in}{1.512856in}}%
\pgfpathlineto{\pgfqpoint{0.879861in}{1.518541in}}%
\pgfpathlineto{\pgfqpoint{0.882011in}{1.521858in}}%
\pgfpathlineto{\pgfqpoint{0.886173in}{1.530859in}}%
\pgfpathlineto{\pgfqpoint{0.888348in}{1.539861in}}%
\pgfpathlineto{\pgfqpoint{0.888428in}{1.548863in}}%
\pgfpathlineto{\pgfqpoint{0.886521in}{1.557864in}}%
\pgfpathlineto{\pgfqpoint{0.882942in}{1.566866in}}%
\pgfpathlineto{\pgfqpoint{0.879861in}{1.572686in}}%
\pgfpathlineto{\pgfqpoint{0.878196in}{1.575867in}}%
\pgfpathlineto{\pgfqpoint{0.872924in}{1.584869in}}%
\pgfpathlineto{\pgfqpoint{0.870129in}{1.589742in}}%
\pgfpathlineto{\pgfqpoint{0.867617in}{1.593870in}}%
\pgfpathlineto{\pgfqpoint{0.862973in}{1.602872in}}%
\pgfpathlineto{\pgfqpoint{0.860398in}{1.610196in}}%
\pgfpathlineto{\pgfqpoint{0.859722in}{1.611873in}}%
\pgfpathlineto{\pgfqpoint{0.858378in}{1.620875in}}%
\pgfpathlineto{\pgfqpoint{0.859695in}{1.629876in}}%
\pgfpathlineto{\pgfqpoint{0.860398in}{1.631416in}}%
\pgfpathlineto{\pgfqpoint{0.863369in}{1.638878in}}%
\pgfpathlineto{\pgfqpoint{0.869273in}{1.647880in}}%
\pgfpathlineto{\pgfqpoint{0.870129in}{1.648855in}}%
\pgfpathlineto{\pgfqpoint{0.876763in}{1.656881in}}%
\pgfpathlineto{\pgfqpoint{0.879861in}{1.660038in}}%
\pgfpathlineto{\pgfqpoint{0.885663in}{1.665883in}}%
\pgfpathlineto{\pgfqpoint{0.889592in}{1.669517in}}%
\pgfpathlineto{\pgfqpoint{0.895911in}{1.674884in}}%
\pgfpathlineto{\pgfqpoint{0.899324in}{1.677750in}}%
\pgfpathlineto{\pgfqpoint{0.908000in}{1.683886in}}%
\pgfpathlineto{\pgfqpoint{0.909055in}{1.684678in}}%
\pgfpathlineto{\pgfqpoint{0.918787in}{1.690139in}}%
\pgfpathlineto{\pgfqpoint{0.926854in}{1.692887in}}%
\pgfpathlineto{\pgfqpoint{0.928518in}{1.693537in}}%
\pgfpathlineto{\pgfqpoint{0.938250in}{1.694756in}}%
\pgfpathlineto{\pgfqpoint{0.947981in}{1.693512in}}%
\pgfpathlineto{\pgfqpoint{0.949794in}{1.692887in}}%
\pgfpathlineto{\pgfqpoint{0.957713in}{1.690506in}}%
\pgfpathlineto{\pgfqpoint{0.967444in}{1.686210in}}%
\pgfpathlineto{\pgfqpoint{0.971908in}{1.683886in}}%
\pgfpathlineto{\pgfqpoint{0.977176in}{1.681301in}}%
\pgfpathlineto{\pgfqpoint{0.986907in}{1.676424in}}%
\pgfpathlineto{\pgfqpoint{0.990347in}{1.674884in}}%
\pgfpathlineto{\pgfqpoint{0.996639in}{1.672035in}}%
\pgfpathlineto{\pgfqpoint{1.006370in}{1.668723in}}%
\pgfpathlineto{\pgfqpoint{1.016102in}{1.666960in}}%
\pgfpathlineto{\pgfqpoint{1.025833in}{1.667034in}}%
\pgfpathlineto{\pgfqpoint{1.035565in}{1.669046in}}%
\pgfpathlineto{\pgfqpoint{1.045296in}{1.672895in}}%
\pgfpathlineto{\pgfqpoint{1.048882in}{1.674884in}}%
\pgfpathlineto{\pgfqpoint{1.055028in}{1.678254in}}%
\pgfpathlineto{\pgfqpoint{1.063566in}{1.683886in}}%
\pgfpathlineto{\pgfqpoint{1.064759in}{1.684721in}}%
\pgfpathlineto{\pgfqpoint{1.074491in}{1.692086in}}%
\pgfpathlineto{\pgfqpoint{1.075565in}{1.692887in}}%
\pgfpathlineto{\pgfqpoint{1.084222in}{1.700287in}}%
\pgfpathlineto{\pgfqpoint{1.086318in}{1.701889in}}%
\pgfpathlineto{\pgfqpoint{1.093953in}{1.709243in}}%
\pgfpathlineto{\pgfqpoint{1.096122in}{1.710890in}}%
\pgfpathlineto{\pgfqpoint{1.103685in}{1.719214in}}%
\pgfpathlineto{\pgfqpoint{1.104648in}{1.719892in}}%
\pgfpathlineto{\pgfqpoint{1.111387in}{1.728893in}}%
\pgfpathlineto{\pgfqpoint{1.112393in}{1.737895in}}%
\pgfpathlineto{\pgfqpoint{1.108571in}{1.746897in}}%
\pgfpathlineto{\pgfqpoint{1.103685in}{1.752971in}}%
\pgfpathlineto{\pgfqpoint{1.102178in}{1.755898in}}%
\pgfpathlineto{\pgfqpoint{1.096495in}{1.764900in}}%
\pgfpathlineto{\pgfqpoint{1.093953in}{1.769077in}}%
\pgfpathlineto{\pgfqpoint{1.091638in}{1.773901in}}%
\pgfpathlineto{\pgfqpoint{1.088476in}{1.782903in}}%
\pgfpathlineto{\pgfqpoint{1.087368in}{1.791904in}}%
\pgfpathlineto{\pgfqpoint{1.089026in}{1.800906in}}%
\pgfpathlineto{\pgfqpoint{1.093905in}{1.809907in}}%
\pgfpathlineto{\pgfqpoint{1.093953in}{1.809960in}}%
\pgfpathlineto{\pgfqpoint{1.103685in}{1.818327in}}%
\pgfpathlineto{\pgfqpoint{1.104742in}{1.818909in}}%
\pgfpathlineto{\pgfqpoint{1.113416in}{1.822342in}}%
\pgfpathlineto{\pgfqpoint{1.123148in}{1.823666in}}%
\pgfpathlineto{\pgfqpoint{1.132879in}{1.822342in}}%
\pgfpathlineto{\pgfqpoint{1.141554in}{1.818909in}}%
\pgfpathlineto{\pgfqpoint{1.142611in}{1.818327in}}%
\pgfpathlineto{\pgfqpoint{1.152342in}{1.809960in}}%
\pgfpathlineto{\pgfqpoint{1.152391in}{1.809907in}}%
\pgfpathlineto{\pgfqpoint{1.157270in}{1.800906in}}%
\pgfpathlineto{\pgfqpoint{1.158928in}{1.791904in}}%
\pgfpathlineto{\pgfqpoint{1.157820in}{1.782903in}}%
\pgfpathlineto{\pgfqpoint{1.154658in}{1.773901in}}%
\pgfpathlineto{\pgfqpoint{1.152342in}{1.769077in}}%
\pgfpathlineto{\pgfqpoint{1.149801in}{1.764900in}}%
\pgfpathlineto{\pgfqpoint{1.144117in}{1.755898in}}%
\pgfpathlineto{\pgfqpoint{1.142611in}{1.752971in}}%
\pgfpathlineto{\pgfqpoint{1.137725in}{1.746897in}}%
\pgfpathlineto{\pgfqpoint{1.133903in}{1.737895in}}%
\pgfpathlineto{\pgfqpoint{1.134909in}{1.728893in}}%
\pgfpathlineto{\pgfqpoint{1.141648in}{1.719892in}}%
\pgfpathlineto{\pgfqpoint{1.142611in}{1.719214in}}%
\pgfpathlineto{\pgfqpoint{1.150174in}{1.710890in}}%
\pgfpathlineto{\pgfqpoint{1.152342in}{1.709243in}}%
\pgfpathlineto{\pgfqpoint{1.159978in}{1.701889in}}%
\pgfpathlineto{\pgfqpoint{1.162074in}{1.700287in}}%
\pgfpathlineto{\pgfqpoint{1.170731in}{1.692887in}}%
\pgfpathlineto{\pgfqpoint{1.171805in}{1.692086in}}%
\pgfpathlineto{\pgfqpoint{1.181537in}{1.684721in}}%
\pgfpathlineto{\pgfqpoint{1.182730in}{1.683886in}}%
\pgfpathlineto{\pgfqpoint{1.191268in}{1.678254in}}%
\pgfpathlineto{\pgfqpoint{1.197414in}{1.674884in}}%
\pgfpathlineto{\pgfqpoint{1.201000in}{1.672895in}}%
\pgfpathlineto{\pgfqpoint{1.210731in}{1.669046in}}%
\pgfpathlineto{\pgfqpoint{1.220463in}{1.667034in}}%
\pgfpathlineto{\pgfqpoint{1.230194in}{1.666960in}}%
\pgfpathlineto{\pgfqpoint{1.239926in}{1.668723in}}%
\pgfpathlineto{\pgfqpoint{1.249657in}{1.672035in}}%
\pgfpathlineto{\pgfqpoint{1.255949in}{1.674884in}}%
\pgfpathlineto{\pgfqpoint{1.259389in}{1.676424in}}%
\pgfpathlineto{\pgfqpoint{1.269120in}{1.681301in}}%
\pgfpathlineto{\pgfqpoint{1.274388in}{1.683886in}}%
\pgfpathlineto{\pgfqpoint{1.278852in}{1.686210in}}%
\pgfpathlineto{\pgfqpoint{1.288583in}{1.690506in}}%
\pgfpathlineto{\pgfqpoint{1.296501in}{1.692887in}}%
\pgfpathlineto{\pgfqpoint{1.298315in}{1.693512in}}%
\pgfpathlineto{\pgfqpoint{1.308046in}{1.694756in}}%
\pgfpathlineto{\pgfqpoint{1.317778in}{1.693537in}}%
\pgfpathlineto{\pgfqpoint{1.319442in}{1.692887in}}%
\pgfpathlineto{\pgfqpoint{1.327509in}{1.690139in}}%
\pgfpathlineto{\pgfqpoint{1.337240in}{1.684678in}}%
\pgfpathlineto{\pgfqpoint{1.338296in}{1.683886in}}%
\pgfpathlineto{\pgfqpoint{1.346972in}{1.677750in}}%
\pgfpathlineto{\pgfqpoint{1.350385in}{1.674884in}}%
\pgfpathlineto{\pgfqpoint{1.356703in}{1.669517in}}%
\pgfpathlineto{\pgfqpoint{1.360632in}{1.665883in}}%
\pgfpathlineto{\pgfqpoint{1.366435in}{1.660038in}}%
\pgfpathlineto{\pgfqpoint{1.369533in}{1.656881in}}%
\pgfpathlineto{\pgfqpoint{1.376166in}{1.648855in}}%
\pgfpathlineto{\pgfqpoint{1.377023in}{1.647880in}}%
\pgfpathlineto{\pgfqpoint{1.382927in}{1.638878in}}%
\pgfpathlineto{\pgfqpoint{1.385898in}{1.631416in}}%
\pgfpathlineto{\pgfqpoint{1.386600in}{1.629876in}}%
\pgfpathlineto{\pgfqpoint{1.387918in}{1.620875in}}%
\pgfpathlineto{\pgfqpoint{1.386574in}{1.611873in}}%
\pgfpathlineto{\pgfqpoint{1.385898in}{1.610196in}}%
\pgfpathlineto{\pgfqpoint{1.383323in}{1.602872in}}%
\pgfpathlineto{\pgfqpoint{1.378679in}{1.593870in}}%
\pgfpathlineto{\pgfqpoint{1.376166in}{1.589742in}}%
\pgfpathlineto{\pgfqpoint{1.373372in}{1.584869in}}%
\pgfpathlineto{\pgfqpoint{1.368099in}{1.575867in}}%
\pgfpathlineto{\pgfqpoint{1.366435in}{1.572686in}}%
\pgfpathlineto{\pgfqpoint{1.363354in}{1.566866in}}%
\pgfpathlineto{\pgfqpoint{1.359775in}{1.557864in}}%
\pgfpathlineto{\pgfqpoint{1.357868in}{1.548863in}}%
\pgfpathlineto{\pgfqpoint{1.357948in}{1.539861in}}%
\pgfpathlineto{\pgfqpoint{1.360123in}{1.530859in}}%
\pgfpathlineto{\pgfqpoint{1.364285in}{1.521858in}}%
\pgfpathlineto{\pgfqpoint{1.366435in}{1.518541in}}%
\pgfpathlineto{\pgfqpoint{1.370078in}{1.512856in}}%
\pgfpathlineto{\pgfqpoint{1.376166in}{1.504958in}}%
\pgfpathlineto{\pgfqpoint{1.377069in}{1.503855in}}%
\pgfpathlineto{\pgfqpoint{1.385032in}{1.494853in}}%
\pgfpathlineto{\pgfqpoint{1.385898in}{1.493859in}}%
\pgfpathlineto{\pgfqpoint{1.393898in}{1.485852in}}%
\pgfpathlineto{\pgfqpoint{1.395629in}{1.483913in}}%
\pgfpathlineto{\pgfqpoint{1.403579in}{1.476850in}}%
\pgfpathlineto{\pgfqpoint{1.405361in}{1.474844in}}%
\pgfpathlineto{\pgfqpoint{1.414359in}{1.467849in}}%
\pgfpathlineto{\pgfqpoint{1.415092in}{1.466958in}}%
\pgfpathlineto{\pgfqpoint{1.424824in}{1.460724in}}%
\pgfpathlineto{\pgfqpoint{1.434555in}{1.459794in}}%
\pgfpathlineto{\pgfqpoint{1.444287in}{1.463329in}}%
\pgfpathlineto{\pgfqpoint{1.450854in}{1.467849in}}%
\pgfpathlineto{\pgfqpoint{1.454018in}{1.469242in}}%
\pgfpathlineto{\pgfqpoint{1.463750in}{1.474500in}}%
\pgfpathlineto{\pgfqpoint{1.468266in}{1.476850in}}%
\pgfpathlineto{\pgfqpoint{1.473481in}{1.478992in}}%
\pgfpathlineto{\pgfqpoint{1.483213in}{1.481917in}}%
\pgfpathlineto{\pgfqpoint{1.492944in}{1.482942in}}%
\pgfpathlineto{\pgfqpoint{1.502676in}{1.481408in}}%
\pgfpathlineto{\pgfqpoint{1.512407in}{1.476895in}}%
\pgfpathlineto{\pgfqpoint{1.512464in}{1.476850in}}%
\pgfpathlineto{\pgfqpoint{1.521510in}{1.467849in}}%
\pgfpathlineto{\pgfqpoint{1.522139in}{1.466871in}}%
\pgfpathlineto{\pgfqpoint{1.525850in}{1.458847in}}%
\pgfpathlineto{\pgfqpoint{1.527282in}{1.449846in}}%
\pgfpathlineto{\pgfqpoint{1.525850in}{1.440844in}}%
\pgfpathlineto{\pgfqpoint{1.522139in}{1.432820in}}%
\pgfpathlineto{\pgfqpoint{1.521510in}{1.431842in}}%
\pgfpathlineto{\pgfqpoint{1.512464in}{1.422841in}}%
\pgfpathlineto{\pgfqpoint{1.512407in}{1.422796in}}%
\pgfpathlineto{\pgfqpoint{1.502676in}{1.418283in}}%
\pgfpathlineto{\pgfqpoint{1.492944in}{1.416750in}}%
\pgfpathlineto{\pgfqpoint{1.483213in}{1.417774in}}%
\pgfpathlineto{\pgfqpoint{1.473481in}{1.420699in}}%
\pgfpathlineto{\pgfqpoint{1.468266in}{1.422841in}}%
\pgfpathlineto{\pgfqpoint{1.463750in}{1.425191in}}%
\pgfpathlineto{\pgfqpoint{1.454018in}{1.430449in}}%
\pgfpathlineto{\pgfqpoint{1.450854in}{1.431842in}}%
\pgfpathlineto{\pgfqpoint{1.444287in}{1.436362in}}%
\pgfpathlineto{\pgfqpoint{1.434555in}{1.439898in}}%
\pgfpathlineto{\pgfqpoint{1.424824in}{1.438967in}}%
\pgfpathlineto{\pgfqpoint{1.415092in}{1.432733in}}%
\pgfpathlineto{\pgfqpoint{1.414359in}{1.431842in}}%
\pgfpathlineto{\pgfqpoint{1.405361in}{1.424847in}}%
\pgfpathlineto{\pgfqpoint{1.403579in}{1.422841in}}%
\pgfpathlineto{\pgfqpoint{1.395629in}{1.415778in}}%
\pgfpathlineto{\pgfqpoint{1.393898in}{1.413839in}}%
\pgfpathlineto{\pgfqpoint{1.385898in}{1.405832in}}%
\pgfpathlineto{\pgfqpoint{1.385032in}{1.404838in}}%
\pgfpathlineto{\pgfqpoint{1.377069in}{1.395836in}}%
\pgfpathlineto{\pgfqpoint{1.376166in}{1.394733in}}%
\pgfpathlineto{\pgfqpoint{1.370078in}{1.386835in}}%
\pgfpathlineto{\pgfqpoint{1.366435in}{1.381150in}}%
\pgfpathlineto{\pgfqpoint{1.364285in}{1.377833in}}%
\pgfpathlineto{\pgfqpoint{1.360123in}{1.368832in}}%
\pgfpathlineto{\pgfqpoint{1.357948in}{1.359830in}}%
\pgfpathlineto{\pgfqpoint{1.357868in}{1.350829in}}%
\pgfpathlineto{\pgfqpoint{1.359775in}{1.341827in}}%
\pgfpathlineto{\pgfqpoint{1.363354in}{1.332825in}}%
\pgfpathlineto{\pgfqpoint{1.366435in}{1.327005in}}%
\pgfpathlineto{\pgfqpoint{1.368099in}{1.323824in}}%
\pgfpathlineto{\pgfqpoint{1.373372in}{1.314822in}}%
\pgfpathlineto{\pgfqpoint{1.376166in}{1.309949in}}%
\pgfpathlineto{\pgfqpoint{1.378679in}{1.305821in}}%
\pgfpathlineto{\pgfqpoint{1.383323in}{1.296819in}}%
\pgfpathlineto{\pgfqpoint{1.385898in}{1.289495in}}%
\pgfpathlineto{\pgfqpoint{1.386574in}{1.287818in}}%
\pgfpathlineto{\pgfqpoint{1.387918in}{1.278816in}}%
\pgfpathlineto{\pgfqpoint{1.386600in}{1.269815in}}%
\pgfpathlineto{\pgfqpoint{1.385898in}{1.268275in}}%
\pgfpathlineto{\pgfqpoint{1.382927in}{1.260813in}}%
\pgfpathlineto{\pgfqpoint{1.377023in}{1.251812in}}%
\pgfpathlineto{\pgfqpoint{1.376166in}{1.250836in}}%
\pgfpathlineto{\pgfqpoint{1.369533in}{1.242810in}}%
\pgfpathlineto{\pgfqpoint{1.366435in}{1.239653in}}%
\pgfpathlineto{\pgfqpoint{1.360632in}{1.233808in}}%
\pgfpathlineto{\pgfqpoint{1.356703in}{1.230174in}}%
\pgfpathlineto{\pgfqpoint{1.350385in}{1.224807in}}%
\pgfpathlineto{\pgfqpoint{1.346972in}{1.221941in}}%
\pgfpathlineto{\pgfqpoint{1.338296in}{1.215805in}}%
\pgfpathlineto{\pgfqpoint{1.337240in}{1.215013in}}%
\pgfpathlineto{\pgfqpoint{1.327509in}{1.209552in}}%
\pgfpathlineto{\pgfqpoint{1.319442in}{1.206804in}}%
\pgfpathlineto{\pgfqpoint{1.317778in}{1.206154in}}%
\pgfpathlineto{\pgfqpoint{1.308046in}{1.204936in}}%
\pgfpathlineto{\pgfqpoint{1.298315in}{1.206179in}}%
\pgfpathlineto{\pgfqpoint{1.296501in}{1.206804in}}%
\pgfpathlineto{\pgfqpoint{1.288583in}{1.209186in}}%
\pgfpathlineto{\pgfqpoint{1.278852in}{1.213481in}}%
\pgfpathlineto{\pgfqpoint{1.274388in}{1.215805in}}%
\pgfpathlineto{\pgfqpoint{1.269120in}{1.218390in}}%
\pgfpathlineto{\pgfqpoint{1.259389in}{1.223267in}}%
\pgfpathlineto{\pgfqpoint{1.255949in}{1.224807in}}%
\pgfpathlineto{\pgfqpoint{1.249657in}{1.227656in}}%
\pgfpathlineto{\pgfqpoint{1.239926in}{1.230968in}}%
\pgfpathlineto{\pgfqpoint{1.230194in}{1.232731in}}%
\pgfpathlineto{\pgfqpoint{1.220463in}{1.232657in}}%
\pgfpathlineto{\pgfqpoint{1.210731in}{1.230645in}}%
\pgfpathlineto{\pgfqpoint{1.201000in}{1.226796in}}%
\pgfpathlineto{\pgfqpoint{1.197414in}{1.224807in}}%
\pgfpathlineto{\pgfqpoint{1.191268in}{1.221437in}}%
\pgfpathlineto{\pgfqpoint{1.182730in}{1.215805in}}%
\pgfpathlineto{\pgfqpoint{1.181537in}{1.214970in}}%
\pgfpathlineto{\pgfqpoint{1.171805in}{1.207605in}}%
\pgfpathlineto{\pgfqpoint{1.170731in}{1.206804in}}%
\pgfpathlineto{\pgfqpoint{1.162074in}{1.199404in}}%
\pgfpathlineto{\pgfqpoint{1.159978in}{1.197802in}}%
\pgfpathlineto{\pgfqpoint{1.152342in}{1.190449in}}%
\pgfpathlineto{\pgfqpoint{1.150174in}{1.188801in}}%
\pgfpathlineto{\pgfqpoint{1.142611in}{1.180477in}}%
\pgfpathlineto{\pgfqpoint{1.141648in}{1.179799in}}%
\pgfpathlineto{\pgfqpoint{1.134909in}{1.170798in}}%
\pgfpathlineto{\pgfqpoint{1.133903in}{1.161796in}}%
\pgfpathlineto{\pgfqpoint{1.137725in}{1.152795in}}%
\pgfpathlineto{\pgfqpoint{1.142611in}{1.146720in}}%
\pgfpathlineto{\pgfqpoint{1.144117in}{1.143793in}}%
\pgfpathlineto{\pgfqpoint{1.149801in}{1.134792in}}%
\pgfpathlineto{\pgfqpoint{1.152342in}{1.130614in}}%
\pgfpathlineto{\pgfqpoint{1.154658in}{1.125790in}}%
\pgfpathlineto{\pgfqpoint{1.157820in}{1.116788in}}%
\pgfpathlineto{\pgfqpoint{1.158928in}{1.107787in}}%
\pgfpathlineto{\pgfqpoint{1.157270in}{1.098785in}}%
\pgfpathlineto{\pgfqpoint{1.152391in}{1.089784in}}%
\pgfpathlineto{\pgfqpoint{1.152342in}{1.089731in}}%
\pgfpathlineto{\pgfqpoint{1.142611in}{1.081364in}}%
\pgfpathlineto{\pgfqpoint{1.141554in}{1.080782in}}%
\pgfpathlineto{\pgfqpoint{1.132879in}{1.077350in}}%
\pgfpathlineto{\pgfqpoint{1.123148in}{1.076025in}}%
\pgfpathlineto{\pgfqpoint{1.113416in}{1.077350in}}%
\pgfpathclose%
\pgfusepath{fill}%
\end{pgfscope}%
\begin{pgfscope}%
\pgfpathrectangle{\pgfqpoint{0.150000in}{0.549691in}}{\pgfqpoint{1.946296in}{1.800309in}}%
\pgfusepath{clip}%
\pgfsetbuttcap%
\pgfsetroundjoin%
\definecolor{currentfill}{rgb}{0.160784,0.160784,0.160784}%
\pgfsetfillcolor{currentfill}%
\pgfsetlinewidth{0.000000pt}%
\definecolor{currentstroke}{rgb}{0.000000,0.000000,0.000000}%
\pgfsetstrokecolor{currentstroke}%
\pgfsetdash{}{0pt}%
\pgfpathmoveto{\pgfqpoint{1.016102in}{1.343276in}}%
\pgfpathlineto{\pgfqpoint{1.025833in}{1.343655in}}%
\pgfpathlineto{\pgfqpoint{1.032631in}{1.350829in}}%
\pgfpathlineto{\pgfqpoint{1.032272in}{1.359830in}}%
\pgfpathlineto{\pgfqpoint{1.025833in}{1.365786in}}%
\pgfpathlineto{\pgfqpoint{1.016102in}{1.366118in}}%
\pgfpathlineto{\pgfqpoint{1.008346in}{1.359830in}}%
\pgfpathlineto{\pgfqpoint{1.007937in}{1.350829in}}%
\pgfpathclose%
\pgfusepath{fill}%
\end{pgfscope}%
\begin{pgfscope}%
\pgfpathrectangle{\pgfqpoint{0.150000in}{0.549691in}}{\pgfqpoint{1.946296in}{1.800309in}}%
\pgfusepath{clip}%
\pgfsetbuttcap%
\pgfsetroundjoin%
\definecolor{currentfill}{rgb}{0.160784,0.160784,0.160784}%
\pgfsetfillcolor{currentfill}%
\pgfsetlinewidth{0.000000pt}%
\definecolor{currentstroke}{rgb}{0.000000,0.000000,0.000000}%
\pgfsetstrokecolor{currentstroke}%
\pgfsetdash{}{0pt}%
\pgfpathmoveto{\pgfqpoint{1.220463in}{1.343655in}}%
\pgfpathlineto{\pgfqpoint{1.230194in}{1.343276in}}%
\pgfpathlineto{\pgfqpoint{1.238359in}{1.350829in}}%
\pgfpathlineto{\pgfqpoint{1.237950in}{1.359830in}}%
\pgfpathlineto{\pgfqpoint{1.230194in}{1.366118in}}%
\pgfpathlineto{\pgfqpoint{1.220463in}{1.365786in}}%
\pgfpathlineto{\pgfqpoint{1.214023in}{1.359830in}}%
\pgfpathlineto{\pgfqpoint{1.213665in}{1.350829in}}%
\pgfpathclose%
\pgfusepath{fill}%
\end{pgfscope}%
\begin{pgfscope}%
\pgfpathrectangle{\pgfqpoint{0.150000in}{0.549691in}}{\pgfqpoint{1.946296in}{1.800309in}}%
\pgfusepath{clip}%
\pgfsetbuttcap%
\pgfsetroundjoin%
\definecolor{currentfill}{rgb}{0.160784,0.160784,0.160784}%
\pgfsetfillcolor{currentfill}%
\pgfsetlinewidth{0.000000pt}%
\definecolor{currentstroke}{rgb}{0.000000,0.000000,0.000000}%
\pgfsetstrokecolor{currentstroke}%
\pgfsetdash{}{0pt}%
\pgfpathmoveto{\pgfqpoint{1.016102in}{1.533573in}}%
\pgfpathlineto{\pgfqpoint{1.025833in}{1.533905in}}%
\pgfpathlineto{\pgfqpoint{1.032272in}{1.539861in}}%
\pgfpathlineto{\pgfqpoint{1.032631in}{1.548863in}}%
\pgfpathlineto{\pgfqpoint{1.025833in}{1.556036in}}%
\pgfpathlineto{\pgfqpoint{1.016102in}{1.556415in}}%
\pgfpathlineto{\pgfqpoint{1.007937in}{1.548863in}}%
\pgfpathlineto{\pgfqpoint{1.008346in}{1.539861in}}%
\pgfpathclose%
\pgfusepath{fill}%
\end{pgfscope}%
\begin{pgfscope}%
\pgfpathrectangle{\pgfqpoint{0.150000in}{0.549691in}}{\pgfqpoint{1.946296in}{1.800309in}}%
\pgfusepath{clip}%
\pgfsetbuttcap%
\pgfsetroundjoin%
\definecolor{currentfill}{rgb}{0.160784,0.160784,0.160784}%
\pgfsetfillcolor{currentfill}%
\pgfsetlinewidth{0.000000pt}%
\definecolor{currentstroke}{rgb}{0.000000,0.000000,0.000000}%
\pgfsetstrokecolor{currentstroke}%
\pgfsetdash{}{0pt}%
\pgfpathmoveto{\pgfqpoint{1.220463in}{1.533905in}}%
\pgfpathlineto{\pgfqpoint{1.230194in}{1.533573in}}%
\pgfpathlineto{\pgfqpoint{1.237950in}{1.539861in}}%
\pgfpathlineto{\pgfqpoint{1.238359in}{1.548863in}}%
\pgfpathlineto{\pgfqpoint{1.230194in}{1.556415in}}%
\pgfpathlineto{\pgfqpoint{1.220463in}{1.556036in}}%
\pgfpathlineto{\pgfqpoint{1.213665in}{1.548863in}}%
\pgfpathlineto{\pgfqpoint{1.214023in}{1.539861in}}%
\pgfpathclose%
\pgfusepath{fill}%
\end{pgfscope}%
\begin{pgfscope}%
\pgfpathrectangle{\pgfqpoint{0.150000in}{0.549691in}}{\pgfqpoint{1.946296in}{1.800309in}}%
\pgfusepath{clip}%
\pgfsetbuttcap%
\pgfsetroundjoin%
\definecolor{currentfill}{rgb}{0.188235,0.188235,0.188235}%
\pgfsetfillcolor{currentfill}%
\pgfsetlinewidth{0.000000pt}%
\definecolor{currentstroke}{rgb}{0.000000,0.000000,0.000000}%
\pgfsetstrokecolor{currentstroke}%
\pgfsetdash{}{0pt}%
\pgfpathmoveto{\pgfqpoint{1.103685in}{0.897783in}}%
\pgfpathlineto{\pgfqpoint{1.113416in}{0.894788in}}%
\pgfpathlineto{\pgfqpoint{1.123148in}{0.893758in}}%
\pgfpathlineto{\pgfqpoint{1.132879in}{0.894788in}}%
\pgfpathlineto{\pgfqpoint{1.142611in}{0.897783in}}%
\pgfpathlineto{\pgfqpoint{1.148784in}{0.900751in}}%
\pgfpathlineto{\pgfqpoint{1.152342in}{0.902940in}}%
\pgfpathlineto{\pgfqpoint{1.161079in}{0.909753in}}%
\pgfpathlineto{\pgfqpoint{1.162074in}{0.910850in}}%
\pgfpathlineto{\pgfqpoint{1.168481in}{0.918754in}}%
\pgfpathlineto{\pgfqpoint{1.171805in}{0.925530in}}%
\pgfpathlineto{\pgfqpoint{1.172881in}{0.927756in}}%
\pgfpathlineto{\pgfqpoint{1.174798in}{0.936758in}}%
\pgfpathlineto{\pgfqpoint{1.174906in}{0.945759in}}%
\pgfpathlineto{\pgfqpoint{1.173994in}{0.954761in}}%
\pgfpathlineto{\pgfqpoint{1.172938in}{0.963762in}}%
\pgfpathlineto{\pgfqpoint{1.172612in}{0.972764in}}%
\pgfpathlineto{\pgfqpoint{1.173806in}{0.981765in}}%
\pgfpathlineto{\pgfqpoint{1.177145in}{0.990767in}}%
\pgfpathlineto{\pgfqpoint{1.181537in}{0.997483in}}%
\pgfpathlineto{\pgfqpoint{1.183152in}{0.999768in}}%
\pgfpathlineto{\pgfqpoint{1.191268in}{1.007651in}}%
\pgfpathlineto{\pgfqpoint{1.192654in}{1.008770in}}%
\pgfpathlineto{\pgfqpoint{1.201000in}{1.013964in}}%
\pgfpathlineto{\pgfqpoint{1.209577in}{1.017771in}}%
\pgfpathlineto{\pgfqpoint{1.210731in}{1.018201in}}%
\pgfpathlineto{\pgfqpoint{1.220463in}{1.020093in}}%
\pgfpathlineto{\pgfqpoint{1.230194in}{1.020163in}}%
\pgfpathlineto{\pgfqpoint{1.239926in}{1.018504in}}%
\pgfpathlineto{\pgfqpoint{1.242215in}{1.017771in}}%
\pgfpathlineto{\pgfqpoint{1.249657in}{1.014930in}}%
\pgfpathlineto{\pgfqpoint{1.259389in}{1.009984in}}%
\pgfpathlineto{\pgfqpoint{1.261523in}{1.008770in}}%
\pgfpathlineto{\pgfqpoint{1.269120in}{1.003163in}}%
\pgfpathlineto{\pgfqpoint{1.273820in}{0.999768in}}%
\pgfpathlineto{\pgfqpoint{1.278852in}{0.994475in}}%
\pgfpathlineto{\pgfqpoint{1.283009in}{0.990767in}}%
\pgfpathlineto{\pgfqpoint{1.288583in}{0.982002in}}%
\pgfpathlineto{\pgfqpoint{1.288804in}{0.981765in}}%
\pgfpathlineto{\pgfqpoint{1.291807in}{0.972764in}}%
\pgfpathlineto{\pgfqpoint{1.290987in}{0.963762in}}%
\pgfpathlineto{\pgfqpoint{1.288583in}{0.955617in}}%
\pgfpathlineto{\pgfqpoint{1.288411in}{0.954761in}}%
\pgfpathlineto{\pgfqpoint{1.286848in}{0.945759in}}%
\pgfpathlineto{\pgfqpoint{1.287033in}{0.936758in}}%
\pgfpathlineto{\pgfqpoint{1.288583in}{0.932511in}}%
\pgfpathlineto{\pgfqpoint{1.291131in}{0.927756in}}%
\pgfpathlineto{\pgfqpoint{1.298315in}{0.921850in}}%
\pgfpathlineto{\pgfqpoint{1.308046in}{0.918886in}}%
\pgfpathlineto{\pgfqpoint{1.317778in}{0.921791in}}%
\pgfpathlineto{\pgfqpoint{1.324184in}{0.927756in}}%
\pgfpathlineto{\pgfqpoint{1.327509in}{0.934784in}}%
\pgfpathlineto{\pgfqpoint{1.328076in}{0.936758in}}%
\pgfpathlineto{\pgfqpoint{1.328221in}{0.945759in}}%
\pgfpathlineto{\pgfqpoint{1.327509in}{0.950975in}}%
\pgfpathlineto{\pgfqpoint{1.326657in}{0.954761in}}%
\pgfpathlineto{\pgfqpoint{1.324311in}{0.963762in}}%
\pgfpathlineto{\pgfqpoint{1.323587in}{0.972764in}}%
\pgfpathlineto{\pgfqpoint{1.326239in}{0.981765in}}%
\pgfpathlineto{\pgfqpoint{1.327509in}{0.983307in}}%
\pgfpathlineto{\pgfqpoint{1.331241in}{0.990767in}}%
\pgfpathlineto{\pgfqpoint{1.337240in}{0.997570in}}%
\pgfpathlineto{\pgfqpoint{1.338690in}{0.999768in}}%
\pgfpathlineto{\pgfqpoint{1.346972in}{1.008393in}}%
\pgfpathlineto{\pgfqpoint{1.347277in}{1.008770in}}%
\pgfpathlineto{\pgfqpoint{1.356703in}{1.017755in}}%
\pgfpathlineto{\pgfqpoint{1.356719in}{1.017771in}}%
\pgfpathlineto{\pgfqpoint{1.366435in}{1.026225in}}%
\pgfpathlineto{\pgfqpoint{1.367057in}{1.026773in}}%
\pgfpathlineto{\pgfqpoint{1.376166in}{1.033967in}}%
\pgfpathlineto{\pgfqpoint{1.378595in}{1.035775in}}%
\pgfpathlineto{\pgfqpoint{1.385898in}{1.040952in}}%
\pgfpathlineto{\pgfqpoint{1.392080in}{1.044776in}}%
\pgfpathlineto{\pgfqpoint{1.395629in}{1.046992in}}%
\pgfpathlineto{\pgfqpoint{1.405361in}{1.051811in}}%
\pgfpathlineto{\pgfqpoint{1.411115in}{1.053778in}}%
\pgfpathlineto{\pgfqpoint{1.415092in}{1.055234in}}%
\pgfpathlineto{\pgfqpoint{1.424824in}{1.057112in}}%
\pgfpathlineto{\pgfqpoint{1.434555in}{1.057393in}}%
\pgfpathlineto{\pgfqpoint{1.444287in}{1.056327in}}%
\pgfpathlineto{\pgfqpoint{1.454018in}{1.054309in}}%
\pgfpathlineto{\pgfqpoint{1.456108in}{1.053778in}}%
\pgfpathlineto{\pgfqpoint{1.463750in}{1.051962in}}%
\pgfpathlineto{\pgfqpoint{1.473481in}{1.049735in}}%
\pgfpathlineto{\pgfqpoint{1.483213in}{1.048104in}}%
\pgfpathlineto{\pgfqpoint{1.492944in}{1.047533in}}%
\pgfpathlineto{\pgfqpoint{1.502676in}{1.048388in}}%
\pgfpathlineto{\pgfqpoint{1.512407in}{1.050904in}}%
\pgfpathlineto{\pgfqpoint{1.518975in}{1.053778in}}%
\pgfpathlineto{\pgfqpoint{1.522139in}{1.055260in}}%
\pgfpathlineto{\pgfqpoint{1.531870in}{1.061600in}}%
\pgfpathlineto{\pgfqpoint{1.533328in}{1.062779in}}%
\pgfpathlineto{\pgfqpoint{1.541602in}{1.070432in}}%
\pgfpathlineto{\pgfqpoint{1.542876in}{1.071781in}}%
\pgfpathlineto{\pgfqpoint{1.549730in}{1.080782in}}%
\pgfpathlineto{\pgfqpoint{1.551333in}{1.083708in}}%
\pgfpathlineto{\pgfqpoint{1.554440in}{1.089784in}}%
\pgfpathlineto{\pgfqpoint{1.557160in}{1.098785in}}%
\pgfpathlineto{\pgfqpoint{1.558084in}{1.107787in}}%
\pgfpathlineto{\pgfqpoint{1.557466in}{1.116788in}}%
\pgfpathlineto{\pgfqpoint{1.555704in}{1.125790in}}%
\pgfpathlineto{\pgfqpoint{1.553296in}{1.134792in}}%
\pgfpathlineto{\pgfqpoint{1.551333in}{1.141860in}}%
\pgfpathlineto{\pgfqpoint{1.550758in}{1.143793in}}%
\pgfpathlineto{\pgfqpoint{1.548576in}{1.152795in}}%
\pgfpathlineto{\pgfqpoint{1.547425in}{1.161796in}}%
\pgfpathlineto{\pgfqpoint{1.547728in}{1.170798in}}%
\pgfpathlineto{\pgfqpoint{1.549759in}{1.179799in}}%
\pgfpathlineto{\pgfqpoint{1.551333in}{1.183478in}}%
\pgfpathlineto{\pgfqpoint{1.553460in}{1.188801in}}%
\pgfpathlineto{\pgfqpoint{1.558669in}{1.197802in}}%
\pgfpathlineto{\pgfqpoint{1.561064in}{1.201085in}}%
\pgfpathlineto{\pgfqpoint{1.565198in}{1.206804in}}%
\pgfpathlineto{\pgfqpoint{1.570796in}{1.213559in}}%
\pgfpathlineto{\pgfqpoint{1.572750in}{1.215805in}}%
\pgfpathlineto{\pgfqpoint{1.580527in}{1.224232in}}%
\pgfpathlineto{\pgfqpoint{1.581119in}{1.224807in}}%
\pgfpathlineto{\pgfqpoint{1.590259in}{1.233794in}}%
\pgfpathlineto{\pgfqpoint{1.590276in}{1.233808in}}%
\pgfpathlineto{\pgfqpoint{1.599990in}{1.242528in}}%
\pgfpathlineto{\pgfqpoint{1.600398in}{1.242810in}}%
\pgfpathlineto{\pgfqpoint{1.609722in}{1.250471in}}%
\pgfpathlineto{\pgfqpoint{1.612098in}{1.251812in}}%
\pgfpathlineto{\pgfqpoint{1.619453in}{1.257361in}}%
\pgfpathlineto{\pgfqpoint{1.627518in}{1.260813in}}%
\pgfpathlineto{\pgfqpoint{1.629185in}{1.261988in}}%
\pgfpathlineto{\pgfqpoint{1.638916in}{1.264441in}}%
\pgfpathlineto{\pgfqpoint{1.648648in}{1.263771in}}%
\pgfpathlineto{\pgfqpoint{1.658379in}{1.261601in}}%
\pgfpathlineto{\pgfqpoint{1.662472in}{1.260813in}}%
\pgfpathlineto{\pgfqpoint{1.668111in}{1.260154in}}%
\pgfpathlineto{\pgfqpoint{1.677842in}{1.260289in}}%
\pgfpathlineto{\pgfqpoint{1.679976in}{1.260813in}}%
\pgfpathlineto{\pgfqpoint{1.687574in}{1.263888in}}%
\pgfpathlineto{\pgfqpoint{1.694023in}{1.269815in}}%
\pgfpathlineto{\pgfqpoint{1.697163in}{1.278816in}}%
\pgfpathlineto{\pgfqpoint{1.693959in}{1.287818in}}%
\pgfpathlineto{\pgfqpoint{1.687574in}{1.294463in}}%
\pgfpathlineto{\pgfqpoint{1.682433in}{1.296819in}}%
\pgfpathlineto{\pgfqpoint{1.677842in}{1.298253in}}%
\pgfpathlineto{\pgfqpoint{1.668111in}{1.298424in}}%
\pgfpathlineto{\pgfqpoint{1.658379in}{1.296979in}}%
\pgfpathlineto{\pgfqpoint{1.657453in}{1.296819in}}%
\pgfpathlineto{\pgfqpoint{1.648648in}{1.294595in}}%
\pgfpathlineto{\pgfqpoint{1.638916in}{1.293837in}}%
\pgfpathlineto{\pgfqpoint{1.629185in}{1.296615in}}%
\pgfpathlineto{\pgfqpoint{1.628928in}{1.296819in}}%
\pgfpathlineto{\pgfqpoint{1.619453in}{1.301975in}}%
\pgfpathlineto{\pgfqpoint{1.615445in}{1.305821in}}%
\pgfpathlineto{\pgfqpoint{1.609722in}{1.310475in}}%
\pgfpathlineto{\pgfqpoint{1.606051in}{1.314822in}}%
\pgfpathlineto{\pgfqpoint{1.599990in}{1.321849in}}%
\pgfpathlineto{\pgfqpoint{1.598677in}{1.323824in}}%
\pgfpathlineto{\pgfqpoint{1.593331in}{1.332825in}}%
\pgfpathlineto{\pgfqpoint{1.590259in}{1.339709in}}%
\pgfpathlineto{\pgfqpoint{1.589467in}{1.341827in}}%
\pgfpathlineto{\pgfqpoint{1.587673in}{1.350829in}}%
\pgfpathlineto{\pgfqpoint{1.587749in}{1.359830in}}%
\pgfpathlineto{\pgfqpoint{1.589795in}{1.368832in}}%
\pgfpathlineto{\pgfqpoint{1.590259in}{1.369899in}}%
\pgfpathlineto{\pgfqpoint{1.594375in}{1.377833in}}%
\pgfpathlineto{\pgfqpoint{1.599990in}{1.385553in}}%
\pgfpathlineto{\pgfqpoint{1.601200in}{1.386835in}}%
\pgfpathlineto{\pgfqpoint{1.609722in}{1.394342in}}%
\pgfpathlineto{\pgfqpoint{1.612192in}{1.395836in}}%
\pgfpathlineto{\pgfqpoint{1.619453in}{1.399899in}}%
\pgfpathlineto{\pgfqpoint{1.629185in}{1.402988in}}%
\pgfpathlineto{\pgfqpoint{1.638916in}{1.404091in}}%
\pgfpathlineto{\pgfqpoint{1.648648in}{1.403790in}}%
\pgfpathlineto{\pgfqpoint{1.658379in}{1.402813in}}%
\pgfpathlineto{\pgfqpoint{1.668111in}{1.401970in}}%
\pgfpathlineto{\pgfqpoint{1.677842in}{1.402070in}}%
\pgfpathlineto{\pgfqpoint{1.687574in}{1.403843in}}%
\pgfpathlineto{\pgfqpoint{1.689980in}{1.404838in}}%
\pgfpathlineto{\pgfqpoint{1.697305in}{1.407913in}}%
\pgfpathlineto{\pgfqpoint{1.705851in}{1.413839in}}%
\pgfpathlineto{\pgfqpoint{1.707037in}{1.414760in}}%
\pgfpathlineto{\pgfqpoint{1.714402in}{1.422841in}}%
\pgfpathlineto{\pgfqpoint{1.716768in}{1.426133in}}%
\pgfpathlineto{\pgfqpoint{1.719978in}{1.431842in}}%
\pgfpathlineto{\pgfqpoint{1.723215in}{1.440844in}}%
\pgfpathlineto{\pgfqpoint{1.724329in}{1.449846in}}%
\pgfpathlineto{\pgfqpoint{1.723215in}{1.458847in}}%
\pgfpathlineto{\pgfqpoint{1.719978in}{1.467849in}}%
\pgfpathlineto{\pgfqpoint{1.716768in}{1.473559in}}%
\pgfpathlineto{\pgfqpoint{1.714402in}{1.476850in}}%
\pgfpathlineto{\pgfqpoint{1.707037in}{1.484931in}}%
\pgfpathlineto{\pgfqpoint{1.705851in}{1.485852in}}%
\pgfpathlineto{\pgfqpoint{1.697305in}{1.491778in}}%
\pgfpathlineto{\pgfqpoint{1.689980in}{1.494853in}}%
\pgfpathlineto{\pgfqpoint{1.687574in}{1.495849in}}%
\pgfpathlineto{\pgfqpoint{1.677842in}{1.497621in}}%
\pgfpathlineto{\pgfqpoint{1.668111in}{1.497721in}}%
\pgfpathlineto{\pgfqpoint{1.658379in}{1.496878in}}%
\pgfpathlineto{\pgfqpoint{1.648648in}{1.495901in}}%
\pgfpathlineto{\pgfqpoint{1.638916in}{1.495600in}}%
\pgfpathlineto{\pgfqpoint{1.629185in}{1.496704in}}%
\pgfpathlineto{\pgfqpoint{1.619453in}{1.499792in}}%
\pgfpathlineto{\pgfqpoint{1.612192in}{1.503855in}}%
\pgfpathlineto{\pgfqpoint{1.609722in}{1.505349in}}%
\pgfpathlineto{\pgfqpoint{1.601200in}{1.512856in}}%
\pgfpathlineto{\pgfqpoint{1.599990in}{1.514138in}}%
\pgfpathlineto{\pgfqpoint{1.594375in}{1.521858in}}%
\pgfpathlineto{\pgfqpoint{1.590259in}{1.529792in}}%
\pgfpathlineto{\pgfqpoint{1.589795in}{1.530859in}}%
\pgfpathlineto{\pgfqpoint{1.587749in}{1.539861in}}%
\pgfpathlineto{\pgfqpoint{1.587673in}{1.548863in}}%
\pgfpathlineto{\pgfqpoint{1.589467in}{1.557864in}}%
\pgfpathlineto{\pgfqpoint{1.590259in}{1.559982in}}%
\pgfpathlineto{\pgfqpoint{1.593331in}{1.566866in}}%
\pgfpathlineto{\pgfqpoint{1.598677in}{1.575867in}}%
\pgfpathlineto{\pgfqpoint{1.599990in}{1.577842in}}%
\pgfpathlineto{\pgfqpoint{1.606051in}{1.584869in}}%
\pgfpathlineto{\pgfqpoint{1.609722in}{1.589216in}}%
\pgfpathlineto{\pgfqpoint{1.615445in}{1.593870in}}%
\pgfpathlineto{\pgfqpoint{1.619453in}{1.597716in}}%
\pgfpathlineto{\pgfqpoint{1.628928in}{1.602872in}}%
\pgfpathlineto{\pgfqpoint{1.629185in}{1.603077in}}%
\pgfpathlineto{\pgfqpoint{1.638916in}{1.605854in}}%
\pgfpathlineto{\pgfqpoint{1.648648in}{1.605096in}}%
\pgfpathlineto{\pgfqpoint{1.657453in}{1.602872in}}%
\pgfpathlineto{\pgfqpoint{1.658379in}{1.602712in}}%
\pgfpathlineto{\pgfqpoint{1.668111in}{1.601267in}}%
\pgfpathlineto{\pgfqpoint{1.677842in}{1.601438in}}%
\pgfpathlineto{\pgfqpoint{1.682433in}{1.602872in}}%
\pgfpathlineto{\pgfqpoint{1.687574in}{1.605228in}}%
\pgfpathlineto{\pgfqpoint{1.693959in}{1.611873in}}%
\pgfpathlineto{\pgfqpoint{1.697163in}{1.620875in}}%
\pgfpathlineto{\pgfqpoint{1.694023in}{1.629876in}}%
\pgfpathlineto{\pgfqpoint{1.687574in}{1.635803in}}%
\pgfpathlineto{\pgfqpoint{1.679976in}{1.638878in}}%
\pgfpathlineto{\pgfqpoint{1.677842in}{1.639402in}}%
\pgfpathlineto{\pgfqpoint{1.668111in}{1.639537in}}%
\pgfpathlineto{\pgfqpoint{1.662472in}{1.638878in}}%
\pgfpathlineto{\pgfqpoint{1.658379in}{1.638090in}}%
\pgfpathlineto{\pgfqpoint{1.648648in}{1.635920in}}%
\pgfpathlineto{\pgfqpoint{1.638916in}{1.635250in}}%
\pgfpathlineto{\pgfqpoint{1.629185in}{1.637703in}}%
\pgfpathlineto{\pgfqpoint{1.627518in}{1.638878in}}%
\pgfpathlineto{\pgfqpoint{1.619453in}{1.642330in}}%
\pgfpathlineto{\pgfqpoint{1.612098in}{1.647880in}}%
\pgfpathlineto{\pgfqpoint{1.609722in}{1.649220in}}%
\pgfpathlineto{\pgfqpoint{1.600398in}{1.656881in}}%
\pgfpathlineto{\pgfqpoint{1.599990in}{1.657163in}}%
\pgfpathlineto{\pgfqpoint{1.590276in}{1.665883in}}%
\pgfpathlineto{\pgfqpoint{1.590259in}{1.665897in}}%
\pgfpathlineto{\pgfqpoint{1.581119in}{1.674884in}}%
\pgfpathlineto{\pgfqpoint{1.580527in}{1.675460in}}%
\pgfpathlineto{\pgfqpoint{1.572750in}{1.683886in}}%
\pgfpathlineto{\pgfqpoint{1.570796in}{1.686133in}}%
\pgfpathlineto{\pgfqpoint{1.565198in}{1.692887in}}%
\pgfpathlineto{\pgfqpoint{1.561064in}{1.698606in}}%
\pgfpathlineto{\pgfqpoint{1.558669in}{1.701889in}}%
\pgfpathlineto{\pgfqpoint{1.553460in}{1.710890in}}%
\pgfpathlineto{\pgfqpoint{1.551333in}{1.716213in}}%
\pgfpathlineto{\pgfqpoint{1.549759in}{1.719892in}}%
\pgfpathlineto{\pgfqpoint{1.547728in}{1.728893in}}%
\pgfpathlineto{\pgfqpoint{1.547425in}{1.737895in}}%
\pgfpathlineto{\pgfqpoint{1.548576in}{1.746897in}}%
\pgfpathlineto{\pgfqpoint{1.550758in}{1.755898in}}%
\pgfpathlineto{\pgfqpoint{1.551333in}{1.757831in}}%
\pgfpathlineto{\pgfqpoint{1.553296in}{1.764900in}}%
\pgfpathlineto{\pgfqpoint{1.555704in}{1.773901in}}%
\pgfpathlineto{\pgfqpoint{1.557466in}{1.782903in}}%
\pgfpathlineto{\pgfqpoint{1.558084in}{1.791904in}}%
\pgfpathlineto{\pgfqpoint{1.557160in}{1.800906in}}%
\pgfpathlineto{\pgfqpoint{1.554440in}{1.809907in}}%
\pgfpathlineto{\pgfqpoint{1.551333in}{1.815983in}}%
\pgfpathlineto{\pgfqpoint{1.549730in}{1.818909in}}%
\pgfpathlineto{\pgfqpoint{1.542876in}{1.827910in}}%
\pgfpathlineto{\pgfqpoint{1.541602in}{1.829259in}}%
\pgfpathlineto{\pgfqpoint{1.533328in}{1.836912in}}%
\pgfpathlineto{\pgfqpoint{1.531870in}{1.838091in}}%
\pgfpathlineto{\pgfqpoint{1.522139in}{1.844431in}}%
\pgfpathlineto{\pgfqpoint{1.518975in}{1.845914in}}%
\pgfpathlineto{\pgfqpoint{1.512407in}{1.848788in}}%
\pgfpathlineto{\pgfqpoint{1.502676in}{1.851303in}}%
\pgfpathlineto{\pgfqpoint{1.492944in}{1.852158in}}%
\pgfpathlineto{\pgfqpoint{1.483213in}{1.851587in}}%
\pgfpathlineto{\pgfqpoint{1.473481in}{1.849956in}}%
\pgfpathlineto{\pgfqpoint{1.463750in}{1.847729in}}%
\pgfpathlineto{\pgfqpoint{1.456108in}{1.845914in}}%
\pgfpathlineto{\pgfqpoint{1.454018in}{1.845382in}}%
\pgfpathlineto{\pgfqpoint{1.444287in}{1.843364in}}%
\pgfpathlineto{\pgfqpoint{1.434555in}{1.842298in}}%
\pgfpathlineto{\pgfqpoint{1.424824in}{1.842579in}}%
\pgfpathlineto{\pgfqpoint{1.415092in}{1.844457in}}%
\pgfpathlineto{\pgfqpoint{1.411115in}{1.845914in}}%
\pgfpathlineto{\pgfqpoint{1.405361in}{1.847881in}}%
\pgfpathlineto{\pgfqpoint{1.395629in}{1.852699in}}%
\pgfpathlineto{\pgfqpoint{1.392080in}{1.854915in}}%
\pgfpathlineto{\pgfqpoint{1.385898in}{1.858739in}}%
\pgfpathlineto{\pgfqpoint{1.378595in}{1.863917in}}%
\pgfpathlineto{\pgfqpoint{1.376166in}{1.865724in}}%
\pgfpathlineto{\pgfqpoint{1.367057in}{1.872918in}}%
\pgfpathlineto{\pgfqpoint{1.366435in}{1.873466in}}%
\pgfpathlineto{\pgfqpoint{1.356719in}{1.881920in}}%
\pgfpathlineto{\pgfqpoint{1.356703in}{1.881936in}}%
\pgfpathlineto{\pgfqpoint{1.347277in}{1.890921in}}%
\pgfpathlineto{\pgfqpoint{1.346972in}{1.891298in}}%
\pgfpathlineto{\pgfqpoint{1.338690in}{1.899923in}}%
\pgfpathlineto{\pgfqpoint{1.337240in}{1.902121in}}%
\pgfpathlineto{\pgfqpoint{1.331241in}{1.908924in}}%
\pgfpathlineto{\pgfqpoint{1.327509in}{1.916384in}}%
\pgfpathlineto{\pgfqpoint{1.326239in}{1.917926in}}%
\pgfpathlineto{\pgfqpoint{1.323587in}{1.926927in}}%
\pgfpathlineto{\pgfqpoint{1.324311in}{1.935929in}}%
\pgfpathlineto{\pgfqpoint{1.326657in}{1.944931in}}%
\pgfpathlineto{\pgfqpoint{1.327509in}{1.948716in}}%
\pgfpathlineto{\pgfqpoint{1.328221in}{1.953932in}}%
\pgfpathlineto{\pgfqpoint{1.328076in}{1.962934in}}%
\pgfpathlineto{\pgfqpoint{1.327509in}{1.964907in}}%
\pgfpathlineto{\pgfqpoint{1.324184in}{1.971935in}}%
\pgfpathlineto{\pgfqpoint{1.317778in}{1.977900in}}%
\pgfpathlineto{\pgfqpoint{1.308046in}{1.980805in}}%
\pgfpathlineto{\pgfqpoint{1.298315in}{1.977841in}}%
\pgfpathlineto{\pgfqpoint{1.291131in}{1.971935in}}%
\pgfpathlineto{\pgfqpoint{1.288583in}{1.967180in}}%
\pgfpathlineto{\pgfqpoint{1.287033in}{1.962934in}}%
\pgfpathlineto{\pgfqpoint{1.286848in}{1.953932in}}%
\pgfpathlineto{\pgfqpoint{1.288411in}{1.944931in}}%
\pgfpathlineto{\pgfqpoint{1.288583in}{1.944074in}}%
\pgfpathlineto{\pgfqpoint{1.290987in}{1.935929in}}%
\pgfpathlineto{\pgfqpoint{1.291807in}{1.926927in}}%
\pgfpathlineto{\pgfqpoint{1.288804in}{1.917926in}}%
\pgfpathlineto{\pgfqpoint{1.288583in}{1.917689in}}%
\pgfpathlineto{\pgfqpoint{1.283009in}{1.908924in}}%
\pgfpathlineto{\pgfqpoint{1.278852in}{1.905216in}}%
\pgfpathlineto{\pgfqpoint{1.273820in}{1.899923in}}%
\pgfpathlineto{\pgfqpoint{1.269120in}{1.896528in}}%
\pgfpathlineto{\pgfqpoint{1.261523in}{1.890921in}}%
\pgfpathlineto{\pgfqpoint{1.259389in}{1.889707in}}%
\pgfpathlineto{\pgfqpoint{1.249657in}{1.884761in}}%
\pgfpathlineto{\pgfqpoint{1.242215in}{1.881920in}}%
\pgfpathlineto{\pgfqpoint{1.239926in}{1.881187in}}%
\pgfpathlineto{\pgfqpoint{1.230194in}{1.879528in}}%
\pgfpathlineto{\pgfqpoint{1.220463in}{1.879598in}}%
\pgfpathlineto{\pgfqpoint{1.210731in}{1.881490in}}%
\pgfpathlineto{\pgfqpoint{1.209577in}{1.881920in}}%
\pgfpathlineto{\pgfqpoint{1.201000in}{1.885727in}}%
\pgfpathlineto{\pgfqpoint{1.192654in}{1.890921in}}%
\pgfpathlineto{\pgfqpoint{1.191268in}{1.892040in}}%
\pgfpathlineto{\pgfqpoint{1.183152in}{1.899923in}}%
\pgfpathlineto{\pgfqpoint{1.181537in}{1.902208in}}%
\pgfpathlineto{\pgfqpoint{1.177145in}{1.908924in}}%
\pgfpathlineto{\pgfqpoint{1.173806in}{1.917926in}}%
\pgfpathlineto{\pgfqpoint{1.172612in}{1.926927in}}%
\pgfpathlineto{\pgfqpoint{1.172938in}{1.935929in}}%
\pgfpathlineto{\pgfqpoint{1.173994in}{1.944931in}}%
\pgfpathlineto{\pgfqpoint{1.174906in}{1.953932in}}%
\pgfpathlineto{\pgfqpoint{1.174798in}{1.962934in}}%
\pgfpathlineto{\pgfqpoint{1.172881in}{1.971935in}}%
\pgfpathlineto{\pgfqpoint{1.171805in}{1.974161in}}%
\pgfpathlineto{\pgfqpoint{1.168481in}{1.980937in}}%
\pgfpathlineto{\pgfqpoint{1.162074in}{1.988841in}}%
\pgfpathlineto{\pgfqpoint{1.161079in}{1.989938in}}%
\pgfpathlineto{\pgfqpoint{1.152342in}{1.996751in}}%
\pgfpathlineto{\pgfqpoint{1.148784in}{1.998940in}}%
\pgfpathlineto{\pgfqpoint{1.142611in}{2.001909in}}%
\pgfpathlineto{\pgfqpoint{1.132879in}{2.004903in}}%
\pgfpathlineto{\pgfqpoint{1.123148in}{2.005933in}}%
\pgfpathlineto{\pgfqpoint{1.113416in}{2.004903in}}%
\pgfpathlineto{\pgfqpoint{1.103685in}{2.001909in}}%
\pgfpathlineto{\pgfqpoint{1.097512in}{1.998940in}}%
\pgfpathlineto{\pgfqpoint{1.093953in}{1.996751in}}%
\pgfpathlineto{\pgfqpoint{1.085217in}{1.989938in}}%
\pgfpathlineto{\pgfqpoint{1.084222in}{1.988841in}}%
\pgfpathlineto{\pgfqpoint{1.077815in}{1.980937in}}%
\pgfpathlineto{\pgfqpoint{1.074491in}{1.974161in}}%
\pgfpathlineto{\pgfqpoint{1.073415in}{1.971935in}}%
\pgfpathlineto{\pgfqpoint{1.071498in}{1.962934in}}%
\pgfpathlineto{\pgfqpoint{1.071390in}{1.953932in}}%
\pgfpathlineto{\pgfqpoint{1.072302in}{1.944931in}}%
\pgfpathlineto{\pgfqpoint{1.073358in}{1.935929in}}%
\pgfpathlineto{\pgfqpoint{1.073684in}{1.926927in}}%
\pgfpathlineto{\pgfqpoint{1.072490in}{1.917926in}}%
\pgfpathlineto{\pgfqpoint{1.069151in}{1.908924in}}%
\pgfpathlineto{\pgfqpoint{1.064759in}{1.902208in}}%
\pgfpathlineto{\pgfqpoint{1.063144in}{1.899923in}}%
\pgfpathlineto{\pgfqpoint{1.055028in}{1.892040in}}%
\pgfpathlineto{\pgfqpoint{1.053642in}{1.890921in}}%
\pgfpathlineto{\pgfqpoint{1.045296in}{1.885727in}}%
\pgfpathlineto{\pgfqpoint{1.036719in}{1.881920in}}%
\pgfpathlineto{\pgfqpoint{1.035565in}{1.881490in}}%
\pgfpathlineto{\pgfqpoint{1.025833in}{1.879598in}}%
\pgfpathlineto{\pgfqpoint{1.016102in}{1.879528in}}%
\pgfpathlineto{\pgfqpoint{1.006370in}{1.881187in}}%
\pgfpathlineto{\pgfqpoint{1.004081in}{1.881920in}}%
\pgfpathlineto{\pgfqpoint{0.996639in}{1.884761in}}%
\pgfpathlineto{\pgfqpoint{0.986907in}{1.889707in}}%
\pgfpathlineto{\pgfqpoint{0.984772in}{1.890921in}}%
\pgfpathlineto{\pgfqpoint{0.977176in}{1.896528in}}%
\pgfpathlineto{\pgfqpoint{0.972476in}{1.899923in}}%
\pgfpathlineto{\pgfqpoint{0.967444in}{1.905216in}}%
\pgfpathlineto{\pgfqpoint{0.963287in}{1.908924in}}%
\pgfpathlineto{\pgfqpoint{0.957713in}{1.917689in}}%
\pgfpathlineto{\pgfqpoint{0.957491in}{1.917926in}}%
\pgfpathlineto{\pgfqpoint{0.954488in}{1.926927in}}%
\pgfpathlineto{\pgfqpoint{0.955308in}{1.935929in}}%
\pgfpathlineto{\pgfqpoint{0.957713in}{1.944074in}}%
\pgfpathlineto{\pgfqpoint{0.957885in}{1.944931in}}%
\pgfpathlineto{\pgfqpoint{0.959448in}{1.953932in}}%
\pgfpathlineto{\pgfqpoint{0.959263in}{1.962934in}}%
\pgfpathlineto{\pgfqpoint{0.957713in}{1.967180in}}%
\pgfpathlineto{\pgfqpoint{0.955165in}{1.971935in}}%
\pgfpathlineto{\pgfqpoint{0.947981in}{1.977841in}}%
\pgfpathlineto{\pgfqpoint{0.938250in}{1.980805in}}%
\pgfpathlineto{\pgfqpoint{0.928518in}{1.977900in}}%
\pgfpathlineto{\pgfqpoint{0.922111in}{1.971935in}}%
\pgfpathlineto{\pgfqpoint{0.918787in}{1.964907in}}%
\pgfpathlineto{\pgfqpoint{0.918220in}{1.962934in}}%
\pgfpathlineto{\pgfqpoint{0.918075in}{1.953932in}}%
\pgfpathlineto{\pgfqpoint{0.918787in}{1.948716in}}%
\pgfpathlineto{\pgfqpoint{0.919639in}{1.944931in}}%
\pgfpathlineto{\pgfqpoint{0.921985in}{1.935929in}}%
\pgfpathlineto{\pgfqpoint{0.922709in}{1.926927in}}%
\pgfpathlineto{\pgfqpoint{0.920057in}{1.917926in}}%
\pgfpathlineto{\pgfqpoint{0.918787in}{1.916384in}}%
\pgfpathlineto{\pgfqpoint{0.915055in}{1.908924in}}%
\pgfpathlineto{\pgfqpoint{0.909055in}{1.902121in}}%
\pgfpathlineto{\pgfqpoint{0.907606in}{1.899923in}}%
\pgfpathlineto{\pgfqpoint{0.899324in}{1.891298in}}%
\pgfpathlineto{\pgfqpoint{0.899019in}{1.890921in}}%
\pgfpathlineto{\pgfqpoint{0.889592in}{1.881936in}}%
\pgfpathlineto{\pgfqpoint{0.889577in}{1.881920in}}%
\pgfpathlineto{\pgfqpoint{0.879861in}{1.873466in}}%
\pgfpathlineto{\pgfqpoint{0.879239in}{1.872918in}}%
\pgfpathlineto{\pgfqpoint{0.870129in}{1.865724in}}%
\pgfpathlineto{\pgfqpoint{0.867700in}{1.863917in}}%
\pgfpathlineto{\pgfqpoint{0.860398in}{1.858739in}}%
\pgfpathlineto{\pgfqpoint{0.854216in}{1.854915in}}%
\pgfpathlineto{\pgfqpoint{0.850667in}{1.852699in}}%
\pgfpathlineto{\pgfqpoint{0.840935in}{1.847881in}}%
\pgfpathlineto{\pgfqpoint{0.835181in}{1.845914in}}%
\pgfpathlineto{\pgfqpoint{0.831204in}{1.844457in}}%
\pgfpathlineto{\pgfqpoint{0.821472in}{1.842579in}}%
\pgfpathlineto{\pgfqpoint{0.811741in}{1.842298in}}%
\pgfpathlineto{\pgfqpoint{0.802009in}{1.843364in}}%
\pgfpathlineto{\pgfqpoint{0.792278in}{1.845382in}}%
\pgfpathlineto{\pgfqpoint{0.790188in}{1.845914in}}%
\pgfpathlineto{\pgfqpoint{0.782546in}{1.847729in}}%
\pgfpathlineto{\pgfqpoint{0.772815in}{1.849956in}}%
\pgfpathlineto{\pgfqpoint{0.763083in}{1.851587in}}%
\pgfpathlineto{\pgfqpoint{0.753352in}{1.852158in}}%
\pgfpathlineto{\pgfqpoint{0.743620in}{1.851303in}}%
\pgfpathlineto{\pgfqpoint{0.733889in}{1.848788in}}%
\pgfpathlineto{\pgfqpoint{0.727321in}{1.845914in}}%
\pgfpathlineto{\pgfqpoint{0.724157in}{1.844431in}}%
\pgfpathlineto{\pgfqpoint{0.714426in}{1.838091in}}%
\pgfpathlineto{\pgfqpoint{0.712968in}{1.836912in}}%
\pgfpathlineto{\pgfqpoint{0.704694in}{1.829259in}}%
\pgfpathlineto{\pgfqpoint{0.703420in}{1.827910in}}%
\pgfpathlineto{\pgfqpoint{0.696566in}{1.818909in}}%
\pgfpathlineto{\pgfqpoint{0.694963in}{1.815983in}}%
\pgfpathlineto{\pgfqpoint{0.691856in}{1.809907in}}%
\pgfpathlineto{\pgfqpoint{0.689136in}{1.800906in}}%
\pgfpathlineto{\pgfqpoint{0.688212in}{1.791904in}}%
\pgfpathlineto{\pgfqpoint{0.688830in}{1.782903in}}%
\pgfpathlineto{\pgfqpoint{0.690592in}{1.773901in}}%
\pgfpathlineto{\pgfqpoint{0.693000in}{1.764900in}}%
\pgfpathlineto{\pgfqpoint{0.694963in}{1.757831in}}%
\pgfpathlineto{\pgfqpoint{0.695538in}{1.755898in}}%
\pgfpathlineto{\pgfqpoint{0.697719in}{1.746897in}}%
\pgfpathlineto{\pgfqpoint{0.698871in}{1.737895in}}%
\pgfpathlineto{\pgfqpoint{0.698568in}{1.728893in}}%
\pgfpathlineto{\pgfqpoint{0.696537in}{1.719892in}}%
\pgfpathlineto{\pgfqpoint{0.694963in}{1.716213in}}%
\pgfpathlineto{\pgfqpoint{0.692836in}{1.710890in}}%
\pgfpathlineto{\pgfqpoint{0.687627in}{1.701889in}}%
\pgfpathlineto{\pgfqpoint{0.685231in}{1.698606in}}%
\pgfpathlineto{\pgfqpoint{0.681098in}{1.692887in}}%
\pgfpathlineto{\pgfqpoint{0.675500in}{1.686133in}}%
\pgfpathlineto{\pgfqpoint{0.673546in}{1.683886in}}%
\pgfpathlineto{\pgfqpoint{0.665768in}{1.675460in}}%
\pgfpathlineto{\pgfqpoint{0.665176in}{1.674884in}}%
\pgfpathlineto{\pgfqpoint{0.656037in}{1.665897in}}%
\pgfpathlineto{\pgfqpoint{0.656019in}{1.665883in}}%
\pgfpathlineto{\pgfqpoint{0.646305in}{1.657163in}}%
\pgfpathlineto{\pgfqpoint{0.645898in}{1.656881in}}%
\pgfpathlineto{\pgfqpoint{0.636574in}{1.649220in}}%
\pgfpathlineto{\pgfqpoint{0.634198in}{1.647880in}}%
\pgfpathlineto{\pgfqpoint{0.626842in}{1.642330in}}%
\pgfpathlineto{\pgfqpoint{0.618777in}{1.638878in}}%
\pgfpathlineto{\pgfqpoint{0.617111in}{1.637703in}}%
\pgfpathlineto{\pgfqpoint{0.607380in}{1.635250in}}%
\pgfpathlineto{\pgfqpoint{0.597648in}{1.635920in}}%
\pgfpathlineto{\pgfqpoint{0.587917in}{1.638090in}}%
\pgfpathlineto{\pgfqpoint{0.583824in}{1.638878in}}%
\pgfpathlineto{\pgfqpoint{0.578185in}{1.639537in}}%
\pgfpathlineto{\pgfqpoint{0.568454in}{1.639402in}}%
\pgfpathlineto{\pgfqpoint{0.566320in}{1.638878in}}%
\pgfpathlineto{\pgfqpoint{0.558722in}{1.635803in}}%
\pgfpathlineto{\pgfqpoint{0.552273in}{1.629876in}}%
\pgfpathlineto{\pgfqpoint{0.549133in}{1.620875in}}%
\pgfpathlineto{\pgfqpoint{0.552337in}{1.611873in}}%
\pgfpathlineto{\pgfqpoint{0.558722in}{1.605228in}}%
\pgfpathlineto{\pgfqpoint{0.563863in}{1.602872in}}%
\pgfpathlineto{\pgfqpoint{0.568454in}{1.601438in}}%
\pgfpathlineto{\pgfqpoint{0.578185in}{1.601267in}}%
\pgfpathlineto{\pgfqpoint{0.587917in}{1.602712in}}%
\pgfpathlineto{\pgfqpoint{0.588843in}{1.602872in}}%
\pgfpathlineto{\pgfqpoint{0.597648in}{1.605096in}}%
\pgfpathlineto{\pgfqpoint{0.607380in}{1.605854in}}%
\pgfpathlineto{\pgfqpoint{0.617111in}{1.603077in}}%
\pgfpathlineto{\pgfqpoint{0.617367in}{1.602872in}}%
\pgfpathlineto{\pgfqpoint{0.626842in}{1.597716in}}%
\pgfpathlineto{\pgfqpoint{0.630851in}{1.593870in}}%
\pgfpathlineto{\pgfqpoint{0.636574in}{1.589216in}}%
\pgfpathlineto{\pgfqpoint{0.640244in}{1.584869in}}%
\pgfpathlineto{\pgfqpoint{0.646305in}{1.577842in}}%
\pgfpathlineto{\pgfqpoint{0.647618in}{1.575867in}}%
\pgfpathlineto{\pgfqpoint{0.652965in}{1.566866in}}%
\pgfpathlineto{\pgfqpoint{0.656037in}{1.559982in}}%
\pgfpathlineto{\pgfqpoint{0.656829in}{1.557864in}}%
\pgfpathlineto{\pgfqpoint{0.658623in}{1.548863in}}%
\pgfpathlineto{\pgfqpoint{0.658547in}{1.539861in}}%
\pgfpathlineto{\pgfqpoint{0.656501in}{1.530859in}}%
\pgfpathlineto{\pgfqpoint{0.656037in}{1.529792in}}%
\pgfpathlineto{\pgfqpoint{0.651921in}{1.521858in}}%
\pgfpathlineto{\pgfqpoint{0.646305in}{1.514138in}}%
\pgfpathlineto{\pgfqpoint{0.645096in}{1.512856in}}%
\pgfpathlineto{\pgfqpoint{0.636574in}{1.505349in}}%
\pgfpathlineto{\pgfqpoint{0.634104in}{1.503855in}}%
\pgfpathlineto{\pgfqpoint{0.626842in}{1.499792in}}%
\pgfpathlineto{\pgfqpoint{0.617111in}{1.496704in}}%
\pgfpathlineto{\pgfqpoint{0.607380in}{1.495600in}}%
\pgfpathlineto{\pgfqpoint{0.597648in}{1.495901in}}%
\pgfpathlineto{\pgfqpoint{0.587917in}{1.496878in}}%
\pgfpathlineto{\pgfqpoint{0.578185in}{1.497721in}}%
\pgfpathlineto{\pgfqpoint{0.568454in}{1.497621in}}%
\pgfpathlineto{\pgfqpoint{0.558722in}{1.495849in}}%
\pgfpathlineto{\pgfqpoint{0.556316in}{1.494853in}}%
\pgfpathlineto{\pgfqpoint{0.548991in}{1.491778in}}%
\pgfpathlineto{\pgfqpoint{0.540445in}{1.485852in}}%
\pgfpathlineto{\pgfqpoint{0.539259in}{1.484931in}}%
\pgfpathlineto{\pgfqpoint{0.531894in}{1.476850in}}%
\pgfpathlineto{\pgfqpoint{0.529528in}{1.473559in}}%
\pgfpathlineto{\pgfqpoint{0.526318in}{1.467849in}}%
\pgfpathlineto{\pgfqpoint{0.523081in}{1.458847in}}%
\pgfpathlineto{\pgfqpoint{0.521967in}{1.449846in}}%
\pgfpathlineto{\pgfqpoint{0.523081in}{1.440844in}}%
\pgfpathlineto{\pgfqpoint{0.526318in}{1.431842in}}%
\pgfpathlineto{\pgfqpoint{0.529528in}{1.426133in}}%
\pgfpathlineto{\pgfqpoint{0.531894in}{1.422841in}}%
\pgfpathlineto{\pgfqpoint{0.539259in}{1.414760in}}%
\pgfpathlineto{\pgfqpoint{0.540445in}{1.413839in}}%
\pgfpathlineto{\pgfqpoint{0.548991in}{1.407913in}}%
\pgfpathlineto{\pgfqpoint{0.556316in}{1.404838in}}%
\pgfpathlineto{\pgfqpoint{0.558722in}{1.403843in}}%
\pgfpathlineto{\pgfqpoint{0.568454in}{1.402070in}}%
\pgfpathlineto{\pgfqpoint{0.578185in}{1.401970in}}%
\pgfpathlineto{\pgfqpoint{0.587917in}{1.402813in}}%
\pgfpathlineto{\pgfqpoint{0.597648in}{1.403790in}}%
\pgfpathlineto{\pgfqpoint{0.607380in}{1.404091in}}%
\pgfpathlineto{\pgfqpoint{0.617111in}{1.402988in}}%
\pgfpathlineto{\pgfqpoint{0.626842in}{1.399899in}}%
\pgfpathlineto{\pgfqpoint{0.634104in}{1.395836in}}%
\pgfpathlineto{\pgfqpoint{0.636574in}{1.394342in}}%
\pgfpathlineto{\pgfqpoint{0.645096in}{1.386835in}}%
\pgfpathlineto{\pgfqpoint{0.646305in}{1.385553in}}%
\pgfpathlineto{\pgfqpoint{0.651921in}{1.377833in}}%
\pgfpathlineto{\pgfqpoint{0.656037in}{1.369899in}}%
\pgfpathlineto{\pgfqpoint{0.656501in}{1.368832in}}%
\pgfpathlineto{\pgfqpoint{0.658547in}{1.359830in}}%
\pgfpathlineto{\pgfqpoint{0.658623in}{1.350829in}}%
\pgfpathlineto{\pgfqpoint{0.656829in}{1.341827in}}%
\pgfpathlineto{\pgfqpoint{0.656037in}{1.339709in}}%
\pgfpathlineto{\pgfqpoint{0.652965in}{1.332825in}}%
\pgfpathlineto{\pgfqpoint{0.647618in}{1.323824in}}%
\pgfpathlineto{\pgfqpoint{0.646305in}{1.321849in}}%
\pgfpathlineto{\pgfqpoint{0.640244in}{1.314822in}}%
\pgfpathlineto{\pgfqpoint{0.636574in}{1.310475in}}%
\pgfpathlineto{\pgfqpoint{0.630851in}{1.305821in}}%
\pgfpathlineto{\pgfqpoint{0.626842in}{1.301975in}}%
\pgfpathlineto{\pgfqpoint{0.617367in}{1.296819in}}%
\pgfpathlineto{\pgfqpoint{0.617111in}{1.296615in}}%
\pgfpathlineto{\pgfqpoint{0.607380in}{1.293837in}}%
\pgfpathlineto{\pgfqpoint{0.597648in}{1.294595in}}%
\pgfpathlineto{\pgfqpoint{0.588843in}{1.296819in}}%
\pgfpathlineto{\pgfqpoint{0.587917in}{1.296979in}}%
\pgfpathlineto{\pgfqpoint{0.578185in}{1.298424in}}%
\pgfpathlineto{\pgfqpoint{0.568454in}{1.298253in}}%
\pgfpathlineto{\pgfqpoint{0.563863in}{1.296819in}}%
\pgfpathlineto{\pgfqpoint{0.558722in}{1.294463in}}%
\pgfpathlineto{\pgfqpoint{0.552337in}{1.287818in}}%
\pgfpathlineto{\pgfqpoint{0.549133in}{1.278816in}}%
\pgfpathlineto{\pgfqpoint{0.552273in}{1.269815in}}%
\pgfpathlineto{\pgfqpoint{0.558722in}{1.263888in}}%
\pgfpathlineto{\pgfqpoint{0.566320in}{1.260813in}}%
\pgfpathlineto{\pgfqpoint{0.568454in}{1.260289in}}%
\pgfpathlineto{\pgfqpoint{0.578185in}{1.260154in}}%
\pgfpathlineto{\pgfqpoint{0.583824in}{1.260813in}}%
\pgfpathlineto{\pgfqpoint{0.587917in}{1.261601in}}%
\pgfpathlineto{\pgfqpoint{0.597648in}{1.263771in}}%
\pgfpathlineto{\pgfqpoint{0.607380in}{1.264441in}}%
\pgfpathlineto{\pgfqpoint{0.617111in}{1.261988in}}%
\pgfpathlineto{\pgfqpoint{0.618777in}{1.260813in}}%
\pgfpathlineto{\pgfqpoint{0.626842in}{1.257361in}}%
\pgfpathlineto{\pgfqpoint{0.634198in}{1.251812in}}%
\pgfpathlineto{\pgfqpoint{0.636574in}{1.250471in}}%
\pgfpathlineto{\pgfqpoint{0.645898in}{1.242810in}}%
\pgfpathlineto{\pgfqpoint{0.646305in}{1.242528in}}%
\pgfpathlineto{\pgfqpoint{0.656019in}{1.233808in}}%
\pgfpathlineto{\pgfqpoint{0.656037in}{1.233794in}}%
\pgfpathlineto{\pgfqpoint{0.665176in}{1.224807in}}%
\pgfpathlineto{\pgfqpoint{0.665768in}{1.224232in}}%
\pgfpathlineto{\pgfqpoint{0.673546in}{1.215805in}}%
\pgfpathlineto{\pgfqpoint{0.675500in}{1.213559in}}%
\pgfpathlineto{\pgfqpoint{0.681098in}{1.206804in}}%
\pgfpathlineto{\pgfqpoint{0.685231in}{1.201085in}}%
\pgfpathlineto{\pgfqpoint{0.687627in}{1.197802in}}%
\pgfpathlineto{\pgfqpoint{0.692836in}{1.188801in}}%
\pgfpathlineto{\pgfqpoint{0.694963in}{1.183478in}}%
\pgfpathlineto{\pgfqpoint{0.696537in}{1.179799in}}%
\pgfpathlineto{\pgfqpoint{0.698568in}{1.170798in}}%
\pgfpathlineto{\pgfqpoint{0.698871in}{1.161796in}}%
\pgfpathlineto{\pgfqpoint{0.697719in}{1.152795in}}%
\pgfpathlineto{\pgfqpoint{0.695538in}{1.143793in}}%
\pgfpathlineto{\pgfqpoint{0.694963in}{1.141860in}}%
\pgfpathlineto{\pgfqpoint{0.693000in}{1.134792in}}%
\pgfpathlineto{\pgfqpoint{0.690592in}{1.125790in}}%
\pgfpathlineto{\pgfqpoint{0.688830in}{1.116788in}}%
\pgfpathlineto{\pgfqpoint{0.688212in}{1.107787in}}%
\pgfpathlineto{\pgfqpoint{0.689136in}{1.098785in}}%
\pgfpathlineto{\pgfqpoint{0.691856in}{1.089784in}}%
\pgfpathlineto{\pgfqpoint{0.694963in}{1.083708in}}%
\pgfpathlineto{\pgfqpoint{0.696566in}{1.080782in}}%
\pgfpathlineto{\pgfqpoint{0.703420in}{1.071781in}}%
\pgfpathlineto{\pgfqpoint{0.704694in}{1.070432in}}%
\pgfpathlineto{\pgfqpoint{0.712968in}{1.062779in}}%
\pgfpathlineto{\pgfqpoint{0.714426in}{1.061600in}}%
\pgfpathlineto{\pgfqpoint{0.724157in}{1.055260in}}%
\pgfpathlineto{\pgfqpoint{0.727321in}{1.053778in}}%
\pgfpathlineto{\pgfqpoint{0.733889in}{1.050904in}}%
\pgfpathlineto{\pgfqpoint{0.743620in}{1.048388in}}%
\pgfpathlineto{\pgfqpoint{0.753352in}{1.047533in}}%
\pgfpathlineto{\pgfqpoint{0.763083in}{1.048104in}}%
\pgfpathlineto{\pgfqpoint{0.772815in}{1.049735in}}%
\pgfpathlineto{\pgfqpoint{0.782546in}{1.051962in}}%
\pgfpathlineto{\pgfqpoint{0.790188in}{1.053778in}}%
\pgfpathlineto{\pgfqpoint{0.792278in}{1.054309in}}%
\pgfpathlineto{\pgfqpoint{0.802009in}{1.056327in}}%
\pgfpathlineto{\pgfqpoint{0.811741in}{1.057393in}}%
\pgfpathlineto{\pgfqpoint{0.821472in}{1.057112in}}%
\pgfpathlineto{\pgfqpoint{0.831204in}{1.055234in}}%
\pgfpathlineto{\pgfqpoint{0.835181in}{1.053778in}}%
\pgfpathlineto{\pgfqpoint{0.840935in}{1.051811in}}%
\pgfpathlineto{\pgfqpoint{0.850667in}{1.046992in}}%
\pgfpathlineto{\pgfqpoint{0.854216in}{1.044776in}}%
\pgfpathlineto{\pgfqpoint{0.860398in}{1.040952in}}%
\pgfpathlineto{\pgfqpoint{0.867700in}{1.035775in}}%
\pgfpathlineto{\pgfqpoint{0.870129in}{1.033967in}}%
\pgfpathlineto{\pgfqpoint{0.879239in}{1.026773in}}%
\pgfpathlineto{\pgfqpoint{0.879861in}{1.026225in}}%
\pgfpathlineto{\pgfqpoint{0.889577in}{1.017771in}}%
\pgfpathlineto{\pgfqpoint{0.889592in}{1.017755in}}%
\pgfpathlineto{\pgfqpoint{0.899019in}{1.008770in}}%
\pgfpathlineto{\pgfqpoint{0.899324in}{1.008393in}}%
\pgfpathlineto{\pgfqpoint{0.907606in}{0.999768in}}%
\pgfpathlineto{\pgfqpoint{0.909055in}{0.997570in}}%
\pgfpathlineto{\pgfqpoint{0.915055in}{0.990767in}}%
\pgfpathlineto{\pgfqpoint{0.918787in}{0.983307in}}%
\pgfpathlineto{\pgfqpoint{0.920057in}{0.981765in}}%
\pgfpathlineto{\pgfqpoint{0.922709in}{0.972764in}}%
\pgfpathlineto{\pgfqpoint{0.921985in}{0.963762in}}%
\pgfpathlineto{\pgfqpoint{0.919639in}{0.954761in}}%
\pgfpathlineto{\pgfqpoint{0.918787in}{0.950975in}}%
\pgfpathlineto{\pgfqpoint{0.918075in}{0.945759in}}%
\pgfpathlineto{\pgfqpoint{0.918220in}{0.936758in}}%
\pgfpathlineto{\pgfqpoint{0.918787in}{0.934784in}}%
\pgfpathlineto{\pgfqpoint{0.922111in}{0.927756in}}%
\pgfpathlineto{\pgfqpoint{0.928518in}{0.921791in}}%
\pgfpathlineto{\pgfqpoint{0.938250in}{0.918886in}}%
\pgfpathlineto{\pgfqpoint{0.947981in}{0.921850in}}%
\pgfpathlineto{\pgfqpoint{0.955165in}{0.927756in}}%
\pgfpathlineto{\pgfqpoint{0.957713in}{0.932511in}}%
\pgfpathlineto{\pgfqpoint{0.959263in}{0.936758in}}%
\pgfpathlineto{\pgfqpoint{0.959448in}{0.945759in}}%
\pgfpathlineto{\pgfqpoint{0.957885in}{0.954761in}}%
\pgfpathlineto{\pgfqpoint{0.957713in}{0.955617in}}%
\pgfpathlineto{\pgfqpoint{0.955308in}{0.963762in}}%
\pgfpathlineto{\pgfqpoint{0.954488in}{0.972764in}}%
\pgfpathlineto{\pgfqpoint{0.957491in}{0.981765in}}%
\pgfpathlineto{\pgfqpoint{0.957713in}{0.982002in}}%
\pgfpathlineto{\pgfqpoint{0.963287in}{0.990767in}}%
\pgfpathlineto{\pgfqpoint{0.967444in}{0.994475in}}%
\pgfpathlineto{\pgfqpoint{0.972476in}{0.999768in}}%
\pgfpathlineto{\pgfqpoint{0.977176in}{1.003163in}}%
\pgfpathlineto{\pgfqpoint{0.984772in}{1.008770in}}%
\pgfpathlineto{\pgfqpoint{0.986907in}{1.009984in}}%
\pgfpathlineto{\pgfqpoint{0.996639in}{1.014930in}}%
\pgfpathlineto{\pgfqpoint{1.004081in}{1.017771in}}%
\pgfpathlineto{\pgfqpoint{1.006370in}{1.018504in}}%
\pgfpathlineto{\pgfqpoint{1.016102in}{1.020163in}}%
\pgfpathlineto{\pgfqpoint{1.025833in}{1.020093in}}%
\pgfpathlineto{\pgfqpoint{1.035565in}{1.018201in}}%
\pgfpathlineto{\pgfqpoint{1.036719in}{1.017771in}}%
\pgfpathlineto{\pgfqpoint{1.045296in}{1.013964in}}%
\pgfpathlineto{\pgfqpoint{1.053642in}{1.008770in}}%
\pgfpathlineto{\pgfqpoint{1.055028in}{1.007651in}}%
\pgfpathlineto{\pgfqpoint{1.063144in}{0.999768in}}%
\pgfpathlineto{\pgfqpoint{1.064759in}{0.997483in}}%
\pgfpathlineto{\pgfqpoint{1.069151in}{0.990767in}}%
\pgfpathlineto{\pgfqpoint{1.072490in}{0.981765in}}%
\pgfpathlineto{\pgfqpoint{1.073684in}{0.972764in}}%
\pgfpathlineto{\pgfqpoint{1.073358in}{0.963762in}}%
\pgfpathlineto{\pgfqpoint{1.072302in}{0.954761in}}%
\pgfpathlineto{\pgfqpoint{1.071390in}{0.945759in}}%
\pgfpathlineto{\pgfqpoint{1.071498in}{0.936758in}}%
\pgfpathlineto{\pgfqpoint{1.073415in}{0.927756in}}%
\pgfpathlineto{\pgfqpoint{1.074491in}{0.925530in}}%
\pgfpathlineto{\pgfqpoint{1.077815in}{0.918754in}}%
\pgfpathlineto{\pgfqpoint{1.084222in}{0.910850in}}%
\pgfpathlineto{\pgfqpoint{1.085217in}{0.909753in}}%
\pgfpathlineto{\pgfqpoint{1.093953in}{0.902940in}}%
\pgfpathlineto{\pgfqpoint{1.097512in}{0.900751in}}%
\pgfpathclose%
\pgfpathmoveto{\pgfqpoint{1.117857in}{1.035775in}}%
\pgfpathlineto{\pgfqpoint{1.113416in}{1.036168in}}%
\pgfpathlineto{\pgfqpoint{1.103685in}{1.038677in}}%
\pgfpathlineto{\pgfqpoint{1.093953in}{1.042599in}}%
\pgfpathlineto{\pgfqpoint{1.089691in}{1.044776in}}%
\pgfpathlineto{\pgfqpoint{1.084222in}{1.047596in}}%
\pgfpathlineto{\pgfqpoint{1.074491in}{1.053210in}}%
\pgfpathlineto{\pgfqpoint{1.073522in}{1.053778in}}%
\pgfpathlineto{\pgfqpoint{1.064759in}{1.059274in}}%
\pgfpathlineto{\pgfqpoint{1.058717in}{1.062779in}}%
\pgfpathlineto{\pgfqpoint{1.055028in}{1.065228in}}%
\pgfpathlineto{\pgfqpoint{1.045296in}{1.070598in}}%
\pgfpathlineto{\pgfqpoint{1.042291in}{1.071781in}}%
\pgfpathlineto{\pgfqpoint{1.035565in}{1.075066in}}%
\pgfpathlineto{\pgfqpoint{1.025833in}{1.077549in}}%
\pgfpathlineto{\pgfqpoint{1.016102in}{1.077641in}}%
\pgfpathlineto{\pgfqpoint{1.006370in}{1.075464in}}%
\pgfpathlineto{\pgfqpoint{0.997602in}{1.071781in}}%
\pgfpathlineto{\pgfqpoint{0.996639in}{1.071454in}}%
\pgfpathlineto{\pgfqpoint{0.986907in}{1.067070in}}%
\pgfpathlineto{\pgfqpoint{0.978401in}{1.062779in}}%
\pgfpathlineto{\pgfqpoint{0.977176in}{1.062239in}}%
\pgfpathlineto{\pgfqpoint{0.967444in}{1.058040in}}%
\pgfpathlineto{\pgfqpoint{0.957713in}{1.054480in}}%
\pgfpathlineto{\pgfqpoint{0.954896in}{1.053778in}}%
\pgfpathlineto{\pgfqpoint{0.947981in}{1.052168in}}%
\pgfpathlineto{\pgfqpoint{0.938250in}{1.051329in}}%
\pgfpathlineto{\pgfqpoint{0.928518in}{1.052152in}}%
\pgfpathlineto{\pgfqpoint{0.922349in}{1.053778in}}%
\pgfpathlineto{\pgfqpoint{0.918787in}{1.054783in}}%
\pgfpathlineto{\pgfqpoint{0.909055in}{1.059309in}}%
\pgfpathlineto{\pgfqpoint{0.903477in}{1.062779in}}%
\pgfpathlineto{\pgfqpoint{0.899324in}{1.065735in}}%
\pgfpathlineto{\pgfqpoint{0.892170in}{1.071781in}}%
\pgfpathlineto{\pgfqpoint{0.889592in}{1.074484in}}%
\pgfpathlineto{\pgfqpoint{0.884077in}{1.080782in}}%
\pgfpathlineto{\pgfqpoint{0.879861in}{1.087475in}}%
\pgfpathlineto{\pgfqpoint{0.878423in}{1.089784in}}%
\pgfpathlineto{\pgfqpoint{0.875111in}{1.098785in}}%
\pgfpathlineto{\pgfqpoint{0.873987in}{1.107787in}}%
\pgfpathlineto{\pgfqpoint{0.874739in}{1.116788in}}%
\pgfpathlineto{\pgfqpoint{0.876885in}{1.125790in}}%
\pgfpathlineto{\pgfqpoint{0.879817in}{1.134792in}}%
\pgfpathlineto{\pgfqpoint{0.879861in}{1.134922in}}%
\pgfpathlineto{\pgfqpoint{0.882895in}{1.143793in}}%
\pgfpathlineto{\pgfqpoint{0.885404in}{1.152795in}}%
\pgfpathlineto{\pgfqpoint{0.886729in}{1.161796in}}%
\pgfpathlineto{\pgfqpoint{0.886380in}{1.170798in}}%
\pgfpathlineto{\pgfqpoint{0.884045in}{1.179799in}}%
\pgfpathlineto{\pgfqpoint{0.879861in}{1.188300in}}%
\pgfpathlineto{\pgfqpoint{0.879617in}{1.188801in}}%
\pgfpathlineto{\pgfqpoint{0.873274in}{1.197802in}}%
\pgfpathlineto{\pgfqpoint{0.870129in}{1.201342in}}%
\pgfpathlineto{\pgfqpoint{0.864979in}{1.206804in}}%
\pgfpathlineto{\pgfqpoint{0.860398in}{1.211041in}}%
\pgfpathlineto{\pgfqpoint{0.854493in}{1.215805in}}%
\pgfpathlineto{\pgfqpoint{0.850667in}{1.218714in}}%
\pgfpathlineto{\pgfqpoint{0.840935in}{1.224582in}}%
\pgfpathlineto{\pgfqpoint{0.840394in}{1.224807in}}%
\pgfpathlineto{\pgfqpoint{0.831204in}{1.228677in}}%
\pgfpathlineto{\pgfqpoint{0.821472in}{1.230837in}}%
\pgfpathlineto{\pgfqpoint{0.811741in}{1.231160in}}%
\pgfpathlineto{\pgfqpoint{0.802009in}{1.229934in}}%
\pgfpathlineto{\pgfqpoint{0.792278in}{1.227613in}}%
\pgfpathlineto{\pgfqpoint{0.782688in}{1.224807in}}%
\pgfpathlineto{\pgfqpoint{0.782546in}{1.224766in}}%
\pgfpathlineto{\pgfqpoint{0.772815in}{1.222054in}}%
\pgfpathlineto{\pgfqpoint{0.763083in}{1.220069in}}%
\pgfpathlineto{\pgfqpoint{0.753352in}{1.219373in}}%
\pgfpathlineto{\pgfqpoint{0.743620in}{1.220414in}}%
\pgfpathlineto{\pgfqpoint{0.733889in}{1.223477in}}%
\pgfpathlineto{\pgfqpoint{0.731393in}{1.224807in}}%
\pgfpathlineto{\pgfqpoint{0.724157in}{1.228707in}}%
\pgfpathlineto{\pgfqpoint{0.717349in}{1.233808in}}%
\pgfpathlineto{\pgfqpoint{0.714426in}{1.236193in}}%
\pgfpathlineto{\pgfqpoint{0.707890in}{1.242810in}}%
\pgfpathlineto{\pgfqpoint{0.704694in}{1.246652in}}%
\pgfpathlineto{\pgfqpoint{0.700943in}{1.251812in}}%
\pgfpathlineto{\pgfqpoint{0.696050in}{1.260813in}}%
\pgfpathlineto{\pgfqpoint{0.694963in}{1.264108in}}%
\pgfpathlineto{\pgfqpoint{0.693205in}{1.269815in}}%
\pgfpathlineto{\pgfqpoint{0.692316in}{1.278816in}}%
\pgfpathlineto{\pgfqpoint{0.693223in}{1.287818in}}%
\pgfpathlineto{\pgfqpoint{0.694963in}{1.294214in}}%
\pgfpathlineto{\pgfqpoint{0.695722in}{1.296819in}}%
\pgfpathlineto{\pgfqpoint{0.699571in}{1.305821in}}%
\pgfpathlineto{\pgfqpoint{0.704111in}{1.314822in}}%
\pgfpathlineto{\pgfqpoint{0.704694in}{1.315955in}}%
\pgfpathlineto{\pgfqpoint{0.709333in}{1.323824in}}%
\pgfpathlineto{\pgfqpoint{0.714073in}{1.332825in}}%
\pgfpathlineto{\pgfqpoint{0.714426in}{1.333717in}}%
\pgfpathlineto{\pgfqpoint{0.718407in}{1.341827in}}%
\pgfpathlineto{\pgfqpoint{0.720761in}{1.350829in}}%
\pgfpathlineto{\pgfqpoint{0.720662in}{1.359830in}}%
\pgfpathlineto{\pgfqpoint{0.717977in}{1.368832in}}%
\pgfpathlineto{\pgfqpoint{0.714426in}{1.375054in}}%
\pgfpathlineto{\pgfqpoint{0.713147in}{1.377833in}}%
\pgfpathlineto{\pgfqpoint{0.707342in}{1.386835in}}%
\pgfpathlineto{\pgfqpoint{0.704694in}{1.390248in}}%
\pgfpathlineto{\pgfqpoint{0.700905in}{1.395836in}}%
\pgfpathlineto{\pgfqpoint{0.694963in}{1.403942in}}%
\pgfpathlineto{\pgfqpoint{0.694349in}{1.404838in}}%
\pgfpathlineto{\pgfqpoint{0.688280in}{1.413839in}}%
\pgfpathlineto{\pgfqpoint{0.685231in}{1.418898in}}%
\pgfpathlineto{\pgfqpoint{0.682878in}{1.422841in}}%
\pgfpathlineto{\pgfqpoint{0.678638in}{1.431842in}}%
\pgfpathlineto{\pgfqpoint{0.675926in}{1.440844in}}%
\pgfpathlineto{\pgfqpoint{0.675500in}{1.444952in}}%
\pgfpathlineto{\pgfqpoint{0.674968in}{1.449846in}}%
\pgfpathlineto{\pgfqpoint{0.675500in}{1.454740in}}%
\pgfpathlineto{\pgfqpoint{0.675926in}{1.458847in}}%
\pgfpathlineto{\pgfqpoint{0.678638in}{1.467849in}}%
\pgfpathlineto{\pgfqpoint{0.682878in}{1.476850in}}%
\pgfpathlineto{\pgfqpoint{0.685231in}{1.480793in}}%
\pgfpathlineto{\pgfqpoint{0.688280in}{1.485852in}}%
\pgfpathlineto{\pgfqpoint{0.694349in}{1.494853in}}%
\pgfpathlineto{\pgfqpoint{0.694963in}{1.495749in}}%
\pgfpathlineto{\pgfqpoint{0.700905in}{1.503855in}}%
\pgfpathlineto{\pgfqpoint{0.704694in}{1.509444in}}%
\pgfpathlineto{\pgfqpoint{0.707342in}{1.512856in}}%
\pgfpathlineto{\pgfqpoint{0.713147in}{1.521858in}}%
\pgfpathlineto{\pgfqpoint{0.714426in}{1.524638in}}%
\pgfpathlineto{\pgfqpoint{0.717977in}{1.530859in}}%
\pgfpathlineto{\pgfqpoint{0.720662in}{1.539861in}}%
\pgfpathlineto{\pgfqpoint{0.720761in}{1.548863in}}%
\pgfpathlineto{\pgfqpoint{0.718407in}{1.557864in}}%
\pgfpathlineto{\pgfqpoint{0.714426in}{1.565974in}}%
\pgfpathlineto{\pgfqpoint{0.714073in}{1.566866in}}%
\pgfpathlineto{\pgfqpoint{0.709333in}{1.575867in}}%
\pgfpathlineto{\pgfqpoint{0.704694in}{1.583736in}}%
\pgfpathlineto{\pgfqpoint{0.704111in}{1.584869in}}%
\pgfpathlineto{\pgfqpoint{0.699571in}{1.593870in}}%
\pgfpathlineto{\pgfqpoint{0.695722in}{1.602872in}}%
\pgfpathlineto{\pgfqpoint{0.694963in}{1.605478in}}%
\pgfpathlineto{\pgfqpoint{0.693223in}{1.611873in}}%
\pgfpathlineto{\pgfqpoint{0.692316in}{1.620875in}}%
\pgfpathlineto{\pgfqpoint{0.693205in}{1.629876in}}%
\pgfpathlineto{\pgfqpoint{0.694963in}{1.635583in}}%
\pgfpathlineto{\pgfqpoint{0.696050in}{1.638878in}}%
\pgfpathlineto{\pgfqpoint{0.700943in}{1.647880in}}%
\pgfpathlineto{\pgfqpoint{0.704694in}{1.653039in}}%
\pgfpathlineto{\pgfqpoint{0.707890in}{1.656881in}}%
\pgfpathlineto{\pgfqpoint{0.714426in}{1.663498in}}%
\pgfpathlineto{\pgfqpoint{0.717349in}{1.665883in}}%
\pgfpathlineto{\pgfqpoint{0.724157in}{1.670984in}}%
\pgfpathlineto{\pgfqpoint{0.731393in}{1.674884in}}%
\pgfpathlineto{\pgfqpoint{0.733889in}{1.676214in}}%
\pgfpathlineto{\pgfqpoint{0.743620in}{1.679277in}}%
\pgfpathlineto{\pgfqpoint{0.753352in}{1.680318in}}%
\pgfpathlineto{\pgfqpoint{0.763083in}{1.679622in}}%
\pgfpathlineto{\pgfqpoint{0.772815in}{1.677637in}}%
\pgfpathlineto{\pgfqpoint{0.782546in}{1.674925in}}%
\pgfpathlineto{\pgfqpoint{0.782688in}{1.674884in}}%
\pgfpathlineto{\pgfqpoint{0.792278in}{1.672078in}}%
\pgfpathlineto{\pgfqpoint{0.802009in}{1.669757in}}%
\pgfpathlineto{\pgfqpoint{0.811741in}{1.668532in}}%
\pgfpathlineto{\pgfqpoint{0.821472in}{1.668854in}}%
\pgfpathlineto{\pgfqpoint{0.831204in}{1.671014in}}%
\pgfpathlineto{\pgfqpoint{0.840394in}{1.674884in}}%
\pgfpathlineto{\pgfqpoint{0.840935in}{1.675110in}}%
\pgfpathlineto{\pgfqpoint{0.850667in}{1.680977in}}%
\pgfpathlineto{\pgfqpoint{0.854493in}{1.683886in}}%
\pgfpathlineto{\pgfqpoint{0.860398in}{1.688650in}}%
\pgfpathlineto{\pgfqpoint{0.864979in}{1.692887in}}%
\pgfpathlineto{\pgfqpoint{0.870129in}{1.698350in}}%
\pgfpathlineto{\pgfqpoint{0.873274in}{1.701889in}}%
\pgfpathlineto{\pgfqpoint{0.879617in}{1.710890in}}%
\pgfpathlineto{\pgfqpoint{0.879861in}{1.711391in}}%
\pgfpathlineto{\pgfqpoint{0.884045in}{1.719892in}}%
\pgfpathlineto{\pgfqpoint{0.886380in}{1.728893in}}%
\pgfpathlineto{\pgfqpoint{0.886729in}{1.737895in}}%
\pgfpathlineto{\pgfqpoint{0.885404in}{1.746897in}}%
\pgfpathlineto{\pgfqpoint{0.882895in}{1.755898in}}%
\pgfpathlineto{\pgfqpoint{0.879861in}{1.764769in}}%
\pgfpathlineto{\pgfqpoint{0.879817in}{1.764900in}}%
\pgfpathlineto{\pgfqpoint{0.876885in}{1.773901in}}%
\pgfpathlineto{\pgfqpoint{0.874739in}{1.782903in}}%
\pgfpathlineto{\pgfqpoint{0.873987in}{1.791904in}}%
\pgfpathlineto{\pgfqpoint{0.875111in}{1.800906in}}%
\pgfpathlineto{\pgfqpoint{0.878423in}{1.809907in}}%
\pgfpathlineto{\pgfqpoint{0.879861in}{1.812216in}}%
\pgfpathlineto{\pgfqpoint{0.884077in}{1.818909in}}%
\pgfpathlineto{\pgfqpoint{0.889592in}{1.825207in}}%
\pgfpathlineto{\pgfqpoint{0.892170in}{1.827910in}}%
\pgfpathlineto{\pgfqpoint{0.899324in}{1.833956in}}%
\pgfpathlineto{\pgfqpoint{0.903477in}{1.836912in}}%
\pgfpathlineto{\pgfqpoint{0.909055in}{1.840382in}}%
\pgfpathlineto{\pgfqpoint{0.918787in}{1.844908in}}%
\pgfpathlineto{\pgfqpoint{0.922349in}{1.845914in}}%
\pgfpathlineto{\pgfqpoint{0.928518in}{1.847539in}}%
\pgfpathlineto{\pgfqpoint{0.938250in}{1.848362in}}%
\pgfpathlineto{\pgfqpoint{0.947981in}{1.847523in}}%
\pgfpathlineto{\pgfqpoint{0.954896in}{1.845914in}}%
\pgfpathlineto{\pgfqpoint{0.957713in}{1.845211in}}%
\pgfpathlineto{\pgfqpoint{0.967444in}{1.841651in}}%
\pgfpathlineto{\pgfqpoint{0.977176in}{1.837452in}}%
\pgfpathlineto{\pgfqpoint{0.978401in}{1.836912in}}%
\pgfpathlineto{\pgfqpoint{0.986907in}{1.832621in}}%
\pgfpathlineto{\pgfqpoint{0.996639in}{1.828237in}}%
\pgfpathlineto{\pgfqpoint{0.997602in}{1.827910in}}%
\pgfpathlineto{\pgfqpoint{1.006370in}{1.824227in}}%
\pgfpathlineto{\pgfqpoint{1.016102in}{1.822050in}}%
\pgfpathlineto{\pgfqpoint{1.025833in}{1.822142in}}%
\pgfpathlineto{\pgfqpoint{1.035565in}{1.824626in}}%
\pgfpathlineto{\pgfqpoint{1.042291in}{1.827910in}}%
\pgfpathlineto{\pgfqpoint{1.045296in}{1.829093in}}%
\pgfpathlineto{\pgfqpoint{1.055028in}{1.834463in}}%
\pgfpathlineto{\pgfqpoint{1.058717in}{1.836912in}}%
\pgfpathlineto{\pgfqpoint{1.064759in}{1.840417in}}%
\pgfpathlineto{\pgfqpoint{1.073522in}{1.845914in}}%
\pgfpathlineto{\pgfqpoint{1.074491in}{1.846481in}}%
\pgfpathlineto{\pgfqpoint{1.084222in}{1.852095in}}%
\pgfpathlineto{\pgfqpoint{1.089691in}{1.854915in}}%
\pgfpathlineto{\pgfqpoint{1.093953in}{1.857092in}}%
\pgfpathlineto{\pgfqpoint{1.103685in}{1.861014in}}%
\pgfpathlineto{\pgfqpoint{1.113416in}{1.863523in}}%
\pgfpathlineto{\pgfqpoint{1.117857in}{1.863917in}}%
\pgfpathlineto{\pgfqpoint{1.123148in}{1.864409in}}%
\pgfpathlineto{\pgfqpoint{1.128439in}{1.863917in}}%
\pgfpathlineto{\pgfqpoint{1.132879in}{1.863523in}}%
\pgfpathlineto{\pgfqpoint{1.142611in}{1.861014in}}%
\pgfpathlineto{\pgfqpoint{1.152342in}{1.857092in}}%
\pgfpathlineto{\pgfqpoint{1.156605in}{1.854915in}}%
\pgfpathlineto{\pgfqpoint{1.162074in}{1.852095in}}%
\pgfpathlineto{\pgfqpoint{1.171805in}{1.846481in}}%
\pgfpathlineto{\pgfqpoint{1.172774in}{1.845914in}}%
\pgfpathlineto{\pgfqpoint{1.181537in}{1.840417in}}%
\pgfpathlineto{\pgfqpoint{1.187579in}{1.836912in}}%
\pgfpathlineto{\pgfqpoint{1.191268in}{1.834463in}}%
\pgfpathlineto{\pgfqpoint{1.201000in}{1.829093in}}%
\pgfpathlineto{\pgfqpoint{1.204005in}{1.827910in}}%
\pgfpathlineto{\pgfqpoint{1.210731in}{1.824626in}}%
\pgfpathlineto{\pgfqpoint{1.220463in}{1.822142in}}%
\pgfpathlineto{\pgfqpoint{1.230194in}{1.822050in}}%
\pgfpathlineto{\pgfqpoint{1.239926in}{1.824227in}}%
\pgfpathlineto{\pgfqpoint{1.248694in}{1.827910in}}%
\pgfpathlineto{\pgfqpoint{1.249657in}{1.828237in}}%
\pgfpathlineto{\pgfqpoint{1.259389in}{1.832621in}}%
\pgfpathlineto{\pgfqpoint{1.267895in}{1.836912in}}%
\pgfpathlineto{\pgfqpoint{1.269120in}{1.837452in}}%
\pgfpathlineto{\pgfqpoint{1.278852in}{1.841651in}}%
\pgfpathlineto{\pgfqpoint{1.288583in}{1.845211in}}%
\pgfpathlineto{\pgfqpoint{1.291400in}{1.845914in}}%
\pgfpathlineto{\pgfqpoint{1.298315in}{1.847523in}}%
\pgfpathlineto{\pgfqpoint{1.308046in}{1.848362in}}%
\pgfpathlineto{\pgfqpoint{1.317778in}{1.847539in}}%
\pgfpathlineto{\pgfqpoint{1.323946in}{1.845914in}}%
\pgfpathlineto{\pgfqpoint{1.327509in}{1.844908in}}%
\pgfpathlineto{\pgfqpoint{1.337240in}{1.840382in}}%
\pgfpathlineto{\pgfqpoint{1.342818in}{1.836912in}}%
\pgfpathlineto{\pgfqpoint{1.346972in}{1.833956in}}%
\pgfpathlineto{\pgfqpoint{1.354125in}{1.827910in}}%
\pgfpathlineto{\pgfqpoint{1.356703in}{1.825207in}}%
\pgfpathlineto{\pgfqpoint{1.362219in}{1.818909in}}%
\pgfpathlineto{\pgfqpoint{1.366435in}{1.812216in}}%
\pgfpathlineto{\pgfqpoint{1.367873in}{1.809907in}}%
\pgfpathlineto{\pgfqpoint{1.371184in}{1.800906in}}%
\pgfpathlineto{\pgfqpoint{1.372309in}{1.791904in}}%
\pgfpathlineto{\pgfqpoint{1.371557in}{1.782903in}}%
\pgfpathlineto{\pgfqpoint{1.369411in}{1.773901in}}%
\pgfpathlineto{\pgfqpoint{1.366479in}{1.764900in}}%
\pgfpathlineto{\pgfqpoint{1.366435in}{1.764769in}}%
\pgfpathlineto{\pgfqpoint{1.363401in}{1.755898in}}%
\pgfpathlineto{\pgfqpoint{1.360892in}{1.746897in}}%
\pgfpathlineto{\pgfqpoint{1.359567in}{1.737895in}}%
\pgfpathlineto{\pgfqpoint{1.359916in}{1.728893in}}%
\pgfpathlineto{\pgfqpoint{1.362251in}{1.719892in}}%
\pgfpathlineto{\pgfqpoint{1.366435in}{1.711391in}}%
\pgfpathlineto{\pgfqpoint{1.366679in}{1.710890in}}%
\pgfpathlineto{\pgfqpoint{1.373022in}{1.701889in}}%
\pgfpathlineto{\pgfqpoint{1.376166in}{1.698350in}}%
\pgfpathlineto{\pgfqpoint{1.381317in}{1.692887in}}%
\pgfpathlineto{\pgfqpoint{1.385898in}{1.688650in}}%
\pgfpathlineto{\pgfqpoint{1.391803in}{1.683886in}}%
\pgfpathlineto{\pgfqpoint{1.395629in}{1.680977in}}%
\pgfpathlineto{\pgfqpoint{1.405361in}{1.675110in}}%
\pgfpathlineto{\pgfqpoint{1.405902in}{1.674884in}}%
\pgfpathlineto{\pgfqpoint{1.415092in}{1.671014in}}%
\pgfpathlineto{\pgfqpoint{1.424824in}{1.668854in}}%
\pgfpathlineto{\pgfqpoint{1.434555in}{1.668532in}}%
\pgfpathlineto{\pgfqpoint{1.444287in}{1.669757in}}%
\pgfpathlineto{\pgfqpoint{1.454018in}{1.672078in}}%
\pgfpathlineto{\pgfqpoint{1.463608in}{1.674884in}}%
\pgfpathlineto{\pgfqpoint{1.463750in}{1.674925in}}%
\pgfpathlineto{\pgfqpoint{1.473481in}{1.677637in}}%
\pgfpathlineto{\pgfqpoint{1.483213in}{1.679622in}}%
\pgfpathlineto{\pgfqpoint{1.492944in}{1.680318in}}%
\pgfpathlineto{\pgfqpoint{1.502676in}{1.679277in}}%
\pgfpathlineto{\pgfqpoint{1.512407in}{1.676214in}}%
\pgfpathlineto{\pgfqpoint{1.514903in}{1.674884in}}%
\pgfpathlineto{\pgfqpoint{1.522139in}{1.670984in}}%
\pgfpathlineto{\pgfqpoint{1.528947in}{1.665883in}}%
\pgfpathlineto{\pgfqpoint{1.531870in}{1.663498in}}%
\pgfpathlineto{\pgfqpoint{1.538406in}{1.656881in}}%
\pgfpathlineto{\pgfqpoint{1.541602in}{1.653039in}}%
\pgfpathlineto{\pgfqpoint{1.545353in}{1.647880in}}%
\pgfpathlineto{\pgfqpoint{1.550246in}{1.638878in}}%
\pgfpathlineto{\pgfqpoint{1.551333in}{1.635583in}}%
\pgfpathlineto{\pgfqpoint{1.553091in}{1.629876in}}%
\pgfpathlineto{\pgfqpoint{1.553980in}{1.620875in}}%
\pgfpathlineto{\pgfqpoint{1.553073in}{1.611873in}}%
\pgfpathlineto{\pgfqpoint{1.551333in}{1.605478in}}%
\pgfpathlineto{\pgfqpoint{1.550574in}{1.602872in}}%
\pgfpathlineto{\pgfqpoint{1.546725in}{1.593870in}}%
\pgfpathlineto{\pgfqpoint{1.542185in}{1.584869in}}%
\pgfpathlineto{\pgfqpoint{1.541602in}{1.583736in}}%
\pgfpathlineto{\pgfqpoint{1.536963in}{1.575867in}}%
\pgfpathlineto{\pgfqpoint{1.532223in}{1.566866in}}%
\pgfpathlineto{\pgfqpoint{1.531870in}{1.565974in}}%
\pgfpathlineto{\pgfqpoint{1.527888in}{1.557864in}}%
\pgfpathlineto{\pgfqpoint{1.525535in}{1.548863in}}%
\pgfpathlineto{\pgfqpoint{1.525634in}{1.539861in}}%
\pgfpathlineto{\pgfqpoint{1.528319in}{1.530859in}}%
\pgfpathlineto{\pgfqpoint{1.531870in}{1.524638in}}%
\pgfpathlineto{\pgfqpoint{1.533149in}{1.521858in}}%
\pgfpathlineto{\pgfqpoint{1.538954in}{1.512856in}}%
\pgfpathlineto{\pgfqpoint{1.541602in}{1.509444in}}%
\pgfpathlineto{\pgfqpoint{1.545391in}{1.503855in}}%
\pgfpathlineto{\pgfqpoint{1.551333in}{1.495749in}}%
\pgfpathlineto{\pgfqpoint{1.551946in}{1.494853in}}%
\pgfpathlineto{\pgfqpoint{1.558016in}{1.485852in}}%
\pgfpathlineto{\pgfqpoint{1.561064in}{1.480793in}}%
\pgfpathlineto{\pgfqpoint{1.563418in}{1.476850in}}%
\pgfpathlineto{\pgfqpoint{1.567658in}{1.467849in}}%
\pgfpathlineto{\pgfqpoint{1.570370in}{1.458847in}}%
\pgfpathlineto{\pgfqpoint{1.570796in}{1.454740in}}%
\pgfpathlineto{\pgfqpoint{1.571328in}{1.449846in}}%
\pgfpathlineto{\pgfqpoint{1.570796in}{1.444952in}}%
\pgfpathlineto{\pgfqpoint{1.570370in}{1.440844in}}%
\pgfpathlineto{\pgfqpoint{1.567658in}{1.431842in}}%
\pgfpathlineto{\pgfqpoint{1.563418in}{1.422841in}}%
\pgfpathlineto{\pgfqpoint{1.561064in}{1.418898in}}%
\pgfpathlineto{\pgfqpoint{1.558016in}{1.413839in}}%
\pgfpathlineto{\pgfqpoint{1.551946in}{1.404838in}}%
\pgfpathlineto{\pgfqpoint{1.551333in}{1.403942in}}%
\pgfpathlineto{\pgfqpoint{1.545391in}{1.395836in}}%
\pgfpathlineto{\pgfqpoint{1.541602in}{1.390248in}}%
\pgfpathlineto{\pgfqpoint{1.538954in}{1.386835in}}%
\pgfpathlineto{\pgfqpoint{1.533149in}{1.377833in}}%
\pgfpathlineto{\pgfqpoint{1.531870in}{1.375054in}}%
\pgfpathlineto{\pgfqpoint{1.528319in}{1.368832in}}%
\pgfpathlineto{\pgfqpoint{1.525634in}{1.359830in}}%
\pgfpathlineto{\pgfqpoint{1.525535in}{1.350829in}}%
\pgfpathlineto{\pgfqpoint{1.527888in}{1.341827in}}%
\pgfpathlineto{\pgfqpoint{1.531870in}{1.333717in}}%
\pgfpathlineto{\pgfqpoint{1.532223in}{1.332825in}}%
\pgfpathlineto{\pgfqpoint{1.536963in}{1.323824in}}%
\pgfpathlineto{\pgfqpoint{1.541602in}{1.315955in}}%
\pgfpathlineto{\pgfqpoint{1.542185in}{1.314822in}}%
\pgfpathlineto{\pgfqpoint{1.546725in}{1.305821in}}%
\pgfpathlineto{\pgfqpoint{1.550574in}{1.296819in}}%
\pgfpathlineto{\pgfqpoint{1.551333in}{1.294214in}}%
\pgfpathlineto{\pgfqpoint{1.553073in}{1.287818in}}%
\pgfpathlineto{\pgfqpoint{1.553980in}{1.278816in}}%
\pgfpathlineto{\pgfqpoint{1.553091in}{1.269815in}}%
\pgfpathlineto{\pgfqpoint{1.551333in}{1.264108in}}%
\pgfpathlineto{\pgfqpoint{1.550246in}{1.260813in}}%
\pgfpathlineto{\pgfqpoint{1.545353in}{1.251812in}}%
\pgfpathlineto{\pgfqpoint{1.541602in}{1.246652in}}%
\pgfpathlineto{\pgfqpoint{1.538406in}{1.242810in}}%
\pgfpathlineto{\pgfqpoint{1.531870in}{1.236193in}}%
\pgfpathlineto{\pgfqpoint{1.528947in}{1.233808in}}%
\pgfpathlineto{\pgfqpoint{1.522139in}{1.228707in}}%
\pgfpathlineto{\pgfqpoint{1.514903in}{1.224807in}}%
\pgfpathlineto{\pgfqpoint{1.512407in}{1.223477in}}%
\pgfpathlineto{\pgfqpoint{1.502676in}{1.220414in}}%
\pgfpathlineto{\pgfqpoint{1.492944in}{1.219373in}}%
\pgfpathlineto{\pgfqpoint{1.483213in}{1.220069in}}%
\pgfpathlineto{\pgfqpoint{1.473481in}{1.222054in}}%
\pgfpathlineto{\pgfqpoint{1.463750in}{1.224766in}}%
\pgfpathlineto{\pgfqpoint{1.463608in}{1.224807in}}%
\pgfpathlineto{\pgfqpoint{1.454018in}{1.227613in}}%
\pgfpathlineto{\pgfqpoint{1.444287in}{1.229934in}}%
\pgfpathlineto{\pgfqpoint{1.434555in}{1.231160in}}%
\pgfpathlineto{\pgfqpoint{1.424824in}{1.230837in}}%
\pgfpathlineto{\pgfqpoint{1.415092in}{1.228677in}}%
\pgfpathlineto{\pgfqpoint{1.405902in}{1.224807in}}%
\pgfpathlineto{\pgfqpoint{1.405361in}{1.224582in}}%
\pgfpathlineto{\pgfqpoint{1.395629in}{1.218714in}}%
\pgfpathlineto{\pgfqpoint{1.391803in}{1.215805in}}%
\pgfpathlineto{\pgfqpoint{1.385898in}{1.211041in}}%
\pgfpathlineto{\pgfqpoint{1.381317in}{1.206804in}}%
\pgfpathlineto{\pgfqpoint{1.376166in}{1.201342in}}%
\pgfpathlineto{\pgfqpoint{1.373022in}{1.197802in}}%
\pgfpathlineto{\pgfqpoint{1.366679in}{1.188801in}}%
\pgfpathlineto{\pgfqpoint{1.366435in}{1.188300in}}%
\pgfpathlineto{\pgfqpoint{1.362251in}{1.179799in}}%
\pgfpathlineto{\pgfqpoint{1.359916in}{1.170798in}}%
\pgfpathlineto{\pgfqpoint{1.359567in}{1.161796in}}%
\pgfpathlineto{\pgfqpoint{1.360892in}{1.152795in}}%
\pgfpathlineto{\pgfqpoint{1.363401in}{1.143793in}}%
\pgfpathlineto{\pgfqpoint{1.366435in}{1.134922in}}%
\pgfpathlineto{\pgfqpoint{1.366479in}{1.134792in}}%
\pgfpathlineto{\pgfqpoint{1.369411in}{1.125790in}}%
\pgfpathlineto{\pgfqpoint{1.371557in}{1.116788in}}%
\pgfpathlineto{\pgfqpoint{1.372309in}{1.107787in}}%
\pgfpathlineto{\pgfqpoint{1.371184in}{1.098785in}}%
\pgfpathlineto{\pgfqpoint{1.367873in}{1.089784in}}%
\pgfpathlineto{\pgfqpoint{1.366435in}{1.087475in}}%
\pgfpathlineto{\pgfqpoint{1.362219in}{1.080782in}}%
\pgfpathlineto{\pgfqpoint{1.356703in}{1.074484in}}%
\pgfpathlineto{\pgfqpoint{1.354125in}{1.071781in}}%
\pgfpathlineto{\pgfqpoint{1.346972in}{1.065735in}}%
\pgfpathlineto{\pgfqpoint{1.342818in}{1.062779in}}%
\pgfpathlineto{\pgfqpoint{1.337240in}{1.059309in}}%
\pgfpathlineto{\pgfqpoint{1.327509in}{1.054783in}}%
\pgfpathlineto{\pgfqpoint{1.323946in}{1.053778in}}%
\pgfpathlineto{\pgfqpoint{1.317778in}{1.052152in}}%
\pgfpathlineto{\pgfqpoint{1.308046in}{1.051329in}}%
\pgfpathlineto{\pgfqpoint{1.298315in}{1.052168in}}%
\pgfpathlineto{\pgfqpoint{1.291400in}{1.053778in}}%
\pgfpathlineto{\pgfqpoint{1.288583in}{1.054480in}}%
\pgfpathlineto{\pgfqpoint{1.278852in}{1.058040in}}%
\pgfpathlineto{\pgfqpoint{1.269120in}{1.062239in}}%
\pgfpathlineto{\pgfqpoint{1.267895in}{1.062779in}}%
\pgfpathlineto{\pgfqpoint{1.259389in}{1.067070in}}%
\pgfpathlineto{\pgfqpoint{1.249657in}{1.071454in}}%
\pgfpathlineto{\pgfqpoint{1.248694in}{1.071781in}}%
\pgfpathlineto{\pgfqpoint{1.239926in}{1.075464in}}%
\pgfpathlineto{\pgfqpoint{1.230194in}{1.077641in}}%
\pgfpathlineto{\pgfqpoint{1.220463in}{1.077549in}}%
\pgfpathlineto{\pgfqpoint{1.210731in}{1.075066in}}%
\pgfpathlineto{\pgfqpoint{1.204005in}{1.071781in}}%
\pgfpathlineto{\pgfqpoint{1.201000in}{1.070598in}}%
\pgfpathlineto{\pgfqpoint{1.191268in}{1.065228in}}%
\pgfpathlineto{\pgfqpoint{1.187579in}{1.062779in}}%
\pgfpathlineto{\pgfqpoint{1.181537in}{1.059274in}}%
\pgfpathlineto{\pgfqpoint{1.172774in}{1.053778in}}%
\pgfpathlineto{\pgfqpoint{1.171805in}{1.053210in}}%
\pgfpathlineto{\pgfqpoint{1.162074in}{1.047596in}}%
\pgfpathlineto{\pgfqpoint{1.156605in}{1.044776in}}%
\pgfpathlineto{\pgfqpoint{1.152342in}{1.042599in}}%
\pgfpathlineto{\pgfqpoint{1.142611in}{1.038677in}}%
\pgfpathlineto{\pgfqpoint{1.132879in}{1.036168in}}%
\pgfpathlineto{\pgfqpoint{1.128439in}{1.035775in}}%
\pgfpathlineto{\pgfqpoint{1.123148in}{1.035282in}}%
\pgfpathclose%
\pgfusepath{fill}%
\end{pgfscope}%
\begin{pgfscope}%
\pgfpathrectangle{\pgfqpoint{0.150000in}{0.549691in}}{\pgfqpoint{1.946296in}{1.800309in}}%
\pgfusepath{clip}%
\pgfsetbuttcap%
\pgfsetroundjoin%
\definecolor{currentfill}{rgb}{0.219608,0.219608,0.219608}%
\pgfsetfillcolor{currentfill}%
\pgfsetlinewidth{0.000000pt}%
\definecolor{currentstroke}{rgb}{0.000000,0.000000,0.000000}%
\pgfsetstrokecolor{currentstroke}%
\pgfsetdash{}{0pt}%
\pgfpathmoveto{\pgfqpoint{1.113416in}{0.826284in}}%
\pgfpathlineto{\pgfqpoint{1.123148in}{0.825116in}}%
\pgfpathlineto{\pgfqpoint{1.132879in}{0.826284in}}%
\pgfpathlineto{\pgfqpoint{1.139915in}{0.828739in}}%
\pgfpathlineto{\pgfqpoint{1.142611in}{0.829509in}}%
\pgfpathlineto{\pgfqpoint{1.152342in}{0.833851in}}%
\pgfpathlineto{\pgfqpoint{1.159220in}{0.837741in}}%
\pgfpathlineto{\pgfqpoint{1.162074in}{0.839140in}}%
\pgfpathlineto{\pgfqpoint{1.171805in}{0.844481in}}%
\pgfpathlineto{\pgfqpoint{1.175863in}{0.846742in}}%
\pgfpathlineto{\pgfqpoint{1.181537in}{0.849631in}}%
\pgfpathlineto{\pgfqpoint{1.191268in}{0.854215in}}%
\pgfpathlineto{\pgfqpoint{1.195173in}{0.855744in}}%
\pgfpathlineto{\pgfqpoint{1.201000in}{0.857935in}}%
\pgfpathlineto{\pgfqpoint{1.210731in}{0.860544in}}%
\pgfpathlineto{\pgfqpoint{1.220463in}{0.861908in}}%
\pgfpathlineto{\pgfqpoint{1.230194in}{0.861958in}}%
\pgfpathlineto{\pgfqpoint{1.239926in}{0.860763in}}%
\pgfpathlineto{\pgfqpoint{1.249657in}{0.858518in}}%
\pgfpathlineto{\pgfqpoint{1.258694in}{0.855744in}}%
\pgfpathlineto{\pgfqpoint{1.259389in}{0.855522in}}%
\pgfpathlineto{\pgfqpoint{1.269120in}{0.852039in}}%
\pgfpathlineto{\pgfqpoint{1.278852in}{0.848629in}}%
\pgfpathlineto{\pgfqpoint{1.285206in}{0.846742in}}%
\pgfpathlineto{\pgfqpoint{1.288583in}{0.845645in}}%
\pgfpathlineto{\pgfqpoint{1.298315in}{0.843490in}}%
\pgfpathlineto{\pgfqpoint{1.308046in}{0.842692in}}%
\pgfpathlineto{\pgfqpoint{1.317778in}{0.843474in}}%
\pgfpathlineto{\pgfqpoint{1.327509in}{0.845914in}}%
\pgfpathlineto{\pgfqpoint{1.329513in}{0.846742in}}%
\pgfpathlineto{\pgfqpoint{1.337240in}{0.849660in}}%
\pgfpathlineto{\pgfqpoint{1.346972in}{0.854575in}}%
\pgfpathlineto{\pgfqpoint{1.348921in}{0.855744in}}%
\pgfpathlineto{\pgfqpoint{1.356703in}{0.860225in}}%
\pgfpathlineto{\pgfqpoint{1.363912in}{0.864745in}}%
\pgfpathlineto{\pgfqpoint{1.366435in}{0.866340in}}%
\pgfpathlineto{\pgfqpoint{1.376166in}{0.872562in}}%
\pgfpathlineto{\pgfqpoint{1.378133in}{0.873747in}}%
\pgfpathlineto{\pgfqpoint{1.385898in}{0.878701in}}%
\pgfpathlineto{\pgfqpoint{1.393171in}{0.882748in}}%
\pgfpathlineto{\pgfqpoint{1.395629in}{0.884277in}}%
\pgfpathlineto{\pgfqpoint{1.405361in}{0.889076in}}%
\pgfpathlineto{\pgfqpoint{1.413216in}{0.891750in}}%
\pgfpathlineto{\pgfqpoint{1.415092in}{0.892508in}}%
\pgfpathlineto{\pgfqpoint{1.424824in}{0.894582in}}%
\pgfpathlineto{\pgfqpoint{1.434555in}{0.894891in}}%
\pgfpathlineto{\pgfqpoint{1.444287in}{0.893715in}}%
\pgfpathlineto{\pgfqpoint{1.452872in}{0.891750in}}%
\pgfpathlineto{\pgfqpoint{1.454018in}{0.891529in}}%
\pgfpathlineto{\pgfqpoint{1.463750in}{0.889226in}}%
\pgfpathlineto{\pgfqpoint{1.473481in}{0.887008in}}%
\pgfpathlineto{\pgfqpoint{1.483213in}{0.885385in}}%
\pgfpathlineto{\pgfqpoint{1.492944in}{0.884816in}}%
\pgfpathlineto{\pgfqpoint{1.502676in}{0.885667in}}%
\pgfpathlineto{\pgfqpoint{1.512407in}{0.888172in}}%
\pgfpathlineto{\pgfqpoint{1.520617in}{0.891750in}}%
\pgfpathlineto{\pgfqpoint{1.522139in}{0.892537in}}%
\pgfpathlineto{\pgfqpoint{1.531870in}{0.899536in}}%
\pgfpathlineto{\pgfqpoint{1.533232in}{0.900751in}}%
\pgfpathlineto{\pgfqpoint{1.541117in}{0.909753in}}%
\pgfpathlineto{\pgfqpoint{1.541602in}{0.910535in}}%
\pgfpathlineto{\pgfqpoint{1.546052in}{0.918754in}}%
\pgfpathlineto{\pgfqpoint{1.549002in}{0.927756in}}%
\pgfpathlineto{\pgfqpoint{1.550302in}{0.936758in}}%
\pgfpathlineto{\pgfqpoint{1.550375in}{0.945759in}}%
\pgfpathlineto{\pgfqpoint{1.549757in}{0.954761in}}%
\pgfpathlineto{\pgfqpoint{1.549041in}{0.963762in}}%
\pgfpathlineto{\pgfqpoint{1.548820in}{0.972764in}}%
\pgfpathlineto{\pgfqpoint{1.549629in}{0.981765in}}%
\pgfpathlineto{\pgfqpoint{1.551333in}{0.988540in}}%
\pgfpathlineto{\pgfqpoint{1.551856in}{0.990767in}}%
\pgfpathlineto{\pgfqpoint{1.555583in}{0.999768in}}%
\pgfpathlineto{\pgfqpoint{1.561010in}{1.008770in}}%
\pgfpathlineto{\pgfqpoint{1.561064in}{1.008839in}}%
\pgfpathlineto{\pgfqpoint{1.567986in}{1.017771in}}%
\pgfpathlineto{\pgfqpoint{1.570796in}{1.020811in}}%
\pgfpathlineto{\pgfqpoint{1.576579in}{1.026773in}}%
\pgfpathlineto{\pgfqpoint{1.580527in}{1.030426in}}%
\pgfpathlineto{\pgfqpoint{1.586972in}{1.035775in}}%
\pgfpathlineto{\pgfqpoint{1.590259in}{1.038374in}}%
\pgfpathlineto{\pgfqpoint{1.599916in}{1.044776in}}%
\pgfpathlineto{\pgfqpoint{1.599990in}{1.044826in}}%
\pgfpathlineto{\pgfqpoint{1.609722in}{1.049846in}}%
\pgfpathlineto{\pgfqpoint{1.619453in}{1.053294in}}%
\pgfpathlineto{\pgfqpoint{1.621861in}{1.053778in}}%
\pgfpathlineto{\pgfqpoint{1.629185in}{1.055354in}}%
\pgfpathlineto{\pgfqpoint{1.638916in}{1.056102in}}%
\pgfpathlineto{\pgfqpoint{1.648648in}{1.055898in}}%
\pgfpathlineto{\pgfqpoint{1.658379in}{1.055235in}}%
\pgfpathlineto{\pgfqpoint{1.668111in}{1.054664in}}%
\pgfpathlineto{\pgfqpoint{1.677842in}{1.054732in}}%
\pgfpathlineto{\pgfqpoint{1.687574in}{1.055934in}}%
\pgfpathlineto{\pgfqpoint{1.697305in}{1.058663in}}%
\pgfpathlineto{\pgfqpoint{1.706191in}{1.062779in}}%
\pgfpathlineto{\pgfqpoint{1.707037in}{1.063228in}}%
\pgfpathlineto{\pgfqpoint{1.716768in}{1.070521in}}%
\pgfpathlineto{\pgfqpoint{1.718082in}{1.071781in}}%
\pgfpathlineto{\pgfqpoint{1.725648in}{1.080782in}}%
\pgfpathlineto{\pgfqpoint{1.726500in}{1.082190in}}%
\pgfpathlineto{\pgfqpoint{1.730367in}{1.089784in}}%
\pgfpathlineto{\pgfqpoint{1.733076in}{1.098785in}}%
\pgfpathlineto{\pgfqpoint{1.733996in}{1.107787in}}%
\pgfpathlineto{\pgfqpoint{1.733381in}{1.116788in}}%
\pgfpathlineto{\pgfqpoint{1.731626in}{1.125790in}}%
\pgfpathlineto{\pgfqpoint{1.729228in}{1.134792in}}%
\pgfpathlineto{\pgfqpoint{1.726739in}{1.143793in}}%
\pgfpathlineto{\pgfqpoint{1.726500in}{1.144853in}}%
\pgfpathlineto{\pgfqpoint{1.724375in}{1.152795in}}%
\pgfpathlineto{\pgfqpoint{1.723103in}{1.161796in}}%
\pgfpathlineto{\pgfqpoint{1.723438in}{1.170798in}}%
\pgfpathlineto{\pgfqpoint{1.725680in}{1.179799in}}%
\pgfpathlineto{\pgfqpoint{1.726500in}{1.181535in}}%
\pgfpathlineto{\pgfqpoint{1.729391in}{1.188801in}}%
\pgfpathlineto{\pgfqpoint{1.734579in}{1.197802in}}%
\pgfpathlineto{\pgfqpoint{1.736231in}{1.200076in}}%
\pgfpathlineto{\pgfqpoint{1.740607in}{1.206804in}}%
\pgfpathlineto{\pgfqpoint{1.745963in}{1.213986in}}%
\pgfpathlineto{\pgfqpoint{1.747243in}{1.215805in}}%
\pgfpathlineto{\pgfqpoint{1.753970in}{1.224807in}}%
\pgfpathlineto{\pgfqpoint{1.755694in}{1.227141in}}%
\pgfpathlineto{\pgfqpoint{1.760581in}{1.233808in}}%
\pgfpathlineto{\pgfqpoint{1.765426in}{1.241007in}}%
\pgfpathlineto{\pgfqpoint{1.766689in}{1.242810in}}%
\pgfpathlineto{\pgfqpoint{1.772002in}{1.251812in}}%
\pgfpathlineto{\pgfqpoint{1.775157in}{1.258960in}}%
\pgfpathlineto{\pgfqpoint{1.776052in}{1.260813in}}%
\pgfpathlineto{\pgfqpoint{1.778690in}{1.269815in}}%
\pgfpathlineto{\pgfqpoint{1.779535in}{1.278816in}}%
\pgfpathlineto{\pgfqpoint{1.778672in}{1.287818in}}%
\pgfpathlineto{\pgfqpoint{1.776343in}{1.296819in}}%
\pgfpathlineto{\pgfqpoint{1.775157in}{1.299943in}}%
\pgfpathlineto{\pgfqpoint{1.773117in}{1.305821in}}%
\pgfpathlineto{\pgfqpoint{1.769431in}{1.314822in}}%
\pgfpathlineto{\pgfqpoint{1.765666in}{1.323824in}}%
\pgfpathlineto{\pgfqpoint{1.765426in}{1.324466in}}%
\pgfpathlineto{\pgfqpoint{1.762426in}{1.332825in}}%
\pgfpathlineto{\pgfqpoint{1.759999in}{1.341827in}}%
\pgfpathlineto{\pgfqpoint{1.758707in}{1.350829in}}%
\pgfpathlineto{\pgfqpoint{1.758761in}{1.359830in}}%
\pgfpathlineto{\pgfqpoint{1.760236in}{1.368832in}}%
\pgfpathlineto{\pgfqpoint{1.763057in}{1.377833in}}%
\pgfpathlineto{\pgfqpoint{1.765426in}{1.383223in}}%
\pgfpathlineto{\pgfqpoint{1.767078in}{1.386835in}}%
\pgfpathlineto{\pgfqpoint{1.772034in}{1.395836in}}%
\pgfpathlineto{\pgfqpoint{1.775157in}{1.401084in}}%
\pgfpathlineto{\pgfqpoint{1.777601in}{1.404838in}}%
\pgfpathlineto{\pgfqpoint{1.783375in}{1.413839in}}%
\pgfpathlineto{\pgfqpoint{1.784889in}{1.416479in}}%
\pgfpathlineto{\pgfqpoint{1.789094in}{1.422841in}}%
\pgfpathlineto{\pgfqpoint{1.793788in}{1.431842in}}%
\pgfpathlineto{\pgfqpoint{1.794620in}{1.434336in}}%
\pgfpathlineto{\pgfqpoint{1.797274in}{1.440844in}}%
\pgfpathlineto{\pgfqpoint{1.798536in}{1.449846in}}%
\pgfpathlineto{\pgfqpoint{1.797274in}{1.458847in}}%
\pgfpathlineto{\pgfqpoint{1.794620in}{1.465355in}}%
\pgfpathlineto{\pgfqpoint{1.793788in}{1.467849in}}%
\pgfpathlineto{\pgfqpoint{1.789094in}{1.476850in}}%
\pgfpathlineto{\pgfqpoint{1.784889in}{1.483212in}}%
\pgfpathlineto{\pgfqpoint{1.783375in}{1.485852in}}%
\pgfpathlineto{\pgfqpoint{1.777601in}{1.494853in}}%
\pgfpathlineto{\pgfqpoint{1.775157in}{1.498607in}}%
\pgfpathlineto{\pgfqpoint{1.772034in}{1.503855in}}%
\pgfpathlineto{\pgfqpoint{1.767078in}{1.512856in}}%
\pgfpathlineto{\pgfqpoint{1.765426in}{1.516468in}}%
\pgfpathlineto{\pgfqpoint{1.763057in}{1.521858in}}%
\pgfpathlineto{\pgfqpoint{1.760236in}{1.530859in}}%
\pgfpathlineto{\pgfqpoint{1.758761in}{1.539861in}}%
\pgfpathlineto{\pgfqpoint{1.758707in}{1.548863in}}%
\pgfpathlineto{\pgfqpoint{1.759999in}{1.557864in}}%
\pgfpathlineto{\pgfqpoint{1.762426in}{1.566866in}}%
\pgfpathlineto{\pgfqpoint{1.765426in}{1.575225in}}%
\pgfpathlineto{\pgfqpoint{1.765666in}{1.575867in}}%
\pgfpathlineto{\pgfqpoint{1.769431in}{1.584869in}}%
\pgfpathlineto{\pgfqpoint{1.773117in}{1.593870in}}%
\pgfpathlineto{\pgfqpoint{1.775157in}{1.599748in}}%
\pgfpathlineto{\pgfqpoint{1.776343in}{1.602872in}}%
\pgfpathlineto{\pgfqpoint{1.778672in}{1.611873in}}%
\pgfpathlineto{\pgfqpoint{1.779535in}{1.620875in}}%
\pgfpathlineto{\pgfqpoint{1.778690in}{1.629876in}}%
\pgfpathlineto{\pgfqpoint{1.776052in}{1.638878in}}%
\pgfpathlineto{\pgfqpoint{1.775157in}{1.640731in}}%
\pgfpathlineto{\pgfqpoint{1.772002in}{1.647880in}}%
\pgfpathlineto{\pgfqpoint{1.766689in}{1.656881in}}%
\pgfpathlineto{\pgfqpoint{1.765426in}{1.658684in}}%
\pgfpathlineto{\pgfqpoint{1.760581in}{1.665883in}}%
\pgfpathlineto{\pgfqpoint{1.755694in}{1.672550in}}%
\pgfpathlineto{\pgfqpoint{1.753970in}{1.674884in}}%
\pgfpathlineto{\pgfqpoint{1.747243in}{1.683886in}}%
\pgfpathlineto{\pgfqpoint{1.745963in}{1.685705in}}%
\pgfpathlineto{\pgfqpoint{1.740607in}{1.692887in}}%
\pgfpathlineto{\pgfqpoint{1.736231in}{1.699615in}}%
\pgfpathlineto{\pgfqpoint{1.734579in}{1.701889in}}%
\pgfpathlineto{\pgfqpoint{1.729391in}{1.710890in}}%
\pgfpathlineto{\pgfqpoint{1.726500in}{1.718156in}}%
\pgfpathlineto{\pgfqpoint{1.725680in}{1.719892in}}%
\pgfpathlineto{\pgfqpoint{1.723438in}{1.728893in}}%
\pgfpathlineto{\pgfqpoint{1.723103in}{1.737895in}}%
\pgfpathlineto{\pgfqpoint{1.724375in}{1.746897in}}%
\pgfpathlineto{\pgfqpoint{1.726500in}{1.754838in}}%
\pgfpathlineto{\pgfqpoint{1.726739in}{1.755898in}}%
\pgfpathlineto{\pgfqpoint{1.729228in}{1.764900in}}%
\pgfpathlineto{\pgfqpoint{1.731626in}{1.773901in}}%
\pgfpathlineto{\pgfqpoint{1.733381in}{1.782903in}}%
\pgfpathlineto{\pgfqpoint{1.733996in}{1.791904in}}%
\pgfpathlineto{\pgfqpoint{1.733076in}{1.800906in}}%
\pgfpathlineto{\pgfqpoint{1.730367in}{1.809907in}}%
\pgfpathlineto{\pgfqpoint{1.726500in}{1.817501in}}%
\pgfpathlineto{\pgfqpoint{1.725648in}{1.818909in}}%
\pgfpathlineto{\pgfqpoint{1.718082in}{1.827910in}}%
\pgfpathlineto{\pgfqpoint{1.716768in}{1.829170in}}%
\pgfpathlineto{\pgfqpoint{1.707037in}{1.836464in}}%
\pgfpathlineto{\pgfqpoint{1.706191in}{1.836912in}}%
\pgfpathlineto{\pgfqpoint{1.697305in}{1.841028in}}%
\pgfpathlineto{\pgfqpoint{1.687574in}{1.843758in}}%
\pgfpathlineto{\pgfqpoint{1.677842in}{1.844960in}}%
\pgfpathlineto{\pgfqpoint{1.668111in}{1.845027in}}%
\pgfpathlineto{\pgfqpoint{1.658379in}{1.844456in}}%
\pgfpathlineto{\pgfqpoint{1.648648in}{1.843793in}}%
\pgfpathlineto{\pgfqpoint{1.638916in}{1.843589in}}%
\pgfpathlineto{\pgfqpoint{1.629185in}{1.844337in}}%
\pgfpathlineto{\pgfqpoint{1.621861in}{1.845914in}}%
\pgfpathlineto{\pgfqpoint{1.619453in}{1.846397in}}%
\pgfpathlineto{\pgfqpoint{1.609722in}{1.849845in}}%
\pgfpathlineto{\pgfqpoint{1.599990in}{1.854865in}}%
\pgfpathlineto{\pgfqpoint{1.599916in}{1.854915in}}%
\pgfpathlineto{\pgfqpoint{1.590259in}{1.861317in}}%
\pgfpathlineto{\pgfqpoint{1.586972in}{1.863917in}}%
\pgfpathlineto{\pgfqpoint{1.580527in}{1.869265in}}%
\pgfpathlineto{\pgfqpoint{1.576579in}{1.872918in}}%
\pgfpathlineto{\pgfqpoint{1.570796in}{1.878880in}}%
\pgfpathlineto{\pgfqpoint{1.567986in}{1.881920in}}%
\pgfpathlineto{\pgfqpoint{1.561064in}{1.890852in}}%
\pgfpathlineto{\pgfqpoint{1.561010in}{1.890921in}}%
\pgfpathlineto{\pgfqpoint{1.555583in}{1.899923in}}%
\pgfpathlineto{\pgfqpoint{1.551856in}{1.908924in}}%
\pgfpathlineto{\pgfqpoint{1.551333in}{1.911152in}}%
\pgfpathlineto{\pgfqpoint{1.549629in}{1.917926in}}%
\pgfpathlineto{\pgfqpoint{1.548820in}{1.926927in}}%
\pgfpathlineto{\pgfqpoint{1.549041in}{1.935929in}}%
\pgfpathlineto{\pgfqpoint{1.549757in}{1.944931in}}%
\pgfpathlineto{\pgfqpoint{1.550375in}{1.953932in}}%
\pgfpathlineto{\pgfqpoint{1.550302in}{1.962934in}}%
\pgfpathlineto{\pgfqpoint{1.549002in}{1.971935in}}%
\pgfpathlineto{\pgfqpoint{1.546052in}{1.980937in}}%
\pgfpathlineto{\pgfqpoint{1.541602in}{1.989156in}}%
\pgfpathlineto{\pgfqpoint{1.541117in}{1.989938in}}%
\pgfpathlineto{\pgfqpoint{1.533232in}{1.998940in}}%
\pgfpathlineto{\pgfqpoint{1.531870in}{2.000155in}}%
\pgfpathlineto{\pgfqpoint{1.522139in}{2.007154in}}%
\pgfpathlineto{\pgfqpoint{1.520617in}{2.007941in}}%
\pgfpathlineto{\pgfqpoint{1.512407in}{2.011519in}}%
\pgfpathlineto{\pgfqpoint{1.502676in}{2.014024in}}%
\pgfpathlineto{\pgfqpoint{1.492944in}{2.014875in}}%
\pgfpathlineto{\pgfqpoint{1.483213in}{2.014306in}}%
\pgfpathlineto{\pgfqpoint{1.473481in}{2.012683in}}%
\pgfpathlineto{\pgfqpoint{1.463750in}{2.010465in}}%
\pgfpathlineto{\pgfqpoint{1.454018in}{2.008162in}}%
\pgfpathlineto{\pgfqpoint{1.452872in}{2.007941in}}%
\pgfpathlineto{\pgfqpoint{1.444287in}{2.005976in}}%
\pgfpathlineto{\pgfqpoint{1.434555in}{2.004800in}}%
\pgfpathlineto{\pgfqpoint{1.424824in}{2.005109in}}%
\pgfpathlineto{\pgfqpoint{1.415092in}{2.007183in}}%
\pgfpathlineto{\pgfqpoint{1.413216in}{2.007941in}}%
\pgfpathlineto{\pgfqpoint{1.405361in}{2.010616in}}%
\pgfpathlineto{\pgfqpoint{1.395629in}{2.015414in}}%
\pgfpathlineto{\pgfqpoint{1.393171in}{2.016943in}}%
\pgfpathlineto{\pgfqpoint{1.385898in}{2.020990in}}%
\pgfpathlineto{\pgfqpoint{1.378133in}{2.025944in}}%
\pgfpathlineto{\pgfqpoint{1.376166in}{2.027129in}}%
\pgfpathlineto{\pgfqpoint{1.366435in}{2.033351in}}%
\pgfpathlineto{\pgfqpoint{1.363912in}{2.034946in}}%
\pgfpathlineto{\pgfqpoint{1.356703in}{2.039466in}}%
\pgfpathlineto{\pgfqpoint{1.348921in}{2.043947in}}%
\pgfpathlineto{\pgfqpoint{1.346972in}{2.045116in}}%
\pgfpathlineto{\pgfqpoint{1.337240in}{2.050031in}}%
\pgfpathlineto{\pgfqpoint{1.329513in}{2.052949in}}%
\pgfpathlineto{\pgfqpoint{1.327509in}{2.053777in}}%
\pgfpathlineto{\pgfqpoint{1.317778in}{2.056217in}}%
\pgfpathlineto{\pgfqpoint{1.308046in}{2.056999in}}%
\pgfpathlineto{\pgfqpoint{1.298315in}{2.056201in}}%
\pgfpathlineto{\pgfqpoint{1.288583in}{2.054046in}}%
\pgfpathlineto{\pgfqpoint{1.285206in}{2.052949in}}%
\pgfpathlineto{\pgfqpoint{1.278852in}{2.051062in}}%
\pgfpathlineto{\pgfqpoint{1.269120in}{2.047652in}}%
\pgfpathlineto{\pgfqpoint{1.259389in}{2.044169in}}%
\pgfpathlineto{\pgfqpoint{1.258694in}{2.043947in}}%
\pgfpathlineto{\pgfqpoint{1.249657in}{2.041173in}}%
\pgfpathlineto{\pgfqpoint{1.239926in}{2.038928in}}%
\pgfpathlineto{\pgfqpoint{1.230194in}{2.037733in}}%
\pgfpathlineto{\pgfqpoint{1.220463in}{2.037783in}}%
\pgfpathlineto{\pgfqpoint{1.210731in}{2.039147in}}%
\pgfpathlineto{\pgfqpoint{1.201000in}{2.041757in}}%
\pgfpathlineto{\pgfqpoint{1.195173in}{2.043947in}}%
\pgfpathlineto{\pgfqpoint{1.191268in}{2.045476in}}%
\pgfpathlineto{\pgfqpoint{1.181537in}{2.050060in}}%
\pgfpathlineto{\pgfqpoint{1.175863in}{2.052949in}}%
\pgfpathlineto{\pgfqpoint{1.171805in}{2.055210in}}%
\pgfpathlineto{\pgfqpoint{1.162074in}{2.060551in}}%
\pgfpathlineto{\pgfqpoint{1.159220in}{2.061951in}}%
\pgfpathlineto{\pgfqpoint{1.152342in}{2.065840in}}%
\pgfpathlineto{\pgfqpoint{1.142611in}{2.070182in}}%
\pgfpathlineto{\pgfqpoint{1.139915in}{2.070952in}}%
\pgfpathlineto{\pgfqpoint{1.132879in}{2.073407in}}%
\pgfpathlineto{\pgfqpoint{1.123148in}{2.074575in}}%
\pgfpathlineto{\pgfqpoint{1.113416in}{2.073407in}}%
\pgfpathlineto{\pgfqpoint{1.106381in}{2.070952in}}%
\pgfpathlineto{\pgfqpoint{1.103685in}{2.070182in}}%
\pgfpathlineto{\pgfqpoint{1.093953in}{2.065840in}}%
\pgfpathlineto{\pgfqpoint{1.087076in}{2.061951in}}%
\pgfpathlineto{\pgfqpoint{1.084222in}{2.060551in}}%
\pgfpathlineto{\pgfqpoint{1.074491in}{2.055210in}}%
\pgfpathlineto{\pgfqpoint{1.070433in}{2.052949in}}%
\pgfpathlineto{\pgfqpoint{1.064759in}{2.050060in}}%
\pgfpathlineto{\pgfqpoint{1.055028in}{2.045476in}}%
\pgfpathlineto{\pgfqpoint{1.051123in}{2.043947in}}%
\pgfpathlineto{\pgfqpoint{1.045296in}{2.041757in}}%
\pgfpathlineto{\pgfqpoint{1.035565in}{2.039147in}}%
\pgfpathlineto{\pgfqpoint{1.025833in}{2.037783in}}%
\pgfpathlineto{\pgfqpoint{1.016102in}{2.037733in}}%
\pgfpathlineto{\pgfqpoint{1.006370in}{2.038928in}}%
\pgfpathlineto{\pgfqpoint{0.996639in}{2.041173in}}%
\pgfpathlineto{\pgfqpoint{0.987602in}{2.043947in}}%
\pgfpathlineto{\pgfqpoint{0.986907in}{2.044169in}}%
\pgfpathlineto{\pgfqpoint{0.977176in}{2.047652in}}%
\pgfpathlineto{\pgfqpoint{0.967444in}{2.051062in}}%
\pgfpathlineto{\pgfqpoint{0.961090in}{2.052949in}}%
\pgfpathlineto{\pgfqpoint{0.957713in}{2.054046in}}%
\pgfpathlineto{\pgfqpoint{0.947981in}{2.056201in}}%
\pgfpathlineto{\pgfqpoint{0.938250in}{2.056999in}}%
\pgfpathlineto{\pgfqpoint{0.928518in}{2.056217in}}%
\pgfpathlineto{\pgfqpoint{0.918787in}{2.053777in}}%
\pgfpathlineto{\pgfqpoint{0.916783in}{2.052949in}}%
\pgfpathlineto{\pgfqpoint{0.909055in}{2.050031in}}%
\pgfpathlineto{\pgfqpoint{0.899324in}{2.045116in}}%
\pgfpathlineto{\pgfqpoint{0.897375in}{2.043947in}}%
\pgfpathlineto{\pgfqpoint{0.889592in}{2.039466in}}%
\pgfpathlineto{\pgfqpoint{0.882384in}{2.034946in}}%
\pgfpathlineto{\pgfqpoint{0.879861in}{2.033351in}}%
\pgfpathlineto{\pgfqpoint{0.870129in}{2.027129in}}%
\pgfpathlineto{\pgfqpoint{0.868163in}{2.025944in}}%
\pgfpathlineto{\pgfqpoint{0.860398in}{2.020990in}}%
\pgfpathlineto{\pgfqpoint{0.853125in}{2.016943in}}%
\pgfpathlineto{\pgfqpoint{0.850667in}{2.015414in}}%
\pgfpathlineto{\pgfqpoint{0.840935in}{2.010616in}}%
\pgfpathlineto{\pgfqpoint{0.833080in}{2.007941in}}%
\pgfpathlineto{\pgfqpoint{0.831204in}{2.007183in}}%
\pgfpathlineto{\pgfqpoint{0.821472in}{2.005109in}}%
\pgfpathlineto{\pgfqpoint{0.811741in}{2.004800in}}%
\pgfpathlineto{\pgfqpoint{0.802009in}{2.005976in}}%
\pgfpathlineto{\pgfqpoint{0.793424in}{2.007941in}}%
\pgfpathlineto{\pgfqpoint{0.792278in}{2.008162in}}%
\pgfpathlineto{\pgfqpoint{0.782546in}{2.010465in}}%
\pgfpathlineto{\pgfqpoint{0.772815in}{2.012683in}}%
\pgfpathlineto{\pgfqpoint{0.763083in}{2.014306in}}%
\pgfpathlineto{\pgfqpoint{0.753352in}{2.014875in}}%
\pgfpathlineto{\pgfqpoint{0.743620in}{2.014024in}}%
\pgfpathlineto{\pgfqpoint{0.733889in}{2.011519in}}%
\pgfpathlineto{\pgfqpoint{0.725679in}{2.007941in}}%
\pgfpathlineto{\pgfqpoint{0.724157in}{2.007154in}}%
\pgfpathlineto{\pgfqpoint{0.714426in}{2.000155in}}%
\pgfpathlineto{\pgfqpoint{0.713064in}{1.998940in}}%
\pgfpathlineto{\pgfqpoint{0.705179in}{1.989938in}}%
\pgfpathlineto{\pgfqpoint{0.704694in}{1.989156in}}%
\pgfpathlineto{\pgfqpoint{0.700244in}{1.980937in}}%
\pgfpathlineto{\pgfqpoint{0.697294in}{1.971935in}}%
\pgfpathlineto{\pgfqpoint{0.695994in}{1.962934in}}%
\pgfpathlineto{\pgfqpoint{0.695921in}{1.953932in}}%
\pgfpathlineto{\pgfqpoint{0.696539in}{1.944931in}}%
\pgfpathlineto{\pgfqpoint{0.697255in}{1.935929in}}%
\pgfpathlineto{\pgfqpoint{0.697476in}{1.926927in}}%
\pgfpathlineto{\pgfqpoint{0.696667in}{1.917926in}}%
\pgfpathlineto{\pgfqpoint{0.694963in}{1.911152in}}%
\pgfpathlineto{\pgfqpoint{0.694440in}{1.908924in}}%
\pgfpathlineto{\pgfqpoint{0.690713in}{1.899923in}}%
\pgfpathlineto{\pgfqpoint{0.685285in}{1.890921in}}%
\pgfpathlineto{\pgfqpoint{0.685231in}{1.890852in}}%
\pgfpathlineto{\pgfqpoint{0.678310in}{1.881920in}}%
\pgfpathlineto{\pgfqpoint{0.675500in}{1.878880in}}%
\pgfpathlineto{\pgfqpoint{0.669717in}{1.872918in}}%
\pgfpathlineto{\pgfqpoint{0.665768in}{1.869265in}}%
\pgfpathlineto{\pgfqpoint{0.659324in}{1.863917in}}%
\pgfpathlineto{\pgfqpoint{0.656037in}{1.861317in}}%
\pgfpathlineto{\pgfqpoint{0.646380in}{1.854915in}}%
\pgfpathlineto{\pgfqpoint{0.646305in}{1.854865in}}%
\pgfpathlineto{\pgfqpoint{0.636574in}{1.849845in}}%
\pgfpathlineto{\pgfqpoint{0.626842in}{1.846397in}}%
\pgfpathlineto{\pgfqpoint{0.624435in}{1.845914in}}%
\pgfpathlineto{\pgfqpoint{0.617111in}{1.844337in}}%
\pgfpathlineto{\pgfqpoint{0.607380in}{1.843589in}}%
\pgfpathlineto{\pgfqpoint{0.597648in}{1.843793in}}%
\pgfpathlineto{\pgfqpoint{0.587917in}{1.844456in}}%
\pgfpathlineto{\pgfqpoint{0.578185in}{1.845027in}}%
\pgfpathlineto{\pgfqpoint{0.568454in}{1.844960in}}%
\pgfpathlineto{\pgfqpoint{0.558722in}{1.843758in}}%
\pgfpathlineto{\pgfqpoint{0.548991in}{1.841028in}}%
\pgfpathlineto{\pgfqpoint{0.540105in}{1.836912in}}%
\pgfpathlineto{\pgfqpoint{0.539259in}{1.836464in}}%
\pgfpathlineto{\pgfqpoint{0.529528in}{1.829170in}}%
\pgfpathlineto{\pgfqpoint{0.528214in}{1.827910in}}%
\pgfpathlineto{\pgfqpoint{0.520647in}{1.818909in}}%
\pgfpathlineto{\pgfqpoint{0.519796in}{1.817501in}}%
\pgfpathlineto{\pgfqpoint{0.515929in}{1.809907in}}%
\pgfpathlineto{\pgfqpoint{0.513220in}{1.800906in}}%
\pgfpathlineto{\pgfqpoint{0.512300in}{1.791904in}}%
\pgfpathlineto{\pgfqpoint{0.512915in}{1.782903in}}%
\pgfpathlineto{\pgfqpoint{0.514670in}{1.773901in}}%
\pgfpathlineto{\pgfqpoint{0.517068in}{1.764900in}}%
\pgfpathlineto{\pgfqpoint{0.519557in}{1.755898in}}%
\pgfpathlineto{\pgfqpoint{0.519796in}{1.754838in}}%
\pgfpathlineto{\pgfqpoint{0.521921in}{1.746897in}}%
\pgfpathlineto{\pgfqpoint{0.523193in}{1.737895in}}%
\pgfpathlineto{\pgfqpoint{0.522858in}{1.728893in}}%
\pgfpathlineto{\pgfqpoint{0.520616in}{1.719892in}}%
\pgfpathlineto{\pgfqpoint{0.519796in}{1.718156in}}%
\pgfpathlineto{\pgfqpoint{0.516905in}{1.710890in}}%
\pgfpathlineto{\pgfqpoint{0.511717in}{1.701889in}}%
\pgfpathlineto{\pgfqpoint{0.510065in}{1.699615in}}%
\pgfpathlineto{\pgfqpoint{0.505689in}{1.692887in}}%
\pgfpathlineto{\pgfqpoint{0.500333in}{1.685705in}}%
\pgfpathlineto{\pgfqpoint{0.499053in}{1.683886in}}%
\pgfpathlineto{\pgfqpoint{0.492326in}{1.674884in}}%
\pgfpathlineto{\pgfqpoint{0.490602in}{1.672550in}}%
\pgfpathlineto{\pgfqpoint{0.485715in}{1.665883in}}%
\pgfpathlineto{\pgfqpoint{0.480870in}{1.658684in}}%
\pgfpathlineto{\pgfqpoint{0.479607in}{1.656881in}}%
\pgfpathlineto{\pgfqpoint{0.474294in}{1.647880in}}%
\pgfpathlineto{\pgfqpoint{0.471139in}{1.640731in}}%
\pgfpathlineto{\pgfqpoint{0.470244in}{1.638878in}}%
\pgfpathlineto{\pgfqpoint{0.467606in}{1.629876in}}%
\pgfpathlineto{\pgfqpoint{0.466761in}{1.620875in}}%
\pgfpathlineto{\pgfqpoint{0.467623in}{1.611873in}}%
\pgfpathlineto{\pgfqpoint{0.469953in}{1.602872in}}%
\pgfpathlineto{\pgfqpoint{0.471139in}{1.599748in}}%
\pgfpathlineto{\pgfqpoint{0.473179in}{1.593870in}}%
\pgfpathlineto{\pgfqpoint{0.476865in}{1.584869in}}%
\pgfpathlineto{\pgfqpoint{0.480630in}{1.575867in}}%
\pgfpathlineto{\pgfqpoint{0.480870in}{1.575225in}}%
\pgfpathlineto{\pgfqpoint{0.483870in}{1.566866in}}%
\pgfpathlineto{\pgfqpoint{0.486297in}{1.557864in}}%
\pgfpathlineto{\pgfqpoint{0.487589in}{1.548863in}}%
\pgfpathlineto{\pgfqpoint{0.487535in}{1.539861in}}%
\pgfpathlineto{\pgfqpoint{0.486060in}{1.530859in}}%
\pgfpathlineto{\pgfqpoint{0.483239in}{1.521858in}}%
\pgfpathlineto{\pgfqpoint{0.480870in}{1.516468in}}%
\pgfpathlineto{\pgfqpoint{0.479218in}{1.512856in}}%
\pgfpathlineto{\pgfqpoint{0.474262in}{1.503855in}}%
\pgfpathlineto{\pgfqpoint{0.471139in}{1.498607in}}%
\pgfpathlineto{\pgfqpoint{0.468695in}{1.494853in}}%
\pgfpathlineto{\pgfqpoint{0.462921in}{1.485852in}}%
\pgfpathlineto{\pgfqpoint{0.461407in}{1.483212in}}%
\pgfpathlineto{\pgfqpoint{0.457202in}{1.476850in}}%
\pgfpathlineto{\pgfqpoint{0.452508in}{1.467849in}}%
\pgfpathlineto{\pgfqpoint{0.451676in}{1.465355in}}%
\pgfpathlineto{\pgfqpoint{0.449022in}{1.458847in}}%
\pgfpathlineto{\pgfqpoint{0.447759in}{1.449846in}}%
\pgfpathlineto{\pgfqpoint{0.449022in}{1.440844in}}%
\pgfpathlineto{\pgfqpoint{0.451676in}{1.434336in}}%
\pgfpathlineto{\pgfqpoint{0.452508in}{1.431842in}}%
\pgfpathlineto{\pgfqpoint{0.457202in}{1.422841in}}%
\pgfpathlineto{\pgfqpoint{0.461407in}{1.416479in}}%
\pgfpathlineto{\pgfqpoint{0.462921in}{1.413839in}}%
\pgfpathlineto{\pgfqpoint{0.468695in}{1.404838in}}%
\pgfpathlineto{\pgfqpoint{0.471139in}{1.401084in}}%
\pgfpathlineto{\pgfqpoint{0.474262in}{1.395836in}}%
\pgfpathlineto{\pgfqpoint{0.479218in}{1.386835in}}%
\pgfpathlineto{\pgfqpoint{0.480870in}{1.383223in}}%
\pgfpathlineto{\pgfqpoint{0.483239in}{1.377833in}}%
\pgfpathlineto{\pgfqpoint{0.486060in}{1.368832in}}%
\pgfpathlineto{\pgfqpoint{0.487535in}{1.359830in}}%
\pgfpathlineto{\pgfqpoint{0.487589in}{1.350829in}}%
\pgfpathlineto{\pgfqpoint{0.486297in}{1.341827in}}%
\pgfpathlineto{\pgfqpoint{0.483870in}{1.332825in}}%
\pgfpathlineto{\pgfqpoint{0.480870in}{1.324466in}}%
\pgfpathlineto{\pgfqpoint{0.480630in}{1.323824in}}%
\pgfpathlineto{\pgfqpoint{0.476865in}{1.314822in}}%
\pgfpathlineto{\pgfqpoint{0.473179in}{1.305821in}}%
\pgfpathlineto{\pgfqpoint{0.471139in}{1.299943in}}%
\pgfpathlineto{\pgfqpoint{0.469953in}{1.296819in}}%
\pgfpathlineto{\pgfqpoint{0.467623in}{1.287818in}}%
\pgfpathlineto{\pgfqpoint{0.466761in}{1.278816in}}%
\pgfpathlineto{\pgfqpoint{0.467606in}{1.269815in}}%
\pgfpathlineto{\pgfqpoint{0.470244in}{1.260813in}}%
\pgfpathlineto{\pgfqpoint{0.471139in}{1.258960in}}%
\pgfpathlineto{\pgfqpoint{0.474294in}{1.251812in}}%
\pgfpathlineto{\pgfqpoint{0.479607in}{1.242810in}}%
\pgfpathlineto{\pgfqpoint{0.480870in}{1.241007in}}%
\pgfpathlineto{\pgfqpoint{0.485715in}{1.233808in}}%
\pgfpathlineto{\pgfqpoint{0.490602in}{1.227141in}}%
\pgfpathlineto{\pgfqpoint{0.492326in}{1.224807in}}%
\pgfpathlineto{\pgfqpoint{0.499053in}{1.215805in}}%
\pgfpathlineto{\pgfqpoint{0.500333in}{1.213986in}}%
\pgfpathlineto{\pgfqpoint{0.505689in}{1.206804in}}%
\pgfpathlineto{\pgfqpoint{0.510065in}{1.200076in}}%
\pgfpathlineto{\pgfqpoint{0.511717in}{1.197802in}}%
\pgfpathlineto{\pgfqpoint{0.516905in}{1.188801in}}%
\pgfpathlineto{\pgfqpoint{0.519796in}{1.181535in}}%
\pgfpathlineto{\pgfqpoint{0.520616in}{1.179799in}}%
\pgfpathlineto{\pgfqpoint{0.522858in}{1.170798in}}%
\pgfpathlineto{\pgfqpoint{0.523193in}{1.161796in}}%
\pgfpathlineto{\pgfqpoint{0.521921in}{1.152795in}}%
\pgfpathlineto{\pgfqpoint{0.519796in}{1.144853in}}%
\pgfpathlineto{\pgfqpoint{0.519557in}{1.143793in}}%
\pgfpathlineto{\pgfqpoint{0.517068in}{1.134792in}}%
\pgfpathlineto{\pgfqpoint{0.514670in}{1.125790in}}%
\pgfpathlineto{\pgfqpoint{0.512915in}{1.116788in}}%
\pgfpathlineto{\pgfqpoint{0.512300in}{1.107787in}}%
\pgfpathlineto{\pgfqpoint{0.513220in}{1.098785in}}%
\pgfpathlineto{\pgfqpoint{0.515929in}{1.089784in}}%
\pgfpathlineto{\pgfqpoint{0.519796in}{1.082190in}}%
\pgfpathlineto{\pgfqpoint{0.520647in}{1.080782in}}%
\pgfpathlineto{\pgfqpoint{0.528214in}{1.071781in}}%
\pgfpathlineto{\pgfqpoint{0.529528in}{1.070521in}}%
\pgfpathlineto{\pgfqpoint{0.539259in}{1.063228in}}%
\pgfpathlineto{\pgfqpoint{0.540105in}{1.062779in}}%
\pgfpathlineto{\pgfqpoint{0.548991in}{1.058663in}}%
\pgfpathlineto{\pgfqpoint{0.558722in}{1.055934in}}%
\pgfpathlineto{\pgfqpoint{0.568454in}{1.054732in}}%
\pgfpathlineto{\pgfqpoint{0.578185in}{1.054664in}}%
\pgfpathlineto{\pgfqpoint{0.587917in}{1.055235in}}%
\pgfpathlineto{\pgfqpoint{0.597648in}{1.055898in}}%
\pgfpathlineto{\pgfqpoint{0.607380in}{1.056102in}}%
\pgfpathlineto{\pgfqpoint{0.617111in}{1.055354in}}%
\pgfpathlineto{\pgfqpoint{0.624435in}{1.053778in}}%
\pgfpathlineto{\pgfqpoint{0.626842in}{1.053294in}}%
\pgfpathlineto{\pgfqpoint{0.636574in}{1.049846in}}%
\pgfpathlineto{\pgfqpoint{0.646305in}{1.044826in}}%
\pgfpathlineto{\pgfqpoint{0.646380in}{1.044776in}}%
\pgfpathlineto{\pgfqpoint{0.656037in}{1.038374in}}%
\pgfpathlineto{\pgfqpoint{0.659324in}{1.035775in}}%
\pgfpathlineto{\pgfqpoint{0.665768in}{1.030426in}}%
\pgfpathlineto{\pgfqpoint{0.669717in}{1.026773in}}%
\pgfpathlineto{\pgfqpoint{0.675500in}{1.020811in}}%
\pgfpathlineto{\pgfqpoint{0.678310in}{1.017771in}}%
\pgfpathlineto{\pgfqpoint{0.685231in}{1.008839in}}%
\pgfpathlineto{\pgfqpoint{0.685285in}{1.008770in}}%
\pgfpathlineto{\pgfqpoint{0.690713in}{0.999768in}}%
\pgfpathlineto{\pgfqpoint{0.694440in}{0.990767in}}%
\pgfpathlineto{\pgfqpoint{0.694963in}{0.988540in}}%
\pgfpathlineto{\pgfqpoint{0.696667in}{0.981765in}}%
\pgfpathlineto{\pgfqpoint{0.697476in}{0.972764in}}%
\pgfpathlineto{\pgfqpoint{0.697255in}{0.963762in}}%
\pgfpathlineto{\pgfqpoint{0.696539in}{0.954761in}}%
\pgfpathlineto{\pgfqpoint{0.695921in}{0.945759in}}%
\pgfpathlineto{\pgfqpoint{0.695994in}{0.936758in}}%
\pgfpathlineto{\pgfqpoint{0.697294in}{0.927756in}}%
\pgfpathlineto{\pgfqpoint{0.700244in}{0.918754in}}%
\pgfpathlineto{\pgfqpoint{0.704694in}{0.910535in}}%
\pgfpathlineto{\pgfqpoint{0.705179in}{0.909753in}}%
\pgfpathlineto{\pgfqpoint{0.713064in}{0.900751in}}%
\pgfpathlineto{\pgfqpoint{0.714426in}{0.899536in}}%
\pgfpathlineto{\pgfqpoint{0.724157in}{0.892537in}}%
\pgfpathlineto{\pgfqpoint{0.725679in}{0.891750in}}%
\pgfpathlineto{\pgfqpoint{0.733889in}{0.888172in}}%
\pgfpathlineto{\pgfqpoint{0.743620in}{0.885667in}}%
\pgfpathlineto{\pgfqpoint{0.753352in}{0.884816in}}%
\pgfpathlineto{\pgfqpoint{0.763083in}{0.885385in}}%
\pgfpathlineto{\pgfqpoint{0.772815in}{0.887008in}}%
\pgfpathlineto{\pgfqpoint{0.782546in}{0.889226in}}%
\pgfpathlineto{\pgfqpoint{0.792278in}{0.891529in}}%
\pgfpathlineto{\pgfqpoint{0.793424in}{0.891750in}}%
\pgfpathlineto{\pgfqpoint{0.802009in}{0.893715in}}%
\pgfpathlineto{\pgfqpoint{0.811741in}{0.894891in}}%
\pgfpathlineto{\pgfqpoint{0.821472in}{0.894582in}}%
\pgfpathlineto{\pgfqpoint{0.831204in}{0.892508in}}%
\pgfpathlineto{\pgfqpoint{0.833080in}{0.891750in}}%
\pgfpathlineto{\pgfqpoint{0.840935in}{0.889076in}}%
\pgfpathlineto{\pgfqpoint{0.850667in}{0.884277in}}%
\pgfpathlineto{\pgfqpoint{0.853125in}{0.882748in}}%
\pgfpathlineto{\pgfqpoint{0.860398in}{0.878701in}}%
\pgfpathlineto{\pgfqpoint{0.868163in}{0.873747in}}%
\pgfpathlineto{\pgfqpoint{0.870129in}{0.872562in}}%
\pgfpathlineto{\pgfqpoint{0.879861in}{0.866340in}}%
\pgfpathlineto{\pgfqpoint{0.882384in}{0.864745in}}%
\pgfpathlineto{\pgfqpoint{0.889592in}{0.860225in}}%
\pgfpathlineto{\pgfqpoint{0.897375in}{0.855744in}}%
\pgfpathlineto{\pgfqpoint{0.899324in}{0.854575in}}%
\pgfpathlineto{\pgfqpoint{0.909055in}{0.849660in}}%
\pgfpathlineto{\pgfqpoint{0.916783in}{0.846742in}}%
\pgfpathlineto{\pgfqpoint{0.918787in}{0.845914in}}%
\pgfpathlineto{\pgfqpoint{0.928518in}{0.843474in}}%
\pgfpathlineto{\pgfqpoint{0.938250in}{0.842692in}}%
\pgfpathlineto{\pgfqpoint{0.947981in}{0.843490in}}%
\pgfpathlineto{\pgfqpoint{0.957713in}{0.845645in}}%
\pgfpathlineto{\pgfqpoint{0.961090in}{0.846742in}}%
\pgfpathlineto{\pgfqpoint{0.967444in}{0.848629in}}%
\pgfpathlineto{\pgfqpoint{0.977176in}{0.852039in}}%
\pgfpathlineto{\pgfqpoint{0.986907in}{0.855522in}}%
\pgfpathlineto{\pgfqpoint{0.987602in}{0.855744in}}%
\pgfpathlineto{\pgfqpoint{0.996639in}{0.858518in}}%
\pgfpathlineto{\pgfqpoint{1.006370in}{0.860763in}}%
\pgfpathlineto{\pgfqpoint{1.016102in}{0.861958in}}%
\pgfpathlineto{\pgfqpoint{1.025833in}{0.861908in}}%
\pgfpathlineto{\pgfqpoint{1.035565in}{0.860544in}}%
\pgfpathlineto{\pgfqpoint{1.045296in}{0.857935in}}%
\pgfpathlineto{\pgfqpoint{1.051123in}{0.855744in}}%
\pgfpathlineto{\pgfqpoint{1.055028in}{0.854215in}}%
\pgfpathlineto{\pgfqpoint{1.064759in}{0.849631in}}%
\pgfpathlineto{\pgfqpoint{1.070433in}{0.846742in}}%
\pgfpathlineto{\pgfqpoint{1.074491in}{0.844481in}}%
\pgfpathlineto{\pgfqpoint{1.084222in}{0.839140in}}%
\pgfpathlineto{\pgfqpoint{1.087076in}{0.837741in}}%
\pgfpathlineto{\pgfqpoint{1.093953in}{0.833851in}}%
\pgfpathlineto{\pgfqpoint{1.103685in}{0.829509in}}%
\pgfpathlineto{\pgfqpoint{1.106381in}{0.828739in}}%
\pgfpathclose%
\pgfpathmoveto{\pgfqpoint{1.097512in}{0.900751in}}%
\pgfpathlineto{\pgfqpoint{1.093953in}{0.902940in}}%
\pgfpathlineto{\pgfqpoint{1.085217in}{0.909753in}}%
\pgfpathlineto{\pgfqpoint{1.084222in}{0.910850in}}%
\pgfpathlineto{\pgfqpoint{1.077815in}{0.918754in}}%
\pgfpathlineto{\pgfqpoint{1.074491in}{0.925530in}}%
\pgfpathlineto{\pgfqpoint{1.073415in}{0.927756in}}%
\pgfpathlineto{\pgfqpoint{1.071498in}{0.936758in}}%
\pgfpathlineto{\pgfqpoint{1.071390in}{0.945759in}}%
\pgfpathlineto{\pgfqpoint{1.072302in}{0.954761in}}%
\pgfpathlineto{\pgfqpoint{1.073358in}{0.963762in}}%
\pgfpathlineto{\pgfqpoint{1.073684in}{0.972764in}}%
\pgfpathlineto{\pgfqpoint{1.072490in}{0.981765in}}%
\pgfpathlineto{\pgfqpoint{1.069151in}{0.990767in}}%
\pgfpathlineto{\pgfqpoint{1.064759in}{0.997483in}}%
\pgfpathlineto{\pgfqpoint{1.063144in}{0.999768in}}%
\pgfpathlineto{\pgfqpoint{1.055028in}{1.007651in}}%
\pgfpathlineto{\pgfqpoint{1.053642in}{1.008770in}}%
\pgfpathlineto{\pgfqpoint{1.045296in}{1.013964in}}%
\pgfpathlineto{\pgfqpoint{1.036719in}{1.017771in}}%
\pgfpathlineto{\pgfqpoint{1.035565in}{1.018201in}}%
\pgfpathlineto{\pgfqpoint{1.025833in}{1.020093in}}%
\pgfpathlineto{\pgfqpoint{1.016102in}{1.020163in}}%
\pgfpathlineto{\pgfqpoint{1.006370in}{1.018504in}}%
\pgfpathlineto{\pgfqpoint{1.004081in}{1.017771in}}%
\pgfpathlineto{\pgfqpoint{0.996639in}{1.014930in}}%
\pgfpathlineto{\pgfqpoint{0.986907in}{1.009984in}}%
\pgfpathlineto{\pgfqpoint{0.984772in}{1.008770in}}%
\pgfpathlineto{\pgfqpoint{0.977176in}{1.003163in}}%
\pgfpathlineto{\pgfqpoint{0.972476in}{0.999768in}}%
\pgfpathlineto{\pgfqpoint{0.967444in}{0.994475in}}%
\pgfpathlineto{\pgfqpoint{0.963287in}{0.990767in}}%
\pgfpathlineto{\pgfqpoint{0.957713in}{0.982002in}}%
\pgfpathlineto{\pgfqpoint{0.957491in}{0.981765in}}%
\pgfpathlineto{\pgfqpoint{0.954488in}{0.972764in}}%
\pgfpathlineto{\pgfqpoint{0.955308in}{0.963762in}}%
\pgfpathlineto{\pgfqpoint{0.957713in}{0.955617in}}%
\pgfpathlineto{\pgfqpoint{0.957885in}{0.954761in}}%
\pgfpathlineto{\pgfqpoint{0.959448in}{0.945759in}}%
\pgfpathlineto{\pgfqpoint{0.959263in}{0.936758in}}%
\pgfpathlineto{\pgfqpoint{0.957713in}{0.932511in}}%
\pgfpathlineto{\pgfqpoint{0.955165in}{0.927756in}}%
\pgfpathlineto{\pgfqpoint{0.947981in}{0.921850in}}%
\pgfpathlineto{\pgfqpoint{0.938250in}{0.918886in}}%
\pgfpathlineto{\pgfqpoint{0.928518in}{0.921791in}}%
\pgfpathlineto{\pgfqpoint{0.922111in}{0.927756in}}%
\pgfpathlineto{\pgfqpoint{0.918787in}{0.934784in}}%
\pgfpathlineto{\pgfqpoint{0.918220in}{0.936758in}}%
\pgfpathlineto{\pgfqpoint{0.918075in}{0.945759in}}%
\pgfpathlineto{\pgfqpoint{0.918787in}{0.950975in}}%
\pgfpathlineto{\pgfqpoint{0.919639in}{0.954761in}}%
\pgfpathlineto{\pgfqpoint{0.921985in}{0.963762in}}%
\pgfpathlineto{\pgfqpoint{0.922709in}{0.972764in}}%
\pgfpathlineto{\pgfqpoint{0.920057in}{0.981765in}}%
\pgfpathlineto{\pgfqpoint{0.918787in}{0.983307in}}%
\pgfpathlineto{\pgfqpoint{0.915055in}{0.990767in}}%
\pgfpathlineto{\pgfqpoint{0.909055in}{0.997570in}}%
\pgfpathlineto{\pgfqpoint{0.907606in}{0.999768in}}%
\pgfpathlineto{\pgfqpoint{0.899324in}{1.008393in}}%
\pgfpathlineto{\pgfqpoint{0.899019in}{1.008770in}}%
\pgfpathlineto{\pgfqpoint{0.889592in}{1.017755in}}%
\pgfpathlineto{\pgfqpoint{0.889577in}{1.017771in}}%
\pgfpathlineto{\pgfqpoint{0.879861in}{1.026225in}}%
\pgfpathlineto{\pgfqpoint{0.879239in}{1.026773in}}%
\pgfpathlineto{\pgfqpoint{0.870129in}{1.033967in}}%
\pgfpathlineto{\pgfqpoint{0.867700in}{1.035775in}}%
\pgfpathlineto{\pgfqpoint{0.860398in}{1.040952in}}%
\pgfpathlineto{\pgfqpoint{0.854216in}{1.044776in}}%
\pgfpathlineto{\pgfqpoint{0.850667in}{1.046992in}}%
\pgfpathlineto{\pgfqpoint{0.840935in}{1.051811in}}%
\pgfpathlineto{\pgfqpoint{0.835181in}{1.053778in}}%
\pgfpathlineto{\pgfqpoint{0.831204in}{1.055234in}}%
\pgfpathlineto{\pgfqpoint{0.821472in}{1.057112in}}%
\pgfpathlineto{\pgfqpoint{0.811741in}{1.057393in}}%
\pgfpathlineto{\pgfqpoint{0.802009in}{1.056327in}}%
\pgfpathlineto{\pgfqpoint{0.792278in}{1.054309in}}%
\pgfpathlineto{\pgfqpoint{0.790188in}{1.053778in}}%
\pgfpathlineto{\pgfqpoint{0.782546in}{1.051962in}}%
\pgfpathlineto{\pgfqpoint{0.772815in}{1.049735in}}%
\pgfpathlineto{\pgfqpoint{0.763083in}{1.048104in}}%
\pgfpathlineto{\pgfqpoint{0.753352in}{1.047533in}}%
\pgfpathlineto{\pgfqpoint{0.743620in}{1.048388in}}%
\pgfpathlineto{\pgfqpoint{0.733889in}{1.050904in}}%
\pgfpathlineto{\pgfqpoint{0.727321in}{1.053778in}}%
\pgfpathlineto{\pgfqpoint{0.724157in}{1.055260in}}%
\pgfpathlineto{\pgfqpoint{0.714426in}{1.061600in}}%
\pgfpathlineto{\pgfqpoint{0.712968in}{1.062779in}}%
\pgfpathlineto{\pgfqpoint{0.704694in}{1.070432in}}%
\pgfpathlineto{\pgfqpoint{0.703420in}{1.071781in}}%
\pgfpathlineto{\pgfqpoint{0.696566in}{1.080782in}}%
\pgfpathlineto{\pgfqpoint{0.694963in}{1.083708in}}%
\pgfpathlineto{\pgfqpoint{0.691856in}{1.089784in}}%
\pgfpathlineto{\pgfqpoint{0.689136in}{1.098785in}}%
\pgfpathlineto{\pgfqpoint{0.688212in}{1.107787in}}%
\pgfpathlineto{\pgfqpoint{0.688830in}{1.116788in}}%
\pgfpathlineto{\pgfqpoint{0.690592in}{1.125790in}}%
\pgfpathlineto{\pgfqpoint{0.693000in}{1.134792in}}%
\pgfpathlineto{\pgfqpoint{0.694963in}{1.141860in}}%
\pgfpathlineto{\pgfqpoint{0.695538in}{1.143793in}}%
\pgfpathlineto{\pgfqpoint{0.697719in}{1.152795in}}%
\pgfpathlineto{\pgfqpoint{0.698871in}{1.161796in}}%
\pgfpathlineto{\pgfqpoint{0.698568in}{1.170798in}}%
\pgfpathlineto{\pgfqpoint{0.696537in}{1.179799in}}%
\pgfpathlineto{\pgfqpoint{0.694963in}{1.183478in}}%
\pgfpathlineto{\pgfqpoint{0.692836in}{1.188801in}}%
\pgfpathlineto{\pgfqpoint{0.687627in}{1.197802in}}%
\pgfpathlineto{\pgfqpoint{0.685231in}{1.201085in}}%
\pgfpathlineto{\pgfqpoint{0.681098in}{1.206804in}}%
\pgfpathlineto{\pgfqpoint{0.675500in}{1.213559in}}%
\pgfpathlineto{\pgfqpoint{0.673546in}{1.215805in}}%
\pgfpathlineto{\pgfqpoint{0.665768in}{1.224232in}}%
\pgfpathlineto{\pgfqpoint{0.665176in}{1.224807in}}%
\pgfpathlineto{\pgfqpoint{0.656037in}{1.233794in}}%
\pgfpathlineto{\pgfqpoint{0.656019in}{1.233808in}}%
\pgfpathlineto{\pgfqpoint{0.646305in}{1.242528in}}%
\pgfpathlineto{\pgfqpoint{0.645898in}{1.242810in}}%
\pgfpathlineto{\pgfqpoint{0.636574in}{1.250471in}}%
\pgfpathlineto{\pgfqpoint{0.634198in}{1.251812in}}%
\pgfpathlineto{\pgfqpoint{0.626842in}{1.257361in}}%
\pgfpathlineto{\pgfqpoint{0.618777in}{1.260813in}}%
\pgfpathlineto{\pgfqpoint{0.617111in}{1.261988in}}%
\pgfpathlineto{\pgfqpoint{0.607380in}{1.264441in}}%
\pgfpathlineto{\pgfqpoint{0.597648in}{1.263771in}}%
\pgfpathlineto{\pgfqpoint{0.587917in}{1.261601in}}%
\pgfpathlineto{\pgfqpoint{0.583824in}{1.260813in}}%
\pgfpathlineto{\pgfqpoint{0.578185in}{1.260154in}}%
\pgfpathlineto{\pgfqpoint{0.568454in}{1.260289in}}%
\pgfpathlineto{\pgfqpoint{0.566320in}{1.260813in}}%
\pgfpathlineto{\pgfqpoint{0.558722in}{1.263888in}}%
\pgfpathlineto{\pgfqpoint{0.552273in}{1.269815in}}%
\pgfpathlineto{\pgfqpoint{0.549133in}{1.278816in}}%
\pgfpathlineto{\pgfqpoint{0.552337in}{1.287818in}}%
\pgfpathlineto{\pgfqpoint{0.558722in}{1.294463in}}%
\pgfpathlineto{\pgfqpoint{0.563863in}{1.296819in}}%
\pgfpathlineto{\pgfqpoint{0.568454in}{1.298253in}}%
\pgfpathlineto{\pgfqpoint{0.578185in}{1.298424in}}%
\pgfpathlineto{\pgfqpoint{0.587917in}{1.296979in}}%
\pgfpathlineto{\pgfqpoint{0.588843in}{1.296819in}}%
\pgfpathlineto{\pgfqpoint{0.597648in}{1.294595in}}%
\pgfpathlineto{\pgfqpoint{0.607380in}{1.293837in}}%
\pgfpathlineto{\pgfqpoint{0.617111in}{1.296615in}}%
\pgfpathlineto{\pgfqpoint{0.617367in}{1.296819in}}%
\pgfpathlineto{\pgfqpoint{0.626842in}{1.301975in}}%
\pgfpathlineto{\pgfqpoint{0.630851in}{1.305821in}}%
\pgfpathlineto{\pgfqpoint{0.636574in}{1.310475in}}%
\pgfpathlineto{\pgfqpoint{0.640244in}{1.314822in}}%
\pgfpathlineto{\pgfqpoint{0.646305in}{1.321849in}}%
\pgfpathlineto{\pgfqpoint{0.647618in}{1.323824in}}%
\pgfpathlineto{\pgfqpoint{0.652965in}{1.332825in}}%
\pgfpathlineto{\pgfqpoint{0.656037in}{1.339709in}}%
\pgfpathlineto{\pgfqpoint{0.656829in}{1.341827in}}%
\pgfpathlineto{\pgfqpoint{0.658623in}{1.350829in}}%
\pgfpathlineto{\pgfqpoint{0.658547in}{1.359830in}}%
\pgfpathlineto{\pgfqpoint{0.656501in}{1.368832in}}%
\pgfpathlineto{\pgfqpoint{0.656037in}{1.369899in}}%
\pgfpathlineto{\pgfqpoint{0.651921in}{1.377833in}}%
\pgfpathlineto{\pgfqpoint{0.646305in}{1.385553in}}%
\pgfpathlineto{\pgfqpoint{0.645096in}{1.386835in}}%
\pgfpathlineto{\pgfqpoint{0.636574in}{1.394342in}}%
\pgfpathlineto{\pgfqpoint{0.634104in}{1.395836in}}%
\pgfpathlineto{\pgfqpoint{0.626842in}{1.399899in}}%
\pgfpathlineto{\pgfqpoint{0.617111in}{1.402988in}}%
\pgfpathlineto{\pgfqpoint{0.607380in}{1.404091in}}%
\pgfpathlineto{\pgfqpoint{0.597648in}{1.403790in}}%
\pgfpathlineto{\pgfqpoint{0.587917in}{1.402813in}}%
\pgfpathlineto{\pgfqpoint{0.578185in}{1.401970in}}%
\pgfpathlineto{\pgfqpoint{0.568454in}{1.402070in}}%
\pgfpathlineto{\pgfqpoint{0.558722in}{1.403843in}}%
\pgfpathlineto{\pgfqpoint{0.556316in}{1.404838in}}%
\pgfpathlineto{\pgfqpoint{0.548991in}{1.407913in}}%
\pgfpathlineto{\pgfqpoint{0.540445in}{1.413839in}}%
\pgfpathlineto{\pgfqpoint{0.539259in}{1.414760in}}%
\pgfpathlineto{\pgfqpoint{0.531894in}{1.422841in}}%
\pgfpathlineto{\pgfqpoint{0.529528in}{1.426133in}}%
\pgfpathlineto{\pgfqpoint{0.526318in}{1.431842in}}%
\pgfpathlineto{\pgfqpoint{0.523081in}{1.440844in}}%
\pgfpathlineto{\pgfqpoint{0.521967in}{1.449846in}}%
\pgfpathlineto{\pgfqpoint{0.523081in}{1.458847in}}%
\pgfpathlineto{\pgfqpoint{0.526318in}{1.467849in}}%
\pgfpathlineto{\pgfqpoint{0.529528in}{1.473559in}}%
\pgfpathlineto{\pgfqpoint{0.531894in}{1.476850in}}%
\pgfpathlineto{\pgfqpoint{0.539259in}{1.484931in}}%
\pgfpathlineto{\pgfqpoint{0.540445in}{1.485852in}}%
\pgfpathlineto{\pgfqpoint{0.548991in}{1.491778in}}%
\pgfpathlineto{\pgfqpoint{0.556316in}{1.494853in}}%
\pgfpathlineto{\pgfqpoint{0.558722in}{1.495849in}}%
\pgfpathlineto{\pgfqpoint{0.568454in}{1.497621in}}%
\pgfpathlineto{\pgfqpoint{0.578185in}{1.497721in}}%
\pgfpathlineto{\pgfqpoint{0.587917in}{1.496878in}}%
\pgfpathlineto{\pgfqpoint{0.597648in}{1.495901in}}%
\pgfpathlineto{\pgfqpoint{0.607380in}{1.495600in}}%
\pgfpathlineto{\pgfqpoint{0.617111in}{1.496704in}}%
\pgfpathlineto{\pgfqpoint{0.626842in}{1.499792in}}%
\pgfpathlineto{\pgfqpoint{0.634104in}{1.503855in}}%
\pgfpathlineto{\pgfqpoint{0.636574in}{1.505349in}}%
\pgfpathlineto{\pgfqpoint{0.645096in}{1.512856in}}%
\pgfpathlineto{\pgfqpoint{0.646305in}{1.514138in}}%
\pgfpathlineto{\pgfqpoint{0.651921in}{1.521858in}}%
\pgfpathlineto{\pgfqpoint{0.656037in}{1.529792in}}%
\pgfpathlineto{\pgfqpoint{0.656501in}{1.530859in}}%
\pgfpathlineto{\pgfqpoint{0.658547in}{1.539861in}}%
\pgfpathlineto{\pgfqpoint{0.658623in}{1.548863in}}%
\pgfpathlineto{\pgfqpoint{0.656829in}{1.557864in}}%
\pgfpathlineto{\pgfqpoint{0.656037in}{1.559982in}}%
\pgfpathlineto{\pgfqpoint{0.652965in}{1.566866in}}%
\pgfpathlineto{\pgfqpoint{0.647618in}{1.575867in}}%
\pgfpathlineto{\pgfqpoint{0.646305in}{1.577842in}}%
\pgfpathlineto{\pgfqpoint{0.640244in}{1.584869in}}%
\pgfpathlineto{\pgfqpoint{0.636574in}{1.589216in}}%
\pgfpathlineto{\pgfqpoint{0.630851in}{1.593870in}}%
\pgfpathlineto{\pgfqpoint{0.626842in}{1.597716in}}%
\pgfpathlineto{\pgfqpoint{0.617367in}{1.602872in}}%
\pgfpathlineto{\pgfqpoint{0.617111in}{1.603077in}}%
\pgfpathlineto{\pgfqpoint{0.607380in}{1.605854in}}%
\pgfpathlineto{\pgfqpoint{0.597648in}{1.605096in}}%
\pgfpathlineto{\pgfqpoint{0.588843in}{1.602872in}}%
\pgfpathlineto{\pgfqpoint{0.587917in}{1.602712in}}%
\pgfpathlineto{\pgfqpoint{0.578185in}{1.601267in}}%
\pgfpathlineto{\pgfqpoint{0.568454in}{1.601438in}}%
\pgfpathlineto{\pgfqpoint{0.563863in}{1.602872in}}%
\pgfpathlineto{\pgfqpoint{0.558722in}{1.605228in}}%
\pgfpathlineto{\pgfqpoint{0.552337in}{1.611873in}}%
\pgfpathlineto{\pgfqpoint{0.549133in}{1.620875in}}%
\pgfpathlineto{\pgfqpoint{0.552273in}{1.629876in}}%
\pgfpathlineto{\pgfqpoint{0.558722in}{1.635803in}}%
\pgfpathlineto{\pgfqpoint{0.566320in}{1.638878in}}%
\pgfpathlineto{\pgfqpoint{0.568454in}{1.639402in}}%
\pgfpathlineto{\pgfqpoint{0.578185in}{1.639537in}}%
\pgfpathlineto{\pgfqpoint{0.583824in}{1.638878in}}%
\pgfpathlineto{\pgfqpoint{0.587917in}{1.638090in}}%
\pgfpathlineto{\pgfqpoint{0.597648in}{1.635920in}}%
\pgfpathlineto{\pgfqpoint{0.607380in}{1.635250in}}%
\pgfpathlineto{\pgfqpoint{0.617111in}{1.637703in}}%
\pgfpathlineto{\pgfqpoint{0.618777in}{1.638878in}}%
\pgfpathlineto{\pgfqpoint{0.626842in}{1.642330in}}%
\pgfpathlineto{\pgfqpoint{0.634198in}{1.647880in}}%
\pgfpathlineto{\pgfqpoint{0.636574in}{1.649220in}}%
\pgfpathlineto{\pgfqpoint{0.645898in}{1.656881in}}%
\pgfpathlineto{\pgfqpoint{0.646305in}{1.657163in}}%
\pgfpathlineto{\pgfqpoint{0.656019in}{1.665883in}}%
\pgfpathlineto{\pgfqpoint{0.656037in}{1.665897in}}%
\pgfpathlineto{\pgfqpoint{0.665176in}{1.674884in}}%
\pgfpathlineto{\pgfqpoint{0.665768in}{1.675460in}}%
\pgfpathlineto{\pgfqpoint{0.673546in}{1.683886in}}%
\pgfpathlineto{\pgfqpoint{0.675500in}{1.686133in}}%
\pgfpathlineto{\pgfqpoint{0.681098in}{1.692887in}}%
\pgfpathlineto{\pgfqpoint{0.685231in}{1.698606in}}%
\pgfpathlineto{\pgfqpoint{0.687627in}{1.701889in}}%
\pgfpathlineto{\pgfqpoint{0.692836in}{1.710890in}}%
\pgfpathlineto{\pgfqpoint{0.694963in}{1.716213in}}%
\pgfpathlineto{\pgfqpoint{0.696537in}{1.719892in}}%
\pgfpathlineto{\pgfqpoint{0.698568in}{1.728893in}}%
\pgfpathlineto{\pgfqpoint{0.698871in}{1.737895in}}%
\pgfpathlineto{\pgfqpoint{0.697719in}{1.746897in}}%
\pgfpathlineto{\pgfqpoint{0.695538in}{1.755898in}}%
\pgfpathlineto{\pgfqpoint{0.694963in}{1.757831in}}%
\pgfpathlineto{\pgfqpoint{0.693000in}{1.764900in}}%
\pgfpathlineto{\pgfqpoint{0.690592in}{1.773901in}}%
\pgfpathlineto{\pgfqpoint{0.688830in}{1.782903in}}%
\pgfpathlineto{\pgfqpoint{0.688212in}{1.791904in}}%
\pgfpathlineto{\pgfqpoint{0.689136in}{1.800906in}}%
\pgfpathlineto{\pgfqpoint{0.691856in}{1.809907in}}%
\pgfpathlineto{\pgfqpoint{0.694963in}{1.815983in}}%
\pgfpathlineto{\pgfqpoint{0.696566in}{1.818909in}}%
\pgfpathlineto{\pgfqpoint{0.703420in}{1.827910in}}%
\pgfpathlineto{\pgfqpoint{0.704694in}{1.829259in}}%
\pgfpathlineto{\pgfqpoint{0.712968in}{1.836912in}}%
\pgfpathlineto{\pgfqpoint{0.714426in}{1.838091in}}%
\pgfpathlineto{\pgfqpoint{0.724157in}{1.844431in}}%
\pgfpathlineto{\pgfqpoint{0.727321in}{1.845914in}}%
\pgfpathlineto{\pgfqpoint{0.733889in}{1.848788in}}%
\pgfpathlineto{\pgfqpoint{0.743620in}{1.851303in}}%
\pgfpathlineto{\pgfqpoint{0.753352in}{1.852158in}}%
\pgfpathlineto{\pgfqpoint{0.763083in}{1.851587in}}%
\pgfpathlineto{\pgfqpoint{0.772815in}{1.849956in}}%
\pgfpathlineto{\pgfqpoint{0.782546in}{1.847729in}}%
\pgfpathlineto{\pgfqpoint{0.790188in}{1.845914in}}%
\pgfpathlineto{\pgfqpoint{0.792278in}{1.845382in}}%
\pgfpathlineto{\pgfqpoint{0.802009in}{1.843364in}}%
\pgfpathlineto{\pgfqpoint{0.811741in}{1.842298in}}%
\pgfpathlineto{\pgfqpoint{0.821472in}{1.842579in}}%
\pgfpathlineto{\pgfqpoint{0.831204in}{1.844457in}}%
\pgfpathlineto{\pgfqpoint{0.835181in}{1.845914in}}%
\pgfpathlineto{\pgfqpoint{0.840935in}{1.847881in}}%
\pgfpathlineto{\pgfqpoint{0.850667in}{1.852699in}}%
\pgfpathlineto{\pgfqpoint{0.854216in}{1.854915in}}%
\pgfpathlineto{\pgfqpoint{0.860398in}{1.858739in}}%
\pgfpathlineto{\pgfqpoint{0.867700in}{1.863917in}}%
\pgfpathlineto{\pgfqpoint{0.870129in}{1.865724in}}%
\pgfpathlineto{\pgfqpoint{0.879239in}{1.872918in}}%
\pgfpathlineto{\pgfqpoint{0.879861in}{1.873466in}}%
\pgfpathlineto{\pgfqpoint{0.889577in}{1.881920in}}%
\pgfpathlineto{\pgfqpoint{0.889592in}{1.881936in}}%
\pgfpathlineto{\pgfqpoint{0.899019in}{1.890921in}}%
\pgfpathlineto{\pgfqpoint{0.899324in}{1.891298in}}%
\pgfpathlineto{\pgfqpoint{0.907606in}{1.899923in}}%
\pgfpathlineto{\pgfqpoint{0.909055in}{1.902121in}}%
\pgfpathlineto{\pgfqpoint{0.915055in}{1.908924in}}%
\pgfpathlineto{\pgfqpoint{0.918787in}{1.916384in}}%
\pgfpathlineto{\pgfqpoint{0.920057in}{1.917926in}}%
\pgfpathlineto{\pgfqpoint{0.922709in}{1.926927in}}%
\pgfpathlineto{\pgfqpoint{0.921985in}{1.935929in}}%
\pgfpathlineto{\pgfqpoint{0.919639in}{1.944930in}}%
\pgfpathlineto{\pgfqpoint{0.918787in}{1.948716in}}%
\pgfpathlineto{\pgfqpoint{0.918075in}{1.953932in}}%
\pgfpathlineto{\pgfqpoint{0.918220in}{1.962934in}}%
\pgfpathlineto{\pgfqpoint{0.918787in}{1.964907in}}%
\pgfpathlineto{\pgfqpoint{0.922111in}{1.971935in}}%
\pgfpathlineto{\pgfqpoint{0.928518in}{1.977900in}}%
\pgfpathlineto{\pgfqpoint{0.938250in}{1.980805in}}%
\pgfpathlineto{\pgfqpoint{0.947981in}{1.977841in}}%
\pgfpathlineto{\pgfqpoint{0.955165in}{1.971935in}}%
\pgfpathlineto{\pgfqpoint{0.957713in}{1.967180in}}%
\pgfpathlineto{\pgfqpoint{0.959263in}{1.962934in}}%
\pgfpathlineto{\pgfqpoint{0.959448in}{1.953932in}}%
\pgfpathlineto{\pgfqpoint{0.957885in}{1.944931in}}%
\pgfpathlineto{\pgfqpoint{0.957713in}{1.944074in}}%
\pgfpathlineto{\pgfqpoint{0.955308in}{1.935929in}}%
\pgfpathlineto{\pgfqpoint{0.954488in}{1.926927in}}%
\pgfpathlineto{\pgfqpoint{0.957491in}{1.917926in}}%
\pgfpathlineto{\pgfqpoint{0.957713in}{1.917689in}}%
\pgfpathlineto{\pgfqpoint{0.963287in}{1.908924in}}%
\pgfpathlineto{\pgfqpoint{0.967444in}{1.905216in}}%
\pgfpathlineto{\pgfqpoint{0.972476in}{1.899923in}}%
\pgfpathlineto{\pgfqpoint{0.977176in}{1.896528in}}%
\pgfpathlineto{\pgfqpoint{0.984772in}{1.890921in}}%
\pgfpathlineto{\pgfqpoint{0.986907in}{1.889707in}}%
\pgfpathlineto{\pgfqpoint{0.996639in}{1.884761in}}%
\pgfpathlineto{\pgfqpoint{1.004081in}{1.881920in}}%
\pgfpathlineto{\pgfqpoint{1.006370in}{1.881187in}}%
\pgfpathlineto{\pgfqpoint{1.016102in}{1.879528in}}%
\pgfpathlineto{\pgfqpoint{1.025833in}{1.879598in}}%
\pgfpathlineto{\pgfqpoint{1.035565in}{1.881490in}}%
\pgfpathlineto{\pgfqpoint{1.036719in}{1.881920in}}%
\pgfpathlineto{\pgfqpoint{1.045296in}{1.885727in}}%
\pgfpathlineto{\pgfqpoint{1.053642in}{1.890921in}}%
\pgfpathlineto{\pgfqpoint{1.055028in}{1.892040in}}%
\pgfpathlineto{\pgfqpoint{1.063144in}{1.899923in}}%
\pgfpathlineto{\pgfqpoint{1.064759in}{1.902208in}}%
\pgfpathlineto{\pgfqpoint{1.069151in}{1.908924in}}%
\pgfpathlineto{\pgfqpoint{1.072490in}{1.917926in}}%
\pgfpathlineto{\pgfqpoint{1.073684in}{1.926927in}}%
\pgfpathlineto{\pgfqpoint{1.073358in}{1.935929in}}%
\pgfpathlineto{\pgfqpoint{1.072302in}{1.944931in}}%
\pgfpathlineto{\pgfqpoint{1.071390in}{1.953932in}}%
\pgfpathlineto{\pgfqpoint{1.071498in}{1.962934in}}%
\pgfpathlineto{\pgfqpoint{1.073415in}{1.971935in}}%
\pgfpathlineto{\pgfqpoint{1.074491in}{1.974161in}}%
\pgfpathlineto{\pgfqpoint{1.077815in}{1.980937in}}%
\pgfpathlineto{\pgfqpoint{1.084222in}{1.988841in}}%
\pgfpathlineto{\pgfqpoint{1.085217in}{1.989938in}}%
\pgfpathlineto{\pgfqpoint{1.093953in}{1.996751in}}%
\pgfpathlineto{\pgfqpoint{1.097512in}{1.998940in}}%
\pgfpathlineto{\pgfqpoint{1.103685in}{2.001909in}}%
\pgfpathlineto{\pgfqpoint{1.113416in}{2.004903in}}%
\pgfpathlineto{\pgfqpoint{1.123148in}{2.005933in}}%
\pgfpathlineto{\pgfqpoint{1.132879in}{2.004903in}}%
\pgfpathlineto{\pgfqpoint{1.142611in}{2.001909in}}%
\pgfpathlineto{\pgfqpoint{1.148784in}{1.998940in}}%
\pgfpathlineto{\pgfqpoint{1.152342in}{1.996751in}}%
\pgfpathlineto{\pgfqpoint{1.161079in}{1.989938in}}%
\pgfpathlineto{\pgfqpoint{1.162074in}{1.988841in}}%
\pgfpathlineto{\pgfqpoint{1.168481in}{1.980937in}}%
\pgfpathlineto{\pgfqpoint{1.171805in}{1.974161in}}%
\pgfpathlineto{\pgfqpoint{1.172881in}{1.971935in}}%
\pgfpathlineto{\pgfqpoint{1.174798in}{1.962934in}}%
\pgfpathlineto{\pgfqpoint{1.174906in}{1.953932in}}%
\pgfpathlineto{\pgfqpoint{1.173994in}{1.944931in}}%
\pgfpathlineto{\pgfqpoint{1.172938in}{1.935929in}}%
\pgfpathlineto{\pgfqpoint{1.172612in}{1.926927in}}%
\pgfpathlineto{\pgfqpoint{1.173806in}{1.917926in}}%
\pgfpathlineto{\pgfqpoint{1.177145in}{1.908924in}}%
\pgfpathlineto{\pgfqpoint{1.181537in}{1.902208in}}%
\pgfpathlineto{\pgfqpoint{1.183152in}{1.899923in}}%
\pgfpathlineto{\pgfqpoint{1.191268in}{1.892040in}}%
\pgfpathlineto{\pgfqpoint{1.192654in}{1.890921in}}%
\pgfpathlineto{\pgfqpoint{1.201000in}{1.885727in}}%
\pgfpathlineto{\pgfqpoint{1.209577in}{1.881920in}}%
\pgfpathlineto{\pgfqpoint{1.210731in}{1.881490in}}%
\pgfpathlineto{\pgfqpoint{1.220463in}{1.879598in}}%
\pgfpathlineto{\pgfqpoint{1.230194in}{1.879528in}}%
\pgfpathlineto{\pgfqpoint{1.239926in}{1.881187in}}%
\pgfpathlineto{\pgfqpoint{1.242215in}{1.881920in}}%
\pgfpathlineto{\pgfqpoint{1.249657in}{1.884761in}}%
\pgfpathlineto{\pgfqpoint{1.259389in}{1.889707in}}%
\pgfpathlineto{\pgfqpoint{1.261523in}{1.890921in}}%
\pgfpathlineto{\pgfqpoint{1.269120in}{1.896528in}}%
\pgfpathlineto{\pgfqpoint{1.273820in}{1.899923in}}%
\pgfpathlineto{\pgfqpoint{1.278852in}{1.905216in}}%
\pgfpathlineto{\pgfqpoint{1.283009in}{1.908924in}}%
\pgfpathlineto{\pgfqpoint{1.288583in}{1.917689in}}%
\pgfpathlineto{\pgfqpoint{1.288804in}{1.917926in}}%
\pgfpathlineto{\pgfqpoint{1.291807in}{1.926927in}}%
\pgfpathlineto{\pgfqpoint{1.290987in}{1.935929in}}%
\pgfpathlineto{\pgfqpoint{1.288583in}{1.944074in}}%
\pgfpathlineto{\pgfqpoint{1.288411in}{1.944931in}}%
\pgfpathlineto{\pgfqpoint{1.286848in}{1.953932in}}%
\pgfpathlineto{\pgfqpoint{1.287033in}{1.962934in}}%
\pgfpathlineto{\pgfqpoint{1.288583in}{1.967180in}}%
\pgfpathlineto{\pgfqpoint{1.291131in}{1.971935in}}%
\pgfpathlineto{\pgfqpoint{1.298315in}{1.977841in}}%
\pgfpathlineto{\pgfqpoint{1.308046in}{1.980805in}}%
\pgfpathlineto{\pgfqpoint{1.317778in}{1.977900in}}%
\pgfpathlineto{\pgfqpoint{1.324184in}{1.971935in}}%
\pgfpathlineto{\pgfqpoint{1.327509in}{1.964907in}}%
\pgfpathlineto{\pgfqpoint{1.328076in}{1.962934in}}%
\pgfpathlineto{\pgfqpoint{1.328221in}{1.953932in}}%
\pgfpathlineto{\pgfqpoint{1.327509in}{1.948716in}}%
\pgfpathlineto{\pgfqpoint{1.326657in}{1.944931in}}%
\pgfpathlineto{\pgfqpoint{1.324311in}{1.935929in}}%
\pgfpathlineto{\pgfqpoint{1.323587in}{1.926927in}}%
\pgfpathlineto{\pgfqpoint{1.326239in}{1.917926in}}%
\pgfpathlineto{\pgfqpoint{1.327509in}{1.916384in}}%
\pgfpathlineto{\pgfqpoint{1.331241in}{1.908924in}}%
\pgfpathlineto{\pgfqpoint{1.337240in}{1.902121in}}%
\pgfpathlineto{\pgfqpoint{1.338690in}{1.899923in}}%
\pgfpathlineto{\pgfqpoint{1.346972in}{1.891298in}}%
\pgfpathlineto{\pgfqpoint{1.347277in}{1.890921in}}%
\pgfpathlineto{\pgfqpoint{1.356703in}{1.881936in}}%
\pgfpathlineto{\pgfqpoint{1.356719in}{1.881920in}}%
\pgfpathlineto{\pgfqpoint{1.366435in}{1.873466in}}%
\pgfpathlineto{\pgfqpoint{1.367057in}{1.872918in}}%
\pgfpathlineto{\pgfqpoint{1.376166in}{1.865724in}}%
\pgfpathlineto{\pgfqpoint{1.378595in}{1.863917in}}%
\pgfpathlineto{\pgfqpoint{1.385898in}{1.858739in}}%
\pgfpathlineto{\pgfqpoint{1.392080in}{1.854915in}}%
\pgfpathlineto{\pgfqpoint{1.395629in}{1.852699in}}%
\pgfpathlineto{\pgfqpoint{1.405361in}{1.847881in}}%
\pgfpathlineto{\pgfqpoint{1.411115in}{1.845914in}}%
\pgfpathlineto{\pgfqpoint{1.415092in}{1.844457in}}%
\pgfpathlineto{\pgfqpoint{1.424824in}{1.842579in}}%
\pgfpathlineto{\pgfqpoint{1.434555in}{1.842298in}}%
\pgfpathlineto{\pgfqpoint{1.444287in}{1.843364in}}%
\pgfpathlineto{\pgfqpoint{1.454018in}{1.845382in}}%
\pgfpathlineto{\pgfqpoint{1.456108in}{1.845914in}}%
\pgfpathlineto{\pgfqpoint{1.463750in}{1.847729in}}%
\pgfpathlineto{\pgfqpoint{1.473481in}{1.849956in}}%
\pgfpathlineto{\pgfqpoint{1.483213in}{1.851587in}}%
\pgfpathlineto{\pgfqpoint{1.492944in}{1.852158in}}%
\pgfpathlineto{\pgfqpoint{1.502676in}{1.851303in}}%
\pgfpathlineto{\pgfqpoint{1.512407in}{1.848788in}}%
\pgfpathlineto{\pgfqpoint{1.518975in}{1.845914in}}%
\pgfpathlineto{\pgfqpoint{1.522139in}{1.844431in}}%
\pgfpathlineto{\pgfqpoint{1.531870in}{1.838091in}}%
\pgfpathlineto{\pgfqpoint{1.533328in}{1.836912in}}%
\pgfpathlineto{\pgfqpoint{1.541602in}{1.829259in}}%
\pgfpathlineto{\pgfqpoint{1.542876in}{1.827910in}}%
\pgfpathlineto{\pgfqpoint{1.549730in}{1.818909in}}%
\pgfpathlineto{\pgfqpoint{1.551333in}{1.815983in}}%
\pgfpathlineto{\pgfqpoint{1.554440in}{1.809907in}}%
\pgfpathlineto{\pgfqpoint{1.557160in}{1.800906in}}%
\pgfpathlineto{\pgfqpoint{1.558084in}{1.791904in}}%
\pgfpathlineto{\pgfqpoint{1.557466in}{1.782903in}}%
\pgfpathlineto{\pgfqpoint{1.555704in}{1.773901in}}%
\pgfpathlineto{\pgfqpoint{1.553296in}{1.764900in}}%
\pgfpathlineto{\pgfqpoint{1.551333in}{1.757831in}}%
\pgfpathlineto{\pgfqpoint{1.550758in}{1.755898in}}%
\pgfpathlineto{\pgfqpoint{1.548576in}{1.746897in}}%
\pgfpathlineto{\pgfqpoint{1.547425in}{1.737895in}}%
\pgfpathlineto{\pgfqpoint{1.547728in}{1.728893in}}%
\pgfpathlineto{\pgfqpoint{1.549759in}{1.719892in}}%
\pgfpathlineto{\pgfqpoint{1.551333in}{1.716213in}}%
\pgfpathlineto{\pgfqpoint{1.553460in}{1.710890in}}%
\pgfpathlineto{\pgfqpoint{1.558669in}{1.701889in}}%
\pgfpathlineto{\pgfqpoint{1.561064in}{1.698606in}}%
\pgfpathlineto{\pgfqpoint{1.565198in}{1.692887in}}%
\pgfpathlineto{\pgfqpoint{1.570796in}{1.686133in}}%
\pgfpathlineto{\pgfqpoint{1.572750in}{1.683886in}}%
\pgfpathlineto{\pgfqpoint{1.580527in}{1.675460in}}%
\pgfpathlineto{\pgfqpoint{1.581119in}{1.674884in}}%
\pgfpathlineto{\pgfqpoint{1.590259in}{1.665897in}}%
\pgfpathlineto{\pgfqpoint{1.590276in}{1.665883in}}%
\pgfpathlineto{\pgfqpoint{1.599990in}{1.657163in}}%
\pgfpathlineto{\pgfqpoint{1.600398in}{1.656881in}}%
\pgfpathlineto{\pgfqpoint{1.609722in}{1.649220in}}%
\pgfpathlineto{\pgfqpoint{1.612098in}{1.647880in}}%
\pgfpathlineto{\pgfqpoint{1.619453in}{1.642330in}}%
\pgfpathlineto{\pgfqpoint{1.627518in}{1.638878in}}%
\pgfpathlineto{\pgfqpoint{1.629185in}{1.637703in}}%
\pgfpathlineto{\pgfqpoint{1.638916in}{1.635250in}}%
\pgfpathlineto{\pgfqpoint{1.648648in}{1.635920in}}%
\pgfpathlineto{\pgfqpoint{1.658379in}{1.638090in}}%
\pgfpathlineto{\pgfqpoint{1.662472in}{1.638878in}}%
\pgfpathlineto{\pgfqpoint{1.668111in}{1.639537in}}%
\pgfpathlineto{\pgfqpoint{1.677842in}{1.639402in}}%
\pgfpathlineto{\pgfqpoint{1.679976in}{1.638878in}}%
\pgfpathlineto{\pgfqpoint{1.687574in}{1.635803in}}%
\pgfpathlineto{\pgfqpoint{1.694023in}{1.629876in}}%
\pgfpathlineto{\pgfqpoint{1.697163in}{1.620875in}}%
\pgfpathlineto{\pgfqpoint{1.693959in}{1.611873in}}%
\pgfpathlineto{\pgfqpoint{1.687574in}{1.605228in}}%
\pgfpathlineto{\pgfqpoint{1.682433in}{1.602872in}}%
\pgfpathlineto{\pgfqpoint{1.677842in}{1.601438in}}%
\pgfpathlineto{\pgfqpoint{1.668111in}{1.601267in}}%
\pgfpathlineto{\pgfqpoint{1.658379in}{1.602712in}}%
\pgfpathlineto{\pgfqpoint{1.657453in}{1.602872in}}%
\pgfpathlineto{\pgfqpoint{1.648648in}{1.605096in}}%
\pgfpathlineto{\pgfqpoint{1.638916in}{1.605854in}}%
\pgfpathlineto{\pgfqpoint{1.629185in}{1.603077in}}%
\pgfpathlineto{\pgfqpoint{1.628928in}{1.602872in}}%
\pgfpathlineto{\pgfqpoint{1.619453in}{1.597716in}}%
\pgfpathlineto{\pgfqpoint{1.615445in}{1.593870in}}%
\pgfpathlineto{\pgfqpoint{1.609722in}{1.589216in}}%
\pgfpathlineto{\pgfqpoint{1.606051in}{1.584869in}}%
\pgfpathlineto{\pgfqpoint{1.599990in}{1.577842in}}%
\pgfpathlineto{\pgfqpoint{1.598677in}{1.575867in}}%
\pgfpathlineto{\pgfqpoint{1.593331in}{1.566866in}}%
\pgfpathlineto{\pgfqpoint{1.590259in}{1.559982in}}%
\pgfpathlineto{\pgfqpoint{1.589467in}{1.557864in}}%
\pgfpathlineto{\pgfqpoint{1.587673in}{1.548863in}}%
\pgfpathlineto{\pgfqpoint{1.587749in}{1.539861in}}%
\pgfpathlineto{\pgfqpoint{1.589795in}{1.530859in}}%
\pgfpathlineto{\pgfqpoint{1.590259in}{1.529792in}}%
\pgfpathlineto{\pgfqpoint{1.594375in}{1.521858in}}%
\pgfpathlineto{\pgfqpoint{1.599990in}{1.514138in}}%
\pgfpathlineto{\pgfqpoint{1.601200in}{1.512856in}}%
\pgfpathlineto{\pgfqpoint{1.609722in}{1.505349in}}%
\pgfpathlineto{\pgfqpoint{1.612192in}{1.503855in}}%
\pgfpathlineto{\pgfqpoint{1.619453in}{1.499792in}}%
\pgfpathlineto{\pgfqpoint{1.629185in}{1.496704in}}%
\pgfpathlineto{\pgfqpoint{1.638916in}{1.495600in}}%
\pgfpathlineto{\pgfqpoint{1.648648in}{1.495901in}}%
\pgfpathlineto{\pgfqpoint{1.658379in}{1.496878in}}%
\pgfpathlineto{\pgfqpoint{1.668111in}{1.497721in}}%
\pgfpathlineto{\pgfqpoint{1.677842in}{1.497621in}}%
\pgfpathlineto{\pgfqpoint{1.687574in}{1.495849in}}%
\pgfpathlineto{\pgfqpoint{1.689980in}{1.494853in}}%
\pgfpathlineto{\pgfqpoint{1.697305in}{1.491778in}}%
\pgfpathlineto{\pgfqpoint{1.705851in}{1.485852in}}%
\pgfpathlineto{\pgfqpoint{1.707037in}{1.484931in}}%
\pgfpathlineto{\pgfqpoint{1.714402in}{1.476850in}}%
\pgfpathlineto{\pgfqpoint{1.716768in}{1.473559in}}%
\pgfpathlineto{\pgfqpoint{1.719978in}{1.467849in}}%
\pgfpathlineto{\pgfqpoint{1.723215in}{1.458847in}}%
\pgfpathlineto{\pgfqpoint{1.724329in}{1.449846in}}%
\pgfpathlineto{\pgfqpoint{1.723215in}{1.440844in}}%
\pgfpathlineto{\pgfqpoint{1.719978in}{1.431842in}}%
\pgfpathlineto{\pgfqpoint{1.716768in}{1.426133in}}%
\pgfpathlineto{\pgfqpoint{1.714402in}{1.422841in}}%
\pgfpathlineto{\pgfqpoint{1.707037in}{1.414760in}}%
\pgfpathlineto{\pgfqpoint{1.705851in}{1.413839in}}%
\pgfpathlineto{\pgfqpoint{1.697305in}{1.407913in}}%
\pgfpathlineto{\pgfqpoint{1.689980in}{1.404838in}}%
\pgfpathlineto{\pgfqpoint{1.687574in}{1.403843in}}%
\pgfpathlineto{\pgfqpoint{1.677842in}{1.402070in}}%
\pgfpathlineto{\pgfqpoint{1.668111in}{1.401970in}}%
\pgfpathlineto{\pgfqpoint{1.658379in}{1.402813in}}%
\pgfpathlineto{\pgfqpoint{1.648648in}{1.403790in}}%
\pgfpathlineto{\pgfqpoint{1.638916in}{1.404091in}}%
\pgfpathlineto{\pgfqpoint{1.629185in}{1.402988in}}%
\pgfpathlineto{\pgfqpoint{1.619453in}{1.399899in}}%
\pgfpathlineto{\pgfqpoint{1.612192in}{1.395836in}}%
\pgfpathlineto{\pgfqpoint{1.609722in}{1.394342in}}%
\pgfpathlineto{\pgfqpoint{1.601200in}{1.386835in}}%
\pgfpathlineto{\pgfqpoint{1.599990in}{1.385553in}}%
\pgfpathlineto{\pgfqpoint{1.594375in}{1.377833in}}%
\pgfpathlineto{\pgfqpoint{1.590259in}{1.369899in}}%
\pgfpathlineto{\pgfqpoint{1.589795in}{1.368832in}}%
\pgfpathlineto{\pgfqpoint{1.587749in}{1.359830in}}%
\pgfpathlineto{\pgfqpoint{1.587673in}{1.350829in}}%
\pgfpathlineto{\pgfqpoint{1.589467in}{1.341827in}}%
\pgfpathlineto{\pgfqpoint{1.590259in}{1.339709in}}%
\pgfpathlineto{\pgfqpoint{1.593331in}{1.332825in}}%
\pgfpathlineto{\pgfqpoint{1.598677in}{1.323824in}}%
\pgfpathlineto{\pgfqpoint{1.599990in}{1.321849in}}%
\pgfpathlineto{\pgfqpoint{1.606051in}{1.314822in}}%
\pgfpathlineto{\pgfqpoint{1.609722in}{1.310475in}}%
\pgfpathlineto{\pgfqpoint{1.615445in}{1.305821in}}%
\pgfpathlineto{\pgfqpoint{1.619453in}{1.301975in}}%
\pgfpathlineto{\pgfqpoint{1.628928in}{1.296819in}}%
\pgfpathlineto{\pgfqpoint{1.629185in}{1.296615in}}%
\pgfpathlineto{\pgfqpoint{1.638916in}{1.293837in}}%
\pgfpathlineto{\pgfqpoint{1.648648in}{1.294595in}}%
\pgfpathlineto{\pgfqpoint{1.657453in}{1.296819in}}%
\pgfpathlineto{\pgfqpoint{1.658379in}{1.296979in}}%
\pgfpathlineto{\pgfqpoint{1.668111in}{1.298424in}}%
\pgfpathlineto{\pgfqpoint{1.677842in}{1.298253in}}%
\pgfpathlineto{\pgfqpoint{1.682433in}{1.296819in}}%
\pgfpathlineto{\pgfqpoint{1.687574in}{1.294463in}}%
\pgfpathlineto{\pgfqpoint{1.693959in}{1.287818in}}%
\pgfpathlineto{\pgfqpoint{1.697163in}{1.278816in}}%
\pgfpathlineto{\pgfqpoint{1.694023in}{1.269815in}}%
\pgfpathlineto{\pgfqpoint{1.687574in}{1.263888in}}%
\pgfpathlineto{\pgfqpoint{1.679976in}{1.260813in}}%
\pgfpathlineto{\pgfqpoint{1.677842in}{1.260289in}}%
\pgfpathlineto{\pgfqpoint{1.668111in}{1.260154in}}%
\pgfpathlineto{\pgfqpoint{1.662472in}{1.260813in}}%
\pgfpathlineto{\pgfqpoint{1.658379in}{1.261601in}}%
\pgfpathlineto{\pgfqpoint{1.648648in}{1.263771in}}%
\pgfpathlineto{\pgfqpoint{1.638916in}{1.264441in}}%
\pgfpathlineto{\pgfqpoint{1.629185in}{1.261988in}}%
\pgfpathlineto{\pgfqpoint{1.627518in}{1.260813in}}%
\pgfpathlineto{\pgfqpoint{1.619453in}{1.257361in}}%
\pgfpathlineto{\pgfqpoint{1.612098in}{1.251812in}}%
\pgfpathlineto{\pgfqpoint{1.609722in}{1.250471in}}%
\pgfpathlineto{\pgfqpoint{1.600398in}{1.242810in}}%
\pgfpathlineto{\pgfqpoint{1.599990in}{1.242528in}}%
\pgfpathlineto{\pgfqpoint{1.590276in}{1.233808in}}%
\pgfpathlineto{\pgfqpoint{1.590259in}{1.233794in}}%
\pgfpathlineto{\pgfqpoint{1.581119in}{1.224807in}}%
\pgfpathlineto{\pgfqpoint{1.580527in}{1.224232in}}%
\pgfpathlineto{\pgfqpoint{1.572750in}{1.215805in}}%
\pgfpathlineto{\pgfqpoint{1.570796in}{1.213559in}}%
\pgfpathlineto{\pgfqpoint{1.565198in}{1.206804in}}%
\pgfpathlineto{\pgfqpoint{1.561064in}{1.201085in}}%
\pgfpathlineto{\pgfqpoint{1.558669in}{1.197802in}}%
\pgfpathlineto{\pgfqpoint{1.553460in}{1.188801in}}%
\pgfpathlineto{\pgfqpoint{1.551333in}{1.183478in}}%
\pgfpathlineto{\pgfqpoint{1.549759in}{1.179799in}}%
\pgfpathlineto{\pgfqpoint{1.547728in}{1.170798in}}%
\pgfpathlineto{\pgfqpoint{1.547425in}{1.161796in}}%
\pgfpathlineto{\pgfqpoint{1.548576in}{1.152795in}}%
\pgfpathlineto{\pgfqpoint{1.550758in}{1.143793in}}%
\pgfpathlineto{\pgfqpoint{1.551333in}{1.141860in}}%
\pgfpathlineto{\pgfqpoint{1.553296in}{1.134792in}}%
\pgfpathlineto{\pgfqpoint{1.555704in}{1.125790in}}%
\pgfpathlineto{\pgfqpoint{1.557466in}{1.116788in}}%
\pgfpathlineto{\pgfqpoint{1.558084in}{1.107787in}}%
\pgfpathlineto{\pgfqpoint{1.557160in}{1.098785in}}%
\pgfpathlineto{\pgfqpoint{1.554440in}{1.089784in}}%
\pgfpathlineto{\pgfqpoint{1.551333in}{1.083708in}}%
\pgfpathlineto{\pgfqpoint{1.549730in}{1.080782in}}%
\pgfpathlineto{\pgfqpoint{1.542876in}{1.071781in}}%
\pgfpathlineto{\pgfqpoint{1.541602in}{1.070432in}}%
\pgfpathlineto{\pgfqpoint{1.533328in}{1.062779in}}%
\pgfpathlineto{\pgfqpoint{1.531870in}{1.061600in}}%
\pgfpathlineto{\pgfqpoint{1.522139in}{1.055260in}}%
\pgfpathlineto{\pgfqpoint{1.518975in}{1.053778in}}%
\pgfpathlineto{\pgfqpoint{1.512407in}{1.050904in}}%
\pgfpathlineto{\pgfqpoint{1.502676in}{1.048388in}}%
\pgfpathlineto{\pgfqpoint{1.492944in}{1.047533in}}%
\pgfpathlineto{\pgfqpoint{1.483213in}{1.048104in}}%
\pgfpathlineto{\pgfqpoint{1.473481in}{1.049735in}}%
\pgfpathlineto{\pgfqpoint{1.463750in}{1.051962in}}%
\pgfpathlineto{\pgfqpoint{1.456108in}{1.053778in}}%
\pgfpathlineto{\pgfqpoint{1.454018in}{1.054309in}}%
\pgfpathlineto{\pgfqpoint{1.444287in}{1.056327in}}%
\pgfpathlineto{\pgfqpoint{1.434555in}{1.057393in}}%
\pgfpathlineto{\pgfqpoint{1.424824in}{1.057112in}}%
\pgfpathlineto{\pgfqpoint{1.415092in}{1.055234in}}%
\pgfpathlineto{\pgfqpoint{1.411115in}{1.053778in}}%
\pgfpathlineto{\pgfqpoint{1.405361in}{1.051811in}}%
\pgfpathlineto{\pgfqpoint{1.395629in}{1.046992in}}%
\pgfpathlineto{\pgfqpoint{1.392080in}{1.044776in}}%
\pgfpathlineto{\pgfqpoint{1.385898in}{1.040952in}}%
\pgfpathlineto{\pgfqpoint{1.378595in}{1.035775in}}%
\pgfpathlineto{\pgfqpoint{1.376166in}{1.033967in}}%
\pgfpathlineto{\pgfqpoint{1.367057in}{1.026773in}}%
\pgfpathlineto{\pgfqpoint{1.366435in}{1.026225in}}%
\pgfpathlineto{\pgfqpoint{1.356719in}{1.017771in}}%
\pgfpathlineto{\pgfqpoint{1.356703in}{1.017755in}}%
\pgfpathlineto{\pgfqpoint{1.347277in}{1.008770in}}%
\pgfpathlineto{\pgfqpoint{1.346972in}{1.008393in}}%
\pgfpathlineto{\pgfqpoint{1.338690in}{0.999768in}}%
\pgfpathlineto{\pgfqpoint{1.337240in}{0.997570in}}%
\pgfpathlineto{\pgfqpoint{1.331241in}{0.990767in}}%
\pgfpathlineto{\pgfqpoint{1.327509in}{0.983307in}}%
\pgfpathlineto{\pgfqpoint{1.326239in}{0.981765in}}%
\pgfpathlineto{\pgfqpoint{1.323587in}{0.972764in}}%
\pgfpathlineto{\pgfqpoint{1.324311in}{0.963762in}}%
\pgfpathlineto{\pgfqpoint{1.326657in}{0.954761in}}%
\pgfpathlineto{\pgfqpoint{1.327509in}{0.950975in}}%
\pgfpathlineto{\pgfqpoint{1.328221in}{0.945759in}}%
\pgfpathlineto{\pgfqpoint{1.328076in}{0.936758in}}%
\pgfpathlineto{\pgfqpoint{1.327509in}{0.934784in}}%
\pgfpathlineto{\pgfqpoint{1.324184in}{0.927756in}}%
\pgfpathlineto{\pgfqpoint{1.317778in}{0.921791in}}%
\pgfpathlineto{\pgfqpoint{1.308046in}{0.918886in}}%
\pgfpathlineto{\pgfqpoint{1.298315in}{0.921850in}}%
\pgfpathlineto{\pgfqpoint{1.291131in}{0.927756in}}%
\pgfpathlineto{\pgfqpoint{1.288583in}{0.932511in}}%
\pgfpathlineto{\pgfqpoint{1.287033in}{0.936758in}}%
\pgfpathlineto{\pgfqpoint{1.286848in}{0.945759in}}%
\pgfpathlineto{\pgfqpoint{1.288411in}{0.954761in}}%
\pgfpathlineto{\pgfqpoint{1.288583in}{0.955617in}}%
\pgfpathlineto{\pgfqpoint{1.290987in}{0.963762in}}%
\pgfpathlineto{\pgfqpoint{1.291807in}{0.972764in}}%
\pgfpathlineto{\pgfqpoint{1.288804in}{0.981765in}}%
\pgfpathlineto{\pgfqpoint{1.288583in}{0.982002in}}%
\pgfpathlineto{\pgfqpoint{1.283009in}{0.990767in}}%
\pgfpathlineto{\pgfqpoint{1.278852in}{0.994475in}}%
\pgfpathlineto{\pgfqpoint{1.273820in}{0.999768in}}%
\pgfpathlineto{\pgfqpoint{1.269120in}{1.003163in}}%
\pgfpathlineto{\pgfqpoint{1.261523in}{1.008770in}}%
\pgfpathlineto{\pgfqpoint{1.259389in}{1.009984in}}%
\pgfpathlineto{\pgfqpoint{1.249657in}{1.014930in}}%
\pgfpathlineto{\pgfqpoint{1.242215in}{1.017771in}}%
\pgfpathlineto{\pgfqpoint{1.239926in}{1.018504in}}%
\pgfpathlineto{\pgfqpoint{1.230194in}{1.020163in}}%
\pgfpathlineto{\pgfqpoint{1.220463in}{1.020093in}}%
\pgfpathlineto{\pgfqpoint{1.210731in}{1.018201in}}%
\pgfpathlineto{\pgfqpoint{1.209577in}{1.017771in}}%
\pgfpathlineto{\pgfqpoint{1.201000in}{1.013964in}}%
\pgfpathlineto{\pgfqpoint{1.192654in}{1.008770in}}%
\pgfpathlineto{\pgfqpoint{1.191268in}{1.007651in}}%
\pgfpathlineto{\pgfqpoint{1.183152in}{0.999768in}}%
\pgfpathlineto{\pgfqpoint{1.181537in}{0.997483in}}%
\pgfpathlineto{\pgfqpoint{1.177145in}{0.990767in}}%
\pgfpathlineto{\pgfqpoint{1.173806in}{0.981765in}}%
\pgfpathlineto{\pgfqpoint{1.172612in}{0.972764in}}%
\pgfpathlineto{\pgfqpoint{1.172938in}{0.963762in}}%
\pgfpathlineto{\pgfqpoint{1.173994in}{0.954761in}}%
\pgfpathlineto{\pgfqpoint{1.174906in}{0.945759in}}%
\pgfpathlineto{\pgfqpoint{1.174798in}{0.936758in}}%
\pgfpathlineto{\pgfqpoint{1.172881in}{0.927756in}}%
\pgfpathlineto{\pgfqpoint{1.171805in}{0.925530in}}%
\pgfpathlineto{\pgfqpoint{1.168481in}{0.918754in}}%
\pgfpathlineto{\pgfqpoint{1.162074in}{0.910850in}}%
\pgfpathlineto{\pgfqpoint{1.161079in}{0.909753in}}%
\pgfpathlineto{\pgfqpoint{1.152342in}{0.902940in}}%
\pgfpathlineto{\pgfqpoint{1.148784in}{0.900751in}}%
\pgfpathlineto{\pgfqpoint{1.142611in}{0.897783in}}%
\pgfpathlineto{\pgfqpoint{1.132879in}{0.894788in}}%
\pgfpathlineto{\pgfqpoint{1.123148in}{0.893758in}}%
\pgfpathlineto{\pgfqpoint{1.113416in}{0.894788in}}%
\pgfpathlineto{\pgfqpoint{1.103685in}{0.897783in}}%
\pgfpathclose%
\pgfusepath{fill}%
\end{pgfscope}%
\begin{pgfscope}%
\pgfpathrectangle{\pgfqpoint{0.150000in}{0.549691in}}{\pgfqpoint{1.946296in}{1.800309in}}%
\pgfusepath{clip}%
\pgfsetbuttcap%
\pgfsetroundjoin%
\definecolor{currentfill}{rgb}{0.250980,0.250980,0.250980}%
\pgfsetfillcolor{currentfill}%
\pgfsetlinewidth{0.000000pt}%
\definecolor{currentstroke}{rgb}{0.000000,0.000000,0.000000}%
\pgfsetstrokecolor{currentstroke}%
\pgfsetdash{}{0pt}%
\pgfpathmoveto{\pgfqpoint{1.103685in}{0.674910in}}%
\pgfpathlineto{\pgfqpoint{1.113416in}{0.673017in}}%
\pgfpathlineto{\pgfqpoint{1.123148in}{0.672366in}}%
\pgfpathlineto{\pgfqpoint{1.132879in}{0.673017in}}%
\pgfpathlineto{\pgfqpoint{1.142611in}{0.674910in}}%
\pgfpathlineto{\pgfqpoint{1.145252in}{0.675713in}}%
\pgfpathlineto{\pgfqpoint{1.152342in}{0.677794in}}%
\pgfpathlineto{\pgfqpoint{1.162074in}{0.681415in}}%
\pgfpathlineto{\pgfqpoint{1.169998in}{0.684714in}}%
\pgfpathlineto{\pgfqpoint{1.171805in}{0.685472in}}%
\pgfpathlineto{\pgfqpoint{1.181537in}{0.689613in}}%
\pgfpathlineto{\pgfqpoint{1.191268in}{0.693443in}}%
\pgfpathlineto{\pgfqpoint{1.192102in}{0.693716in}}%
\pgfpathlineto{\pgfqpoint{1.201000in}{0.696774in}}%
\pgfpathlineto{\pgfqpoint{1.210731in}{0.699160in}}%
\pgfpathlineto{\pgfqpoint{1.220463in}{0.700407in}}%
\pgfpathlineto{\pgfqpoint{1.230194in}{0.700453in}}%
\pgfpathlineto{\pgfqpoint{1.239926in}{0.699360in}}%
\pgfpathlineto{\pgfqpoint{1.249657in}{0.697308in}}%
\pgfpathlineto{\pgfqpoint{1.259389in}{0.694576in}}%
\pgfpathlineto{\pgfqpoint{1.262127in}{0.693716in}}%
\pgfpathlineto{\pgfqpoint{1.269120in}{0.691624in}}%
\pgfpathlineto{\pgfqpoint{1.278852in}{0.688775in}}%
\pgfpathlineto{\pgfqpoint{1.288583in}{0.686360in}}%
\pgfpathlineto{\pgfqpoint{1.298315in}{0.684715in}}%
\pgfpathlineto{\pgfqpoint{1.298322in}{0.684714in}}%
\pgfpathlineto{\pgfqpoint{1.308046in}{0.684109in}}%
\pgfpathlineto{\pgfqpoint{1.317778in}{0.684703in}}%
\pgfpathlineto{\pgfqpoint{1.317838in}{0.684714in}}%
\pgfpathlineto{\pgfqpoint{1.327509in}{0.686566in}}%
\pgfpathlineto{\pgfqpoint{1.337240in}{0.689637in}}%
\pgfpathlineto{\pgfqpoint{1.346906in}{0.693716in}}%
\pgfpathlineto{\pgfqpoint{1.346972in}{0.693745in}}%
\pgfpathlineto{\pgfqpoint{1.356703in}{0.698868in}}%
\pgfpathlineto{\pgfqpoint{1.363418in}{0.702717in}}%
\pgfpathlineto{\pgfqpoint{1.366435in}{0.704617in}}%
\pgfpathlineto{\pgfqpoint{1.376166in}{0.710818in}}%
\pgfpathlineto{\pgfqpoint{1.377668in}{0.711719in}}%
\pgfpathlineto{\pgfqpoint{1.385898in}{0.717416in}}%
\pgfpathlineto{\pgfqpoint{1.391370in}{0.720720in}}%
\pgfpathlineto{\pgfqpoint{1.395629in}{0.723851in}}%
\pgfpathlineto{\pgfqpoint{1.405361in}{0.729525in}}%
\pgfpathlineto{\pgfqpoint{1.405851in}{0.729722in}}%
\pgfpathlineto{\pgfqpoint{1.415092in}{0.734540in}}%
\pgfpathlineto{\pgfqpoint{1.424824in}{0.737214in}}%
\pgfpathlineto{\pgfqpoint{1.434555in}{0.737613in}}%
\pgfpathlineto{\pgfqpoint{1.444287in}{0.736097in}}%
\pgfpathlineto{\pgfqpoint{1.454018in}{0.733223in}}%
\pgfpathlineto{\pgfqpoint{1.463682in}{0.729722in}}%
\pgfpathlineto{\pgfqpoint{1.463750in}{0.729703in}}%
\pgfpathlineto{\pgfqpoint{1.473481in}{0.727081in}}%
\pgfpathlineto{\pgfqpoint{1.483213in}{0.725161in}}%
\pgfpathlineto{\pgfqpoint{1.492944in}{0.724489in}}%
\pgfpathlineto{\pgfqpoint{1.502676in}{0.725495in}}%
\pgfpathlineto{\pgfqpoint{1.512407in}{0.728457in}}%
\pgfpathlineto{\pgfqpoint{1.514863in}{0.729722in}}%
\pgfpathlineto{\pgfqpoint{1.522139in}{0.734577in}}%
\pgfpathlineto{\pgfqpoint{1.526609in}{0.738724in}}%
\pgfpathlineto{\pgfqpoint{1.531870in}{0.745516in}}%
\pgfpathlineto{\pgfqpoint{1.533249in}{0.747725in}}%
\pgfpathlineto{\pgfqpoint{1.536980in}{0.756727in}}%
\pgfpathlineto{\pgfqpoint{1.539307in}{0.765728in}}%
\pgfpathlineto{\pgfqpoint{1.540844in}{0.774730in}}%
\pgfpathlineto{\pgfqpoint{1.541602in}{0.779518in}}%
\pgfpathlineto{\pgfqpoint{1.542184in}{0.783731in}}%
\pgfpathlineto{\pgfqpoint{1.543924in}{0.792733in}}%
\pgfpathlineto{\pgfqpoint{1.546694in}{0.801734in}}%
\pgfpathlineto{\pgfqpoint{1.550919in}{0.810736in}}%
\pgfpathlineto{\pgfqpoint{1.551333in}{0.811361in}}%
\pgfpathlineto{\pgfqpoint{1.556505in}{0.819737in}}%
\pgfpathlineto{\pgfqpoint{1.561064in}{0.825392in}}%
\pgfpathlineto{\pgfqpoint{1.563738in}{0.828739in}}%
\pgfpathlineto{\pgfqpoint{1.570796in}{0.835968in}}%
\pgfpathlineto{\pgfqpoint{1.572612in}{0.837741in}}%
\pgfpathlineto{\pgfqpoint{1.580527in}{0.844440in}}%
\pgfpathlineto{\pgfqpoint{1.583558in}{0.846742in}}%
\pgfpathlineto{\pgfqpoint{1.590259in}{0.851393in}}%
\pgfpathlineto{\pgfqpoint{1.597735in}{0.855744in}}%
\pgfpathlineto{\pgfqpoint{1.599990in}{0.857004in}}%
\pgfpathlineto{\pgfqpoint{1.609722in}{0.861196in}}%
\pgfpathlineto{\pgfqpoint{1.619453in}{0.864075in}}%
\pgfpathlineto{\pgfqpoint{1.623447in}{0.864745in}}%
\pgfpathlineto{\pgfqpoint{1.629185in}{0.865715in}}%
\pgfpathlineto{\pgfqpoint{1.638916in}{0.866304in}}%
\pgfpathlineto{\pgfqpoint{1.648648in}{0.866143in}}%
\pgfpathlineto{\pgfqpoint{1.658379in}{0.865623in}}%
\pgfpathlineto{\pgfqpoint{1.668111in}{0.865173in}}%
\pgfpathlineto{\pgfqpoint{1.677842in}{0.865227in}}%
\pgfpathlineto{\pgfqpoint{1.687574in}{0.866171in}}%
\pgfpathlineto{\pgfqpoint{1.697305in}{0.868316in}}%
\pgfpathlineto{\pgfqpoint{1.707037in}{0.871859in}}%
\pgfpathlineto{\pgfqpoint{1.710705in}{0.873747in}}%
\pgfpathlineto{\pgfqpoint{1.716768in}{0.877053in}}%
\pgfpathlineto{\pgfqpoint{1.724936in}{0.882748in}}%
\pgfpathlineto{\pgfqpoint{1.726500in}{0.883967in}}%
\pgfpathlineto{\pgfqpoint{1.734914in}{0.891750in}}%
\pgfpathlineto{\pgfqpoint{1.736231in}{0.893197in}}%
\pgfpathlineto{\pgfqpoint{1.742389in}{0.900751in}}%
\pgfpathlineto{\pgfqpoint{1.745963in}{0.906360in}}%
\pgfpathlineto{\pgfqpoint{1.748004in}{0.909753in}}%
\pgfpathlineto{\pgfqpoint{1.751834in}{0.918754in}}%
\pgfpathlineto{\pgfqpoint{1.754153in}{0.927756in}}%
\pgfpathlineto{\pgfqpoint{1.755174in}{0.936758in}}%
\pgfpathlineto{\pgfqpoint{1.755231in}{0.945759in}}%
\pgfpathlineto{\pgfqpoint{1.754746in}{0.954761in}}%
\pgfpathlineto{\pgfqpoint{1.754183in}{0.963762in}}%
\pgfpathlineto{\pgfqpoint{1.754009in}{0.972764in}}%
\pgfpathlineto{\pgfqpoint{1.754645in}{0.981765in}}%
\pgfpathlineto{\pgfqpoint{1.755694in}{0.987072in}}%
\pgfpathlineto{\pgfqpoint{1.756419in}{0.990767in}}%
\pgfpathlineto{\pgfqpoint{1.759531in}{0.999768in}}%
\pgfpathlineto{\pgfqpoint{1.764063in}{1.008770in}}%
\pgfpathlineto{\pgfqpoint{1.765426in}{1.010856in}}%
\pgfpathlineto{\pgfqpoint{1.770129in}{1.017771in}}%
\pgfpathlineto{\pgfqpoint{1.775157in}{1.023969in}}%
\pgfpathlineto{\pgfqpoint{1.777645in}{1.026773in}}%
\pgfpathlineto{\pgfqpoint{1.784889in}{1.034095in}}%
\pgfpathlineto{\pgfqpoint{1.786805in}{1.035775in}}%
\pgfpathlineto{\pgfqpoint{1.794620in}{1.042303in}}%
\pgfpathlineto{\pgfqpoint{1.798238in}{1.044776in}}%
\pgfpathlineto{\pgfqpoint{1.804351in}{1.048994in}}%
\pgfpathlineto{\pgfqpoint{1.813407in}{1.053778in}}%
\pgfpathlineto{\pgfqpoint{1.814083in}{1.054160in}}%
\pgfpathlineto{\pgfqpoint{1.823814in}{1.058068in}}%
\pgfpathlineto{\pgfqpoint{1.833546in}{1.060631in}}%
\pgfpathlineto{\pgfqpoint{1.843277in}{1.062240in}}%
\pgfpathlineto{\pgfqpoint{1.847832in}{1.062779in}}%
\pgfpathlineto{\pgfqpoint{1.853009in}{1.063480in}}%
\pgfpathlineto{\pgfqpoint{1.862740in}{1.064901in}}%
\pgfpathlineto{\pgfqpoint{1.872472in}{1.067054in}}%
\pgfpathlineto{\pgfqpoint{1.882203in}{1.070505in}}%
\pgfpathlineto{\pgfqpoint{1.884592in}{1.071781in}}%
\pgfpathlineto{\pgfqpoint{1.891935in}{1.076647in}}%
\pgfpathlineto{\pgfqpoint{1.896417in}{1.080782in}}%
\pgfpathlineto{\pgfqpoint{1.901666in}{1.087512in}}%
\pgfpathlineto{\pgfqpoint{1.903034in}{1.089784in}}%
\pgfpathlineto{\pgfqpoint{1.906236in}{1.098785in}}%
\pgfpathlineto{\pgfqpoint{1.907324in}{1.107787in}}%
\pgfpathlineto{\pgfqpoint{1.906597in}{1.116788in}}%
\pgfpathlineto{\pgfqpoint{1.904522in}{1.125790in}}%
\pgfpathlineto{\pgfqpoint{1.901687in}{1.134792in}}%
\pgfpathlineto{\pgfqpoint{1.901666in}{1.134854in}}%
\pgfpathlineto{\pgfqpoint{1.897881in}{1.143793in}}%
\pgfpathlineto{\pgfqpoint{1.894775in}{1.152795in}}%
\pgfpathlineto{\pgfqpoint{1.893135in}{1.161796in}}%
\pgfpathlineto{\pgfqpoint{1.893567in}{1.170798in}}%
\pgfpathlineto{\pgfqpoint{1.896458in}{1.179799in}}%
\pgfpathlineto{\pgfqpoint{1.901666in}{1.188348in}}%
\pgfpathlineto{\pgfqpoint{1.901879in}{1.188801in}}%
\pgfpathlineto{\pgfqpoint{1.908013in}{1.197802in}}%
\pgfpathlineto{\pgfqpoint{1.911398in}{1.201742in}}%
\pgfpathlineto{\pgfqpoint{1.914970in}{1.206804in}}%
\pgfpathlineto{\pgfqpoint{1.921129in}{1.214417in}}%
\pgfpathlineto{\pgfqpoint{1.922103in}{1.215805in}}%
\pgfpathlineto{\pgfqpoint{1.928807in}{1.224807in}}%
\pgfpathlineto{\pgfqpoint{1.930861in}{1.227597in}}%
\pgfpathlineto{\pgfqpoint{1.935022in}{1.233808in}}%
\pgfpathlineto{\pgfqpoint{1.940561in}{1.242810in}}%
\pgfpathlineto{\pgfqpoint{1.940592in}{1.242871in}}%
\pgfpathlineto{\pgfqpoint{1.945002in}{1.251812in}}%
\pgfpathlineto{\pgfqpoint{1.948322in}{1.260813in}}%
\pgfpathlineto{\pgfqpoint{1.950324in}{1.269758in}}%
\pgfpathlineto{\pgfqpoint{1.950336in}{1.269815in}}%
\pgfpathlineto{\pgfqpoint{1.950978in}{1.278816in}}%
\pgfpathlineto{\pgfqpoint{1.950324in}{1.287811in}}%
\pgfpathlineto{\pgfqpoint{1.950323in}{1.287818in}}%
\pgfpathlineto{\pgfqpoint{1.948544in}{1.296819in}}%
\pgfpathlineto{\pgfqpoint{1.945933in}{1.305821in}}%
\pgfpathlineto{\pgfqpoint{1.942853in}{1.314822in}}%
\pgfpathlineto{\pgfqpoint{1.940592in}{1.321291in}}%
\pgfpathlineto{\pgfqpoint{1.939662in}{1.323824in}}%
\pgfpathlineto{\pgfqpoint{1.936709in}{1.332825in}}%
\pgfpathlineto{\pgfqpoint{1.934491in}{1.341827in}}%
\pgfpathlineto{\pgfqpoint{1.933309in}{1.350829in}}%
\pgfpathlineto{\pgfqpoint{1.933359in}{1.359830in}}%
\pgfpathlineto{\pgfqpoint{1.934707in}{1.368832in}}%
\pgfpathlineto{\pgfqpoint{1.937286in}{1.377833in}}%
\pgfpathlineto{\pgfqpoint{1.940592in}{1.386063in}}%
\pgfpathlineto{\pgfqpoint{1.940887in}{1.386835in}}%
\pgfpathlineto{\pgfqpoint{1.945028in}{1.395836in}}%
\pgfpathlineto{\pgfqpoint{1.949505in}{1.404838in}}%
\pgfpathlineto{\pgfqpoint{1.950324in}{1.406509in}}%
\pgfpathlineto{\pgfqpoint{1.953891in}{1.413839in}}%
\pgfpathlineto{\pgfqpoint{1.957805in}{1.422841in}}%
\pgfpathlineto{\pgfqpoint{1.960055in}{1.429399in}}%
\pgfpathlineto{\pgfqpoint{1.960923in}{1.431842in}}%
\pgfpathlineto{\pgfqpoint{1.962969in}{1.440844in}}%
\pgfpathlineto{\pgfqpoint{1.963673in}{1.449846in}}%
\pgfpathlineto{\pgfqpoint{1.962969in}{1.458847in}}%
\pgfpathlineto{\pgfqpoint{1.960923in}{1.467849in}}%
\pgfpathlineto{\pgfqpoint{1.960055in}{1.470292in}}%
\pgfpathlineto{\pgfqpoint{1.957805in}{1.476850in}}%
\pgfpathlineto{\pgfqpoint{1.953891in}{1.485852in}}%
\pgfpathlineto{\pgfqpoint{1.950324in}{1.493182in}}%
\pgfpathlineto{\pgfqpoint{1.949505in}{1.494853in}}%
\pgfpathlineto{\pgfqpoint{1.945028in}{1.503855in}}%
\pgfpathlineto{\pgfqpoint{1.940887in}{1.512856in}}%
\pgfpathlineto{\pgfqpoint{1.940592in}{1.513628in}}%
\pgfpathlineto{\pgfqpoint{1.937286in}{1.521858in}}%
\pgfpathlineto{\pgfqpoint{1.934707in}{1.530859in}}%
\pgfpathlineto{\pgfqpoint{1.933359in}{1.539861in}}%
\pgfpathlineto{\pgfqpoint{1.933309in}{1.548863in}}%
\pgfpathlineto{\pgfqpoint{1.934491in}{1.557864in}}%
\pgfpathlineto{\pgfqpoint{1.936709in}{1.566866in}}%
\pgfpathlineto{\pgfqpoint{1.939662in}{1.575867in}}%
\pgfpathlineto{\pgfqpoint{1.940592in}{1.578400in}}%
\pgfpathlineto{\pgfqpoint{1.942853in}{1.584869in}}%
\pgfpathlineto{\pgfqpoint{1.945933in}{1.593870in}}%
\pgfpathlineto{\pgfqpoint{1.948544in}{1.602872in}}%
\pgfpathlineto{\pgfqpoint{1.950323in}{1.611873in}}%
\pgfpathlineto{\pgfqpoint{1.950324in}{1.611880in}}%
\pgfpathlineto{\pgfqpoint{1.950978in}{1.620875in}}%
\pgfpathlineto{\pgfqpoint{1.950336in}{1.629876in}}%
\pgfpathlineto{\pgfqpoint{1.950324in}{1.629933in}}%
\pgfpathlineto{\pgfqpoint{1.948322in}{1.638878in}}%
\pgfpathlineto{\pgfqpoint{1.945002in}{1.647880in}}%
\pgfpathlineto{\pgfqpoint{1.940592in}{1.656820in}}%
\pgfpathlineto{\pgfqpoint{1.940561in}{1.656881in}}%
\pgfpathlineto{\pgfqpoint{1.935022in}{1.665883in}}%
\pgfpathlineto{\pgfqpoint{1.930861in}{1.672094in}}%
\pgfpathlineto{\pgfqpoint{1.928807in}{1.674884in}}%
\pgfpathlineto{\pgfqpoint{1.922103in}{1.683886in}}%
\pgfpathlineto{\pgfqpoint{1.921129in}{1.685274in}}%
\pgfpathlineto{\pgfqpoint{1.914970in}{1.692887in}}%
\pgfpathlineto{\pgfqpoint{1.911398in}{1.697949in}}%
\pgfpathlineto{\pgfqpoint{1.908013in}{1.701889in}}%
\pgfpathlineto{\pgfqpoint{1.901879in}{1.710890in}}%
\pgfpathlineto{\pgfqpoint{1.901666in}{1.711343in}}%
\pgfpathlineto{\pgfqpoint{1.896458in}{1.719892in}}%
\pgfpathlineto{\pgfqpoint{1.893567in}{1.728893in}}%
\pgfpathlineto{\pgfqpoint{1.893135in}{1.737895in}}%
\pgfpathlineto{\pgfqpoint{1.894775in}{1.746897in}}%
\pgfpathlineto{\pgfqpoint{1.897881in}{1.755898in}}%
\pgfpathlineto{\pgfqpoint{1.901666in}{1.764837in}}%
\pgfpathlineto{\pgfqpoint{1.901687in}{1.764900in}}%
\pgfpathlineto{\pgfqpoint{1.904522in}{1.773901in}}%
\pgfpathlineto{\pgfqpoint{1.906597in}{1.782903in}}%
\pgfpathlineto{\pgfqpoint{1.907324in}{1.791904in}}%
\pgfpathlineto{\pgfqpoint{1.906236in}{1.800906in}}%
\pgfpathlineto{\pgfqpoint{1.903034in}{1.809907in}}%
\pgfpathlineto{\pgfqpoint{1.901666in}{1.812179in}}%
\pgfpathlineto{\pgfqpoint{1.896417in}{1.818909in}}%
\pgfpathlineto{\pgfqpoint{1.891935in}{1.823044in}}%
\pgfpathlineto{\pgfqpoint{1.884592in}{1.827910in}}%
\pgfpathlineto{\pgfqpoint{1.882203in}{1.829186in}}%
\pgfpathlineto{\pgfqpoint{1.872472in}{1.832637in}}%
\pgfpathlineto{\pgfqpoint{1.862740in}{1.834790in}}%
\pgfpathlineto{\pgfqpoint{1.853009in}{1.836211in}}%
\pgfpathlineto{\pgfqpoint{1.847832in}{1.836912in}}%
\pgfpathlineto{\pgfqpoint{1.843277in}{1.837451in}}%
\pgfpathlineto{\pgfqpoint{1.833546in}{1.839060in}}%
\pgfpathlineto{\pgfqpoint{1.823814in}{1.841623in}}%
\pgfpathlineto{\pgfqpoint{1.814083in}{1.845531in}}%
\pgfpathlineto{\pgfqpoint{1.813407in}{1.845914in}}%
\pgfpathlineto{\pgfqpoint{1.804351in}{1.850697in}}%
\pgfpathlineto{\pgfqpoint{1.798238in}{1.854915in}}%
\pgfpathlineto{\pgfqpoint{1.794620in}{1.857388in}}%
\pgfpathlineto{\pgfqpoint{1.786805in}{1.863917in}}%
\pgfpathlineto{\pgfqpoint{1.784889in}{1.865596in}}%
\pgfpathlineto{\pgfqpoint{1.777645in}{1.872918in}}%
\pgfpathlineto{\pgfqpoint{1.775157in}{1.875722in}}%
\pgfpathlineto{\pgfqpoint{1.770129in}{1.881920in}}%
\pgfpathlineto{\pgfqpoint{1.765426in}{1.888835in}}%
\pgfpathlineto{\pgfqpoint{1.764063in}{1.890921in}}%
\pgfpathlineto{\pgfqpoint{1.759531in}{1.899923in}}%
\pgfpathlineto{\pgfqpoint{1.756419in}{1.908924in}}%
\pgfpathlineto{\pgfqpoint{1.755694in}{1.912619in}}%
\pgfpathlineto{\pgfqpoint{1.754645in}{1.917926in}}%
\pgfpathlineto{\pgfqpoint{1.754009in}{1.926927in}}%
\pgfpathlineto{\pgfqpoint{1.754183in}{1.935929in}}%
\pgfpathlineto{\pgfqpoint{1.754746in}{1.944931in}}%
\pgfpathlineto{\pgfqpoint{1.755231in}{1.953932in}}%
\pgfpathlineto{\pgfqpoint{1.755174in}{1.962934in}}%
\pgfpathlineto{\pgfqpoint{1.754153in}{1.971935in}}%
\pgfpathlineto{\pgfqpoint{1.751834in}{1.980937in}}%
\pgfpathlineto{\pgfqpoint{1.748004in}{1.989938in}}%
\pgfpathlineto{\pgfqpoint{1.745963in}{1.993331in}}%
\pgfpathlineto{\pgfqpoint{1.742389in}{1.998940in}}%
\pgfpathlineto{\pgfqpoint{1.736231in}{2.006495in}}%
\pgfpathlineto{\pgfqpoint{1.734914in}{2.007941in}}%
\pgfpathlineto{\pgfqpoint{1.726500in}{2.015724in}}%
\pgfpathlineto{\pgfqpoint{1.724936in}{2.016943in}}%
\pgfpathlineto{\pgfqpoint{1.716768in}{2.022639in}}%
\pgfpathlineto{\pgfqpoint{1.710705in}{2.025944in}}%
\pgfpathlineto{\pgfqpoint{1.707037in}{2.027832in}}%
\pgfpathlineto{\pgfqpoint{1.697305in}{2.031375in}}%
\pgfpathlineto{\pgfqpoint{1.687574in}{2.033520in}}%
\pgfpathlineto{\pgfqpoint{1.677842in}{2.034465in}}%
\pgfpathlineto{\pgfqpoint{1.668111in}{2.034518in}}%
\pgfpathlineto{\pgfqpoint{1.658379in}{2.034069in}}%
\pgfpathlineto{\pgfqpoint{1.648648in}{2.033548in}}%
\pgfpathlineto{\pgfqpoint{1.638916in}{2.033387in}}%
\pgfpathlineto{\pgfqpoint{1.629185in}{2.033976in}}%
\pgfpathlineto{\pgfqpoint{1.623447in}{2.034946in}}%
\pgfpathlineto{\pgfqpoint{1.619453in}{2.035616in}}%
\pgfpathlineto{\pgfqpoint{1.609722in}{2.038495in}}%
\pgfpathlineto{\pgfqpoint{1.599990in}{2.042687in}}%
\pgfpathlineto{\pgfqpoint{1.597735in}{2.043947in}}%
\pgfpathlineto{\pgfqpoint{1.590259in}{2.048298in}}%
\pgfpathlineto{\pgfqpoint{1.583558in}{2.052949in}}%
\pgfpathlineto{\pgfqpoint{1.580527in}{2.055251in}}%
\pgfpathlineto{\pgfqpoint{1.572612in}{2.061951in}}%
\pgfpathlineto{\pgfqpoint{1.570796in}{2.063723in}}%
\pgfpathlineto{\pgfqpoint{1.563738in}{2.070952in}}%
\pgfpathlineto{\pgfqpoint{1.561064in}{2.074299in}}%
\pgfpathlineto{\pgfqpoint{1.556505in}{2.079954in}}%
\pgfpathlineto{\pgfqpoint{1.551333in}{2.088330in}}%
\pgfpathlineto{\pgfqpoint{1.550919in}{2.088955in}}%
\pgfpathlineto{\pgfqpoint{1.546694in}{2.097957in}}%
\pgfpathlineto{\pgfqpoint{1.543924in}{2.106958in}}%
\pgfpathlineto{\pgfqpoint{1.542184in}{2.115960in}}%
\pgfpathlineto{\pgfqpoint{1.541602in}{2.120173in}}%
\pgfpathlineto{\pgfqpoint{1.540844in}{2.124961in}}%
\pgfpathlineto{\pgfqpoint{1.539307in}{2.133963in}}%
\pgfpathlineto{\pgfqpoint{1.536980in}{2.142964in}}%
\pgfpathlineto{\pgfqpoint{1.533249in}{2.151966in}}%
\pgfpathlineto{\pgfqpoint{1.531870in}{2.154176in}}%
\pgfpathlineto{\pgfqpoint{1.526609in}{2.160968in}}%
\pgfpathlineto{\pgfqpoint{1.522139in}{2.165114in}}%
\pgfpathlineto{\pgfqpoint{1.514863in}{2.169969in}}%
\pgfpathlineto{\pgfqpoint{1.512407in}{2.171234in}}%
\pgfpathlineto{\pgfqpoint{1.502676in}{2.174196in}}%
\pgfpathlineto{\pgfqpoint{1.492944in}{2.175202in}}%
\pgfpathlineto{\pgfqpoint{1.483213in}{2.174530in}}%
\pgfpathlineto{\pgfqpoint{1.473481in}{2.172610in}}%
\pgfpathlineto{\pgfqpoint{1.463750in}{2.169988in}}%
\pgfpathlineto{\pgfqpoint{1.463682in}{2.169969in}}%
\pgfpathlineto{\pgfqpoint{1.454018in}{2.166468in}}%
\pgfpathlineto{\pgfqpoint{1.444287in}{2.163594in}}%
\pgfpathlineto{\pgfqpoint{1.434555in}{2.162078in}}%
\pgfpathlineto{\pgfqpoint{1.424824in}{2.162477in}}%
\pgfpathlineto{\pgfqpoint{1.415092in}{2.165151in}}%
\pgfpathlineto{\pgfqpoint{1.405851in}{2.169969in}}%
\pgfpathlineto{\pgfqpoint{1.405361in}{2.170166in}}%
\pgfpathlineto{\pgfqpoint{1.395629in}{2.175840in}}%
\pgfpathlineto{\pgfqpoint{1.391370in}{2.178971in}}%
\pgfpathlineto{\pgfqpoint{1.385898in}{2.182275in}}%
\pgfpathlineto{\pgfqpoint{1.377668in}{2.187972in}}%
\pgfpathlineto{\pgfqpoint{1.376166in}{2.188873in}}%
\pgfpathlineto{\pgfqpoint{1.366435in}{2.195074in}}%
\pgfpathlineto{\pgfqpoint{1.363418in}{2.196974in}}%
\pgfpathlineto{\pgfqpoint{1.356703in}{2.200823in}}%
\pgfpathlineto{\pgfqpoint{1.346972in}{2.205946in}}%
\pgfpathlineto{\pgfqpoint{1.346906in}{2.205975in}}%
\pgfpathlineto{\pgfqpoint{1.337240in}{2.210054in}}%
\pgfpathlineto{\pgfqpoint{1.327509in}{2.213125in}}%
\pgfpathlineto{\pgfqpoint{1.317838in}{2.214977in}}%
\pgfpathlineto{\pgfqpoint{1.317778in}{2.214988in}}%
\pgfpathlineto{\pgfqpoint{1.308046in}{2.215582in}}%
\pgfpathlineto{\pgfqpoint{1.298322in}{2.214977in}}%
\pgfpathlineto{\pgfqpoint{1.298315in}{2.214976in}}%
\pgfpathlineto{\pgfqpoint{1.288583in}{2.213331in}}%
\pgfpathlineto{\pgfqpoint{1.278852in}{2.210916in}}%
\pgfpathlineto{\pgfqpoint{1.269120in}{2.208067in}}%
\pgfpathlineto{\pgfqpoint{1.262127in}{2.205975in}}%
\pgfpathlineto{\pgfqpoint{1.259389in}{2.205115in}}%
\pgfpathlineto{\pgfqpoint{1.249657in}{2.202384in}}%
\pgfpathlineto{\pgfqpoint{1.239926in}{2.200331in}}%
\pgfpathlineto{\pgfqpoint{1.230194in}{2.199238in}}%
\pgfpathlineto{\pgfqpoint{1.220463in}{2.199284in}}%
\pgfpathlineto{\pgfqpoint{1.210731in}{2.200531in}}%
\pgfpathlineto{\pgfqpoint{1.201000in}{2.202917in}}%
\pgfpathlineto{\pgfqpoint{1.192102in}{2.205975in}}%
\pgfpathlineto{\pgfqpoint{1.191268in}{2.206248in}}%
\pgfpathlineto{\pgfqpoint{1.181537in}{2.210078in}}%
\pgfpathlineto{\pgfqpoint{1.171805in}{2.214219in}}%
\pgfpathlineto{\pgfqpoint{1.169998in}{2.214977in}}%
\pgfpathlineto{\pgfqpoint{1.162074in}{2.218276in}}%
\pgfpathlineto{\pgfqpoint{1.152342in}{2.221897in}}%
\pgfpathlineto{\pgfqpoint{1.145252in}{2.223978in}}%
\pgfpathlineto{\pgfqpoint{1.142611in}{2.224781in}}%
\pgfpathlineto{\pgfqpoint{1.132879in}{2.226674in}}%
\pgfpathlineto{\pgfqpoint{1.123148in}{2.227325in}}%
\pgfpathlineto{\pgfqpoint{1.113416in}{2.226674in}}%
\pgfpathlineto{\pgfqpoint{1.103685in}{2.224781in}}%
\pgfpathlineto{\pgfqpoint{1.101043in}{2.223978in}}%
\pgfpathlineto{\pgfqpoint{1.093953in}{2.221897in}}%
\pgfpathlineto{\pgfqpoint{1.084222in}{2.218276in}}%
\pgfpathlineto{\pgfqpoint{1.076297in}{2.214977in}}%
\pgfpathlineto{\pgfqpoint{1.074491in}{2.214219in}}%
\pgfpathlineto{\pgfqpoint{1.064759in}{2.210078in}}%
\pgfpathlineto{\pgfqpoint{1.055028in}{2.206248in}}%
\pgfpathlineto{\pgfqpoint{1.054193in}{2.205975in}}%
\pgfpathlineto{\pgfqpoint{1.045296in}{2.202917in}}%
\pgfpathlineto{\pgfqpoint{1.035565in}{2.200531in}}%
\pgfpathlineto{\pgfqpoint{1.025833in}{2.199284in}}%
\pgfpathlineto{\pgfqpoint{1.016102in}{2.199238in}}%
\pgfpathlineto{\pgfqpoint{1.006370in}{2.200331in}}%
\pgfpathlineto{\pgfqpoint{0.996639in}{2.202384in}}%
\pgfpathlineto{\pgfqpoint{0.986907in}{2.205115in}}%
\pgfpathlineto{\pgfqpoint{0.984169in}{2.205975in}}%
\pgfpathlineto{\pgfqpoint{0.977176in}{2.208067in}}%
\pgfpathlineto{\pgfqpoint{0.967444in}{2.210916in}}%
\pgfpathlineto{\pgfqpoint{0.957713in}{2.213331in}}%
\pgfpathlineto{\pgfqpoint{0.947981in}{2.214976in}}%
\pgfpathlineto{\pgfqpoint{0.947974in}{2.214977in}}%
\pgfpathlineto{\pgfqpoint{0.938250in}{2.215582in}}%
\pgfpathlineto{\pgfqpoint{0.928518in}{2.214988in}}%
\pgfpathlineto{\pgfqpoint{0.928458in}{2.214977in}}%
\pgfpathlineto{\pgfqpoint{0.918787in}{2.213125in}}%
\pgfpathlineto{\pgfqpoint{0.909055in}{2.210054in}}%
\pgfpathlineto{\pgfqpoint{0.899390in}{2.205975in}}%
\pgfpathlineto{\pgfqpoint{0.899324in}{2.205946in}}%
\pgfpathlineto{\pgfqpoint{0.889592in}{2.200823in}}%
\pgfpathlineto{\pgfqpoint{0.882878in}{2.196974in}}%
\pgfpathlineto{\pgfqpoint{0.879861in}{2.195074in}}%
\pgfpathlineto{\pgfqpoint{0.870129in}{2.188873in}}%
\pgfpathlineto{\pgfqpoint{0.868628in}{2.187972in}}%
\pgfpathlineto{\pgfqpoint{0.860398in}{2.182275in}}%
\pgfpathlineto{\pgfqpoint{0.854926in}{2.178971in}}%
\pgfpathlineto{\pgfqpoint{0.850667in}{2.175840in}}%
\pgfpathlineto{\pgfqpoint{0.840935in}{2.170166in}}%
\pgfpathlineto{\pgfqpoint{0.840445in}{2.169969in}}%
\pgfpathlineto{\pgfqpoint{0.831204in}{2.165151in}}%
\pgfpathlineto{\pgfqpoint{0.821472in}{2.162477in}}%
\pgfpathlineto{\pgfqpoint{0.811741in}{2.162078in}}%
\pgfpathlineto{\pgfqpoint{0.802009in}{2.163594in}}%
\pgfpathlineto{\pgfqpoint{0.792278in}{2.166468in}}%
\pgfpathlineto{\pgfqpoint{0.782614in}{2.169969in}}%
\pgfpathlineto{\pgfqpoint{0.782546in}{2.169988in}}%
\pgfpathlineto{\pgfqpoint{0.772815in}{2.172610in}}%
\pgfpathlineto{\pgfqpoint{0.763083in}{2.174530in}}%
\pgfpathlineto{\pgfqpoint{0.753352in}{2.175202in}}%
\pgfpathlineto{\pgfqpoint{0.743620in}{2.174196in}}%
\pgfpathlineto{\pgfqpoint{0.733889in}{2.171234in}}%
\pgfpathlineto{\pgfqpoint{0.731433in}{2.169969in}}%
\pgfpathlineto{\pgfqpoint{0.724157in}{2.165114in}}%
\pgfpathlineto{\pgfqpoint{0.719687in}{2.160968in}}%
\pgfpathlineto{\pgfqpoint{0.714426in}{2.154176in}}%
\pgfpathlineto{\pgfqpoint{0.713046in}{2.151966in}}%
\pgfpathlineto{\pgfqpoint{0.709316in}{2.142964in}}%
\pgfpathlineto{\pgfqpoint{0.706989in}{2.133963in}}%
\pgfpathlineto{\pgfqpoint{0.705452in}{2.124961in}}%
\pgfpathlineto{\pgfqpoint{0.704694in}{2.120173in}}%
\pgfpathlineto{\pgfqpoint{0.704112in}{2.115960in}}%
\pgfpathlineto{\pgfqpoint{0.702372in}{2.106958in}}%
\pgfpathlineto{\pgfqpoint{0.699602in}{2.097957in}}%
\pgfpathlineto{\pgfqpoint{0.695376in}{2.088955in}}%
\pgfpathlineto{\pgfqpoint{0.694963in}{2.088330in}}%
\pgfpathlineto{\pgfqpoint{0.689791in}{2.079954in}}%
\pgfpathlineto{\pgfqpoint{0.685231in}{2.074299in}}%
\pgfpathlineto{\pgfqpoint{0.682558in}{2.070952in}}%
\pgfpathlineto{\pgfqpoint{0.675500in}{2.063723in}}%
\pgfpathlineto{\pgfqpoint{0.673684in}{2.061951in}}%
\pgfpathlineto{\pgfqpoint{0.665768in}{2.055251in}}%
\pgfpathlineto{\pgfqpoint{0.662737in}{2.052949in}}%
\pgfpathlineto{\pgfqpoint{0.656037in}{2.048298in}}%
\pgfpathlineto{\pgfqpoint{0.648561in}{2.043947in}}%
\pgfpathlineto{\pgfqpoint{0.646305in}{2.042687in}}%
\pgfpathlineto{\pgfqpoint{0.636574in}{2.038495in}}%
\pgfpathlineto{\pgfqpoint{0.626842in}{2.035616in}}%
\pgfpathlineto{\pgfqpoint{0.622848in}{2.034946in}}%
\pgfpathlineto{\pgfqpoint{0.617111in}{2.033976in}}%
\pgfpathlineto{\pgfqpoint{0.607380in}{2.033387in}}%
\pgfpathlineto{\pgfqpoint{0.597648in}{2.033548in}}%
\pgfpathlineto{\pgfqpoint{0.587917in}{2.034069in}}%
\pgfpathlineto{\pgfqpoint{0.578185in}{2.034518in}}%
\pgfpathlineto{\pgfqpoint{0.568454in}{2.034465in}}%
\pgfpathlineto{\pgfqpoint{0.558722in}{2.033520in}}%
\pgfpathlineto{\pgfqpoint{0.548991in}{2.031375in}}%
\pgfpathlineto{\pgfqpoint{0.539259in}{2.027832in}}%
\pgfpathlineto{\pgfqpoint{0.535591in}{2.025944in}}%
\pgfpathlineto{\pgfqpoint{0.529528in}{2.022639in}}%
\pgfpathlineto{\pgfqpoint{0.521360in}{2.016943in}}%
\pgfpathlineto{\pgfqpoint{0.519796in}{2.015724in}}%
\pgfpathlineto{\pgfqpoint{0.511382in}{2.007941in}}%
\pgfpathlineto{\pgfqpoint{0.510065in}{2.006495in}}%
\pgfpathlineto{\pgfqpoint{0.503907in}{1.998940in}}%
\pgfpathlineto{\pgfqpoint{0.500333in}{1.993331in}}%
\pgfpathlineto{\pgfqpoint{0.498292in}{1.989938in}}%
\pgfpathlineto{\pgfqpoint{0.494462in}{1.980937in}}%
\pgfpathlineto{\pgfqpoint{0.492143in}{1.971935in}}%
\pgfpathlineto{\pgfqpoint{0.491122in}{1.962934in}}%
\pgfpathlineto{\pgfqpoint{0.491065in}{1.953932in}}%
\pgfpathlineto{\pgfqpoint{0.491550in}{1.944931in}}%
\pgfpathlineto{\pgfqpoint{0.492113in}{1.935929in}}%
\pgfpathlineto{\pgfqpoint{0.492287in}{1.926927in}}%
\pgfpathlineto{\pgfqpoint{0.491651in}{1.917926in}}%
\pgfpathlineto{\pgfqpoint{0.490602in}{1.912619in}}%
\pgfpathlineto{\pgfqpoint{0.489877in}{1.908924in}}%
\pgfpathlineto{\pgfqpoint{0.486765in}{1.899923in}}%
\pgfpathlineto{\pgfqpoint{0.482233in}{1.890921in}}%
\pgfpathlineto{\pgfqpoint{0.480870in}{1.888835in}}%
\pgfpathlineto{\pgfqpoint{0.476167in}{1.881920in}}%
\pgfpathlineto{\pgfqpoint{0.471139in}{1.875722in}}%
\pgfpathlineto{\pgfqpoint{0.468650in}{1.872918in}}%
\pgfpathlineto{\pgfqpoint{0.461407in}{1.865596in}}%
\pgfpathlineto{\pgfqpoint{0.459491in}{1.863917in}}%
\pgfpathlineto{\pgfqpoint{0.451676in}{1.857388in}}%
\pgfpathlineto{\pgfqpoint{0.448058in}{1.854915in}}%
\pgfpathlineto{\pgfqpoint{0.441944in}{1.850697in}}%
\pgfpathlineto{\pgfqpoint{0.432889in}{1.845914in}}%
\pgfpathlineto{\pgfqpoint{0.432213in}{1.845531in}}%
\pgfpathlineto{\pgfqpoint{0.422481in}{1.841623in}}%
\pgfpathlineto{\pgfqpoint{0.412750in}{1.839060in}}%
\pgfpathlineto{\pgfqpoint{0.403018in}{1.837451in}}%
\pgfpathlineto{\pgfqpoint{0.398464in}{1.836912in}}%
\pgfpathlineto{\pgfqpoint{0.393287in}{1.836211in}}%
\pgfpathlineto{\pgfqpoint{0.383556in}{1.834790in}}%
\pgfpathlineto{\pgfqpoint{0.373824in}{1.832637in}}%
\pgfpathlineto{\pgfqpoint{0.364093in}{1.829186in}}%
\pgfpathlineto{\pgfqpoint{0.361704in}{1.827910in}}%
\pgfpathlineto{\pgfqpoint{0.354361in}{1.823044in}}%
\pgfpathlineto{\pgfqpoint{0.349878in}{1.818909in}}%
\pgfpathlineto{\pgfqpoint{0.344630in}{1.812179in}}%
\pgfpathlineto{\pgfqpoint{0.343262in}{1.809907in}}%
\pgfpathlineto{\pgfqpoint{0.340060in}{1.800906in}}%
\pgfpathlineto{\pgfqpoint{0.338972in}{1.791904in}}%
\pgfpathlineto{\pgfqpoint{0.339699in}{1.782903in}}%
\pgfpathlineto{\pgfqpoint{0.341774in}{1.773901in}}%
\pgfpathlineto{\pgfqpoint{0.344609in}{1.764900in}}%
\pgfpathlineto{\pgfqpoint{0.344630in}{1.764837in}}%
\pgfpathlineto{\pgfqpoint{0.348415in}{1.755898in}}%
\pgfpathlineto{\pgfqpoint{0.351521in}{1.746897in}}%
\pgfpathlineto{\pgfqpoint{0.353161in}{1.737895in}}%
\pgfpathlineto{\pgfqpoint{0.352729in}{1.728893in}}%
\pgfpathlineto{\pgfqpoint{0.349838in}{1.719892in}}%
\pgfpathlineto{\pgfqpoint{0.344630in}{1.711343in}}%
\pgfpathlineto{\pgfqpoint{0.344416in}{1.710890in}}%
\pgfpathlineto{\pgfqpoint{0.338283in}{1.701889in}}%
\pgfpathlineto{\pgfqpoint{0.334898in}{1.697949in}}%
\pgfpathlineto{\pgfqpoint{0.331326in}{1.692887in}}%
\pgfpathlineto{\pgfqpoint{0.325167in}{1.685274in}}%
\pgfpathlineto{\pgfqpoint{0.324192in}{1.683886in}}%
\pgfpathlineto{\pgfqpoint{0.317489in}{1.674884in}}%
\pgfpathlineto{\pgfqpoint{0.315435in}{1.672094in}}%
\pgfpathlineto{\pgfqpoint{0.311274in}{1.665883in}}%
\pgfpathlineto{\pgfqpoint{0.305735in}{1.656881in}}%
\pgfpathlineto{\pgfqpoint{0.305704in}{1.656820in}}%
\pgfpathlineto{\pgfqpoint{0.301294in}{1.647880in}}%
\pgfpathlineto{\pgfqpoint{0.297974in}{1.638878in}}%
\pgfpathlineto{\pgfqpoint{0.295972in}{1.629933in}}%
\pgfpathlineto{\pgfqpoint{0.295960in}{1.629876in}}%
\pgfpathlineto{\pgfqpoint{0.295318in}{1.620875in}}%
\pgfpathlineto{\pgfqpoint{0.295972in}{1.611880in}}%
\pgfpathlineto{\pgfqpoint{0.295973in}{1.611873in}}%
\pgfpathlineto{\pgfqpoint{0.297752in}{1.602872in}}%
\pgfpathlineto{\pgfqpoint{0.300363in}{1.593870in}}%
\pgfpathlineto{\pgfqpoint{0.303443in}{1.584869in}}%
\pgfpathlineto{\pgfqpoint{0.305704in}{1.578400in}}%
\pgfpathlineto{\pgfqpoint{0.306634in}{1.575867in}}%
\pgfpathlineto{\pgfqpoint{0.309587in}{1.566866in}}%
\pgfpathlineto{\pgfqpoint{0.311805in}{1.557864in}}%
\pgfpathlineto{\pgfqpoint{0.312987in}{1.548863in}}%
\pgfpathlineto{\pgfqpoint{0.312937in}{1.539861in}}%
\pgfpathlineto{\pgfqpoint{0.311589in}{1.530859in}}%
\pgfpathlineto{\pgfqpoint{0.309010in}{1.521858in}}%
\pgfpathlineto{\pgfqpoint{0.305704in}{1.513628in}}%
\pgfpathlineto{\pgfqpoint{0.305409in}{1.512856in}}%
\pgfpathlineto{\pgfqpoint{0.301268in}{1.503855in}}%
\pgfpathlineto{\pgfqpoint{0.296791in}{1.494853in}}%
\pgfpathlineto{\pgfqpoint{0.295972in}{1.493182in}}%
\pgfpathlineto{\pgfqpoint{0.292405in}{1.485852in}}%
\pgfpathlineto{\pgfqpoint{0.288491in}{1.476850in}}%
\pgfpathlineto{\pgfqpoint{0.286241in}{1.470292in}}%
\pgfpathlineto{\pgfqpoint{0.285373in}{1.467849in}}%
\pgfpathlineto{\pgfqpoint{0.283327in}{1.458847in}}%
\pgfpathlineto{\pgfqpoint{0.282623in}{1.449846in}}%
\pgfpathlineto{\pgfqpoint{0.283327in}{1.440844in}}%
\pgfpathlineto{\pgfqpoint{0.285373in}{1.431842in}}%
\pgfpathlineto{\pgfqpoint{0.286241in}{1.429399in}}%
\pgfpathlineto{\pgfqpoint{0.288491in}{1.422841in}}%
\pgfpathlineto{\pgfqpoint{0.292405in}{1.413839in}}%
\pgfpathlineto{\pgfqpoint{0.295972in}{1.406509in}}%
\pgfpathlineto{\pgfqpoint{0.296791in}{1.404838in}}%
\pgfpathlineto{\pgfqpoint{0.301268in}{1.395836in}}%
\pgfpathlineto{\pgfqpoint{0.305409in}{1.386835in}}%
\pgfpathlineto{\pgfqpoint{0.305704in}{1.386063in}}%
\pgfpathlineto{\pgfqpoint{0.309010in}{1.377833in}}%
\pgfpathlineto{\pgfqpoint{0.311589in}{1.368832in}}%
\pgfpathlineto{\pgfqpoint{0.312937in}{1.359830in}}%
\pgfpathlineto{\pgfqpoint{0.312987in}{1.350829in}}%
\pgfpathlineto{\pgfqpoint{0.311805in}{1.341827in}}%
\pgfpathlineto{\pgfqpoint{0.309587in}{1.332825in}}%
\pgfpathlineto{\pgfqpoint{0.306634in}{1.323824in}}%
\pgfpathlineto{\pgfqpoint{0.305704in}{1.321291in}}%
\pgfpathlineto{\pgfqpoint{0.303443in}{1.314822in}}%
\pgfpathlineto{\pgfqpoint{0.300363in}{1.305821in}}%
\pgfpathlineto{\pgfqpoint{0.297752in}{1.296819in}}%
\pgfpathlineto{\pgfqpoint{0.295973in}{1.287818in}}%
\pgfpathlineto{\pgfqpoint{0.295972in}{1.287811in}}%
\pgfpathlineto{\pgfqpoint{0.295318in}{1.278816in}}%
\pgfpathlineto{\pgfqpoint{0.295960in}{1.269815in}}%
\pgfpathlineto{\pgfqpoint{0.295972in}{1.269758in}}%
\pgfpathlineto{\pgfqpoint{0.297974in}{1.260813in}}%
\pgfpathlineto{\pgfqpoint{0.301294in}{1.251812in}}%
\pgfpathlineto{\pgfqpoint{0.305704in}{1.242871in}}%
\pgfpathlineto{\pgfqpoint{0.305735in}{1.242810in}}%
\pgfpathlineto{\pgfqpoint{0.311274in}{1.233808in}}%
\pgfpathlineto{\pgfqpoint{0.315435in}{1.227597in}}%
\pgfpathlineto{\pgfqpoint{0.317489in}{1.224807in}}%
\pgfpathlineto{\pgfqpoint{0.324192in}{1.215805in}}%
\pgfpathlineto{\pgfqpoint{0.325167in}{1.214417in}}%
\pgfpathlineto{\pgfqpoint{0.331326in}{1.206804in}}%
\pgfpathlineto{\pgfqpoint{0.334898in}{1.201742in}}%
\pgfpathlineto{\pgfqpoint{0.338283in}{1.197802in}}%
\pgfpathlineto{\pgfqpoint{0.344416in}{1.188801in}}%
\pgfpathlineto{\pgfqpoint{0.344630in}{1.188348in}}%
\pgfpathlineto{\pgfqpoint{0.349838in}{1.179799in}}%
\pgfpathlineto{\pgfqpoint{0.352729in}{1.170798in}}%
\pgfpathlineto{\pgfqpoint{0.353161in}{1.161796in}}%
\pgfpathlineto{\pgfqpoint{0.351521in}{1.152795in}}%
\pgfpathlineto{\pgfqpoint{0.348415in}{1.143793in}}%
\pgfpathlineto{\pgfqpoint{0.344630in}{1.134854in}}%
\pgfpathlineto{\pgfqpoint{0.344609in}{1.134792in}}%
\pgfpathlineto{\pgfqpoint{0.341774in}{1.125790in}}%
\pgfpathlineto{\pgfqpoint{0.339699in}{1.116788in}}%
\pgfpathlineto{\pgfqpoint{0.338972in}{1.107787in}}%
\pgfpathlineto{\pgfqpoint{0.340060in}{1.098785in}}%
\pgfpathlineto{\pgfqpoint{0.343262in}{1.089784in}}%
\pgfpathlineto{\pgfqpoint{0.344630in}{1.087512in}}%
\pgfpathlineto{\pgfqpoint{0.349878in}{1.080782in}}%
\pgfpathlineto{\pgfqpoint{0.354361in}{1.076647in}}%
\pgfpathlineto{\pgfqpoint{0.361704in}{1.071781in}}%
\pgfpathlineto{\pgfqpoint{0.364093in}{1.070505in}}%
\pgfpathlineto{\pgfqpoint{0.373824in}{1.067054in}}%
\pgfpathlineto{\pgfqpoint{0.383556in}{1.064901in}}%
\pgfpathlineto{\pgfqpoint{0.393287in}{1.063480in}}%
\pgfpathlineto{\pgfqpoint{0.398464in}{1.062779in}}%
\pgfpathlineto{\pgfqpoint{0.403018in}{1.062240in}}%
\pgfpathlineto{\pgfqpoint{0.412750in}{1.060631in}}%
\pgfpathlineto{\pgfqpoint{0.422481in}{1.058068in}}%
\pgfpathlineto{\pgfqpoint{0.432213in}{1.054160in}}%
\pgfpathlineto{\pgfqpoint{0.432889in}{1.053778in}}%
\pgfpathlineto{\pgfqpoint{0.441944in}{1.048994in}}%
\pgfpathlineto{\pgfqpoint{0.448058in}{1.044776in}}%
\pgfpathlineto{\pgfqpoint{0.451676in}{1.042303in}}%
\pgfpathlineto{\pgfqpoint{0.459491in}{1.035775in}}%
\pgfpathlineto{\pgfqpoint{0.461407in}{1.034095in}}%
\pgfpathlineto{\pgfqpoint{0.468650in}{1.026773in}}%
\pgfpathlineto{\pgfqpoint{0.471139in}{1.023969in}}%
\pgfpathlineto{\pgfqpoint{0.476167in}{1.017771in}}%
\pgfpathlineto{\pgfqpoint{0.480870in}{1.010856in}}%
\pgfpathlineto{\pgfqpoint{0.482233in}{1.008770in}}%
\pgfpathlineto{\pgfqpoint{0.486765in}{0.999768in}}%
\pgfpathlineto{\pgfqpoint{0.489877in}{0.990767in}}%
\pgfpathlineto{\pgfqpoint{0.490602in}{0.987072in}}%
\pgfpathlineto{\pgfqpoint{0.491651in}{0.981765in}}%
\pgfpathlineto{\pgfqpoint{0.492287in}{0.972764in}}%
\pgfpathlineto{\pgfqpoint{0.492113in}{0.963762in}}%
\pgfpathlineto{\pgfqpoint{0.491550in}{0.954761in}}%
\pgfpathlineto{\pgfqpoint{0.491065in}{0.945759in}}%
\pgfpathlineto{\pgfqpoint{0.491122in}{0.936758in}}%
\pgfpathlineto{\pgfqpoint{0.492143in}{0.927756in}}%
\pgfpathlineto{\pgfqpoint{0.494462in}{0.918754in}}%
\pgfpathlineto{\pgfqpoint{0.498292in}{0.909753in}}%
\pgfpathlineto{\pgfqpoint{0.500333in}{0.906360in}}%
\pgfpathlineto{\pgfqpoint{0.503907in}{0.900751in}}%
\pgfpathlineto{\pgfqpoint{0.510065in}{0.893197in}}%
\pgfpathlineto{\pgfqpoint{0.511382in}{0.891750in}}%
\pgfpathlineto{\pgfqpoint{0.519796in}{0.883967in}}%
\pgfpathlineto{\pgfqpoint{0.521360in}{0.882748in}}%
\pgfpathlineto{\pgfqpoint{0.529528in}{0.877053in}}%
\pgfpathlineto{\pgfqpoint{0.535591in}{0.873747in}}%
\pgfpathlineto{\pgfqpoint{0.539259in}{0.871859in}}%
\pgfpathlineto{\pgfqpoint{0.548991in}{0.868316in}}%
\pgfpathlineto{\pgfqpoint{0.558722in}{0.866171in}}%
\pgfpathlineto{\pgfqpoint{0.568454in}{0.865227in}}%
\pgfpathlineto{\pgfqpoint{0.578185in}{0.865173in}}%
\pgfpathlineto{\pgfqpoint{0.587917in}{0.865623in}}%
\pgfpathlineto{\pgfqpoint{0.597648in}{0.866143in}}%
\pgfpathlineto{\pgfqpoint{0.607380in}{0.866304in}}%
\pgfpathlineto{\pgfqpoint{0.617111in}{0.865715in}}%
\pgfpathlineto{\pgfqpoint{0.622848in}{0.864745in}}%
\pgfpathlineto{\pgfqpoint{0.626842in}{0.864075in}}%
\pgfpathlineto{\pgfqpoint{0.636574in}{0.861196in}}%
\pgfpathlineto{\pgfqpoint{0.646305in}{0.857004in}}%
\pgfpathlineto{\pgfqpoint{0.648561in}{0.855744in}}%
\pgfpathlineto{\pgfqpoint{0.656037in}{0.851393in}}%
\pgfpathlineto{\pgfqpoint{0.662737in}{0.846742in}}%
\pgfpathlineto{\pgfqpoint{0.665768in}{0.844440in}}%
\pgfpathlineto{\pgfqpoint{0.673684in}{0.837741in}}%
\pgfpathlineto{\pgfqpoint{0.675500in}{0.835968in}}%
\pgfpathlineto{\pgfqpoint{0.682558in}{0.828739in}}%
\pgfpathlineto{\pgfqpoint{0.685231in}{0.825392in}}%
\pgfpathlineto{\pgfqpoint{0.689791in}{0.819737in}}%
\pgfpathlineto{\pgfqpoint{0.694963in}{0.811361in}}%
\pgfpathlineto{\pgfqpoint{0.695376in}{0.810736in}}%
\pgfpathlineto{\pgfqpoint{0.699602in}{0.801734in}}%
\pgfpathlineto{\pgfqpoint{0.702372in}{0.792733in}}%
\pgfpathlineto{\pgfqpoint{0.704112in}{0.783731in}}%
\pgfpathlineto{\pgfqpoint{0.704694in}{0.779518in}}%
\pgfpathlineto{\pgfqpoint{0.705452in}{0.774730in}}%
\pgfpathlineto{\pgfqpoint{0.706989in}{0.765728in}}%
\pgfpathlineto{\pgfqpoint{0.709316in}{0.756727in}}%
\pgfpathlineto{\pgfqpoint{0.713046in}{0.747725in}}%
\pgfpathlineto{\pgfqpoint{0.714426in}{0.745516in}}%
\pgfpathlineto{\pgfqpoint{0.719687in}{0.738724in}}%
\pgfpathlineto{\pgfqpoint{0.724157in}{0.734577in}}%
\pgfpathlineto{\pgfqpoint{0.731433in}{0.729722in}}%
\pgfpathlineto{\pgfqpoint{0.733889in}{0.728457in}}%
\pgfpathlineto{\pgfqpoint{0.743620in}{0.725495in}}%
\pgfpathlineto{\pgfqpoint{0.753352in}{0.724489in}}%
\pgfpathlineto{\pgfqpoint{0.763083in}{0.725161in}}%
\pgfpathlineto{\pgfqpoint{0.772815in}{0.727081in}}%
\pgfpathlineto{\pgfqpoint{0.782546in}{0.729703in}}%
\pgfpathlineto{\pgfqpoint{0.782614in}{0.729722in}}%
\pgfpathlineto{\pgfqpoint{0.792278in}{0.733223in}}%
\pgfpathlineto{\pgfqpoint{0.802009in}{0.736097in}}%
\pgfpathlineto{\pgfqpoint{0.811741in}{0.737613in}}%
\pgfpathlineto{\pgfqpoint{0.821472in}{0.737214in}}%
\pgfpathlineto{\pgfqpoint{0.831204in}{0.734540in}}%
\pgfpathlineto{\pgfqpoint{0.840445in}{0.729722in}}%
\pgfpathlineto{\pgfqpoint{0.840935in}{0.729525in}}%
\pgfpathlineto{\pgfqpoint{0.850667in}{0.723851in}}%
\pgfpathlineto{\pgfqpoint{0.854926in}{0.720720in}}%
\pgfpathlineto{\pgfqpoint{0.860398in}{0.717416in}}%
\pgfpathlineto{\pgfqpoint{0.868628in}{0.711719in}}%
\pgfpathlineto{\pgfqpoint{0.870129in}{0.710818in}}%
\pgfpathlineto{\pgfqpoint{0.879861in}{0.704617in}}%
\pgfpathlineto{\pgfqpoint{0.882878in}{0.702717in}}%
\pgfpathlineto{\pgfqpoint{0.889592in}{0.698868in}}%
\pgfpathlineto{\pgfqpoint{0.899324in}{0.693745in}}%
\pgfpathlineto{\pgfqpoint{0.899390in}{0.693716in}}%
\pgfpathlineto{\pgfqpoint{0.909055in}{0.689637in}}%
\pgfpathlineto{\pgfqpoint{0.918787in}{0.686566in}}%
\pgfpathlineto{\pgfqpoint{0.928458in}{0.684714in}}%
\pgfpathlineto{\pgfqpoint{0.928518in}{0.684703in}}%
\pgfpathlineto{\pgfqpoint{0.938250in}{0.684109in}}%
\pgfpathlineto{\pgfqpoint{0.947974in}{0.684714in}}%
\pgfpathlineto{\pgfqpoint{0.947981in}{0.684715in}}%
\pgfpathlineto{\pgfqpoint{0.957713in}{0.686360in}}%
\pgfpathlineto{\pgfqpoint{0.967444in}{0.688775in}}%
\pgfpathlineto{\pgfqpoint{0.977176in}{0.691624in}}%
\pgfpathlineto{\pgfqpoint{0.984169in}{0.693716in}}%
\pgfpathlineto{\pgfqpoint{0.986907in}{0.694576in}}%
\pgfpathlineto{\pgfqpoint{0.996639in}{0.697308in}}%
\pgfpathlineto{\pgfqpoint{1.006370in}{0.699360in}}%
\pgfpathlineto{\pgfqpoint{1.016102in}{0.700453in}}%
\pgfpathlineto{\pgfqpoint{1.025833in}{0.700407in}}%
\pgfpathlineto{\pgfqpoint{1.035565in}{0.699160in}}%
\pgfpathlineto{\pgfqpoint{1.045296in}{0.696774in}}%
\pgfpathlineto{\pgfqpoint{1.054193in}{0.693716in}}%
\pgfpathlineto{\pgfqpoint{1.055028in}{0.693443in}}%
\pgfpathlineto{\pgfqpoint{1.064759in}{0.689613in}}%
\pgfpathlineto{\pgfqpoint{1.074491in}{0.685472in}}%
\pgfpathlineto{\pgfqpoint{1.076297in}{0.684714in}}%
\pgfpathlineto{\pgfqpoint{1.084222in}{0.681415in}}%
\pgfpathlineto{\pgfqpoint{1.093953in}{0.677794in}}%
\pgfpathlineto{\pgfqpoint{1.101043in}{0.675713in}}%
\pgfpathclose%
\pgfpathmoveto{\pgfqpoint{1.106381in}{0.828739in}}%
\pgfpathlineto{\pgfqpoint{1.103685in}{0.829509in}}%
\pgfpathlineto{\pgfqpoint{1.093953in}{0.833851in}}%
\pgfpathlineto{\pgfqpoint{1.087076in}{0.837741in}}%
\pgfpathlineto{\pgfqpoint{1.084222in}{0.839140in}}%
\pgfpathlineto{\pgfqpoint{1.074491in}{0.844481in}}%
\pgfpathlineto{\pgfqpoint{1.070433in}{0.846742in}}%
\pgfpathlineto{\pgfqpoint{1.064759in}{0.849631in}}%
\pgfpathlineto{\pgfqpoint{1.055028in}{0.854215in}}%
\pgfpathlineto{\pgfqpoint{1.051123in}{0.855744in}}%
\pgfpathlineto{\pgfqpoint{1.045296in}{0.857935in}}%
\pgfpathlineto{\pgfqpoint{1.035565in}{0.860544in}}%
\pgfpathlineto{\pgfqpoint{1.025833in}{0.861908in}}%
\pgfpathlineto{\pgfqpoint{1.016102in}{0.861958in}}%
\pgfpathlineto{\pgfqpoint{1.006370in}{0.860763in}}%
\pgfpathlineto{\pgfqpoint{0.996639in}{0.858518in}}%
\pgfpathlineto{\pgfqpoint{0.987602in}{0.855744in}}%
\pgfpathlineto{\pgfqpoint{0.986907in}{0.855522in}}%
\pgfpathlineto{\pgfqpoint{0.977176in}{0.852039in}}%
\pgfpathlineto{\pgfqpoint{0.967444in}{0.848629in}}%
\pgfpathlineto{\pgfqpoint{0.961090in}{0.846742in}}%
\pgfpathlineto{\pgfqpoint{0.957713in}{0.845645in}}%
\pgfpathlineto{\pgfqpoint{0.947981in}{0.843490in}}%
\pgfpathlineto{\pgfqpoint{0.938250in}{0.842692in}}%
\pgfpathlineto{\pgfqpoint{0.928518in}{0.843474in}}%
\pgfpathlineto{\pgfqpoint{0.918787in}{0.845914in}}%
\pgfpathlineto{\pgfqpoint{0.916783in}{0.846742in}}%
\pgfpathlineto{\pgfqpoint{0.909055in}{0.849660in}}%
\pgfpathlineto{\pgfqpoint{0.899324in}{0.854575in}}%
\pgfpathlineto{\pgfqpoint{0.897375in}{0.855744in}}%
\pgfpathlineto{\pgfqpoint{0.889592in}{0.860225in}}%
\pgfpathlineto{\pgfqpoint{0.882384in}{0.864745in}}%
\pgfpathlineto{\pgfqpoint{0.879861in}{0.866340in}}%
\pgfpathlineto{\pgfqpoint{0.870129in}{0.872562in}}%
\pgfpathlineto{\pgfqpoint{0.868163in}{0.873747in}}%
\pgfpathlineto{\pgfqpoint{0.860398in}{0.878701in}}%
\pgfpathlineto{\pgfqpoint{0.853125in}{0.882748in}}%
\pgfpathlineto{\pgfqpoint{0.850667in}{0.884277in}}%
\pgfpathlineto{\pgfqpoint{0.840935in}{0.889076in}}%
\pgfpathlineto{\pgfqpoint{0.833080in}{0.891750in}}%
\pgfpathlineto{\pgfqpoint{0.831204in}{0.892508in}}%
\pgfpathlineto{\pgfqpoint{0.821472in}{0.894582in}}%
\pgfpathlineto{\pgfqpoint{0.811741in}{0.894891in}}%
\pgfpathlineto{\pgfqpoint{0.802009in}{0.893715in}}%
\pgfpathlineto{\pgfqpoint{0.793424in}{0.891750in}}%
\pgfpathlineto{\pgfqpoint{0.792278in}{0.891529in}}%
\pgfpathlineto{\pgfqpoint{0.782546in}{0.889226in}}%
\pgfpathlineto{\pgfqpoint{0.772815in}{0.887008in}}%
\pgfpathlineto{\pgfqpoint{0.763083in}{0.885385in}}%
\pgfpathlineto{\pgfqpoint{0.753352in}{0.884816in}}%
\pgfpathlineto{\pgfqpoint{0.743620in}{0.885667in}}%
\pgfpathlineto{\pgfqpoint{0.733889in}{0.888172in}}%
\pgfpathlineto{\pgfqpoint{0.725679in}{0.891750in}}%
\pgfpathlineto{\pgfqpoint{0.724157in}{0.892537in}}%
\pgfpathlineto{\pgfqpoint{0.714426in}{0.899536in}}%
\pgfpathlineto{\pgfqpoint{0.713064in}{0.900751in}}%
\pgfpathlineto{\pgfqpoint{0.705179in}{0.909753in}}%
\pgfpathlineto{\pgfqpoint{0.704694in}{0.910535in}}%
\pgfpathlineto{\pgfqpoint{0.700244in}{0.918754in}}%
\pgfpathlineto{\pgfqpoint{0.697294in}{0.927756in}}%
\pgfpathlineto{\pgfqpoint{0.695994in}{0.936758in}}%
\pgfpathlineto{\pgfqpoint{0.695921in}{0.945759in}}%
\pgfpathlineto{\pgfqpoint{0.696539in}{0.954761in}}%
\pgfpathlineto{\pgfqpoint{0.697255in}{0.963762in}}%
\pgfpathlineto{\pgfqpoint{0.697476in}{0.972764in}}%
\pgfpathlineto{\pgfqpoint{0.696667in}{0.981765in}}%
\pgfpathlineto{\pgfqpoint{0.694963in}{0.988540in}}%
\pgfpathlineto{\pgfqpoint{0.694440in}{0.990767in}}%
\pgfpathlineto{\pgfqpoint{0.690713in}{0.999768in}}%
\pgfpathlineto{\pgfqpoint{0.685285in}{1.008770in}}%
\pgfpathlineto{\pgfqpoint{0.685231in}{1.008839in}}%
\pgfpathlineto{\pgfqpoint{0.678310in}{1.017771in}}%
\pgfpathlineto{\pgfqpoint{0.675500in}{1.020811in}}%
\pgfpathlineto{\pgfqpoint{0.669717in}{1.026773in}}%
\pgfpathlineto{\pgfqpoint{0.665768in}{1.030426in}}%
\pgfpathlineto{\pgfqpoint{0.659324in}{1.035775in}}%
\pgfpathlineto{\pgfqpoint{0.656037in}{1.038374in}}%
\pgfpathlineto{\pgfqpoint{0.646380in}{1.044776in}}%
\pgfpathlineto{\pgfqpoint{0.646305in}{1.044826in}}%
\pgfpathlineto{\pgfqpoint{0.636574in}{1.049846in}}%
\pgfpathlineto{\pgfqpoint{0.626842in}{1.053294in}}%
\pgfpathlineto{\pgfqpoint{0.624435in}{1.053778in}}%
\pgfpathlineto{\pgfqpoint{0.617111in}{1.055354in}}%
\pgfpathlineto{\pgfqpoint{0.607380in}{1.056102in}}%
\pgfpathlineto{\pgfqpoint{0.597648in}{1.055898in}}%
\pgfpathlineto{\pgfqpoint{0.587917in}{1.055235in}}%
\pgfpathlineto{\pgfqpoint{0.578185in}{1.054664in}}%
\pgfpathlineto{\pgfqpoint{0.568454in}{1.054732in}}%
\pgfpathlineto{\pgfqpoint{0.558722in}{1.055934in}}%
\pgfpathlineto{\pgfqpoint{0.548991in}{1.058663in}}%
\pgfpathlineto{\pgfqpoint{0.540105in}{1.062779in}}%
\pgfpathlineto{\pgfqpoint{0.539259in}{1.063228in}}%
\pgfpathlineto{\pgfqpoint{0.529528in}{1.070521in}}%
\pgfpathlineto{\pgfqpoint{0.528214in}{1.071781in}}%
\pgfpathlineto{\pgfqpoint{0.520647in}{1.080782in}}%
\pgfpathlineto{\pgfqpoint{0.519796in}{1.082190in}}%
\pgfpathlineto{\pgfqpoint{0.515929in}{1.089784in}}%
\pgfpathlineto{\pgfqpoint{0.513220in}{1.098785in}}%
\pgfpathlineto{\pgfqpoint{0.512300in}{1.107787in}}%
\pgfpathlineto{\pgfqpoint{0.512915in}{1.116788in}}%
\pgfpathlineto{\pgfqpoint{0.514670in}{1.125790in}}%
\pgfpathlineto{\pgfqpoint{0.517068in}{1.134792in}}%
\pgfpathlineto{\pgfqpoint{0.519557in}{1.143793in}}%
\pgfpathlineto{\pgfqpoint{0.519796in}{1.144853in}}%
\pgfpathlineto{\pgfqpoint{0.521921in}{1.152795in}}%
\pgfpathlineto{\pgfqpoint{0.523193in}{1.161796in}}%
\pgfpathlineto{\pgfqpoint{0.522858in}{1.170798in}}%
\pgfpathlineto{\pgfqpoint{0.520616in}{1.179799in}}%
\pgfpathlineto{\pgfqpoint{0.519796in}{1.181535in}}%
\pgfpathlineto{\pgfqpoint{0.516905in}{1.188801in}}%
\pgfpathlineto{\pgfqpoint{0.511717in}{1.197802in}}%
\pgfpathlineto{\pgfqpoint{0.510065in}{1.200076in}}%
\pgfpathlineto{\pgfqpoint{0.505689in}{1.206804in}}%
\pgfpathlineto{\pgfqpoint{0.500333in}{1.213986in}}%
\pgfpathlineto{\pgfqpoint{0.499053in}{1.215805in}}%
\pgfpathlineto{\pgfqpoint{0.492326in}{1.224807in}}%
\pgfpathlineto{\pgfqpoint{0.490602in}{1.227141in}}%
\pgfpathlineto{\pgfqpoint{0.485715in}{1.233808in}}%
\pgfpathlineto{\pgfqpoint{0.480870in}{1.241007in}}%
\pgfpathlineto{\pgfqpoint{0.479607in}{1.242810in}}%
\pgfpathlineto{\pgfqpoint{0.474294in}{1.251812in}}%
\pgfpathlineto{\pgfqpoint{0.471139in}{1.258960in}}%
\pgfpathlineto{\pgfqpoint{0.470244in}{1.260813in}}%
\pgfpathlineto{\pgfqpoint{0.467606in}{1.269815in}}%
\pgfpathlineto{\pgfqpoint{0.466761in}{1.278816in}}%
\pgfpathlineto{\pgfqpoint{0.467623in}{1.287818in}}%
\pgfpathlineto{\pgfqpoint{0.469953in}{1.296819in}}%
\pgfpathlineto{\pgfqpoint{0.471139in}{1.299943in}}%
\pgfpathlineto{\pgfqpoint{0.473179in}{1.305821in}}%
\pgfpathlineto{\pgfqpoint{0.476865in}{1.314822in}}%
\pgfpathlineto{\pgfqpoint{0.480630in}{1.323824in}}%
\pgfpathlineto{\pgfqpoint{0.480870in}{1.324466in}}%
\pgfpathlineto{\pgfqpoint{0.483870in}{1.332825in}}%
\pgfpathlineto{\pgfqpoint{0.486297in}{1.341827in}}%
\pgfpathlineto{\pgfqpoint{0.487589in}{1.350829in}}%
\pgfpathlineto{\pgfqpoint{0.487535in}{1.359830in}}%
\pgfpathlineto{\pgfqpoint{0.486060in}{1.368832in}}%
\pgfpathlineto{\pgfqpoint{0.483239in}{1.377833in}}%
\pgfpathlineto{\pgfqpoint{0.480870in}{1.383223in}}%
\pgfpathlineto{\pgfqpoint{0.479218in}{1.386835in}}%
\pgfpathlineto{\pgfqpoint{0.474262in}{1.395836in}}%
\pgfpathlineto{\pgfqpoint{0.471139in}{1.401084in}}%
\pgfpathlineto{\pgfqpoint{0.468695in}{1.404838in}}%
\pgfpathlineto{\pgfqpoint{0.462921in}{1.413839in}}%
\pgfpathlineto{\pgfqpoint{0.461407in}{1.416479in}}%
\pgfpathlineto{\pgfqpoint{0.457202in}{1.422841in}}%
\pgfpathlineto{\pgfqpoint{0.452508in}{1.431842in}}%
\pgfpathlineto{\pgfqpoint{0.451676in}{1.434336in}}%
\pgfpathlineto{\pgfqpoint{0.449022in}{1.440844in}}%
\pgfpathlineto{\pgfqpoint{0.447759in}{1.449846in}}%
\pgfpathlineto{\pgfqpoint{0.449022in}{1.458847in}}%
\pgfpathlineto{\pgfqpoint{0.451676in}{1.465355in}}%
\pgfpathlineto{\pgfqpoint{0.452508in}{1.467849in}}%
\pgfpathlineto{\pgfqpoint{0.457202in}{1.476850in}}%
\pgfpathlineto{\pgfqpoint{0.461407in}{1.483212in}}%
\pgfpathlineto{\pgfqpoint{0.462921in}{1.485852in}}%
\pgfpathlineto{\pgfqpoint{0.468695in}{1.494853in}}%
\pgfpathlineto{\pgfqpoint{0.471139in}{1.498607in}}%
\pgfpathlineto{\pgfqpoint{0.474262in}{1.503855in}}%
\pgfpathlineto{\pgfqpoint{0.479218in}{1.512856in}}%
\pgfpathlineto{\pgfqpoint{0.480870in}{1.516468in}}%
\pgfpathlineto{\pgfqpoint{0.483239in}{1.521858in}}%
\pgfpathlineto{\pgfqpoint{0.486060in}{1.530859in}}%
\pgfpathlineto{\pgfqpoint{0.487535in}{1.539861in}}%
\pgfpathlineto{\pgfqpoint{0.487589in}{1.548863in}}%
\pgfpathlineto{\pgfqpoint{0.486297in}{1.557864in}}%
\pgfpathlineto{\pgfqpoint{0.483870in}{1.566866in}}%
\pgfpathlineto{\pgfqpoint{0.480870in}{1.575225in}}%
\pgfpathlineto{\pgfqpoint{0.480630in}{1.575867in}}%
\pgfpathlineto{\pgfqpoint{0.476865in}{1.584869in}}%
\pgfpathlineto{\pgfqpoint{0.473179in}{1.593870in}}%
\pgfpathlineto{\pgfqpoint{0.471139in}{1.599748in}}%
\pgfpathlineto{\pgfqpoint{0.469953in}{1.602872in}}%
\pgfpathlineto{\pgfqpoint{0.467623in}{1.611873in}}%
\pgfpathlineto{\pgfqpoint{0.466761in}{1.620875in}}%
\pgfpathlineto{\pgfqpoint{0.467606in}{1.629876in}}%
\pgfpathlineto{\pgfqpoint{0.470244in}{1.638878in}}%
\pgfpathlineto{\pgfqpoint{0.471139in}{1.640731in}}%
\pgfpathlineto{\pgfqpoint{0.474294in}{1.647880in}}%
\pgfpathlineto{\pgfqpoint{0.479607in}{1.656881in}}%
\pgfpathlineto{\pgfqpoint{0.480870in}{1.658684in}}%
\pgfpathlineto{\pgfqpoint{0.485715in}{1.665883in}}%
\pgfpathlineto{\pgfqpoint{0.490602in}{1.672550in}}%
\pgfpathlineto{\pgfqpoint{0.492326in}{1.674884in}}%
\pgfpathlineto{\pgfqpoint{0.499053in}{1.683886in}}%
\pgfpathlineto{\pgfqpoint{0.500333in}{1.685705in}}%
\pgfpathlineto{\pgfqpoint{0.505689in}{1.692887in}}%
\pgfpathlineto{\pgfqpoint{0.510065in}{1.699615in}}%
\pgfpathlineto{\pgfqpoint{0.511717in}{1.701889in}}%
\pgfpathlineto{\pgfqpoint{0.516905in}{1.710890in}}%
\pgfpathlineto{\pgfqpoint{0.519796in}{1.718156in}}%
\pgfpathlineto{\pgfqpoint{0.520616in}{1.719892in}}%
\pgfpathlineto{\pgfqpoint{0.522858in}{1.728893in}}%
\pgfpathlineto{\pgfqpoint{0.523193in}{1.737895in}}%
\pgfpathlineto{\pgfqpoint{0.521921in}{1.746897in}}%
\pgfpathlineto{\pgfqpoint{0.519796in}{1.754838in}}%
\pgfpathlineto{\pgfqpoint{0.519557in}{1.755898in}}%
\pgfpathlineto{\pgfqpoint{0.517068in}{1.764900in}}%
\pgfpathlineto{\pgfqpoint{0.514670in}{1.773901in}}%
\pgfpathlineto{\pgfqpoint{0.512915in}{1.782903in}}%
\pgfpathlineto{\pgfqpoint{0.512300in}{1.791904in}}%
\pgfpathlineto{\pgfqpoint{0.513220in}{1.800906in}}%
\pgfpathlineto{\pgfqpoint{0.515929in}{1.809907in}}%
\pgfpathlineto{\pgfqpoint{0.519796in}{1.817501in}}%
\pgfpathlineto{\pgfqpoint{0.520647in}{1.818909in}}%
\pgfpathlineto{\pgfqpoint{0.528214in}{1.827910in}}%
\pgfpathlineto{\pgfqpoint{0.529528in}{1.829170in}}%
\pgfpathlineto{\pgfqpoint{0.539259in}{1.836464in}}%
\pgfpathlineto{\pgfqpoint{0.540105in}{1.836912in}}%
\pgfpathlineto{\pgfqpoint{0.548991in}{1.841028in}}%
\pgfpathlineto{\pgfqpoint{0.558722in}{1.843758in}}%
\pgfpathlineto{\pgfqpoint{0.568454in}{1.844960in}}%
\pgfpathlineto{\pgfqpoint{0.578185in}{1.845027in}}%
\pgfpathlineto{\pgfqpoint{0.587917in}{1.844456in}}%
\pgfpathlineto{\pgfqpoint{0.597648in}{1.843793in}}%
\pgfpathlineto{\pgfqpoint{0.607380in}{1.843589in}}%
\pgfpathlineto{\pgfqpoint{0.617111in}{1.844337in}}%
\pgfpathlineto{\pgfqpoint{0.624435in}{1.845914in}}%
\pgfpathlineto{\pgfqpoint{0.626842in}{1.846397in}}%
\pgfpathlineto{\pgfqpoint{0.636574in}{1.849845in}}%
\pgfpathlineto{\pgfqpoint{0.646305in}{1.854865in}}%
\pgfpathlineto{\pgfqpoint{0.646380in}{1.854915in}}%
\pgfpathlineto{\pgfqpoint{0.656037in}{1.861317in}}%
\pgfpathlineto{\pgfqpoint{0.659324in}{1.863917in}}%
\pgfpathlineto{\pgfqpoint{0.665768in}{1.869265in}}%
\pgfpathlineto{\pgfqpoint{0.669717in}{1.872918in}}%
\pgfpathlineto{\pgfqpoint{0.675500in}{1.878880in}}%
\pgfpathlineto{\pgfqpoint{0.678310in}{1.881920in}}%
\pgfpathlineto{\pgfqpoint{0.685231in}{1.890852in}}%
\pgfpathlineto{\pgfqpoint{0.685285in}{1.890921in}}%
\pgfpathlineto{\pgfqpoint{0.690713in}{1.899923in}}%
\pgfpathlineto{\pgfqpoint{0.694440in}{1.908924in}}%
\pgfpathlineto{\pgfqpoint{0.694963in}{1.911152in}}%
\pgfpathlineto{\pgfqpoint{0.696667in}{1.917926in}}%
\pgfpathlineto{\pgfqpoint{0.697476in}{1.926927in}}%
\pgfpathlineto{\pgfqpoint{0.697255in}{1.935929in}}%
\pgfpathlineto{\pgfqpoint{0.696539in}{1.944931in}}%
\pgfpathlineto{\pgfqpoint{0.695921in}{1.953932in}}%
\pgfpathlineto{\pgfqpoint{0.695994in}{1.962934in}}%
\pgfpathlineto{\pgfqpoint{0.697294in}{1.971935in}}%
\pgfpathlineto{\pgfqpoint{0.700244in}{1.980937in}}%
\pgfpathlineto{\pgfqpoint{0.704694in}{1.989156in}}%
\pgfpathlineto{\pgfqpoint{0.705179in}{1.989938in}}%
\pgfpathlineto{\pgfqpoint{0.713064in}{1.998940in}}%
\pgfpathlineto{\pgfqpoint{0.714426in}{2.000155in}}%
\pgfpathlineto{\pgfqpoint{0.724157in}{2.007154in}}%
\pgfpathlineto{\pgfqpoint{0.725679in}{2.007941in}}%
\pgfpathlineto{\pgfqpoint{0.733889in}{2.011519in}}%
\pgfpathlineto{\pgfqpoint{0.743620in}{2.014024in}}%
\pgfpathlineto{\pgfqpoint{0.753352in}{2.014875in}}%
\pgfpathlineto{\pgfqpoint{0.763083in}{2.014306in}}%
\pgfpathlineto{\pgfqpoint{0.772815in}{2.012683in}}%
\pgfpathlineto{\pgfqpoint{0.782546in}{2.010465in}}%
\pgfpathlineto{\pgfqpoint{0.792278in}{2.008162in}}%
\pgfpathlineto{\pgfqpoint{0.793424in}{2.007941in}}%
\pgfpathlineto{\pgfqpoint{0.802009in}{2.005976in}}%
\pgfpathlineto{\pgfqpoint{0.811741in}{2.004800in}}%
\pgfpathlineto{\pgfqpoint{0.821472in}{2.005109in}}%
\pgfpathlineto{\pgfqpoint{0.831204in}{2.007183in}}%
\pgfpathlineto{\pgfqpoint{0.833080in}{2.007941in}}%
\pgfpathlineto{\pgfqpoint{0.840935in}{2.010616in}}%
\pgfpathlineto{\pgfqpoint{0.850667in}{2.015414in}}%
\pgfpathlineto{\pgfqpoint{0.853125in}{2.016943in}}%
\pgfpathlineto{\pgfqpoint{0.860398in}{2.020990in}}%
\pgfpathlineto{\pgfqpoint{0.868163in}{2.025944in}}%
\pgfpathlineto{\pgfqpoint{0.870129in}{2.027129in}}%
\pgfpathlineto{\pgfqpoint{0.879861in}{2.033351in}}%
\pgfpathlineto{\pgfqpoint{0.882384in}{2.034946in}}%
\pgfpathlineto{\pgfqpoint{0.889592in}{2.039466in}}%
\pgfpathlineto{\pgfqpoint{0.897375in}{2.043947in}}%
\pgfpathlineto{\pgfqpoint{0.899324in}{2.045116in}}%
\pgfpathlineto{\pgfqpoint{0.909055in}{2.050031in}}%
\pgfpathlineto{\pgfqpoint{0.916783in}{2.052949in}}%
\pgfpathlineto{\pgfqpoint{0.918787in}{2.053777in}}%
\pgfpathlineto{\pgfqpoint{0.928518in}{2.056217in}}%
\pgfpathlineto{\pgfqpoint{0.938250in}{2.056999in}}%
\pgfpathlineto{\pgfqpoint{0.947981in}{2.056201in}}%
\pgfpathlineto{\pgfqpoint{0.957713in}{2.054046in}}%
\pgfpathlineto{\pgfqpoint{0.961090in}{2.052949in}}%
\pgfpathlineto{\pgfqpoint{0.967444in}{2.051062in}}%
\pgfpathlineto{\pgfqpoint{0.977176in}{2.047652in}}%
\pgfpathlineto{\pgfqpoint{0.986907in}{2.044169in}}%
\pgfpathlineto{\pgfqpoint{0.987602in}{2.043947in}}%
\pgfpathlineto{\pgfqpoint{0.996639in}{2.041173in}}%
\pgfpathlineto{\pgfqpoint{1.006370in}{2.038928in}}%
\pgfpathlineto{\pgfqpoint{1.016102in}{2.037733in}}%
\pgfpathlineto{\pgfqpoint{1.025833in}{2.037783in}}%
\pgfpathlineto{\pgfqpoint{1.035565in}{2.039147in}}%
\pgfpathlineto{\pgfqpoint{1.045296in}{2.041757in}}%
\pgfpathlineto{\pgfqpoint{1.051123in}{2.043947in}}%
\pgfpathlineto{\pgfqpoint{1.055028in}{2.045476in}}%
\pgfpathlineto{\pgfqpoint{1.064759in}{2.050060in}}%
\pgfpathlineto{\pgfqpoint{1.070433in}{2.052949in}}%
\pgfpathlineto{\pgfqpoint{1.074491in}{2.055210in}}%
\pgfpathlineto{\pgfqpoint{1.084222in}{2.060551in}}%
\pgfpathlineto{\pgfqpoint{1.087076in}{2.061951in}}%
\pgfpathlineto{\pgfqpoint{1.093953in}{2.065840in}}%
\pgfpathlineto{\pgfqpoint{1.103685in}{2.070182in}}%
\pgfpathlineto{\pgfqpoint{1.106381in}{2.070952in}}%
\pgfpathlineto{\pgfqpoint{1.113416in}{2.073407in}}%
\pgfpathlineto{\pgfqpoint{1.123148in}{2.074575in}}%
\pgfpathlineto{\pgfqpoint{1.132879in}{2.073407in}}%
\pgfpathlineto{\pgfqpoint{1.139915in}{2.070952in}}%
\pgfpathlineto{\pgfqpoint{1.142611in}{2.070182in}}%
\pgfpathlineto{\pgfqpoint{1.152342in}{2.065840in}}%
\pgfpathlineto{\pgfqpoint{1.159220in}{2.061951in}}%
\pgfpathlineto{\pgfqpoint{1.162074in}{2.060551in}}%
\pgfpathlineto{\pgfqpoint{1.171805in}{2.055210in}}%
\pgfpathlineto{\pgfqpoint{1.175863in}{2.052949in}}%
\pgfpathlineto{\pgfqpoint{1.181537in}{2.050060in}}%
\pgfpathlineto{\pgfqpoint{1.191268in}{2.045476in}}%
\pgfpathlineto{\pgfqpoint{1.195173in}{2.043947in}}%
\pgfpathlineto{\pgfqpoint{1.201000in}{2.041757in}}%
\pgfpathlineto{\pgfqpoint{1.210731in}{2.039147in}}%
\pgfpathlineto{\pgfqpoint{1.220463in}{2.037783in}}%
\pgfpathlineto{\pgfqpoint{1.230194in}{2.037733in}}%
\pgfpathlineto{\pgfqpoint{1.239926in}{2.038928in}}%
\pgfpathlineto{\pgfqpoint{1.249657in}{2.041173in}}%
\pgfpathlineto{\pgfqpoint{1.258694in}{2.043947in}}%
\pgfpathlineto{\pgfqpoint{1.259389in}{2.044169in}}%
\pgfpathlineto{\pgfqpoint{1.269120in}{2.047652in}}%
\pgfpathlineto{\pgfqpoint{1.278852in}{2.051062in}}%
\pgfpathlineto{\pgfqpoint{1.285206in}{2.052949in}}%
\pgfpathlineto{\pgfqpoint{1.288583in}{2.054046in}}%
\pgfpathlineto{\pgfqpoint{1.298315in}{2.056201in}}%
\pgfpathlineto{\pgfqpoint{1.308046in}{2.056999in}}%
\pgfpathlineto{\pgfqpoint{1.317778in}{2.056217in}}%
\pgfpathlineto{\pgfqpoint{1.327509in}{2.053777in}}%
\pgfpathlineto{\pgfqpoint{1.329513in}{2.052949in}}%
\pgfpathlineto{\pgfqpoint{1.337240in}{2.050031in}}%
\pgfpathlineto{\pgfqpoint{1.346972in}{2.045116in}}%
\pgfpathlineto{\pgfqpoint{1.348921in}{2.043947in}}%
\pgfpathlineto{\pgfqpoint{1.356703in}{2.039466in}}%
\pgfpathlineto{\pgfqpoint{1.363912in}{2.034946in}}%
\pgfpathlineto{\pgfqpoint{1.366435in}{2.033351in}}%
\pgfpathlineto{\pgfqpoint{1.376166in}{2.027129in}}%
\pgfpathlineto{\pgfqpoint{1.378133in}{2.025944in}}%
\pgfpathlineto{\pgfqpoint{1.385898in}{2.020990in}}%
\pgfpathlineto{\pgfqpoint{1.393171in}{2.016943in}}%
\pgfpathlineto{\pgfqpoint{1.395629in}{2.015414in}}%
\pgfpathlineto{\pgfqpoint{1.405361in}{2.010616in}}%
\pgfpathlineto{\pgfqpoint{1.413216in}{2.007941in}}%
\pgfpathlineto{\pgfqpoint{1.415092in}{2.007183in}}%
\pgfpathlineto{\pgfqpoint{1.424824in}{2.005109in}}%
\pgfpathlineto{\pgfqpoint{1.434555in}{2.004800in}}%
\pgfpathlineto{\pgfqpoint{1.444287in}{2.005976in}}%
\pgfpathlineto{\pgfqpoint{1.452872in}{2.007941in}}%
\pgfpathlineto{\pgfqpoint{1.454018in}{2.008162in}}%
\pgfpathlineto{\pgfqpoint{1.463750in}{2.010465in}}%
\pgfpathlineto{\pgfqpoint{1.473481in}{2.012683in}}%
\pgfpathlineto{\pgfqpoint{1.483213in}{2.014306in}}%
\pgfpathlineto{\pgfqpoint{1.492944in}{2.014875in}}%
\pgfpathlineto{\pgfqpoint{1.502676in}{2.014024in}}%
\pgfpathlineto{\pgfqpoint{1.512407in}{2.011519in}}%
\pgfpathlineto{\pgfqpoint{1.520617in}{2.007941in}}%
\pgfpathlineto{\pgfqpoint{1.522139in}{2.007154in}}%
\pgfpathlineto{\pgfqpoint{1.531870in}{2.000155in}}%
\pgfpathlineto{\pgfqpoint{1.533232in}{1.998940in}}%
\pgfpathlineto{\pgfqpoint{1.541117in}{1.989938in}}%
\pgfpathlineto{\pgfqpoint{1.541602in}{1.989156in}}%
\pgfpathlineto{\pgfqpoint{1.546052in}{1.980937in}}%
\pgfpathlineto{\pgfqpoint{1.549002in}{1.971935in}}%
\pgfpathlineto{\pgfqpoint{1.550302in}{1.962934in}}%
\pgfpathlineto{\pgfqpoint{1.550375in}{1.953932in}}%
\pgfpathlineto{\pgfqpoint{1.549757in}{1.944931in}}%
\pgfpathlineto{\pgfqpoint{1.549041in}{1.935929in}}%
\pgfpathlineto{\pgfqpoint{1.548820in}{1.926927in}}%
\pgfpathlineto{\pgfqpoint{1.549629in}{1.917926in}}%
\pgfpathlineto{\pgfqpoint{1.551333in}{1.911152in}}%
\pgfpathlineto{\pgfqpoint{1.551856in}{1.908924in}}%
\pgfpathlineto{\pgfqpoint{1.555583in}{1.899923in}}%
\pgfpathlineto{\pgfqpoint{1.561010in}{1.890921in}}%
\pgfpathlineto{\pgfqpoint{1.561064in}{1.890852in}}%
\pgfpathlineto{\pgfqpoint{1.567986in}{1.881920in}}%
\pgfpathlineto{\pgfqpoint{1.570796in}{1.878880in}}%
\pgfpathlineto{\pgfqpoint{1.576579in}{1.872918in}}%
\pgfpathlineto{\pgfqpoint{1.580527in}{1.869265in}}%
\pgfpathlineto{\pgfqpoint{1.586972in}{1.863917in}}%
\pgfpathlineto{\pgfqpoint{1.590259in}{1.861317in}}%
\pgfpathlineto{\pgfqpoint{1.599916in}{1.854915in}}%
\pgfpathlineto{\pgfqpoint{1.599990in}{1.854865in}}%
\pgfpathlineto{\pgfqpoint{1.609722in}{1.849845in}}%
\pgfpathlineto{\pgfqpoint{1.619453in}{1.846397in}}%
\pgfpathlineto{\pgfqpoint{1.621861in}{1.845914in}}%
\pgfpathlineto{\pgfqpoint{1.629185in}{1.844337in}}%
\pgfpathlineto{\pgfqpoint{1.638916in}{1.843589in}}%
\pgfpathlineto{\pgfqpoint{1.648648in}{1.843793in}}%
\pgfpathlineto{\pgfqpoint{1.658379in}{1.844456in}}%
\pgfpathlineto{\pgfqpoint{1.668111in}{1.845027in}}%
\pgfpathlineto{\pgfqpoint{1.677842in}{1.844960in}}%
\pgfpathlineto{\pgfqpoint{1.687574in}{1.843758in}}%
\pgfpathlineto{\pgfqpoint{1.697305in}{1.841028in}}%
\pgfpathlineto{\pgfqpoint{1.706191in}{1.836912in}}%
\pgfpathlineto{\pgfqpoint{1.707037in}{1.836464in}}%
\pgfpathlineto{\pgfqpoint{1.716768in}{1.829170in}}%
\pgfpathlineto{\pgfqpoint{1.718082in}{1.827910in}}%
\pgfpathlineto{\pgfqpoint{1.725648in}{1.818909in}}%
\pgfpathlineto{\pgfqpoint{1.726500in}{1.817501in}}%
\pgfpathlineto{\pgfqpoint{1.730367in}{1.809907in}}%
\pgfpathlineto{\pgfqpoint{1.733076in}{1.800906in}}%
\pgfpathlineto{\pgfqpoint{1.733996in}{1.791904in}}%
\pgfpathlineto{\pgfqpoint{1.733381in}{1.782903in}}%
\pgfpathlineto{\pgfqpoint{1.731626in}{1.773901in}}%
\pgfpathlineto{\pgfqpoint{1.729228in}{1.764900in}}%
\pgfpathlineto{\pgfqpoint{1.726739in}{1.755898in}}%
\pgfpathlineto{\pgfqpoint{1.726500in}{1.754838in}}%
\pgfpathlineto{\pgfqpoint{1.724375in}{1.746897in}}%
\pgfpathlineto{\pgfqpoint{1.723103in}{1.737895in}}%
\pgfpathlineto{\pgfqpoint{1.723438in}{1.728893in}}%
\pgfpathlineto{\pgfqpoint{1.725680in}{1.719892in}}%
\pgfpathlineto{\pgfqpoint{1.726500in}{1.718156in}}%
\pgfpathlineto{\pgfqpoint{1.729391in}{1.710890in}}%
\pgfpathlineto{\pgfqpoint{1.734579in}{1.701889in}}%
\pgfpathlineto{\pgfqpoint{1.736231in}{1.699615in}}%
\pgfpathlineto{\pgfqpoint{1.740607in}{1.692887in}}%
\pgfpathlineto{\pgfqpoint{1.745963in}{1.685705in}}%
\pgfpathlineto{\pgfqpoint{1.747243in}{1.683886in}}%
\pgfpathlineto{\pgfqpoint{1.753970in}{1.674884in}}%
\pgfpathlineto{\pgfqpoint{1.755694in}{1.672550in}}%
\pgfpathlineto{\pgfqpoint{1.760581in}{1.665883in}}%
\pgfpathlineto{\pgfqpoint{1.765426in}{1.658684in}}%
\pgfpathlineto{\pgfqpoint{1.766689in}{1.656881in}}%
\pgfpathlineto{\pgfqpoint{1.772002in}{1.647880in}}%
\pgfpathlineto{\pgfqpoint{1.775157in}{1.640731in}}%
\pgfpathlineto{\pgfqpoint{1.776052in}{1.638878in}}%
\pgfpathlineto{\pgfqpoint{1.778690in}{1.629876in}}%
\pgfpathlineto{\pgfqpoint{1.779535in}{1.620875in}}%
\pgfpathlineto{\pgfqpoint{1.778672in}{1.611873in}}%
\pgfpathlineto{\pgfqpoint{1.776343in}{1.602872in}}%
\pgfpathlineto{\pgfqpoint{1.775157in}{1.599748in}}%
\pgfpathlineto{\pgfqpoint{1.773117in}{1.593870in}}%
\pgfpathlineto{\pgfqpoint{1.769431in}{1.584869in}}%
\pgfpathlineto{\pgfqpoint{1.765666in}{1.575867in}}%
\pgfpathlineto{\pgfqpoint{1.765426in}{1.575225in}}%
\pgfpathlineto{\pgfqpoint{1.762426in}{1.566866in}}%
\pgfpathlineto{\pgfqpoint{1.759999in}{1.557864in}}%
\pgfpathlineto{\pgfqpoint{1.758707in}{1.548863in}}%
\pgfpathlineto{\pgfqpoint{1.758761in}{1.539861in}}%
\pgfpathlineto{\pgfqpoint{1.760236in}{1.530859in}}%
\pgfpathlineto{\pgfqpoint{1.763057in}{1.521858in}}%
\pgfpathlineto{\pgfqpoint{1.765426in}{1.516468in}}%
\pgfpathlineto{\pgfqpoint{1.767078in}{1.512856in}}%
\pgfpathlineto{\pgfqpoint{1.772034in}{1.503855in}}%
\pgfpathlineto{\pgfqpoint{1.775157in}{1.498607in}}%
\pgfpathlineto{\pgfqpoint{1.777601in}{1.494853in}}%
\pgfpathlineto{\pgfqpoint{1.783375in}{1.485852in}}%
\pgfpathlineto{\pgfqpoint{1.784889in}{1.483212in}}%
\pgfpathlineto{\pgfqpoint{1.789094in}{1.476850in}}%
\pgfpathlineto{\pgfqpoint{1.793788in}{1.467849in}}%
\pgfpathlineto{\pgfqpoint{1.794620in}{1.465355in}}%
\pgfpathlineto{\pgfqpoint{1.797274in}{1.458847in}}%
\pgfpathlineto{\pgfqpoint{1.798536in}{1.449846in}}%
\pgfpathlineto{\pgfqpoint{1.797274in}{1.440844in}}%
\pgfpathlineto{\pgfqpoint{1.794620in}{1.434336in}}%
\pgfpathlineto{\pgfqpoint{1.793788in}{1.431842in}}%
\pgfpathlineto{\pgfqpoint{1.789094in}{1.422841in}}%
\pgfpathlineto{\pgfqpoint{1.784889in}{1.416479in}}%
\pgfpathlineto{\pgfqpoint{1.783375in}{1.413839in}}%
\pgfpathlineto{\pgfqpoint{1.777601in}{1.404838in}}%
\pgfpathlineto{\pgfqpoint{1.775157in}{1.401084in}}%
\pgfpathlineto{\pgfqpoint{1.772034in}{1.395836in}}%
\pgfpathlineto{\pgfqpoint{1.767078in}{1.386835in}}%
\pgfpathlineto{\pgfqpoint{1.765426in}{1.383223in}}%
\pgfpathlineto{\pgfqpoint{1.763057in}{1.377833in}}%
\pgfpathlineto{\pgfqpoint{1.760236in}{1.368832in}}%
\pgfpathlineto{\pgfqpoint{1.758761in}{1.359830in}}%
\pgfpathlineto{\pgfqpoint{1.758707in}{1.350829in}}%
\pgfpathlineto{\pgfqpoint{1.759999in}{1.341827in}}%
\pgfpathlineto{\pgfqpoint{1.762426in}{1.332825in}}%
\pgfpathlineto{\pgfqpoint{1.765426in}{1.324466in}}%
\pgfpathlineto{\pgfqpoint{1.765666in}{1.323824in}}%
\pgfpathlineto{\pgfqpoint{1.769431in}{1.314822in}}%
\pgfpathlineto{\pgfqpoint{1.773117in}{1.305821in}}%
\pgfpathlineto{\pgfqpoint{1.775157in}{1.299943in}}%
\pgfpathlineto{\pgfqpoint{1.776343in}{1.296819in}}%
\pgfpathlineto{\pgfqpoint{1.778672in}{1.287818in}}%
\pgfpathlineto{\pgfqpoint{1.779535in}{1.278816in}}%
\pgfpathlineto{\pgfqpoint{1.778690in}{1.269815in}}%
\pgfpathlineto{\pgfqpoint{1.776052in}{1.260813in}}%
\pgfpathlineto{\pgfqpoint{1.775157in}{1.258960in}}%
\pgfpathlineto{\pgfqpoint{1.772002in}{1.251812in}}%
\pgfpathlineto{\pgfqpoint{1.766689in}{1.242810in}}%
\pgfpathlineto{\pgfqpoint{1.765426in}{1.241007in}}%
\pgfpathlineto{\pgfqpoint{1.760581in}{1.233808in}}%
\pgfpathlineto{\pgfqpoint{1.755694in}{1.227141in}}%
\pgfpathlineto{\pgfqpoint{1.753970in}{1.224807in}}%
\pgfpathlineto{\pgfqpoint{1.747243in}{1.215805in}}%
\pgfpathlineto{\pgfqpoint{1.745963in}{1.213986in}}%
\pgfpathlineto{\pgfqpoint{1.740607in}{1.206804in}}%
\pgfpathlineto{\pgfqpoint{1.736231in}{1.200076in}}%
\pgfpathlineto{\pgfqpoint{1.734579in}{1.197802in}}%
\pgfpathlineto{\pgfqpoint{1.729391in}{1.188801in}}%
\pgfpathlineto{\pgfqpoint{1.726500in}{1.181535in}}%
\pgfpathlineto{\pgfqpoint{1.725680in}{1.179799in}}%
\pgfpathlineto{\pgfqpoint{1.723438in}{1.170798in}}%
\pgfpathlineto{\pgfqpoint{1.723103in}{1.161796in}}%
\pgfpathlineto{\pgfqpoint{1.724375in}{1.152795in}}%
\pgfpathlineto{\pgfqpoint{1.726500in}{1.144853in}}%
\pgfpathlineto{\pgfqpoint{1.726739in}{1.143793in}}%
\pgfpathlineto{\pgfqpoint{1.729228in}{1.134792in}}%
\pgfpathlineto{\pgfqpoint{1.731626in}{1.125790in}}%
\pgfpathlineto{\pgfqpoint{1.733381in}{1.116788in}}%
\pgfpathlineto{\pgfqpoint{1.733996in}{1.107787in}}%
\pgfpathlineto{\pgfqpoint{1.733076in}{1.098785in}}%
\pgfpathlineto{\pgfqpoint{1.730367in}{1.089784in}}%
\pgfpathlineto{\pgfqpoint{1.726500in}{1.082190in}}%
\pgfpathlineto{\pgfqpoint{1.725648in}{1.080782in}}%
\pgfpathlineto{\pgfqpoint{1.718082in}{1.071781in}}%
\pgfpathlineto{\pgfqpoint{1.716768in}{1.070521in}}%
\pgfpathlineto{\pgfqpoint{1.707037in}{1.063228in}}%
\pgfpathlineto{\pgfqpoint{1.706191in}{1.062779in}}%
\pgfpathlineto{\pgfqpoint{1.697305in}{1.058663in}}%
\pgfpathlineto{\pgfqpoint{1.687574in}{1.055934in}}%
\pgfpathlineto{\pgfqpoint{1.677842in}{1.054732in}}%
\pgfpathlineto{\pgfqpoint{1.668111in}{1.054664in}}%
\pgfpathlineto{\pgfqpoint{1.658379in}{1.055235in}}%
\pgfpathlineto{\pgfqpoint{1.648648in}{1.055898in}}%
\pgfpathlineto{\pgfqpoint{1.638916in}{1.056102in}}%
\pgfpathlineto{\pgfqpoint{1.629185in}{1.055354in}}%
\pgfpathlineto{\pgfqpoint{1.621861in}{1.053778in}}%
\pgfpathlineto{\pgfqpoint{1.619453in}{1.053294in}}%
\pgfpathlineto{\pgfqpoint{1.609722in}{1.049846in}}%
\pgfpathlineto{\pgfqpoint{1.599990in}{1.044826in}}%
\pgfpathlineto{\pgfqpoint{1.599916in}{1.044776in}}%
\pgfpathlineto{\pgfqpoint{1.590259in}{1.038374in}}%
\pgfpathlineto{\pgfqpoint{1.586972in}{1.035775in}}%
\pgfpathlineto{\pgfqpoint{1.580527in}{1.030426in}}%
\pgfpathlineto{\pgfqpoint{1.576579in}{1.026773in}}%
\pgfpathlineto{\pgfqpoint{1.570796in}{1.020811in}}%
\pgfpathlineto{\pgfqpoint{1.567986in}{1.017771in}}%
\pgfpathlineto{\pgfqpoint{1.561064in}{1.008839in}}%
\pgfpathlineto{\pgfqpoint{1.561010in}{1.008770in}}%
\pgfpathlineto{\pgfqpoint{1.555583in}{0.999768in}}%
\pgfpathlineto{\pgfqpoint{1.551856in}{0.990767in}}%
\pgfpathlineto{\pgfqpoint{1.551333in}{0.988540in}}%
\pgfpathlineto{\pgfqpoint{1.549629in}{0.981765in}}%
\pgfpathlineto{\pgfqpoint{1.548820in}{0.972764in}}%
\pgfpathlineto{\pgfqpoint{1.549041in}{0.963762in}}%
\pgfpathlineto{\pgfqpoint{1.549757in}{0.954761in}}%
\pgfpathlineto{\pgfqpoint{1.550375in}{0.945759in}}%
\pgfpathlineto{\pgfqpoint{1.550302in}{0.936758in}}%
\pgfpathlineto{\pgfqpoint{1.549002in}{0.927756in}}%
\pgfpathlineto{\pgfqpoint{1.546052in}{0.918754in}}%
\pgfpathlineto{\pgfqpoint{1.541602in}{0.910535in}}%
\pgfpathlineto{\pgfqpoint{1.541117in}{0.909753in}}%
\pgfpathlineto{\pgfqpoint{1.533232in}{0.900751in}}%
\pgfpathlineto{\pgfqpoint{1.531870in}{0.899536in}}%
\pgfpathlineto{\pgfqpoint{1.522139in}{0.892537in}}%
\pgfpathlineto{\pgfqpoint{1.520617in}{0.891750in}}%
\pgfpathlineto{\pgfqpoint{1.512407in}{0.888172in}}%
\pgfpathlineto{\pgfqpoint{1.502676in}{0.885667in}}%
\pgfpathlineto{\pgfqpoint{1.492944in}{0.884816in}}%
\pgfpathlineto{\pgfqpoint{1.483213in}{0.885385in}}%
\pgfpathlineto{\pgfqpoint{1.473481in}{0.887008in}}%
\pgfpathlineto{\pgfqpoint{1.463750in}{0.889226in}}%
\pgfpathlineto{\pgfqpoint{1.454018in}{0.891529in}}%
\pgfpathlineto{\pgfqpoint{1.452872in}{0.891750in}}%
\pgfpathlineto{\pgfqpoint{1.444287in}{0.893715in}}%
\pgfpathlineto{\pgfqpoint{1.434555in}{0.894891in}}%
\pgfpathlineto{\pgfqpoint{1.424824in}{0.894582in}}%
\pgfpathlineto{\pgfqpoint{1.415092in}{0.892508in}}%
\pgfpathlineto{\pgfqpoint{1.413216in}{0.891750in}}%
\pgfpathlineto{\pgfqpoint{1.405361in}{0.889076in}}%
\pgfpathlineto{\pgfqpoint{1.395629in}{0.884277in}}%
\pgfpathlineto{\pgfqpoint{1.393171in}{0.882748in}}%
\pgfpathlineto{\pgfqpoint{1.385898in}{0.878701in}}%
\pgfpathlineto{\pgfqpoint{1.378133in}{0.873747in}}%
\pgfpathlineto{\pgfqpoint{1.376166in}{0.872562in}}%
\pgfpathlineto{\pgfqpoint{1.366435in}{0.866340in}}%
\pgfpathlineto{\pgfqpoint{1.363912in}{0.864745in}}%
\pgfpathlineto{\pgfqpoint{1.356703in}{0.860225in}}%
\pgfpathlineto{\pgfqpoint{1.348921in}{0.855744in}}%
\pgfpathlineto{\pgfqpoint{1.346972in}{0.854575in}}%
\pgfpathlineto{\pgfqpoint{1.337240in}{0.849660in}}%
\pgfpathlineto{\pgfqpoint{1.329513in}{0.846742in}}%
\pgfpathlineto{\pgfqpoint{1.327509in}{0.845914in}}%
\pgfpathlineto{\pgfqpoint{1.317778in}{0.843474in}}%
\pgfpathlineto{\pgfqpoint{1.308046in}{0.842692in}}%
\pgfpathlineto{\pgfqpoint{1.298315in}{0.843490in}}%
\pgfpathlineto{\pgfqpoint{1.288583in}{0.845645in}}%
\pgfpathlineto{\pgfqpoint{1.285206in}{0.846742in}}%
\pgfpathlineto{\pgfqpoint{1.278852in}{0.848629in}}%
\pgfpathlineto{\pgfqpoint{1.269120in}{0.852039in}}%
\pgfpathlineto{\pgfqpoint{1.259389in}{0.855522in}}%
\pgfpathlineto{\pgfqpoint{1.258694in}{0.855744in}}%
\pgfpathlineto{\pgfqpoint{1.249657in}{0.858518in}}%
\pgfpathlineto{\pgfqpoint{1.239926in}{0.860763in}}%
\pgfpathlineto{\pgfqpoint{1.230194in}{0.861958in}}%
\pgfpathlineto{\pgfqpoint{1.220463in}{0.861908in}}%
\pgfpathlineto{\pgfqpoint{1.210731in}{0.860544in}}%
\pgfpathlineto{\pgfqpoint{1.201000in}{0.857935in}}%
\pgfpathlineto{\pgfqpoint{1.195173in}{0.855744in}}%
\pgfpathlineto{\pgfqpoint{1.191268in}{0.854215in}}%
\pgfpathlineto{\pgfqpoint{1.181537in}{0.849631in}}%
\pgfpathlineto{\pgfqpoint{1.175863in}{0.846742in}}%
\pgfpathlineto{\pgfqpoint{1.171805in}{0.844481in}}%
\pgfpathlineto{\pgfqpoint{1.162074in}{0.839140in}}%
\pgfpathlineto{\pgfqpoint{1.159220in}{0.837741in}}%
\pgfpathlineto{\pgfqpoint{1.152342in}{0.833851in}}%
\pgfpathlineto{\pgfqpoint{1.142611in}{0.829509in}}%
\pgfpathlineto{\pgfqpoint{1.139915in}{0.828739in}}%
\pgfpathlineto{\pgfqpoint{1.132879in}{0.826284in}}%
\pgfpathlineto{\pgfqpoint{1.123148in}{0.825116in}}%
\pgfpathlineto{\pgfqpoint{1.113416in}{0.826284in}}%
\pgfpathclose%
\pgfusepath{fill}%
\end{pgfscope}%
\begin{pgfscope}%
\pgfpathrectangle{\pgfqpoint{0.150000in}{0.549691in}}{\pgfqpoint{1.946296in}{1.800309in}}%
\pgfusepath{clip}%
\pgfsetbuttcap%
\pgfsetroundjoin%
\definecolor{currentfill}{rgb}{0.278431,0.278431,0.278431}%
\pgfsetfillcolor{currentfill}%
\pgfsetlinewidth{0.000000pt}%
\definecolor{currentstroke}{rgb}{0.000000,0.000000,0.000000}%
\pgfsetstrokecolor{currentstroke}%
\pgfsetdash{}{0pt}%
\pgfpathmoveto{\pgfqpoint{0.733889in}{0.558680in}}%
\pgfpathlineto{\pgfqpoint{0.743620in}{0.555698in}}%
\pgfpathlineto{\pgfqpoint{0.753352in}{0.554686in}}%
\pgfpathlineto{\pgfqpoint{0.763083in}{0.555363in}}%
\pgfpathlineto{\pgfqpoint{0.772815in}{0.557295in}}%
\pgfpathlineto{\pgfqpoint{0.777970in}{0.558693in}}%
\pgfpathlineto{\pgfqpoint{0.782546in}{0.560281in}}%
\pgfpathlineto{\pgfqpoint{0.792278in}{0.563788in}}%
\pgfpathlineto{\pgfqpoint{0.802009in}{0.566646in}}%
\pgfpathlineto{\pgfqpoint{0.808773in}{0.567694in}}%
\pgfpathlineto{\pgfqpoint{0.811741in}{0.568300in}}%
\pgfpathlineto{\pgfqpoint{0.821472in}{0.567777in}}%
\pgfpathlineto{\pgfqpoint{0.821702in}{0.567694in}}%
\pgfpathlineto{\pgfqpoint{0.831204in}{0.565097in}}%
\pgfpathlineto{\pgfqpoint{0.840935in}{0.560052in}}%
\pgfpathlineto{\pgfqpoint{0.842745in}{0.558693in}}%
\pgfpathlineto{\pgfqpoint{0.850667in}{0.554044in}}%
\pgfpathlineto{\pgfqpoint{0.856551in}{0.549691in}}%
\pgfpathlineto{\pgfqpoint{0.860398in}{0.549691in}}%
\pgfpathlineto{\pgfqpoint{0.870129in}{0.549691in}}%
\pgfpathlineto{\pgfqpoint{0.879861in}{0.549691in}}%
\pgfpathlineto{\pgfqpoint{0.889592in}{0.549691in}}%
\pgfpathlineto{\pgfqpoint{0.899324in}{0.549691in}}%
\pgfpathlineto{\pgfqpoint{0.909055in}{0.549691in}}%
\pgfpathlineto{\pgfqpoint{0.918787in}{0.549691in}}%
\pgfpathlineto{\pgfqpoint{0.928518in}{0.549691in}}%
\pgfpathlineto{\pgfqpoint{0.938250in}{0.549691in}}%
\pgfpathlineto{\pgfqpoint{0.947981in}{0.549691in}}%
\pgfpathlineto{\pgfqpoint{0.957713in}{0.549691in}}%
\pgfpathlineto{\pgfqpoint{0.967444in}{0.549691in}}%
\pgfpathlineto{\pgfqpoint{0.977176in}{0.549691in}}%
\pgfpathlineto{\pgfqpoint{0.986907in}{0.549691in}}%
\pgfpathlineto{\pgfqpoint{0.996639in}{0.549691in}}%
\pgfpathlineto{\pgfqpoint{1.006370in}{0.549691in}}%
\pgfpathlineto{\pgfqpoint{1.016102in}{0.549691in}}%
\pgfpathlineto{\pgfqpoint{1.025833in}{0.549691in}}%
\pgfpathlineto{\pgfqpoint{1.035565in}{0.549691in}}%
\pgfpathlineto{\pgfqpoint{1.045296in}{0.549691in}}%
\pgfpathlineto{\pgfqpoint{1.055028in}{0.549691in}}%
\pgfpathlineto{\pgfqpoint{1.064759in}{0.549691in}}%
\pgfpathlineto{\pgfqpoint{1.074491in}{0.549691in}}%
\pgfpathlineto{\pgfqpoint{1.084222in}{0.549691in}}%
\pgfpathlineto{\pgfqpoint{1.093953in}{0.549691in}}%
\pgfpathlineto{\pgfqpoint{1.103685in}{0.549691in}}%
\pgfpathlineto{\pgfqpoint{1.113416in}{0.549691in}}%
\pgfpathlineto{\pgfqpoint{1.123148in}{0.549691in}}%
\pgfpathlineto{\pgfqpoint{1.132879in}{0.549691in}}%
\pgfpathlineto{\pgfqpoint{1.142611in}{0.549691in}}%
\pgfpathlineto{\pgfqpoint{1.152342in}{0.549691in}}%
\pgfpathlineto{\pgfqpoint{1.162074in}{0.549691in}}%
\pgfpathlineto{\pgfqpoint{1.171805in}{0.549691in}}%
\pgfpathlineto{\pgfqpoint{1.181537in}{0.549691in}}%
\pgfpathlineto{\pgfqpoint{1.191268in}{0.549691in}}%
\pgfpathlineto{\pgfqpoint{1.201000in}{0.549691in}}%
\pgfpathlineto{\pgfqpoint{1.210731in}{0.549691in}}%
\pgfpathlineto{\pgfqpoint{1.220463in}{0.549691in}}%
\pgfpathlineto{\pgfqpoint{1.230194in}{0.549691in}}%
\pgfpathlineto{\pgfqpoint{1.239926in}{0.549691in}}%
\pgfpathlineto{\pgfqpoint{1.249657in}{0.549691in}}%
\pgfpathlineto{\pgfqpoint{1.259389in}{0.549691in}}%
\pgfpathlineto{\pgfqpoint{1.269120in}{0.549691in}}%
\pgfpathlineto{\pgfqpoint{1.278852in}{0.549691in}}%
\pgfpathlineto{\pgfqpoint{1.288583in}{0.549691in}}%
\pgfpathlineto{\pgfqpoint{1.298315in}{0.549691in}}%
\pgfpathlineto{\pgfqpoint{1.308046in}{0.549691in}}%
\pgfpathlineto{\pgfqpoint{1.317778in}{0.549691in}}%
\pgfpathlineto{\pgfqpoint{1.327509in}{0.549691in}}%
\pgfpathlineto{\pgfqpoint{1.337240in}{0.549691in}}%
\pgfpathlineto{\pgfqpoint{1.346972in}{0.549691in}}%
\pgfpathlineto{\pgfqpoint{1.356703in}{0.549691in}}%
\pgfpathlineto{\pgfqpoint{1.366435in}{0.549691in}}%
\pgfpathlineto{\pgfqpoint{1.376166in}{0.549691in}}%
\pgfpathlineto{\pgfqpoint{1.385898in}{0.549691in}}%
\pgfpathlineto{\pgfqpoint{1.389745in}{0.549691in}}%
\pgfpathlineto{\pgfqpoint{1.395629in}{0.554044in}}%
\pgfpathlineto{\pgfqpoint{1.403551in}{0.558693in}}%
\pgfpathlineto{\pgfqpoint{1.405361in}{0.560052in}}%
\pgfpathlineto{\pgfqpoint{1.415092in}{0.565097in}}%
\pgfpathlineto{\pgfqpoint{1.424594in}{0.567694in}}%
\pgfpathlineto{\pgfqpoint{1.424824in}{0.567777in}}%
\pgfpathlineto{\pgfqpoint{1.434555in}{0.568300in}}%
\pgfpathlineto{\pgfqpoint{1.437522in}{0.567694in}}%
\pgfpathlineto{\pgfqpoint{1.444287in}{0.566646in}}%
\pgfpathlineto{\pgfqpoint{1.454018in}{0.563788in}}%
\pgfpathlineto{\pgfqpoint{1.463750in}{0.560281in}}%
\pgfpathlineto{\pgfqpoint{1.468326in}{0.558693in}}%
\pgfpathlineto{\pgfqpoint{1.473481in}{0.557295in}}%
\pgfpathlineto{\pgfqpoint{1.483213in}{0.555363in}}%
\pgfpathlineto{\pgfqpoint{1.492944in}{0.554686in}}%
\pgfpathlineto{\pgfqpoint{1.502676in}{0.555698in}}%
\pgfpathlineto{\pgfqpoint{1.512407in}{0.558680in}}%
\pgfpathlineto{\pgfqpoint{1.512432in}{0.558693in}}%
\pgfpathlineto{\pgfqpoint{1.522139in}{0.565134in}}%
\pgfpathlineto{\pgfqpoint{1.524913in}{0.567694in}}%
\pgfpathlineto{\pgfqpoint{1.531870in}{0.576142in}}%
\pgfpathlineto{\pgfqpoint{1.532238in}{0.576696in}}%
\pgfpathlineto{\pgfqpoint{1.536809in}{0.585697in}}%
\pgfpathlineto{\pgfqpoint{1.540589in}{0.594699in}}%
\pgfpathlineto{\pgfqpoint{1.541602in}{0.597183in}}%
\pgfpathlineto{\pgfqpoint{1.543922in}{0.603700in}}%
\pgfpathlineto{\pgfqpoint{1.547623in}{0.612702in}}%
\pgfpathlineto{\pgfqpoint{1.551333in}{0.619760in}}%
\pgfpathlineto{\pgfqpoint{1.552287in}{0.621703in}}%
\pgfpathlineto{\pgfqpoint{1.558062in}{0.630705in}}%
\pgfpathlineto{\pgfqpoint{1.561064in}{0.634362in}}%
\pgfpathlineto{\pgfqpoint{1.565410in}{0.639707in}}%
\pgfpathlineto{\pgfqpoint{1.570796in}{0.645090in}}%
\pgfpathlineto{\pgfqpoint{1.574594in}{0.648708in}}%
\pgfpathlineto{\pgfqpoint{1.580527in}{0.653509in}}%
\pgfpathlineto{\pgfqpoint{1.586314in}{0.657710in}}%
\pgfpathlineto{\pgfqpoint{1.590259in}{0.660251in}}%
\pgfpathlineto{\pgfqpoint{1.599990in}{0.665505in}}%
\pgfpathlineto{\pgfqpoint{1.602887in}{0.666711in}}%
\pgfpathlineto{\pgfqpoint{1.609722in}{0.669346in}}%
\pgfpathlineto{\pgfqpoint{1.619453in}{0.671922in}}%
\pgfpathlineto{\pgfqpoint{1.629185in}{0.673383in}}%
\pgfpathlineto{\pgfqpoint{1.638916in}{0.673906in}}%
\pgfpathlineto{\pgfqpoint{1.648648in}{0.673763in}}%
\pgfpathlineto{\pgfqpoint{1.658379in}{0.673301in}}%
\pgfpathlineto{\pgfqpoint{1.668111in}{0.672902in}}%
\pgfpathlineto{\pgfqpoint{1.677842in}{0.672949in}}%
\pgfpathlineto{\pgfqpoint{1.687574in}{0.673788in}}%
\pgfpathlineto{\pgfqpoint{1.697305in}{0.675692in}}%
\pgfpathlineto{\pgfqpoint{1.697369in}{0.675713in}}%
\pgfpathlineto{\pgfqpoint{1.707037in}{0.678731in}}%
\pgfpathlineto{\pgfqpoint{1.716768in}{0.683026in}}%
\pgfpathlineto{\pgfqpoint{1.719759in}{0.684714in}}%
\pgfpathlineto{\pgfqpoint{1.726500in}{0.688546in}}%
\pgfpathlineto{\pgfqpoint{1.734160in}{0.693716in}}%
\pgfpathlineto{\pgfqpoint{1.736231in}{0.695185in}}%
\pgfpathlineto{\pgfqpoint{1.745741in}{0.702717in}}%
\pgfpathlineto{\pgfqpoint{1.745963in}{0.702911in}}%
\pgfpathlineto{\pgfqpoint{1.755519in}{0.711719in}}%
\pgfpathlineto{\pgfqpoint{1.755694in}{0.711905in}}%
\pgfpathlineto{\pgfqpoint{1.763922in}{0.720720in}}%
\pgfpathlineto{\pgfqpoint{1.765426in}{0.722682in}}%
\pgfpathlineto{\pgfqpoint{1.771047in}{0.729722in}}%
\pgfpathlineto{\pgfqpoint{1.775157in}{0.736390in}}%
\pgfpathlineto{\pgfqpoint{1.776731in}{0.738724in}}%
\pgfpathlineto{\pgfqpoint{1.781093in}{0.747725in}}%
\pgfpathlineto{\pgfqpoint{1.783989in}{0.756727in}}%
\pgfpathlineto{\pgfqpoint{1.784889in}{0.761208in}}%
\pgfpathlineto{\pgfqpoint{1.785935in}{0.765728in}}%
\pgfpathlineto{\pgfqpoint{1.787310in}{0.774730in}}%
\pgfpathlineto{\pgfqpoint{1.788585in}{0.783731in}}%
\pgfpathlineto{\pgfqpoint{1.790367in}{0.792733in}}%
\pgfpathlineto{\pgfqpoint{1.793204in}{0.801734in}}%
\pgfpathlineto{\pgfqpoint{1.794620in}{0.804679in}}%
\pgfpathlineto{\pgfqpoint{1.798179in}{0.810736in}}%
\pgfpathlineto{\pgfqpoint{1.804351in}{0.818193in}}%
\pgfpathlineto{\pgfqpoint{1.806021in}{0.819737in}}%
\pgfpathlineto{\pgfqpoint{1.814083in}{0.825447in}}%
\pgfpathlineto{\pgfqpoint{1.820631in}{0.828739in}}%
\pgfpathlineto{\pgfqpoint{1.823814in}{0.830048in}}%
\pgfpathlineto{\pgfqpoint{1.833546in}{0.832673in}}%
\pgfpathlineto{\pgfqpoint{1.843277in}{0.834321in}}%
\pgfpathlineto{\pgfqpoint{1.853009in}{0.835501in}}%
\pgfpathlineto{\pgfqpoint{1.862740in}{0.836773in}}%
\pgfpathlineto{\pgfqpoint{1.867627in}{0.837741in}}%
\pgfpathlineto{\pgfqpoint{1.872472in}{0.838572in}}%
\pgfpathlineto{\pgfqpoint{1.882203in}{0.841251in}}%
\pgfpathlineto{\pgfqpoint{1.891935in}{0.845286in}}%
\pgfpathlineto{\pgfqpoint{1.894458in}{0.846742in}}%
\pgfpathlineto{\pgfqpoint{1.901666in}{0.850544in}}%
\pgfpathlineto{\pgfqpoint{1.909277in}{0.855744in}}%
\pgfpathlineto{\pgfqpoint{1.911398in}{0.857135in}}%
\pgfpathlineto{\pgfqpoint{1.920927in}{0.864745in}}%
\pgfpathlineto{\pgfqpoint{1.921129in}{0.864908in}}%
\pgfpathlineto{\pgfqpoint{1.930652in}{0.873747in}}%
\pgfpathlineto{\pgfqpoint{1.930861in}{0.873952in}}%
\pgfpathlineto{\pgfqpoint{1.939004in}{0.882748in}}%
\pgfpathlineto{\pgfqpoint{1.940592in}{0.884664in}}%
\pgfpathlineto{\pgfqpoint{1.946181in}{0.891750in}}%
\pgfpathlineto{\pgfqpoint{1.950324in}{0.897985in}}%
\pgfpathlineto{\pgfqpoint{1.952149in}{0.900751in}}%
\pgfpathlineto{\pgfqpoint{1.956793in}{0.909753in}}%
\pgfpathlineto{\pgfqpoint{1.960055in}{0.918695in}}%
\pgfpathlineto{\pgfqpoint{1.960077in}{0.918754in}}%
\pgfpathlineto{\pgfqpoint{1.962136in}{0.927756in}}%
\pgfpathlineto{\pgfqpoint{1.963043in}{0.936758in}}%
\pgfpathlineto{\pgfqpoint{1.963094in}{0.945759in}}%
\pgfpathlineto{\pgfqpoint{1.962663in}{0.954761in}}%
\pgfpathlineto{\pgfqpoint{1.962163in}{0.963762in}}%
\pgfpathlineto{\pgfqpoint{1.962009in}{0.972764in}}%
\pgfpathlineto{\pgfqpoint{1.962573in}{0.981765in}}%
\pgfpathlineto{\pgfqpoint{1.964153in}{0.990767in}}%
\pgfpathlineto{\pgfqpoint{1.966938in}{0.999768in}}%
\pgfpathlineto{\pgfqpoint{1.969787in}{1.006090in}}%
\pgfpathlineto{\pgfqpoint{1.971090in}{1.008770in}}%
\pgfpathlineto{\pgfqpoint{1.976771in}{1.017771in}}%
\pgfpathlineto{\pgfqpoint{1.979518in}{1.021420in}}%
\pgfpathlineto{\pgfqpoint{1.984060in}{1.026773in}}%
\pgfpathlineto{\pgfqpoint{1.989250in}{1.032262in}}%
\pgfpathlineto{\pgfqpoint{1.993161in}{1.035775in}}%
\pgfpathlineto{\pgfqpoint{1.998981in}{1.040756in}}%
\pgfpathlineto{\pgfqpoint{2.004759in}{1.044776in}}%
\pgfpathlineto{\pgfqpoint{2.008713in}{1.047553in}}%
\pgfpathlineto{\pgfqpoint{2.018444in}{1.052895in}}%
\pgfpathlineto{\pgfqpoint{2.020545in}{1.053778in}}%
\pgfpathlineto{\pgfqpoint{2.028175in}{1.057209in}}%
\pgfpathlineto{\pgfqpoint{2.037907in}{1.060632in}}%
\pgfpathlineto{\pgfqpoint{2.044952in}{1.062779in}}%
\pgfpathlineto{\pgfqpoint{2.047638in}{1.063716in}}%
\pgfpathlineto{\pgfqpoint{2.057370in}{1.067212in}}%
\pgfpathlineto{\pgfqpoint{2.067101in}{1.071440in}}%
\pgfpathlineto{\pgfqpoint{2.067701in}{1.071781in}}%
\pgfpathlineto{\pgfqpoint{2.076833in}{1.078216in}}%
\pgfpathlineto{\pgfqpoint{2.079600in}{1.080782in}}%
\pgfpathlineto{\pgfqpoint{2.086564in}{1.089761in}}%
\pgfpathlineto{\pgfqpoint{2.086578in}{1.089784in}}%
\pgfpathlineto{\pgfqpoint{2.089801in}{1.098785in}}%
\pgfpathlineto{\pgfqpoint{2.090896in}{1.107787in}}%
\pgfpathlineto{\pgfqpoint{2.090164in}{1.116788in}}%
\pgfpathlineto{\pgfqpoint{2.088076in}{1.125790in}}%
\pgfpathlineto{\pgfqpoint{2.086564in}{1.130558in}}%
\pgfpathlineto{\pgfqpoint{2.084847in}{1.134792in}}%
\pgfpathlineto{\pgfqpoint{2.081056in}{1.143793in}}%
\pgfpathlineto{\pgfqpoint{2.077966in}{1.152795in}}%
\pgfpathlineto{\pgfqpoint{2.076833in}{1.159052in}}%
\pgfpathlineto{\pgfqpoint{2.076178in}{1.161796in}}%
\pgfpathlineto{\pgfqpoint{2.076743in}{1.170798in}}%
\pgfpathlineto{\pgfqpoint{2.076833in}{1.171010in}}%
\pgfpathlineto{\pgfqpoint{2.079640in}{1.179799in}}%
\pgfpathlineto{\pgfqpoint{2.085095in}{1.188801in}}%
\pgfpathlineto{\pgfqpoint{2.086564in}{1.190475in}}%
\pgfpathlineto{\pgfqpoint{2.091590in}{1.197802in}}%
\pgfpathlineto{\pgfqpoint{2.096296in}{1.203246in}}%
\pgfpathlineto{\pgfqpoint{2.096296in}{1.206804in}}%
\pgfpathlineto{\pgfqpoint{2.096296in}{1.215805in}}%
\pgfpathlineto{\pgfqpoint{2.096296in}{1.224807in}}%
\pgfpathlineto{\pgfqpoint{2.096296in}{1.233808in}}%
\pgfpathlineto{\pgfqpoint{2.096296in}{1.242810in}}%
\pgfpathlineto{\pgfqpoint{2.096296in}{1.251812in}}%
\pgfpathlineto{\pgfqpoint{2.096296in}{1.260813in}}%
\pgfpathlineto{\pgfqpoint{2.096296in}{1.269815in}}%
\pgfpathlineto{\pgfqpoint{2.096296in}{1.278816in}}%
\pgfpathlineto{\pgfqpoint{2.096296in}{1.287818in}}%
\pgfpathlineto{\pgfqpoint{2.096296in}{1.296819in}}%
\pgfpathlineto{\pgfqpoint{2.096296in}{1.305821in}}%
\pgfpathlineto{\pgfqpoint{2.096296in}{1.314822in}}%
\pgfpathlineto{\pgfqpoint{2.096296in}{1.323824in}}%
\pgfpathlineto{\pgfqpoint{2.096296in}{1.332825in}}%
\pgfpathlineto{\pgfqpoint{2.096296in}{1.341827in}}%
\pgfpathlineto{\pgfqpoint{2.096296in}{1.350829in}}%
\pgfpathlineto{\pgfqpoint{2.096296in}{1.359830in}}%
\pgfpathlineto{\pgfqpoint{2.096296in}{1.368832in}}%
\pgfpathlineto{\pgfqpoint{2.096296in}{1.377833in}}%
\pgfpathlineto{\pgfqpoint{2.096296in}{1.386835in}}%
\pgfpathlineto{\pgfqpoint{2.096296in}{1.395836in}}%
\pgfpathlineto{\pgfqpoint{2.096296in}{1.404838in}}%
\pgfpathlineto{\pgfqpoint{2.096296in}{1.413839in}}%
\pgfpathlineto{\pgfqpoint{2.096296in}{1.422841in}}%
\pgfpathlineto{\pgfqpoint{2.096296in}{1.431842in}}%
\pgfpathlineto{\pgfqpoint{2.096296in}{1.440844in}}%
\pgfpathlineto{\pgfqpoint{2.096296in}{1.449846in}}%
\pgfpathlineto{\pgfqpoint{2.096296in}{1.458847in}}%
\pgfpathlineto{\pgfqpoint{2.096296in}{1.467849in}}%
\pgfpathlineto{\pgfqpoint{2.096296in}{1.476850in}}%
\pgfpathlineto{\pgfqpoint{2.096296in}{1.485852in}}%
\pgfpathlineto{\pgfqpoint{2.096296in}{1.494853in}}%
\pgfpathlineto{\pgfqpoint{2.096296in}{1.503855in}}%
\pgfpathlineto{\pgfqpoint{2.096296in}{1.512856in}}%
\pgfpathlineto{\pgfqpoint{2.096296in}{1.521858in}}%
\pgfpathlineto{\pgfqpoint{2.096296in}{1.530859in}}%
\pgfpathlineto{\pgfqpoint{2.096296in}{1.539861in}}%
\pgfpathlineto{\pgfqpoint{2.096296in}{1.548863in}}%
\pgfpathlineto{\pgfqpoint{2.096296in}{1.557864in}}%
\pgfpathlineto{\pgfqpoint{2.096296in}{1.566866in}}%
\pgfpathlineto{\pgfqpoint{2.096296in}{1.575867in}}%
\pgfpathlineto{\pgfqpoint{2.096296in}{1.584869in}}%
\pgfpathlineto{\pgfqpoint{2.096296in}{1.593870in}}%
\pgfpathlineto{\pgfqpoint{2.096296in}{1.602872in}}%
\pgfpathlineto{\pgfqpoint{2.096296in}{1.611873in}}%
\pgfpathlineto{\pgfqpoint{2.096296in}{1.620875in}}%
\pgfpathlineto{\pgfqpoint{2.096296in}{1.629876in}}%
\pgfpathlineto{\pgfqpoint{2.096296in}{1.638878in}}%
\pgfpathlineto{\pgfqpoint{2.096296in}{1.647880in}}%
\pgfpathlineto{\pgfqpoint{2.096296in}{1.656881in}}%
\pgfpathlineto{\pgfqpoint{2.096296in}{1.665883in}}%
\pgfpathlineto{\pgfqpoint{2.096296in}{1.674884in}}%
\pgfpathlineto{\pgfqpoint{2.096296in}{1.683886in}}%
\pgfpathlineto{\pgfqpoint{2.096296in}{1.692887in}}%
\pgfpathlineto{\pgfqpoint{2.096296in}{1.696446in}}%
\pgfpathlineto{\pgfqpoint{2.091590in}{1.701889in}}%
\pgfpathlineto{\pgfqpoint{2.086564in}{1.709217in}}%
\pgfpathlineto{\pgfqpoint{2.085095in}{1.710890in}}%
\pgfpathlineto{\pgfqpoint{2.079640in}{1.719892in}}%
\pgfpathlineto{\pgfqpoint{2.076833in}{1.728681in}}%
\pgfpathlineto{\pgfqpoint{2.076743in}{1.728893in}}%
\pgfpathlineto{\pgfqpoint{2.076178in}{1.737895in}}%
\pgfpathlineto{\pgfqpoint{2.076833in}{1.740640in}}%
\pgfpathlineto{\pgfqpoint{2.077966in}{1.746897in}}%
\pgfpathlineto{\pgfqpoint{2.081056in}{1.755898in}}%
\pgfpathlineto{\pgfqpoint{2.084847in}{1.764900in}}%
\pgfpathlineto{\pgfqpoint{2.086564in}{1.769133in}}%
\pgfpathlineto{\pgfqpoint{2.088076in}{1.773901in}}%
\pgfpathlineto{\pgfqpoint{2.090164in}{1.782903in}}%
\pgfpathlineto{\pgfqpoint{2.090896in}{1.791904in}}%
\pgfpathlineto{\pgfqpoint{2.089801in}{1.800906in}}%
\pgfpathlineto{\pgfqpoint{2.086578in}{1.809907in}}%
\pgfpathlineto{\pgfqpoint{2.086564in}{1.809931in}}%
\pgfpathlineto{\pgfqpoint{2.079600in}{1.818909in}}%
\pgfpathlineto{\pgfqpoint{2.076833in}{1.821476in}}%
\pgfpathlineto{\pgfqpoint{2.067701in}{1.827910in}}%
\pgfpathlineto{\pgfqpoint{2.067101in}{1.828251in}}%
\pgfpathlineto{\pgfqpoint{2.057370in}{1.832479in}}%
\pgfpathlineto{\pgfqpoint{2.047638in}{1.835976in}}%
\pgfpathlineto{\pgfqpoint{2.044952in}{1.836912in}}%
\pgfpathlineto{\pgfqpoint{2.037907in}{1.839059in}}%
\pgfpathlineto{\pgfqpoint{2.028175in}{1.842482in}}%
\pgfpathlineto{\pgfqpoint{2.020545in}{1.845914in}}%
\pgfpathlineto{\pgfqpoint{2.018444in}{1.846796in}}%
\pgfpathlineto{\pgfqpoint{2.008713in}{1.852138in}}%
\pgfpathlineto{\pgfqpoint{2.004759in}{1.854915in}}%
\pgfpathlineto{\pgfqpoint{1.998981in}{1.858935in}}%
\pgfpathlineto{\pgfqpoint{1.993161in}{1.863917in}}%
\pgfpathlineto{\pgfqpoint{1.989250in}{1.867429in}}%
\pgfpathlineto{\pgfqpoint{1.984060in}{1.872918in}}%
\pgfpathlineto{\pgfqpoint{1.979518in}{1.878271in}}%
\pgfpathlineto{\pgfqpoint{1.976771in}{1.881920in}}%
\pgfpathlineto{\pgfqpoint{1.971090in}{1.890921in}}%
\pgfpathlineto{\pgfqpoint{1.969787in}{1.893601in}}%
\pgfpathlineto{\pgfqpoint{1.966938in}{1.899923in}}%
\pgfpathlineto{\pgfqpoint{1.964153in}{1.908924in}}%
\pgfpathlineto{\pgfqpoint{1.962573in}{1.917926in}}%
\pgfpathlineto{\pgfqpoint{1.962009in}{1.926927in}}%
\pgfpathlineto{\pgfqpoint{1.962163in}{1.935929in}}%
\pgfpathlineto{\pgfqpoint{1.962663in}{1.944931in}}%
\pgfpathlineto{\pgfqpoint{1.963094in}{1.953932in}}%
\pgfpathlineto{\pgfqpoint{1.963043in}{1.962934in}}%
\pgfpathlineto{\pgfqpoint{1.962136in}{1.971935in}}%
\pgfpathlineto{\pgfqpoint{1.960077in}{1.980937in}}%
\pgfpathlineto{\pgfqpoint{1.960055in}{1.980996in}}%
\pgfpathlineto{\pgfqpoint{1.956793in}{1.989938in}}%
\pgfpathlineto{\pgfqpoint{1.952149in}{1.998940in}}%
\pgfpathlineto{\pgfqpoint{1.950324in}{2.001706in}}%
\pgfpathlineto{\pgfqpoint{1.946181in}{2.007941in}}%
\pgfpathlineto{\pgfqpoint{1.940592in}{2.015027in}}%
\pgfpathlineto{\pgfqpoint{1.939004in}{2.016943in}}%
\pgfpathlineto{\pgfqpoint{1.930861in}{2.025739in}}%
\pgfpathlineto{\pgfqpoint{1.930652in}{2.025944in}}%
\pgfpathlineto{\pgfqpoint{1.921129in}{2.034784in}}%
\pgfpathlineto{\pgfqpoint{1.920927in}{2.034946in}}%
\pgfpathlineto{\pgfqpoint{1.911398in}{2.042556in}}%
\pgfpathlineto{\pgfqpoint{1.909277in}{2.043947in}}%
\pgfpathlineto{\pgfqpoint{1.901666in}{2.049147in}}%
\pgfpathlineto{\pgfqpoint{1.894458in}{2.052949in}}%
\pgfpathlineto{\pgfqpoint{1.891935in}{2.054405in}}%
\pgfpathlineto{\pgfqpoint{1.882203in}{2.058440in}}%
\pgfpathlineto{\pgfqpoint{1.872472in}{2.061119in}}%
\pgfpathlineto{\pgfqpoint{1.867627in}{2.061951in}}%
\pgfpathlineto{\pgfqpoint{1.862740in}{2.062918in}}%
\pgfpathlineto{\pgfqpoint{1.853009in}{2.064191in}}%
\pgfpathlineto{\pgfqpoint{1.843277in}{2.065370in}}%
\pgfpathlineto{\pgfqpoint{1.833546in}{2.067018in}}%
\pgfpathlineto{\pgfqpoint{1.823814in}{2.069643in}}%
\pgfpathlineto{\pgfqpoint{1.820631in}{2.070952in}}%
\pgfpathlineto{\pgfqpoint{1.814083in}{2.074244in}}%
\pgfpathlineto{\pgfqpoint{1.806021in}{2.079954in}}%
\pgfpathlineto{\pgfqpoint{1.804351in}{2.081498in}}%
\pgfpathlineto{\pgfqpoint{1.798179in}{2.088955in}}%
\pgfpathlineto{\pgfqpoint{1.794620in}{2.095012in}}%
\pgfpathlineto{\pgfqpoint{1.793204in}{2.097957in}}%
\pgfpathlineto{\pgfqpoint{1.790367in}{2.106958in}}%
\pgfpathlineto{\pgfqpoint{1.788585in}{2.115960in}}%
\pgfpathlineto{\pgfqpoint{1.787310in}{2.124961in}}%
\pgfpathlineto{\pgfqpoint{1.785935in}{2.133963in}}%
\pgfpathlineto{\pgfqpoint{1.784889in}{2.138483in}}%
\pgfpathlineto{\pgfqpoint{1.783989in}{2.142964in}}%
\pgfpathlineto{\pgfqpoint{1.781093in}{2.151966in}}%
\pgfpathlineto{\pgfqpoint{1.776731in}{2.160968in}}%
\pgfpathlineto{\pgfqpoint{1.775157in}{2.163301in}}%
\pgfpathlineto{\pgfqpoint{1.771047in}{2.169969in}}%
\pgfpathlineto{\pgfqpoint{1.765426in}{2.177009in}}%
\pgfpathlineto{\pgfqpoint{1.763922in}{2.178971in}}%
\pgfpathlineto{\pgfqpoint{1.755694in}{2.187786in}}%
\pgfpathlineto{\pgfqpoint{1.755519in}{2.187972in}}%
\pgfpathlineto{\pgfqpoint{1.745963in}{2.196780in}}%
\pgfpathlineto{\pgfqpoint{1.745741in}{2.196974in}}%
\pgfpathlineto{\pgfqpoint{1.736231in}{2.204507in}}%
\pgfpathlineto{\pgfqpoint{1.734160in}{2.205975in}}%
\pgfpathlineto{\pgfqpoint{1.726500in}{2.211145in}}%
\pgfpathlineto{\pgfqpoint{1.719759in}{2.214977in}}%
\pgfpathlineto{\pgfqpoint{1.716768in}{2.216665in}}%
\pgfpathlineto{\pgfqpoint{1.707037in}{2.220961in}}%
\pgfpathlineto{\pgfqpoint{1.697369in}{2.223978in}}%
\pgfpathlineto{\pgfqpoint{1.697305in}{2.223999in}}%
\pgfpathlineto{\pgfqpoint{1.687574in}{2.225903in}}%
\pgfpathlineto{\pgfqpoint{1.677842in}{2.226742in}}%
\pgfpathlineto{\pgfqpoint{1.668111in}{2.226789in}}%
\pgfpathlineto{\pgfqpoint{1.658379in}{2.226390in}}%
\pgfpathlineto{\pgfqpoint{1.648648in}{2.225928in}}%
\pgfpathlineto{\pgfqpoint{1.638916in}{2.225785in}}%
\pgfpathlineto{\pgfqpoint{1.629185in}{2.226308in}}%
\pgfpathlineto{\pgfqpoint{1.619453in}{2.227769in}}%
\pgfpathlineto{\pgfqpoint{1.609722in}{2.230345in}}%
\pgfpathlineto{\pgfqpoint{1.602887in}{2.232980in}}%
\pgfpathlineto{\pgfqpoint{1.599990in}{2.234186in}}%
\pgfpathlineto{\pgfqpoint{1.590259in}{2.239441in}}%
\pgfpathlineto{\pgfqpoint{1.586314in}{2.241981in}}%
\pgfpathlineto{\pgfqpoint{1.580527in}{2.246182in}}%
\pgfpathlineto{\pgfqpoint{1.574594in}{2.250983in}}%
\pgfpathlineto{\pgfqpoint{1.570796in}{2.254601in}}%
\pgfpathlineto{\pgfqpoint{1.565410in}{2.259985in}}%
\pgfpathlineto{\pgfqpoint{1.561064in}{2.265329in}}%
\pgfpathlineto{\pgfqpoint{1.558062in}{2.268986in}}%
\pgfpathlineto{\pgfqpoint{1.552287in}{2.277988in}}%
\pgfpathlineto{\pgfqpoint{1.551333in}{2.279931in}}%
\pgfpathlineto{\pgfqpoint{1.547623in}{2.286989in}}%
\pgfpathlineto{\pgfqpoint{1.543922in}{2.295991in}}%
\pgfpathlineto{\pgfqpoint{1.541602in}{2.302508in}}%
\pgfpathlineto{\pgfqpoint{1.540589in}{2.304992in}}%
\pgfpathlineto{\pgfqpoint{1.536809in}{2.313994in}}%
\pgfpathlineto{\pgfqpoint{1.532238in}{2.322995in}}%
\pgfpathlineto{\pgfqpoint{1.531870in}{2.323550in}}%
\pgfpathlineto{\pgfqpoint{1.524913in}{2.331997in}}%
\pgfpathlineto{\pgfqpoint{1.522139in}{2.334557in}}%
\pgfpathlineto{\pgfqpoint{1.512432in}{2.340998in}}%
\pgfpathlineto{\pgfqpoint{1.512407in}{2.341011in}}%
\pgfpathlineto{\pgfqpoint{1.502676in}{2.343993in}}%
\pgfpathlineto{\pgfqpoint{1.492944in}{2.345005in}}%
\pgfpathlineto{\pgfqpoint{1.483213in}{2.344328in}}%
\pgfpathlineto{\pgfqpoint{1.473481in}{2.342396in}}%
\pgfpathlineto{\pgfqpoint{1.468326in}{2.340998in}}%
\pgfpathlineto{\pgfqpoint{1.463750in}{2.339410in}}%
\pgfpathlineto{\pgfqpoint{1.454018in}{2.335903in}}%
\pgfpathlineto{\pgfqpoint{1.444287in}{2.333045in}}%
\pgfpathlineto{\pgfqpoint{1.437522in}{2.331997in}}%
\pgfpathlineto{\pgfqpoint{1.434555in}{2.331392in}}%
\pgfpathlineto{\pgfqpoint{1.424824in}{2.331914in}}%
\pgfpathlineto{\pgfqpoint{1.424594in}{2.331997in}}%
\pgfpathlineto{\pgfqpoint{1.415092in}{2.334594in}}%
\pgfpathlineto{\pgfqpoint{1.405361in}{2.339640in}}%
\pgfpathlineto{\pgfqpoint{1.403551in}{2.340998in}}%
\pgfpathlineto{\pgfqpoint{1.395629in}{2.345647in}}%
\pgfpathlineto{\pgfqpoint{1.389745in}{2.350000in}}%
\pgfpathlineto{\pgfqpoint{1.385898in}{2.350000in}}%
\pgfpathlineto{\pgfqpoint{1.376166in}{2.350000in}}%
\pgfpathlineto{\pgfqpoint{1.366435in}{2.350000in}}%
\pgfpathlineto{\pgfqpoint{1.356703in}{2.350000in}}%
\pgfpathlineto{\pgfqpoint{1.346972in}{2.350000in}}%
\pgfpathlineto{\pgfqpoint{1.337240in}{2.350000in}}%
\pgfpathlineto{\pgfqpoint{1.327509in}{2.350000in}}%
\pgfpathlineto{\pgfqpoint{1.317778in}{2.350000in}}%
\pgfpathlineto{\pgfqpoint{1.308046in}{2.350000in}}%
\pgfpathlineto{\pgfqpoint{1.298315in}{2.350000in}}%
\pgfpathlineto{\pgfqpoint{1.288583in}{2.350000in}}%
\pgfpathlineto{\pgfqpoint{1.278852in}{2.350000in}}%
\pgfpathlineto{\pgfqpoint{1.269120in}{2.350000in}}%
\pgfpathlineto{\pgfqpoint{1.259389in}{2.350000in}}%
\pgfpathlineto{\pgfqpoint{1.249657in}{2.350000in}}%
\pgfpathlineto{\pgfqpoint{1.239926in}{2.350000in}}%
\pgfpathlineto{\pgfqpoint{1.230194in}{2.350000in}}%
\pgfpathlineto{\pgfqpoint{1.220463in}{2.350000in}}%
\pgfpathlineto{\pgfqpoint{1.210731in}{2.350000in}}%
\pgfpathlineto{\pgfqpoint{1.201000in}{2.350000in}}%
\pgfpathlineto{\pgfqpoint{1.191268in}{2.350000in}}%
\pgfpathlineto{\pgfqpoint{1.181537in}{2.350000in}}%
\pgfpathlineto{\pgfqpoint{1.171805in}{2.350000in}}%
\pgfpathlineto{\pgfqpoint{1.162074in}{2.350000in}}%
\pgfpathlineto{\pgfqpoint{1.152342in}{2.350000in}}%
\pgfpathlineto{\pgfqpoint{1.142611in}{2.350000in}}%
\pgfpathlineto{\pgfqpoint{1.132879in}{2.350000in}}%
\pgfpathlineto{\pgfqpoint{1.123148in}{2.350000in}}%
\pgfpathlineto{\pgfqpoint{1.113416in}{2.350000in}}%
\pgfpathlineto{\pgfqpoint{1.103685in}{2.350000in}}%
\pgfpathlineto{\pgfqpoint{1.093953in}{2.350000in}}%
\pgfpathlineto{\pgfqpoint{1.084222in}{2.350000in}}%
\pgfpathlineto{\pgfqpoint{1.074491in}{2.350000in}}%
\pgfpathlineto{\pgfqpoint{1.064759in}{2.350000in}}%
\pgfpathlineto{\pgfqpoint{1.055028in}{2.350000in}}%
\pgfpathlineto{\pgfqpoint{1.045296in}{2.350000in}}%
\pgfpathlineto{\pgfqpoint{1.035565in}{2.350000in}}%
\pgfpathlineto{\pgfqpoint{1.025833in}{2.350000in}}%
\pgfpathlineto{\pgfqpoint{1.016102in}{2.350000in}}%
\pgfpathlineto{\pgfqpoint{1.006370in}{2.350000in}}%
\pgfpathlineto{\pgfqpoint{0.996639in}{2.350000in}}%
\pgfpathlineto{\pgfqpoint{0.986907in}{2.350000in}}%
\pgfpathlineto{\pgfqpoint{0.977176in}{2.350000in}}%
\pgfpathlineto{\pgfqpoint{0.967444in}{2.350000in}}%
\pgfpathlineto{\pgfqpoint{0.957713in}{2.350000in}}%
\pgfpathlineto{\pgfqpoint{0.947981in}{2.350000in}}%
\pgfpathlineto{\pgfqpoint{0.938250in}{2.350000in}}%
\pgfpathlineto{\pgfqpoint{0.928518in}{2.350000in}}%
\pgfpathlineto{\pgfqpoint{0.918787in}{2.350000in}}%
\pgfpathlineto{\pgfqpoint{0.909055in}{2.350000in}}%
\pgfpathlineto{\pgfqpoint{0.899324in}{2.350000in}}%
\pgfpathlineto{\pgfqpoint{0.889592in}{2.350000in}}%
\pgfpathlineto{\pgfqpoint{0.879861in}{2.350000in}}%
\pgfpathlineto{\pgfqpoint{0.870129in}{2.350000in}}%
\pgfpathlineto{\pgfqpoint{0.860398in}{2.350000in}}%
\pgfpathlineto{\pgfqpoint{0.856551in}{2.350000in}}%
\pgfpathlineto{\pgfqpoint{0.850667in}{2.345647in}}%
\pgfpathlineto{\pgfqpoint{0.842745in}{2.340998in}}%
\pgfpathlineto{\pgfqpoint{0.840935in}{2.339640in}}%
\pgfpathlineto{\pgfqpoint{0.831204in}{2.334594in}}%
\pgfpathlineto{\pgfqpoint{0.821702in}{2.331997in}}%
\pgfpathlineto{\pgfqpoint{0.821472in}{2.331914in}}%
\pgfpathlineto{\pgfqpoint{0.811741in}{2.331392in}}%
\pgfpathlineto{\pgfqpoint{0.808773in}{2.331997in}}%
\pgfpathlineto{\pgfqpoint{0.802009in}{2.333045in}}%
\pgfpathlineto{\pgfqpoint{0.792278in}{2.335903in}}%
\pgfpathlineto{\pgfqpoint{0.782546in}{2.339410in}}%
\pgfpathlineto{\pgfqpoint{0.777970in}{2.340998in}}%
\pgfpathlineto{\pgfqpoint{0.772815in}{2.342396in}}%
\pgfpathlineto{\pgfqpoint{0.763083in}{2.344328in}}%
\pgfpathlineto{\pgfqpoint{0.753352in}{2.345005in}}%
\pgfpathlineto{\pgfqpoint{0.743620in}{2.343993in}}%
\pgfpathlineto{\pgfqpoint{0.733889in}{2.341011in}}%
\pgfpathlineto{\pgfqpoint{0.733864in}{2.340998in}}%
\pgfpathlineto{\pgfqpoint{0.724157in}{2.334557in}}%
\pgfpathlineto{\pgfqpoint{0.721382in}{2.331997in}}%
\pgfpathlineto{\pgfqpoint{0.714426in}{2.323550in}}%
\pgfpathlineto{\pgfqpoint{0.714058in}{2.322995in}}%
\pgfpathlineto{\pgfqpoint{0.709487in}{2.313994in}}%
\pgfpathlineto{\pgfqpoint{0.705707in}{2.304992in}}%
\pgfpathlineto{\pgfqpoint{0.704694in}{2.302508in}}%
\pgfpathlineto{\pgfqpoint{0.702373in}{2.295991in}}%
\pgfpathlineto{\pgfqpoint{0.698672in}{2.286989in}}%
\pgfpathlineto{\pgfqpoint{0.694963in}{2.279931in}}%
\pgfpathlineto{\pgfqpoint{0.694009in}{2.277988in}}%
\pgfpathlineto{\pgfqpoint{0.688233in}{2.268986in}}%
\pgfpathlineto{\pgfqpoint{0.685231in}{2.265329in}}%
\pgfpathlineto{\pgfqpoint{0.680886in}{2.259985in}}%
\pgfpathlineto{\pgfqpoint{0.675500in}{2.254601in}}%
\pgfpathlineto{\pgfqpoint{0.671702in}{2.250983in}}%
\pgfpathlineto{\pgfqpoint{0.665768in}{2.246182in}}%
\pgfpathlineto{\pgfqpoint{0.659981in}{2.241981in}}%
\pgfpathlineto{\pgfqpoint{0.656037in}{2.239441in}}%
\pgfpathlineto{\pgfqpoint{0.646305in}{2.234186in}}%
\pgfpathlineto{\pgfqpoint{0.643408in}{2.232980in}}%
\pgfpathlineto{\pgfqpoint{0.636574in}{2.230345in}}%
\pgfpathlineto{\pgfqpoint{0.626842in}{2.227769in}}%
\pgfpathlineto{\pgfqpoint{0.617111in}{2.226308in}}%
\pgfpathlineto{\pgfqpoint{0.607380in}{2.225785in}}%
\pgfpathlineto{\pgfqpoint{0.597648in}{2.225928in}}%
\pgfpathlineto{\pgfqpoint{0.587917in}{2.226390in}}%
\pgfpathlineto{\pgfqpoint{0.578185in}{2.226789in}}%
\pgfpathlineto{\pgfqpoint{0.568454in}{2.226742in}}%
\pgfpathlineto{\pgfqpoint{0.558722in}{2.225903in}}%
\pgfpathlineto{\pgfqpoint{0.548991in}{2.223999in}}%
\pgfpathlineto{\pgfqpoint{0.548927in}{2.223978in}}%
\pgfpathlineto{\pgfqpoint{0.539259in}{2.220961in}}%
\pgfpathlineto{\pgfqpoint{0.529528in}{2.216665in}}%
\pgfpathlineto{\pgfqpoint{0.526537in}{2.214977in}}%
\pgfpathlineto{\pgfqpoint{0.519796in}{2.211145in}}%
\pgfpathlineto{\pgfqpoint{0.512136in}{2.205975in}}%
\pgfpathlineto{\pgfqpoint{0.510065in}{2.204507in}}%
\pgfpathlineto{\pgfqpoint{0.500555in}{2.196974in}}%
\pgfpathlineto{\pgfqpoint{0.500333in}{2.196780in}}%
\pgfpathlineto{\pgfqpoint{0.490777in}{2.187972in}}%
\pgfpathlineto{\pgfqpoint{0.490602in}{2.187786in}}%
\pgfpathlineto{\pgfqpoint{0.482374in}{2.178971in}}%
\pgfpathlineto{\pgfqpoint{0.480870in}{2.177009in}}%
\pgfpathlineto{\pgfqpoint{0.475249in}{2.169969in}}%
\pgfpathlineto{\pgfqpoint{0.471139in}{2.163301in}}%
\pgfpathlineto{\pgfqpoint{0.469565in}{2.160968in}}%
\pgfpathlineto{\pgfqpoint{0.465203in}{2.151966in}}%
\pgfpathlineto{\pgfqpoint{0.462307in}{2.142964in}}%
\pgfpathlineto{\pgfqpoint{0.461407in}{2.138483in}}%
\pgfpathlineto{\pgfqpoint{0.460361in}{2.133963in}}%
\pgfpathlineto{\pgfqpoint{0.458986in}{2.124961in}}%
\pgfpathlineto{\pgfqpoint{0.457711in}{2.115960in}}%
\pgfpathlineto{\pgfqpoint{0.455929in}{2.106958in}}%
\pgfpathlineto{\pgfqpoint{0.453091in}{2.097957in}}%
\pgfpathlineto{\pgfqpoint{0.451676in}{2.095012in}}%
\pgfpathlineto{\pgfqpoint{0.448117in}{2.088955in}}%
\pgfpathlineto{\pgfqpoint{0.441944in}{2.081498in}}%
\pgfpathlineto{\pgfqpoint{0.440274in}{2.079954in}}%
\pgfpathlineto{\pgfqpoint{0.432213in}{2.074244in}}%
\pgfpathlineto{\pgfqpoint{0.425665in}{2.070952in}}%
\pgfpathlineto{\pgfqpoint{0.422481in}{2.069643in}}%
\pgfpathlineto{\pgfqpoint{0.412750in}{2.067018in}}%
\pgfpathlineto{\pgfqpoint{0.403018in}{2.065370in}}%
\pgfpathlineto{\pgfqpoint{0.393287in}{2.064191in}}%
\pgfpathlineto{\pgfqpoint{0.383556in}{2.062918in}}%
\pgfpathlineto{\pgfqpoint{0.378669in}{2.061951in}}%
\pgfpathlineto{\pgfqpoint{0.373824in}{2.061119in}}%
\pgfpathlineto{\pgfqpoint{0.364093in}{2.058440in}}%
\pgfpathlineto{\pgfqpoint{0.354361in}{2.054405in}}%
\pgfpathlineto{\pgfqpoint{0.351838in}{2.052949in}}%
\pgfpathlineto{\pgfqpoint{0.344630in}{2.049147in}}%
\pgfpathlineto{\pgfqpoint{0.337019in}{2.043947in}}%
\pgfpathlineto{\pgfqpoint{0.334898in}{2.042556in}}%
\pgfpathlineto{\pgfqpoint{0.325368in}{2.034946in}}%
\pgfpathlineto{\pgfqpoint{0.325167in}{2.034784in}}%
\pgfpathlineto{\pgfqpoint{0.315644in}{2.025944in}}%
\pgfpathlineto{\pgfqpoint{0.315435in}{2.025739in}}%
\pgfpathlineto{\pgfqpoint{0.307291in}{2.016943in}}%
\pgfpathlineto{\pgfqpoint{0.305704in}{2.015027in}}%
\pgfpathlineto{\pgfqpoint{0.300115in}{2.007941in}}%
\pgfpathlineto{\pgfqpoint{0.295972in}{2.001706in}}%
\pgfpathlineto{\pgfqpoint{0.294147in}{1.998940in}}%
\pgfpathlineto{\pgfqpoint{0.289503in}{1.989938in}}%
\pgfpathlineto{\pgfqpoint{0.286241in}{1.980996in}}%
\pgfpathlineto{\pgfqpoint{0.286218in}{1.980937in}}%
\pgfpathlineto{\pgfqpoint{0.284160in}{1.971935in}}%
\pgfpathlineto{\pgfqpoint{0.283253in}{1.962934in}}%
\pgfpathlineto{\pgfqpoint{0.283202in}{1.953932in}}%
\pgfpathlineto{\pgfqpoint{0.283633in}{1.944931in}}%
\pgfpathlineto{\pgfqpoint{0.284133in}{1.935929in}}%
\pgfpathlineto{\pgfqpoint{0.284287in}{1.926927in}}%
\pgfpathlineto{\pgfqpoint{0.283722in}{1.917926in}}%
\pgfpathlineto{\pgfqpoint{0.282143in}{1.908924in}}%
\pgfpathlineto{\pgfqpoint{0.279358in}{1.899923in}}%
\pgfpathlineto{\pgfqpoint{0.276509in}{1.893601in}}%
\pgfpathlineto{\pgfqpoint{0.275205in}{1.890921in}}%
\pgfpathlineto{\pgfqpoint{0.269525in}{1.881920in}}%
\pgfpathlineto{\pgfqpoint{0.266778in}{1.878271in}}%
\pgfpathlineto{\pgfqpoint{0.262236in}{1.872918in}}%
\pgfpathlineto{\pgfqpoint{0.257046in}{1.867429in}}%
\pgfpathlineto{\pgfqpoint{0.253135in}{1.863917in}}%
\pgfpathlineto{\pgfqpoint{0.247315in}{1.858935in}}%
\pgfpathlineto{\pgfqpoint{0.241537in}{1.854915in}}%
\pgfpathlineto{\pgfqpoint{0.237583in}{1.852138in}}%
\pgfpathlineto{\pgfqpoint{0.227852in}{1.846796in}}%
\pgfpathlineto{\pgfqpoint{0.225751in}{1.845914in}}%
\pgfpathlineto{\pgfqpoint{0.218120in}{1.842482in}}%
\pgfpathlineto{\pgfqpoint{0.208389in}{1.839059in}}%
\pgfpathlineto{\pgfqpoint{0.201343in}{1.836912in}}%
\pgfpathlineto{\pgfqpoint{0.198657in}{1.835976in}}%
\pgfpathlineto{\pgfqpoint{0.188926in}{1.832479in}}%
\pgfpathlineto{\pgfqpoint{0.179194in}{1.828251in}}%
\pgfpathlineto{\pgfqpoint{0.178595in}{1.827910in}}%
\pgfpathlineto{\pgfqpoint{0.169463in}{1.821476in}}%
\pgfpathlineto{\pgfqpoint{0.166695in}{1.818909in}}%
\pgfpathlineto{\pgfqpoint{0.159731in}{1.809931in}}%
\pgfpathlineto{\pgfqpoint{0.159717in}{1.809907in}}%
\pgfpathlineto{\pgfqpoint{0.156494in}{1.800906in}}%
\pgfpathlineto{\pgfqpoint{0.155400in}{1.791904in}}%
\pgfpathlineto{\pgfqpoint{0.156132in}{1.782903in}}%
\pgfpathlineto{\pgfqpoint{0.158220in}{1.773901in}}%
\pgfpathlineto{\pgfqpoint{0.159731in}{1.769133in}}%
\pgfpathlineto{\pgfqpoint{0.161449in}{1.764900in}}%
\pgfpathlineto{\pgfqpoint{0.165240in}{1.755898in}}%
\pgfpathlineto{\pgfqpoint{0.168329in}{1.746897in}}%
\pgfpathlineto{\pgfqpoint{0.169463in}{1.740640in}}%
\pgfpathlineto{\pgfqpoint{0.170117in}{1.737895in}}%
\pgfpathlineto{\pgfqpoint{0.169552in}{1.728893in}}%
\pgfpathlineto{\pgfqpoint{0.169463in}{1.728681in}}%
\pgfpathlineto{\pgfqpoint{0.166655in}{1.719892in}}%
\pgfpathlineto{\pgfqpoint{0.161201in}{1.710890in}}%
\pgfpathlineto{\pgfqpoint{0.159731in}{1.709217in}}%
\pgfpathlineto{\pgfqpoint{0.154706in}{1.701889in}}%
\pgfpathlineto{\pgfqpoint{0.150000in}{1.696446in}}%
\pgfpathlineto{\pgfqpoint{0.150000in}{1.692887in}}%
\pgfpathlineto{\pgfqpoint{0.150000in}{1.683886in}}%
\pgfpathlineto{\pgfqpoint{0.150000in}{1.674884in}}%
\pgfpathlineto{\pgfqpoint{0.150000in}{1.665883in}}%
\pgfpathlineto{\pgfqpoint{0.150000in}{1.656881in}}%
\pgfpathlineto{\pgfqpoint{0.150000in}{1.647880in}}%
\pgfpathlineto{\pgfqpoint{0.150000in}{1.638878in}}%
\pgfpathlineto{\pgfqpoint{0.150000in}{1.629876in}}%
\pgfpathlineto{\pgfqpoint{0.150000in}{1.620875in}}%
\pgfpathlineto{\pgfqpoint{0.150000in}{1.611873in}}%
\pgfpathlineto{\pgfqpoint{0.150000in}{1.602872in}}%
\pgfpathlineto{\pgfqpoint{0.150000in}{1.593870in}}%
\pgfpathlineto{\pgfqpoint{0.150000in}{1.584869in}}%
\pgfpathlineto{\pgfqpoint{0.150000in}{1.575867in}}%
\pgfpathlineto{\pgfqpoint{0.150000in}{1.566866in}}%
\pgfpathlineto{\pgfqpoint{0.150000in}{1.557864in}}%
\pgfpathlineto{\pgfqpoint{0.150000in}{1.548863in}}%
\pgfpathlineto{\pgfqpoint{0.150000in}{1.539861in}}%
\pgfpathlineto{\pgfqpoint{0.150000in}{1.530859in}}%
\pgfpathlineto{\pgfqpoint{0.150000in}{1.521858in}}%
\pgfpathlineto{\pgfqpoint{0.150000in}{1.512856in}}%
\pgfpathlineto{\pgfqpoint{0.150000in}{1.503855in}}%
\pgfpathlineto{\pgfqpoint{0.150000in}{1.494853in}}%
\pgfpathlineto{\pgfqpoint{0.150000in}{1.485852in}}%
\pgfpathlineto{\pgfqpoint{0.150000in}{1.476850in}}%
\pgfpathlineto{\pgfqpoint{0.150000in}{1.467849in}}%
\pgfpathlineto{\pgfqpoint{0.150000in}{1.458847in}}%
\pgfpathlineto{\pgfqpoint{0.150000in}{1.449846in}}%
\pgfpathlineto{\pgfqpoint{0.150000in}{1.440844in}}%
\pgfpathlineto{\pgfqpoint{0.150000in}{1.431842in}}%
\pgfpathlineto{\pgfqpoint{0.150000in}{1.422841in}}%
\pgfpathlineto{\pgfqpoint{0.150000in}{1.413839in}}%
\pgfpathlineto{\pgfqpoint{0.150000in}{1.404838in}}%
\pgfpathlineto{\pgfqpoint{0.150000in}{1.395836in}}%
\pgfpathlineto{\pgfqpoint{0.150000in}{1.386835in}}%
\pgfpathlineto{\pgfqpoint{0.150000in}{1.377833in}}%
\pgfpathlineto{\pgfqpoint{0.150000in}{1.368832in}}%
\pgfpathlineto{\pgfqpoint{0.150000in}{1.359830in}}%
\pgfpathlineto{\pgfqpoint{0.150000in}{1.350829in}}%
\pgfpathlineto{\pgfqpoint{0.150000in}{1.341827in}}%
\pgfpathlineto{\pgfqpoint{0.150000in}{1.332825in}}%
\pgfpathlineto{\pgfqpoint{0.150000in}{1.323824in}}%
\pgfpathlineto{\pgfqpoint{0.150000in}{1.314822in}}%
\pgfpathlineto{\pgfqpoint{0.150000in}{1.305821in}}%
\pgfpathlineto{\pgfqpoint{0.150000in}{1.296819in}}%
\pgfpathlineto{\pgfqpoint{0.150000in}{1.287818in}}%
\pgfpathlineto{\pgfqpoint{0.150000in}{1.278816in}}%
\pgfpathlineto{\pgfqpoint{0.150000in}{1.269815in}}%
\pgfpathlineto{\pgfqpoint{0.150000in}{1.260813in}}%
\pgfpathlineto{\pgfqpoint{0.150000in}{1.251812in}}%
\pgfpathlineto{\pgfqpoint{0.150000in}{1.242810in}}%
\pgfpathlineto{\pgfqpoint{0.150000in}{1.233808in}}%
\pgfpathlineto{\pgfqpoint{0.150000in}{1.224807in}}%
\pgfpathlineto{\pgfqpoint{0.150000in}{1.215805in}}%
\pgfpathlineto{\pgfqpoint{0.150000in}{1.206804in}}%
\pgfpathlineto{\pgfqpoint{0.150000in}{1.203246in}}%
\pgfpathlineto{\pgfqpoint{0.154706in}{1.197802in}}%
\pgfpathlineto{\pgfqpoint{0.159731in}{1.190475in}}%
\pgfpathlineto{\pgfqpoint{0.161201in}{1.188801in}}%
\pgfpathlineto{\pgfqpoint{0.166655in}{1.179799in}}%
\pgfpathlineto{\pgfqpoint{0.169463in}{1.171010in}}%
\pgfpathlineto{\pgfqpoint{0.169552in}{1.170798in}}%
\pgfpathlineto{\pgfqpoint{0.170117in}{1.161796in}}%
\pgfpathlineto{\pgfqpoint{0.169463in}{1.159052in}}%
\pgfpathlineto{\pgfqpoint{0.168329in}{1.152795in}}%
\pgfpathlineto{\pgfqpoint{0.165240in}{1.143793in}}%
\pgfpathlineto{\pgfqpoint{0.161449in}{1.134792in}}%
\pgfpathlineto{\pgfqpoint{0.159731in}{1.130558in}}%
\pgfpathlineto{\pgfqpoint{0.158220in}{1.125790in}}%
\pgfpathlineto{\pgfqpoint{0.156132in}{1.116788in}}%
\pgfpathlineto{\pgfqpoint{0.155400in}{1.107787in}}%
\pgfpathlineto{\pgfqpoint{0.156494in}{1.098785in}}%
\pgfpathlineto{\pgfqpoint{0.159717in}{1.089784in}}%
\pgfpathlineto{\pgfqpoint{0.159731in}{1.089761in}}%
\pgfpathlineto{\pgfqpoint{0.166695in}{1.080782in}}%
\pgfpathlineto{\pgfqpoint{0.169463in}{1.078216in}}%
\pgfpathlineto{\pgfqpoint{0.178595in}{1.071781in}}%
\pgfpathlineto{\pgfqpoint{0.179194in}{1.071440in}}%
\pgfpathlineto{\pgfqpoint{0.188926in}{1.067212in}}%
\pgfpathlineto{\pgfqpoint{0.198657in}{1.063716in}}%
\pgfpathlineto{\pgfqpoint{0.201343in}{1.062779in}}%
\pgfpathlineto{\pgfqpoint{0.208389in}{1.060632in}}%
\pgfpathlineto{\pgfqpoint{0.218120in}{1.057209in}}%
\pgfpathlineto{\pgfqpoint{0.225751in}{1.053778in}}%
\pgfpathlineto{\pgfqpoint{0.227852in}{1.052895in}}%
\pgfpathlineto{\pgfqpoint{0.237583in}{1.047553in}}%
\pgfpathlineto{\pgfqpoint{0.241537in}{1.044776in}}%
\pgfpathlineto{\pgfqpoint{0.247315in}{1.040756in}}%
\pgfpathlineto{\pgfqpoint{0.253135in}{1.035775in}}%
\pgfpathlineto{\pgfqpoint{0.257046in}{1.032262in}}%
\pgfpathlineto{\pgfqpoint{0.262236in}{1.026773in}}%
\pgfpathlineto{\pgfqpoint{0.266778in}{1.021420in}}%
\pgfpathlineto{\pgfqpoint{0.269525in}{1.017771in}}%
\pgfpathlineto{\pgfqpoint{0.275205in}{1.008770in}}%
\pgfpathlineto{\pgfqpoint{0.276509in}{1.006090in}}%
\pgfpathlineto{\pgfqpoint{0.279358in}{0.999768in}}%
\pgfpathlineto{\pgfqpoint{0.282143in}{0.990767in}}%
\pgfpathlineto{\pgfqpoint{0.283722in}{0.981765in}}%
\pgfpathlineto{\pgfqpoint{0.284287in}{0.972764in}}%
\pgfpathlineto{\pgfqpoint{0.284133in}{0.963762in}}%
\pgfpathlineto{\pgfqpoint{0.283633in}{0.954761in}}%
\pgfpathlineto{\pgfqpoint{0.283202in}{0.945759in}}%
\pgfpathlineto{\pgfqpoint{0.283253in}{0.936758in}}%
\pgfpathlineto{\pgfqpoint{0.284160in}{0.927756in}}%
\pgfpathlineto{\pgfqpoint{0.286218in}{0.918754in}}%
\pgfpathlineto{\pgfqpoint{0.286241in}{0.918695in}}%
\pgfpathlineto{\pgfqpoint{0.289503in}{0.909753in}}%
\pgfpathlineto{\pgfqpoint{0.294147in}{0.900751in}}%
\pgfpathlineto{\pgfqpoint{0.295972in}{0.897985in}}%
\pgfpathlineto{\pgfqpoint{0.300115in}{0.891750in}}%
\pgfpathlineto{\pgfqpoint{0.305704in}{0.884664in}}%
\pgfpathlineto{\pgfqpoint{0.307291in}{0.882748in}}%
\pgfpathlineto{\pgfqpoint{0.315435in}{0.873952in}}%
\pgfpathlineto{\pgfqpoint{0.315644in}{0.873747in}}%
\pgfpathlineto{\pgfqpoint{0.325167in}{0.864908in}}%
\pgfpathlineto{\pgfqpoint{0.325368in}{0.864745in}}%
\pgfpathlineto{\pgfqpoint{0.334898in}{0.857135in}}%
\pgfpathlineto{\pgfqpoint{0.337019in}{0.855744in}}%
\pgfpathlineto{\pgfqpoint{0.344630in}{0.850544in}}%
\pgfpathlineto{\pgfqpoint{0.351838in}{0.846742in}}%
\pgfpathlineto{\pgfqpoint{0.354361in}{0.845286in}}%
\pgfpathlineto{\pgfqpoint{0.364093in}{0.841251in}}%
\pgfpathlineto{\pgfqpoint{0.373824in}{0.838572in}}%
\pgfpathlineto{\pgfqpoint{0.378669in}{0.837741in}}%
\pgfpathlineto{\pgfqpoint{0.383556in}{0.836773in}}%
\pgfpathlineto{\pgfqpoint{0.393287in}{0.835501in}}%
\pgfpathlineto{\pgfqpoint{0.403018in}{0.834321in}}%
\pgfpathlineto{\pgfqpoint{0.412750in}{0.832673in}}%
\pgfpathlineto{\pgfqpoint{0.422481in}{0.830048in}}%
\pgfpathlineto{\pgfqpoint{0.425665in}{0.828739in}}%
\pgfpathlineto{\pgfqpoint{0.432213in}{0.825447in}}%
\pgfpathlineto{\pgfqpoint{0.440274in}{0.819737in}}%
\pgfpathlineto{\pgfqpoint{0.441944in}{0.818193in}}%
\pgfpathlineto{\pgfqpoint{0.448117in}{0.810736in}}%
\pgfpathlineto{\pgfqpoint{0.451676in}{0.804679in}}%
\pgfpathlineto{\pgfqpoint{0.453091in}{0.801734in}}%
\pgfpathlineto{\pgfqpoint{0.455929in}{0.792733in}}%
\pgfpathlineto{\pgfqpoint{0.457711in}{0.783731in}}%
\pgfpathlineto{\pgfqpoint{0.458986in}{0.774730in}}%
\pgfpathlineto{\pgfqpoint{0.460361in}{0.765728in}}%
\pgfpathlineto{\pgfqpoint{0.461407in}{0.761208in}}%
\pgfpathlineto{\pgfqpoint{0.462307in}{0.756727in}}%
\pgfpathlineto{\pgfqpoint{0.465203in}{0.747725in}}%
\pgfpathlineto{\pgfqpoint{0.469565in}{0.738724in}}%
\pgfpathlineto{\pgfqpoint{0.471139in}{0.736390in}}%
\pgfpathlineto{\pgfqpoint{0.475249in}{0.729722in}}%
\pgfpathlineto{\pgfqpoint{0.480870in}{0.722682in}}%
\pgfpathlineto{\pgfqpoint{0.482374in}{0.720720in}}%
\pgfpathlineto{\pgfqpoint{0.490602in}{0.711905in}}%
\pgfpathlineto{\pgfqpoint{0.490777in}{0.711719in}}%
\pgfpathlineto{\pgfqpoint{0.500333in}{0.702911in}}%
\pgfpathlineto{\pgfqpoint{0.500555in}{0.702717in}}%
\pgfpathlineto{\pgfqpoint{0.510065in}{0.695185in}}%
\pgfpathlineto{\pgfqpoint{0.512136in}{0.693716in}}%
\pgfpathlineto{\pgfqpoint{0.519796in}{0.688546in}}%
\pgfpathlineto{\pgfqpoint{0.526537in}{0.684714in}}%
\pgfpathlineto{\pgfqpoint{0.529528in}{0.683026in}}%
\pgfpathlineto{\pgfqpoint{0.539259in}{0.678731in}}%
\pgfpathlineto{\pgfqpoint{0.548927in}{0.675713in}}%
\pgfpathlineto{\pgfqpoint{0.548991in}{0.675692in}}%
\pgfpathlineto{\pgfqpoint{0.558722in}{0.673788in}}%
\pgfpathlineto{\pgfqpoint{0.568454in}{0.672949in}}%
\pgfpathlineto{\pgfqpoint{0.578185in}{0.672902in}}%
\pgfpathlineto{\pgfqpoint{0.587917in}{0.673301in}}%
\pgfpathlineto{\pgfqpoint{0.597648in}{0.673763in}}%
\pgfpathlineto{\pgfqpoint{0.607380in}{0.673906in}}%
\pgfpathlineto{\pgfqpoint{0.617111in}{0.673383in}}%
\pgfpathlineto{\pgfqpoint{0.626842in}{0.671922in}}%
\pgfpathlineto{\pgfqpoint{0.636574in}{0.669346in}}%
\pgfpathlineto{\pgfqpoint{0.643408in}{0.666711in}}%
\pgfpathlineto{\pgfqpoint{0.646305in}{0.665505in}}%
\pgfpathlineto{\pgfqpoint{0.656037in}{0.660251in}}%
\pgfpathlineto{\pgfqpoint{0.659981in}{0.657710in}}%
\pgfpathlineto{\pgfqpoint{0.665768in}{0.653509in}}%
\pgfpathlineto{\pgfqpoint{0.671702in}{0.648708in}}%
\pgfpathlineto{\pgfqpoint{0.675500in}{0.645090in}}%
\pgfpathlineto{\pgfqpoint{0.680886in}{0.639707in}}%
\pgfpathlineto{\pgfqpoint{0.685231in}{0.634362in}}%
\pgfpathlineto{\pgfqpoint{0.688233in}{0.630705in}}%
\pgfpathlineto{\pgfqpoint{0.694009in}{0.621703in}}%
\pgfpathlineto{\pgfqpoint{0.694963in}{0.619760in}}%
\pgfpathlineto{\pgfqpoint{0.698672in}{0.612702in}}%
\pgfpathlineto{\pgfqpoint{0.702373in}{0.603700in}}%
\pgfpathlineto{\pgfqpoint{0.704694in}{0.597183in}}%
\pgfpathlineto{\pgfqpoint{0.705707in}{0.594699in}}%
\pgfpathlineto{\pgfqpoint{0.709487in}{0.585697in}}%
\pgfpathlineto{\pgfqpoint{0.714058in}{0.576696in}}%
\pgfpathlineto{\pgfqpoint{0.714426in}{0.576142in}}%
\pgfpathlineto{\pgfqpoint{0.721382in}{0.567694in}}%
\pgfpathlineto{\pgfqpoint{0.724157in}{0.565134in}}%
\pgfpathlineto{\pgfqpoint{0.733864in}{0.558693in}}%
\pgfpathclose%
\pgfpathmoveto{\pgfqpoint{1.101043in}{0.675713in}}%
\pgfpathlineto{\pgfqpoint{1.093953in}{0.677794in}}%
\pgfpathlineto{\pgfqpoint{1.084222in}{0.681415in}}%
\pgfpathlineto{\pgfqpoint{1.076297in}{0.684714in}}%
\pgfpathlineto{\pgfqpoint{1.074491in}{0.685472in}}%
\pgfpathlineto{\pgfqpoint{1.064759in}{0.689613in}}%
\pgfpathlineto{\pgfqpoint{1.055028in}{0.693443in}}%
\pgfpathlineto{\pgfqpoint{1.054193in}{0.693716in}}%
\pgfpathlineto{\pgfqpoint{1.045296in}{0.696774in}}%
\pgfpathlineto{\pgfqpoint{1.035565in}{0.699160in}}%
\pgfpathlineto{\pgfqpoint{1.025833in}{0.700407in}}%
\pgfpathlineto{\pgfqpoint{1.016102in}{0.700453in}}%
\pgfpathlineto{\pgfqpoint{1.006370in}{0.699360in}}%
\pgfpathlineto{\pgfqpoint{0.996639in}{0.697308in}}%
\pgfpathlineto{\pgfqpoint{0.986907in}{0.694576in}}%
\pgfpathlineto{\pgfqpoint{0.984169in}{0.693716in}}%
\pgfpathlineto{\pgfqpoint{0.977176in}{0.691624in}}%
\pgfpathlineto{\pgfqpoint{0.967444in}{0.688775in}}%
\pgfpathlineto{\pgfqpoint{0.957713in}{0.686360in}}%
\pgfpathlineto{\pgfqpoint{0.947981in}{0.684715in}}%
\pgfpathlineto{\pgfqpoint{0.947974in}{0.684714in}}%
\pgfpathlineto{\pgfqpoint{0.938250in}{0.684109in}}%
\pgfpathlineto{\pgfqpoint{0.928518in}{0.684703in}}%
\pgfpathlineto{\pgfqpoint{0.928458in}{0.684714in}}%
\pgfpathlineto{\pgfqpoint{0.918787in}{0.686566in}}%
\pgfpathlineto{\pgfqpoint{0.909055in}{0.689637in}}%
\pgfpathlineto{\pgfqpoint{0.899390in}{0.693716in}}%
\pgfpathlineto{\pgfqpoint{0.899324in}{0.693745in}}%
\pgfpathlineto{\pgfqpoint{0.889592in}{0.698868in}}%
\pgfpathlineto{\pgfqpoint{0.882878in}{0.702717in}}%
\pgfpathlineto{\pgfqpoint{0.879861in}{0.704617in}}%
\pgfpathlineto{\pgfqpoint{0.870129in}{0.710818in}}%
\pgfpathlineto{\pgfqpoint{0.868628in}{0.711719in}}%
\pgfpathlineto{\pgfqpoint{0.860398in}{0.717416in}}%
\pgfpathlineto{\pgfqpoint{0.854926in}{0.720720in}}%
\pgfpathlineto{\pgfqpoint{0.850667in}{0.723851in}}%
\pgfpathlineto{\pgfqpoint{0.840935in}{0.729525in}}%
\pgfpathlineto{\pgfqpoint{0.840445in}{0.729722in}}%
\pgfpathlineto{\pgfqpoint{0.831204in}{0.734540in}}%
\pgfpathlineto{\pgfqpoint{0.821472in}{0.737214in}}%
\pgfpathlineto{\pgfqpoint{0.811741in}{0.737613in}}%
\pgfpathlineto{\pgfqpoint{0.802009in}{0.736097in}}%
\pgfpathlineto{\pgfqpoint{0.792278in}{0.733223in}}%
\pgfpathlineto{\pgfqpoint{0.782614in}{0.729722in}}%
\pgfpathlineto{\pgfqpoint{0.782546in}{0.729703in}}%
\pgfpathlineto{\pgfqpoint{0.772815in}{0.727081in}}%
\pgfpathlineto{\pgfqpoint{0.763083in}{0.725161in}}%
\pgfpathlineto{\pgfqpoint{0.753352in}{0.724489in}}%
\pgfpathlineto{\pgfqpoint{0.743620in}{0.725495in}}%
\pgfpathlineto{\pgfqpoint{0.733889in}{0.728457in}}%
\pgfpathlineto{\pgfqpoint{0.731433in}{0.729722in}}%
\pgfpathlineto{\pgfqpoint{0.724157in}{0.734577in}}%
\pgfpathlineto{\pgfqpoint{0.719687in}{0.738724in}}%
\pgfpathlineto{\pgfqpoint{0.714426in}{0.745516in}}%
\pgfpathlineto{\pgfqpoint{0.713046in}{0.747725in}}%
\pgfpathlineto{\pgfqpoint{0.709316in}{0.756727in}}%
\pgfpathlineto{\pgfqpoint{0.706989in}{0.765728in}}%
\pgfpathlineto{\pgfqpoint{0.705452in}{0.774730in}}%
\pgfpathlineto{\pgfqpoint{0.704694in}{0.779518in}}%
\pgfpathlineto{\pgfqpoint{0.704112in}{0.783731in}}%
\pgfpathlineto{\pgfqpoint{0.702372in}{0.792733in}}%
\pgfpathlineto{\pgfqpoint{0.699602in}{0.801734in}}%
\pgfpathlineto{\pgfqpoint{0.695376in}{0.810736in}}%
\pgfpathlineto{\pgfqpoint{0.694963in}{0.811361in}}%
\pgfpathlineto{\pgfqpoint{0.689791in}{0.819737in}}%
\pgfpathlineto{\pgfqpoint{0.685231in}{0.825392in}}%
\pgfpathlineto{\pgfqpoint{0.682558in}{0.828739in}}%
\pgfpathlineto{\pgfqpoint{0.675500in}{0.835968in}}%
\pgfpathlineto{\pgfqpoint{0.673684in}{0.837741in}}%
\pgfpathlineto{\pgfqpoint{0.665768in}{0.844440in}}%
\pgfpathlineto{\pgfqpoint{0.662737in}{0.846742in}}%
\pgfpathlineto{\pgfqpoint{0.656037in}{0.851393in}}%
\pgfpathlineto{\pgfqpoint{0.648561in}{0.855744in}}%
\pgfpathlineto{\pgfqpoint{0.646305in}{0.857004in}}%
\pgfpathlineto{\pgfqpoint{0.636574in}{0.861196in}}%
\pgfpathlineto{\pgfqpoint{0.626842in}{0.864075in}}%
\pgfpathlineto{\pgfqpoint{0.622848in}{0.864745in}}%
\pgfpathlineto{\pgfqpoint{0.617111in}{0.865715in}}%
\pgfpathlineto{\pgfqpoint{0.607380in}{0.866304in}}%
\pgfpathlineto{\pgfqpoint{0.597648in}{0.866143in}}%
\pgfpathlineto{\pgfqpoint{0.587917in}{0.865623in}}%
\pgfpathlineto{\pgfqpoint{0.578185in}{0.865173in}}%
\pgfpathlineto{\pgfqpoint{0.568454in}{0.865227in}}%
\pgfpathlineto{\pgfqpoint{0.558722in}{0.866171in}}%
\pgfpathlineto{\pgfqpoint{0.548991in}{0.868316in}}%
\pgfpathlineto{\pgfqpoint{0.539259in}{0.871859in}}%
\pgfpathlineto{\pgfqpoint{0.535591in}{0.873747in}}%
\pgfpathlineto{\pgfqpoint{0.529528in}{0.877053in}}%
\pgfpathlineto{\pgfqpoint{0.521360in}{0.882748in}}%
\pgfpathlineto{\pgfqpoint{0.519796in}{0.883967in}}%
\pgfpathlineto{\pgfqpoint{0.511382in}{0.891750in}}%
\pgfpathlineto{\pgfqpoint{0.510065in}{0.893197in}}%
\pgfpathlineto{\pgfqpoint{0.503907in}{0.900751in}}%
\pgfpathlineto{\pgfqpoint{0.500333in}{0.906360in}}%
\pgfpathlineto{\pgfqpoint{0.498292in}{0.909753in}}%
\pgfpathlineto{\pgfqpoint{0.494462in}{0.918754in}}%
\pgfpathlineto{\pgfqpoint{0.492143in}{0.927756in}}%
\pgfpathlineto{\pgfqpoint{0.491122in}{0.936758in}}%
\pgfpathlineto{\pgfqpoint{0.491065in}{0.945759in}}%
\pgfpathlineto{\pgfqpoint{0.491550in}{0.954761in}}%
\pgfpathlineto{\pgfqpoint{0.492113in}{0.963762in}}%
\pgfpathlineto{\pgfqpoint{0.492287in}{0.972764in}}%
\pgfpathlineto{\pgfqpoint{0.491651in}{0.981765in}}%
\pgfpathlineto{\pgfqpoint{0.490602in}{0.987072in}}%
\pgfpathlineto{\pgfqpoint{0.489877in}{0.990767in}}%
\pgfpathlineto{\pgfqpoint{0.486765in}{0.999768in}}%
\pgfpathlineto{\pgfqpoint{0.482233in}{1.008770in}}%
\pgfpathlineto{\pgfqpoint{0.480870in}{1.010856in}}%
\pgfpathlineto{\pgfqpoint{0.476167in}{1.017771in}}%
\pgfpathlineto{\pgfqpoint{0.471139in}{1.023969in}}%
\pgfpathlineto{\pgfqpoint{0.468650in}{1.026773in}}%
\pgfpathlineto{\pgfqpoint{0.461407in}{1.034095in}}%
\pgfpathlineto{\pgfqpoint{0.459491in}{1.035775in}}%
\pgfpathlineto{\pgfqpoint{0.451676in}{1.042303in}}%
\pgfpathlineto{\pgfqpoint{0.448058in}{1.044776in}}%
\pgfpathlineto{\pgfqpoint{0.441944in}{1.048994in}}%
\pgfpathlineto{\pgfqpoint{0.432889in}{1.053778in}}%
\pgfpathlineto{\pgfqpoint{0.432213in}{1.054160in}}%
\pgfpathlineto{\pgfqpoint{0.422481in}{1.058068in}}%
\pgfpathlineto{\pgfqpoint{0.412750in}{1.060631in}}%
\pgfpathlineto{\pgfqpoint{0.403018in}{1.062240in}}%
\pgfpathlineto{\pgfqpoint{0.398464in}{1.062779in}}%
\pgfpathlineto{\pgfqpoint{0.393287in}{1.063480in}}%
\pgfpathlineto{\pgfqpoint{0.383556in}{1.064901in}}%
\pgfpathlineto{\pgfqpoint{0.373824in}{1.067054in}}%
\pgfpathlineto{\pgfqpoint{0.364093in}{1.070505in}}%
\pgfpathlineto{\pgfqpoint{0.361704in}{1.071781in}}%
\pgfpathlineto{\pgfqpoint{0.354361in}{1.076647in}}%
\pgfpathlineto{\pgfqpoint{0.349878in}{1.080782in}}%
\pgfpathlineto{\pgfqpoint{0.344630in}{1.087512in}}%
\pgfpathlineto{\pgfqpoint{0.343262in}{1.089784in}}%
\pgfpathlineto{\pgfqpoint{0.340060in}{1.098785in}}%
\pgfpathlineto{\pgfqpoint{0.338972in}{1.107787in}}%
\pgfpathlineto{\pgfqpoint{0.339699in}{1.116788in}}%
\pgfpathlineto{\pgfqpoint{0.341774in}{1.125790in}}%
\pgfpathlineto{\pgfqpoint{0.344609in}{1.134792in}}%
\pgfpathlineto{\pgfqpoint{0.344630in}{1.134854in}}%
\pgfpathlineto{\pgfqpoint{0.348415in}{1.143793in}}%
\pgfpathlineto{\pgfqpoint{0.351521in}{1.152795in}}%
\pgfpathlineto{\pgfqpoint{0.353161in}{1.161796in}}%
\pgfpathlineto{\pgfqpoint{0.352729in}{1.170798in}}%
\pgfpathlineto{\pgfqpoint{0.349838in}{1.179799in}}%
\pgfpathlineto{\pgfqpoint{0.344630in}{1.188348in}}%
\pgfpathlineto{\pgfqpoint{0.344416in}{1.188801in}}%
\pgfpathlineto{\pgfqpoint{0.338283in}{1.197802in}}%
\pgfpathlineto{\pgfqpoint{0.334898in}{1.201742in}}%
\pgfpathlineto{\pgfqpoint{0.331326in}{1.206804in}}%
\pgfpathlineto{\pgfqpoint{0.325167in}{1.214417in}}%
\pgfpathlineto{\pgfqpoint{0.324192in}{1.215805in}}%
\pgfpathlineto{\pgfqpoint{0.317489in}{1.224807in}}%
\pgfpathlineto{\pgfqpoint{0.315435in}{1.227597in}}%
\pgfpathlineto{\pgfqpoint{0.311274in}{1.233808in}}%
\pgfpathlineto{\pgfqpoint{0.305735in}{1.242810in}}%
\pgfpathlineto{\pgfqpoint{0.305704in}{1.242871in}}%
\pgfpathlineto{\pgfqpoint{0.301294in}{1.251812in}}%
\pgfpathlineto{\pgfqpoint{0.297974in}{1.260813in}}%
\pgfpathlineto{\pgfqpoint{0.295972in}{1.269758in}}%
\pgfpathlineto{\pgfqpoint{0.295960in}{1.269815in}}%
\pgfpathlineto{\pgfqpoint{0.295318in}{1.278816in}}%
\pgfpathlineto{\pgfqpoint{0.295972in}{1.287811in}}%
\pgfpathlineto{\pgfqpoint{0.295973in}{1.287818in}}%
\pgfpathlineto{\pgfqpoint{0.297752in}{1.296819in}}%
\pgfpathlineto{\pgfqpoint{0.300363in}{1.305821in}}%
\pgfpathlineto{\pgfqpoint{0.303443in}{1.314822in}}%
\pgfpathlineto{\pgfqpoint{0.305704in}{1.321291in}}%
\pgfpathlineto{\pgfqpoint{0.306634in}{1.323824in}}%
\pgfpathlineto{\pgfqpoint{0.309587in}{1.332825in}}%
\pgfpathlineto{\pgfqpoint{0.311805in}{1.341827in}}%
\pgfpathlineto{\pgfqpoint{0.312987in}{1.350829in}}%
\pgfpathlineto{\pgfqpoint{0.312937in}{1.359830in}}%
\pgfpathlineto{\pgfqpoint{0.311589in}{1.368832in}}%
\pgfpathlineto{\pgfqpoint{0.309010in}{1.377833in}}%
\pgfpathlineto{\pgfqpoint{0.305704in}{1.386063in}}%
\pgfpathlineto{\pgfqpoint{0.305409in}{1.386835in}}%
\pgfpathlineto{\pgfqpoint{0.301268in}{1.395836in}}%
\pgfpathlineto{\pgfqpoint{0.296791in}{1.404838in}}%
\pgfpathlineto{\pgfqpoint{0.295972in}{1.406509in}}%
\pgfpathlineto{\pgfqpoint{0.292405in}{1.413839in}}%
\pgfpathlineto{\pgfqpoint{0.288491in}{1.422841in}}%
\pgfpathlineto{\pgfqpoint{0.286241in}{1.429399in}}%
\pgfpathlineto{\pgfqpoint{0.285373in}{1.431842in}}%
\pgfpathlineto{\pgfqpoint{0.283327in}{1.440844in}}%
\pgfpathlineto{\pgfqpoint{0.282623in}{1.449846in}}%
\pgfpathlineto{\pgfqpoint{0.283327in}{1.458847in}}%
\pgfpathlineto{\pgfqpoint{0.285373in}{1.467849in}}%
\pgfpathlineto{\pgfqpoint{0.286241in}{1.470292in}}%
\pgfpathlineto{\pgfqpoint{0.288491in}{1.476850in}}%
\pgfpathlineto{\pgfqpoint{0.292405in}{1.485852in}}%
\pgfpathlineto{\pgfqpoint{0.295972in}{1.493182in}}%
\pgfpathlineto{\pgfqpoint{0.296791in}{1.494853in}}%
\pgfpathlineto{\pgfqpoint{0.301268in}{1.503855in}}%
\pgfpathlineto{\pgfqpoint{0.305409in}{1.512856in}}%
\pgfpathlineto{\pgfqpoint{0.305704in}{1.513628in}}%
\pgfpathlineto{\pgfqpoint{0.309010in}{1.521858in}}%
\pgfpathlineto{\pgfqpoint{0.311589in}{1.530859in}}%
\pgfpathlineto{\pgfqpoint{0.312937in}{1.539861in}}%
\pgfpathlineto{\pgfqpoint{0.312987in}{1.548863in}}%
\pgfpathlineto{\pgfqpoint{0.311805in}{1.557864in}}%
\pgfpathlineto{\pgfqpoint{0.309587in}{1.566866in}}%
\pgfpathlineto{\pgfqpoint{0.306634in}{1.575867in}}%
\pgfpathlineto{\pgfqpoint{0.305704in}{1.578400in}}%
\pgfpathlineto{\pgfqpoint{0.303443in}{1.584869in}}%
\pgfpathlineto{\pgfqpoint{0.300363in}{1.593870in}}%
\pgfpathlineto{\pgfqpoint{0.297752in}{1.602872in}}%
\pgfpathlineto{\pgfqpoint{0.295973in}{1.611873in}}%
\pgfpathlineto{\pgfqpoint{0.295972in}{1.611880in}}%
\pgfpathlineto{\pgfqpoint{0.295318in}{1.620875in}}%
\pgfpathlineto{\pgfqpoint{0.295960in}{1.629876in}}%
\pgfpathlineto{\pgfqpoint{0.295972in}{1.629933in}}%
\pgfpathlineto{\pgfqpoint{0.297974in}{1.638878in}}%
\pgfpathlineto{\pgfqpoint{0.301294in}{1.647880in}}%
\pgfpathlineto{\pgfqpoint{0.305704in}{1.656820in}}%
\pgfpathlineto{\pgfqpoint{0.305735in}{1.656881in}}%
\pgfpathlineto{\pgfqpoint{0.311274in}{1.665883in}}%
\pgfpathlineto{\pgfqpoint{0.315435in}{1.672094in}}%
\pgfpathlineto{\pgfqpoint{0.317489in}{1.674884in}}%
\pgfpathlineto{\pgfqpoint{0.324192in}{1.683886in}}%
\pgfpathlineto{\pgfqpoint{0.325167in}{1.685274in}}%
\pgfpathlineto{\pgfqpoint{0.331326in}{1.692887in}}%
\pgfpathlineto{\pgfqpoint{0.334898in}{1.697949in}}%
\pgfpathlineto{\pgfqpoint{0.338283in}{1.701889in}}%
\pgfpathlineto{\pgfqpoint{0.344416in}{1.710890in}}%
\pgfpathlineto{\pgfqpoint{0.344630in}{1.711343in}}%
\pgfpathlineto{\pgfqpoint{0.349838in}{1.719892in}}%
\pgfpathlineto{\pgfqpoint{0.352729in}{1.728893in}}%
\pgfpathlineto{\pgfqpoint{0.353161in}{1.737895in}}%
\pgfpathlineto{\pgfqpoint{0.351521in}{1.746897in}}%
\pgfpathlineto{\pgfqpoint{0.348415in}{1.755898in}}%
\pgfpathlineto{\pgfqpoint{0.344630in}{1.764837in}}%
\pgfpathlineto{\pgfqpoint{0.344609in}{1.764900in}}%
\pgfpathlineto{\pgfqpoint{0.341774in}{1.773901in}}%
\pgfpathlineto{\pgfqpoint{0.339699in}{1.782903in}}%
\pgfpathlineto{\pgfqpoint{0.338972in}{1.791904in}}%
\pgfpathlineto{\pgfqpoint{0.340060in}{1.800906in}}%
\pgfpathlineto{\pgfqpoint{0.343262in}{1.809907in}}%
\pgfpathlineto{\pgfqpoint{0.344630in}{1.812179in}}%
\pgfpathlineto{\pgfqpoint{0.349878in}{1.818909in}}%
\pgfpathlineto{\pgfqpoint{0.354361in}{1.823044in}}%
\pgfpathlineto{\pgfqpoint{0.361704in}{1.827910in}}%
\pgfpathlineto{\pgfqpoint{0.364093in}{1.829186in}}%
\pgfpathlineto{\pgfqpoint{0.373824in}{1.832637in}}%
\pgfpathlineto{\pgfqpoint{0.383556in}{1.834790in}}%
\pgfpathlineto{\pgfqpoint{0.393287in}{1.836211in}}%
\pgfpathlineto{\pgfqpoint{0.398464in}{1.836912in}}%
\pgfpathlineto{\pgfqpoint{0.403018in}{1.837451in}}%
\pgfpathlineto{\pgfqpoint{0.412750in}{1.839060in}}%
\pgfpathlineto{\pgfqpoint{0.422481in}{1.841623in}}%
\pgfpathlineto{\pgfqpoint{0.432213in}{1.845531in}}%
\pgfpathlineto{\pgfqpoint{0.432889in}{1.845914in}}%
\pgfpathlineto{\pgfqpoint{0.441944in}{1.850697in}}%
\pgfpathlineto{\pgfqpoint{0.448058in}{1.854915in}}%
\pgfpathlineto{\pgfqpoint{0.451676in}{1.857388in}}%
\pgfpathlineto{\pgfqpoint{0.459491in}{1.863917in}}%
\pgfpathlineto{\pgfqpoint{0.461407in}{1.865596in}}%
\pgfpathlineto{\pgfqpoint{0.468650in}{1.872918in}}%
\pgfpathlineto{\pgfqpoint{0.471139in}{1.875722in}}%
\pgfpathlineto{\pgfqpoint{0.476167in}{1.881920in}}%
\pgfpathlineto{\pgfqpoint{0.480870in}{1.888835in}}%
\pgfpathlineto{\pgfqpoint{0.482233in}{1.890921in}}%
\pgfpathlineto{\pgfqpoint{0.486765in}{1.899923in}}%
\pgfpathlineto{\pgfqpoint{0.489877in}{1.908924in}}%
\pgfpathlineto{\pgfqpoint{0.490602in}{1.912619in}}%
\pgfpathlineto{\pgfqpoint{0.491651in}{1.917926in}}%
\pgfpathlineto{\pgfqpoint{0.492287in}{1.926927in}}%
\pgfpathlineto{\pgfqpoint{0.492113in}{1.935929in}}%
\pgfpathlineto{\pgfqpoint{0.491550in}{1.944931in}}%
\pgfpathlineto{\pgfqpoint{0.491065in}{1.953932in}}%
\pgfpathlineto{\pgfqpoint{0.491122in}{1.962934in}}%
\pgfpathlineto{\pgfqpoint{0.492143in}{1.971935in}}%
\pgfpathlineto{\pgfqpoint{0.494462in}{1.980937in}}%
\pgfpathlineto{\pgfqpoint{0.498292in}{1.989938in}}%
\pgfpathlineto{\pgfqpoint{0.500333in}{1.993331in}}%
\pgfpathlineto{\pgfqpoint{0.503907in}{1.998940in}}%
\pgfpathlineto{\pgfqpoint{0.510065in}{2.006495in}}%
\pgfpathlineto{\pgfqpoint{0.511382in}{2.007941in}}%
\pgfpathlineto{\pgfqpoint{0.519796in}{2.015724in}}%
\pgfpathlineto{\pgfqpoint{0.521360in}{2.016943in}}%
\pgfpathlineto{\pgfqpoint{0.529528in}{2.022639in}}%
\pgfpathlineto{\pgfqpoint{0.535591in}{2.025944in}}%
\pgfpathlineto{\pgfqpoint{0.539259in}{2.027832in}}%
\pgfpathlineto{\pgfqpoint{0.548991in}{2.031375in}}%
\pgfpathlineto{\pgfqpoint{0.558722in}{2.033520in}}%
\pgfpathlineto{\pgfqpoint{0.568454in}{2.034465in}}%
\pgfpathlineto{\pgfqpoint{0.578185in}{2.034518in}}%
\pgfpathlineto{\pgfqpoint{0.587917in}{2.034069in}}%
\pgfpathlineto{\pgfqpoint{0.597648in}{2.033548in}}%
\pgfpathlineto{\pgfqpoint{0.607380in}{2.033387in}}%
\pgfpathlineto{\pgfqpoint{0.617111in}{2.033976in}}%
\pgfpathlineto{\pgfqpoint{0.622848in}{2.034946in}}%
\pgfpathlineto{\pgfqpoint{0.626842in}{2.035616in}}%
\pgfpathlineto{\pgfqpoint{0.636574in}{2.038495in}}%
\pgfpathlineto{\pgfqpoint{0.646305in}{2.042687in}}%
\pgfpathlineto{\pgfqpoint{0.648561in}{2.043947in}}%
\pgfpathlineto{\pgfqpoint{0.656037in}{2.048298in}}%
\pgfpathlineto{\pgfqpoint{0.662737in}{2.052949in}}%
\pgfpathlineto{\pgfqpoint{0.665768in}{2.055251in}}%
\pgfpathlineto{\pgfqpoint{0.673684in}{2.061951in}}%
\pgfpathlineto{\pgfqpoint{0.675500in}{2.063723in}}%
\pgfpathlineto{\pgfqpoint{0.682558in}{2.070952in}}%
\pgfpathlineto{\pgfqpoint{0.685231in}{2.074299in}}%
\pgfpathlineto{\pgfqpoint{0.689791in}{2.079954in}}%
\pgfpathlineto{\pgfqpoint{0.694963in}{2.088330in}}%
\pgfpathlineto{\pgfqpoint{0.695376in}{2.088955in}}%
\pgfpathlineto{\pgfqpoint{0.699602in}{2.097957in}}%
\pgfpathlineto{\pgfqpoint{0.702372in}{2.106958in}}%
\pgfpathlineto{\pgfqpoint{0.704112in}{2.115960in}}%
\pgfpathlineto{\pgfqpoint{0.704694in}{2.120173in}}%
\pgfpathlineto{\pgfqpoint{0.705452in}{2.124961in}}%
\pgfpathlineto{\pgfqpoint{0.706989in}{2.133963in}}%
\pgfpathlineto{\pgfqpoint{0.709316in}{2.142964in}}%
\pgfpathlineto{\pgfqpoint{0.713046in}{2.151966in}}%
\pgfpathlineto{\pgfqpoint{0.714426in}{2.154176in}}%
\pgfpathlineto{\pgfqpoint{0.719687in}{2.160968in}}%
\pgfpathlineto{\pgfqpoint{0.724157in}{2.165114in}}%
\pgfpathlineto{\pgfqpoint{0.731433in}{2.169969in}}%
\pgfpathlineto{\pgfqpoint{0.733889in}{2.171234in}}%
\pgfpathlineto{\pgfqpoint{0.743620in}{2.174196in}}%
\pgfpathlineto{\pgfqpoint{0.753352in}{2.175202in}}%
\pgfpathlineto{\pgfqpoint{0.763083in}{2.174530in}}%
\pgfpathlineto{\pgfqpoint{0.772815in}{2.172610in}}%
\pgfpathlineto{\pgfqpoint{0.782546in}{2.169988in}}%
\pgfpathlineto{\pgfqpoint{0.782614in}{2.169969in}}%
\pgfpathlineto{\pgfqpoint{0.792278in}{2.166468in}}%
\pgfpathlineto{\pgfqpoint{0.802009in}{2.163594in}}%
\pgfpathlineto{\pgfqpoint{0.811741in}{2.162078in}}%
\pgfpathlineto{\pgfqpoint{0.821472in}{2.162477in}}%
\pgfpathlineto{\pgfqpoint{0.831204in}{2.165151in}}%
\pgfpathlineto{\pgfqpoint{0.840445in}{2.169969in}}%
\pgfpathlineto{\pgfqpoint{0.840935in}{2.170166in}}%
\pgfpathlineto{\pgfqpoint{0.850667in}{2.175840in}}%
\pgfpathlineto{\pgfqpoint{0.854926in}{2.178971in}}%
\pgfpathlineto{\pgfqpoint{0.860398in}{2.182275in}}%
\pgfpathlineto{\pgfqpoint{0.868628in}{2.187972in}}%
\pgfpathlineto{\pgfqpoint{0.870129in}{2.188873in}}%
\pgfpathlineto{\pgfqpoint{0.879861in}{2.195074in}}%
\pgfpathlineto{\pgfqpoint{0.882878in}{2.196974in}}%
\pgfpathlineto{\pgfqpoint{0.889592in}{2.200823in}}%
\pgfpathlineto{\pgfqpoint{0.899324in}{2.205946in}}%
\pgfpathlineto{\pgfqpoint{0.899390in}{2.205975in}}%
\pgfpathlineto{\pgfqpoint{0.909055in}{2.210054in}}%
\pgfpathlineto{\pgfqpoint{0.918787in}{2.213125in}}%
\pgfpathlineto{\pgfqpoint{0.928458in}{2.214977in}}%
\pgfpathlineto{\pgfqpoint{0.928518in}{2.214988in}}%
\pgfpathlineto{\pgfqpoint{0.938250in}{2.215582in}}%
\pgfpathlineto{\pgfqpoint{0.947974in}{2.214977in}}%
\pgfpathlineto{\pgfqpoint{0.947981in}{2.214976in}}%
\pgfpathlineto{\pgfqpoint{0.957713in}{2.213331in}}%
\pgfpathlineto{\pgfqpoint{0.967444in}{2.210916in}}%
\pgfpathlineto{\pgfqpoint{0.977176in}{2.208067in}}%
\pgfpathlineto{\pgfqpoint{0.984169in}{2.205975in}}%
\pgfpathlineto{\pgfqpoint{0.986907in}{2.205115in}}%
\pgfpathlineto{\pgfqpoint{0.996639in}{2.202384in}}%
\pgfpathlineto{\pgfqpoint{1.006370in}{2.200331in}}%
\pgfpathlineto{\pgfqpoint{1.016102in}{2.199238in}}%
\pgfpathlineto{\pgfqpoint{1.025833in}{2.199284in}}%
\pgfpathlineto{\pgfqpoint{1.035565in}{2.200531in}}%
\pgfpathlineto{\pgfqpoint{1.045296in}{2.202917in}}%
\pgfpathlineto{\pgfqpoint{1.054193in}{2.205975in}}%
\pgfpathlineto{\pgfqpoint{1.055028in}{2.206248in}}%
\pgfpathlineto{\pgfqpoint{1.064759in}{2.210078in}}%
\pgfpathlineto{\pgfqpoint{1.074491in}{2.214219in}}%
\pgfpathlineto{\pgfqpoint{1.076297in}{2.214977in}}%
\pgfpathlineto{\pgfqpoint{1.084222in}{2.218276in}}%
\pgfpathlineto{\pgfqpoint{1.093953in}{2.221897in}}%
\pgfpathlineto{\pgfqpoint{1.101043in}{2.223978in}}%
\pgfpathlineto{\pgfqpoint{1.103685in}{2.224781in}}%
\pgfpathlineto{\pgfqpoint{1.113416in}{2.226674in}}%
\pgfpathlineto{\pgfqpoint{1.123148in}{2.227325in}}%
\pgfpathlineto{\pgfqpoint{1.132879in}{2.226674in}}%
\pgfpathlineto{\pgfqpoint{1.142611in}{2.224781in}}%
\pgfpathlineto{\pgfqpoint{1.145252in}{2.223978in}}%
\pgfpathlineto{\pgfqpoint{1.152342in}{2.221897in}}%
\pgfpathlineto{\pgfqpoint{1.162074in}{2.218276in}}%
\pgfpathlineto{\pgfqpoint{1.169998in}{2.214977in}}%
\pgfpathlineto{\pgfqpoint{1.171805in}{2.214219in}}%
\pgfpathlineto{\pgfqpoint{1.181537in}{2.210078in}}%
\pgfpathlineto{\pgfqpoint{1.191268in}{2.206248in}}%
\pgfpathlineto{\pgfqpoint{1.192102in}{2.205975in}}%
\pgfpathlineto{\pgfqpoint{1.201000in}{2.202917in}}%
\pgfpathlineto{\pgfqpoint{1.210731in}{2.200531in}}%
\pgfpathlineto{\pgfqpoint{1.220463in}{2.199284in}}%
\pgfpathlineto{\pgfqpoint{1.230194in}{2.199238in}}%
\pgfpathlineto{\pgfqpoint{1.239926in}{2.200331in}}%
\pgfpathlineto{\pgfqpoint{1.249657in}{2.202384in}}%
\pgfpathlineto{\pgfqpoint{1.259389in}{2.205115in}}%
\pgfpathlineto{\pgfqpoint{1.262127in}{2.205975in}}%
\pgfpathlineto{\pgfqpoint{1.269120in}{2.208067in}}%
\pgfpathlineto{\pgfqpoint{1.278852in}{2.210916in}}%
\pgfpathlineto{\pgfqpoint{1.288583in}{2.213331in}}%
\pgfpathlineto{\pgfqpoint{1.298315in}{2.214976in}}%
\pgfpathlineto{\pgfqpoint{1.298322in}{2.214977in}}%
\pgfpathlineto{\pgfqpoint{1.308046in}{2.215582in}}%
\pgfpathlineto{\pgfqpoint{1.317778in}{2.214988in}}%
\pgfpathlineto{\pgfqpoint{1.317838in}{2.214977in}}%
\pgfpathlineto{\pgfqpoint{1.327509in}{2.213125in}}%
\pgfpathlineto{\pgfqpoint{1.337240in}{2.210054in}}%
\pgfpathlineto{\pgfqpoint{1.346906in}{2.205975in}}%
\pgfpathlineto{\pgfqpoint{1.346972in}{2.205946in}}%
\pgfpathlineto{\pgfqpoint{1.356703in}{2.200823in}}%
\pgfpathlineto{\pgfqpoint{1.363418in}{2.196974in}}%
\pgfpathlineto{\pgfqpoint{1.366435in}{2.195074in}}%
\pgfpathlineto{\pgfqpoint{1.376166in}{2.188873in}}%
\pgfpathlineto{\pgfqpoint{1.377668in}{2.187972in}}%
\pgfpathlineto{\pgfqpoint{1.385898in}{2.182275in}}%
\pgfpathlineto{\pgfqpoint{1.391370in}{2.178971in}}%
\pgfpathlineto{\pgfqpoint{1.395629in}{2.175840in}}%
\pgfpathlineto{\pgfqpoint{1.405361in}{2.170166in}}%
\pgfpathlineto{\pgfqpoint{1.405851in}{2.169969in}}%
\pgfpathlineto{\pgfqpoint{1.415092in}{2.165151in}}%
\pgfpathlineto{\pgfqpoint{1.424824in}{2.162477in}}%
\pgfpathlineto{\pgfqpoint{1.434555in}{2.162078in}}%
\pgfpathlineto{\pgfqpoint{1.444287in}{2.163594in}}%
\pgfpathlineto{\pgfqpoint{1.454018in}{2.166468in}}%
\pgfpathlineto{\pgfqpoint{1.463682in}{2.169969in}}%
\pgfpathlineto{\pgfqpoint{1.463750in}{2.169988in}}%
\pgfpathlineto{\pgfqpoint{1.473481in}{2.172610in}}%
\pgfpathlineto{\pgfqpoint{1.483213in}{2.174530in}}%
\pgfpathlineto{\pgfqpoint{1.492944in}{2.175202in}}%
\pgfpathlineto{\pgfqpoint{1.502676in}{2.174196in}}%
\pgfpathlineto{\pgfqpoint{1.512407in}{2.171234in}}%
\pgfpathlineto{\pgfqpoint{1.514863in}{2.169969in}}%
\pgfpathlineto{\pgfqpoint{1.522139in}{2.165114in}}%
\pgfpathlineto{\pgfqpoint{1.526609in}{2.160968in}}%
\pgfpathlineto{\pgfqpoint{1.531870in}{2.154176in}}%
\pgfpathlineto{\pgfqpoint{1.533249in}{2.151966in}}%
\pgfpathlineto{\pgfqpoint{1.536980in}{2.142964in}}%
\pgfpathlineto{\pgfqpoint{1.539307in}{2.133963in}}%
\pgfpathlineto{\pgfqpoint{1.540844in}{2.124961in}}%
\pgfpathlineto{\pgfqpoint{1.541602in}{2.120173in}}%
\pgfpathlineto{\pgfqpoint{1.542184in}{2.115960in}}%
\pgfpathlineto{\pgfqpoint{1.543924in}{2.106958in}}%
\pgfpathlineto{\pgfqpoint{1.546694in}{2.097957in}}%
\pgfpathlineto{\pgfqpoint{1.550919in}{2.088955in}}%
\pgfpathlineto{\pgfqpoint{1.551333in}{2.088330in}}%
\pgfpathlineto{\pgfqpoint{1.556505in}{2.079954in}}%
\pgfpathlineto{\pgfqpoint{1.561064in}{2.074299in}}%
\pgfpathlineto{\pgfqpoint{1.563738in}{2.070952in}}%
\pgfpathlineto{\pgfqpoint{1.570796in}{2.063723in}}%
\pgfpathlineto{\pgfqpoint{1.572612in}{2.061951in}}%
\pgfpathlineto{\pgfqpoint{1.580527in}{2.055251in}}%
\pgfpathlineto{\pgfqpoint{1.583558in}{2.052949in}}%
\pgfpathlineto{\pgfqpoint{1.590259in}{2.048298in}}%
\pgfpathlineto{\pgfqpoint{1.597735in}{2.043947in}}%
\pgfpathlineto{\pgfqpoint{1.599990in}{2.042687in}}%
\pgfpathlineto{\pgfqpoint{1.609722in}{2.038495in}}%
\pgfpathlineto{\pgfqpoint{1.619453in}{2.035616in}}%
\pgfpathlineto{\pgfqpoint{1.623447in}{2.034946in}}%
\pgfpathlineto{\pgfqpoint{1.629185in}{2.033976in}}%
\pgfpathlineto{\pgfqpoint{1.638916in}{2.033387in}}%
\pgfpathlineto{\pgfqpoint{1.648648in}{2.033548in}}%
\pgfpathlineto{\pgfqpoint{1.658379in}{2.034069in}}%
\pgfpathlineto{\pgfqpoint{1.668111in}{2.034518in}}%
\pgfpathlineto{\pgfqpoint{1.677842in}{2.034465in}}%
\pgfpathlineto{\pgfqpoint{1.687574in}{2.033520in}}%
\pgfpathlineto{\pgfqpoint{1.697305in}{2.031375in}}%
\pgfpathlineto{\pgfqpoint{1.707037in}{2.027832in}}%
\pgfpathlineto{\pgfqpoint{1.710705in}{2.025944in}}%
\pgfpathlineto{\pgfqpoint{1.716768in}{2.022639in}}%
\pgfpathlineto{\pgfqpoint{1.724936in}{2.016943in}}%
\pgfpathlineto{\pgfqpoint{1.726500in}{2.015724in}}%
\pgfpathlineto{\pgfqpoint{1.734914in}{2.007941in}}%
\pgfpathlineto{\pgfqpoint{1.736231in}{2.006495in}}%
\pgfpathlineto{\pgfqpoint{1.742389in}{1.998940in}}%
\pgfpathlineto{\pgfqpoint{1.745963in}{1.993331in}}%
\pgfpathlineto{\pgfqpoint{1.748004in}{1.989938in}}%
\pgfpathlineto{\pgfqpoint{1.751834in}{1.980937in}}%
\pgfpathlineto{\pgfqpoint{1.754153in}{1.971935in}}%
\pgfpathlineto{\pgfqpoint{1.755174in}{1.962934in}}%
\pgfpathlineto{\pgfqpoint{1.755231in}{1.953932in}}%
\pgfpathlineto{\pgfqpoint{1.754746in}{1.944931in}}%
\pgfpathlineto{\pgfqpoint{1.754183in}{1.935929in}}%
\pgfpathlineto{\pgfqpoint{1.754009in}{1.926927in}}%
\pgfpathlineto{\pgfqpoint{1.754645in}{1.917926in}}%
\pgfpathlineto{\pgfqpoint{1.755694in}{1.912619in}}%
\pgfpathlineto{\pgfqpoint{1.756419in}{1.908924in}}%
\pgfpathlineto{\pgfqpoint{1.759531in}{1.899923in}}%
\pgfpathlineto{\pgfqpoint{1.764063in}{1.890921in}}%
\pgfpathlineto{\pgfqpoint{1.765426in}{1.888835in}}%
\pgfpathlineto{\pgfqpoint{1.770129in}{1.881920in}}%
\pgfpathlineto{\pgfqpoint{1.775157in}{1.875722in}}%
\pgfpathlineto{\pgfqpoint{1.777645in}{1.872918in}}%
\pgfpathlineto{\pgfqpoint{1.784889in}{1.865596in}}%
\pgfpathlineto{\pgfqpoint{1.786805in}{1.863917in}}%
\pgfpathlineto{\pgfqpoint{1.794620in}{1.857388in}}%
\pgfpathlineto{\pgfqpoint{1.798238in}{1.854915in}}%
\pgfpathlineto{\pgfqpoint{1.804351in}{1.850697in}}%
\pgfpathlineto{\pgfqpoint{1.813407in}{1.845914in}}%
\pgfpathlineto{\pgfqpoint{1.814083in}{1.845531in}}%
\pgfpathlineto{\pgfqpoint{1.823814in}{1.841623in}}%
\pgfpathlineto{\pgfqpoint{1.833546in}{1.839060in}}%
\pgfpathlineto{\pgfqpoint{1.843277in}{1.837451in}}%
\pgfpathlineto{\pgfqpoint{1.847832in}{1.836912in}}%
\pgfpathlineto{\pgfqpoint{1.853009in}{1.836211in}}%
\pgfpathlineto{\pgfqpoint{1.862740in}{1.834790in}}%
\pgfpathlineto{\pgfqpoint{1.872472in}{1.832637in}}%
\pgfpathlineto{\pgfqpoint{1.882203in}{1.829186in}}%
\pgfpathlineto{\pgfqpoint{1.884592in}{1.827910in}}%
\pgfpathlineto{\pgfqpoint{1.891935in}{1.823044in}}%
\pgfpathlineto{\pgfqpoint{1.896417in}{1.818909in}}%
\pgfpathlineto{\pgfqpoint{1.901666in}{1.812179in}}%
\pgfpathlineto{\pgfqpoint{1.903034in}{1.809907in}}%
\pgfpathlineto{\pgfqpoint{1.906236in}{1.800906in}}%
\pgfpathlineto{\pgfqpoint{1.907324in}{1.791904in}}%
\pgfpathlineto{\pgfqpoint{1.906597in}{1.782903in}}%
\pgfpathlineto{\pgfqpoint{1.904522in}{1.773901in}}%
\pgfpathlineto{\pgfqpoint{1.901687in}{1.764900in}}%
\pgfpathlineto{\pgfqpoint{1.901666in}{1.764837in}}%
\pgfpathlineto{\pgfqpoint{1.897881in}{1.755898in}}%
\pgfpathlineto{\pgfqpoint{1.894775in}{1.746897in}}%
\pgfpathlineto{\pgfqpoint{1.893135in}{1.737895in}}%
\pgfpathlineto{\pgfqpoint{1.893567in}{1.728893in}}%
\pgfpathlineto{\pgfqpoint{1.896458in}{1.719892in}}%
\pgfpathlineto{\pgfqpoint{1.901666in}{1.711343in}}%
\pgfpathlineto{\pgfqpoint{1.901879in}{1.710890in}}%
\pgfpathlineto{\pgfqpoint{1.908013in}{1.701889in}}%
\pgfpathlineto{\pgfqpoint{1.911398in}{1.697949in}}%
\pgfpathlineto{\pgfqpoint{1.914970in}{1.692887in}}%
\pgfpathlineto{\pgfqpoint{1.921129in}{1.685274in}}%
\pgfpathlineto{\pgfqpoint{1.922103in}{1.683886in}}%
\pgfpathlineto{\pgfqpoint{1.928807in}{1.674884in}}%
\pgfpathlineto{\pgfqpoint{1.930861in}{1.672094in}}%
\pgfpathlineto{\pgfqpoint{1.935022in}{1.665883in}}%
\pgfpathlineto{\pgfqpoint{1.940561in}{1.656881in}}%
\pgfpathlineto{\pgfqpoint{1.940592in}{1.656820in}}%
\pgfpathlineto{\pgfqpoint{1.945002in}{1.647880in}}%
\pgfpathlineto{\pgfqpoint{1.948322in}{1.638878in}}%
\pgfpathlineto{\pgfqpoint{1.950324in}{1.629933in}}%
\pgfpathlineto{\pgfqpoint{1.950336in}{1.629876in}}%
\pgfpathlineto{\pgfqpoint{1.950978in}{1.620875in}}%
\pgfpathlineto{\pgfqpoint{1.950324in}{1.611880in}}%
\pgfpathlineto{\pgfqpoint{1.950323in}{1.611873in}}%
\pgfpathlineto{\pgfqpoint{1.948544in}{1.602872in}}%
\pgfpathlineto{\pgfqpoint{1.945933in}{1.593870in}}%
\pgfpathlineto{\pgfqpoint{1.942853in}{1.584869in}}%
\pgfpathlineto{\pgfqpoint{1.940592in}{1.578400in}}%
\pgfpathlineto{\pgfqpoint{1.939662in}{1.575867in}}%
\pgfpathlineto{\pgfqpoint{1.936709in}{1.566866in}}%
\pgfpathlineto{\pgfqpoint{1.934491in}{1.557864in}}%
\pgfpathlineto{\pgfqpoint{1.933309in}{1.548863in}}%
\pgfpathlineto{\pgfqpoint{1.933359in}{1.539861in}}%
\pgfpathlineto{\pgfqpoint{1.934707in}{1.530859in}}%
\pgfpathlineto{\pgfqpoint{1.937286in}{1.521858in}}%
\pgfpathlineto{\pgfqpoint{1.940592in}{1.513628in}}%
\pgfpathlineto{\pgfqpoint{1.940887in}{1.512856in}}%
\pgfpathlineto{\pgfqpoint{1.945028in}{1.503855in}}%
\pgfpathlineto{\pgfqpoint{1.949505in}{1.494853in}}%
\pgfpathlineto{\pgfqpoint{1.950324in}{1.493182in}}%
\pgfpathlineto{\pgfqpoint{1.953891in}{1.485852in}}%
\pgfpathlineto{\pgfqpoint{1.957805in}{1.476850in}}%
\pgfpathlineto{\pgfqpoint{1.960055in}{1.470292in}}%
\pgfpathlineto{\pgfqpoint{1.960923in}{1.467849in}}%
\pgfpathlineto{\pgfqpoint{1.962969in}{1.458847in}}%
\pgfpathlineto{\pgfqpoint{1.963673in}{1.449846in}}%
\pgfpathlineto{\pgfqpoint{1.962969in}{1.440844in}}%
\pgfpathlineto{\pgfqpoint{1.960923in}{1.431842in}}%
\pgfpathlineto{\pgfqpoint{1.960055in}{1.429399in}}%
\pgfpathlineto{\pgfqpoint{1.957805in}{1.422841in}}%
\pgfpathlineto{\pgfqpoint{1.953891in}{1.413839in}}%
\pgfpathlineto{\pgfqpoint{1.950324in}{1.406509in}}%
\pgfpathlineto{\pgfqpoint{1.949505in}{1.404838in}}%
\pgfpathlineto{\pgfqpoint{1.945028in}{1.395836in}}%
\pgfpathlineto{\pgfqpoint{1.940887in}{1.386835in}}%
\pgfpathlineto{\pgfqpoint{1.940592in}{1.386063in}}%
\pgfpathlineto{\pgfqpoint{1.937286in}{1.377833in}}%
\pgfpathlineto{\pgfqpoint{1.934707in}{1.368832in}}%
\pgfpathlineto{\pgfqpoint{1.933359in}{1.359830in}}%
\pgfpathlineto{\pgfqpoint{1.933309in}{1.350829in}}%
\pgfpathlineto{\pgfqpoint{1.934491in}{1.341827in}}%
\pgfpathlineto{\pgfqpoint{1.936709in}{1.332825in}}%
\pgfpathlineto{\pgfqpoint{1.939662in}{1.323824in}}%
\pgfpathlineto{\pgfqpoint{1.940592in}{1.321291in}}%
\pgfpathlineto{\pgfqpoint{1.942853in}{1.314822in}}%
\pgfpathlineto{\pgfqpoint{1.945933in}{1.305821in}}%
\pgfpathlineto{\pgfqpoint{1.948544in}{1.296819in}}%
\pgfpathlineto{\pgfqpoint{1.950323in}{1.287818in}}%
\pgfpathlineto{\pgfqpoint{1.950324in}{1.287811in}}%
\pgfpathlineto{\pgfqpoint{1.950978in}{1.278816in}}%
\pgfpathlineto{\pgfqpoint{1.950336in}{1.269815in}}%
\pgfpathlineto{\pgfqpoint{1.950324in}{1.269758in}}%
\pgfpathlineto{\pgfqpoint{1.948322in}{1.260813in}}%
\pgfpathlineto{\pgfqpoint{1.945002in}{1.251812in}}%
\pgfpathlineto{\pgfqpoint{1.940592in}{1.242871in}}%
\pgfpathlineto{\pgfqpoint{1.940561in}{1.242810in}}%
\pgfpathlineto{\pgfqpoint{1.935022in}{1.233808in}}%
\pgfpathlineto{\pgfqpoint{1.930861in}{1.227597in}}%
\pgfpathlineto{\pgfqpoint{1.928807in}{1.224807in}}%
\pgfpathlineto{\pgfqpoint{1.922103in}{1.215805in}}%
\pgfpathlineto{\pgfqpoint{1.921129in}{1.214417in}}%
\pgfpathlineto{\pgfqpoint{1.914970in}{1.206804in}}%
\pgfpathlineto{\pgfqpoint{1.911398in}{1.201742in}}%
\pgfpathlineto{\pgfqpoint{1.908013in}{1.197802in}}%
\pgfpathlineto{\pgfqpoint{1.901879in}{1.188801in}}%
\pgfpathlineto{\pgfqpoint{1.901666in}{1.188348in}}%
\pgfpathlineto{\pgfqpoint{1.896458in}{1.179799in}}%
\pgfpathlineto{\pgfqpoint{1.893567in}{1.170798in}}%
\pgfpathlineto{\pgfqpoint{1.893135in}{1.161796in}}%
\pgfpathlineto{\pgfqpoint{1.894775in}{1.152795in}}%
\pgfpathlineto{\pgfqpoint{1.897881in}{1.143793in}}%
\pgfpathlineto{\pgfqpoint{1.901666in}{1.134854in}}%
\pgfpathlineto{\pgfqpoint{1.901687in}{1.134792in}}%
\pgfpathlineto{\pgfqpoint{1.904522in}{1.125790in}}%
\pgfpathlineto{\pgfqpoint{1.906597in}{1.116788in}}%
\pgfpathlineto{\pgfqpoint{1.907324in}{1.107787in}}%
\pgfpathlineto{\pgfqpoint{1.906236in}{1.098785in}}%
\pgfpathlineto{\pgfqpoint{1.903034in}{1.089784in}}%
\pgfpathlineto{\pgfqpoint{1.901666in}{1.087512in}}%
\pgfpathlineto{\pgfqpoint{1.896417in}{1.080782in}}%
\pgfpathlineto{\pgfqpoint{1.891935in}{1.076647in}}%
\pgfpathlineto{\pgfqpoint{1.884592in}{1.071781in}}%
\pgfpathlineto{\pgfqpoint{1.882203in}{1.070505in}}%
\pgfpathlineto{\pgfqpoint{1.872472in}{1.067054in}}%
\pgfpathlineto{\pgfqpoint{1.862740in}{1.064901in}}%
\pgfpathlineto{\pgfqpoint{1.853009in}{1.063480in}}%
\pgfpathlineto{\pgfqpoint{1.847832in}{1.062779in}}%
\pgfpathlineto{\pgfqpoint{1.843277in}{1.062240in}}%
\pgfpathlineto{\pgfqpoint{1.833546in}{1.060631in}}%
\pgfpathlineto{\pgfqpoint{1.823814in}{1.058068in}}%
\pgfpathlineto{\pgfqpoint{1.814083in}{1.054160in}}%
\pgfpathlineto{\pgfqpoint{1.813407in}{1.053778in}}%
\pgfpathlineto{\pgfqpoint{1.804351in}{1.048994in}}%
\pgfpathlineto{\pgfqpoint{1.798238in}{1.044776in}}%
\pgfpathlineto{\pgfqpoint{1.794620in}{1.042303in}}%
\pgfpathlineto{\pgfqpoint{1.786805in}{1.035775in}}%
\pgfpathlineto{\pgfqpoint{1.784889in}{1.034095in}}%
\pgfpathlineto{\pgfqpoint{1.777645in}{1.026773in}}%
\pgfpathlineto{\pgfqpoint{1.775157in}{1.023969in}}%
\pgfpathlineto{\pgfqpoint{1.770129in}{1.017771in}}%
\pgfpathlineto{\pgfqpoint{1.765426in}{1.010856in}}%
\pgfpathlineto{\pgfqpoint{1.764063in}{1.008770in}}%
\pgfpathlineto{\pgfqpoint{1.759531in}{0.999768in}}%
\pgfpathlineto{\pgfqpoint{1.756419in}{0.990767in}}%
\pgfpathlineto{\pgfqpoint{1.755694in}{0.987072in}}%
\pgfpathlineto{\pgfqpoint{1.754645in}{0.981765in}}%
\pgfpathlineto{\pgfqpoint{1.754009in}{0.972764in}}%
\pgfpathlineto{\pgfqpoint{1.754183in}{0.963762in}}%
\pgfpathlineto{\pgfqpoint{1.754746in}{0.954761in}}%
\pgfpathlineto{\pgfqpoint{1.755231in}{0.945759in}}%
\pgfpathlineto{\pgfqpoint{1.755174in}{0.936758in}}%
\pgfpathlineto{\pgfqpoint{1.754153in}{0.927756in}}%
\pgfpathlineto{\pgfqpoint{1.751834in}{0.918754in}}%
\pgfpathlineto{\pgfqpoint{1.748004in}{0.909753in}}%
\pgfpathlineto{\pgfqpoint{1.745963in}{0.906360in}}%
\pgfpathlineto{\pgfqpoint{1.742389in}{0.900751in}}%
\pgfpathlineto{\pgfqpoint{1.736231in}{0.893197in}}%
\pgfpathlineto{\pgfqpoint{1.734914in}{0.891750in}}%
\pgfpathlineto{\pgfqpoint{1.726500in}{0.883967in}}%
\pgfpathlineto{\pgfqpoint{1.724936in}{0.882748in}}%
\pgfpathlineto{\pgfqpoint{1.716768in}{0.877053in}}%
\pgfpathlineto{\pgfqpoint{1.710705in}{0.873747in}}%
\pgfpathlineto{\pgfqpoint{1.707037in}{0.871859in}}%
\pgfpathlineto{\pgfqpoint{1.697305in}{0.868316in}}%
\pgfpathlineto{\pgfqpoint{1.687574in}{0.866171in}}%
\pgfpathlineto{\pgfqpoint{1.677842in}{0.865227in}}%
\pgfpathlineto{\pgfqpoint{1.668111in}{0.865173in}}%
\pgfpathlineto{\pgfqpoint{1.658379in}{0.865623in}}%
\pgfpathlineto{\pgfqpoint{1.648648in}{0.866143in}}%
\pgfpathlineto{\pgfqpoint{1.638916in}{0.866304in}}%
\pgfpathlineto{\pgfqpoint{1.629185in}{0.865715in}}%
\pgfpathlineto{\pgfqpoint{1.623447in}{0.864745in}}%
\pgfpathlineto{\pgfqpoint{1.619453in}{0.864075in}}%
\pgfpathlineto{\pgfqpoint{1.609722in}{0.861196in}}%
\pgfpathlineto{\pgfqpoint{1.599990in}{0.857004in}}%
\pgfpathlineto{\pgfqpoint{1.597735in}{0.855744in}}%
\pgfpathlineto{\pgfqpoint{1.590259in}{0.851393in}}%
\pgfpathlineto{\pgfqpoint{1.583558in}{0.846742in}}%
\pgfpathlineto{\pgfqpoint{1.580527in}{0.844440in}}%
\pgfpathlineto{\pgfqpoint{1.572612in}{0.837741in}}%
\pgfpathlineto{\pgfqpoint{1.570796in}{0.835968in}}%
\pgfpathlineto{\pgfqpoint{1.563738in}{0.828739in}}%
\pgfpathlineto{\pgfqpoint{1.561064in}{0.825392in}}%
\pgfpathlineto{\pgfqpoint{1.556505in}{0.819737in}}%
\pgfpathlineto{\pgfqpoint{1.551333in}{0.811361in}}%
\pgfpathlineto{\pgfqpoint{1.550919in}{0.810736in}}%
\pgfpathlineto{\pgfqpoint{1.546694in}{0.801734in}}%
\pgfpathlineto{\pgfqpoint{1.543924in}{0.792733in}}%
\pgfpathlineto{\pgfqpoint{1.542184in}{0.783731in}}%
\pgfpathlineto{\pgfqpoint{1.541602in}{0.779518in}}%
\pgfpathlineto{\pgfqpoint{1.540844in}{0.774730in}}%
\pgfpathlineto{\pgfqpoint{1.539307in}{0.765728in}}%
\pgfpathlineto{\pgfqpoint{1.536980in}{0.756727in}}%
\pgfpathlineto{\pgfqpoint{1.533249in}{0.747725in}}%
\pgfpathlineto{\pgfqpoint{1.531870in}{0.745516in}}%
\pgfpathlineto{\pgfqpoint{1.526609in}{0.738724in}}%
\pgfpathlineto{\pgfqpoint{1.522139in}{0.734577in}}%
\pgfpathlineto{\pgfqpoint{1.514863in}{0.729722in}}%
\pgfpathlineto{\pgfqpoint{1.512407in}{0.728457in}}%
\pgfpathlineto{\pgfqpoint{1.502676in}{0.725495in}}%
\pgfpathlineto{\pgfqpoint{1.492944in}{0.724489in}}%
\pgfpathlineto{\pgfqpoint{1.483213in}{0.725161in}}%
\pgfpathlineto{\pgfqpoint{1.473481in}{0.727081in}}%
\pgfpathlineto{\pgfqpoint{1.463750in}{0.729703in}}%
\pgfpathlineto{\pgfqpoint{1.463682in}{0.729722in}}%
\pgfpathlineto{\pgfqpoint{1.454018in}{0.733223in}}%
\pgfpathlineto{\pgfqpoint{1.444287in}{0.736097in}}%
\pgfpathlineto{\pgfqpoint{1.434555in}{0.737613in}}%
\pgfpathlineto{\pgfqpoint{1.424824in}{0.737214in}}%
\pgfpathlineto{\pgfqpoint{1.415092in}{0.734540in}}%
\pgfpathlineto{\pgfqpoint{1.405851in}{0.729722in}}%
\pgfpathlineto{\pgfqpoint{1.405361in}{0.729525in}}%
\pgfpathlineto{\pgfqpoint{1.395629in}{0.723851in}}%
\pgfpathlineto{\pgfqpoint{1.391370in}{0.720720in}}%
\pgfpathlineto{\pgfqpoint{1.385898in}{0.717416in}}%
\pgfpathlineto{\pgfqpoint{1.377668in}{0.711719in}}%
\pgfpathlineto{\pgfqpoint{1.376166in}{0.710818in}}%
\pgfpathlineto{\pgfqpoint{1.366435in}{0.704617in}}%
\pgfpathlineto{\pgfqpoint{1.363418in}{0.702717in}}%
\pgfpathlineto{\pgfqpoint{1.356703in}{0.698868in}}%
\pgfpathlineto{\pgfqpoint{1.346972in}{0.693745in}}%
\pgfpathlineto{\pgfqpoint{1.346906in}{0.693716in}}%
\pgfpathlineto{\pgfqpoint{1.337240in}{0.689637in}}%
\pgfpathlineto{\pgfqpoint{1.327509in}{0.686566in}}%
\pgfpathlineto{\pgfqpoint{1.317838in}{0.684714in}}%
\pgfpathlineto{\pgfqpoint{1.317778in}{0.684703in}}%
\pgfpathlineto{\pgfqpoint{1.308046in}{0.684109in}}%
\pgfpathlineto{\pgfqpoint{1.298322in}{0.684714in}}%
\pgfpathlineto{\pgfqpoint{1.298315in}{0.684715in}}%
\pgfpathlineto{\pgfqpoint{1.288583in}{0.686360in}}%
\pgfpathlineto{\pgfqpoint{1.278852in}{0.688775in}}%
\pgfpathlineto{\pgfqpoint{1.269120in}{0.691624in}}%
\pgfpathlineto{\pgfqpoint{1.262127in}{0.693716in}}%
\pgfpathlineto{\pgfqpoint{1.259389in}{0.694576in}}%
\pgfpathlineto{\pgfqpoint{1.249657in}{0.697308in}}%
\pgfpathlineto{\pgfqpoint{1.239926in}{0.699360in}}%
\pgfpathlineto{\pgfqpoint{1.230194in}{0.700453in}}%
\pgfpathlineto{\pgfqpoint{1.220463in}{0.700407in}}%
\pgfpathlineto{\pgfqpoint{1.210731in}{0.699160in}}%
\pgfpathlineto{\pgfqpoint{1.201000in}{0.696774in}}%
\pgfpathlineto{\pgfqpoint{1.192102in}{0.693716in}}%
\pgfpathlineto{\pgfqpoint{1.191268in}{0.693443in}}%
\pgfpathlineto{\pgfqpoint{1.181537in}{0.689613in}}%
\pgfpathlineto{\pgfqpoint{1.171805in}{0.685472in}}%
\pgfpathlineto{\pgfqpoint{1.169998in}{0.684714in}}%
\pgfpathlineto{\pgfqpoint{1.162074in}{0.681415in}}%
\pgfpathlineto{\pgfqpoint{1.152342in}{0.677794in}}%
\pgfpathlineto{\pgfqpoint{1.145252in}{0.675713in}}%
\pgfpathlineto{\pgfqpoint{1.142611in}{0.674910in}}%
\pgfpathlineto{\pgfqpoint{1.132879in}{0.673017in}}%
\pgfpathlineto{\pgfqpoint{1.123148in}{0.672366in}}%
\pgfpathlineto{\pgfqpoint{1.113416in}{0.673017in}}%
\pgfpathlineto{\pgfqpoint{1.103685in}{0.674910in}}%
\pgfpathclose%
\pgfusepath{fill}%
\end{pgfscope}%
\begin{pgfscope}%
\pgfpathrectangle{\pgfqpoint{0.150000in}{0.549691in}}{\pgfqpoint{1.946296in}{1.800309in}}%
\pgfusepath{clip}%
\pgfsetbuttcap%
\pgfsetroundjoin%
\definecolor{currentfill}{rgb}{0.309804,0.309804,0.309804}%
\pgfsetfillcolor{currentfill}%
\pgfsetlinewidth{0.000000pt}%
\definecolor{currentstroke}{rgb}{0.000000,0.000000,0.000000}%
\pgfsetstrokecolor{currentstroke}%
\pgfsetdash{}{0pt}%
\pgfpathmoveto{\pgfqpoint{0.354361in}{0.553499in}}%
\pgfpathlineto{\pgfqpoint{0.361734in}{0.549691in}}%
\pgfpathlineto{\pgfqpoint{0.364093in}{0.549691in}}%
\pgfpathlineto{\pgfqpoint{0.373824in}{0.549691in}}%
\pgfpathlineto{\pgfqpoint{0.383556in}{0.549691in}}%
\pgfpathlineto{\pgfqpoint{0.393287in}{0.549691in}}%
\pgfpathlineto{\pgfqpoint{0.403018in}{0.549691in}}%
\pgfpathlineto{\pgfqpoint{0.412750in}{0.549691in}}%
\pgfpathlineto{\pgfqpoint{0.422481in}{0.549691in}}%
\pgfpathlineto{\pgfqpoint{0.432213in}{0.549691in}}%
\pgfpathlineto{\pgfqpoint{0.441944in}{0.549691in}}%
\pgfpathlineto{\pgfqpoint{0.451676in}{0.549691in}}%
\pgfpathlineto{\pgfqpoint{0.461407in}{0.549691in}}%
\pgfpathlineto{\pgfqpoint{0.471139in}{0.549691in}}%
\pgfpathlineto{\pgfqpoint{0.480870in}{0.549691in}}%
\pgfpathlineto{\pgfqpoint{0.490602in}{0.549691in}}%
\pgfpathlineto{\pgfqpoint{0.500333in}{0.549691in}}%
\pgfpathlineto{\pgfqpoint{0.510065in}{0.549691in}}%
\pgfpathlineto{\pgfqpoint{0.519796in}{0.549691in}}%
\pgfpathlineto{\pgfqpoint{0.529528in}{0.549691in}}%
\pgfpathlineto{\pgfqpoint{0.539259in}{0.549691in}}%
\pgfpathlineto{\pgfqpoint{0.548991in}{0.549691in}}%
\pgfpathlineto{\pgfqpoint{0.558722in}{0.549691in}}%
\pgfpathlineto{\pgfqpoint{0.568454in}{0.549691in}}%
\pgfpathlineto{\pgfqpoint{0.578185in}{0.549691in}}%
\pgfpathlineto{\pgfqpoint{0.587917in}{0.549691in}}%
\pgfpathlineto{\pgfqpoint{0.597648in}{0.549691in}}%
\pgfpathlineto{\pgfqpoint{0.607380in}{0.549691in}}%
\pgfpathlineto{\pgfqpoint{0.617111in}{0.549691in}}%
\pgfpathlineto{\pgfqpoint{0.626842in}{0.549691in}}%
\pgfpathlineto{\pgfqpoint{0.636574in}{0.549691in}}%
\pgfpathlineto{\pgfqpoint{0.646305in}{0.549691in}}%
\pgfpathlineto{\pgfqpoint{0.656037in}{0.549691in}}%
\pgfpathlineto{\pgfqpoint{0.665768in}{0.549691in}}%
\pgfpathlineto{\pgfqpoint{0.675500in}{0.549691in}}%
\pgfpathlineto{\pgfqpoint{0.685231in}{0.549691in}}%
\pgfpathlineto{\pgfqpoint{0.694963in}{0.549691in}}%
\pgfpathlineto{\pgfqpoint{0.704694in}{0.549691in}}%
\pgfpathlineto{\pgfqpoint{0.714426in}{0.549691in}}%
\pgfpathlineto{\pgfqpoint{0.724157in}{0.549691in}}%
\pgfpathlineto{\pgfqpoint{0.733889in}{0.549691in}}%
\pgfpathlineto{\pgfqpoint{0.743620in}{0.549691in}}%
\pgfpathlineto{\pgfqpoint{0.753352in}{0.549691in}}%
\pgfpathlineto{\pgfqpoint{0.763083in}{0.549691in}}%
\pgfpathlineto{\pgfqpoint{0.772815in}{0.549691in}}%
\pgfpathlineto{\pgfqpoint{0.782546in}{0.549691in}}%
\pgfpathlineto{\pgfqpoint{0.792278in}{0.549691in}}%
\pgfpathlineto{\pgfqpoint{0.802009in}{0.549691in}}%
\pgfpathlineto{\pgfqpoint{0.811741in}{0.549691in}}%
\pgfpathlineto{\pgfqpoint{0.821472in}{0.549691in}}%
\pgfpathlineto{\pgfqpoint{0.831204in}{0.549691in}}%
\pgfpathlineto{\pgfqpoint{0.840935in}{0.549691in}}%
\pgfpathlineto{\pgfqpoint{0.850667in}{0.549691in}}%
\pgfpathlineto{\pgfqpoint{0.856551in}{0.549691in}}%
\pgfpathlineto{\pgfqpoint{0.850667in}{0.554044in}}%
\pgfpathlineto{\pgfqpoint{0.842745in}{0.558693in}}%
\pgfpathlineto{\pgfqpoint{0.840935in}{0.560052in}}%
\pgfpathlineto{\pgfqpoint{0.831204in}{0.565097in}}%
\pgfpathlineto{\pgfqpoint{0.821702in}{0.567694in}}%
\pgfpathlineto{\pgfqpoint{0.821472in}{0.567777in}}%
\pgfpathlineto{\pgfqpoint{0.811741in}{0.568300in}}%
\pgfpathlineto{\pgfqpoint{0.808773in}{0.567694in}}%
\pgfpathlineto{\pgfqpoint{0.802009in}{0.566646in}}%
\pgfpathlineto{\pgfqpoint{0.792278in}{0.563788in}}%
\pgfpathlineto{\pgfqpoint{0.782546in}{0.560281in}}%
\pgfpathlineto{\pgfqpoint{0.777970in}{0.558693in}}%
\pgfpathlineto{\pgfqpoint{0.772815in}{0.557295in}}%
\pgfpathlineto{\pgfqpoint{0.763083in}{0.555363in}}%
\pgfpathlineto{\pgfqpoint{0.753352in}{0.554686in}}%
\pgfpathlineto{\pgfqpoint{0.743620in}{0.555698in}}%
\pgfpathlineto{\pgfqpoint{0.733889in}{0.558680in}}%
\pgfpathlineto{\pgfqpoint{0.733864in}{0.558693in}}%
\pgfpathlineto{\pgfqpoint{0.724157in}{0.565134in}}%
\pgfpathlineto{\pgfqpoint{0.721382in}{0.567694in}}%
\pgfpathlineto{\pgfqpoint{0.714426in}{0.576142in}}%
\pgfpathlineto{\pgfqpoint{0.714058in}{0.576696in}}%
\pgfpathlineto{\pgfqpoint{0.709487in}{0.585697in}}%
\pgfpathlineto{\pgfqpoint{0.705707in}{0.594699in}}%
\pgfpathlineto{\pgfqpoint{0.704694in}{0.597183in}}%
\pgfpathlineto{\pgfqpoint{0.702373in}{0.603700in}}%
\pgfpathlineto{\pgfqpoint{0.698672in}{0.612702in}}%
\pgfpathlineto{\pgfqpoint{0.694963in}{0.619760in}}%
\pgfpathlineto{\pgfqpoint{0.694009in}{0.621703in}}%
\pgfpathlineto{\pgfqpoint{0.688233in}{0.630705in}}%
\pgfpathlineto{\pgfqpoint{0.685231in}{0.634362in}}%
\pgfpathlineto{\pgfqpoint{0.680886in}{0.639707in}}%
\pgfpathlineto{\pgfqpoint{0.675500in}{0.645090in}}%
\pgfpathlineto{\pgfqpoint{0.671702in}{0.648708in}}%
\pgfpathlineto{\pgfqpoint{0.665768in}{0.653509in}}%
\pgfpathlineto{\pgfqpoint{0.659981in}{0.657710in}}%
\pgfpathlineto{\pgfqpoint{0.656037in}{0.660251in}}%
\pgfpathlineto{\pgfqpoint{0.646305in}{0.665505in}}%
\pgfpathlineto{\pgfqpoint{0.643408in}{0.666711in}}%
\pgfpathlineto{\pgfqpoint{0.636574in}{0.669346in}}%
\pgfpathlineto{\pgfqpoint{0.626842in}{0.671922in}}%
\pgfpathlineto{\pgfqpoint{0.617111in}{0.673383in}}%
\pgfpathlineto{\pgfqpoint{0.607380in}{0.673906in}}%
\pgfpathlineto{\pgfqpoint{0.597648in}{0.673763in}}%
\pgfpathlineto{\pgfqpoint{0.587917in}{0.673301in}}%
\pgfpathlineto{\pgfqpoint{0.578185in}{0.672902in}}%
\pgfpathlineto{\pgfqpoint{0.568454in}{0.672949in}}%
\pgfpathlineto{\pgfqpoint{0.558722in}{0.673788in}}%
\pgfpathlineto{\pgfqpoint{0.548991in}{0.675692in}}%
\pgfpathlineto{\pgfqpoint{0.548927in}{0.675713in}}%
\pgfpathlineto{\pgfqpoint{0.539259in}{0.678731in}}%
\pgfpathlineto{\pgfqpoint{0.529528in}{0.683026in}}%
\pgfpathlineto{\pgfqpoint{0.526537in}{0.684714in}}%
\pgfpathlineto{\pgfqpoint{0.519796in}{0.688546in}}%
\pgfpathlineto{\pgfqpoint{0.512136in}{0.693716in}}%
\pgfpathlineto{\pgfqpoint{0.510065in}{0.695185in}}%
\pgfpathlineto{\pgfqpoint{0.500555in}{0.702717in}}%
\pgfpathlineto{\pgfqpoint{0.500333in}{0.702911in}}%
\pgfpathlineto{\pgfqpoint{0.490777in}{0.711719in}}%
\pgfpathlineto{\pgfqpoint{0.490602in}{0.711905in}}%
\pgfpathlineto{\pgfqpoint{0.482374in}{0.720720in}}%
\pgfpathlineto{\pgfqpoint{0.480870in}{0.722682in}}%
\pgfpathlineto{\pgfqpoint{0.475249in}{0.729722in}}%
\pgfpathlineto{\pgfqpoint{0.471139in}{0.736390in}}%
\pgfpathlineto{\pgfqpoint{0.469565in}{0.738724in}}%
\pgfpathlineto{\pgfqpoint{0.465203in}{0.747725in}}%
\pgfpathlineto{\pgfqpoint{0.462307in}{0.756727in}}%
\pgfpathlineto{\pgfqpoint{0.461407in}{0.761208in}}%
\pgfpathlineto{\pgfqpoint{0.460361in}{0.765728in}}%
\pgfpathlineto{\pgfqpoint{0.458986in}{0.774730in}}%
\pgfpathlineto{\pgfqpoint{0.457711in}{0.783731in}}%
\pgfpathlineto{\pgfqpoint{0.455929in}{0.792733in}}%
\pgfpathlineto{\pgfqpoint{0.453091in}{0.801734in}}%
\pgfpathlineto{\pgfqpoint{0.451676in}{0.804679in}}%
\pgfpathlineto{\pgfqpoint{0.448117in}{0.810736in}}%
\pgfpathlineto{\pgfqpoint{0.441944in}{0.818193in}}%
\pgfpathlineto{\pgfqpoint{0.440274in}{0.819737in}}%
\pgfpathlineto{\pgfqpoint{0.432213in}{0.825447in}}%
\pgfpathlineto{\pgfqpoint{0.425665in}{0.828739in}}%
\pgfpathlineto{\pgfqpoint{0.422481in}{0.830048in}}%
\pgfpathlineto{\pgfqpoint{0.412750in}{0.832673in}}%
\pgfpathlineto{\pgfqpoint{0.403018in}{0.834321in}}%
\pgfpathlineto{\pgfqpoint{0.393287in}{0.835501in}}%
\pgfpathlineto{\pgfqpoint{0.383556in}{0.836773in}}%
\pgfpathlineto{\pgfqpoint{0.378669in}{0.837741in}}%
\pgfpathlineto{\pgfqpoint{0.373824in}{0.838572in}}%
\pgfpathlineto{\pgfqpoint{0.364093in}{0.841251in}}%
\pgfpathlineto{\pgfqpoint{0.354361in}{0.845286in}}%
\pgfpathlineto{\pgfqpoint{0.351838in}{0.846742in}}%
\pgfpathlineto{\pgfqpoint{0.344630in}{0.850544in}}%
\pgfpathlineto{\pgfqpoint{0.337019in}{0.855744in}}%
\pgfpathlineto{\pgfqpoint{0.334898in}{0.857135in}}%
\pgfpathlineto{\pgfqpoint{0.325368in}{0.864745in}}%
\pgfpathlineto{\pgfqpoint{0.325167in}{0.864908in}}%
\pgfpathlineto{\pgfqpoint{0.315644in}{0.873747in}}%
\pgfpathlineto{\pgfqpoint{0.315435in}{0.873952in}}%
\pgfpathlineto{\pgfqpoint{0.307291in}{0.882748in}}%
\pgfpathlineto{\pgfqpoint{0.305704in}{0.884664in}}%
\pgfpathlineto{\pgfqpoint{0.300115in}{0.891750in}}%
\pgfpathlineto{\pgfqpoint{0.295972in}{0.897985in}}%
\pgfpathlineto{\pgfqpoint{0.294147in}{0.900751in}}%
\pgfpathlineto{\pgfqpoint{0.289503in}{0.909753in}}%
\pgfpathlineto{\pgfqpoint{0.286241in}{0.918695in}}%
\pgfpathlineto{\pgfqpoint{0.286218in}{0.918754in}}%
\pgfpathlineto{\pgfqpoint{0.284160in}{0.927756in}}%
\pgfpathlineto{\pgfqpoint{0.283253in}{0.936758in}}%
\pgfpathlineto{\pgfqpoint{0.283202in}{0.945759in}}%
\pgfpathlineto{\pgfqpoint{0.283633in}{0.954761in}}%
\pgfpathlineto{\pgfqpoint{0.284133in}{0.963762in}}%
\pgfpathlineto{\pgfqpoint{0.284287in}{0.972764in}}%
\pgfpathlineto{\pgfqpoint{0.283722in}{0.981765in}}%
\pgfpathlineto{\pgfqpoint{0.282143in}{0.990767in}}%
\pgfpathlineto{\pgfqpoint{0.279358in}{0.999768in}}%
\pgfpathlineto{\pgfqpoint{0.276509in}{1.006090in}}%
\pgfpathlineto{\pgfqpoint{0.275205in}{1.008770in}}%
\pgfpathlineto{\pgfqpoint{0.269525in}{1.017771in}}%
\pgfpathlineto{\pgfqpoint{0.266778in}{1.021420in}}%
\pgfpathlineto{\pgfqpoint{0.262236in}{1.026773in}}%
\pgfpathlineto{\pgfqpoint{0.257046in}{1.032262in}}%
\pgfpathlineto{\pgfqpoint{0.253135in}{1.035775in}}%
\pgfpathlineto{\pgfqpoint{0.247315in}{1.040756in}}%
\pgfpathlineto{\pgfqpoint{0.241537in}{1.044776in}}%
\pgfpathlineto{\pgfqpoint{0.237583in}{1.047553in}}%
\pgfpathlineto{\pgfqpoint{0.227852in}{1.052895in}}%
\pgfpathlineto{\pgfqpoint{0.225751in}{1.053778in}}%
\pgfpathlineto{\pgfqpoint{0.218120in}{1.057209in}}%
\pgfpathlineto{\pgfqpoint{0.208389in}{1.060632in}}%
\pgfpathlineto{\pgfqpoint{0.201343in}{1.062779in}}%
\pgfpathlineto{\pgfqpoint{0.198657in}{1.063716in}}%
\pgfpathlineto{\pgfqpoint{0.188926in}{1.067212in}}%
\pgfpathlineto{\pgfqpoint{0.179194in}{1.071440in}}%
\pgfpathlineto{\pgfqpoint{0.178595in}{1.071781in}}%
\pgfpathlineto{\pgfqpoint{0.169463in}{1.078216in}}%
\pgfpathlineto{\pgfqpoint{0.166695in}{1.080782in}}%
\pgfpathlineto{\pgfqpoint{0.159731in}{1.089761in}}%
\pgfpathlineto{\pgfqpoint{0.159717in}{1.089784in}}%
\pgfpathlineto{\pgfqpoint{0.156494in}{1.098785in}}%
\pgfpathlineto{\pgfqpoint{0.155400in}{1.107787in}}%
\pgfpathlineto{\pgfqpoint{0.156132in}{1.116788in}}%
\pgfpathlineto{\pgfqpoint{0.158220in}{1.125790in}}%
\pgfpathlineto{\pgfqpoint{0.159731in}{1.130558in}}%
\pgfpathlineto{\pgfqpoint{0.161449in}{1.134792in}}%
\pgfpathlineto{\pgfqpoint{0.165240in}{1.143793in}}%
\pgfpathlineto{\pgfqpoint{0.168329in}{1.152795in}}%
\pgfpathlineto{\pgfqpoint{0.169463in}{1.159052in}}%
\pgfpathlineto{\pgfqpoint{0.170117in}{1.161796in}}%
\pgfpathlineto{\pgfqpoint{0.169552in}{1.170798in}}%
\pgfpathlineto{\pgfqpoint{0.169463in}{1.171010in}}%
\pgfpathlineto{\pgfqpoint{0.166655in}{1.179799in}}%
\pgfpathlineto{\pgfqpoint{0.161201in}{1.188801in}}%
\pgfpathlineto{\pgfqpoint{0.159731in}{1.190475in}}%
\pgfpathlineto{\pgfqpoint{0.154706in}{1.197802in}}%
\pgfpathlineto{\pgfqpoint{0.150000in}{1.203246in}}%
\pgfpathlineto{\pgfqpoint{0.150000in}{1.197802in}}%
\pgfpathlineto{\pgfqpoint{0.150000in}{1.188801in}}%
\pgfpathlineto{\pgfqpoint{0.150000in}{1.179799in}}%
\pgfpathlineto{\pgfqpoint{0.150000in}{1.170798in}}%
\pgfpathlineto{\pgfqpoint{0.150000in}{1.161796in}}%
\pgfpathlineto{\pgfqpoint{0.150000in}{1.152795in}}%
\pgfpathlineto{\pgfqpoint{0.150000in}{1.143793in}}%
\pgfpathlineto{\pgfqpoint{0.150000in}{1.134792in}}%
\pgfpathlineto{\pgfqpoint{0.150000in}{1.125790in}}%
\pgfpathlineto{\pgfqpoint{0.150000in}{1.116788in}}%
\pgfpathlineto{\pgfqpoint{0.150000in}{1.107787in}}%
\pgfpathlineto{\pgfqpoint{0.150000in}{1.098785in}}%
\pgfpathlineto{\pgfqpoint{0.150000in}{1.089784in}}%
\pgfpathlineto{\pgfqpoint{0.150000in}{1.080782in}}%
\pgfpathlineto{\pgfqpoint{0.150000in}{1.071781in}}%
\pgfpathlineto{\pgfqpoint{0.150000in}{1.062779in}}%
\pgfpathlineto{\pgfqpoint{0.150000in}{1.053778in}}%
\pgfpathlineto{\pgfqpoint{0.150000in}{1.044776in}}%
\pgfpathlineto{\pgfqpoint{0.150000in}{1.035775in}}%
\pgfpathlineto{\pgfqpoint{0.150000in}{1.026773in}}%
\pgfpathlineto{\pgfqpoint{0.150000in}{1.017771in}}%
\pgfpathlineto{\pgfqpoint{0.150000in}{1.008770in}}%
\pgfpathlineto{\pgfqpoint{0.150000in}{0.999768in}}%
\pgfpathlineto{\pgfqpoint{0.150000in}{0.990767in}}%
\pgfpathlineto{\pgfqpoint{0.150000in}{0.981765in}}%
\pgfpathlineto{\pgfqpoint{0.150000in}{0.972764in}}%
\pgfpathlineto{\pgfqpoint{0.150000in}{0.963762in}}%
\pgfpathlineto{\pgfqpoint{0.150000in}{0.954761in}}%
\pgfpathlineto{\pgfqpoint{0.150000in}{0.945759in}}%
\pgfpathlineto{\pgfqpoint{0.150000in}{0.936758in}}%
\pgfpathlineto{\pgfqpoint{0.150000in}{0.927756in}}%
\pgfpathlineto{\pgfqpoint{0.150000in}{0.918754in}}%
\pgfpathlineto{\pgfqpoint{0.150000in}{0.909753in}}%
\pgfpathlineto{\pgfqpoint{0.150000in}{0.900751in}}%
\pgfpathlineto{\pgfqpoint{0.150000in}{0.891750in}}%
\pgfpathlineto{\pgfqpoint{0.150000in}{0.882748in}}%
\pgfpathlineto{\pgfqpoint{0.150000in}{0.873747in}}%
\pgfpathlineto{\pgfqpoint{0.150000in}{0.864745in}}%
\pgfpathlineto{\pgfqpoint{0.150000in}{0.855744in}}%
\pgfpathlineto{\pgfqpoint{0.150000in}{0.846742in}}%
\pgfpathlineto{\pgfqpoint{0.150000in}{0.837741in}}%
\pgfpathlineto{\pgfqpoint{0.150000in}{0.828739in}}%
\pgfpathlineto{\pgfqpoint{0.150000in}{0.819737in}}%
\pgfpathlineto{\pgfqpoint{0.150000in}{0.810736in}}%
\pgfpathlineto{\pgfqpoint{0.150000in}{0.801734in}}%
\pgfpathlineto{\pgfqpoint{0.150000in}{0.792733in}}%
\pgfpathlineto{\pgfqpoint{0.150000in}{0.783731in}}%
\pgfpathlineto{\pgfqpoint{0.150000in}{0.774730in}}%
\pgfpathlineto{\pgfqpoint{0.150000in}{0.765728in}}%
\pgfpathlineto{\pgfqpoint{0.150000in}{0.756727in}}%
\pgfpathlineto{\pgfqpoint{0.150000in}{0.747725in}}%
\pgfpathlineto{\pgfqpoint{0.150000in}{0.745543in}}%
\pgfpathlineto{\pgfqpoint{0.154117in}{0.738724in}}%
\pgfpathlineto{\pgfqpoint{0.159731in}{0.732040in}}%
\pgfpathlineto{\pgfqpoint{0.162224in}{0.729722in}}%
\pgfpathlineto{\pgfqpoint{0.169463in}{0.724524in}}%
\pgfpathlineto{\pgfqpoint{0.176434in}{0.720720in}}%
\pgfpathlineto{\pgfqpoint{0.179194in}{0.719483in}}%
\pgfpathlineto{\pgfqpoint{0.188926in}{0.716146in}}%
\pgfpathlineto{\pgfqpoint{0.198657in}{0.713386in}}%
\pgfpathlineto{\pgfqpoint{0.204716in}{0.711719in}}%
\pgfpathlineto{\pgfqpoint{0.208389in}{0.710843in}}%
\pgfpathlineto{\pgfqpoint{0.218120in}{0.708162in}}%
\pgfpathlineto{\pgfqpoint{0.227852in}{0.704735in}}%
\pgfpathlineto{\pgfqpoint{0.232233in}{0.702717in}}%
\pgfpathlineto{\pgfqpoint{0.237583in}{0.700475in}}%
\pgfpathlineto{\pgfqpoint{0.247315in}{0.695258in}}%
\pgfpathlineto{\pgfqpoint{0.249653in}{0.693716in}}%
\pgfpathlineto{\pgfqpoint{0.257046in}{0.689075in}}%
\pgfpathlineto{\pgfqpoint{0.262946in}{0.684714in}}%
\pgfpathlineto{\pgfqpoint{0.266778in}{0.681901in}}%
\pgfpathlineto{\pgfqpoint{0.274258in}{0.675713in}}%
\pgfpathlineto{\pgfqpoint{0.276509in}{0.673784in}}%
\pgfpathlineto{\pgfqpoint{0.284156in}{0.666711in}}%
\pgfpathlineto{\pgfqpoint{0.286241in}{0.664629in}}%
\pgfpathlineto{\pgfqpoint{0.292931in}{0.657710in}}%
\pgfpathlineto{\pgfqpoint{0.295972in}{0.654165in}}%
\pgfpathlineto{\pgfqpoint{0.300686in}{0.648708in}}%
\pgfpathlineto{\pgfqpoint{0.305704in}{0.641869in}}%
\pgfpathlineto{\pgfqpoint{0.307371in}{0.639707in}}%
\pgfpathlineto{\pgfqpoint{0.313011in}{0.630705in}}%
\pgfpathlineto{\pgfqpoint{0.315435in}{0.625756in}}%
\pgfpathlineto{\pgfqpoint{0.317616in}{0.621703in}}%
\pgfpathlineto{\pgfqpoint{0.321321in}{0.612702in}}%
\pgfpathlineto{\pgfqpoint{0.324219in}{0.603700in}}%
\pgfpathlineto{\pgfqpoint{0.325167in}{0.600303in}}%
\pgfpathlineto{\pgfqpoint{0.326969in}{0.594699in}}%
\pgfpathlineto{\pgfqpoint{0.329953in}{0.585697in}}%
\pgfpathlineto{\pgfqpoint{0.333560in}{0.576696in}}%
\pgfpathlineto{\pgfqpoint{0.334898in}{0.574142in}}%
\pgfpathlineto{\pgfqpoint{0.339010in}{0.567694in}}%
\pgfpathlineto{\pgfqpoint{0.344630in}{0.560998in}}%
\pgfpathlineto{\pgfqpoint{0.347135in}{0.558693in}}%
\pgfpathclose%
\pgfusepath{fill}%
\end{pgfscope}%
\begin{pgfscope}%
\pgfpathrectangle{\pgfqpoint{0.150000in}{0.549691in}}{\pgfqpoint{1.946296in}{1.800309in}}%
\pgfusepath{clip}%
\pgfsetbuttcap%
\pgfsetroundjoin%
\definecolor{currentfill}{rgb}{0.309804,0.309804,0.309804}%
\pgfsetfillcolor{currentfill}%
\pgfsetlinewidth{0.000000pt}%
\definecolor{currentstroke}{rgb}{0.000000,0.000000,0.000000}%
\pgfsetstrokecolor{currentstroke}%
\pgfsetdash{}{0pt}%
\pgfpathmoveto{\pgfqpoint{1.395629in}{0.549691in}}%
\pgfpathlineto{\pgfqpoint{1.405361in}{0.549691in}}%
\pgfpathlineto{\pgfqpoint{1.415092in}{0.549691in}}%
\pgfpathlineto{\pgfqpoint{1.424824in}{0.549691in}}%
\pgfpathlineto{\pgfqpoint{1.434555in}{0.549691in}}%
\pgfpathlineto{\pgfqpoint{1.444287in}{0.549691in}}%
\pgfpathlineto{\pgfqpoint{1.454018in}{0.549691in}}%
\pgfpathlineto{\pgfqpoint{1.463750in}{0.549691in}}%
\pgfpathlineto{\pgfqpoint{1.473481in}{0.549691in}}%
\pgfpathlineto{\pgfqpoint{1.483213in}{0.549691in}}%
\pgfpathlineto{\pgfqpoint{1.492944in}{0.549691in}}%
\pgfpathlineto{\pgfqpoint{1.502676in}{0.549691in}}%
\pgfpathlineto{\pgfqpoint{1.512407in}{0.549691in}}%
\pgfpathlineto{\pgfqpoint{1.522139in}{0.549691in}}%
\pgfpathlineto{\pgfqpoint{1.531870in}{0.549691in}}%
\pgfpathlineto{\pgfqpoint{1.541602in}{0.549691in}}%
\pgfpathlineto{\pgfqpoint{1.551333in}{0.549691in}}%
\pgfpathlineto{\pgfqpoint{1.561064in}{0.549691in}}%
\pgfpathlineto{\pgfqpoint{1.570796in}{0.549691in}}%
\pgfpathlineto{\pgfqpoint{1.580527in}{0.549691in}}%
\pgfpathlineto{\pgfqpoint{1.590259in}{0.549691in}}%
\pgfpathlineto{\pgfqpoint{1.599990in}{0.549691in}}%
\pgfpathlineto{\pgfqpoint{1.609722in}{0.549691in}}%
\pgfpathlineto{\pgfqpoint{1.619453in}{0.549691in}}%
\pgfpathlineto{\pgfqpoint{1.629185in}{0.549691in}}%
\pgfpathlineto{\pgfqpoint{1.638916in}{0.549691in}}%
\pgfpathlineto{\pgfqpoint{1.648648in}{0.549691in}}%
\pgfpathlineto{\pgfqpoint{1.658379in}{0.549691in}}%
\pgfpathlineto{\pgfqpoint{1.668111in}{0.549691in}}%
\pgfpathlineto{\pgfqpoint{1.677842in}{0.549691in}}%
\pgfpathlineto{\pgfqpoint{1.687574in}{0.549691in}}%
\pgfpathlineto{\pgfqpoint{1.697305in}{0.549691in}}%
\pgfpathlineto{\pgfqpoint{1.707037in}{0.549691in}}%
\pgfpathlineto{\pgfqpoint{1.716768in}{0.549691in}}%
\pgfpathlineto{\pgfqpoint{1.726500in}{0.549691in}}%
\pgfpathlineto{\pgfqpoint{1.736231in}{0.549691in}}%
\pgfpathlineto{\pgfqpoint{1.745963in}{0.549691in}}%
\pgfpathlineto{\pgfqpoint{1.755694in}{0.549691in}}%
\pgfpathlineto{\pgfqpoint{1.765426in}{0.549691in}}%
\pgfpathlineto{\pgfqpoint{1.775157in}{0.549691in}}%
\pgfpathlineto{\pgfqpoint{1.784889in}{0.549691in}}%
\pgfpathlineto{\pgfqpoint{1.794620in}{0.549691in}}%
\pgfpathlineto{\pgfqpoint{1.804351in}{0.549691in}}%
\pgfpathlineto{\pgfqpoint{1.814083in}{0.549691in}}%
\pgfpathlineto{\pgfqpoint{1.823814in}{0.549691in}}%
\pgfpathlineto{\pgfqpoint{1.833546in}{0.549691in}}%
\pgfpathlineto{\pgfqpoint{1.843277in}{0.549691in}}%
\pgfpathlineto{\pgfqpoint{1.853009in}{0.549691in}}%
\pgfpathlineto{\pgfqpoint{1.862740in}{0.549691in}}%
\pgfpathlineto{\pgfqpoint{1.872472in}{0.549691in}}%
\pgfpathlineto{\pgfqpoint{1.882203in}{0.549691in}}%
\pgfpathlineto{\pgfqpoint{1.884562in}{0.549691in}}%
\pgfpathlineto{\pgfqpoint{1.891935in}{0.553499in}}%
\pgfpathlineto{\pgfqpoint{1.899161in}{0.558693in}}%
\pgfpathlineto{\pgfqpoint{1.901666in}{0.560998in}}%
\pgfpathlineto{\pgfqpoint{1.907286in}{0.567694in}}%
\pgfpathlineto{\pgfqpoint{1.911398in}{0.574142in}}%
\pgfpathlineto{\pgfqpoint{1.912735in}{0.576696in}}%
\pgfpathlineto{\pgfqpoint{1.916343in}{0.585697in}}%
\pgfpathlineto{\pgfqpoint{1.919327in}{0.594699in}}%
\pgfpathlineto{\pgfqpoint{1.921129in}{0.600303in}}%
\pgfpathlineto{\pgfqpoint{1.922077in}{0.603700in}}%
\pgfpathlineto{\pgfqpoint{1.924975in}{0.612702in}}%
\pgfpathlineto{\pgfqpoint{1.928680in}{0.621703in}}%
\pgfpathlineto{\pgfqpoint{1.930861in}{0.625756in}}%
\pgfpathlineto{\pgfqpoint{1.933285in}{0.630705in}}%
\pgfpathlineto{\pgfqpoint{1.938925in}{0.639707in}}%
\pgfpathlineto{\pgfqpoint{1.940592in}{0.641869in}}%
\pgfpathlineto{\pgfqpoint{1.945610in}{0.648708in}}%
\pgfpathlineto{\pgfqpoint{1.950324in}{0.654165in}}%
\pgfpathlineto{\pgfqpoint{1.953365in}{0.657710in}}%
\pgfpathlineto{\pgfqpoint{1.960055in}{0.664629in}}%
\pgfpathlineto{\pgfqpoint{1.962140in}{0.666711in}}%
\pgfpathlineto{\pgfqpoint{1.969787in}{0.673784in}}%
\pgfpathlineto{\pgfqpoint{1.972038in}{0.675713in}}%
\pgfpathlineto{\pgfqpoint{1.979518in}{0.681901in}}%
\pgfpathlineto{\pgfqpoint{1.983350in}{0.684714in}}%
\pgfpathlineto{\pgfqpoint{1.989250in}{0.689075in}}%
\pgfpathlineto{\pgfqpoint{1.996643in}{0.693716in}}%
\pgfpathlineto{\pgfqpoint{1.998981in}{0.695258in}}%
\pgfpathlineto{\pgfqpoint{2.008713in}{0.700475in}}%
\pgfpathlineto{\pgfqpoint{2.014062in}{0.702717in}}%
\pgfpathlineto{\pgfqpoint{2.018444in}{0.704735in}}%
\pgfpathlineto{\pgfqpoint{2.028175in}{0.708162in}}%
\pgfpathlineto{\pgfqpoint{2.037907in}{0.710843in}}%
\pgfpathlineto{\pgfqpoint{2.041579in}{0.711719in}}%
\pgfpathlineto{\pgfqpoint{2.047638in}{0.713386in}}%
\pgfpathlineto{\pgfqpoint{2.057370in}{0.716146in}}%
\pgfpathlineto{\pgfqpoint{2.067101in}{0.719483in}}%
\pgfpathlineto{\pgfqpoint{2.069862in}{0.720720in}}%
\pgfpathlineto{\pgfqpoint{2.076833in}{0.724524in}}%
\pgfpathlineto{\pgfqpoint{2.084072in}{0.729722in}}%
\pgfpathlineto{\pgfqpoint{2.086564in}{0.732040in}}%
\pgfpathlineto{\pgfqpoint{2.092179in}{0.738724in}}%
\pgfpathlineto{\pgfqpoint{2.096296in}{0.745543in}}%
\pgfpathlineto{\pgfqpoint{2.096296in}{0.747725in}}%
\pgfpathlineto{\pgfqpoint{2.096296in}{0.756727in}}%
\pgfpathlineto{\pgfqpoint{2.096296in}{0.765728in}}%
\pgfpathlineto{\pgfqpoint{2.096296in}{0.774730in}}%
\pgfpathlineto{\pgfqpoint{2.096296in}{0.783731in}}%
\pgfpathlineto{\pgfqpoint{2.096296in}{0.792733in}}%
\pgfpathlineto{\pgfqpoint{2.096296in}{0.801734in}}%
\pgfpathlineto{\pgfqpoint{2.096296in}{0.810736in}}%
\pgfpathlineto{\pgfqpoint{2.096296in}{0.819737in}}%
\pgfpathlineto{\pgfqpoint{2.096296in}{0.828739in}}%
\pgfpathlineto{\pgfqpoint{2.096296in}{0.837741in}}%
\pgfpathlineto{\pgfqpoint{2.096296in}{0.846742in}}%
\pgfpathlineto{\pgfqpoint{2.096296in}{0.855744in}}%
\pgfpathlineto{\pgfqpoint{2.096296in}{0.864745in}}%
\pgfpathlineto{\pgfqpoint{2.096296in}{0.873747in}}%
\pgfpathlineto{\pgfqpoint{2.096296in}{0.882748in}}%
\pgfpathlineto{\pgfqpoint{2.096296in}{0.891750in}}%
\pgfpathlineto{\pgfqpoint{2.096296in}{0.900751in}}%
\pgfpathlineto{\pgfqpoint{2.096296in}{0.909753in}}%
\pgfpathlineto{\pgfqpoint{2.096296in}{0.918754in}}%
\pgfpathlineto{\pgfqpoint{2.096296in}{0.927756in}}%
\pgfpathlineto{\pgfqpoint{2.096296in}{0.936758in}}%
\pgfpathlineto{\pgfqpoint{2.096296in}{0.945759in}}%
\pgfpathlineto{\pgfqpoint{2.096296in}{0.954761in}}%
\pgfpathlineto{\pgfqpoint{2.096296in}{0.963762in}}%
\pgfpathlineto{\pgfqpoint{2.096296in}{0.972764in}}%
\pgfpathlineto{\pgfqpoint{2.096296in}{0.981765in}}%
\pgfpathlineto{\pgfqpoint{2.096296in}{0.990767in}}%
\pgfpathlineto{\pgfqpoint{2.096296in}{0.999768in}}%
\pgfpathlineto{\pgfqpoint{2.096296in}{1.008770in}}%
\pgfpathlineto{\pgfqpoint{2.096296in}{1.017771in}}%
\pgfpathlineto{\pgfqpoint{2.096296in}{1.026773in}}%
\pgfpathlineto{\pgfqpoint{2.096296in}{1.035775in}}%
\pgfpathlineto{\pgfqpoint{2.096296in}{1.044776in}}%
\pgfpathlineto{\pgfqpoint{2.096296in}{1.053778in}}%
\pgfpathlineto{\pgfqpoint{2.096296in}{1.062779in}}%
\pgfpathlineto{\pgfqpoint{2.096296in}{1.071781in}}%
\pgfpathlineto{\pgfqpoint{2.096296in}{1.080782in}}%
\pgfpathlineto{\pgfqpoint{2.096296in}{1.089784in}}%
\pgfpathlineto{\pgfqpoint{2.096296in}{1.098785in}}%
\pgfpathlineto{\pgfqpoint{2.096296in}{1.107787in}}%
\pgfpathlineto{\pgfqpoint{2.096296in}{1.116788in}}%
\pgfpathlineto{\pgfqpoint{2.096296in}{1.125790in}}%
\pgfpathlineto{\pgfqpoint{2.096296in}{1.134792in}}%
\pgfpathlineto{\pgfqpoint{2.096296in}{1.143793in}}%
\pgfpathlineto{\pgfqpoint{2.096296in}{1.152795in}}%
\pgfpathlineto{\pgfqpoint{2.096296in}{1.161796in}}%
\pgfpathlineto{\pgfqpoint{2.096296in}{1.170798in}}%
\pgfpathlineto{\pgfqpoint{2.096296in}{1.179799in}}%
\pgfpathlineto{\pgfqpoint{2.096296in}{1.188801in}}%
\pgfpathlineto{\pgfqpoint{2.096296in}{1.197802in}}%
\pgfpathlineto{\pgfqpoint{2.096296in}{1.203246in}}%
\pgfpathlineto{\pgfqpoint{2.091590in}{1.197802in}}%
\pgfpathlineto{\pgfqpoint{2.086564in}{1.190475in}}%
\pgfpathlineto{\pgfqpoint{2.085095in}{1.188801in}}%
\pgfpathlineto{\pgfqpoint{2.079640in}{1.179799in}}%
\pgfpathlineto{\pgfqpoint{2.076833in}{1.171010in}}%
\pgfpathlineto{\pgfqpoint{2.076743in}{1.170798in}}%
\pgfpathlineto{\pgfqpoint{2.076178in}{1.161796in}}%
\pgfpathlineto{\pgfqpoint{2.076833in}{1.159052in}}%
\pgfpathlineto{\pgfqpoint{2.077966in}{1.152795in}}%
\pgfpathlineto{\pgfqpoint{2.081056in}{1.143793in}}%
\pgfpathlineto{\pgfqpoint{2.084847in}{1.134792in}}%
\pgfpathlineto{\pgfqpoint{2.086564in}{1.130558in}}%
\pgfpathlineto{\pgfqpoint{2.088076in}{1.125790in}}%
\pgfpathlineto{\pgfqpoint{2.090164in}{1.116788in}}%
\pgfpathlineto{\pgfqpoint{2.090896in}{1.107787in}}%
\pgfpathlineto{\pgfqpoint{2.089801in}{1.098785in}}%
\pgfpathlineto{\pgfqpoint{2.086578in}{1.089784in}}%
\pgfpathlineto{\pgfqpoint{2.086564in}{1.089761in}}%
\pgfpathlineto{\pgfqpoint{2.079600in}{1.080782in}}%
\pgfpathlineto{\pgfqpoint{2.076833in}{1.078216in}}%
\pgfpathlineto{\pgfqpoint{2.067701in}{1.071781in}}%
\pgfpathlineto{\pgfqpoint{2.067101in}{1.071440in}}%
\pgfpathlineto{\pgfqpoint{2.057370in}{1.067212in}}%
\pgfpathlineto{\pgfqpoint{2.047638in}{1.063716in}}%
\pgfpathlineto{\pgfqpoint{2.044952in}{1.062779in}}%
\pgfpathlineto{\pgfqpoint{2.037907in}{1.060632in}}%
\pgfpathlineto{\pgfqpoint{2.028175in}{1.057209in}}%
\pgfpathlineto{\pgfqpoint{2.020545in}{1.053778in}}%
\pgfpathlineto{\pgfqpoint{2.018444in}{1.052895in}}%
\pgfpathlineto{\pgfqpoint{2.008713in}{1.047553in}}%
\pgfpathlineto{\pgfqpoint{2.004759in}{1.044776in}}%
\pgfpathlineto{\pgfqpoint{1.998981in}{1.040756in}}%
\pgfpathlineto{\pgfqpoint{1.993161in}{1.035775in}}%
\pgfpathlineto{\pgfqpoint{1.989250in}{1.032262in}}%
\pgfpathlineto{\pgfqpoint{1.984060in}{1.026773in}}%
\pgfpathlineto{\pgfqpoint{1.979518in}{1.021420in}}%
\pgfpathlineto{\pgfqpoint{1.976771in}{1.017771in}}%
\pgfpathlineto{\pgfqpoint{1.971090in}{1.008770in}}%
\pgfpathlineto{\pgfqpoint{1.969787in}{1.006090in}}%
\pgfpathlineto{\pgfqpoint{1.966938in}{0.999768in}}%
\pgfpathlineto{\pgfqpoint{1.964153in}{0.990767in}}%
\pgfpathlineto{\pgfqpoint{1.962573in}{0.981765in}}%
\pgfpathlineto{\pgfqpoint{1.962009in}{0.972764in}}%
\pgfpathlineto{\pgfqpoint{1.962163in}{0.963762in}}%
\pgfpathlineto{\pgfqpoint{1.962663in}{0.954761in}}%
\pgfpathlineto{\pgfqpoint{1.963094in}{0.945759in}}%
\pgfpathlineto{\pgfqpoint{1.963043in}{0.936758in}}%
\pgfpathlineto{\pgfqpoint{1.962136in}{0.927756in}}%
\pgfpathlineto{\pgfqpoint{1.960077in}{0.918754in}}%
\pgfpathlineto{\pgfqpoint{1.960055in}{0.918695in}}%
\pgfpathlineto{\pgfqpoint{1.956793in}{0.909753in}}%
\pgfpathlineto{\pgfqpoint{1.952149in}{0.900751in}}%
\pgfpathlineto{\pgfqpoint{1.950324in}{0.897985in}}%
\pgfpathlineto{\pgfqpoint{1.946181in}{0.891750in}}%
\pgfpathlineto{\pgfqpoint{1.940592in}{0.884664in}}%
\pgfpathlineto{\pgfqpoint{1.939004in}{0.882748in}}%
\pgfpathlineto{\pgfqpoint{1.930861in}{0.873952in}}%
\pgfpathlineto{\pgfqpoint{1.930652in}{0.873747in}}%
\pgfpathlineto{\pgfqpoint{1.921129in}{0.864908in}}%
\pgfpathlineto{\pgfqpoint{1.920927in}{0.864745in}}%
\pgfpathlineto{\pgfqpoint{1.911398in}{0.857135in}}%
\pgfpathlineto{\pgfqpoint{1.909277in}{0.855744in}}%
\pgfpathlineto{\pgfqpoint{1.901666in}{0.850544in}}%
\pgfpathlineto{\pgfqpoint{1.894458in}{0.846742in}}%
\pgfpathlineto{\pgfqpoint{1.891935in}{0.845286in}}%
\pgfpathlineto{\pgfqpoint{1.882203in}{0.841251in}}%
\pgfpathlineto{\pgfqpoint{1.872472in}{0.838572in}}%
\pgfpathlineto{\pgfqpoint{1.867627in}{0.837741in}}%
\pgfpathlineto{\pgfqpoint{1.862740in}{0.836773in}}%
\pgfpathlineto{\pgfqpoint{1.853009in}{0.835501in}}%
\pgfpathlineto{\pgfqpoint{1.843277in}{0.834321in}}%
\pgfpathlineto{\pgfqpoint{1.833546in}{0.832673in}}%
\pgfpathlineto{\pgfqpoint{1.823814in}{0.830048in}}%
\pgfpathlineto{\pgfqpoint{1.820631in}{0.828739in}}%
\pgfpathlineto{\pgfqpoint{1.814083in}{0.825447in}}%
\pgfpathlineto{\pgfqpoint{1.806021in}{0.819737in}}%
\pgfpathlineto{\pgfqpoint{1.804351in}{0.818193in}}%
\pgfpathlineto{\pgfqpoint{1.798179in}{0.810736in}}%
\pgfpathlineto{\pgfqpoint{1.794620in}{0.804679in}}%
\pgfpathlineto{\pgfqpoint{1.793204in}{0.801734in}}%
\pgfpathlineto{\pgfqpoint{1.790367in}{0.792733in}}%
\pgfpathlineto{\pgfqpoint{1.788585in}{0.783731in}}%
\pgfpathlineto{\pgfqpoint{1.787310in}{0.774730in}}%
\pgfpathlineto{\pgfqpoint{1.785935in}{0.765728in}}%
\pgfpathlineto{\pgfqpoint{1.784889in}{0.761208in}}%
\pgfpathlineto{\pgfqpoint{1.783989in}{0.756727in}}%
\pgfpathlineto{\pgfqpoint{1.781093in}{0.747725in}}%
\pgfpathlineto{\pgfqpoint{1.776731in}{0.738724in}}%
\pgfpathlineto{\pgfqpoint{1.775157in}{0.736390in}}%
\pgfpathlineto{\pgfqpoint{1.771047in}{0.729722in}}%
\pgfpathlineto{\pgfqpoint{1.765426in}{0.722682in}}%
\pgfpathlineto{\pgfqpoint{1.763922in}{0.720720in}}%
\pgfpathlineto{\pgfqpoint{1.755694in}{0.711905in}}%
\pgfpathlineto{\pgfqpoint{1.755519in}{0.711719in}}%
\pgfpathlineto{\pgfqpoint{1.745963in}{0.702911in}}%
\pgfpathlineto{\pgfqpoint{1.745741in}{0.702717in}}%
\pgfpathlineto{\pgfqpoint{1.736231in}{0.695185in}}%
\pgfpathlineto{\pgfqpoint{1.734160in}{0.693716in}}%
\pgfpathlineto{\pgfqpoint{1.726500in}{0.688546in}}%
\pgfpathlineto{\pgfqpoint{1.719759in}{0.684714in}}%
\pgfpathlineto{\pgfqpoint{1.716768in}{0.683026in}}%
\pgfpathlineto{\pgfqpoint{1.707037in}{0.678731in}}%
\pgfpathlineto{\pgfqpoint{1.697369in}{0.675713in}}%
\pgfpathlineto{\pgfqpoint{1.697305in}{0.675692in}}%
\pgfpathlineto{\pgfqpoint{1.687574in}{0.673788in}}%
\pgfpathlineto{\pgfqpoint{1.677842in}{0.672949in}}%
\pgfpathlineto{\pgfqpoint{1.668111in}{0.672902in}}%
\pgfpathlineto{\pgfqpoint{1.658379in}{0.673301in}}%
\pgfpathlineto{\pgfqpoint{1.648648in}{0.673763in}}%
\pgfpathlineto{\pgfqpoint{1.638916in}{0.673906in}}%
\pgfpathlineto{\pgfqpoint{1.629185in}{0.673383in}}%
\pgfpathlineto{\pgfqpoint{1.619453in}{0.671922in}}%
\pgfpathlineto{\pgfqpoint{1.609722in}{0.669346in}}%
\pgfpathlineto{\pgfqpoint{1.602887in}{0.666711in}}%
\pgfpathlineto{\pgfqpoint{1.599990in}{0.665505in}}%
\pgfpathlineto{\pgfqpoint{1.590259in}{0.660251in}}%
\pgfpathlineto{\pgfqpoint{1.586314in}{0.657710in}}%
\pgfpathlineto{\pgfqpoint{1.580527in}{0.653509in}}%
\pgfpathlineto{\pgfqpoint{1.574594in}{0.648708in}}%
\pgfpathlineto{\pgfqpoint{1.570796in}{0.645090in}}%
\pgfpathlineto{\pgfqpoint{1.565410in}{0.639707in}}%
\pgfpathlineto{\pgfqpoint{1.561064in}{0.634362in}}%
\pgfpathlineto{\pgfqpoint{1.558062in}{0.630705in}}%
\pgfpathlineto{\pgfqpoint{1.552287in}{0.621703in}}%
\pgfpathlineto{\pgfqpoint{1.551333in}{0.619760in}}%
\pgfpathlineto{\pgfqpoint{1.547623in}{0.612702in}}%
\pgfpathlineto{\pgfqpoint{1.543922in}{0.603700in}}%
\pgfpathlineto{\pgfqpoint{1.541602in}{0.597183in}}%
\pgfpathlineto{\pgfqpoint{1.540589in}{0.594699in}}%
\pgfpathlineto{\pgfqpoint{1.536809in}{0.585697in}}%
\pgfpathlineto{\pgfqpoint{1.532238in}{0.576696in}}%
\pgfpathlineto{\pgfqpoint{1.531870in}{0.576142in}}%
\pgfpathlineto{\pgfqpoint{1.524913in}{0.567694in}}%
\pgfpathlineto{\pgfqpoint{1.522139in}{0.565134in}}%
\pgfpathlineto{\pgfqpoint{1.512432in}{0.558693in}}%
\pgfpathlineto{\pgfqpoint{1.512407in}{0.558680in}}%
\pgfpathlineto{\pgfqpoint{1.502676in}{0.555698in}}%
\pgfpathlineto{\pgfqpoint{1.492944in}{0.554686in}}%
\pgfpathlineto{\pgfqpoint{1.483213in}{0.555363in}}%
\pgfpathlineto{\pgfqpoint{1.473481in}{0.557295in}}%
\pgfpathlineto{\pgfqpoint{1.468326in}{0.558693in}}%
\pgfpathlineto{\pgfqpoint{1.463750in}{0.560281in}}%
\pgfpathlineto{\pgfqpoint{1.454018in}{0.563788in}}%
\pgfpathlineto{\pgfqpoint{1.444287in}{0.566646in}}%
\pgfpathlineto{\pgfqpoint{1.437522in}{0.567694in}}%
\pgfpathlineto{\pgfqpoint{1.434555in}{0.568300in}}%
\pgfpathlineto{\pgfqpoint{1.424824in}{0.567777in}}%
\pgfpathlineto{\pgfqpoint{1.424594in}{0.567694in}}%
\pgfpathlineto{\pgfqpoint{1.415092in}{0.565097in}}%
\pgfpathlineto{\pgfqpoint{1.405361in}{0.560052in}}%
\pgfpathlineto{\pgfqpoint{1.403551in}{0.558693in}}%
\pgfpathlineto{\pgfqpoint{1.395629in}{0.554044in}}%
\pgfpathlineto{\pgfqpoint{1.389745in}{0.549691in}}%
\pgfpathclose%
\pgfusepath{fill}%
\end{pgfscope}%
\begin{pgfscope}%
\pgfpathrectangle{\pgfqpoint{0.150000in}{0.549691in}}{\pgfqpoint{1.946296in}{1.800309in}}%
\pgfusepath{clip}%
\pgfsetbuttcap%
\pgfsetroundjoin%
\definecolor{currentfill}{rgb}{0.309804,0.309804,0.309804}%
\pgfsetfillcolor{currentfill}%
\pgfsetlinewidth{0.000000pt}%
\definecolor{currentstroke}{rgb}{0.000000,0.000000,0.000000}%
\pgfsetstrokecolor{currentstroke}%
\pgfsetdash{}{0pt}%
\pgfpathmoveto{\pgfqpoint{0.154706in}{1.701889in}}%
\pgfpathlineto{\pgfqpoint{0.159731in}{1.709217in}}%
\pgfpathlineto{\pgfqpoint{0.161201in}{1.710890in}}%
\pgfpathlineto{\pgfqpoint{0.166655in}{1.719892in}}%
\pgfpathlineto{\pgfqpoint{0.169463in}{1.728681in}}%
\pgfpathlineto{\pgfqpoint{0.169552in}{1.728893in}}%
\pgfpathlineto{\pgfqpoint{0.170117in}{1.737895in}}%
\pgfpathlineto{\pgfqpoint{0.169463in}{1.740640in}}%
\pgfpathlineto{\pgfqpoint{0.168329in}{1.746897in}}%
\pgfpathlineto{\pgfqpoint{0.165240in}{1.755898in}}%
\pgfpathlineto{\pgfqpoint{0.161449in}{1.764900in}}%
\pgfpathlineto{\pgfqpoint{0.159731in}{1.769133in}}%
\pgfpathlineto{\pgfqpoint{0.158220in}{1.773901in}}%
\pgfpathlineto{\pgfqpoint{0.156132in}{1.782903in}}%
\pgfpathlineto{\pgfqpoint{0.155400in}{1.791904in}}%
\pgfpathlineto{\pgfqpoint{0.156494in}{1.800906in}}%
\pgfpathlineto{\pgfqpoint{0.159717in}{1.809907in}}%
\pgfpathlineto{\pgfqpoint{0.159731in}{1.809931in}}%
\pgfpathlineto{\pgfqpoint{0.166695in}{1.818909in}}%
\pgfpathlineto{\pgfqpoint{0.169463in}{1.821476in}}%
\pgfpathlineto{\pgfqpoint{0.178595in}{1.827910in}}%
\pgfpathlineto{\pgfqpoint{0.179194in}{1.828251in}}%
\pgfpathlineto{\pgfqpoint{0.188926in}{1.832479in}}%
\pgfpathlineto{\pgfqpoint{0.198657in}{1.835976in}}%
\pgfpathlineto{\pgfqpoint{0.201343in}{1.836912in}}%
\pgfpathlineto{\pgfqpoint{0.208389in}{1.839059in}}%
\pgfpathlineto{\pgfqpoint{0.218120in}{1.842482in}}%
\pgfpathlineto{\pgfqpoint{0.225751in}{1.845914in}}%
\pgfpathlineto{\pgfqpoint{0.227852in}{1.846796in}}%
\pgfpathlineto{\pgfqpoint{0.237583in}{1.852138in}}%
\pgfpathlineto{\pgfqpoint{0.241537in}{1.854915in}}%
\pgfpathlineto{\pgfqpoint{0.247315in}{1.858935in}}%
\pgfpathlineto{\pgfqpoint{0.253135in}{1.863917in}}%
\pgfpathlineto{\pgfqpoint{0.257046in}{1.867429in}}%
\pgfpathlineto{\pgfqpoint{0.262236in}{1.872918in}}%
\pgfpathlineto{\pgfqpoint{0.266778in}{1.878271in}}%
\pgfpathlineto{\pgfqpoint{0.269525in}{1.881920in}}%
\pgfpathlineto{\pgfqpoint{0.275205in}{1.890921in}}%
\pgfpathlineto{\pgfqpoint{0.276509in}{1.893601in}}%
\pgfpathlineto{\pgfqpoint{0.279358in}{1.899923in}}%
\pgfpathlineto{\pgfqpoint{0.282143in}{1.908924in}}%
\pgfpathlineto{\pgfqpoint{0.283722in}{1.917926in}}%
\pgfpathlineto{\pgfqpoint{0.284287in}{1.926927in}}%
\pgfpathlineto{\pgfqpoint{0.284133in}{1.935929in}}%
\pgfpathlineto{\pgfqpoint{0.283633in}{1.944931in}}%
\pgfpathlineto{\pgfqpoint{0.283202in}{1.953932in}}%
\pgfpathlineto{\pgfqpoint{0.283253in}{1.962934in}}%
\pgfpathlineto{\pgfqpoint{0.284160in}{1.971935in}}%
\pgfpathlineto{\pgfqpoint{0.286218in}{1.980937in}}%
\pgfpathlineto{\pgfqpoint{0.286241in}{1.980996in}}%
\pgfpathlineto{\pgfqpoint{0.289503in}{1.989938in}}%
\pgfpathlineto{\pgfqpoint{0.294147in}{1.998940in}}%
\pgfpathlineto{\pgfqpoint{0.295972in}{2.001706in}}%
\pgfpathlineto{\pgfqpoint{0.300115in}{2.007941in}}%
\pgfpathlineto{\pgfqpoint{0.305704in}{2.015027in}}%
\pgfpathlineto{\pgfqpoint{0.307291in}{2.016943in}}%
\pgfpathlineto{\pgfqpoint{0.315435in}{2.025739in}}%
\pgfpathlineto{\pgfqpoint{0.315644in}{2.025944in}}%
\pgfpathlineto{\pgfqpoint{0.325167in}{2.034784in}}%
\pgfpathlineto{\pgfqpoint{0.325368in}{2.034946in}}%
\pgfpathlineto{\pgfqpoint{0.334898in}{2.042556in}}%
\pgfpathlineto{\pgfqpoint{0.337019in}{2.043947in}}%
\pgfpathlineto{\pgfqpoint{0.344630in}{2.049147in}}%
\pgfpathlineto{\pgfqpoint{0.351838in}{2.052949in}}%
\pgfpathlineto{\pgfqpoint{0.354361in}{2.054405in}}%
\pgfpathlineto{\pgfqpoint{0.364093in}{2.058440in}}%
\pgfpathlineto{\pgfqpoint{0.373824in}{2.061119in}}%
\pgfpathlineto{\pgfqpoint{0.378669in}{2.061951in}}%
\pgfpathlineto{\pgfqpoint{0.383556in}{2.062918in}}%
\pgfpathlineto{\pgfqpoint{0.393287in}{2.064191in}}%
\pgfpathlineto{\pgfqpoint{0.403018in}{2.065370in}}%
\pgfpathlineto{\pgfqpoint{0.412750in}{2.067018in}}%
\pgfpathlineto{\pgfqpoint{0.422481in}{2.069643in}}%
\pgfpathlineto{\pgfqpoint{0.425665in}{2.070952in}}%
\pgfpathlineto{\pgfqpoint{0.432213in}{2.074244in}}%
\pgfpathlineto{\pgfqpoint{0.440274in}{2.079954in}}%
\pgfpathlineto{\pgfqpoint{0.441944in}{2.081498in}}%
\pgfpathlineto{\pgfqpoint{0.448117in}{2.088955in}}%
\pgfpathlineto{\pgfqpoint{0.451676in}{2.095012in}}%
\pgfpathlineto{\pgfqpoint{0.453091in}{2.097957in}}%
\pgfpathlineto{\pgfqpoint{0.455929in}{2.106958in}}%
\pgfpathlineto{\pgfqpoint{0.457711in}{2.115960in}}%
\pgfpathlineto{\pgfqpoint{0.458986in}{2.124961in}}%
\pgfpathlineto{\pgfqpoint{0.460361in}{2.133963in}}%
\pgfpathlineto{\pgfqpoint{0.461407in}{2.138483in}}%
\pgfpathlineto{\pgfqpoint{0.462307in}{2.142964in}}%
\pgfpathlineto{\pgfqpoint{0.465203in}{2.151966in}}%
\pgfpathlineto{\pgfqpoint{0.469565in}{2.160968in}}%
\pgfpathlineto{\pgfqpoint{0.471139in}{2.163301in}}%
\pgfpathlineto{\pgfqpoint{0.475249in}{2.169969in}}%
\pgfpathlineto{\pgfqpoint{0.480870in}{2.177009in}}%
\pgfpathlineto{\pgfqpoint{0.482374in}{2.178971in}}%
\pgfpathlineto{\pgfqpoint{0.490602in}{2.187786in}}%
\pgfpathlineto{\pgfqpoint{0.490777in}{2.187972in}}%
\pgfpathlineto{\pgfqpoint{0.500333in}{2.196780in}}%
\pgfpathlineto{\pgfqpoint{0.500555in}{2.196974in}}%
\pgfpathlineto{\pgfqpoint{0.510065in}{2.204507in}}%
\pgfpathlineto{\pgfqpoint{0.512136in}{2.205975in}}%
\pgfpathlineto{\pgfqpoint{0.519796in}{2.211145in}}%
\pgfpathlineto{\pgfqpoint{0.526537in}{2.214977in}}%
\pgfpathlineto{\pgfqpoint{0.529528in}{2.216665in}}%
\pgfpathlineto{\pgfqpoint{0.539259in}{2.220961in}}%
\pgfpathlineto{\pgfqpoint{0.548927in}{2.223978in}}%
\pgfpathlineto{\pgfqpoint{0.548991in}{2.223999in}}%
\pgfpathlineto{\pgfqpoint{0.558722in}{2.225903in}}%
\pgfpathlineto{\pgfqpoint{0.568454in}{2.226742in}}%
\pgfpathlineto{\pgfqpoint{0.578185in}{2.226789in}}%
\pgfpathlineto{\pgfqpoint{0.587917in}{2.226390in}}%
\pgfpathlineto{\pgfqpoint{0.597648in}{2.225928in}}%
\pgfpathlineto{\pgfqpoint{0.607380in}{2.225785in}}%
\pgfpathlineto{\pgfqpoint{0.617111in}{2.226308in}}%
\pgfpathlineto{\pgfqpoint{0.626842in}{2.227769in}}%
\pgfpathlineto{\pgfqpoint{0.636574in}{2.230345in}}%
\pgfpathlineto{\pgfqpoint{0.643408in}{2.232980in}}%
\pgfpathlineto{\pgfqpoint{0.646305in}{2.234186in}}%
\pgfpathlineto{\pgfqpoint{0.656037in}{2.239441in}}%
\pgfpathlineto{\pgfqpoint{0.659981in}{2.241981in}}%
\pgfpathlineto{\pgfqpoint{0.665768in}{2.246182in}}%
\pgfpathlineto{\pgfqpoint{0.671702in}{2.250983in}}%
\pgfpathlineto{\pgfqpoint{0.675500in}{2.254601in}}%
\pgfpathlineto{\pgfqpoint{0.680886in}{2.259985in}}%
\pgfpathlineto{\pgfqpoint{0.685231in}{2.265329in}}%
\pgfpathlineto{\pgfqpoint{0.688233in}{2.268986in}}%
\pgfpathlineto{\pgfqpoint{0.694009in}{2.277988in}}%
\pgfpathlineto{\pgfqpoint{0.694963in}{2.279931in}}%
\pgfpathlineto{\pgfqpoint{0.698672in}{2.286989in}}%
\pgfpathlineto{\pgfqpoint{0.702373in}{2.295991in}}%
\pgfpathlineto{\pgfqpoint{0.704694in}{2.302508in}}%
\pgfpathlineto{\pgfqpoint{0.705707in}{2.304992in}}%
\pgfpathlineto{\pgfqpoint{0.709487in}{2.313994in}}%
\pgfpathlineto{\pgfqpoint{0.714058in}{2.322995in}}%
\pgfpathlineto{\pgfqpoint{0.714426in}{2.323550in}}%
\pgfpathlineto{\pgfqpoint{0.721382in}{2.331997in}}%
\pgfpathlineto{\pgfqpoint{0.724157in}{2.334557in}}%
\pgfpathlineto{\pgfqpoint{0.733864in}{2.340998in}}%
\pgfpathlineto{\pgfqpoint{0.733889in}{2.341011in}}%
\pgfpathlineto{\pgfqpoint{0.743620in}{2.343993in}}%
\pgfpathlineto{\pgfqpoint{0.753352in}{2.345005in}}%
\pgfpathlineto{\pgfqpoint{0.763083in}{2.344328in}}%
\pgfpathlineto{\pgfqpoint{0.772815in}{2.342396in}}%
\pgfpathlineto{\pgfqpoint{0.777970in}{2.340998in}}%
\pgfpathlineto{\pgfqpoint{0.782546in}{2.339410in}}%
\pgfpathlineto{\pgfqpoint{0.792278in}{2.335903in}}%
\pgfpathlineto{\pgfqpoint{0.802009in}{2.333045in}}%
\pgfpathlineto{\pgfqpoint{0.808773in}{2.331997in}}%
\pgfpathlineto{\pgfqpoint{0.811741in}{2.331392in}}%
\pgfpathlineto{\pgfqpoint{0.821472in}{2.331914in}}%
\pgfpathlineto{\pgfqpoint{0.821702in}{2.331997in}}%
\pgfpathlineto{\pgfqpoint{0.831204in}{2.334594in}}%
\pgfpathlineto{\pgfqpoint{0.840935in}{2.339640in}}%
\pgfpathlineto{\pgfqpoint{0.842745in}{2.340998in}}%
\pgfpathlineto{\pgfqpoint{0.850667in}{2.345647in}}%
\pgfpathlineto{\pgfqpoint{0.856551in}{2.350000in}}%
\pgfpathlineto{\pgfqpoint{0.850667in}{2.350000in}}%
\pgfpathlineto{\pgfqpoint{0.840935in}{2.350000in}}%
\pgfpathlineto{\pgfqpoint{0.831204in}{2.350000in}}%
\pgfpathlineto{\pgfqpoint{0.821472in}{2.350000in}}%
\pgfpathlineto{\pgfqpoint{0.811741in}{2.350000in}}%
\pgfpathlineto{\pgfqpoint{0.802009in}{2.350000in}}%
\pgfpathlineto{\pgfqpoint{0.792278in}{2.350000in}}%
\pgfpathlineto{\pgfqpoint{0.782546in}{2.350000in}}%
\pgfpathlineto{\pgfqpoint{0.772815in}{2.350000in}}%
\pgfpathlineto{\pgfqpoint{0.763083in}{2.350000in}}%
\pgfpathlineto{\pgfqpoint{0.753352in}{2.350000in}}%
\pgfpathlineto{\pgfqpoint{0.743620in}{2.350000in}}%
\pgfpathlineto{\pgfqpoint{0.733889in}{2.350000in}}%
\pgfpathlineto{\pgfqpoint{0.724157in}{2.350000in}}%
\pgfpathlineto{\pgfqpoint{0.714426in}{2.350000in}}%
\pgfpathlineto{\pgfqpoint{0.704694in}{2.350000in}}%
\pgfpathlineto{\pgfqpoint{0.694963in}{2.350000in}}%
\pgfpathlineto{\pgfqpoint{0.685231in}{2.350000in}}%
\pgfpathlineto{\pgfqpoint{0.675500in}{2.350000in}}%
\pgfpathlineto{\pgfqpoint{0.665768in}{2.350000in}}%
\pgfpathlineto{\pgfqpoint{0.656037in}{2.350000in}}%
\pgfpathlineto{\pgfqpoint{0.646305in}{2.350000in}}%
\pgfpathlineto{\pgfqpoint{0.636574in}{2.350000in}}%
\pgfpathlineto{\pgfqpoint{0.626842in}{2.350000in}}%
\pgfpathlineto{\pgfqpoint{0.617111in}{2.350000in}}%
\pgfpathlineto{\pgfqpoint{0.607380in}{2.350000in}}%
\pgfpathlineto{\pgfqpoint{0.597648in}{2.350000in}}%
\pgfpathlineto{\pgfqpoint{0.587917in}{2.350000in}}%
\pgfpathlineto{\pgfqpoint{0.578185in}{2.350000in}}%
\pgfpathlineto{\pgfqpoint{0.568454in}{2.350000in}}%
\pgfpathlineto{\pgfqpoint{0.558722in}{2.350000in}}%
\pgfpathlineto{\pgfqpoint{0.548991in}{2.350000in}}%
\pgfpathlineto{\pgfqpoint{0.539259in}{2.350000in}}%
\pgfpathlineto{\pgfqpoint{0.529528in}{2.350000in}}%
\pgfpathlineto{\pgfqpoint{0.519796in}{2.350000in}}%
\pgfpathlineto{\pgfqpoint{0.510065in}{2.350000in}}%
\pgfpathlineto{\pgfqpoint{0.500333in}{2.350000in}}%
\pgfpathlineto{\pgfqpoint{0.490602in}{2.350000in}}%
\pgfpathlineto{\pgfqpoint{0.480870in}{2.350000in}}%
\pgfpathlineto{\pgfqpoint{0.471139in}{2.350000in}}%
\pgfpathlineto{\pgfqpoint{0.461407in}{2.350000in}}%
\pgfpathlineto{\pgfqpoint{0.451676in}{2.350000in}}%
\pgfpathlineto{\pgfqpoint{0.441944in}{2.350000in}}%
\pgfpathlineto{\pgfqpoint{0.432213in}{2.350000in}}%
\pgfpathlineto{\pgfqpoint{0.422481in}{2.350000in}}%
\pgfpathlineto{\pgfqpoint{0.412750in}{2.350000in}}%
\pgfpathlineto{\pgfqpoint{0.403018in}{2.350000in}}%
\pgfpathlineto{\pgfqpoint{0.393287in}{2.350000in}}%
\pgfpathlineto{\pgfqpoint{0.383556in}{2.350000in}}%
\pgfpathlineto{\pgfqpoint{0.373824in}{2.350000in}}%
\pgfpathlineto{\pgfqpoint{0.364093in}{2.350000in}}%
\pgfpathlineto{\pgfqpoint{0.361734in}{2.350000in}}%
\pgfpathlineto{\pgfqpoint{0.354361in}{2.346192in}}%
\pgfpathlineto{\pgfqpoint{0.347135in}{2.340998in}}%
\pgfpathlineto{\pgfqpoint{0.344630in}{2.338693in}}%
\pgfpathlineto{\pgfqpoint{0.339010in}{2.331997in}}%
\pgfpathlineto{\pgfqpoint{0.334898in}{2.325549in}}%
\pgfpathlineto{\pgfqpoint{0.333560in}{2.322995in}}%
\pgfpathlineto{\pgfqpoint{0.329953in}{2.313994in}}%
\pgfpathlineto{\pgfqpoint{0.326969in}{2.304992in}}%
\pgfpathlineto{\pgfqpoint{0.325167in}{2.299388in}}%
\pgfpathlineto{\pgfqpoint{0.324219in}{2.295991in}}%
\pgfpathlineto{\pgfqpoint{0.321321in}{2.286989in}}%
\pgfpathlineto{\pgfqpoint{0.317616in}{2.277988in}}%
\pgfpathlineto{\pgfqpoint{0.315435in}{2.273935in}}%
\pgfpathlineto{\pgfqpoint{0.313011in}{2.268986in}}%
\pgfpathlineto{\pgfqpoint{0.307371in}{2.259985in}}%
\pgfpathlineto{\pgfqpoint{0.305704in}{2.257822in}}%
\pgfpathlineto{\pgfqpoint{0.300686in}{2.250983in}}%
\pgfpathlineto{\pgfqpoint{0.295972in}{2.245526in}}%
\pgfpathlineto{\pgfqpoint{0.292931in}{2.241981in}}%
\pgfpathlineto{\pgfqpoint{0.286241in}{2.235062in}}%
\pgfpathlineto{\pgfqpoint{0.284156in}{2.232980in}}%
\pgfpathlineto{\pgfqpoint{0.276509in}{2.225907in}}%
\pgfpathlineto{\pgfqpoint{0.274258in}{2.223978in}}%
\pgfpathlineto{\pgfqpoint{0.266778in}{2.217790in}}%
\pgfpathlineto{\pgfqpoint{0.262946in}{2.214977in}}%
\pgfpathlineto{\pgfqpoint{0.257046in}{2.210616in}}%
\pgfpathlineto{\pgfqpoint{0.249653in}{2.205975in}}%
\pgfpathlineto{\pgfqpoint{0.247315in}{2.204433in}}%
\pgfpathlineto{\pgfqpoint{0.237583in}{2.199216in}}%
\pgfpathlineto{\pgfqpoint{0.232233in}{2.196974in}}%
\pgfpathlineto{\pgfqpoint{0.227852in}{2.194956in}}%
\pgfpathlineto{\pgfqpoint{0.218120in}{2.191529in}}%
\pgfpathlineto{\pgfqpoint{0.208389in}{2.188849in}}%
\pgfpathlineto{\pgfqpoint{0.204716in}{2.187972in}}%
\pgfpathlineto{\pgfqpoint{0.198657in}{2.186305in}}%
\pgfpathlineto{\pgfqpoint{0.188926in}{2.183545in}}%
\pgfpathlineto{\pgfqpoint{0.179194in}{2.180208in}}%
\pgfpathlineto{\pgfqpoint{0.176434in}{2.178971in}}%
\pgfpathlineto{\pgfqpoint{0.169463in}{2.175168in}}%
\pgfpathlineto{\pgfqpoint{0.162224in}{2.169969in}}%
\pgfpathlineto{\pgfqpoint{0.159731in}{2.167651in}}%
\pgfpathlineto{\pgfqpoint{0.154117in}{2.160968in}}%
\pgfpathlineto{\pgfqpoint{0.150000in}{2.154148in}}%
\pgfpathlineto{\pgfqpoint{0.150000in}{2.151966in}}%
\pgfpathlineto{\pgfqpoint{0.150000in}{2.142964in}}%
\pgfpathlineto{\pgfqpoint{0.150000in}{2.133963in}}%
\pgfpathlineto{\pgfqpoint{0.150000in}{2.124961in}}%
\pgfpathlineto{\pgfqpoint{0.150000in}{2.115960in}}%
\pgfpathlineto{\pgfqpoint{0.150000in}{2.106958in}}%
\pgfpathlineto{\pgfqpoint{0.150000in}{2.097957in}}%
\pgfpathlineto{\pgfqpoint{0.150000in}{2.088955in}}%
\pgfpathlineto{\pgfqpoint{0.150000in}{2.079954in}}%
\pgfpathlineto{\pgfqpoint{0.150000in}{2.070952in}}%
\pgfpathlineto{\pgfqpoint{0.150000in}{2.061951in}}%
\pgfpathlineto{\pgfqpoint{0.150000in}{2.052949in}}%
\pgfpathlineto{\pgfqpoint{0.150000in}{2.043947in}}%
\pgfpathlineto{\pgfqpoint{0.150000in}{2.034946in}}%
\pgfpathlineto{\pgfqpoint{0.150000in}{2.025944in}}%
\pgfpathlineto{\pgfqpoint{0.150000in}{2.016943in}}%
\pgfpathlineto{\pgfqpoint{0.150000in}{2.007941in}}%
\pgfpathlineto{\pgfqpoint{0.150000in}{1.998940in}}%
\pgfpathlineto{\pgfqpoint{0.150000in}{1.989938in}}%
\pgfpathlineto{\pgfqpoint{0.150000in}{1.980937in}}%
\pgfpathlineto{\pgfqpoint{0.150000in}{1.971935in}}%
\pgfpathlineto{\pgfqpoint{0.150000in}{1.962934in}}%
\pgfpathlineto{\pgfqpoint{0.150000in}{1.953932in}}%
\pgfpathlineto{\pgfqpoint{0.150000in}{1.944931in}}%
\pgfpathlineto{\pgfqpoint{0.150000in}{1.935929in}}%
\pgfpathlineto{\pgfqpoint{0.150000in}{1.926927in}}%
\pgfpathlineto{\pgfqpoint{0.150000in}{1.917926in}}%
\pgfpathlineto{\pgfqpoint{0.150000in}{1.908924in}}%
\pgfpathlineto{\pgfqpoint{0.150000in}{1.899923in}}%
\pgfpathlineto{\pgfqpoint{0.150000in}{1.890921in}}%
\pgfpathlineto{\pgfqpoint{0.150000in}{1.881920in}}%
\pgfpathlineto{\pgfqpoint{0.150000in}{1.872918in}}%
\pgfpathlineto{\pgfqpoint{0.150000in}{1.863917in}}%
\pgfpathlineto{\pgfqpoint{0.150000in}{1.854915in}}%
\pgfpathlineto{\pgfqpoint{0.150000in}{1.845914in}}%
\pgfpathlineto{\pgfqpoint{0.150000in}{1.836912in}}%
\pgfpathlineto{\pgfqpoint{0.150000in}{1.827910in}}%
\pgfpathlineto{\pgfqpoint{0.150000in}{1.818909in}}%
\pgfpathlineto{\pgfqpoint{0.150000in}{1.809907in}}%
\pgfpathlineto{\pgfqpoint{0.150000in}{1.800906in}}%
\pgfpathlineto{\pgfqpoint{0.150000in}{1.791904in}}%
\pgfpathlineto{\pgfqpoint{0.150000in}{1.782903in}}%
\pgfpathlineto{\pgfqpoint{0.150000in}{1.773901in}}%
\pgfpathlineto{\pgfqpoint{0.150000in}{1.764900in}}%
\pgfpathlineto{\pgfqpoint{0.150000in}{1.755898in}}%
\pgfpathlineto{\pgfqpoint{0.150000in}{1.746897in}}%
\pgfpathlineto{\pgfqpoint{0.150000in}{1.737895in}}%
\pgfpathlineto{\pgfqpoint{0.150000in}{1.728893in}}%
\pgfpathlineto{\pgfqpoint{0.150000in}{1.719892in}}%
\pgfpathlineto{\pgfqpoint{0.150000in}{1.710890in}}%
\pgfpathlineto{\pgfqpoint{0.150000in}{1.701889in}}%
\pgfpathlineto{\pgfqpoint{0.150000in}{1.696446in}}%
\pgfpathclose%
\pgfusepath{fill}%
\end{pgfscope}%
\begin{pgfscope}%
\pgfpathrectangle{\pgfqpoint{0.150000in}{0.549691in}}{\pgfqpoint{1.946296in}{1.800309in}}%
\pgfusepath{clip}%
\pgfsetbuttcap%
\pgfsetroundjoin%
\definecolor{currentfill}{rgb}{0.309804,0.309804,0.309804}%
\pgfsetfillcolor{currentfill}%
\pgfsetlinewidth{0.000000pt}%
\definecolor{currentstroke}{rgb}{0.000000,0.000000,0.000000}%
\pgfsetstrokecolor{currentstroke}%
\pgfsetdash{}{0pt}%
\pgfpathmoveto{\pgfqpoint{2.096296in}{1.696446in}}%
\pgfpathlineto{\pgfqpoint{2.096296in}{1.701889in}}%
\pgfpathlineto{\pgfqpoint{2.096296in}{1.710890in}}%
\pgfpathlineto{\pgfqpoint{2.096296in}{1.719892in}}%
\pgfpathlineto{\pgfqpoint{2.096296in}{1.728893in}}%
\pgfpathlineto{\pgfqpoint{2.096296in}{1.737895in}}%
\pgfpathlineto{\pgfqpoint{2.096296in}{1.746897in}}%
\pgfpathlineto{\pgfqpoint{2.096296in}{1.755898in}}%
\pgfpathlineto{\pgfqpoint{2.096296in}{1.764900in}}%
\pgfpathlineto{\pgfqpoint{2.096296in}{1.773901in}}%
\pgfpathlineto{\pgfqpoint{2.096296in}{1.782903in}}%
\pgfpathlineto{\pgfqpoint{2.096296in}{1.791904in}}%
\pgfpathlineto{\pgfqpoint{2.096296in}{1.800906in}}%
\pgfpathlineto{\pgfqpoint{2.096296in}{1.809907in}}%
\pgfpathlineto{\pgfqpoint{2.096296in}{1.818909in}}%
\pgfpathlineto{\pgfqpoint{2.096296in}{1.827910in}}%
\pgfpathlineto{\pgfqpoint{2.096296in}{1.836912in}}%
\pgfpathlineto{\pgfqpoint{2.096296in}{1.845914in}}%
\pgfpathlineto{\pgfqpoint{2.096296in}{1.854915in}}%
\pgfpathlineto{\pgfqpoint{2.096296in}{1.863917in}}%
\pgfpathlineto{\pgfqpoint{2.096296in}{1.872918in}}%
\pgfpathlineto{\pgfqpoint{2.096296in}{1.881920in}}%
\pgfpathlineto{\pgfqpoint{2.096296in}{1.890921in}}%
\pgfpathlineto{\pgfqpoint{2.096296in}{1.899923in}}%
\pgfpathlineto{\pgfqpoint{2.096296in}{1.908924in}}%
\pgfpathlineto{\pgfqpoint{2.096296in}{1.917926in}}%
\pgfpathlineto{\pgfqpoint{2.096296in}{1.926927in}}%
\pgfpathlineto{\pgfqpoint{2.096296in}{1.935929in}}%
\pgfpathlineto{\pgfqpoint{2.096296in}{1.944931in}}%
\pgfpathlineto{\pgfqpoint{2.096296in}{1.953932in}}%
\pgfpathlineto{\pgfqpoint{2.096296in}{1.962934in}}%
\pgfpathlineto{\pgfqpoint{2.096296in}{1.971935in}}%
\pgfpathlineto{\pgfqpoint{2.096296in}{1.980937in}}%
\pgfpathlineto{\pgfqpoint{2.096296in}{1.989938in}}%
\pgfpathlineto{\pgfqpoint{2.096296in}{1.998940in}}%
\pgfpathlineto{\pgfqpoint{2.096296in}{2.007941in}}%
\pgfpathlineto{\pgfqpoint{2.096296in}{2.016943in}}%
\pgfpathlineto{\pgfqpoint{2.096296in}{2.025944in}}%
\pgfpathlineto{\pgfqpoint{2.096296in}{2.034946in}}%
\pgfpathlineto{\pgfqpoint{2.096296in}{2.043947in}}%
\pgfpathlineto{\pgfqpoint{2.096296in}{2.052949in}}%
\pgfpathlineto{\pgfqpoint{2.096296in}{2.061951in}}%
\pgfpathlineto{\pgfqpoint{2.096296in}{2.070952in}}%
\pgfpathlineto{\pgfqpoint{2.096296in}{2.079954in}}%
\pgfpathlineto{\pgfqpoint{2.096296in}{2.088955in}}%
\pgfpathlineto{\pgfqpoint{2.096296in}{2.097957in}}%
\pgfpathlineto{\pgfqpoint{2.096296in}{2.106958in}}%
\pgfpathlineto{\pgfqpoint{2.096296in}{2.115960in}}%
\pgfpathlineto{\pgfqpoint{2.096296in}{2.124961in}}%
\pgfpathlineto{\pgfqpoint{2.096296in}{2.133963in}}%
\pgfpathlineto{\pgfqpoint{2.096296in}{2.142964in}}%
\pgfpathlineto{\pgfqpoint{2.096296in}{2.151966in}}%
\pgfpathlineto{\pgfqpoint{2.096296in}{2.154148in}}%
\pgfpathlineto{\pgfqpoint{2.092179in}{2.160968in}}%
\pgfpathlineto{\pgfqpoint{2.086564in}{2.167651in}}%
\pgfpathlineto{\pgfqpoint{2.084072in}{2.169969in}}%
\pgfpathlineto{\pgfqpoint{2.076833in}{2.175168in}}%
\pgfpathlineto{\pgfqpoint{2.069862in}{2.178971in}}%
\pgfpathlineto{\pgfqpoint{2.067101in}{2.180208in}}%
\pgfpathlineto{\pgfqpoint{2.057370in}{2.183545in}}%
\pgfpathlineto{\pgfqpoint{2.047638in}{2.186305in}}%
\pgfpathlineto{\pgfqpoint{2.041579in}{2.187972in}}%
\pgfpathlineto{\pgfqpoint{2.037907in}{2.188849in}}%
\pgfpathlineto{\pgfqpoint{2.028175in}{2.191529in}}%
\pgfpathlineto{\pgfqpoint{2.018444in}{2.194956in}}%
\pgfpathlineto{\pgfqpoint{2.014062in}{2.196974in}}%
\pgfpathlineto{\pgfqpoint{2.008713in}{2.199216in}}%
\pgfpathlineto{\pgfqpoint{1.998981in}{2.204433in}}%
\pgfpathlineto{\pgfqpoint{1.996643in}{2.205975in}}%
\pgfpathlineto{\pgfqpoint{1.989250in}{2.210616in}}%
\pgfpathlineto{\pgfqpoint{1.983350in}{2.214977in}}%
\pgfpathlineto{\pgfqpoint{1.979518in}{2.217790in}}%
\pgfpathlineto{\pgfqpoint{1.972038in}{2.223978in}}%
\pgfpathlineto{\pgfqpoint{1.969787in}{2.225907in}}%
\pgfpathlineto{\pgfqpoint{1.962140in}{2.232980in}}%
\pgfpathlineto{\pgfqpoint{1.960055in}{2.235062in}}%
\pgfpathlineto{\pgfqpoint{1.953365in}{2.241981in}}%
\pgfpathlineto{\pgfqpoint{1.950324in}{2.245526in}}%
\pgfpathlineto{\pgfqpoint{1.945610in}{2.250983in}}%
\pgfpathlineto{\pgfqpoint{1.940592in}{2.257822in}}%
\pgfpathlineto{\pgfqpoint{1.938925in}{2.259985in}}%
\pgfpathlineto{\pgfqpoint{1.933285in}{2.268986in}}%
\pgfpathlineto{\pgfqpoint{1.930861in}{2.273935in}}%
\pgfpathlineto{\pgfqpoint{1.928680in}{2.277988in}}%
\pgfpathlineto{\pgfqpoint{1.924975in}{2.286989in}}%
\pgfpathlineto{\pgfqpoint{1.922077in}{2.295991in}}%
\pgfpathlineto{\pgfqpoint{1.921129in}{2.299388in}}%
\pgfpathlineto{\pgfqpoint{1.919327in}{2.304992in}}%
\pgfpathlineto{\pgfqpoint{1.916343in}{2.313994in}}%
\pgfpathlineto{\pgfqpoint{1.912735in}{2.322995in}}%
\pgfpathlineto{\pgfqpoint{1.911398in}{2.325549in}}%
\pgfpathlineto{\pgfqpoint{1.907286in}{2.331997in}}%
\pgfpathlineto{\pgfqpoint{1.901666in}{2.338693in}}%
\pgfpathlineto{\pgfqpoint{1.899161in}{2.340998in}}%
\pgfpathlineto{\pgfqpoint{1.891935in}{2.346192in}}%
\pgfpathlineto{\pgfqpoint{1.884562in}{2.350000in}}%
\pgfpathlineto{\pgfqpoint{1.882203in}{2.350000in}}%
\pgfpathlineto{\pgfqpoint{1.872472in}{2.350000in}}%
\pgfpathlineto{\pgfqpoint{1.862740in}{2.350000in}}%
\pgfpathlineto{\pgfqpoint{1.853009in}{2.350000in}}%
\pgfpathlineto{\pgfqpoint{1.843277in}{2.350000in}}%
\pgfpathlineto{\pgfqpoint{1.833546in}{2.350000in}}%
\pgfpathlineto{\pgfqpoint{1.823814in}{2.350000in}}%
\pgfpathlineto{\pgfqpoint{1.814083in}{2.350000in}}%
\pgfpathlineto{\pgfqpoint{1.804351in}{2.350000in}}%
\pgfpathlineto{\pgfqpoint{1.794620in}{2.350000in}}%
\pgfpathlineto{\pgfqpoint{1.784889in}{2.350000in}}%
\pgfpathlineto{\pgfqpoint{1.775157in}{2.350000in}}%
\pgfpathlineto{\pgfqpoint{1.765426in}{2.350000in}}%
\pgfpathlineto{\pgfqpoint{1.755694in}{2.350000in}}%
\pgfpathlineto{\pgfqpoint{1.745963in}{2.350000in}}%
\pgfpathlineto{\pgfqpoint{1.736231in}{2.350000in}}%
\pgfpathlineto{\pgfqpoint{1.726500in}{2.350000in}}%
\pgfpathlineto{\pgfqpoint{1.716768in}{2.350000in}}%
\pgfpathlineto{\pgfqpoint{1.707037in}{2.350000in}}%
\pgfpathlineto{\pgfqpoint{1.697305in}{2.350000in}}%
\pgfpathlineto{\pgfqpoint{1.687574in}{2.350000in}}%
\pgfpathlineto{\pgfqpoint{1.677842in}{2.350000in}}%
\pgfpathlineto{\pgfqpoint{1.668111in}{2.350000in}}%
\pgfpathlineto{\pgfqpoint{1.658379in}{2.350000in}}%
\pgfpathlineto{\pgfqpoint{1.648648in}{2.350000in}}%
\pgfpathlineto{\pgfqpoint{1.638916in}{2.350000in}}%
\pgfpathlineto{\pgfqpoint{1.629185in}{2.350000in}}%
\pgfpathlineto{\pgfqpoint{1.619453in}{2.350000in}}%
\pgfpathlineto{\pgfqpoint{1.609722in}{2.350000in}}%
\pgfpathlineto{\pgfqpoint{1.599990in}{2.350000in}}%
\pgfpathlineto{\pgfqpoint{1.590259in}{2.350000in}}%
\pgfpathlineto{\pgfqpoint{1.580527in}{2.350000in}}%
\pgfpathlineto{\pgfqpoint{1.570796in}{2.350000in}}%
\pgfpathlineto{\pgfqpoint{1.561064in}{2.350000in}}%
\pgfpathlineto{\pgfqpoint{1.551333in}{2.350000in}}%
\pgfpathlineto{\pgfqpoint{1.541602in}{2.350000in}}%
\pgfpathlineto{\pgfqpoint{1.531870in}{2.350000in}}%
\pgfpathlineto{\pgfqpoint{1.522139in}{2.350000in}}%
\pgfpathlineto{\pgfqpoint{1.512407in}{2.350000in}}%
\pgfpathlineto{\pgfqpoint{1.502676in}{2.350000in}}%
\pgfpathlineto{\pgfqpoint{1.492944in}{2.350000in}}%
\pgfpathlineto{\pgfqpoint{1.483213in}{2.350000in}}%
\pgfpathlineto{\pgfqpoint{1.473481in}{2.350000in}}%
\pgfpathlineto{\pgfqpoint{1.463750in}{2.350000in}}%
\pgfpathlineto{\pgfqpoint{1.454018in}{2.350000in}}%
\pgfpathlineto{\pgfqpoint{1.444287in}{2.350000in}}%
\pgfpathlineto{\pgfqpoint{1.434555in}{2.350000in}}%
\pgfpathlineto{\pgfqpoint{1.424824in}{2.350000in}}%
\pgfpathlineto{\pgfqpoint{1.415092in}{2.350000in}}%
\pgfpathlineto{\pgfqpoint{1.405361in}{2.350000in}}%
\pgfpathlineto{\pgfqpoint{1.395629in}{2.350000in}}%
\pgfpathlineto{\pgfqpoint{1.389745in}{2.350000in}}%
\pgfpathlineto{\pgfqpoint{1.395629in}{2.345647in}}%
\pgfpathlineto{\pgfqpoint{1.403551in}{2.340998in}}%
\pgfpathlineto{\pgfqpoint{1.405361in}{2.339640in}}%
\pgfpathlineto{\pgfqpoint{1.415092in}{2.334594in}}%
\pgfpathlineto{\pgfqpoint{1.424594in}{2.331997in}}%
\pgfpathlineto{\pgfqpoint{1.424824in}{2.331914in}}%
\pgfpathlineto{\pgfqpoint{1.434555in}{2.331392in}}%
\pgfpathlineto{\pgfqpoint{1.437522in}{2.331997in}}%
\pgfpathlineto{\pgfqpoint{1.444287in}{2.333045in}}%
\pgfpathlineto{\pgfqpoint{1.454018in}{2.335903in}}%
\pgfpathlineto{\pgfqpoint{1.463750in}{2.339410in}}%
\pgfpathlineto{\pgfqpoint{1.468326in}{2.340998in}}%
\pgfpathlineto{\pgfqpoint{1.473481in}{2.342396in}}%
\pgfpathlineto{\pgfqpoint{1.483213in}{2.344328in}}%
\pgfpathlineto{\pgfqpoint{1.492944in}{2.345005in}}%
\pgfpathlineto{\pgfqpoint{1.502676in}{2.343993in}}%
\pgfpathlineto{\pgfqpoint{1.512407in}{2.341011in}}%
\pgfpathlineto{\pgfqpoint{1.512432in}{2.340998in}}%
\pgfpathlineto{\pgfqpoint{1.522139in}{2.334557in}}%
\pgfpathlineto{\pgfqpoint{1.524913in}{2.331997in}}%
\pgfpathlineto{\pgfqpoint{1.531870in}{2.323550in}}%
\pgfpathlineto{\pgfqpoint{1.532238in}{2.322995in}}%
\pgfpathlineto{\pgfqpoint{1.536809in}{2.313994in}}%
\pgfpathlineto{\pgfqpoint{1.540589in}{2.304992in}}%
\pgfpathlineto{\pgfqpoint{1.541602in}{2.302508in}}%
\pgfpathlineto{\pgfqpoint{1.543922in}{2.295991in}}%
\pgfpathlineto{\pgfqpoint{1.547623in}{2.286989in}}%
\pgfpathlineto{\pgfqpoint{1.551333in}{2.279931in}}%
\pgfpathlineto{\pgfqpoint{1.552287in}{2.277988in}}%
\pgfpathlineto{\pgfqpoint{1.558062in}{2.268986in}}%
\pgfpathlineto{\pgfqpoint{1.561064in}{2.265329in}}%
\pgfpathlineto{\pgfqpoint{1.565410in}{2.259985in}}%
\pgfpathlineto{\pgfqpoint{1.570796in}{2.254601in}}%
\pgfpathlineto{\pgfqpoint{1.574594in}{2.250983in}}%
\pgfpathlineto{\pgfqpoint{1.580527in}{2.246182in}}%
\pgfpathlineto{\pgfqpoint{1.586314in}{2.241981in}}%
\pgfpathlineto{\pgfqpoint{1.590259in}{2.239441in}}%
\pgfpathlineto{\pgfqpoint{1.599990in}{2.234186in}}%
\pgfpathlineto{\pgfqpoint{1.602887in}{2.232980in}}%
\pgfpathlineto{\pgfqpoint{1.609722in}{2.230345in}}%
\pgfpathlineto{\pgfqpoint{1.619453in}{2.227769in}}%
\pgfpathlineto{\pgfqpoint{1.629185in}{2.226308in}}%
\pgfpathlineto{\pgfqpoint{1.638916in}{2.225785in}}%
\pgfpathlineto{\pgfqpoint{1.648648in}{2.225928in}}%
\pgfpathlineto{\pgfqpoint{1.658379in}{2.226390in}}%
\pgfpathlineto{\pgfqpoint{1.668111in}{2.226789in}}%
\pgfpathlineto{\pgfqpoint{1.677842in}{2.226742in}}%
\pgfpathlineto{\pgfqpoint{1.687574in}{2.225903in}}%
\pgfpathlineto{\pgfqpoint{1.697305in}{2.223999in}}%
\pgfpathlineto{\pgfqpoint{1.697369in}{2.223978in}}%
\pgfpathlineto{\pgfqpoint{1.707037in}{2.220961in}}%
\pgfpathlineto{\pgfqpoint{1.716768in}{2.216665in}}%
\pgfpathlineto{\pgfqpoint{1.719759in}{2.214977in}}%
\pgfpathlineto{\pgfqpoint{1.726500in}{2.211145in}}%
\pgfpathlineto{\pgfqpoint{1.734160in}{2.205975in}}%
\pgfpathlineto{\pgfqpoint{1.736231in}{2.204507in}}%
\pgfpathlineto{\pgfqpoint{1.745741in}{2.196974in}}%
\pgfpathlineto{\pgfqpoint{1.745963in}{2.196780in}}%
\pgfpathlineto{\pgfqpoint{1.755519in}{2.187972in}}%
\pgfpathlineto{\pgfqpoint{1.755694in}{2.187786in}}%
\pgfpathlineto{\pgfqpoint{1.763922in}{2.178971in}}%
\pgfpathlineto{\pgfqpoint{1.765426in}{2.177009in}}%
\pgfpathlineto{\pgfqpoint{1.771047in}{2.169969in}}%
\pgfpathlineto{\pgfqpoint{1.775157in}{2.163301in}}%
\pgfpathlineto{\pgfqpoint{1.776731in}{2.160968in}}%
\pgfpathlineto{\pgfqpoint{1.781093in}{2.151966in}}%
\pgfpathlineto{\pgfqpoint{1.783989in}{2.142964in}}%
\pgfpathlineto{\pgfqpoint{1.784889in}{2.138483in}}%
\pgfpathlineto{\pgfqpoint{1.785935in}{2.133963in}}%
\pgfpathlineto{\pgfqpoint{1.787310in}{2.124961in}}%
\pgfpathlineto{\pgfqpoint{1.788585in}{2.115960in}}%
\pgfpathlineto{\pgfqpoint{1.790367in}{2.106958in}}%
\pgfpathlineto{\pgfqpoint{1.793204in}{2.097957in}}%
\pgfpathlineto{\pgfqpoint{1.794620in}{2.095012in}}%
\pgfpathlineto{\pgfqpoint{1.798179in}{2.088955in}}%
\pgfpathlineto{\pgfqpoint{1.804351in}{2.081498in}}%
\pgfpathlineto{\pgfqpoint{1.806021in}{2.079954in}}%
\pgfpathlineto{\pgfqpoint{1.814083in}{2.074244in}}%
\pgfpathlineto{\pgfqpoint{1.820631in}{2.070952in}}%
\pgfpathlineto{\pgfqpoint{1.823814in}{2.069643in}}%
\pgfpathlineto{\pgfqpoint{1.833546in}{2.067018in}}%
\pgfpathlineto{\pgfqpoint{1.843277in}{2.065370in}}%
\pgfpathlineto{\pgfqpoint{1.853009in}{2.064191in}}%
\pgfpathlineto{\pgfqpoint{1.862740in}{2.062918in}}%
\pgfpathlineto{\pgfqpoint{1.867627in}{2.061951in}}%
\pgfpathlineto{\pgfqpoint{1.872472in}{2.061119in}}%
\pgfpathlineto{\pgfqpoint{1.882203in}{2.058440in}}%
\pgfpathlineto{\pgfqpoint{1.891935in}{2.054405in}}%
\pgfpathlineto{\pgfqpoint{1.894458in}{2.052949in}}%
\pgfpathlineto{\pgfqpoint{1.901666in}{2.049147in}}%
\pgfpathlineto{\pgfqpoint{1.909277in}{2.043947in}}%
\pgfpathlineto{\pgfqpoint{1.911398in}{2.042556in}}%
\pgfpathlineto{\pgfqpoint{1.920927in}{2.034946in}}%
\pgfpathlineto{\pgfqpoint{1.921129in}{2.034784in}}%
\pgfpathlineto{\pgfqpoint{1.930652in}{2.025944in}}%
\pgfpathlineto{\pgfqpoint{1.930861in}{2.025739in}}%
\pgfpathlineto{\pgfqpoint{1.939004in}{2.016943in}}%
\pgfpathlineto{\pgfqpoint{1.940592in}{2.015027in}}%
\pgfpathlineto{\pgfqpoint{1.946181in}{2.007941in}}%
\pgfpathlineto{\pgfqpoint{1.950324in}{2.001706in}}%
\pgfpathlineto{\pgfqpoint{1.952149in}{1.998940in}}%
\pgfpathlineto{\pgfqpoint{1.956793in}{1.989938in}}%
\pgfpathlineto{\pgfqpoint{1.960055in}{1.980996in}}%
\pgfpathlineto{\pgfqpoint{1.960077in}{1.980937in}}%
\pgfpathlineto{\pgfqpoint{1.962136in}{1.971935in}}%
\pgfpathlineto{\pgfqpoint{1.963043in}{1.962934in}}%
\pgfpathlineto{\pgfqpoint{1.963094in}{1.953932in}}%
\pgfpathlineto{\pgfqpoint{1.962663in}{1.944931in}}%
\pgfpathlineto{\pgfqpoint{1.962163in}{1.935929in}}%
\pgfpathlineto{\pgfqpoint{1.962009in}{1.926927in}}%
\pgfpathlineto{\pgfqpoint{1.962573in}{1.917926in}}%
\pgfpathlineto{\pgfqpoint{1.964153in}{1.908924in}}%
\pgfpathlineto{\pgfqpoint{1.966938in}{1.899923in}}%
\pgfpathlineto{\pgfqpoint{1.969787in}{1.893601in}}%
\pgfpathlineto{\pgfqpoint{1.971090in}{1.890921in}}%
\pgfpathlineto{\pgfqpoint{1.976771in}{1.881920in}}%
\pgfpathlineto{\pgfqpoint{1.979518in}{1.878271in}}%
\pgfpathlineto{\pgfqpoint{1.984060in}{1.872918in}}%
\pgfpathlineto{\pgfqpoint{1.989250in}{1.867429in}}%
\pgfpathlineto{\pgfqpoint{1.993161in}{1.863917in}}%
\pgfpathlineto{\pgfqpoint{1.998981in}{1.858935in}}%
\pgfpathlineto{\pgfqpoint{2.004759in}{1.854915in}}%
\pgfpathlineto{\pgfqpoint{2.008713in}{1.852138in}}%
\pgfpathlineto{\pgfqpoint{2.018444in}{1.846796in}}%
\pgfpathlineto{\pgfqpoint{2.020545in}{1.845914in}}%
\pgfpathlineto{\pgfqpoint{2.028175in}{1.842482in}}%
\pgfpathlineto{\pgfqpoint{2.037907in}{1.839059in}}%
\pgfpathlineto{\pgfqpoint{2.044952in}{1.836912in}}%
\pgfpathlineto{\pgfqpoint{2.047638in}{1.835976in}}%
\pgfpathlineto{\pgfqpoint{2.057370in}{1.832479in}}%
\pgfpathlineto{\pgfqpoint{2.067101in}{1.828251in}}%
\pgfpathlineto{\pgfqpoint{2.067701in}{1.827910in}}%
\pgfpathlineto{\pgfqpoint{2.076833in}{1.821476in}}%
\pgfpathlineto{\pgfqpoint{2.079600in}{1.818909in}}%
\pgfpathlineto{\pgfqpoint{2.086564in}{1.809931in}}%
\pgfpathlineto{\pgfqpoint{2.086578in}{1.809907in}}%
\pgfpathlineto{\pgfqpoint{2.089801in}{1.800906in}}%
\pgfpathlineto{\pgfqpoint{2.090896in}{1.791904in}}%
\pgfpathlineto{\pgfqpoint{2.090164in}{1.782903in}}%
\pgfpathlineto{\pgfqpoint{2.088076in}{1.773901in}}%
\pgfpathlineto{\pgfqpoint{2.086564in}{1.769133in}}%
\pgfpathlineto{\pgfqpoint{2.084847in}{1.764900in}}%
\pgfpathlineto{\pgfqpoint{2.081056in}{1.755898in}}%
\pgfpathlineto{\pgfqpoint{2.077966in}{1.746897in}}%
\pgfpathlineto{\pgfqpoint{2.076833in}{1.740640in}}%
\pgfpathlineto{\pgfqpoint{2.076178in}{1.737895in}}%
\pgfpathlineto{\pgfqpoint{2.076743in}{1.728893in}}%
\pgfpathlineto{\pgfqpoint{2.076833in}{1.728681in}}%
\pgfpathlineto{\pgfqpoint{2.079640in}{1.719892in}}%
\pgfpathlineto{\pgfqpoint{2.085095in}{1.710890in}}%
\pgfpathlineto{\pgfqpoint{2.086564in}{1.709217in}}%
\pgfpathlineto{\pgfqpoint{2.091590in}{1.701889in}}%
\pgfpathclose%
\pgfusepath{fill}%
\end{pgfscope}%
\begin{pgfscope}%
\pgfpathrectangle{\pgfqpoint{0.150000in}{0.549691in}}{\pgfqpoint{1.946296in}{1.800309in}}%
\pgfusepath{clip}%
\pgfsetbuttcap%
\pgfsetroundjoin%
\definecolor{currentfill}{rgb}{0.337255,0.337255,0.337255}%
\pgfsetfillcolor{currentfill}%
\pgfsetlinewidth{0.000000pt}%
\definecolor{currentstroke}{rgb}{0.000000,0.000000,0.000000}%
\pgfsetstrokecolor{currentstroke}%
\pgfsetdash{}{0pt}%
\pgfpathmoveto{\pgfqpoint{0.159731in}{0.549691in}}%
\pgfpathlineto{\pgfqpoint{0.169463in}{0.549691in}}%
\pgfpathlineto{\pgfqpoint{0.179194in}{0.549691in}}%
\pgfpathlineto{\pgfqpoint{0.188926in}{0.549691in}}%
\pgfpathlineto{\pgfqpoint{0.198657in}{0.549691in}}%
\pgfpathlineto{\pgfqpoint{0.208389in}{0.549691in}}%
\pgfpathlineto{\pgfqpoint{0.218120in}{0.549691in}}%
\pgfpathlineto{\pgfqpoint{0.227852in}{0.549691in}}%
\pgfpathlineto{\pgfqpoint{0.237583in}{0.549691in}}%
\pgfpathlineto{\pgfqpoint{0.247315in}{0.549691in}}%
\pgfpathlineto{\pgfqpoint{0.257046in}{0.549691in}}%
\pgfpathlineto{\pgfqpoint{0.266778in}{0.549691in}}%
\pgfpathlineto{\pgfqpoint{0.276509in}{0.549691in}}%
\pgfpathlineto{\pgfqpoint{0.286241in}{0.549691in}}%
\pgfpathlineto{\pgfqpoint{0.295972in}{0.549691in}}%
\pgfpathlineto{\pgfqpoint{0.305704in}{0.549691in}}%
\pgfpathlineto{\pgfqpoint{0.315435in}{0.549691in}}%
\pgfpathlineto{\pgfqpoint{0.325167in}{0.549691in}}%
\pgfpathlineto{\pgfqpoint{0.334898in}{0.549691in}}%
\pgfpathlineto{\pgfqpoint{0.344630in}{0.549691in}}%
\pgfpathlineto{\pgfqpoint{0.354361in}{0.549691in}}%
\pgfpathlineto{\pgfqpoint{0.361734in}{0.549691in}}%
\pgfpathlineto{\pgfqpoint{0.354361in}{0.553499in}}%
\pgfpathlineto{\pgfqpoint{0.347135in}{0.558693in}}%
\pgfpathlineto{\pgfqpoint{0.344630in}{0.560998in}}%
\pgfpathlineto{\pgfqpoint{0.339010in}{0.567694in}}%
\pgfpathlineto{\pgfqpoint{0.334898in}{0.574142in}}%
\pgfpathlineto{\pgfqpoint{0.333560in}{0.576696in}}%
\pgfpathlineto{\pgfqpoint{0.329953in}{0.585697in}}%
\pgfpathlineto{\pgfqpoint{0.326969in}{0.594699in}}%
\pgfpathlineto{\pgfqpoint{0.325167in}{0.600303in}}%
\pgfpathlineto{\pgfqpoint{0.324219in}{0.603700in}}%
\pgfpathlineto{\pgfqpoint{0.321321in}{0.612702in}}%
\pgfpathlineto{\pgfqpoint{0.317616in}{0.621703in}}%
\pgfpathlineto{\pgfqpoint{0.315435in}{0.625756in}}%
\pgfpathlineto{\pgfqpoint{0.313011in}{0.630705in}}%
\pgfpathlineto{\pgfqpoint{0.307371in}{0.639707in}}%
\pgfpathlineto{\pgfqpoint{0.305704in}{0.641869in}}%
\pgfpathlineto{\pgfqpoint{0.300686in}{0.648708in}}%
\pgfpathlineto{\pgfqpoint{0.295972in}{0.654165in}}%
\pgfpathlineto{\pgfqpoint{0.292931in}{0.657710in}}%
\pgfpathlineto{\pgfqpoint{0.286241in}{0.664629in}}%
\pgfpathlineto{\pgfqpoint{0.284156in}{0.666711in}}%
\pgfpathlineto{\pgfqpoint{0.276509in}{0.673784in}}%
\pgfpathlineto{\pgfqpoint{0.274258in}{0.675713in}}%
\pgfpathlineto{\pgfqpoint{0.266778in}{0.681901in}}%
\pgfpathlineto{\pgfqpoint{0.262946in}{0.684714in}}%
\pgfpathlineto{\pgfqpoint{0.257046in}{0.689075in}}%
\pgfpathlineto{\pgfqpoint{0.249653in}{0.693716in}}%
\pgfpathlineto{\pgfqpoint{0.247315in}{0.695258in}}%
\pgfpathlineto{\pgfqpoint{0.237583in}{0.700475in}}%
\pgfpathlineto{\pgfqpoint{0.232233in}{0.702717in}}%
\pgfpathlineto{\pgfqpoint{0.227852in}{0.704735in}}%
\pgfpathlineto{\pgfqpoint{0.218120in}{0.708162in}}%
\pgfpathlineto{\pgfqpoint{0.208389in}{0.710843in}}%
\pgfpathlineto{\pgfqpoint{0.204716in}{0.711719in}}%
\pgfpathlineto{\pgfqpoint{0.198657in}{0.713386in}}%
\pgfpathlineto{\pgfqpoint{0.188926in}{0.716146in}}%
\pgfpathlineto{\pgfqpoint{0.179194in}{0.719483in}}%
\pgfpathlineto{\pgfqpoint{0.176434in}{0.720720in}}%
\pgfpathlineto{\pgfqpoint{0.169463in}{0.724524in}}%
\pgfpathlineto{\pgfqpoint{0.162224in}{0.729722in}}%
\pgfpathlineto{\pgfqpoint{0.159731in}{0.732040in}}%
\pgfpathlineto{\pgfqpoint{0.154117in}{0.738724in}}%
\pgfpathlineto{\pgfqpoint{0.150000in}{0.745543in}}%
\pgfpathlineto{\pgfqpoint{0.150000in}{0.738724in}}%
\pgfpathlineto{\pgfqpoint{0.150000in}{0.729722in}}%
\pgfpathlineto{\pgfqpoint{0.150000in}{0.720720in}}%
\pgfpathlineto{\pgfqpoint{0.150000in}{0.711719in}}%
\pgfpathlineto{\pgfqpoint{0.150000in}{0.702717in}}%
\pgfpathlineto{\pgfqpoint{0.150000in}{0.693716in}}%
\pgfpathlineto{\pgfqpoint{0.150000in}{0.684714in}}%
\pgfpathlineto{\pgfqpoint{0.150000in}{0.675713in}}%
\pgfpathlineto{\pgfqpoint{0.150000in}{0.666711in}}%
\pgfpathlineto{\pgfqpoint{0.150000in}{0.657710in}}%
\pgfpathlineto{\pgfqpoint{0.150000in}{0.648708in}}%
\pgfpathlineto{\pgfqpoint{0.150000in}{0.639707in}}%
\pgfpathlineto{\pgfqpoint{0.150000in}{0.630705in}}%
\pgfpathlineto{\pgfqpoint{0.150000in}{0.621703in}}%
\pgfpathlineto{\pgfqpoint{0.150000in}{0.612702in}}%
\pgfpathlineto{\pgfqpoint{0.150000in}{0.603700in}}%
\pgfpathlineto{\pgfqpoint{0.150000in}{0.594699in}}%
\pgfpathlineto{\pgfqpoint{0.150000in}{0.585697in}}%
\pgfpathlineto{\pgfqpoint{0.150000in}{0.576696in}}%
\pgfpathlineto{\pgfqpoint{0.150000in}{0.567694in}}%
\pgfpathlineto{\pgfqpoint{0.150000in}{0.558693in}}%
\pgfpathlineto{\pgfqpoint{0.150000in}{0.549691in}}%
\pgfpathclose%
\pgfusepath{fill}%
\end{pgfscope}%
\begin{pgfscope}%
\pgfpathrectangle{\pgfqpoint{0.150000in}{0.549691in}}{\pgfqpoint{1.946296in}{1.800309in}}%
\pgfusepath{clip}%
\pgfsetbuttcap%
\pgfsetroundjoin%
\definecolor{currentfill}{rgb}{0.337255,0.337255,0.337255}%
\pgfsetfillcolor{currentfill}%
\pgfsetlinewidth{0.000000pt}%
\definecolor{currentstroke}{rgb}{0.000000,0.000000,0.000000}%
\pgfsetstrokecolor{currentstroke}%
\pgfsetdash{}{0pt}%
\pgfpathmoveto{\pgfqpoint{1.891935in}{0.549691in}}%
\pgfpathlineto{\pgfqpoint{1.901666in}{0.549691in}}%
\pgfpathlineto{\pgfqpoint{1.911398in}{0.549691in}}%
\pgfpathlineto{\pgfqpoint{1.921129in}{0.549691in}}%
\pgfpathlineto{\pgfqpoint{1.930861in}{0.549691in}}%
\pgfpathlineto{\pgfqpoint{1.940592in}{0.549691in}}%
\pgfpathlineto{\pgfqpoint{1.950324in}{0.549691in}}%
\pgfpathlineto{\pgfqpoint{1.960055in}{0.549691in}}%
\pgfpathlineto{\pgfqpoint{1.969787in}{0.549691in}}%
\pgfpathlineto{\pgfqpoint{1.979518in}{0.549691in}}%
\pgfpathlineto{\pgfqpoint{1.989250in}{0.549691in}}%
\pgfpathlineto{\pgfqpoint{1.998981in}{0.549691in}}%
\pgfpathlineto{\pgfqpoint{2.008713in}{0.549691in}}%
\pgfpathlineto{\pgfqpoint{2.018444in}{0.549691in}}%
\pgfpathlineto{\pgfqpoint{2.028175in}{0.549691in}}%
\pgfpathlineto{\pgfqpoint{2.037907in}{0.549691in}}%
\pgfpathlineto{\pgfqpoint{2.047638in}{0.549691in}}%
\pgfpathlineto{\pgfqpoint{2.057370in}{0.549691in}}%
\pgfpathlineto{\pgfqpoint{2.067101in}{0.549691in}}%
\pgfpathlineto{\pgfqpoint{2.076833in}{0.549691in}}%
\pgfpathlineto{\pgfqpoint{2.086564in}{0.549691in}}%
\pgfpathlineto{\pgfqpoint{2.096296in}{0.549691in}}%
\pgfpathlineto{\pgfqpoint{2.096296in}{0.558693in}}%
\pgfpathlineto{\pgfqpoint{2.096296in}{0.567694in}}%
\pgfpathlineto{\pgfqpoint{2.096296in}{0.576696in}}%
\pgfpathlineto{\pgfqpoint{2.096296in}{0.585697in}}%
\pgfpathlineto{\pgfqpoint{2.096296in}{0.594699in}}%
\pgfpathlineto{\pgfqpoint{2.096296in}{0.603700in}}%
\pgfpathlineto{\pgfqpoint{2.096296in}{0.612702in}}%
\pgfpathlineto{\pgfqpoint{2.096296in}{0.621703in}}%
\pgfpathlineto{\pgfqpoint{2.096296in}{0.630705in}}%
\pgfpathlineto{\pgfqpoint{2.096296in}{0.639707in}}%
\pgfpathlineto{\pgfqpoint{2.096296in}{0.648708in}}%
\pgfpathlineto{\pgfqpoint{2.096296in}{0.657710in}}%
\pgfpathlineto{\pgfqpoint{2.096296in}{0.666711in}}%
\pgfpathlineto{\pgfqpoint{2.096296in}{0.675713in}}%
\pgfpathlineto{\pgfqpoint{2.096296in}{0.684714in}}%
\pgfpathlineto{\pgfqpoint{2.096296in}{0.693716in}}%
\pgfpathlineto{\pgfqpoint{2.096296in}{0.702717in}}%
\pgfpathlineto{\pgfqpoint{2.096296in}{0.711719in}}%
\pgfpathlineto{\pgfqpoint{2.096296in}{0.720720in}}%
\pgfpathlineto{\pgfqpoint{2.096296in}{0.729722in}}%
\pgfpathlineto{\pgfqpoint{2.096296in}{0.738724in}}%
\pgfpathlineto{\pgfqpoint{2.096296in}{0.745543in}}%
\pgfpathlineto{\pgfqpoint{2.092179in}{0.738724in}}%
\pgfpathlineto{\pgfqpoint{2.086564in}{0.732040in}}%
\pgfpathlineto{\pgfqpoint{2.084072in}{0.729722in}}%
\pgfpathlineto{\pgfqpoint{2.076833in}{0.724524in}}%
\pgfpathlineto{\pgfqpoint{2.069862in}{0.720720in}}%
\pgfpathlineto{\pgfqpoint{2.067101in}{0.719483in}}%
\pgfpathlineto{\pgfqpoint{2.057370in}{0.716146in}}%
\pgfpathlineto{\pgfqpoint{2.047638in}{0.713386in}}%
\pgfpathlineto{\pgfqpoint{2.041579in}{0.711719in}}%
\pgfpathlineto{\pgfqpoint{2.037907in}{0.710843in}}%
\pgfpathlineto{\pgfqpoint{2.028175in}{0.708162in}}%
\pgfpathlineto{\pgfqpoint{2.018444in}{0.704735in}}%
\pgfpathlineto{\pgfqpoint{2.014062in}{0.702717in}}%
\pgfpathlineto{\pgfqpoint{2.008713in}{0.700475in}}%
\pgfpathlineto{\pgfqpoint{1.998981in}{0.695258in}}%
\pgfpathlineto{\pgfqpoint{1.996643in}{0.693716in}}%
\pgfpathlineto{\pgfqpoint{1.989250in}{0.689075in}}%
\pgfpathlineto{\pgfqpoint{1.983350in}{0.684714in}}%
\pgfpathlineto{\pgfqpoint{1.979518in}{0.681901in}}%
\pgfpathlineto{\pgfqpoint{1.972038in}{0.675713in}}%
\pgfpathlineto{\pgfqpoint{1.969787in}{0.673784in}}%
\pgfpathlineto{\pgfqpoint{1.962140in}{0.666711in}}%
\pgfpathlineto{\pgfqpoint{1.960055in}{0.664629in}}%
\pgfpathlineto{\pgfqpoint{1.953365in}{0.657710in}}%
\pgfpathlineto{\pgfqpoint{1.950324in}{0.654165in}}%
\pgfpathlineto{\pgfqpoint{1.945610in}{0.648708in}}%
\pgfpathlineto{\pgfqpoint{1.940592in}{0.641869in}}%
\pgfpathlineto{\pgfqpoint{1.938925in}{0.639707in}}%
\pgfpathlineto{\pgfqpoint{1.933285in}{0.630705in}}%
\pgfpathlineto{\pgfqpoint{1.930861in}{0.625756in}}%
\pgfpathlineto{\pgfqpoint{1.928680in}{0.621703in}}%
\pgfpathlineto{\pgfqpoint{1.924975in}{0.612702in}}%
\pgfpathlineto{\pgfqpoint{1.922077in}{0.603700in}}%
\pgfpathlineto{\pgfqpoint{1.921129in}{0.600303in}}%
\pgfpathlineto{\pgfqpoint{1.919327in}{0.594699in}}%
\pgfpathlineto{\pgfqpoint{1.916343in}{0.585697in}}%
\pgfpathlineto{\pgfqpoint{1.912735in}{0.576696in}}%
\pgfpathlineto{\pgfqpoint{1.911398in}{0.574142in}}%
\pgfpathlineto{\pgfqpoint{1.907286in}{0.567694in}}%
\pgfpathlineto{\pgfqpoint{1.901666in}{0.560998in}}%
\pgfpathlineto{\pgfqpoint{1.899161in}{0.558693in}}%
\pgfpathlineto{\pgfqpoint{1.891935in}{0.553499in}}%
\pgfpathlineto{\pgfqpoint{1.884562in}{0.549691in}}%
\pgfpathclose%
\pgfusepath{fill}%
\end{pgfscope}%
\begin{pgfscope}%
\pgfpathrectangle{\pgfqpoint{0.150000in}{0.549691in}}{\pgfqpoint{1.946296in}{1.800309in}}%
\pgfusepath{clip}%
\pgfsetbuttcap%
\pgfsetroundjoin%
\definecolor{currentfill}{rgb}{0.337255,0.337255,0.337255}%
\pgfsetfillcolor{currentfill}%
\pgfsetlinewidth{0.000000pt}%
\definecolor{currentstroke}{rgb}{0.000000,0.000000,0.000000}%
\pgfsetstrokecolor{currentstroke}%
\pgfsetdash{}{0pt}%
\pgfpathmoveto{\pgfqpoint{0.154117in}{2.160968in}}%
\pgfpathlineto{\pgfqpoint{0.159731in}{2.167651in}}%
\pgfpathlineto{\pgfqpoint{0.162224in}{2.169969in}}%
\pgfpathlineto{\pgfqpoint{0.169463in}{2.175168in}}%
\pgfpathlineto{\pgfqpoint{0.176434in}{2.178971in}}%
\pgfpathlineto{\pgfqpoint{0.179194in}{2.180208in}}%
\pgfpathlineto{\pgfqpoint{0.188926in}{2.183545in}}%
\pgfpathlineto{\pgfqpoint{0.198657in}{2.186305in}}%
\pgfpathlineto{\pgfqpoint{0.204716in}{2.187972in}}%
\pgfpathlineto{\pgfqpoint{0.208389in}{2.188849in}}%
\pgfpathlineto{\pgfqpoint{0.218120in}{2.191529in}}%
\pgfpathlineto{\pgfqpoint{0.227852in}{2.194956in}}%
\pgfpathlineto{\pgfqpoint{0.232233in}{2.196974in}}%
\pgfpathlineto{\pgfqpoint{0.237583in}{2.199216in}}%
\pgfpathlineto{\pgfqpoint{0.247315in}{2.204433in}}%
\pgfpathlineto{\pgfqpoint{0.249653in}{2.205975in}}%
\pgfpathlineto{\pgfqpoint{0.257046in}{2.210616in}}%
\pgfpathlineto{\pgfqpoint{0.262946in}{2.214977in}}%
\pgfpathlineto{\pgfqpoint{0.266778in}{2.217790in}}%
\pgfpathlineto{\pgfqpoint{0.274258in}{2.223978in}}%
\pgfpathlineto{\pgfqpoint{0.276509in}{2.225907in}}%
\pgfpathlineto{\pgfqpoint{0.284156in}{2.232980in}}%
\pgfpathlineto{\pgfqpoint{0.286241in}{2.235062in}}%
\pgfpathlineto{\pgfqpoint{0.292931in}{2.241981in}}%
\pgfpathlineto{\pgfqpoint{0.295972in}{2.245526in}}%
\pgfpathlineto{\pgfqpoint{0.300686in}{2.250983in}}%
\pgfpathlineto{\pgfqpoint{0.305704in}{2.257822in}}%
\pgfpathlineto{\pgfqpoint{0.307371in}{2.259985in}}%
\pgfpathlineto{\pgfqpoint{0.313011in}{2.268986in}}%
\pgfpathlineto{\pgfqpoint{0.315435in}{2.273935in}}%
\pgfpathlineto{\pgfqpoint{0.317616in}{2.277988in}}%
\pgfpathlineto{\pgfqpoint{0.321321in}{2.286989in}}%
\pgfpathlineto{\pgfqpoint{0.324219in}{2.295991in}}%
\pgfpathlineto{\pgfqpoint{0.325167in}{2.299388in}}%
\pgfpathlineto{\pgfqpoint{0.326969in}{2.304992in}}%
\pgfpathlineto{\pgfqpoint{0.329953in}{2.313994in}}%
\pgfpathlineto{\pgfqpoint{0.333560in}{2.322995in}}%
\pgfpathlineto{\pgfqpoint{0.334898in}{2.325549in}}%
\pgfpathlineto{\pgfqpoint{0.339010in}{2.331997in}}%
\pgfpathlineto{\pgfqpoint{0.344630in}{2.338693in}}%
\pgfpathlineto{\pgfqpoint{0.347135in}{2.340998in}}%
\pgfpathlineto{\pgfqpoint{0.354361in}{2.346192in}}%
\pgfpathlineto{\pgfqpoint{0.361734in}{2.350000in}}%
\pgfpathlineto{\pgfqpoint{0.354361in}{2.350000in}}%
\pgfpathlineto{\pgfqpoint{0.344630in}{2.350000in}}%
\pgfpathlineto{\pgfqpoint{0.334898in}{2.350000in}}%
\pgfpathlineto{\pgfqpoint{0.325167in}{2.350000in}}%
\pgfpathlineto{\pgfqpoint{0.315435in}{2.350000in}}%
\pgfpathlineto{\pgfqpoint{0.305704in}{2.350000in}}%
\pgfpathlineto{\pgfqpoint{0.295972in}{2.350000in}}%
\pgfpathlineto{\pgfqpoint{0.286241in}{2.350000in}}%
\pgfpathlineto{\pgfqpoint{0.276509in}{2.350000in}}%
\pgfpathlineto{\pgfqpoint{0.266778in}{2.350000in}}%
\pgfpathlineto{\pgfqpoint{0.257046in}{2.350000in}}%
\pgfpathlineto{\pgfqpoint{0.247315in}{2.350000in}}%
\pgfpathlineto{\pgfqpoint{0.237583in}{2.350000in}}%
\pgfpathlineto{\pgfqpoint{0.227852in}{2.350000in}}%
\pgfpathlineto{\pgfqpoint{0.218120in}{2.350000in}}%
\pgfpathlineto{\pgfqpoint{0.208389in}{2.350000in}}%
\pgfpathlineto{\pgfqpoint{0.198657in}{2.350000in}}%
\pgfpathlineto{\pgfqpoint{0.188926in}{2.350000in}}%
\pgfpathlineto{\pgfqpoint{0.179194in}{2.350000in}}%
\pgfpathlineto{\pgfqpoint{0.169463in}{2.350000in}}%
\pgfpathlineto{\pgfqpoint{0.159731in}{2.350000in}}%
\pgfpathlineto{\pgfqpoint{0.150000in}{2.350000in}}%
\pgfpathlineto{\pgfqpoint{0.150000in}{2.340998in}}%
\pgfpathlineto{\pgfqpoint{0.150000in}{2.331997in}}%
\pgfpathlineto{\pgfqpoint{0.150000in}{2.322995in}}%
\pgfpathlineto{\pgfqpoint{0.150000in}{2.313994in}}%
\pgfpathlineto{\pgfqpoint{0.150000in}{2.304992in}}%
\pgfpathlineto{\pgfqpoint{0.150000in}{2.295991in}}%
\pgfpathlineto{\pgfqpoint{0.150000in}{2.286989in}}%
\pgfpathlineto{\pgfqpoint{0.150000in}{2.277988in}}%
\pgfpathlineto{\pgfqpoint{0.150000in}{2.268986in}}%
\pgfpathlineto{\pgfqpoint{0.150000in}{2.259985in}}%
\pgfpathlineto{\pgfqpoint{0.150000in}{2.250983in}}%
\pgfpathlineto{\pgfqpoint{0.150000in}{2.241981in}}%
\pgfpathlineto{\pgfqpoint{0.150000in}{2.232980in}}%
\pgfpathlineto{\pgfqpoint{0.150000in}{2.223978in}}%
\pgfpathlineto{\pgfqpoint{0.150000in}{2.214977in}}%
\pgfpathlineto{\pgfqpoint{0.150000in}{2.205975in}}%
\pgfpathlineto{\pgfqpoint{0.150000in}{2.196974in}}%
\pgfpathlineto{\pgfqpoint{0.150000in}{2.187972in}}%
\pgfpathlineto{\pgfqpoint{0.150000in}{2.178971in}}%
\pgfpathlineto{\pgfqpoint{0.150000in}{2.169969in}}%
\pgfpathlineto{\pgfqpoint{0.150000in}{2.160968in}}%
\pgfpathlineto{\pgfqpoint{0.150000in}{2.154148in}}%
\pgfpathclose%
\pgfusepath{fill}%
\end{pgfscope}%
\begin{pgfscope}%
\pgfpathrectangle{\pgfqpoint{0.150000in}{0.549691in}}{\pgfqpoint{1.946296in}{1.800309in}}%
\pgfusepath{clip}%
\pgfsetbuttcap%
\pgfsetroundjoin%
\definecolor{currentfill}{rgb}{0.337255,0.337255,0.337255}%
\pgfsetfillcolor{currentfill}%
\pgfsetlinewidth{0.000000pt}%
\definecolor{currentstroke}{rgb}{0.000000,0.000000,0.000000}%
\pgfsetstrokecolor{currentstroke}%
\pgfsetdash{}{0pt}%
\pgfpathmoveto{\pgfqpoint{2.096296in}{2.154148in}}%
\pgfpathlineto{\pgfqpoint{2.096296in}{2.160968in}}%
\pgfpathlineto{\pgfqpoint{2.096296in}{2.169969in}}%
\pgfpathlineto{\pgfqpoint{2.096296in}{2.178971in}}%
\pgfpathlineto{\pgfqpoint{2.096296in}{2.187972in}}%
\pgfpathlineto{\pgfqpoint{2.096296in}{2.196974in}}%
\pgfpathlineto{\pgfqpoint{2.096296in}{2.205975in}}%
\pgfpathlineto{\pgfqpoint{2.096296in}{2.214977in}}%
\pgfpathlineto{\pgfqpoint{2.096296in}{2.223978in}}%
\pgfpathlineto{\pgfqpoint{2.096296in}{2.232980in}}%
\pgfpathlineto{\pgfqpoint{2.096296in}{2.241981in}}%
\pgfpathlineto{\pgfqpoint{2.096296in}{2.250983in}}%
\pgfpathlineto{\pgfqpoint{2.096296in}{2.259985in}}%
\pgfpathlineto{\pgfqpoint{2.096296in}{2.268986in}}%
\pgfpathlineto{\pgfqpoint{2.096296in}{2.277988in}}%
\pgfpathlineto{\pgfqpoint{2.096296in}{2.286989in}}%
\pgfpathlineto{\pgfqpoint{2.096296in}{2.295991in}}%
\pgfpathlineto{\pgfqpoint{2.096296in}{2.304992in}}%
\pgfpathlineto{\pgfqpoint{2.096296in}{2.313994in}}%
\pgfpathlineto{\pgfqpoint{2.096296in}{2.322995in}}%
\pgfpathlineto{\pgfqpoint{2.096296in}{2.331997in}}%
\pgfpathlineto{\pgfqpoint{2.096296in}{2.340998in}}%
\pgfpathlineto{\pgfqpoint{2.096296in}{2.350000in}}%
\pgfpathlineto{\pgfqpoint{2.086564in}{2.350000in}}%
\pgfpathlineto{\pgfqpoint{2.076833in}{2.350000in}}%
\pgfpathlineto{\pgfqpoint{2.067101in}{2.350000in}}%
\pgfpathlineto{\pgfqpoint{2.057370in}{2.350000in}}%
\pgfpathlineto{\pgfqpoint{2.047638in}{2.350000in}}%
\pgfpathlineto{\pgfqpoint{2.037907in}{2.350000in}}%
\pgfpathlineto{\pgfqpoint{2.028175in}{2.350000in}}%
\pgfpathlineto{\pgfqpoint{2.018444in}{2.350000in}}%
\pgfpathlineto{\pgfqpoint{2.008713in}{2.350000in}}%
\pgfpathlineto{\pgfqpoint{1.998981in}{2.350000in}}%
\pgfpathlineto{\pgfqpoint{1.989250in}{2.350000in}}%
\pgfpathlineto{\pgfqpoint{1.979518in}{2.350000in}}%
\pgfpathlineto{\pgfqpoint{1.969787in}{2.350000in}}%
\pgfpathlineto{\pgfqpoint{1.960055in}{2.350000in}}%
\pgfpathlineto{\pgfqpoint{1.950324in}{2.350000in}}%
\pgfpathlineto{\pgfqpoint{1.940592in}{2.350000in}}%
\pgfpathlineto{\pgfqpoint{1.930861in}{2.350000in}}%
\pgfpathlineto{\pgfqpoint{1.921129in}{2.350000in}}%
\pgfpathlineto{\pgfqpoint{1.911398in}{2.350000in}}%
\pgfpathlineto{\pgfqpoint{1.901666in}{2.350000in}}%
\pgfpathlineto{\pgfqpoint{1.891935in}{2.350000in}}%
\pgfpathlineto{\pgfqpoint{1.884562in}{2.350000in}}%
\pgfpathlineto{\pgfqpoint{1.891935in}{2.346192in}}%
\pgfpathlineto{\pgfqpoint{1.899161in}{2.340998in}}%
\pgfpathlineto{\pgfqpoint{1.901666in}{2.338693in}}%
\pgfpathlineto{\pgfqpoint{1.907286in}{2.331997in}}%
\pgfpathlineto{\pgfqpoint{1.911398in}{2.325549in}}%
\pgfpathlineto{\pgfqpoint{1.912735in}{2.322995in}}%
\pgfpathlineto{\pgfqpoint{1.916343in}{2.313994in}}%
\pgfpathlineto{\pgfqpoint{1.919327in}{2.304992in}}%
\pgfpathlineto{\pgfqpoint{1.921129in}{2.299388in}}%
\pgfpathlineto{\pgfqpoint{1.922077in}{2.295991in}}%
\pgfpathlineto{\pgfqpoint{1.924975in}{2.286989in}}%
\pgfpathlineto{\pgfqpoint{1.928680in}{2.277988in}}%
\pgfpathlineto{\pgfqpoint{1.930861in}{2.273935in}}%
\pgfpathlineto{\pgfqpoint{1.933285in}{2.268986in}}%
\pgfpathlineto{\pgfqpoint{1.938925in}{2.259985in}}%
\pgfpathlineto{\pgfqpoint{1.940592in}{2.257822in}}%
\pgfpathlineto{\pgfqpoint{1.945610in}{2.250983in}}%
\pgfpathlineto{\pgfqpoint{1.950324in}{2.245526in}}%
\pgfpathlineto{\pgfqpoint{1.953365in}{2.241981in}}%
\pgfpathlineto{\pgfqpoint{1.960055in}{2.235062in}}%
\pgfpathlineto{\pgfqpoint{1.962140in}{2.232980in}}%
\pgfpathlineto{\pgfqpoint{1.969787in}{2.225907in}}%
\pgfpathlineto{\pgfqpoint{1.972038in}{2.223978in}}%
\pgfpathlineto{\pgfqpoint{1.979518in}{2.217790in}}%
\pgfpathlineto{\pgfqpoint{1.983350in}{2.214977in}}%
\pgfpathlineto{\pgfqpoint{1.989250in}{2.210616in}}%
\pgfpathlineto{\pgfqpoint{1.996643in}{2.205975in}}%
\pgfpathlineto{\pgfqpoint{1.998981in}{2.204433in}}%
\pgfpathlineto{\pgfqpoint{2.008713in}{2.199216in}}%
\pgfpathlineto{\pgfqpoint{2.014062in}{2.196974in}}%
\pgfpathlineto{\pgfqpoint{2.018444in}{2.194956in}}%
\pgfpathlineto{\pgfqpoint{2.028175in}{2.191529in}}%
\pgfpathlineto{\pgfqpoint{2.037907in}{2.188849in}}%
\pgfpathlineto{\pgfqpoint{2.041579in}{2.187972in}}%
\pgfpathlineto{\pgfqpoint{2.047638in}{2.186305in}}%
\pgfpathlineto{\pgfqpoint{2.057370in}{2.183545in}}%
\pgfpathlineto{\pgfqpoint{2.067101in}{2.180208in}}%
\pgfpathlineto{\pgfqpoint{2.069862in}{2.178971in}}%
\pgfpathlineto{\pgfqpoint{2.076833in}{2.175168in}}%
\pgfpathlineto{\pgfqpoint{2.084072in}{2.169969in}}%
\pgfpathlineto{\pgfqpoint{2.086564in}{2.167651in}}%
\pgfpathlineto{\pgfqpoint{2.092179in}{2.160968in}}%
\pgfpathclose%
\pgfusepath{fill}%
\end{pgfscope}%
\begin{pgfscope}%
\pgfsetbuttcap%
\pgfsetroundjoin%
\definecolor{currentfill}{rgb}{0.000000,0.000000,0.000000}%
\pgfsetfillcolor{currentfill}%
\pgfsetlinewidth{0.803000pt}%
\definecolor{currentstroke}{rgb}{0.000000,0.000000,0.000000}%
\pgfsetstrokecolor{currentstroke}%
\pgfsetdash{}{0pt}%
\pgfsys@defobject{currentmarker}{\pgfqpoint{0.000000in}{-0.048611in}}{\pgfqpoint{0.000000in}{0.000000in}}{%
\pgfpathmoveto{\pgfqpoint{0.000000in}{0.000000in}}%
\pgfpathlineto{\pgfqpoint{0.000000in}{-0.048611in}}%
\pgfusepath{stroke,fill}%
}%
\begin{pgfscope}%
\pgfsys@transformshift{0.150000in}{0.549691in}%
\pgfsys@useobject{currentmarker}{}%
\end{pgfscope}%
\end{pgfscope}%
\begin{pgfscope}%
\definecolor{textcolor}{rgb}{0.000000,0.000000,0.000000}%
\pgfsetstrokecolor{textcolor}%
\pgfsetfillcolor{textcolor}%
\pgftext[x=0.150000in,y=0.452469in,,top]{\color{textcolor}\rmfamily\fontsize{10.000000}{12.000000}\selectfont \(\displaystyle -5.0\)}%
\end{pgfscope}%
\begin{pgfscope}%
\pgfsetbuttcap%
\pgfsetroundjoin%
\definecolor{currentfill}{rgb}{0.000000,0.000000,0.000000}%
\pgfsetfillcolor{currentfill}%
\pgfsetlinewidth{0.803000pt}%
\definecolor{currentstroke}{rgb}{0.000000,0.000000,0.000000}%
\pgfsetstrokecolor{currentstroke}%
\pgfsetdash{}{0pt}%
\pgfsys@defobject{currentmarker}{\pgfqpoint{0.000000in}{-0.048611in}}{\pgfqpoint{0.000000in}{0.000000in}}{%
\pgfpathmoveto{\pgfqpoint{0.000000in}{0.000000in}}%
\pgfpathlineto{\pgfqpoint{0.000000in}{-0.048611in}}%
\pgfusepath{stroke,fill}%
}%
\begin{pgfscope}%
\pgfsys@transformshift{0.636574in}{0.549691in}%
\pgfsys@useobject{currentmarker}{}%
\end{pgfscope}%
\end{pgfscope}%
\begin{pgfscope}%
\definecolor{textcolor}{rgb}{0.000000,0.000000,0.000000}%
\pgfsetstrokecolor{textcolor}%
\pgfsetfillcolor{textcolor}%
\pgftext[x=0.636574in,y=0.452469in,,top]{\color{textcolor}\rmfamily\fontsize{10.000000}{12.000000}\selectfont \(\displaystyle -2.5\)}%
\end{pgfscope}%
\begin{pgfscope}%
\pgfsetbuttcap%
\pgfsetroundjoin%
\definecolor{currentfill}{rgb}{0.000000,0.000000,0.000000}%
\pgfsetfillcolor{currentfill}%
\pgfsetlinewidth{0.803000pt}%
\definecolor{currentstroke}{rgb}{0.000000,0.000000,0.000000}%
\pgfsetstrokecolor{currentstroke}%
\pgfsetdash{}{0pt}%
\pgfsys@defobject{currentmarker}{\pgfqpoint{0.000000in}{-0.048611in}}{\pgfqpoint{0.000000in}{0.000000in}}{%
\pgfpathmoveto{\pgfqpoint{0.000000in}{0.000000in}}%
\pgfpathlineto{\pgfqpoint{0.000000in}{-0.048611in}}%
\pgfusepath{stroke,fill}%
}%
\begin{pgfscope}%
\pgfsys@transformshift{1.123148in}{0.549691in}%
\pgfsys@useobject{currentmarker}{}%
\end{pgfscope}%
\end{pgfscope}%
\begin{pgfscope}%
\definecolor{textcolor}{rgb}{0.000000,0.000000,0.000000}%
\pgfsetstrokecolor{textcolor}%
\pgfsetfillcolor{textcolor}%
\pgftext[x=1.123148in,y=0.452469in,,top]{\color{textcolor}\rmfamily\fontsize{10.000000}{12.000000}\selectfont \(\displaystyle 0.0\)}%
\end{pgfscope}%
\begin{pgfscope}%
\pgfsetbuttcap%
\pgfsetroundjoin%
\definecolor{currentfill}{rgb}{0.000000,0.000000,0.000000}%
\pgfsetfillcolor{currentfill}%
\pgfsetlinewidth{0.803000pt}%
\definecolor{currentstroke}{rgb}{0.000000,0.000000,0.000000}%
\pgfsetstrokecolor{currentstroke}%
\pgfsetdash{}{0pt}%
\pgfsys@defobject{currentmarker}{\pgfqpoint{0.000000in}{-0.048611in}}{\pgfqpoint{0.000000in}{0.000000in}}{%
\pgfpathmoveto{\pgfqpoint{0.000000in}{0.000000in}}%
\pgfpathlineto{\pgfqpoint{0.000000in}{-0.048611in}}%
\pgfusepath{stroke,fill}%
}%
\begin{pgfscope}%
\pgfsys@transformshift{1.609722in}{0.549691in}%
\pgfsys@useobject{currentmarker}{}%
\end{pgfscope}%
\end{pgfscope}%
\begin{pgfscope}%
\definecolor{textcolor}{rgb}{0.000000,0.000000,0.000000}%
\pgfsetstrokecolor{textcolor}%
\pgfsetfillcolor{textcolor}%
\pgftext[x=1.609722in,y=0.452469in,,top]{\color{textcolor}\rmfamily\fontsize{10.000000}{12.000000}\selectfont \(\displaystyle 2.5\)}%
\end{pgfscope}%
\begin{pgfscope}%
\pgfsetbuttcap%
\pgfsetroundjoin%
\definecolor{currentfill}{rgb}{0.000000,0.000000,0.000000}%
\pgfsetfillcolor{currentfill}%
\pgfsetlinewidth{0.803000pt}%
\definecolor{currentstroke}{rgb}{0.000000,0.000000,0.000000}%
\pgfsetstrokecolor{currentstroke}%
\pgfsetdash{}{0pt}%
\pgfsys@defobject{currentmarker}{\pgfqpoint{0.000000in}{-0.048611in}}{\pgfqpoint{0.000000in}{0.000000in}}{%
\pgfpathmoveto{\pgfqpoint{0.000000in}{0.000000in}}%
\pgfpathlineto{\pgfqpoint{0.000000in}{-0.048611in}}%
\pgfusepath{stroke,fill}%
}%
\begin{pgfscope}%
\pgfsys@transformshift{2.096296in}{0.549691in}%
\pgfsys@useobject{currentmarker}{}%
\end{pgfscope}%
\end{pgfscope}%
\begin{pgfscope}%
\definecolor{textcolor}{rgb}{0.000000,0.000000,0.000000}%
\pgfsetstrokecolor{textcolor}%
\pgfsetfillcolor{textcolor}%
\pgftext[x=2.096296in,y=0.452469in,,top]{\color{textcolor}\rmfamily\fontsize{10.000000}{12.000000}\selectfont \(\displaystyle 5.0\)}%
\end{pgfscope}%
\begin{pgfscope}%
\definecolor{textcolor}{rgb}{0.000000,0.000000,0.000000}%
\pgfsetstrokecolor{textcolor}%
\pgfsetfillcolor{textcolor}%
\pgftext[x=1.123148in,y=0.273457in,,top]{\color{textcolor}\rmfamily\fontsize{10.000000}{12.000000}\selectfont \(\displaystyle x\)}%
\end{pgfscope}%
\begin{pgfscope}%
\pgfsetbuttcap%
\pgfsetroundjoin%
\definecolor{currentfill}{rgb}{0.000000,0.000000,0.000000}%
\pgfsetfillcolor{currentfill}%
\pgfsetlinewidth{0.803000pt}%
\definecolor{currentstroke}{rgb}{0.000000,0.000000,0.000000}%
\pgfsetstrokecolor{currentstroke}%
\pgfsetdash{}{0pt}%
\pgfsys@defobject{currentmarker}{\pgfqpoint{0.000000in}{0.000000in}}{\pgfqpoint{0.048611in}{0.000000in}}{%
\pgfpathmoveto{\pgfqpoint{0.000000in}{0.000000in}}%
\pgfpathlineto{\pgfqpoint{0.048611in}{0.000000in}}%
\pgfusepath{stroke,fill}%
}%
\begin{pgfscope}%
\pgfsys@transformshift{2.096296in}{0.729722in}%
\pgfsys@useobject{currentmarker}{}%
\end{pgfscope}%
\end{pgfscope}%
\begin{pgfscope}%
\definecolor{textcolor}{rgb}{0.000000,0.000000,0.000000}%
\pgfsetstrokecolor{textcolor}%
\pgfsetfillcolor{textcolor}%
\pgftext[x=2.193518in,y=0.681497in,left,base]{\color{textcolor}\rmfamily\fontsize{10.000000}{12.000000}\selectfont \(\displaystyle -4\)}%
\end{pgfscope}%
\begin{pgfscope}%
\pgfsetbuttcap%
\pgfsetroundjoin%
\definecolor{currentfill}{rgb}{0.000000,0.000000,0.000000}%
\pgfsetfillcolor{currentfill}%
\pgfsetlinewidth{0.803000pt}%
\definecolor{currentstroke}{rgb}{0.000000,0.000000,0.000000}%
\pgfsetstrokecolor{currentstroke}%
\pgfsetdash{}{0pt}%
\pgfsys@defobject{currentmarker}{\pgfqpoint{0.000000in}{0.000000in}}{\pgfqpoint{0.048611in}{0.000000in}}{%
\pgfpathmoveto{\pgfqpoint{0.000000in}{0.000000in}}%
\pgfpathlineto{\pgfqpoint{0.048611in}{0.000000in}}%
\pgfusepath{stroke,fill}%
}%
\begin{pgfscope}%
\pgfsys@transformshift{2.096296in}{1.089784in}%
\pgfsys@useobject{currentmarker}{}%
\end{pgfscope}%
\end{pgfscope}%
\begin{pgfscope}%
\definecolor{textcolor}{rgb}{0.000000,0.000000,0.000000}%
\pgfsetstrokecolor{textcolor}%
\pgfsetfillcolor{textcolor}%
\pgftext[x=2.193518in,y=1.041559in,left,base]{\color{textcolor}\rmfamily\fontsize{10.000000}{12.000000}\selectfont \(\displaystyle -2\)}%
\end{pgfscope}%
\begin{pgfscope}%
\pgfsetbuttcap%
\pgfsetroundjoin%
\definecolor{currentfill}{rgb}{0.000000,0.000000,0.000000}%
\pgfsetfillcolor{currentfill}%
\pgfsetlinewidth{0.803000pt}%
\definecolor{currentstroke}{rgb}{0.000000,0.000000,0.000000}%
\pgfsetstrokecolor{currentstroke}%
\pgfsetdash{}{0pt}%
\pgfsys@defobject{currentmarker}{\pgfqpoint{0.000000in}{0.000000in}}{\pgfqpoint{0.048611in}{0.000000in}}{%
\pgfpathmoveto{\pgfqpoint{0.000000in}{0.000000in}}%
\pgfpathlineto{\pgfqpoint{0.048611in}{0.000000in}}%
\pgfusepath{stroke,fill}%
}%
\begin{pgfscope}%
\pgfsys@transformshift{2.096296in}{1.449846in}%
\pgfsys@useobject{currentmarker}{}%
\end{pgfscope}%
\end{pgfscope}%
\begin{pgfscope}%
\definecolor{textcolor}{rgb}{0.000000,0.000000,0.000000}%
\pgfsetstrokecolor{textcolor}%
\pgfsetfillcolor{textcolor}%
\pgftext[x=2.193518in,y=1.401620in,left,base]{\color{textcolor}\rmfamily\fontsize{10.000000}{12.000000}\selectfont \(\displaystyle 0\)}%
\end{pgfscope}%
\begin{pgfscope}%
\pgfsetbuttcap%
\pgfsetroundjoin%
\definecolor{currentfill}{rgb}{0.000000,0.000000,0.000000}%
\pgfsetfillcolor{currentfill}%
\pgfsetlinewidth{0.803000pt}%
\definecolor{currentstroke}{rgb}{0.000000,0.000000,0.000000}%
\pgfsetstrokecolor{currentstroke}%
\pgfsetdash{}{0pt}%
\pgfsys@defobject{currentmarker}{\pgfqpoint{0.000000in}{0.000000in}}{\pgfqpoint{0.048611in}{0.000000in}}{%
\pgfpathmoveto{\pgfqpoint{0.000000in}{0.000000in}}%
\pgfpathlineto{\pgfqpoint{0.048611in}{0.000000in}}%
\pgfusepath{stroke,fill}%
}%
\begin{pgfscope}%
\pgfsys@transformshift{2.096296in}{1.809907in}%
\pgfsys@useobject{currentmarker}{}%
\end{pgfscope}%
\end{pgfscope}%
\begin{pgfscope}%
\definecolor{textcolor}{rgb}{0.000000,0.000000,0.000000}%
\pgfsetstrokecolor{textcolor}%
\pgfsetfillcolor{textcolor}%
\pgftext[x=2.193518in,y=1.761682in,left,base]{\color{textcolor}\rmfamily\fontsize{10.000000}{12.000000}\selectfont \(\displaystyle 2\)}%
\end{pgfscope}%
\begin{pgfscope}%
\pgfsetbuttcap%
\pgfsetroundjoin%
\definecolor{currentfill}{rgb}{0.000000,0.000000,0.000000}%
\pgfsetfillcolor{currentfill}%
\pgfsetlinewidth{0.803000pt}%
\definecolor{currentstroke}{rgb}{0.000000,0.000000,0.000000}%
\pgfsetstrokecolor{currentstroke}%
\pgfsetdash{}{0pt}%
\pgfsys@defobject{currentmarker}{\pgfqpoint{0.000000in}{0.000000in}}{\pgfqpoint{0.048611in}{0.000000in}}{%
\pgfpathmoveto{\pgfqpoint{0.000000in}{0.000000in}}%
\pgfpathlineto{\pgfqpoint{0.048611in}{0.000000in}}%
\pgfusepath{stroke,fill}%
}%
\begin{pgfscope}%
\pgfsys@transformshift{2.096296in}{2.169969in}%
\pgfsys@useobject{currentmarker}{}%
\end{pgfscope}%
\end{pgfscope}%
\begin{pgfscope}%
\definecolor{textcolor}{rgb}{0.000000,0.000000,0.000000}%
\pgfsetstrokecolor{textcolor}%
\pgfsetfillcolor{textcolor}%
\pgftext[x=2.193518in,y=2.121744in,left,base]{\color{textcolor}\rmfamily\fontsize{10.000000}{12.000000}\selectfont \(\displaystyle 4\)}%
\end{pgfscope}%
\begin{pgfscope}%
\definecolor{textcolor}{rgb}{0.000000,0.000000,0.000000}%
\pgfsetstrokecolor{textcolor}%
\pgfsetfillcolor{textcolor}%
\pgftext[x=2.426543in,y=1.449846in,,top,rotate=90.000000]{\color{textcolor}\rmfamily\fontsize{10.000000}{12.000000}\selectfont \(\displaystyle y\)}%
\end{pgfscope}%
\begin{pgfscope}%
\pgfpathrectangle{\pgfqpoint{0.150000in}{0.549691in}}{\pgfqpoint{1.946296in}{1.800309in}}%
\pgfusepath{clip}%
\pgfsetrectcap%
\pgfsetroundjoin%
\pgfsetlinewidth{2.007500pt}%
\definecolor{currentstroke}{rgb}{1.000000,0.498039,0.054902}%
\pgfsetstrokecolor{currentstroke}%
\pgfsetdash{}{0pt}%
\pgfpathmoveto{\pgfqpoint{2.096296in}{2.350000in}}%
\pgfpathlineto{\pgfqpoint{2.076833in}{2.331997in}}%
\pgfpathlineto{\pgfqpoint{2.047430in}{2.304800in}}%
\pgfpathlineto{\pgfqpoint{2.014711in}{2.274535in}}%
\pgfpathlineto{\pgfqpoint{1.973379in}{2.236303in}}%
\pgfpathlineto{\pgfqpoint{1.910181in}{2.177845in}}%
\pgfpathlineto{\pgfqpoint{1.834242in}{2.107603in}}%
\pgfpathlineto{\pgfqpoint{1.761728in}{2.040528in}}%
\pgfpathlineto{\pgfqpoint{1.671696in}{1.957249in}}%
\pgfpathlineto{\pgfqpoint{1.590813in}{1.882432in}}%
\pgfpathlineto{\pgfqpoint{1.501654in}{1.799961in}}%
\pgfpathlineto{\pgfqpoint{1.418002in}{1.722583in}}%
\pgfpathlineto{\pgfqpoint{1.337965in}{1.648549in}}%
\pgfpathlineto{\pgfqpoint{1.254217in}{1.571084in}}%
\pgfpathlineto{\pgfqpoint{1.187064in}{1.508968in}}%
\pgfpathlineto{\pgfqpoint{1.114573in}{1.441914in}}%
\pgfpathlineto{\pgfqpoint{1.052845in}{1.384816in}}%
\pgfpathlineto{\pgfqpoint{1.008058in}{1.343388in}}%
\pgfpathlineto{\pgfqpoint{0.963413in}{1.302092in}}%
\pgfpathlineto{\pgfqpoint{0.915386in}{1.257667in}}%
\pgfpathlineto{\pgfqpoint{0.881347in}{1.226182in}}%
\pgfpathlineto{\pgfqpoint{0.867763in}{1.213617in}}%
\pgfpathlineto{\pgfqpoint{0.871953in}{1.217493in}}%
\pgfpathlineto{\pgfqpoint{0.892581in}{1.236573in}}%
\pgfpathlineto{\pgfqpoint{0.926970in}{1.268382in}}%
\pgfpathlineto{\pgfqpoint{0.962454in}{1.301205in}}%
\pgfpathlineto{\pgfqpoint{0.986730in}{1.323660in}}%
\pgfpathlineto{\pgfqpoint{0.999655in}{1.335616in}}%
\pgfpathlineto{\pgfqpoint{1.004610in}{1.340199in}}%
\pgfpathlineto{\pgfqpoint{1.003702in}{1.339359in}}%
\pgfpathlineto{\pgfqpoint{0.997261in}{1.333401in}}%
\pgfpathlineto{\pgfqpoint{0.984235in}{1.321352in}}%
\pgfpathlineto{\pgfqpoint{0.963377in}{1.302059in}}%
\pgfpathlineto{\pgfqpoint{0.936765in}{1.277442in}}%
\pgfpathlineto{\pgfqpoint{0.913324in}{1.255760in}}%
\pgfpathlineto{\pgfqpoint{0.902185in}{1.245456in}}%
\pgfpathlineto{\pgfqpoint{0.905767in}{1.248770in}}%
\pgfpathlineto{\pgfqpoint{0.921534in}{1.263355in}}%
\pgfpathlineto{\pgfqpoint{0.942491in}{1.282740in}}%
\pgfpathlineto{\pgfqpoint{0.959634in}{1.298597in}}%
\pgfpathlineto{\pgfqpoint{0.968013in}{1.306347in}}%
\pgfpathlineto{\pgfqpoint{0.966959in}{1.305372in}}%
\pgfpathlineto{\pgfqpoint{0.957565in}{1.296682in}}%
\pgfpathlineto{\pgfqpoint{0.942562in}{1.282805in}}%
\pgfpathlineto{\pgfqpoint{0.927315in}{1.268702in}}%
\pgfpathlineto{\pgfqpoint{0.917985in}{1.260071in}}%
\pgfpathlineto{\pgfqpoint{0.917769in}{1.259872in}}%
\pgfpathlineto{\pgfqpoint{0.925842in}{1.267339in}}%
\pgfpathlineto{\pgfqpoint{0.938109in}{1.278686in}}%
\pgfpathlineto{\pgfqpoint{0.949124in}{1.288874in}}%
\pgfpathlineto{\pgfqpoint{0.954970in}{1.294282in}}%
\pgfpathlineto{\pgfqpoint{0.954376in}{1.293733in}}%
\pgfpathlineto{\pgfqpoint{0.948154in}{1.287977in}}%
\pgfpathlineto{\pgfqpoint{0.938811in}{1.279335in}}%
\pgfpathlineto{\pgfqpoint{0.930100in}{1.271277in}}%
\pgfpathlineto{\pgfqpoint{0.925497in}{1.267020in}}%
\pgfpathlineto{\pgfqpoint{0.926498in}{1.267946in}}%
\pgfpathlineto{\pgfqpoint{0.932129in}{1.273155in}}%
\pgfpathlineto{\pgfqpoint{0.939596in}{1.280061in}}%
\pgfpathlineto{\pgfqpoint{0.945706in}{1.285713in}}%
\pgfpathlineto{\pgfqpoint{0.948311in}{1.288123in}}%
\pgfpathlineto{\pgfqpoint{0.946860in}{1.286780in}}%
\pgfpathlineto{\pgfqpoint{0.942254in}{1.282520in}}%
\pgfpathlineto{\pgfqpoint{0.936480in}{1.277180in}}%
\pgfpathlineto{\pgfqpoint{0.931909in}{1.272951in}}%
\pgfpathlineto{\pgfqpoint{0.930283in}{1.271447in}}%
\pgfpathlineto{\pgfqpoint{0.931981in}{1.273017in}}%
\pgfpathlineto{\pgfqpoint{0.935968in}{1.276706in}}%
\pgfpathlineto{\pgfqpoint{0.940388in}{1.280794in}}%
\pgfpathlineto{\pgfqpoint{0.943448in}{1.283625in}}%
\pgfpathlineto{\pgfqpoint{0.944128in}{1.284254in}}%
\pgfpathlineto{\pgfqpoint{0.942416in}{1.282670in}}%
\pgfpathlineto{\pgfqpoint{0.939185in}{1.279681in}}%
\pgfpathlineto{\pgfqpoint{0.935828in}{1.276576in}}%
\pgfpathlineto{\pgfqpoint{0.933694in}{1.274602in}}%
\pgfpathlineto{\pgfqpoint{0.933529in}{1.274449in}}%
\pgfpathlineto{\pgfqpoint{0.935203in}{1.275998in}}%
\pgfpathlineto{\pgfqpoint{0.937850in}{1.278447in}}%
\pgfpathlineto{\pgfqpoint{0.940310in}{1.280722in}}%
\pgfpathlineto{\pgfqpoint{0.941637in}{1.281949in}}%
\pgfpathlineto{\pgfqpoint{0.941436in}{1.281763in}}%
\pgfpathlineto{\pgfqpoint{0.939936in}{1.280376in}}%
\pgfpathlineto{\pgfqpoint{0.937844in}{1.278441in}}%
\pgfpathlineto{\pgfqpoint{0.936041in}{1.276773in}}%
\pgfpathlineto{\pgfqpoint{0.935220in}{1.276014in}}%
\pgfpathlineto{\pgfqpoint{0.935614in}{1.276378in}}%
\pgfpathlineto{\pgfqpoint{0.936943in}{1.277607in}}%
\pgfpathlineto{\pgfqpoint{0.938578in}{1.279119in}}%
\pgfpathlineto{\pgfqpoint{0.939839in}{1.280286in}}%
\pgfpathlineto{\pgfqpoint{0.940269in}{1.280684in}}%
\pgfpathlineto{\pgfqpoint{0.939786in}{1.280237in}}%
\pgfpathlineto{\pgfqpoint{0.938667in}{1.279202in}}%
\pgfpathlineto{\pgfqpoint{0.937415in}{1.278044in}}%
\pgfpathlineto{\pgfqpoint{0.936538in}{1.277233in}}%
\pgfpathlineto{\pgfqpoint{0.936351in}{1.277059in}}%
\pgfpathlineto{\pgfqpoint{0.936858in}{1.277529in}}%
\pgfpathlineto{\pgfqpoint{0.937788in}{1.278389in}}%
\pgfpathlineto{\pgfqpoint{0.938727in}{1.279258in}}%
\pgfpathlineto{\pgfqpoint{0.939303in}{1.279790in}}%
\pgfpathlineto{\pgfqpoint{0.939326in}{1.279812in}}%
\pgfpathlineto{\pgfqpoint{0.938842in}{1.279364in}}%
\pgfusepath{stroke}%
\end{pgfscope}%
\begin{pgfscope}%
\pgfpathrectangle{\pgfqpoint{0.150000in}{0.549691in}}{\pgfqpoint{1.946296in}{1.800309in}}%
\pgfusepath{clip}%
\pgfsetrectcap%
\pgfsetroundjoin%
\pgfsetlinewidth{2.007500pt}%
\definecolor{currentstroke}{rgb}{0.839216,0.152941,0.156863}%
\pgfsetstrokecolor{currentstroke}%
\pgfsetdash{}{0pt}%
\pgfpathmoveto{\pgfqpoint{2.096296in}{2.350000in}}%
\pgfpathlineto{\pgfqpoint{2.076833in}{2.331997in}}%
\pgfpathlineto{\pgfqpoint{2.064947in}{2.321003in}}%
\pgfpathlineto{\pgfqpoint{2.056480in}{2.313171in}}%
\pgfpathlineto{\pgfqpoint{2.049548in}{2.306758in}}%
\pgfpathlineto{\pgfqpoint{2.043225in}{2.300910in}}%
\pgfpathlineto{\pgfqpoint{2.036929in}{2.295086in}}%
\pgfpathlineto{\pgfqpoint{2.030159in}{2.288824in}}%
\pgfpathlineto{\pgfqpoint{2.022351in}{2.281602in}}%
\pgfpathlineto{\pgfqpoint{2.012743in}{2.272714in}}%
\pgfpathlineto{\pgfqpoint{2.000207in}{2.261118in}}%
\pgfpathlineto{\pgfqpoint{1.983150in}{2.245341in}}%
\pgfpathlineto{\pgfqpoint{1.959968in}{2.223898in}}%
\pgfpathlineto{\pgfqpoint{1.931591in}{2.197649in}}%
\pgfpathlineto{\pgfqpoint{1.905362in}{2.173388in}}%
\pgfpathlineto{\pgfqpoint{1.888262in}{2.157570in}}%
\pgfpathlineto{\pgfqpoint{1.878088in}{2.148159in}}%
\pgfpathlineto{\pgfqpoint{1.871424in}{2.141996in}}%
\pgfpathlineto{\pgfqpoint{1.866590in}{2.137523in}}%
\pgfpathlineto{\pgfqpoint{1.862793in}{2.134012in}}%
\pgfpathlineto{\pgfqpoint{1.859624in}{2.131080in}}%
\pgfpathlineto{\pgfqpoint{1.856846in}{2.128510in}}%
\pgfpathlineto{\pgfqpoint{1.854307in}{2.126162in}}%
\pgfpathlineto{\pgfqpoint{1.851902in}{2.123938in}}%
\pgfpathlineto{\pgfqpoint{1.849548in}{2.121760in}}%
\pgfpathlineto{\pgfqpoint{1.847173in}{2.119563in}}%
\pgfpathlineto{\pgfqpoint{1.844705in}{2.117280in}}%
\pgfpathlineto{\pgfqpoint{1.842066in}{2.114839in}}%
\pgfpathlineto{\pgfqpoint{1.839161in}{2.112152in}}%
\pgfpathlineto{\pgfqpoint{1.835868in}{2.109106in}}%
\pgfpathlineto{\pgfqpoint{1.832018in}{2.105545in}}%
\pgfpathlineto{\pgfqpoint{1.827367in}{2.101243in}}%
\pgfpathlineto{\pgfqpoint{1.821556in}{2.095868in}}%
\pgfpathlineto{\pgfqpoint{1.814036in}{2.088912in}}%
\pgfpathlineto{\pgfqpoint{1.803980in}{2.079610in}}%
\pgfpathlineto{\pgfqpoint{1.790216in}{2.066879in}}%
\pgfpathlineto{\pgfqpoint{1.771486in}{2.049553in}}%
\pgfpathlineto{\pgfqpoint{1.747845in}{2.027686in}}%
\pgfpathlineto{\pgfqpoint{1.723513in}{2.005179in}}%
\pgfpathlineto{\pgfqpoint{1.705303in}{1.988335in}}%
\pgfpathlineto{\pgfqpoint{1.694344in}{1.978198in}}%
\pgfpathlineto{\pgfqpoint{1.687792in}{1.972137in}}%
\pgfpathlineto{\pgfqpoint{1.683617in}{1.968275in}}%
\pgfpathlineto{\pgfqpoint{1.680797in}{1.965667in}}%
\pgfpathlineto{\pgfqpoint{1.678808in}{1.963827in}}%
\pgfpathlineto{\pgfqpoint{1.677359in}{1.962487in}}%
\pgfpathlineto{\pgfqpoint{1.676279in}{1.961488in}}%
\pgfpathlineto{\pgfqpoint{1.675460in}{1.960730in}}%
\pgfpathlineto{\pgfqpoint{1.674830in}{1.960148in}}%
\pgfpathlineto{\pgfqpoint{1.674341in}{1.959695in}}%
\pgfpathlineto{\pgfqpoint{1.673958in}{1.959341in}}%
\pgfpathlineto{\pgfqpoint{1.673657in}{1.959062in}}%
\pgfpathlineto{\pgfqpoint{1.673418in}{1.958841in}}%
\pgfpathlineto{\pgfqpoint{1.673228in}{1.958666in}}%
\pgfpathlineto{\pgfqpoint{1.673077in}{1.958526in}}%
\pgfpathlineto{\pgfqpoint{1.672957in}{1.958415in}}%
\pgfpathlineto{\pgfqpoint{1.672860in}{1.958325in}}%
\pgfpathlineto{\pgfqpoint{1.672783in}{1.958254in}}%
\pgfpathlineto{\pgfqpoint{1.672721in}{1.958196in}}%
\pgfpathlineto{\pgfqpoint{1.672671in}{1.958150in}}%
\pgfpathlineto{\pgfqpoint{1.672631in}{1.958113in}}%
\pgfpathlineto{\pgfqpoint{1.672598in}{1.958083in}}%
\pgfpathlineto{\pgfqpoint{1.672572in}{1.958059in}}%
\pgfpathlineto{\pgfqpoint{1.672551in}{1.958040in}}%
\pgfpathlineto{\pgfqpoint{1.672535in}{1.958024in}}%
\pgfpathlineto{\pgfqpoint{1.672521in}{1.958011in}}%
\pgfpathlineto{\pgfqpoint{1.672510in}{1.958001in}}%
\pgfpathlineto{\pgfqpoint{1.672501in}{1.957993in}}%
\pgfpathlineto{\pgfqpoint{1.672494in}{1.957987in}}%
\pgfpathlineto{\pgfqpoint{1.672488in}{1.957981in}}%
\pgfpathlineto{\pgfqpoint{1.672484in}{1.957977in}}%
\pgfpathlineto{\pgfqpoint{1.672480in}{1.957974in}}%
\pgfpathlineto{\pgfqpoint{1.672477in}{1.957971in}}%
\pgfpathlineto{\pgfqpoint{1.672475in}{1.957969in}}%
\pgfpathlineto{\pgfqpoint{1.672473in}{1.957967in}}%
\pgfpathlineto{\pgfqpoint{1.672471in}{1.957965in}}%
\pgfpathlineto{\pgfqpoint{1.672470in}{1.957964in}}%
\pgfpathlineto{\pgfqpoint{1.672469in}{1.957963in}}%
\pgfpathlineto{\pgfqpoint{1.672468in}{1.957962in}}%
\pgfpathlineto{\pgfqpoint{1.672467in}{1.957962in}}%
\pgfpathlineto{\pgfqpoint{1.672467in}{1.957961in}}%
\pgfpathlineto{\pgfqpoint{1.672466in}{1.957961in}}%
\pgfpathlineto{\pgfqpoint{1.672466in}{1.957961in}}%
\pgfpathlineto{\pgfqpoint{1.672466in}{1.957960in}}%
\pgfpathlineto{\pgfqpoint{1.672465in}{1.957960in}}%
\pgfpathlineto{\pgfqpoint{1.672465in}{1.957960in}}%
\pgfpathlineto{\pgfqpoint{1.672465in}{1.957960in}}%
\pgfpathlineto{\pgfqpoint{1.672465in}{1.957960in}}%
\pgfpathlineto{\pgfqpoint{1.672465in}{1.957960in}}%
\pgfpathlineto{\pgfqpoint{1.672465in}{1.957960in}}%
\pgfpathlineto{\pgfqpoint{1.672465in}{1.957959in}}%
\pgfpathlineto{\pgfqpoint{1.672465in}{1.957959in}}%
\pgfpathlineto{\pgfqpoint{1.672465in}{1.957959in}}%
\pgfpathlineto{\pgfqpoint{1.672465in}{1.957959in}}%
\pgfpathlineto{\pgfqpoint{1.672465in}{1.957959in}}%
\pgfpathlineto{\pgfqpoint{1.672465in}{1.957959in}}%
\pgfpathlineto{\pgfqpoint{1.672465in}{1.957959in}}%
\pgfpathlineto{\pgfqpoint{1.672465in}{1.957959in}}%
\pgfpathlineto{\pgfqpoint{1.672465in}{1.957959in}}%
\pgfpathlineto{\pgfqpoint{1.672465in}{1.957959in}}%
\pgfpathlineto{\pgfqpoint{1.672465in}{1.957959in}}%
\pgfusepath{stroke}%
\end{pgfscope}%
\begin{pgfscope}%
\pgfpathrectangle{\pgfqpoint{0.150000in}{0.549691in}}{\pgfqpoint{1.946296in}{1.800309in}}%
\pgfusepath{clip}%
\pgfsetrectcap%
\pgfsetroundjoin%
\pgfsetlinewidth{2.007500pt}%
\definecolor{currentstroke}{rgb}{0.172549,0.627451,0.172549}%
\pgfsetstrokecolor{currentstroke}%
\pgfsetdash{}{0pt}%
\pgfpathmoveto{\pgfqpoint{2.096296in}{2.350000in}}%
\pgfpathlineto{\pgfqpoint{2.094349in}{2.348200in}}%
\pgfpathlineto{\pgfqpoint{2.092411in}{2.346406in}}%
\pgfpathlineto{\pgfqpoint{2.090485in}{2.344625in}}%
\pgfpathlineto{\pgfqpoint{2.088578in}{2.342861in}}%
\pgfpathlineto{\pgfqpoint{2.086693in}{2.341118in}}%
\pgfpathlineto{\pgfqpoint{2.084837in}{2.339400in}}%
\pgfpathlineto{\pgfqpoint{2.083011in}{2.337712in}}%
\pgfpathlineto{\pgfqpoint{2.081221in}{2.336056in}}%
\pgfpathlineto{\pgfqpoint{2.079468in}{2.334435in}}%
\pgfpathlineto{\pgfqpoint{2.077755in}{2.332850in}}%
\pgfpathlineto{\pgfqpoint{2.076083in}{2.331303in}}%
\pgfpathlineto{\pgfqpoint{2.074452in}{2.329795in}}%
\pgfpathlineto{\pgfqpoint{2.072864in}{2.328325in}}%
\pgfpathlineto{\pgfqpoint{2.071316in}{2.326894in}}%
\pgfpathlineto{\pgfqpoint{2.069810in}{2.325501in}}%
\pgfpathlineto{\pgfqpoint{2.068343in}{2.324144in}}%
\pgfpathlineto{\pgfqpoint{2.066914in}{2.322822in}}%
\pgfpathlineto{\pgfqpoint{2.065521in}{2.321533in}}%
\pgfpathlineto{\pgfqpoint{2.064162in}{2.320277in}}%
\pgfpathlineto{\pgfqpoint{2.062836in}{2.319050in}}%
\pgfpathlineto{\pgfqpoint{2.061540in}{2.317851in}}%
\pgfpathlineto{\pgfqpoint{2.060272in}{2.316679in}}%
\pgfpathlineto{\pgfqpoint{2.059030in}{2.315529in}}%
\pgfpathlineto{\pgfqpoint{2.057811in}{2.314402in}}%
\pgfpathlineto{\pgfqpoint{2.056613in}{2.313293in}}%
\pgfpathlineto{\pgfqpoint{2.055433in}{2.312202in}}%
\pgfpathlineto{\pgfqpoint{2.054270in}{2.311126in}}%
\pgfpathlineto{\pgfqpoint{2.053121in}{2.310063in}}%
\pgfpathlineto{\pgfqpoint{2.051983in}{2.309011in}}%
\pgfpathlineto{\pgfqpoint{2.050855in}{2.307967in}}%
\pgfpathlineto{\pgfqpoint{2.049733in}{2.306930in}}%
\pgfpathlineto{\pgfqpoint{2.048617in}{2.305897in}}%
\pgfpathlineto{\pgfqpoint{2.047502in}{2.304867in}}%
\pgfpathlineto{\pgfqpoint{2.046388in}{2.303836in}}%
\pgfpathlineto{\pgfqpoint{2.045272in}{2.302803in}}%
\pgfpathlineto{\pgfqpoint{2.044150in}{2.301766in}}%
\pgfpathlineto{\pgfqpoint{2.043022in}{2.300722in}}%
\pgfpathlineto{\pgfqpoint{2.041884in}{2.299669in}}%
\pgfpathlineto{\pgfqpoint{2.040733in}{2.298605in}}%
\pgfpathlineto{\pgfqpoint{2.039567in}{2.297526in}}%
\pgfpathlineto{\pgfqpoint{2.038383in}{2.296432in}}%
\pgfpathlineto{\pgfqpoint{2.037179in}{2.295317in}}%
\pgfpathlineto{\pgfqpoint{2.035950in}{2.294181in}}%
\pgfpathlineto{\pgfqpoint{2.034695in}{2.293020in}}%
\pgfpathlineto{\pgfqpoint{2.033408in}{2.291830in}}%
\pgfpathlineto{\pgfqpoint{2.032088in}{2.290608in}}%
\pgfpathlineto{\pgfqpoint{2.030729in}{2.289351in}}%
\pgfpathlineto{\pgfqpoint{2.029328in}{2.288055in}}%
\pgfpathlineto{\pgfqpoint{2.027880in}{2.286716in}}%
\pgfpathlineto{\pgfqpoint{2.026382in}{2.285330in}}%
\pgfpathlineto{\pgfqpoint{2.024827in}{2.283892in}}%
\pgfpathlineto{\pgfqpoint{2.023212in}{2.282398in}}%
\pgfpathlineto{\pgfqpoint{2.021529in}{2.280842in}}%
\pgfpathlineto{\pgfqpoint{2.019775in}{2.279219in}}%
\pgfpathlineto{\pgfqpoint{2.017943in}{2.277524in}}%
\pgfpathlineto{\pgfqpoint{2.016027in}{2.275752in}}%
\pgfpathlineto{\pgfqpoint{2.014021in}{2.273897in}}%
\pgfpathlineto{\pgfqpoint{2.011920in}{2.271953in}}%
\pgfpathlineto{\pgfqpoint{2.009716in}{2.269914in}}%
\pgfpathlineto{\pgfqpoint{2.007405in}{2.267777in}}%
\pgfpathlineto{\pgfqpoint{2.004982in}{2.265535in}}%
\pgfpathlineto{\pgfqpoint{2.002442in}{2.263186in}}%
\pgfpathlineto{\pgfqpoint{1.999781in}{2.260725in}}%
\pgfpathlineto{\pgfqpoint{1.996998in}{2.258150in}}%
\pgfpathlineto{\pgfqpoint{1.994092in}{2.255462in}}%
\pgfpathlineto{\pgfqpoint{1.991063in}{2.252660in}}%
\pgfpathlineto{\pgfqpoint{1.987915in}{2.249748in}}%
\pgfpathlineto{\pgfqpoint{1.984652in}{2.246730in}}%
\pgfpathlineto{\pgfqpoint{1.981281in}{2.243612in}}%
\pgfpathlineto{\pgfqpoint{1.977811in}{2.240402in}}%
\pgfpathlineto{\pgfqpoint{1.974253in}{2.237111in}}%
\pgfpathlineto{\pgfqpoint{1.970620in}{2.233751in}}%
\pgfpathlineto{\pgfqpoint{1.966925in}{2.230333in}}%
\pgfpathlineto{\pgfqpoint{1.963185in}{2.226873in}}%
\pgfpathlineto{\pgfqpoint{1.959414in}{2.223386in}}%
\pgfpathlineto{\pgfqpoint{1.955630in}{2.219885in}}%
\pgfpathlineto{\pgfqpoint{1.951849in}{2.216388in}}%
\pgfpathlineto{\pgfqpoint{1.948088in}{2.212908in}}%
\pgfpathlineto{\pgfqpoint{1.944362in}{2.209462in}}%
\pgfpathlineto{\pgfqpoint{1.940687in}{2.206063in}}%
\pgfpathlineto{\pgfqpoint{1.937077in}{2.202723in}}%
\pgfpathlineto{\pgfqpoint{1.933544in}{2.199456in}}%
\pgfpathlineto{\pgfqpoint{1.930102in}{2.196272in}}%
\pgfpathlineto{\pgfqpoint{1.926758in}{2.193179in}}%
\pgfpathlineto{\pgfqpoint{1.923523in}{2.190186in}}%
\pgfpathlineto{\pgfqpoint{1.920402in}{2.187300in}}%
\pgfpathlineto{\pgfqpoint{1.917401in}{2.184523in}}%
\pgfpathlineto{\pgfqpoint{1.914523in}{2.181861in}}%
\pgfpathlineto{\pgfqpoint{1.911769in}{2.179314in}}%
\pgfpathlineto{\pgfqpoint{1.909142in}{2.176884in}}%
\pgfpathlineto{\pgfqpoint{1.906638in}{2.174568in}}%
\pgfpathlineto{\pgfqpoint{1.904258in}{2.172366in}}%
\pgfpathlineto{\pgfqpoint{1.901997in}{2.170275in}}%
\pgfpathlineto{\pgfqpoint{1.899852in}{2.168291in}}%
\pgfpathlineto{\pgfqpoint{1.897819in}{2.166410in}}%
\pgfpathlineto{\pgfqpoint{1.895892in}{2.164628in}}%
\pgfpathlineto{\pgfqpoint{1.894068in}{2.162941in}}%
\pgfpathlineto{\pgfqpoint{1.892340in}{2.161342in}}%
\pgfpathlineto{\pgfqpoint{1.890703in}{2.159828in}}%
\pgfpathlineto{\pgfqpoint{1.889152in}{2.158393in}}%
\pgfusepath{stroke}%
\end{pgfscope}%
\begin{pgfscope}%
\pgfpathrectangle{\pgfqpoint{0.150000in}{0.549691in}}{\pgfqpoint{1.946296in}{1.800309in}}%
\pgfusepath{clip}%
\pgfsetrectcap%
\pgfsetroundjoin%
\pgfsetlinewidth{2.007500pt}%
\definecolor{currentstroke}{rgb}{0.121569,0.466667,0.705882}%
\pgfsetstrokecolor{currentstroke}%
\pgfsetdash{}{0pt}%
\pgfpathmoveto{\pgfqpoint{2.096296in}{2.350000in}}%
\pgfpathlineto{\pgfqpoint{2.094349in}{2.348200in}}%
\pgfpathlineto{\pgfqpoint{2.093003in}{2.346954in}}%
\pgfpathlineto{\pgfqpoint{2.091918in}{2.345950in}}%
\pgfpathlineto{\pgfqpoint{2.090988in}{2.345091in}}%
\pgfpathlineto{\pgfqpoint{2.090164in}{2.344328in}}%
\pgfpathlineto{\pgfqpoint{2.089418in}{2.343638in}}%
\pgfpathlineto{\pgfqpoint{2.088732in}{2.343004in}}%
\pgfpathlineto{\pgfqpoint{2.088095in}{2.342414in}}%
\pgfpathlineto{\pgfqpoint{2.087498in}{2.341862in}}%
\pgfpathlineto{\pgfqpoint{2.086935in}{2.341341in}}%
\pgfpathlineto{\pgfqpoint{2.086401in}{2.340848in}}%
\pgfpathlineto{\pgfqpoint{2.085893in}{2.340377in}}%
\pgfpathlineto{\pgfqpoint{2.085407in}{2.339928in}}%
\pgfpathlineto{\pgfqpoint{2.084941in}{2.339497in}}%
\pgfpathlineto{\pgfqpoint{2.084494in}{2.339083in}}%
\pgfpathlineto{\pgfqpoint{2.084062in}{2.338684in}}%
\pgfpathlineto{\pgfqpoint{2.083645in}{2.338298in}}%
\pgfpathlineto{\pgfqpoint{2.083242in}{2.337925in}}%
\pgfpathlineto{\pgfqpoint{2.082851in}{2.337564in}}%
\pgfpathlineto{\pgfqpoint{2.082472in}{2.337213in}}%
\pgfpathlineto{\pgfqpoint{2.082103in}{2.336872in}}%
\pgfpathlineto{\pgfqpoint{2.081744in}{2.336540in}}%
\pgfpathlineto{\pgfqpoint{2.081395in}{2.336217in}}%
\pgfpathlineto{\pgfqpoint{2.081054in}{2.335902in}}%
\pgfpathlineto{\pgfqpoint{2.080721in}{2.335594in}}%
\pgfpathlineto{\pgfqpoint{2.080396in}{2.335293in}}%
\pgfpathlineto{\pgfqpoint{2.080079in}{2.334999in}}%
\pgfpathlineto{\pgfqpoint{2.079768in}{2.334712in}}%
\pgfpathlineto{\pgfqpoint{2.079463in}{2.334430in}}%
\pgfpathlineto{\pgfqpoint{2.079165in}{2.334154in}}%
\pgfpathlineto{\pgfqpoint{2.078873in}{2.333884in}}%
\pgfpathlineto{\pgfqpoint{2.078586in}{2.333618in}}%
\pgfpathlineto{\pgfqpoint{2.078304in}{2.333358in}}%
\pgfpathlineto{\pgfqpoint{2.078027in}{2.333102in}}%
\pgfpathlineto{\pgfqpoint{2.077756in}{2.332850in}}%
\pgfpathlineto{\pgfqpoint{2.077489in}{2.332603in}}%
\pgfpathlineto{\pgfqpoint{2.077226in}{2.332360in}}%
\pgfpathlineto{\pgfqpoint{2.076968in}{2.332121in}}%
\pgfpathlineto{\pgfqpoint{2.076713in}{2.331886in}}%
\pgfpathlineto{\pgfqpoint{2.076463in}{2.331655in}}%
\pgfpathlineto{\pgfqpoint{2.076216in}{2.331427in}}%
\pgfpathlineto{\pgfqpoint{2.075974in}{2.331202in}}%
\pgfpathlineto{\pgfqpoint{2.075734in}{2.330981in}}%
\pgfpathlineto{\pgfqpoint{2.075498in}{2.330762in}}%
\pgfpathlineto{\pgfqpoint{2.075265in}{2.330547in}}%
\pgfpathlineto{\pgfqpoint{2.075036in}{2.330335in}}%
\pgfpathlineto{\pgfqpoint{2.074809in}{2.330125in}}%
\pgfpathlineto{\pgfqpoint{2.074586in}{2.329919in}}%
\pgfpathlineto{\pgfqpoint{2.074365in}{2.329714in}}%
\pgfpathlineto{\pgfqpoint{2.074147in}{2.329513in}}%
\pgfpathlineto{\pgfqpoint{2.073932in}{2.329314in}}%
\pgfpathlineto{\pgfqpoint{2.073720in}{2.329117in}}%
\pgfpathlineto{\pgfqpoint{2.073510in}{2.328923in}}%
\pgfpathlineto{\pgfqpoint{2.073302in}{2.328731in}}%
\pgfpathlineto{\pgfqpoint{2.073097in}{2.328541in}}%
\pgfpathlineto{\pgfqpoint{2.072894in}{2.328353in}}%
\pgfpathlineto{\pgfqpoint{2.072693in}{2.328168in}}%
\pgfpathlineto{\pgfqpoint{2.072495in}{2.327984in}}%
\pgfpathlineto{\pgfqpoint{2.072299in}{2.327803in}}%
\pgfpathlineto{\pgfqpoint{2.072105in}{2.327623in}}%
\pgfpathlineto{\pgfqpoint{2.071912in}{2.327446in}}%
\pgfpathlineto{\pgfqpoint{2.071722in}{2.327270in}}%
\pgfpathlineto{\pgfqpoint{2.071534in}{2.327096in}}%
\pgfpathlineto{\pgfqpoint{2.071348in}{2.326923in}}%
\pgfpathlineto{\pgfqpoint{2.071163in}{2.326753in}}%
\pgfpathlineto{\pgfqpoint{2.070981in}{2.326584in}}%
\pgfpathlineto{\pgfqpoint{2.070800in}{2.326416in}}%
\pgfpathlineto{\pgfqpoint{2.070620in}{2.326250in}}%
\pgfpathlineto{\pgfqpoint{2.070443in}{2.326086in}}%
\pgfpathlineto{\pgfqpoint{2.070267in}{2.325923in}}%
\pgfpathlineto{\pgfqpoint{2.070092in}{2.325762in}}%
\pgfpathlineto{\pgfqpoint{2.069920in}{2.325602in}}%
\pgfpathlineto{\pgfqpoint{2.069748in}{2.325444in}}%
\pgfpathlineto{\pgfqpoint{2.069579in}{2.325287in}}%
\pgfpathlineto{\pgfqpoint{2.069410in}{2.325131in}}%
\pgfpathlineto{\pgfqpoint{2.069243in}{2.324977in}}%
\pgfpathlineto{\pgfqpoint{2.069078in}{2.324824in}}%
\pgfpathlineto{\pgfqpoint{2.068914in}{2.324672in}}%
\pgfpathlineto{\pgfqpoint{2.068751in}{2.324521in}}%
\pgfpathlineto{\pgfqpoint{2.068590in}{2.324372in}}%
\pgfpathlineto{\pgfqpoint{2.068429in}{2.324224in}}%
\pgfpathlineto{\pgfqpoint{2.068271in}{2.324077in}}%
\pgfpathlineto{\pgfqpoint{2.068113in}{2.323931in}}%
\pgfpathlineto{\pgfqpoint{2.067956in}{2.323786in}}%
\pgfpathlineto{\pgfqpoint{2.067801in}{2.323643in}}%
\pgfpathlineto{\pgfqpoint{2.067647in}{2.323500in}}%
\pgfpathlineto{\pgfqpoint{2.067494in}{2.323359in}}%
\pgfpathlineto{\pgfqpoint{2.067342in}{2.323218in}}%
\pgfpathlineto{\pgfqpoint{2.067191in}{2.323079in}}%
\pgfpathlineto{\pgfqpoint{2.067042in}{2.322940in}}%
\pgfpathlineto{\pgfqpoint{2.066893in}{2.322803in}}%
\pgfpathlineto{\pgfqpoint{2.066746in}{2.322666in}}%
\pgfpathlineto{\pgfqpoint{2.066599in}{2.322531in}}%
\pgfpathlineto{\pgfqpoint{2.066454in}{2.322396in}}%
\pgfpathlineto{\pgfqpoint{2.066309in}{2.322262in}}%
\pgfpathlineto{\pgfqpoint{2.066166in}{2.322130in}}%
\pgfpathlineto{\pgfqpoint{2.066023in}{2.321998in}}%
\pgfpathlineto{\pgfqpoint{2.065881in}{2.321867in}}%
\pgfpathlineto{\pgfqpoint{2.065740in}{2.321736in}}%
\pgfpathlineto{\pgfqpoint{2.065601in}{2.321607in}}%
\pgfusepath{stroke}%
\end{pgfscope}%
\begin{pgfscope}%
\pgfsetrectcap%
\pgfsetmiterjoin%
\pgfsetlinewidth{0.803000pt}%
\definecolor{currentstroke}{rgb}{0.000000,0.000000,0.000000}%
\pgfsetstrokecolor{currentstroke}%
\pgfsetdash{}{0pt}%
\pgfpathmoveto{\pgfqpoint{0.150000in}{0.549691in}}%
\pgfpathlineto{\pgfqpoint{0.150000in}{2.350000in}}%
\pgfusepath{stroke}%
\end{pgfscope}%
\begin{pgfscope}%
\pgfsetrectcap%
\pgfsetmiterjoin%
\pgfsetlinewidth{0.803000pt}%
\definecolor{currentstroke}{rgb}{0.000000,0.000000,0.000000}%
\pgfsetstrokecolor{currentstroke}%
\pgfsetdash{}{0pt}%
\pgfpathmoveto{\pgfqpoint{2.096296in}{0.549691in}}%
\pgfpathlineto{\pgfqpoint{2.096296in}{2.350000in}}%
\pgfusepath{stroke}%
\end{pgfscope}%
\begin{pgfscope}%
\pgfsetrectcap%
\pgfsetmiterjoin%
\pgfsetlinewidth{0.803000pt}%
\definecolor{currentstroke}{rgb}{0.000000,0.000000,0.000000}%
\pgfsetstrokecolor{currentstroke}%
\pgfsetdash{}{0pt}%
\pgfpathmoveto{\pgfqpoint{0.150000in}{0.549691in}}%
\pgfpathlineto{\pgfqpoint{2.096296in}{0.549691in}}%
\pgfusepath{stroke}%
\end{pgfscope}%
\begin{pgfscope}%
\pgfsetrectcap%
\pgfsetmiterjoin%
\pgfsetlinewidth{0.803000pt}%
\definecolor{currentstroke}{rgb}{0.000000,0.000000,0.000000}%
\pgfsetstrokecolor{currentstroke}%
\pgfsetdash{}{0pt}%
\pgfpathmoveto{\pgfqpoint{0.150000in}{2.350000in}}%
\pgfpathlineto{\pgfqpoint{2.096296in}{2.350000in}}%
\pgfusepath{stroke}%
\end{pgfscope}%
\begin{pgfscope}%
\pgfsetbuttcap%
\pgfsetmiterjoin%
\definecolor{currentfill}{rgb}{1.000000,1.000000,1.000000}%
\pgfsetfillcolor{currentfill}%
\pgfsetfillopacity{0.800000}%
\pgfsetlinewidth{1.003750pt}%
\definecolor{currentstroke}{rgb}{0.800000,0.800000,0.800000}%
\pgfsetstrokecolor{currentstroke}%
\pgfsetstrokeopacity{0.800000}%
\pgfsetdash{}{0pt}%
\pgfpathmoveto{\pgfqpoint{1.214711in}{0.605247in}}%
\pgfpathlineto{\pgfqpoint{2.018518in}{0.605247in}}%
\pgfpathquadraticcurveto{\pgfqpoint{2.040740in}{0.605247in}}{\pgfqpoint{2.040740in}{0.627469in}}%
\pgfpathlineto{\pgfqpoint{2.040740in}{1.236111in}}%
\pgfpathquadraticcurveto{\pgfqpoint{2.040740in}{1.258333in}}{\pgfqpoint{2.018518in}{1.258333in}}%
\pgfpathlineto{\pgfqpoint{1.214711in}{1.258333in}}%
\pgfpathquadraticcurveto{\pgfqpoint{1.192489in}{1.258333in}}{\pgfqpoint{1.192489in}{1.236111in}}%
\pgfpathlineto{\pgfqpoint{1.192489in}{0.627469in}}%
\pgfpathquadraticcurveto{\pgfqpoint{1.192489in}{0.605247in}}{\pgfqpoint{1.214711in}{0.605247in}}%
\pgfpathclose%
\pgfusepath{stroke,fill}%
\end{pgfscope}%
\begin{pgfscope}%
\pgfsetrectcap%
\pgfsetroundjoin%
\pgfsetlinewidth{2.007500pt}%
\definecolor{currentstroke}{rgb}{1.000000,0.498039,0.054902}%
\pgfsetstrokecolor{currentstroke}%
\pgfsetdash{}{0pt}%
\pgfpathmoveto{\pgfqpoint{1.236934in}{1.175000in}}%
\pgfpathlineto{\pgfqpoint{1.459156in}{1.175000in}}%
\pgfusepath{stroke}%
\end{pgfscope}%
\begin{pgfscope}%
\definecolor{textcolor}{rgb}{0.000000,0.000000,0.000000}%
\pgfsetstrokecolor{textcolor}%
\pgfsetfillcolor{textcolor}%
\pgftext[x=1.548045in,y=1.136111in,left,base]{\color{textcolor}\rmfamily\fontsize{8.000000}{9.600000}\selectfont SGD+M}%
\end{pgfscope}%
\begin{pgfscope}%
\pgfsetrectcap%
\pgfsetroundjoin%
\pgfsetlinewidth{2.007500pt}%
\definecolor{currentstroke}{rgb}{0.839216,0.152941,0.156863}%
\pgfsetstrokecolor{currentstroke}%
\pgfsetdash{}{0pt}%
\pgfpathmoveto{\pgfqpoint{1.236934in}{1.020062in}}%
\pgfpathlineto{\pgfqpoint{1.459156in}{1.020062in}}%
\pgfusepath{stroke}%
\end{pgfscope}%
\begin{pgfscope}%
\definecolor{textcolor}{rgb}{0.000000,0.000000,0.000000}%
\pgfsetstrokecolor{textcolor}%
\pgfsetfillcolor{textcolor}%
\pgftext[x=1.548045in,y=0.981173in,left,base]{\color{textcolor}\rmfamily\fontsize{8.000000}{9.600000}\selectfont SGD}%
\end{pgfscope}%
\begin{pgfscope}%
\pgfsetrectcap%
\pgfsetroundjoin%
\pgfsetlinewidth{2.007500pt}%
\definecolor{currentstroke}{rgb}{0.172549,0.627451,0.172549}%
\pgfsetstrokecolor{currentstroke}%
\pgfsetdash{}{0pt}%
\pgfpathmoveto{\pgfqpoint{1.236934in}{0.865123in}}%
\pgfpathlineto{\pgfqpoint{1.459156in}{0.865123in}}%
\pgfusepath{stroke}%
\end{pgfscope}%
\begin{pgfscope}%
\definecolor{textcolor}{rgb}{0.000000,0.000000,0.000000}%
\pgfsetstrokecolor{textcolor}%
\pgfsetfillcolor{textcolor}%
\pgftext[x=1.548045in,y=0.826234in,left,base]{\color{textcolor}\rmfamily\fontsize{8.000000}{9.600000}\selectfont Adam}%
\end{pgfscope}%
\begin{pgfscope}%
\pgfsetrectcap%
\pgfsetroundjoin%
\pgfsetlinewidth{2.007500pt}%
\definecolor{currentstroke}{rgb}{0.121569,0.466667,0.705882}%
\pgfsetstrokecolor{currentstroke}%
\pgfsetdash{}{0pt}%
\pgfpathmoveto{\pgfqpoint{1.236934in}{0.710185in}}%
\pgfpathlineto{\pgfqpoint{1.459156in}{0.710185in}}%
\pgfusepath{stroke}%
\end{pgfscope}%
\begin{pgfscope}%
\definecolor{textcolor}{rgb}{0.000000,0.000000,0.000000}%
\pgfsetstrokecolor{textcolor}%
\pgfsetfillcolor{textcolor}%
\pgftext[x=1.548045in,y=0.671296in,left,base]{\color{textcolor}\rmfamily\fontsize{8.000000}{9.600000}\selectfont Adagrad}%
\end{pgfscope}%
\end{pgfpicture}%
\makeatother%
\endgroup%
}
    \end{tabular}
    \caption{Optimisation of the Rastrigin function (A=1).}
    \label{fig:pca_sg}
\end{figure}
\noindent Stochastic Gradient Descent with Momentum (SGD+M) converged to the smallest local minima and closest to the global minima; therefore it is considered to perform the best. However, it should be noted that all optimisers got stuck at non-global minima.


\section{Exercise 2}
\textbf{Exercise 2.1:} SVM optimisation on Iris Dataset

\noindent Below are the median validation accuracies for classifying the \text{Iris Versicolor} and \text{Iris Virginica} classes of the Iris Dataset across 100 independent trainings.

\begin{equation*}
\text{Acc}_\text{SGD} = 0.96\quad
\text{Acc}_\text{Adam} = 0.92\quad
\text{Acc}_\text{Random} = 0.40
\end{equation*}

\noindent As expected the accuracies do not reach 100\%, this can be attributed to the classes not being linearly separable. Counter-intuitively random classification is less than 50\%; this may be attributed to the different element count of each class in the validation set.



\end{document}

