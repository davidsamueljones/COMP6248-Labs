\documentclass[11pt,a4paper]{article}
\usepackage[top=1.2cm, bottom=1.8cm, left=1.8cm, right=1.8cm]{geometry}

\usepackage{float}
\usepackage{subfig}
\usepackage{graphicx}
\usepackage{xcolor}
\usepackage[utf8]{inputenc}
\usepackage{enumitem}
\usepackage{siunitx}
\usepackage[newfloat]{minted}
\usepackage{caption}
\usepackage{amsmath}
\usepackage{amsfonts,amssymb}
\usepackage[makeroom]{cancel}

% Declarations for tikz drawings
\usepackage{tikz}
\usepackage{pgfplots}
\usetikzlibrary{calc}
\definecolor{lightgreen}{HTML}{90EE90}
\newcommand*{\boxcolor}{lightgreen}
\makeatletter
\renewcommand{\boxed}[1]{\textcolor{\boxcolor}{%
\tikz[baseline={([yshift=-1ex]current bounding box.center)}] \node [rectangle, minimum width=5ex,rounded corners,draw,line width=0.25mm] {\normalcolor\m@th$\displaystyle#1$};}}
 \makeatother

 % Fix for symbol errors in code listings (see https://tex.stackexchange.com/a/343506)
 \usepackage{etoolbox,xpatch}
 \makeatletter
 \AtBeginEnvironment{minted}{\dontdofcolorbox}
 \def\dontdofcolorbox{\renewcommand\fcolorbox[4][]{##4}}
 \xpatchcmd{\inputminted}{\minted@fvset}{\minted@fvset\dontdofcolorbox}{}{}
 \xpatchcmd{\mintinline}{\minted@fvset}{\minted@fvset\dontdofcolorbox}{}{}
 \makeatother
 % Fix for distance of captions from listings
 \captionsetup[listing]{skip=-10pt}

% \usepackage[style=authoryear, backend=biber]{biblatex}
% \addbibresource{main.bib}
\DeclareMathOperator*{\argmin}{arg\,min}
\DeclareMathOperator*{\argmax}{arg\,max}

\title{COMP6248: Lab Exercise 2}
\author{
David Jones (dsj1n15@soton.ac.uk)}
\date{}
\setlength{\intextsep}{1mm}

\definecolor{mintedbackground}{rgb}{0.95,0.95,0.95}
\newmintedfile[pythoncode]{python}{
    bgcolor=mintedbackground,
    style=friendly,
    % fontfamily=fi4,
    fontsize=\small,
    linenos=true,
    numberblanklines=true,
    numbersep=5pt,
    gobble=0,
    frame=leftline,
    framerule=0.4pt,
    framesep=2mm,
    funcnamehighlighting=true,
    tabsize=4,
    obeytabs=false,
    mathescape=false
    samepage=false,
    showspaces=false,
    showtabs =false,
    texcl=false,
}

\newcommand{\norm}[1]{\left\lVert#1\right\rVert}

\begin{document}

\maketitle
\textbf{Task:} PyTorch Autograd
\vspace{-0.5em}
\section{Exercise 1}
\textbf{Exercise 1.1:} Implementation of matrix factorisation using gradient descent.
\begin{listing}[H]
\pythoncode{code/gd_factorise_ad.py}
\caption{Matrix factorisation using gradient descent with PyTorch’s AD.}
\label{lst:gd}
\end{listing}
\noindent\textbf{Exercise 1.2:} Rank 2 reconstruction loss for Iris Dataset.\\
\vspace{-1.5em}
\begin{alignat*}{1}
    &\textbf{A} = \text{Iris Dataset} - \mu \\
    \text{loss}_{\text{GD}} =\norm{\mathbf{A}-\mathbf{\hat{A}}_{\text{GD}}}^2_\text{F}=15.232\quad
    &\text{loss}_{\text{SVD}} =\norm{\mathbf{A}-\mathbf{\hat{A}}_{\text{SVD}}}^2_\text{F}=15.229
\end{alignat*}
As expected $\text{loss}_{\text{SVD}}$ is less than $\text{loss}_{\text{GD}}$, yet $\mathbf{\hat{A}}_\text{GD}$ is a close approximation of the optimum.


\vspace{1em}
\noindent\textbf{Exercise 1.3:} PCA vs GD Rank Reduction.\\
\begin{figure}[H]
    \centering
    \begin{tabular}{cc}
    \subfloat[PCA]{%% Creator: Matplotlib, PGF backend
%%
%% To include the figure in your LaTeX document, write
%%   \input{<filename>.pgf}
%%
%% Make sure the required packages are loaded in your preamble
%%   \usepackage{pgf}
%%
%% Figures using additional raster images can only be included by \input if
%% they are in the same directory as the main LaTeX file. For loading figures
%% from other directories you can use the `import` package
%%   \usepackage{import}
%% and then include the figures with
%%   \import{<path to file>}{<filename>.pgf}
%%
%% Matplotlib used the following preamble
%%
\begingroup%
\makeatletter%
\begin{pgfpicture}%
\pgfpathrectangle{\pgfpointorigin}{\pgfqpoint{3.300000in}{2.000000in}}%
\pgfusepath{use as bounding box, clip}%
\begin{pgfscope}%
\pgfsetbuttcap%
\pgfsetmiterjoin%
\definecolor{currentfill}{rgb}{1.000000,1.000000,1.000000}%
\pgfsetfillcolor{currentfill}%
\pgfsetlinewidth{0.000000pt}%
\definecolor{currentstroke}{rgb}{1.000000,1.000000,1.000000}%
\pgfsetstrokecolor{currentstroke}%
\pgfsetdash{}{0pt}%
\pgfpathmoveto{\pgfqpoint{0.000000in}{0.000000in}}%
\pgfpathlineto{\pgfqpoint{3.300000in}{0.000000in}}%
\pgfpathlineto{\pgfqpoint{3.300000in}{2.000000in}}%
\pgfpathlineto{\pgfqpoint{0.000000in}{2.000000in}}%
\pgfpathclose%
\pgfusepath{fill}%
\end{pgfscope}%
\begin{pgfscope}%
\pgfsetbuttcap%
\pgfsetmiterjoin%
\definecolor{currentfill}{rgb}{1.000000,1.000000,1.000000}%
\pgfsetfillcolor{currentfill}%
\pgfsetlinewidth{0.000000pt}%
\definecolor{currentstroke}{rgb}{0.000000,0.000000,0.000000}%
\pgfsetstrokecolor{currentstroke}%
\pgfsetstrokeopacity{0.000000}%
\pgfsetdash{}{0pt}%
\pgfpathmoveto{\pgfqpoint{0.532717in}{0.370679in}}%
\pgfpathlineto{\pgfqpoint{3.150000in}{0.370679in}}%
\pgfpathlineto{\pgfqpoint{3.150000in}{1.850000in}}%
\pgfpathlineto{\pgfqpoint{0.532717in}{1.850000in}}%
\pgfpathclose%
\pgfusepath{fill}%
\end{pgfscope}%
\begin{pgfscope}%
\pgfpathrectangle{\pgfqpoint{0.532717in}{0.370679in}}{\pgfqpoint{2.617283in}{1.479321in}}%
\pgfusepath{clip}%
\pgfsetbuttcap%
\pgfsetroundjoin%
\definecolor{currentfill}{rgb}{1.000000,0.411765,0.380392}%
\pgfsetfillcolor{currentfill}%
\pgfsetfillopacity{0.500000}%
\pgfsetlinewidth{1.003750pt}%
\definecolor{currentstroke}{rgb}{1.000000,0.411765,0.380392}%
\pgfsetstrokecolor{currentstroke}%
\pgfsetstrokeopacity{0.500000}%
\pgfsetdash{}{0pt}%
\pgfpathmoveto{\pgfqpoint{0.968267in}{0.947046in}}%
\pgfpathcurveto{\pgfqpoint{0.979317in}{0.947046in}}{\pgfqpoint{0.989916in}{0.951437in}}{\pgfqpoint{0.997730in}{0.959250in}}%
\pgfpathcurveto{\pgfqpoint{1.005543in}{0.967064in}}{\pgfqpoint{1.009933in}{0.977663in}}{\pgfqpoint{1.009933in}{0.988713in}}%
\pgfpathcurveto{\pgfqpoint{1.009933in}{0.999763in}}{\pgfqpoint{1.005543in}{1.010362in}}{\pgfqpoint{0.997730in}{1.018176in}}%
\pgfpathcurveto{\pgfqpoint{0.989916in}{1.025989in}}{\pgfqpoint{0.979317in}{1.030380in}}{\pgfqpoint{0.968267in}{1.030380in}}%
\pgfpathcurveto{\pgfqpoint{0.957217in}{1.030380in}}{\pgfqpoint{0.946618in}{1.025989in}}{\pgfqpoint{0.938804in}{1.018176in}}%
\pgfpathcurveto{\pgfqpoint{0.930990in}{1.010362in}}{\pgfqpoint{0.926600in}{0.999763in}}{\pgfqpoint{0.926600in}{0.988713in}}%
\pgfpathcurveto{\pgfqpoint{0.926600in}{0.977663in}}{\pgfqpoint{0.930990in}{0.967064in}}{\pgfqpoint{0.938804in}{0.959250in}}%
\pgfpathcurveto{\pgfqpoint{0.946618in}{0.951437in}}{\pgfqpoint{0.957217in}{0.947046in}}{\pgfqpoint{0.968267in}{0.947046in}}%
\pgfpathclose%
\pgfusepath{stroke,fill}%
\end{pgfscope}%
\begin{pgfscope}%
\pgfpathrectangle{\pgfqpoint{0.532717in}{0.370679in}}{\pgfqpoint{2.617283in}{1.479321in}}%
\pgfusepath{clip}%
\pgfsetbuttcap%
\pgfsetroundjoin%
\definecolor{currentfill}{rgb}{1.000000,0.411765,0.380392}%
\pgfsetfillcolor{currentfill}%
\pgfsetfillopacity{0.500000}%
\pgfsetlinewidth{1.003750pt}%
\definecolor{currentstroke}{rgb}{1.000000,0.411765,0.380392}%
\pgfsetstrokecolor{currentstroke}%
\pgfsetstrokeopacity{0.500000}%
\pgfsetdash{}{0pt}%
\pgfpathmoveto{\pgfqpoint{0.958890in}{1.174142in}}%
\pgfpathcurveto{\pgfqpoint{0.969940in}{1.174142in}}{\pgfqpoint{0.980539in}{1.178532in}}{\pgfqpoint{0.988352in}{1.186346in}}%
\pgfpathcurveto{\pgfqpoint{0.996166in}{1.194159in}}{\pgfqpoint{1.000556in}{1.204758in}}{\pgfqpoint{1.000556in}{1.215808in}}%
\pgfpathcurveto{\pgfqpoint{1.000556in}{1.226859in}}{\pgfqpoint{0.996166in}{1.237458in}}{\pgfqpoint{0.988352in}{1.245271in}}%
\pgfpathcurveto{\pgfqpoint{0.980539in}{1.253085in}}{\pgfqpoint{0.969940in}{1.257475in}}{\pgfqpoint{0.958890in}{1.257475in}}%
\pgfpathcurveto{\pgfqpoint{0.947839in}{1.257475in}}{\pgfqpoint{0.937240in}{1.253085in}}{\pgfqpoint{0.929427in}{1.245271in}}%
\pgfpathcurveto{\pgfqpoint{0.921613in}{1.237458in}}{\pgfqpoint{0.917223in}{1.226859in}}{\pgfqpoint{0.917223in}{1.215808in}}%
\pgfpathcurveto{\pgfqpoint{0.917223in}{1.204758in}}{\pgfqpoint{0.921613in}{1.194159in}}{\pgfqpoint{0.929427in}{1.186346in}}%
\pgfpathcurveto{\pgfqpoint{0.937240in}{1.178532in}}{\pgfqpoint{0.947839in}{1.174142in}}{\pgfqpoint{0.958890in}{1.174142in}}%
\pgfpathclose%
\pgfusepath{stroke,fill}%
\end{pgfscope}%
\begin{pgfscope}%
\pgfpathrectangle{\pgfqpoint{0.532717in}{0.370679in}}{\pgfqpoint{2.617283in}{1.479321in}}%
\pgfusepath{clip}%
\pgfsetbuttcap%
\pgfsetroundjoin%
\definecolor{currentfill}{rgb}{1.000000,0.411765,0.380392}%
\pgfsetfillcolor{currentfill}%
\pgfsetfillopacity{0.500000}%
\pgfsetlinewidth{1.003750pt}%
\definecolor{currentstroke}{rgb}{1.000000,0.411765,0.380392}%
\pgfsetstrokecolor{currentstroke}%
\pgfsetstrokeopacity{0.500000}%
\pgfsetdash{}{0pt}%
\pgfpathmoveto{\pgfqpoint{0.906437in}{1.159399in}}%
\pgfpathcurveto{\pgfqpoint{0.917487in}{1.159399in}}{\pgfqpoint{0.928086in}{1.163789in}}{\pgfqpoint{0.935900in}{1.171603in}}%
\pgfpathcurveto{\pgfqpoint{0.943713in}{1.179416in}}{\pgfqpoint{0.948104in}{1.190015in}}{\pgfqpoint{0.948104in}{1.201065in}}%
\pgfpathcurveto{\pgfqpoint{0.948104in}{1.212115in}}{\pgfqpoint{0.943713in}{1.222714in}}{\pgfqpoint{0.935900in}{1.230528in}}%
\pgfpathcurveto{\pgfqpoint{0.928086in}{1.238342in}}{\pgfqpoint{0.917487in}{1.242732in}}{\pgfqpoint{0.906437in}{1.242732in}}%
\pgfpathcurveto{\pgfqpoint{0.895387in}{1.242732in}}{\pgfqpoint{0.884788in}{1.238342in}}{\pgfqpoint{0.876974in}{1.230528in}}%
\pgfpathcurveto{\pgfqpoint{0.869161in}{1.222714in}}{\pgfqpoint{0.864770in}{1.212115in}}{\pgfqpoint{0.864770in}{1.201065in}}%
\pgfpathcurveto{\pgfqpoint{0.864770in}{1.190015in}}{\pgfqpoint{0.869161in}{1.179416in}}{\pgfqpoint{0.876974in}{1.171603in}}%
\pgfpathcurveto{\pgfqpoint{0.884788in}{1.163789in}}{\pgfqpoint{0.895387in}{1.159399in}}{\pgfqpoint{0.906437in}{1.159399in}}%
\pgfpathclose%
\pgfusepath{stroke,fill}%
\end{pgfscope}%
\begin{pgfscope}%
\pgfpathrectangle{\pgfqpoint{0.532717in}{0.370679in}}{\pgfqpoint{2.617283in}{1.479321in}}%
\pgfusepath{clip}%
\pgfsetbuttcap%
\pgfsetroundjoin%
\definecolor{currentfill}{rgb}{1.000000,0.411765,0.380392}%
\pgfsetfillcolor{currentfill}%
\pgfsetfillopacity{0.500000}%
\pgfsetlinewidth{1.003750pt}%
\definecolor{currentstroke}{rgb}{1.000000,0.411765,0.380392}%
\pgfsetstrokecolor{currentstroke}%
\pgfsetstrokeopacity{0.500000}%
\pgfsetdash{}{0pt}%
\pgfpathmoveto{\pgfqpoint{0.949553in}{1.238938in}}%
\pgfpathcurveto{\pgfqpoint{0.960604in}{1.238938in}}{\pgfqpoint{0.971203in}{1.243328in}}{\pgfqpoint{0.979016in}{1.251141in}}%
\pgfpathcurveto{\pgfqpoint{0.986830in}{1.258955in}}{\pgfqpoint{0.991220in}{1.269554in}}{\pgfqpoint{0.991220in}{1.280604in}}%
\pgfpathcurveto{\pgfqpoint{0.991220in}{1.291654in}}{\pgfqpoint{0.986830in}{1.302253in}}{\pgfqpoint{0.979016in}{1.310067in}}%
\pgfpathcurveto{\pgfqpoint{0.971203in}{1.317881in}}{\pgfqpoint{0.960604in}{1.322271in}}{\pgfqpoint{0.949553in}{1.322271in}}%
\pgfpathcurveto{\pgfqpoint{0.938503in}{1.322271in}}{\pgfqpoint{0.927904in}{1.317881in}}{\pgfqpoint{0.920091in}{1.310067in}}%
\pgfpathcurveto{\pgfqpoint{0.912277in}{1.302253in}}{\pgfqpoint{0.907887in}{1.291654in}}{\pgfqpoint{0.907887in}{1.280604in}}%
\pgfpathcurveto{\pgfqpoint{0.907887in}{1.269554in}}{\pgfqpoint{0.912277in}{1.258955in}}{\pgfqpoint{0.920091in}{1.251141in}}%
\pgfpathcurveto{\pgfqpoint{0.927904in}{1.243328in}}{\pgfqpoint{0.938503in}{1.238938in}}{\pgfqpoint{0.949553in}{1.238938in}}%
\pgfpathclose%
\pgfusepath{stroke,fill}%
\end{pgfscope}%
\begin{pgfscope}%
\pgfpathrectangle{\pgfqpoint{0.532717in}{0.370679in}}{\pgfqpoint{2.617283in}{1.479321in}}%
\pgfusepath{clip}%
\pgfsetbuttcap%
\pgfsetroundjoin%
\definecolor{currentfill}{rgb}{1.000000,0.411765,0.380392}%
\pgfsetfillcolor{currentfill}%
\pgfsetfillopacity{0.500000}%
\pgfsetlinewidth{1.003750pt}%
\definecolor{currentstroke}{rgb}{1.000000,0.411765,0.380392}%
\pgfsetstrokecolor{currentstroke}%
\pgfsetstrokeopacity{0.500000}%
\pgfsetdash{}{0pt}%
\pgfpathmoveto{\pgfqpoint{0.954919in}{0.943697in}}%
\pgfpathcurveto{\pgfqpoint{0.965970in}{0.943697in}}{\pgfqpoint{0.976569in}{0.948087in}}{\pgfqpoint{0.984382in}{0.955901in}}%
\pgfpathcurveto{\pgfqpoint{0.992196in}{0.963715in}}{\pgfqpoint{0.996586in}{0.974314in}}{\pgfqpoint{0.996586in}{0.985364in}}%
\pgfpathcurveto{\pgfqpoint{0.996586in}{0.996414in}}{\pgfqpoint{0.992196in}{1.007013in}}{\pgfqpoint{0.984382in}{1.014827in}}%
\pgfpathcurveto{\pgfqpoint{0.976569in}{1.022640in}}{\pgfqpoint{0.965970in}{1.027030in}}{\pgfqpoint{0.954919in}{1.027030in}}%
\pgfpathcurveto{\pgfqpoint{0.943869in}{1.027030in}}{\pgfqpoint{0.933270in}{1.022640in}}{\pgfqpoint{0.925457in}{1.014827in}}%
\pgfpathcurveto{\pgfqpoint{0.917643in}{1.007013in}}{\pgfqpoint{0.913253in}{0.996414in}}{\pgfqpoint{0.913253in}{0.985364in}}%
\pgfpathcurveto{\pgfqpoint{0.913253in}{0.974314in}}{\pgfqpoint{0.917643in}{0.963715in}}{\pgfqpoint{0.925457in}{0.955901in}}%
\pgfpathcurveto{\pgfqpoint{0.933270in}{0.948087in}}{\pgfqpoint{0.943869in}{0.943697in}}{\pgfqpoint{0.954919in}{0.943697in}}%
\pgfpathclose%
\pgfusepath{stroke,fill}%
\end{pgfscope}%
\begin{pgfscope}%
\pgfpathrectangle{\pgfqpoint{0.532717in}{0.370679in}}{\pgfqpoint{2.617283in}{1.479321in}}%
\pgfusepath{clip}%
\pgfsetbuttcap%
\pgfsetroundjoin%
\definecolor{currentfill}{rgb}{1.000000,0.411765,0.380392}%
\pgfsetfillcolor{currentfill}%
\pgfsetfillopacity{0.500000}%
\pgfsetlinewidth{1.003750pt}%
\definecolor{currentstroke}{rgb}{1.000000,0.411765,0.380392}%
\pgfsetstrokecolor{currentstroke}%
\pgfsetstrokeopacity{0.500000}%
\pgfsetdash{}{0pt}%
\pgfpathmoveto{\pgfqpoint{1.089847in}{0.754273in}}%
\pgfpathcurveto{\pgfqpoint{1.100897in}{0.754273in}}{\pgfqpoint{1.111496in}{0.758664in}}{\pgfqpoint{1.119310in}{0.766477in}}%
\pgfpathcurveto{\pgfqpoint{1.127123in}{0.774291in}}{\pgfqpoint{1.131514in}{0.784890in}}{\pgfqpoint{1.131514in}{0.795940in}}%
\pgfpathcurveto{\pgfqpoint{1.131514in}{0.806990in}}{\pgfqpoint{1.127123in}{0.817589in}}{\pgfqpoint{1.119310in}{0.825403in}}%
\pgfpathcurveto{\pgfqpoint{1.111496in}{0.833216in}}{\pgfqpoint{1.100897in}{0.837607in}}{\pgfqpoint{1.089847in}{0.837607in}}%
\pgfpathcurveto{\pgfqpoint{1.078797in}{0.837607in}}{\pgfqpoint{1.068198in}{0.833216in}}{\pgfqpoint{1.060384in}{0.825403in}}%
\pgfpathcurveto{\pgfqpoint{1.052570in}{0.817589in}}{\pgfqpoint{1.048180in}{0.806990in}}{\pgfqpoint{1.048180in}{0.795940in}}%
\pgfpathcurveto{\pgfqpoint{1.048180in}{0.784890in}}{\pgfqpoint{1.052570in}{0.774291in}}{\pgfqpoint{1.060384in}{0.766477in}}%
\pgfpathcurveto{\pgfqpoint{1.068198in}{0.758664in}}{\pgfqpoint{1.078797in}{0.754273in}}{\pgfqpoint{1.089847in}{0.754273in}}%
\pgfpathclose%
\pgfusepath{stroke,fill}%
\end{pgfscope}%
\begin{pgfscope}%
\pgfpathrectangle{\pgfqpoint{0.532717in}{0.370679in}}{\pgfqpoint{2.617283in}{1.479321in}}%
\pgfusepath{clip}%
\pgfsetbuttcap%
\pgfsetroundjoin%
\definecolor{currentfill}{rgb}{1.000000,0.411765,0.380392}%
\pgfsetfillcolor{currentfill}%
\pgfsetfillopacity{0.500000}%
\pgfsetlinewidth{1.003750pt}%
\definecolor{currentstroke}{rgb}{1.000000,0.411765,0.380392}%
\pgfsetstrokecolor{currentstroke}%
\pgfsetstrokeopacity{0.500000}%
\pgfsetdash{}{0pt}%
\pgfpathmoveto{\pgfqpoint{0.927165in}{1.134115in}}%
\pgfpathcurveto{\pgfqpoint{0.938215in}{1.134115in}}{\pgfqpoint{0.948814in}{1.138505in}}{\pgfqpoint{0.956627in}{1.146319in}}%
\pgfpathcurveto{\pgfqpoint{0.964441in}{1.154132in}}{\pgfqpoint{0.968831in}{1.164731in}}{\pgfqpoint{0.968831in}{1.175781in}}%
\pgfpathcurveto{\pgfqpoint{0.968831in}{1.186831in}}{\pgfqpoint{0.964441in}{1.197430in}}{\pgfqpoint{0.956627in}{1.205244in}}%
\pgfpathcurveto{\pgfqpoint{0.948814in}{1.213058in}}{\pgfqpoint{0.938215in}{1.217448in}}{\pgfqpoint{0.927165in}{1.217448in}}%
\pgfpathcurveto{\pgfqpoint{0.916114in}{1.217448in}}{\pgfqpoint{0.905515in}{1.213058in}}{\pgfqpoint{0.897702in}{1.205244in}}%
\pgfpathcurveto{\pgfqpoint{0.889888in}{1.197430in}}{\pgfqpoint{0.885498in}{1.186831in}}{\pgfqpoint{0.885498in}{1.175781in}}%
\pgfpathcurveto{\pgfqpoint{0.885498in}{1.164731in}}{\pgfqpoint{0.889888in}{1.154132in}}{\pgfqpoint{0.897702in}{1.146319in}}%
\pgfpathcurveto{\pgfqpoint{0.905515in}{1.138505in}}{\pgfqpoint{0.916114in}{1.134115in}}{\pgfqpoint{0.927165in}{1.134115in}}%
\pgfpathclose%
\pgfusepath{stroke,fill}%
\end{pgfscope}%
\begin{pgfscope}%
\pgfpathrectangle{\pgfqpoint{0.532717in}{0.370679in}}{\pgfqpoint{2.617283in}{1.479321in}}%
\pgfusepath{clip}%
\pgfsetbuttcap%
\pgfsetroundjoin%
\definecolor{currentfill}{rgb}{1.000000,0.411765,0.380392}%
\pgfsetfillcolor{currentfill}%
\pgfsetfillopacity{0.500000}%
\pgfsetlinewidth{1.003750pt}%
\definecolor{currentstroke}{rgb}{1.000000,0.411765,0.380392}%
\pgfsetstrokecolor{currentstroke}%
\pgfsetstrokeopacity{0.500000}%
\pgfsetdash{}{0pt}%
\pgfpathmoveto{\pgfqpoint{0.985625in}{1.018540in}}%
\pgfpathcurveto{\pgfqpoint{0.996675in}{1.018540in}}{\pgfqpoint{1.007274in}{1.022931in}}{\pgfqpoint{1.015088in}{1.030744in}}%
\pgfpathcurveto{\pgfqpoint{1.022902in}{1.038558in}}{\pgfqpoint{1.027292in}{1.049157in}}{\pgfqpoint{1.027292in}{1.060207in}}%
\pgfpathcurveto{\pgfqpoint{1.027292in}{1.071257in}}{\pgfqpoint{1.022902in}{1.081856in}}{\pgfqpoint{1.015088in}{1.089670in}}%
\pgfpathcurveto{\pgfqpoint{1.007274in}{1.097483in}}{\pgfqpoint{0.996675in}{1.101874in}}{\pgfqpoint{0.985625in}{1.101874in}}%
\pgfpathcurveto{\pgfqpoint{0.974575in}{1.101874in}}{\pgfqpoint{0.963976in}{1.097483in}}{\pgfqpoint{0.956162in}{1.089670in}}%
\pgfpathcurveto{\pgfqpoint{0.948349in}{1.081856in}}{\pgfqpoint{0.943959in}{1.071257in}}{\pgfqpoint{0.943959in}{1.060207in}}%
\pgfpathcurveto{\pgfqpoint{0.943959in}{1.049157in}}{\pgfqpoint{0.948349in}{1.038558in}}{\pgfqpoint{0.956162in}{1.030744in}}%
\pgfpathcurveto{\pgfqpoint{0.963976in}{1.022931in}}{\pgfqpoint{0.974575in}{1.018540in}}{\pgfqpoint{0.985625in}{1.018540in}}%
\pgfpathclose%
\pgfusepath{stroke,fill}%
\end{pgfscope}%
\begin{pgfscope}%
\pgfpathrectangle{\pgfqpoint{0.532717in}{0.370679in}}{\pgfqpoint{2.617283in}{1.479321in}}%
\pgfusepath{clip}%
\pgfsetbuttcap%
\pgfsetroundjoin%
\definecolor{currentfill}{rgb}{1.000000,0.411765,0.380392}%
\pgfsetfillcolor{currentfill}%
\pgfsetfillopacity{0.500000}%
\pgfsetlinewidth{1.003750pt}%
\definecolor{currentstroke}{rgb}{1.000000,0.411765,0.380392}%
\pgfsetstrokecolor{currentstroke}%
\pgfsetstrokeopacity{0.500000}%
\pgfsetdash{}{0pt}%
\pgfpathmoveto{\pgfqpoint{0.906997in}{1.357791in}}%
\pgfpathcurveto{\pgfqpoint{0.918047in}{1.357791in}}{\pgfqpoint{0.928646in}{1.362181in}}{\pgfqpoint{0.936459in}{1.369995in}}%
\pgfpathcurveto{\pgfqpoint{0.944273in}{1.377808in}}{\pgfqpoint{0.948663in}{1.388407in}}{\pgfqpoint{0.948663in}{1.399458in}}%
\pgfpathcurveto{\pgfqpoint{0.948663in}{1.410508in}}{\pgfqpoint{0.944273in}{1.421107in}}{\pgfqpoint{0.936459in}{1.428920in}}%
\pgfpathcurveto{\pgfqpoint{0.928646in}{1.436734in}}{\pgfqpoint{0.918047in}{1.441124in}}{\pgfqpoint{0.906997in}{1.441124in}}%
\pgfpathcurveto{\pgfqpoint{0.895946in}{1.441124in}}{\pgfqpoint{0.885347in}{1.436734in}}{\pgfqpoint{0.877534in}{1.428920in}}%
\pgfpathcurveto{\pgfqpoint{0.869720in}{1.421107in}}{\pgfqpoint{0.865330in}{1.410508in}}{\pgfqpoint{0.865330in}{1.399458in}}%
\pgfpathcurveto{\pgfqpoint{0.865330in}{1.388407in}}{\pgfqpoint{0.869720in}{1.377808in}}{\pgfqpoint{0.877534in}{1.369995in}}%
\pgfpathcurveto{\pgfqpoint{0.885347in}{1.362181in}}{\pgfqpoint{0.895946in}{1.357791in}}{\pgfqpoint{0.906997in}{1.357791in}}%
\pgfpathclose%
\pgfusepath{stroke,fill}%
\end{pgfscope}%
\begin{pgfscope}%
\pgfpathrectangle{\pgfqpoint{0.532717in}{0.370679in}}{\pgfqpoint{2.617283in}{1.479321in}}%
\pgfusepath{clip}%
\pgfsetbuttcap%
\pgfsetroundjoin%
\definecolor{currentfill}{rgb}{1.000000,0.411765,0.380392}%
\pgfsetfillcolor{currentfill}%
\pgfsetfillopacity{0.500000}%
\pgfsetlinewidth{1.003750pt}%
\definecolor{currentstroke}{rgb}{1.000000,0.411765,0.380392}%
\pgfsetstrokecolor{currentstroke}%
\pgfsetstrokeopacity{0.500000}%
\pgfsetdash{}{0pt}%
\pgfpathmoveto{\pgfqpoint{0.971383in}{1.145368in}}%
\pgfpathcurveto{\pgfqpoint{0.982433in}{1.145368in}}{\pgfqpoint{0.993032in}{1.149759in}}{\pgfqpoint{1.000846in}{1.157572in}}%
\pgfpathcurveto{\pgfqpoint{1.008659in}{1.165386in}}{\pgfqpoint{1.013049in}{1.175985in}}{\pgfqpoint{1.013049in}{1.187035in}}%
\pgfpathcurveto{\pgfqpoint{1.013049in}{1.198085in}}{\pgfqpoint{1.008659in}{1.208684in}}{\pgfqpoint{1.000846in}{1.216498in}}%
\pgfpathcurveto{\pgfqpoint{0.993032in}{1.224311in}}{\pgfqpoint{0.982433in}{1.228702in}}{\pgfqpoint{0.971383in}{1.228702in}}%
\pgfpathcurveto{\pgfqpoint{0.960333in}{1.228702in}}{\pgfqpoint{0.949734in}{1.224311in}}{\pgfqpoint{0.941920in}{1.216498in}}%
\pgfpathcurveto{\pgfqpoint{0.934106in}{1.208684in}}{\pgfqpoint{0.929716in}{1.198085in}}{\pgfqpoint{0.929716in}{1.187035in}}%
\pgfpathcurveto{\pgfqpoint{0.929716in}{1.175985in}}{\pgfqpoint{0.934106in}{1.165386in}}{\pgfqpoint{0.941920in}{1.157572in}}%
\pgfpathcurveto{\pgfqpoint{0.949734in}{1.149759in}}{\pgfqpoint{0.960333in}{1.145368in}}{\pgfqpoint{0.971383in}{1.145368in}}%
\pgfpathclose%
\pgfusepath{stroke,fill}%
\end{pgfscope}%
\begin{pgfscope}%
\pgfpathrectangle{\pgfqpoint{0.532717in}{0.370679in}}{\pgfqpoint{2.617283in}{1.479321in}}%
\pgfusepath{clip}%
\pgfsetbuttcap%
\pgfsetroundjoin%
\definecolor{currentfill}{rgb}{1.000000,0.411765,0.380392}%
\pgfsetfillcolor{currentfill}%
\pgfsetfillopacity{0.500000}%
\pgfsetlinewidth{1.003750pt}%
\definecolor{currentstroke}{rgb}{1.000000,0.411765,0.380392}%
\pgfsetstrokecolor{currentstroke}%
\pgfsetstrokeopacity{0.500000}%
\pgfsetdash{}{0pt}%
\pgfpathmoveto{\pgfqpoint{1.021697in}{0.798143in}}%
\pgfpathcurveto{\pgfqpoint{1.032747in}{0.798143in}}{\pgfqpoint{1.043346in}{0.802533in}}{\pgfqpoint{1.051160in}{0.810347in}}%
\pgfpathcurveto{\pgfqpoint{1.058973in}{0.818160in}}{\pgfqpoint{1.063364in}{0.828760in}}{\pgfqpoint{1.063364in}{0.839810in}}%
\pgfpathcurveto{\pgfqpoint{1.063364in}{0.850860in}}{\pgfqpoint{1.058973in}{0.861459in}}{\pgfqpoint{1.051160in}{0.869272in}}%
\pgfpathcurveto{\pgfqpoint{1.043346in}{0.877086in}}{\pgfqpoint{1.032747in}{0.881476in}}{\pgfqpoint{1.021697in}{0.881476in}}%
\pgfpathcurveto{\pgfqpoint{1.010647in}{0.881476in}}{\pgfqpoint{1.000048in}{0.877086in}}{\pgfqpoint{0.992234in}{0.869272in}}%
\pgfpathcurveto{\pgfqpoint{0.984421in}{0.861459in}}{\pgfqpoint{0.980030in}{0.850860in}}{\pgfqpoint{0.980030in}{0.839810in}}%
\pgfpathcurveto{\pgfqpoint{0.980030in}{0.828760in}}{\pgfqpoint{0.984421in}{0.818160in}}{\pgfqpoint{0.992234in}{0.810347in}}%
\pgfpathcurveto{\pgfqpoint{1.000048in}{0.802533in}}{\pgfqpoint{1.010647in}{0.798143in}}{\pgfqpoint{1.021697in}{0.798143in}}%
\pgfpathclose%
\pgfusepath{stroke,fill}%
\end{pgfscope}%
\begin{pgfscope}%
\pgfpathrectangle{\pgfqpoint{0.532717in}{0.370679in}}{\pgfqpoint{2.617283in}{1.479321in}}%
\pgfusepath{clip}%
\pgfsetbuttcap%
\pgfsetroundjoin%
\definecolor{currentfill}{rgb}{1.000000,0.411765,0.380392}%
\pgfsetfillcolor{currentfill}%
\pgfsetfillopacity{0.500000}%
\pgfsetlinewidth{1.003750pt}%
\definecolor{currentstroke}{rgb}{1.000000,0.411765,0.380392}%
\pgfsetstrokecolor{currentstroke}%
\pgfsetstrokeopacity{0.500000}%
\pgfsetdash{}{0pt}%
\pgfpathmoveto{\pgfqpoint{0.989637in}{1.086685in}}%
\pgfpathcurveto{\pgfqpoint{1.000687in}{1.086685in}}{\pgfqpoint{1.011286in}{1.091075in}}{\pgfqpoint{1.019099in}{1.098889in}}%
\pgfpathcurveto{\pgfqpoint{1.026913in}{1.106703in}}{\pgfqpoint{1.031303in}{1.117302in}}{\pgfqpoint{1.031303in}{1.128352in}}%
\pgfpathcurveto{\pgfqpoint{1.031303in}{1.139402in}}{\pgfqpoint{1.026913in}{1.150001in}}{\pgfqpoint{1.019099in}{1.157815in}}%
\pgfpathcurveto{\pgfqpoint{1.011286in}{1.165628in}}{\pgfqpoint{1.000687in}{1.170019in}}{\pgfqpoint{0.989637in}{1.170019in}}%
\pgfpathcurveto{\pgfqpoint{0.978586in}{1.170019in}}{\pgfqpoint{0.967987in}{1.165628in}}{\pgfqpoint{0.960174in}{1.157815in}}%
\pgfpathcurveto{\pgfqpoint{0.952360in}{1.150001in}}{\pgfqpoint{0.947970in}{1.139402in}}{\pgfqpoint{0.947970in}{1.128352in}}%
\pgfpathcurveto{\pgfqpoint{0.947970in}{1.117302in}}{\pgfqpoint{0.952360in}{1.106703in}}{\pgfqpoint{0.960174in}{1.098889in}}%
\pgfpathcurveto{\pgfqpoint{0.967987in}{1.091075in}}{\pgfqpoint{0.978586in}{1.086685in}}{\pgfqpoint{0.989637in}{1.086685in}}%
\pgfpathclose%
\pgfusepath{stroke,fill}%
\end{pgfscope}%
\begin{pgfscope}%
\pgfpathrectangle{\pgfqpoint{0.532717in}{0.370679in}}{\pgfqpoint{2.617283in}{1.479321in}}%
\pgfusepath{clip}%
\pgfsetbuttcap%
\pgfsetroundjoin%
\definecolor{currentfill}{rgb}{1.000000,0.411765,0.380392}%
\pgfsetfillcolor{currentfill}%
\pgfsetfillopacity{0.500000}%
\pgfsetlinewidth{1.003750pt}%
\definecolor{currentstroke}{rgb}{1.000000,0.411765,0.380392}%
\pgfsetstrokecolor{currentstroke}%
\pgfsetstrokeopacity{0.500000}%
\pgfsetdash{}{0pt}%
\pgfpathmoveto{\pgfqpoint{0.937225in}{1.200772in}}%
\pgfpathcurveto{\pgfqpoint{0.948275in}{1.200772in}}{\pgfqpoint{0.958874in}{1.205163in}}{\pgfqpoint{0.966688in}{1.212976in}}%
\pgfpathcurveto{\pgfqpoint{0.974502in}{1.220790in}}{\pgfqpoint{0.978892in}{1.231389in}}{\pgfqpoint{0.978892in}{1.242439in}}%
\pgfpathcurveto{\pgfqpoint{0.978892in}{1.253489in}}{\pgfqpoint{0.974502in}{1.264088in}}{\pgfqpoint{0.966688in}{1.271902in}}%
\pgfpathcurveto{\pgfqpoint{0.958874in}{1.279716in}}{\pgfqpoint{0.948275in}{1.284106in}}{\pgfqpoint{0.937225in}{1.284106in}}%
\pgfpathcurveto{\pgfqpoint{0.926175in}{1.284106in}}{\pgfqpoint{0.915576in}{1.279716in}}{\pgfqpoint{0.907763in}{1.271902in}}%
\pgfpathcurveto{\pgfqpoint{0.899949in}{1.264088in}}{\pgfqpoint{0.895559in}{1.253489in}}{\pgfqpoint{0.895559in}{1.242439in}}%
\pgfpathcurveto{\pgfqpoint{0.895559in}{1.231389in}}{\pgfqpoint{0.899949in}{1.220790in}}{\pgfqpoint{0.907763in}{1.212976in}}%
\pgfpathcurveto{\pgfqpoint{0.915576in}{1.205163in}}{\pgfqpoint{0.926175in}{1.200772in}}{\pgfqpoint{0.937225in}{1.200772in}}%
\pgfpathclose%
\pgfusepath{stroke,fill}%
\end{pgfscope}%
\begin{pgfscope}%
\pgfpathrectangle{\pgfqpoint{0.532717in}{0.370679in}}{\pgfqpoint{2.617283in}{1.479321in}}%
\pgfusepath{clip}%
\pgfsetbuttcap%
\pgfsetroundjoin%
\definecolor{currentfill}{rgb}{1.000000,0.411765,0.380392}%
\pgfsetfillcolor{currentfill}%
\pgfsetfillopacity{0.500000}%
\pgfsetlinewidth{1.003750pt}%
\definecolor{currentstroke}{rgb}{1.000000,0.411765,0.380392}%
\pgfsetstrokecolor{currentstroke}%
\pgfsetstrokeopacity{0.500000}%
\pgfsetdash{}{0pt}%
\pgfpathmoveto{\pgfqpoint{0.805584in}{1.326888in}}%
\pgfpathcurveto{\pgfqpoint{0.816635in}{1.326888in}}{\pgfqpoint{0.827234in}{1.331278in}}{\pgfqpoint{0.835047in}{1.339092in}}%
\pgfpathcurveto{\pgfqpoint{0.842861in}{1.346905in}}{\pgfqpoint{0.847251in}{1.357504in}}{\pgfqpoint{0.847251in}{1.368554in}}%
\pgfpathcurveto{\pgfqpoint{0.847251in}{1.379604in}}{\pgfqpoint{0.842861in}{1.390203in}}{\pgfqpoint{0.835047in}{1.398017in}}%
\pgfpathcurveto{\pgfqpoint{0.827234in}{1.405831in}}{\pgfqpoint{0.816635in}{1.410221in}}{\pgfqpoint{0.805584in}{1.410221in}}%
\pgfpathcurveto{\pgfqpoint{0.794534in}{1.410221in}}{\pgfqpoint{0.783935in}{1.405831in}}{\pgfqpoint{0.776122in}{1.398017in}}%
\pgfpathcurveto{\pgfqpoint{0.768308in}{1.390203in}}{\pgfqpoint{0.763918in}{1.379604in}}{\pgfqpoint{0.763918in}{1.368554in}}%
\pgfpathcurveto{\pgfqpoint{0.763918in}{1.357504in}}{\pgfqpoint{0.768308in}{1.346905in}}{\pgfqpoint{0.776122in}{1.339092in}}%
\pgfpathcurveto{\pgfqpoint{0.783935in}{1.331278in}}{\pgfqpoint{0.794534in}{1.326888in}}{\pgfqpoint{0.805584in}{1.326888in}}%
\pgfpathclose%
\pgfusepath{stroke,fill}%
\end{pgfscope}%
\begin{pgfscope}%
\pgfpathrectangle{\pgfqpoint{0.532717in}{0.370679in}}{\pgfqpoint{2.617283in}{1.479321in}}%
\pgfusepath{clip}%
\pgfsetbuttcap%
\pgfsetroundjoin%
\definecolor{currentfill}{rgb}{1.000000,0.411765,0.380392}%
\pgfsetfillcolor{currentfill}%
\pgfsetfillopacity{0.500000}%
\pgfsetlinewidth{1.003750pt}%
\definecolor{currentstroke}{rgb}{1.000000,0.411765,0.380392}%
\pgfsetstrokecolor{currentstroke}%
\pgfsetstrokeopacity{0.500000}%
\pgfsetdash{}{0pt}%
\pgfpathmoveto{\pgfqpoint{0.980495in}{0.553611in}}%
\pgfpathcurveto{\pgfqpoint{0.991545in}{0.553611in}}{\pgfqpoint{1.002144in}{0.558001in}}{\pgfqpoint{1.009958in}{0.565815in}}%
\pgfpathcurveto{\pgfqpoint{1.017771in}{0.573628in}}{\pgfqpoint{1.022161in}{0.584227in}}{\pgfqpoint{1.022161in}{0.595278in}}%
\pgfpathcurveto{\pgfqpoint{1.022161in}{0.606328in}}{\pgfqpoint{1.017771in}{0.616927in}}{\pgfqpoint{1.009958in}{0.624740in}}%
\pgfpathcurveto{\pgfqpoint{1.002144in}{0.632554in}}{\pgfqpoint{0.991545in}{0.636944in}}{\pgfqpoint{0.980495in}{0.636944in}}%
\pgfpathcurveto{\pgfqpoint{0.969445in}{0.636944in}}{\pgfqpoint{0.958846in}{0.632554in}}{\pgfqpoint{0.951032in}{0.624740in}}%
\pgfpathcurveto{\pgfqpoint{0.943218in}{0.616927in}}{\pgfqpoint{0.938828in}{0.606328in}}{\pgfqpoint{0.938828in}{0.595278in}}%
\pgfpathcurveto{\pgfqpoint{0.938828in}{0.584227in}}{\pgfqpoint{0.943218in}{0.573628in}}{\pgfqpoint{0.951032in}{0.565815in}}%
\pgfpathcurveto{\pgfqpoint{0.958846in}{0.558001in}}{\pgfqpoint{0.969445in}{0.553611in}}{\pgfqpoint{0.980495in}{0.553611in}}%
\pgfpathclose%
\pgfusepath{stroke,fill}%
\end{pgfscope}%
\begin{pgfscope}%
\pgfpathrectangle{\pgfqpoint{0.532717in}{0.370679in}}{\pgfqpoint{2.617283in}{1.479321in}}%
\pgfusepath{clip}%
\pgfsetbuttcap%
\pgfsetroundjoin%
\definecolor{currentfill}{rgb}{1.000000,0.411765,0.380392}%
\pgfsetfillcolor{currentfill}%
\pgfsetfillopacity{0.500000}%
\pgfsetlinewidth{1.003750pt}%
\definecolor{currentstroke}{rgb}{1.000000,0.411765,0.380392}%
\pgfsetstrokecolor{currentstroke}%
\pgfsetstrokeopacity{0.500000}%
\pgfsetdash{}{0pt}%
\pgfpathmoveto{\pgfqpoint{1.058581in}{0.481038in}}%
\pgfpathcurveto{\pgfqpoint{1.069632in}{0.481038in}}{\pgfqpoint{1.080231in}{0.485428in}}{\pgfqpoint{1.088044in}{0.493242in}}%
\pgfpathcurveto{\pgfqpoint{1.095858in}{0.501055in}}{\pgfqpoint{1.100248in}{0.511654in}}{\pgfqpoint{1.100248in}{0.522705in}}%
\pgfpathcurveto{\pgfqpoint{1.100248in}{0.533755in}}{\pgfqpoint{1.095858in}{0.544354in}}{\pgfqpoint{1.088044in}{0.552167in}}%
\pgfpathcurveto{\pgfqpoint{1.080231in}{0.559981in}}{\pgfqpoint{1.069632in}{0.564371in}}{\pgfqpoint{1.058581in}{0.564371in}}%
\pgfpathcurveto{\pgfqpoint{1.047531in}{0.564371in}}{\pgfqpoint{1.036932in}{0.559981in}}{\pgfqpoint{1.029119in}{0.552167in}}%
\pgfpathcurveto{\pgfqpoint{1.021305in}{0.544354in}}{\pgfqpoint{1.016915in}{0.533755in}}{\pgfqpoint{1.016915in}{0.522705in}}%
\pgfpathcurveto{\pgfqpoint{1.016915in}{0.511654in}}{\pgfqpoint{1.021305in}{0.501055in}}{\pgfqpoint{1.029119in}{0.493242in}}%
\pgfpathcurveto{\pgfqpoint{1.036932in}{0.485428in}}{\pgfqpoint{1.047531in}{0.481038in}}{\pgfqpoint{1.058581in}{0.481038in}}%
\pgfpathclose%
\pgfusepath{stroke,fill}%
\end{pgfscope}%
\begin{pgfscope}%
\pgfpathrectangle{\pgfqpoint{0.532717in}{0.370679in}}{\pgfqpoint{2.617283in}{1.479321in}}%
\pgfusepath{clip}%
\pgfsetbuttcap%
\pgfsetroundjoin%
\definecolor{currentfill}{rgb}{1.000000,0.411765,0.380392}%
\pgfsetfillcolor{currentfill}%
\pgfsetfillopacity{0.500000}%
\pgfsetlinewidth{1.003750pt}%
\definecolor{currentstroke}{rgb}{1.000000,0.411765,0.380392}%
\pgfsetstrokecolor{currentstroke}%
\pgfsetstrokeopacity{0.500000}%
\pgfsetdash{}{0pt}%
\pgfpathmoveto{\pgfqpoint{0.986815in}{0.722094in}}%
\pgfpathcurveto{\pgfqpoint{0.997865in}{0.722094in}}{\pgfqpoint{1.008464in}{0.726484in}}{\pgfqpoint{1.016278in}{0.734297in}}%
\pgfpathcurveto{\pgfqpoint{1.024091in}{0.742111in}}{\pgfqpoint{1.028481in}{0.752710in}}{\pgfqpoint{1.028481in}{0.763760in}}%
\pgfpathcurveto{\pgfqpoint{1.028481in}{0.774810in}}{\pgfqpoint{1.024091in}{0.785409in}}{\pgfqpoint{1.016278in}{0.793223in}}%
\pgfpathcurveto{\pgfqpoint{1.008464in}{0.801037in}}{\pgfqpoint{0.997865in}{0.805427in}}{\pgfqpoint{0.986815in}{0.805427in}}%
\pgfpathcurveto{\pgfqpoint{0.975765in}{0.805427in}}{\pgfqpoint{0.965166in}{0.801037in}}{\pgfqpoint{0.957352in}{0.793223in}}%
\pgfpathcurveto{\pgfqpoint{0.949538in}{0.785409in}}{\pgfqpoint{0.945148in}{0.774810in}}{\pgfqpoint{0.945148in}{0.763760in}}%
\pgfpathcurveto{\pgfqpoint{0.945148in}{0.752710in}}{\pgfqpoint{0.949538in}{0.742111in}}{\pgfqpoint{0.957352in}{0.734297in}}%
\pgfpathcurveto{\pgfqpoint{0.965166in}{0.726484in}}{\pgfqpoint{0.975765in}{0.722094in}}{\pgfqpoint{0.986815in}{0.722094in}}%
\pgfpathclose%
\pgfusepath{stroke,fill}%
\end{pgfscope}%
\begin{pgfscope}%
\pgfpathrectangle{\pgfqpoint{0.532717in}{0.370679in}}{\pgfqpoint{2.617283in}{1.479321in}}%
\pgfusepath{clip}%
\pgfsetbuttcap%
\pgfsetroundjoin%
\definecolor{currentfill}{rgb}{1.000000,0.411765,0.380392}%
\pgfsetfillcolor{currentfill}%
\pgfsetfillopacity{0.500000}%
\pgfsetlinewidth{1.003750pt}%
\definecolor{currentstroke}{rgb}{1.000000,0.411765,0.380392}%
\pgfsetstrokecolor{currentstroke}%
\pgfsetstrokeopacity{0.500000}%
\pgfsetdash{}{0pt}%
\pgfpathmoveto{\pgfqpoint{0.979058in}{0.950466in}}%
\pgfpathcurveto{\pgfqpoint{0.990108in}{0.950466in}}{\pgfqpoint{1.000707in}{0.954856in}}{\pgfqpoint{1.008520in}{0.962669in}}%
\pgfpathcurveto{\pgfqpoint{1.016334in}{0.970483in}}{\pgfqpoint{1.020724in}{0.981082in}}{\pgfqpoint{1.020724in}{0.992132in}}%
\pgfpathcurveto{\pgfqpoint{1.020724in}{1.003182in}}{\pgfqpoint{1.016334in}{1.013781in}}{\pgfqpoint{1.008520in}{1.021595in}}%
\pgfpathcurveto{\pgfqpoint{1.000707in}{1.029409in}}{\pgfqpoint{0.990108in}{1.033799in}}{\pgfqpoint{0.979058in}{1.033799in}}%
\pgfpathcurveto{\pgfqpoint{0.968007in}{1.033799in}}{\pgfqpoint{0.957408in}{1.029409in}}{\pgfqpoint{0.949595in}{1.021595in}}%
\pgfpathcurveto{\pgfqpoint{0.941781in}{1.013781in}}{\pgfqpoint{0.937391in}{1.003182in}}{\pgfqpoint{0.937391in}{0.992132in}}%
\pgfpathcurveto{\pgfqpoint{0.937391in}{0.981082in}}{\pgfqpoint{0.941781in}{0.970483in}}{\pgfqpoint{0.949595in}{0.962669in}}%
\pgfpathcurveto{\pgfqpoint{0.957408in}{0.954856in}}{\pgfqpoint{0.968007in}{0.950466in}}{\pgfqpoint{0.979058in}{0.950466in}}%
\pgfpathclose%
\pgfusepath{stroke,fill}%
\end{pgfscope}%
\begin{pgfscope}%
\pgfpathrectangle{\pgfqpoint{0.532717in}{0.370679in}}{\pgfqpoint{2.617283in}{1.479321in}}%
\pgfusepath{clip}%
\pgfsetbuttcap%
\pgfsetroundjoin%
\definecolor{currentfill}{rgb}{1.000000,0.411765,0.380392}%
\pgfsetfillcolor{currentfill}%
\pgfsetfillopacity{0.500000}%
\pgfsetlinewidth{1.003750pt}%
\definecolor{currentstroke}{rgb}{1.000000,0.411765,0.380392}%
\pgfsetstrokecolor{currentstroke}%
\pgfsetstrokeopacity{0.500000}%
\pgfsetdash{}{0pt}%
\pgfpathmoveto{\pgfqpoint{1.114150in}{0.694103in}}%
\pgfpathcurveto{\pgfqpoint{1.125200in}{0.694103in}}{\pgfqpoint{1.135799in}{0.698493in}}{\pgfqpoint{1.143613in}{0.706307in}}%
\pgfpathcurveto{\pgfqpoint{1.151427in}{0.714121in}}{\pgfqpoint{1.155817in}{0.724720in}}{\pgfqpoint{1.155817in}{0.735770in}}%
\pgfpathcurveto{\pgfqpoint{1.155817in}{0.746820in}}{\pgfqpoint{1.151427in}{0.757419in}}{\pgfqpoint{1.143613in}{0.765233in}}%
\pgfpathcurveto{\pgfqpoint{1.135799in}{0.773046in}}{\pgfqpoint{1.125200in}{0.777436in}}{\pgfqpoint{1.114150in}{0.777436in}}%
\pgfpathcurveto{\pgfqpoint{1.103100in}{0.777436in}}{\pgfqpoint{1.092501in}{0.773046in}}{\pgfqpoint{1.084687in}{0.765233in}}%
\pgfpathcurveto{\pgfqpoint{1.076874in}{0.757419in}}{\pgfqpoint{1.072483in}{0.746820in}}{\pgfqpoint{1.072483in}{0.735770in}}%
\pgfpathcurveto{\pgfqpoint{1.072483in}{0.724720in}}{\pgfqpoint{1.076874in}{0.714121in}}{\pgfqpoint{1.084687in}{0.706307in}}%
\pgfpathcurveto{\pgfqpoint{1.092501in}{0.698493in}}{\pgfqpoint{1.103100in}{0.694103in}}{\pgfqpoint{1.114150in}{0.694103in}}%
\pgfpathclose%
\pgfusepath{stroke,fill}%
\end{pgfscope}%
\begin{pgfscope}%
\pgfpathrectangle{\pgfqpoint{0.532717in}{0.370679in}}{\pgfqpoint{2.617283in}{1.479321in}}%
\pgfusepath{clip}%
\pgfsetbuttcap%
\pgfsetroundjoin%
\definecolor{currentfill}{rgb}{1.000000,0.411765,0.380392}%
\pgfsetfillcolor{currentfill}%
\pgfsetfillopacity{0.500000}%
\pgfsetlinewidth{1.003750pt}%
\definecolor{currentstroke}{rgb}{1.000000,0.411765,0.380392}%
\pgfsetstrokecolor{currentstroke}%
\pgfsetstrokeopacity{0.500000}%
\pgfsetdash{}{0pt}%
\pgfpathmoveto{\pgfqpoint{0.997394in}{0.858313in}}%
\pgfpathcurveto{\pgfqpoint{1.008444in}{0.858313in}}{\pgfqpoint{1.019043in}{0.862703in}}{\pgfqpoint{1.026857in}{0.870517in}}%
\pgfpathcurveto{\pgfqpoint{1.034670in}{0.878331in}}{\pgfqpoint{1.039060in}{0.888930in}}{\pgfqpoint{1.039060in}{0.899980in}}%
\pgfpathcurveto{\pgfqpoint{1.039060in}{0.911030in}}{\pgfqpoint{1.034670in}{0.921629in}}{\pgfqpoint{1.026857in}{0.929443in}}%
\pgfpathcurveto{\pgfqpoint{1.019043in}{0.937256in}}{\pgfqpoint{1.008444in}{0.941647in}}{\pgfqpoint{0.997394in}{0.941647in}}%
\pgfpathcurveto{\pgfqpoint{0.986344in}{0.941647in}}{\pgfqpoint{0.975745in}{0.937256in}}{\pgfqpoint{0.967931in}{0.929443in}}%
\pgfpathcurveto{\pgfqpoint{0.960117in}{0.921629in}}{\pgfqpoint{0.955727in}{0.911030in}}{\pgfqpoint{0.955727in}{0.899980in}}%
\pgfpathcurveto{\pgfqpoint{0.955727in}{0.888930in}}{\pgfqpoint{0.960117in}{0.878331in}}{\pgfqpoint{0.967931in}{0.870517in}}%
\pgfpathcurveto{\pgfqpoint{0.975745in}{0.862703in}}{\pgfqpoint{0.986344in}{0.858313in}}{\pgfqpoint{0.997394in}{0.858313in}}%
\pgfpathclose%
\pgfusepath{stroke,fill}%
\end{pgfscope}%
\begin{pgfscope}%
\pgfpathrectangle{\pgfqpoint{0.532717in}{0.370679in}}{\pgfqpoint{2.617283in}{1.479321in}}%
\pgfusepath{clip}%
\pgfsetbuttcap%
\pgfsetroundjoin%
\definecolor{currentfill}{rgb}{1.000000,0.411765,0.380392}%
\pgfsetfillcolor{currentfill}%
\pgfsetfillopacity{0.500000}%
\pgfsetlinewidth{1.003750pt}%
\definecolor{currentstroke}{rgb}{1.000000,0.411765,0.380392}%
\pgfsetstrokecolor{currentstroke}%
\pgfsetstrokeopacity{0.500000}%
\pgfsetdash{}{0pt}%
\pgfpathmoveto{\pgfqpoint{1.080635in}{0.914430in}}%
\pgfpathcurveto{\pgfqpoint{1.091685in}{0.914430in}}{\pgfqpoint{1.102284in}{0.918820in}}{\pgfqpoint{1.110098in}{0.926634in}}%
\pgfpathcurveto{\pgfqpoint{1.117911in}{0.934448in}}{\pgfqpoint{1.122301in}{0.945047in}}{\pgfqpoint{1.122301in}{0.956097in}}%
\pgfpathcurveto{\pgfqpoint{1.122301in}{0.967147in}}{\pgfqpoint{1.117911in}{0.977746in}}{\pgfqpoint{1.110098in}{0.985560in}}%
\pgfpathcurveto{\pgfqpoint{1.102284in}{0.993373in}}{\pgfqpoint{1.091685in}{0.997763in}}{\pgfqpoint{1.080635in}{0.997763in}}%
\pgfpathcurveto{\pgfqpoint{1.069585in}{0.997763in}}{\pgfqpoint{1.058986in}{0.993373in}}{\pgfqpoint{1.051172in}{0.985560in}}%
\pgfpathcurveto{\pgfqpoint{1.043358in}{0.977746in}}{\pgfqpoint{1.038968in}{0.967147in}}{\pgfqpoint{1.038968in}{0.956097in}}%
\pgfpathcurveto{\pgfqpoint{1.038968in}{0.945047in}}{\pgfqpoint{1.043358in}{0.934448in}}{\pgfqpoint{1.051172in}{0.926634in}}%
\pgfpathcurveto{\pgfqpoint{1.058986in}{0.918820in}}{\pgfqpoint{1.069585in}{0.914430in}}{\pgfqpoint{1.080635in}{0.914430in}}%
\pgfpathclose%
\pgfusepath{stroke,fill}%
\end{pgfscope}%
\begin{pgfscope}%
\pgfpathrectangle{\pgfqpoint{0.532717in}{0.370679in}}{\pgfqpoint{2.617283in}{1.479321in}}%
\pgfusepath{clip}%
\pgfsetbuttcap%
\pgfsetroundjoin%
\definecolor{currentfill}{rgb}{1.000000,0.411765,0.380392}%
\pgfsetfillcolor{currentfill}%
\pgfsetfillopacity{0.500000}%
\pgfsetlinewidth{1.003750pt}%
\definecolor{currentstroke}{rgb}{1.000000,0.411765,0.380392}%
\pgfsetstrokecolor{currentstroke}%
\pgfsetstrokeopacity{0.500000}%
\pgfsetdash{}{0pt}%
\pgfpathmoveto{\pgfqpoint{1.010659in}{0.895132in}}%
\pgfpathcurveto{\pgfqpoint{1.021709in}{0.895132in}}{\pgfqpoint{1.032308in}{0.899522in}}{\pgfqpoint{1.040121in}{0.907336in}}%
\pgfpathcurveto{\pgfqpoint{1.047935in}{0.915149in}}{\pgfqpoint{1.052325in}{0.925748in}}{\pgfqpoint{1.052325in}{0.936798in}}%
\pgfpathcurveto{\pgfqpoint{1.052325in}{0.947848in}}{\pgfqpoint{1.047935in}{0.958447in}}{\pgfqpoint{1.040121in}{0.966261in}}%
\pgfpathcurveto{\pgfqpoint{1.032308in}{0.974075in}}{\pgfqpoint{1.021709in}{0.978465in}}{\pgfqpoint{1.010659in}{0.978465in}}%
\pgfpathcurveto{\pgfqpoint{0.999608in}{0.978465in}}{\pgfqpoint{0.989009in}{0.974075in}}{\pgfqpoint{0.981196in}{0.966261in}}%
\pgfpathcurveto{\pgfqpoint{0.973382in}{0.958447in}}{\pgfqpoint{0.968992in}{0.947848in}}{\pgfqpoint{0.968992in}{0.936798in}}%
\pgfpathcurveto{\pgfqpoint{0.968992in}{0.925748in}}{\pgfqpoint{0.973382in}{0.915149in}}{\pgfqpoint{0.981196in}{0.907336in}}%
\pgfpathcurveto{\pgfqpoint{0.989009in}{0.899522in}}{\pgfqpoint{0.999608in}{0.895132in}}{\pgfqpoint{1.010659in}{0.895132in}}%
\pgfpathclose%
\pgfusepath{stroke,fill}%
\end{pgfscope}%
\begin{pgfscope}%
\pgfpathrectangle{\pgfqpoint{0.532717in}{0.370679in}}{\pgfqpoint{2.617283in}{1.479321in}}%
\pgfusepath{clip}%
\pgfsetbuttcap%
\pgfsetroundjoin%
\definecolor{currentfill}{rgb}{1.000000,0.411765,0.380392}%
\pgfsetfillcolor{currentfill}%
\pgfsetfillopacity{0.500000}%
\pgfsetlinewidth{1.003750pt}%
\definecolor{currentstroke}{rgb}{1.000000,0.411765,0.380392}%
\pgfsetstrokecolor{currentstroke}%
\pgfsetstrokeopacity{0.500000}%
\pgfsetdash{}{0pt}%
\pgfpathmoveto{\pgfqpoint{0.808394in}{1.031717in}}%
\pgfpathcurveto{\pgfqpoint{0.819444in}{1.031717in}}{\pgfqpoint{0.830043in}{1.036108in}}{\pgfqpoint{0.837857in}{1.043921in}}%
\pgfpathcurveto{\pgfqpoint{0.845670in}{1.051735in}}{\pgfqpoint{0.850061in}{1.062334in}}{\pgfqpoint{0.850061in}{1.073384in}}%
\pgfpathcurveto{\pgfqpoint{0.850061in}{1.084434in}}{\pgfqpoint{0.845670in}{1.095033in}}{\pgfqpoint{0.837857in}{1.102847in}}%
\pgfpathcurveto{\pgfqpoint{0.830043in}{1.110661in}}{\pgfqpoint{0.819444in}{1.115051in}}{\pgfqpoint{0.808394in}{1.115051in}}%
\pgfpathcurveto{\pgfqpoint{0.797344in}{1.115051in}}{\pgfqpoint{0.786745in}{1.110661in}}{\pgfqpoint{0.778931in}{1.102847in}}%
\pgfpathcurveto{\pgfqpoint{0.771117in}{1.095033in}}{\pgfqpoint{0.766727in}{1.084434in}}{\pgfqpoint{0.766727in}{1.073384in}}%
\pgfpathcurveto{\pgfqpoint{0.766727in}{1.062334in}}{\pgfqpoint{0.771117in}{1.051735in}}{\pgfqpoint{0.778931in}{1.043921in}}%
\pgfpathcurveto{\pgfqpoint{0.786745in}{1.036108in}}{\pgfqpoint{0.797344in}{1.031717in}}{\pgfqpoint{0.808394in}{1.031717in}}%
\pgfpathclose%
\pgfusepath{stroke,fill}%
\end{pgfscope}%
\begin{pgfscope}%
\pgfpathrectangle{\pgfqpoint{0.532717in}{0.370679in}}{\pgfqpoint{2.617283in}{1.479321in}}%
\pgfusepath{clip}%
\pgfsetbuttcap%
\pgfsetroundjoin%
\definecolor{currentfill}{rgb}{1.000000,0.411765,0.380392}%
\pgfsetfillcolor{currentfill}%
\pgfsetfillopacity{0.500000}%
\pgfsetlinewidth{1.003750pt}%
\definecolor{currentstroke}{rgb}{1.000000,0.411765,0.380392}%
\pgfsetstrokecolor{currentstroke}%
\pgfsetstrokeopacity{0.500000}%
\pgfsetdash{}{0pt}%
\pgfpathmoveto{\pgfqpoint{1.082861in}{1.048237in}}%
\pgfpathcurveto{\pgfqpoint{1.093911in}{1.048237in}}{\pgfqpoint{1.104510in}{1.052628in}}{\pgfqpoint{1.112324in}{1.060441in}}%
\pgfpathcurveto{\pgfqpoint{1.120137in}{1.068255in}}{\pgfqpoint{1.124528in}{1.078854in}}{\pgfqpoint{1.124528in}{1.089904in}}%
\pgfpathcurveto{\pgfqpoint{1.124528in}{1.100954in}}{\pgfqpoint{1.120137in}{1.111553in}}{\pgfqpoint{1.112324in}{1.119367in}}%
\pgfpathcurveto{\pgfqpoint{1.104510in}{1.127180in}}{\pgfqpoint{1.093911in}{1.131571in}}{\pgfqpoint{1.082861in}{1.131571in}}%
\pgfpathcurveto{\pgfqpoint{1.071811in}{1.131571in}}{\pgfqpoint{1.061212in}{1.127180in}}{\pgfqpoint{1.053398in}{1.119367in}}%
\pgfpathcurveto{\pgfqpoint{1.045585in}{1.111553in}}{\pgfqpoint{1.041194in}{1.100954in}}{\pgfqpoint{1.041194in}{1.089904in}}%
\pgfpathcurveto{\pgfqpoint{1.041194in}{1.078854in}}{\pgfqpoint{1.045585in}{1.068255in}}{\pgfqpoint{1.053398in}{1.060441in}}%
\pgfpathcurveto{\pgfqpoint{1.061212in}{1.052628in}}{\pgfqpoint{1.071811in}{1.048237in}}{\pgfqpoint{1.082861in}{1.048237in}}%
\pgfpathclose%
\pgfusepath{stroke,fill}%
\end{pgfscope}%
\begin{pgfscope}%
\pgfpathrectangle{\pgfqpoint{0.532717in}{0.370679in}}{\pgfqpoint{2.617283in}{1.479321in}}%
\pgfusepath{clip}%
\pgfsetbuttcap%
\pgfsetroundjoin%
\definecolor{currentfill}{rgb}{1.000000,0.411765,0.380392}%
\pgfsetfillcolor{currentfill}%
\pgfsetfillopacity{0.500000}%
\pgfsetlinewidth{1.003750pt}%
\definecolor{currentstroke}{rgb}{1.000000,0.411765,0.380392}%
\pgfsetstrokecolor{currentstroke}%
\pgfsetstrokeopacity{0.500000}%
\pgfsetdash{}{0pt}%
\pgfpathmoveto{\pgfqpoint{1.066911in}{1.110820in}}%
\pgfpathcurveto{\pgfqpoint{1.077961in}{1.110820in}}{\pgfqpoint{1.088560in}{1.115210in}}{\pgfqpoint{1.096373in}{1.123024in}}%
\pgfpathcurveto{\pgfqpoint{1.104187in}{1.130837in}}{\pgfqpoint{1.108577in}{1.141436in}}{\pgfqpoint{1.108577in}{1.152487in}}%
\pgfpathcurveto{\pgfqpoint{1.108577in}{1.163537in}}{\pgfqpoint{1.104187in}{1.174136in}}{\pgfqpoint{1.096373in}{1.181949in}}%
\pgfpathcurveto{\pgfqpoint{1.088560in}{1.189763in}}{\pgfqpoint{1.077961in}{1.194153in}}{\pgfqpoint{1.066911in}{1.194153in}}%
\pgfpathcurveto{\pgfqpoint{1.055860in}{1.194153in}}{\pgfqpoint{1.045261in}{1.189763in}}{\pgfqpoint{1.037448in}{1.181949in}}%
\pgfpathcurveto{\pgfqpoint{1.029634in}{1.174136in}}{\pgfqpoint{1.025244in}{1.163537in}}{\pgfqpoint{1.025244in}{1.152487in}}%
\pgfpathcurveto{\pgfqpoint{1.025244in}{1.141436in}}{\pgfqpoint{1.029634in}{1.130837in}}{\pgfqpoint{1.037448in}{1.123024in}}%
\pgfpathcurveto{\pgfqpoint{1.045261in}{1.115210in}}{\pgfqpoint{1.055860in}{1.110820in}}{\pgfqpoint{1.066911in}{1.110820in}}%
\pgfpathclose%
\pgfusepath{stroke,fill}%
\end{pgfscope}%
\begin{pgfscope}%
\pgfpathrectangle{\pgfqpoint{0.532717in}{0.370679in}}{\pgfqpoint{2.617283in}{1.479321in}}%
\pgfusepath{clip}%
\pgfsetbuttcap%
\pgfsetroundjoin%
\definecolor{currentfill}{rgb}{1.000000,0.411765,0.380392}%
\pgfsetfillcolor{currentfill}%
\pgfsetfillopacity{0.500000}%
\pgfsetlinewidth{1.003750pt}%
\definecolor{currentstroke}{rgb}{1.000000,0.411765,0.380392}%
\pgfsetstrokecolor{currentstroke}%
\pgfsetstrokeopacity{0.500000}%
\pgfsetdash{}{0pt}%
\pgfpathmoveto{\pgfqpoint{1.021279in}{1.160182in}}%
\pgfpathcurveto{\pgfqpoint{1.032329in}{1.160182in}}{\pgfqpoint{1.042928in}{1.164572in}}{\pgfqpoint{1.050742in}{1.172386in}}%
\pgfpathcurveto{\pgfqpoint{1.058555in}{1.180199in}}{\pgfqpoint{1.062946in}{1.190798in}}{\pgfqpoint{1.062946in}{1.201848in}}%
\pgfpathcurveto{\pgfqpoint{1.062946in}{1.212898in}}{\pgfqpoint{1.058555in}{1.223497in}}{\pgfqpoint{1.050742in}{1.231311in}}%
\pgfpathcurveto{\pgfqpoint{1.042928in}{1.239125in}}{\pgfqpoint{1.032329in}{1.243515in}}{\pgfqpoint{1.021279in}{1.243515in}}%
\pgfpathcurveto{\pgfqpoint{1.010229in}{1.243515in}}{\pgfqpoint{0.999630in}{1.239125in}}{\pgfqpoint{0.991816in}{1.231311in}}%
\pgfpathcurveto{\pgfqpoint{0.984003in}{1.223497in}}{\pgfqpoint{0.979612in}{1.212898in}}{\pgfqpoint{0.979612in}{1.201848in}}%
\pgfpathcurveto{\pgfqpoint{0.979612in}{1.190798in}}{\pgfqpoint{0.984003in}{1.180199in}}{\pgfqpoint{0.991816in}{1.172386in}}%
\pgfpathcurveto{\pgfqpoint{0.999630in}{1.164572in}}{\pgfqpoint{1.010229in}{1.160182in}}{\pgfqpoint{1.021279in}{1.160182in}}%
\pgfpathclose%
\pgfusepath{stroke,fill}%
\end{pgfscope}%
\begin{pgfscope}%
\pgfpathrectangle{\pgfqpoint{0.532717in}{0.370679in}}{\pgfqpoint{2.617283in}{1.479321in}}%
\pgfusepath{clip}%
\pgfsetbuttcap%
\pgfsetroundjoin%
\definecolor{currentfill}{rgb}{1.000000,0.411765,0.380392}%
\pgfsetfillcolor{currentfill}%
\pgfsetfillopacity{0.500000}%
\pgfsetlinewidth{1.003750pt}%
\definecolor{currentstroke}{rgb}{1.000000,0.411765,0.380392}%
\pgfsetstrokecolor{currentstroke}%
\pgfsetstrokeopacity{0.500000}%
\pgfsetdash{}{0pt}%
\pgfpathmoveto{\pgfqpoint{1.032965in}{1.033424in}}%
\pgfpathcurveto{\pgfqpoint{1.044015in}{1.033424in}}{\pgfqpoint{1.054614in}{1.037814in}}{\pgfqpoint{1.062428in}{1.045628in}}%
\pgfpathcurveto{\pgfqpoint{1.070241in}{1.053441in}}{\pgfqpoint{1.074632in}{1.064040in}}{\pgfqpoint{1.074632in}{1.075091in}}%
\pgfpathcurveto{\pgfqpoint{1.074632in}{1.086141in}}{\pgfqpoint{1.070241in}{1.096740in}}{\pgfqpoint{1.062428in}{1.104553in}}%
\pgfpathcurveto{\pgfqpoint{1.054614in}{1.112367in}}{\pgfqpoint{1.044015in}{1.116757in}}{\pgfqpoint{1.032965in}{1.116757in}}%
\pgfpathcurveto{\pgfqpoint{1.021915in}{1.116757in}}{\pgfqpoint{1.011316in}{1.112367in}}{\pgfqpoint{1.003502in}{1.104553in}}%
\pgfpathcurveto{\pgfqpoint{0.995689in}{1.096740in}}{\pgfqpoint{0.991298in}{1.086141in}}{\pgfqpoint{0.991298in}{1.075091in}}%
\pgfpathcurveto{\pgfqpoint{0.991298in}{1.064040in}}{\pgfqpoint{0.995689in}{1.053441in}}{\pgfqpoint{1.003502in}{1.045628in}}%
\pgfpathcurveto{\pgfqpoint{1.011316in}{1.037814in}}{\pgfqpoint{1.021915in}{1.033424in}}{\pgfqpoint{1.032965in}{1.033424in}}%
\pgfpathclose%
\pgfusepath{stroke,fill}%
\end{pgfscope}%
\begin{pgfscope}%
\pgfpathrectangle{\pgfqpoint{0.532717in}{0.370679in}}{\pgfqpoint{2.617283in}{1.479321in}}%
\pgfusepath{clip}%
\pgfsetbuttcap%
\pgfsetroundjoin%
\definecolor{currentfill}{rgb}{1.000000,0.411765,0.380392}%
\pgfsetfillcolor{currentfill}%
\pgfsetfillopacity{0.500000}%
\pgfsetlinewidth{1.003750pt}%
\definecolor{currentstroke}{rgb}{1.000000,0.411765,0.380392}%
\pgfsetstrokecolor{currentstroke}%
\pgfsetstrokeopacity{0.500000}%
\pgfsetdash{}{0pt}%
\pgfpathmoveto{\pgfqpoint{1.004898in}{0.925041in}}%
\pgfpathcurveto{\pgfqpoint{1.015948in}{0.925041in}}{\pgfqpoint{1.026547in}{0.929431in}}{\pgfqpoint{1.034361in}{0.937245in}}%
\pgfpathcurveto{\pgfqpoint{1.042175in}{0.945059in}}{\pgfqpoint{1.046565in}{0.955658in}}{\pgfqpoint{1.046565in}{0.966708in}}%
\pgfpathcurveto{\pgfqpoint{1.046565in}{0.977758in}}{\pgfqpoint{1.042175in}{0.988357in}}{\pgfqpoint{1.034361in}{0.996171in}}%
\pgfpathcurveto{\pgfqpoint{1.026547in}{1.003984in}}{\pgfqpoint{1.015948in}{1.008375in}}{\pgfqpoint{1.004898in}{1.008375in}}%
\pgfpathcurveto{\pgfqpoint{0.993848in}{1.008375in}}{\pgfqpoint{0.983249in}{1.003984in}}{\pgfqpoint{0.975435in}{0.996171in}}%
\pgfpathcurveto{\pgfqpoint{0.967622in}{0.988357in}}{\pgfqpoint{0.963231in}{0.977758in}}{\pgfqpoint{0.963231in}{0.966708in}}%
\pgfpathcurveto{\pgfqpoint{0.963231in}{0.955658in}}{\pgfqpoint{0.967622in}{0.945059in}}{\pgfqpoint{0.975435in}{0.937245in}}%
\pgfpathcurveto{\pgfqpoint{0.983249in}{0.929431in}}{\pgfqpoint{0.993848in}{0.925041in}}{\pgfqpoint{1.004898in}{0.925041in}}%
\pgfpathclose%
\pgfusepath{stroke,fill}%
\end{pgfscope}%
\begin{pgfscope}%
\pgfpathrectangle{\pgfqpoint{0.532717in}{0.370679in}}{\pgfqpoint{2.617283in}{1.479321in}}%
\pgfusepath{clip}%
\pgfsetbuttcap%
\pgfsetroundjoin%
\definecolor{currentfill}{rgb}{1.000000,0.411765,0.380392}%
\pgfsetfillcolor{currentfill}%
\pgfsetfillopacity{0.500000}%
\pgfsetlinewidth{1.003750pt}%
\definecolor{currentstroke}{rgb}{1.000000,0.411765,0.380392}%
\pgfsetstrokecolor{currentstroke}%
\pgfsetstrokeopacity{0.500000}%
\pgfsetdash{}{0pt}%
\pgfpathmoveto{\pgfqpoint{0.981614in}{0.950395in}}%
\pgfpathcurveto{\pgfqpoint{0.992664in}{0.950395in}}{\pgfqpoint{1.003263in}{0.954786in}}{\pgfqpoint{1.011077in}{0.962599in}}%
\pgfpathcurveto{\pgfqpoint{1.018890in}{0.970413in}}{\pgfqpoint{1.023281in}{0.981012in}}{\pgfqpoint{1.023281in}{0.992062in}}%
\pgfpathcurveto{\pgfqpoint{1.023281in}{1.003112in}}{\pgfqpoint{1.018890in}{1.013711in}}{\pgfqpoint{1.011077in}{1.021525in}}%
\pgfpathcurveto{\pgfqpoint{1.003263in}{1.029338in}}{\pgfqpoint{0.992664in}{1.033729in}}{\pgfqpoint{0.981614in}{1.033729in}}%
\pgfpathcurveto{\pgfqpoint{0.970564in}{1.033729in}}{\pgfqpoint{0.959965in}{1.029338in}}{\pgfqpoint{0.952151in}{1.021525in}}%
\pgfpathcurveto{\pgfqpoint{0.944338in}{1.013711in}}{\pgfqpoint{0.939947in}{1.003112in}}{\pgfqpoint{0.939947in}{0.992062in}}%
\pgfpathcurveto{\pgfqpoint{0.939947in}{0.981012in}}{\pgfqpoint{0.944338in}{0.970413in}}{\pgfqpoint{0.952151in}{0.962599in}}%
\pgfpathcurveto{\pgfqpoint{0.959965in}{0.954786in}}{\pgfqpoint{0.970564in}{0.950395in}}{\pgfqpoint{0.981614in}{0.950395in}}%
\pgfpathclose%
\pgfusepath{stroke,fill}%
\end{pgfscope}%
\begin{pgfscope}%
\pgfpathrectangle{\pgfqpoint{0.532717in}{0.370679in}}{\pgfqpoint{2.617283in}{1.479321in}}%
\pgfusepath{clip}%
\pgfsetbuttcap%
\pgfsetroundjoin%
\definecolor{currentfill}{rgb}{1.000000,0.411765,0.380392}%
\pgfsetfillcolor{currentfill}%
\pgfsetfillopacity{0.500000}%
\pgfsetlinewidth{1.003750pt}%
\definecolor{currentstroke}{rgb}{1.000000,0.411765,0.380392}%
\pgfsetstrokecolor{currentstroke}%
\pgfsetstrokeopacity{0.500000}%
\pgfsetdash{}{0pt}%
\pgfpathmoveto{\pgfqpoint{0.983711in}{1.183533in}}%
\pgfpathcurveto{\pgfqpoint{0.994761in}{1.183533in}}{\pgfqpoint{1.005360in}{1.187924in}}{\pgfqpoint{1.013174in}{1.195737in}}%
\pgfpathcurveto{\pgfqpoint{1.020987in}{1.203551in}}{\pgfqpoint{1.025378in}{1.214150in}}{\pgfqpoint{1.025378in}{1.225200in}}%
\pgfpathcurveto{\pgfqpoint{1.025378in}{1.236250in}}{\pgfqpoint{1.020987in}{1.246849in}}{\pgfqpoint{1.013174in}{1.254663in}}%
\pgfpathcurveto{\pgfqpoint{1.005360in}{1.262476in}}{\pgfqpoint{0.994761in}{1.266867in}}{\pgfqpoint{0.983711in}{1.266867in}}%
\pgfpathcurveto{\pgfqpoint{0.972661in}{1.266867in}}{\pgfqpoint{0.962062in}{1.262476in}}{\pgfqpoint{0.954248in}{1.254663in}}%
\pgfpathcurveto{\pgfqpoint{0.946435in}{1.246849in}}{\pgfqpoint{0.942044in}{1.236250in}}{\pgfqpoint{0.942044in}{1.225200in}}%
\pgfpathcurveto{\pgfqpoint{0.942044in}{1.214150in}}{\pgfqpoint{0.946435in}{1.203551in}}{\pgfqpoint{0.954248in}{1.195737in}}%
\pgfpathcurveto{\pgfqpoint{0.962062in}{1.187924in}}{\pgfqpoint{0.972661in}{1.183533in}}{\pgfqpoint{0.983711in}{1.183533in}}%
\pgfpathclose%
\pgfusepath{stroke,fill}%
\end{pgfscope}%
\begin{pgfscope}%
\pgfpathrectangle{\pgfqpoint{0.532717in}{0.370679in}}{\pgfqpoint{2.617283in}{1.479321in}}%
\pgfusepath{clip}%
\pgfsetbuttcap%
\pgfsetroundjoin%
\definecolor{currentfill}{rgb}{1.000000,0.411765,0.380392}%
\pgfsetfillcolor{currentfill}%
\pgfsetfillopacity{0.500000}%
\pgfsetlinewidth{1.003750pt}%
\definecolor{currentstroke}{rgb}{1.000000,0.411765,0.380392}%
\pgfsetstrokecolor{currentstroke}%
\pgfsetstrokeopacity{0.500000}%
\pgfsetdash{}{0pt}%
\pgfpathmoveto{\pgfqpoint{0.997058in}{1.186883in}}%
\pgfpathcurveto{\pgfqpoint{1.008108in}{1.186883in}}{\pgfqpoint{1.018707in}{1.191273in}}{\pgfqpoint{1.026521in}{1.199086in}}%
\pgfpathcurveto{\pgfqpoint{1.034335in}{1.206900in}}{\pgfqpoint{1.038725in}{1.217499in}}{\pgfqpoint{1.038725in}{1.228549in}}%
\pgfpathcurveto{\pgfqpoint{1.038725in}{1.239599in}}{\pgfqpoint{1.034335in}{1.250198in}}{\pgfqpoint{1.026521in}{1.258012in}}%
\pgfpathcurveto{\pgfqpoint{1.018707in}{1.265826in}}{\pgfqpoint{1.008108in}{1.270216in}}{\pgfqpoint{0.997058in}{1.270216in}}%
\pgfpathcurveto{\pgfqpoint{0.986008in}{1.270216in}}{\pgfqpoint{0.975409in}{1.265826in}}{\pgfqpoint{0.967595in}{1.258012in}}%
\pgfpathcurveto{\pgfqpoint{0.959782in}{1.250198in}}{\pgfqpoint{0.955392in}{1.239599in}}{\pgfqpoint{0.955392in}{1.228549in}}%
\pgfpathcurveto{\pgfqpoint{0.955392in}{1.217499in}}{\pgfqpoint{0.959782in}{1.206900in}}{\pgfqpoint{0.967595in}{1.199086in}}%
\pgfpathcurveto{\pgfqpoint{0.975409in}{1.191273in}}{\pgfqpoint{0.986008in}{1.186883in}}{\pgfqpoint{0.997058in}{1.186883in}}%
\pgfpathclose%
\pgfusepath{stroke,fill}%
\end{pgfscope}%
\begin{pgfscope}%
\pgfpathrectangle{\pgfqpoint{0.532717in}{0.370679in}}{\pgfqpoint{2.617283in}{1.479321in}}%
\pgfusepath{clip}%
\pgfsetbuttcap%
\pgfsetroundjoin%
\definecolor{currentfill}{rgb}{1.000000,0.411765,0.380392}%
\pgfsetfillcolor{currentfill}%
\pgfsetfillopacity{0.500000}%
\pgfsetlinewidth{1.003750pt}%
\definecolor{currentstroke}{rgb}{1.000000,0.411765,0.380392}%
\pgfsetstrokecolor{currentstroke}%
\pgfsetstrokeopacity{0.500000}%
\pgfsetdash{}{0pt}%
\pgfpathmoveto{\pgfqpoint{1.050700in}{0.905179in}}%
\pgfpathcurveto{\pgfqpoint{1.061751in}{0.905179in}}{\pgfqpoint{1.072350in}{0.909569in}}{\pgfqpoint{1.080163in}{0.917383in}}%
\pgfpathcurveto{\pgfqpoint{1.087977in}{0.925196in}}{\pgfqpoint{1.092367in}{0.935796in}}{\pgfqpoint{1.092367in}{0.946846in}}%
\pgfpathcurveto{\pgfqpoint{1.092367in}{0.957896in}}{\pgfqpoint{1.087977in}{0.968495in}}{\pgfqpoint{1.080163in}{0.976308in}}%
\pgfpathcurveto{\pgfqpoint{1.072350in}{0.984122in}}{\pgfqpoint{1.061751in}{0.988512in}}{\pgfqpoint{1.050700in}{0.988512in}}%
\pgfpathcurveto{\pgfqpoint{1.039650in}{0.988512in}}{\pgfqpoint{1.029051in}{0.984122in}}{\pgfqpoint{1.021238in}{0.976308in}}%
\pgfpathcurveto{\pgfqpoint{1.013424in}{0.968495in}}{\pgfqpoint{1.009034in}{0.957896in}}{\pgfqpoint{1.009034in}{0.946846in}}%
\pgfpathcurveto{\pgfqpoint{1.009034in}{0.935796in}}{\pgfqpoint{1.013424in}{0.925196in}}{\pgfqpoint{1.021238in}{0.917383in}}%
\pgfpathcurveto{\pgfqpoint{1.029051in}{0.909569in}}{\pgfqpoint{1.039650in}{0.905179in}}{\pgfqpoint{1.050700in}{0.905179in}}%
\pgfpathclose%
\pgfusepath{stroke,fill}%
\end{pgfscope}%
\begin{pgfscope}%
\pgfpathrectangle{\pgfqpoint{0.532717in}{0.370679in}}{\pgfqpoint{2.617283in}{1.479321in}}%
\pgfusepath{clip}%
\pgfsetbuttcap%
\pgfsetroundjoin%
\definecolor{currentfill}{rgb}{1.000000,0.411765,0.380392}%
\pgfsetfillcolor{currentfill}%
\pgfsetfillopacity{0.500000}%
\pgfsetlinewidth{1.003750pt}%
\definecolor{currentstroke}{rgb}{1.000000,0.411765,0.380392}%
\pgfsetstrokecolor{currentstroke}%
\pgfsetstrokeopacity{0.500000}%
\pgfsetdash{}{0pt}%
\pgfpathmoveto{\pgfqpoint{0.979264in}{0.721227in}}%
\pgfpathcurveto{\pgfqpoint{0.990314in}{0.721227in}}{\pgfqpoint{1.000913in}{0.725617in}}{\pgfqpoint{1.008727in}{0.733431in}}%
\pgfpathcurveto{\pgfqpoint{1.016540in}{0.741245in}}{\pgfqpoint{1.020931in}{0.751844in}}{\pgfqpoint{1.020931in}{0.762894in}}%
\pgfpathcurveto{\pgfqpoint{1.020931in}{0.773944in}}{\pgfqpoint{1.016540in}{0.784543in}}{\pgfqpoint{1.008727in}{0.792357in}}%
\pgfpathcurveto{\pgfqpoint{1.000913in}{0.800170in}}{\pgfqpoint{0.990314in}{0.804561in}}{\pgfqpoint{0.979264in}{0.804561in}}%
\pgfpathcurveto{\pgfqpoint{0.968214in}{0.804561in}}{\pgfqpoint{0.957615in}{0.800170in}}{\pgfqpoint{0.949801in}{0.792357in}}%
\pgfpathcurveto{\pgfqpoint{0.941987in}{0.784543in}}{\pgfqpoint{0.937597in}{0.773944in}}{\pgfqpoint{0.937597in}{0.762894in}}%
\pgfpathcurveto{\pgfqpoint{0.937597in}{0.751844in}}{\pgfqpoint{0.941987in}{0.741245in}}{\pgfqpoint{0.949801in}{0.733431in}}%
\pgfpathcurveto{\pgfqpoint{0.957615in}{0.725617in}}{\pgfqpoint{0.968214in}{0.721227in}}{\pgfqpoint{0.979264in}{0.721227in}}%
\pgfpathclose%
\pgfusepath{stroke,fill}%
\end{pgfscope}%
\begin{pgfscope}%
\pgfpathrectangle{\pgfqpoint{0.532717in}{0.370679in}}{\pgfqpoint{2.617283in}{1.479321in}}%
\pgfusepath{clip}%
\pgfsetbuttcap%
\pgfsetroundjoin%
\definecolor{currentfill}{rgb}{1.000000,0.411765,0.380392}%
\pgfsetfillcolor{currentfill}%
\pgfsetfillopacity{0.500000}%
\pgfsetlinewidth{1.003750pt}%
\definecolor{currentstroke}{rgb}{1.000000,0.411765,0.380392}%
\pgfsetstrokecolor{currentstroke}%
\pgfsetstrokeopacity{0.500000}%
\pgfsetdash{}{0pt}%
\pgfpathmoveto{\pgfqpoint{0.994443in}{0.593053in}}%
\pgfpathcurveto{\pgfqpoint{1.005493in}{0.593053in}}{\pgfqpoint{1.016092in}{0.597443in}}{\pgfqpoint{1.023906in}{0.605256in}}%
\pgfpathcurveto{\pgfqpoint{1.031719in}{0.613070in}}{\pgfqpoint{1.036110in}{0.623669in}}{\pgfqpoint{1.036110in}{0.634719in}}%
\pgfpathcurveto{\pgfqpoint{1.036110in}{0.645769in}}{\pgfqpoint{1.031719in}{0.656368in}}{\pgfqpoint{1.023906in}{0.664182in}}%
\pgfpathcurveto{\pgfqpoint{1.016092in}{0.671996in}}{\pgfqpoint{1.005493in}{0.676386in}}{\pgfqpoint{0.994443in}{0.676386in}}%
\pgfpathcurveto{\pgfqpoint{0.983393in}{0.676386in}}{\pgfqpoint{0.972794in}{0.671996in}}{\pgfqpoint{0.964980in}{0.664182in}}%
\pgfpathcurveto{\pgfqpoint{0.957166in}{0.656368in}}{\pgfqpoint{0.952776in}{0.645769in}}{\pgfqpoint{0.952776in}{0.634719in}}%
\pgfpathcurveto{\pgfqpoint{0.952776in}{0.623669in}}{\pgfqpoint{0.957166in}{0.613070in}}{\pgfqpoint{0.964980in}{0.605256in}}%
\pgfpathcurveto{\pgfqpoint{0.972794in}{0.597443in}}{\pgfqpoint{0.983393in}{0.593053in}}{\pgfqpoint{0.994443in}{0.593053in}}%
\pgfpathclose%
\pgfusepath{stroke,fill}%
\end{pgfscope}%
\begin{pgfscope}%
\pgfpathrectangle{\pgfqpoint{0.532717in}{0.370679in}}{\pgfqpoint{2.617283in}{1.479321in}}%
\pgfusepath{clip}%
\pgfsetbuttcap%
\pgfsetroundjoin%
\definecolor{currentfill}{rgb}{1.000000,0.411765,0.380392}%
\pgfsetfillcolor{currentfill}%
\pgfsetfillopacity{0.500000}%
\pgfsetlinewidth{1.003750pt}%
\definecolor{currentstroke}{rgb}{1.000000,0.411765,0.380392}%
\pgfsetstrokecolor{currentstroke}%
\pgfsetstrokeopacity{0.500000}%
\pgfsetdash{}{0pt}%
\pgfpathmoveto{\pgfqpoint{0.971383in}{1.145368in}}%
\pgfpathcurveto{\pgfqpoint{0.982433in}{1.145368in}}{\pgfqpoint{0.993032in}{1.149759in}}{\pgfqpoint{1.000846in}{1.157572in}}%
\pgfpathcurveto{\pgfqpoint{1.008659in}{1.165386in}}{\pgfqpoint{1.013049in}{1.175985in}}{\pgfqpoint{1.013049in}{1.187035in}}%
\pgfpathcurveto{\pgfqpoint{1.013049in}{1.198085in}}{\pgfqpoint{1.008659in}{1.208684in}}{\pgfqpoint{1.000846in}{1.216498in}}%
\pgfpathcurveto{\pgfqpoint{0.993032in}{1.224311in}}{\pgfqpoint{0.982433in}{1.228702in}}{\pgfqpoint{0.971383in}{1.228702in}}%
\pgfpathcurveto{\pgfqpoint{0.960333in}{1.228702in}}{\pgfqpoint{0.949734in}{1.224311in}}{\pgfqpoint{0.941920in}{1.216498in}}%
\pgfpathcurveto{\pgfqpoint{0.934106in}{1.208684in}}{\pgfqpoint{0.929716in}{1.198085in}}{\pgfqpoint{0.929716in}{1.187035in}}%
\pgfpathcurveto{\pgfqpoint{0.929716in}{1.175985in}}{\pgfqpoint{0.934106in}{1.165386in}}{\pgfqpoint{0.941920in}{1.157572in}}%
\pgfpathcurveto{\pgfqpoint{0.949734in}{1.149759in}}{\pgfqpoint{0.960333in}{1.145368in}}{\pgfqpoint{0.971383in}{1.145368in}}%
\pgfpathclose%
\pgfusepath{stroke,fill}%
\end{pgfscope}%
\begin{pgfscope}%
\pgfpathrectangle{\pgfqpoint{0.532717in}{0.370679in}}{\pgfqpoint{2.617283in}{1.479321in}}%
\pgfusepath{clip}%
\pgfsetbuttcap%
\pgfsetroundjoin%
\definecolor{currentfill}{rgb}{1.000000,0.411765,0.380392}%
\pgfsetfillcolor{currentfill}%
\pgfsetfillopacity{0.500000}%
\pgfsetlinewidth{1.003750pt}%
\definecolor{currentstroke}{rgb}{1.000000,0.411765,0.380392}%
\pgfsetstrokecolor{currentstroke}%
\pgfsetstrokeopacity{0.500000}%
\pgfsetdash{}{0pt}%
\pgfpathmoveto{\pgfqpoint{0.913299in}{1.061204in}}%
\pgfpathcurveto{\pgfqpoint{0.924349in}{1.061204in}}{\pgfqpoint{0.934948in}{1.065594in}}{\pgfqpoint{0.942762in}{1.073408in}}%
\pgfpathcurveto{\pgfqpoint{0.950575in}{1.081221in}}{\pgfqpoint{0.954966in}{1.091820in}}{\pgfqpoint{0.954966in}{1.102870in}}%
\pgfpathcurveto{\pgfqpoint{0.954966in}{1.113921in}}{\pgfqpoint{0.950575in}{1.124520in}}{\pgfqpoint{0.942762in}{1.132333in}}%
\pgfpathcurveto{\pgfqpoint{0.934948in}{1.140147in}}{\pgfqpoint{0.924349in}{1.144537in}}{\pgfqpoint{0.913299in}{1.144537in}}%
\pgfpathcurveto{\pgfqpoint{0.902249in}{1.144537in}}{\pgfqpoint{0.891650in}{1.140147in}}{\pgfqpoint{0.883836in}{1.132333in}}%
\pgfpathcurveto{\pgfqpoint{0.876023in}{1.124520in}}{\pgfqpoint{0.871632in}{1.113921in}}{\pgfqpoint{0.871632in}{1.102870in}}%
\pgfpathcurveto{\pgfqpoint{0.871632in}{1.091820in}}{\pgfqpoint{0.876023in}{1.081221in}}{\pgfqpoint{0.883836in}{1.073408in}}%
\pgfpathcurveto{\pgfqpoint{0.891650in}{1.065594in}}{\pgfqpoint{0.902249in}{1.061204in}}{\pgfqpoint{0.913299in}{1.061204in}}%
\pgfpathclose%
\pgfusepath{stroke,fill}%
\end{pgfscope}%
\begin{pgfscope}%
\pgfpathrectangle{\pgfqpoint{0.532717in}{0.370679in}}{\pgfqpoint{2.617283in}{1.479321in}}%
\pgfusepath{clip}%
\pgfsetbuttcap%
\pgfsetroundjoin%
\definecolor{currentfill}{rgb}{1.000000,0.411765,0.380392}%
\pgfsetfillcolor{currentfill}%
\pgfsetfillopacity{0.500000}%
\pgfsetlinewidth{1.003750pt}%
\definecolor{currentstroke}{rgb}{1.000000,0.411765,0.380392}%
\pgfsetstrokecolor{currentstroke}%
\pgfsetstrokeopacity{0.500000}%
\pgfsetdash{}{0pt}%
\pgfpathmoveto{\pgfqpoint{0.986002in}{0.818801in}}%
\pgfpathcurveto{\pgfqpoint{0.997052in}{0.818801in}}{\pgfqpoint{1.007651in}{0.823192in}}{\pgfqpoint{1.015465in}{0.831005in}}%
\pgfpathcurveto{\pgfqpoint{1.023279in}{0.838819in}}{\pgfqpoint{1.027669in}{0.849418in}}{\pgfqpoint{1.027669in}{0.860468in}}%
\pgfpathcurveto{\pgfqpoint{1.027669in}{0.871518in}}{\pgfqpoint{1.023279in}{0.882117in}}{\pgfqpoint{1.015465in}{0.889931in}}%
\pgfpathcurveto{\pgfqpoint{1.007651in}{0.897744in}}{\pgfqpoint{0.997052in}{0.902135in}}{\pgfqpoint{0.986002in}{0.902135in}}%
\pgfpathcurveto{\pgfqpoint{0.974952in}{0.902135in}}{\pgfqpoint{0.964353in}{0.897744in}}{\pgfqpoint{0.956539in}{0.889931in}}%
\pgfpathcurveto{\pgfqpoint{0.948726in}{0.882117in}}{\pgfqpoint{0.944336in}{0.871518in}}{\pgfqpoint{0.944336in}{0.860468in}}%
\pgfpathcurveto{\pgfqpoint{0.944336in}{0.849418in}}{\pgfqpoint{0.948726in}{0.838819in}}{\pgfqpoint{0.956539in}{0.831005in}}%
\pgfpathcurveto{\pgfqpoint{0.964353in}{0.823192in}}{\pgfqpoint{0.974952in}{0.818801in}}{\pgfqpoint{0.986002in}{0.818801in}}%
\pgfpathclose%
\pgfusepath{stroke,fill}%
\end{pgfscope}%
\begin{pgfscope}%
\pgfpathrectangle{\pgfqpoint{0.532717in}{0.370679in}}{\pgfqpoint{2.617283in}{1.479321in}}%
\pgfusepath{clip}%
\pgfsetbuttcap%
\pgfsetroundjoin%
\definecolor{currentfill}{rgb}{1.000000,0.411765,0.380392}%
\pgfsetfillcolor{currentfill}%
\pgfsetfillopacity{0.500000}%
\pgfsetlinewidth{1.003750pt}%
\definecolor{currentstroke}{rgb}{1.000000,0.411765,0.380392}%
\pgfsetstrokecolor{currentstroke}%
\pgfsetstrokeopacity{0.500000}%
\pgfsetdash{}{0pt}%
\pgfpathmoveto{\pgfqpoint{0.971383in}{1.145368in}}%
\pgfpathcurveto{\pgfqpoint{0.982433in}{1.145368in}}{\pgfqpoint{0.993032in}{1.149759in}}{\pgfqpoint{1.000846in}{1.157572in}}%
\pgfpathcurveto{\pgfqpoint{1.008659in}{1.165386in}}{\pgfqpoint{1.013049in}{1.175985in}}{\pgfqpoint{1.013049in}{1.187035in}}%
\pgfpathcurveto{\pgfqpoint{1.013049in}{1.198085in}}{\pgfqpoint{1.008659in}{1.208684in}}{\pgfqpoint{1.000846in}{1.216498in}}%
\pgfpathcurveto{\pgfqpoint{0.993032in}{1.224311in}}{\pgfqpoint{0.982433in}{1.228702in}}{\pgfqpoint{0.971383in}{1.228702in}}%
\pgfpathcurveto{\pgfqpoint{0.960333in}{1.228702in}}{\pgfqpoint{0.949734in}{1.224311in}}{\pgfqpoint{0.941920in}{1.216498in}}%
\pgfpathcurveto{\pgfqpoint{0.934106in}{1.208684in}}{\pgfqpoint{0.929716in}{1.198085in}}{\pgfqpoint{0.929716in}{1.187035in}}%
\pgfpathcurveto{\pgfqpoint{0.929716in}{1.175985in}}{\pgfqpoint{0.934106in}{1.165386in}}{\pgfqpoint{0.941920in}{1.157572in}}%
\pgfpathcurveto{\pgfqpoint{0.949734in}{1.149759in}}{\pgfqpoint{0.960333in}{1.145368in}}{\pgfqpoint{0.971383in}{1.145368in}}%
\pgfpathclose%
\pgfusepath{stroke,fill}%
\end{pgfscope}%
\begin{pgfscope}%
\pgfpathrectangle{\pgfqpoint{0.532717in}{0.370679in}}{\pgfqpoint{2.617283in}{1.479321in}}%
\pgfusepath{clip}%
\pgfsetbuttcap%
\pgfsetroundjoin%
\definecolor{currentfill}{rgb}{1.000000,0.411765,0.380392}%
\pgfsetfillcolor{currentfill}%
\pgfsetfillopacity{0.500000}%
\pgfsetlinewidth{1.003750pt}%
\definecolor{currentstroke}{rgb}{1.000000,0.411765,0.380392}%
\pgfsetstrokecolor{currentstroke}%
\pgfsetstrokeopacity{0.500000}%
\pgfsetdash{}{0pt}%
\pgfpathmoveto{\pgfqpoint{0.878765in}{1.316347in}}%
\pgfpathcurveto{\pgfqpoint{0.889815in}{1.316347in}}{\pgfqpoint{0.900414in}{1.320737in}}{\pgfqpoint{0.908227in}{1.328551in}}%
\pgfpathcurveto{\pgfqpoint{0.916041in}{1.336364in}}{\pgfqpoint{0.920431in}{1.346963in}}{\pgfqpoint{0.920431in}{1.358014in}}%
\pgfpathcurveto{\pgfqpoint{0.920431in}{1.369064in}}{\pgfqpoint{0.916041in}{1.379663in}}{\pgfqpoint{0.908227in}{1.387476in}}%
\pgfpathcurveto{\pgfqpoint{0.900414in}{1.395290in}}{\pgfqpoint{0.889815in}{1.399680in}}{\pgfqpoint{0.878765in}{1.399680in}}%
\pgfpathcurveto{\pgfqpoint{0.867714in}{1.399680in}}{\pgfqpoint{0.857115in}{1.395290in}}{\pgfqpoint{0.849302in}{1.387476in}}%
\pgfpathcurveto{\pgfqpoint{0.841488in}{1.379663in}}{\pgfqpoint{0.837098in}{1.369064in}}{\pgfqpoint{0.837098in}{1.358014in}}%
\pgfpathcurveto{\pgfqpoint{0.837098in}{1.346963in}}{\pgfqpoint{0.841488in}{1.336364in}}{\pgfqpoint{0.849302in}{1.328551in}}%
\pgfpathcurveto{\pgfqpoint{0.857115in}{1.320737in}}{\pgfqpoint{0.867714in}{1.316347in}}{\pgfqpoint{0.878765in}{1.316347in}}%
\pgfpathclose%
\pgfusepath{stroke,fill}%
\end{pgfscope}%
\begin{pgfscope}%
\pgfpathrectangle{\pgfqpoint{0.532717in}{0.370679in}}{\pgfqpoint{2.617283in}{1.479321in}}%
\pgfusepath{clip}%
\pgfsetbuttcap%
\pgfsetroundjoin%
\definecolor{currentfill}{rgb}{1.000000,0.411765,0.380392}%
\pgfsetfillcolor{currentfill}%
\pgfsetfillopacity{0.500000}%
\pgfsetlinewidth{1.003750pt}%
\definecolor{currentstroke}{rgb}{1.000000,0.411765,0.380392}%
\pgfsetstrokecolor{currentstroke}%
\pgfsetstrokeopacity{0.500000}%
\pgfsetdash{}{0pt}%
\pgfpathmoveto{\pgfqpoint{0.996499in}{0.988490in}}%
\pgfpathcurveto{\pgfqpoint{1.007549in}{0.988490in}}{\pgfqpoint{1.018148in}{0.992881in}}{\pgfqpoint{1.025961in}{1.000694in}}%
\pgfpathcurveto{\pgfqpoint{1.033775in}{1.008508in}}{\pgfqpoint{1.038165in}{1.019107in}}{\pgfqpoint{1.038165in}{1.030157in}}%
\pgfpathcurveto{\pgfqpoint{1.038165in}{1.041207in}}{\pgfqpoint{1.033775in}{1.051806in}}{\pgfqpoint{1.025961in}{1.059620in}}%
\pgfpathcurveto{\pgfqpoint{1.018148in}{1.067433in}}{\pgfqpoint{1.007549in}{1.071824in}}{\pgfqpoint{0.996499in}{1.071824in}}%
\pgfpathcurveto{\pgfqpoint{0.985449in}{1.071824in}}{\pgfqpoint{0.974849in}{1.067433in}}{\pgfqpoint{0.967036in}{1.059620in}}%
\pgfpathcurveto{\pgfqpoint{0.959222in}{1.051806in}}{\pgfqpoint{0.954832in}{1.041207in}}{\pgfqpoint{0.954832in}{1.030157in}}%
\pgfpathcurveto{\pgfqpoint{0.954832in}{1.019107in}}{\pgfqpoint{0.959222in}{1.008508in}}{\pgfqpoint{0.967036in}{1.000694in}}%
\pgfpathcurveto{\pgfqpoint{0.974849in}{0.992881in}}{\pgfqpoint{0.985449in}{0.988490in}}{\pgfqpoint{0.996499in}{0.988490in}}%
\pgfpathclose%
\pgfusepath{stroke,fill}%
\end{pgfscope}%
\begin{pgfscope}%
\pgfpathrectangle{\pgfqpoint{0.532717in}{0.370679in}}{\pgfqpoint{2.617283in}{1.479321in}}%
\pgfusepath{clip}%
\pgfsetbuttcap%
\pgfsetroundjoin%
\definecolor{currentfill}{rgb}{1.000000,0.411765,0.380392}%
\pgfsetfillcolor{currentfill}%
\pgfsetfillopacity{0.500000}%
\pgfsetlinewidth{1.003750pt}%
\definecolor{currentstroke}{rgb}{1.000000,0.411765,0.380392}%
\pgfsetstrokecolor{currentstroke}%
\pgfsetstrokeopacity{0.500000}%
\pgfsetdash{}{0pt}%
\pgfpathmoveto{\pgfqpoint{0.942426in}{0.972471in}}%
\pgfpathcurveto{\pgfqpoint{0.953476in}{0.972471in}}{\pgfqpoint{0.964075in}{0.976861in}}{\pgfqpoint{0.971889in}{0.984675in}}%
\pgfpathcurveto{\pgfqpoint{0.979703in}{0.992488in}}{\pgfqpoint{0.984093in}{1.003087in}}{\pgfqpoint{0.984093in}{1.014137in}}%
\pgfpathcurveto{\pgfqpoint{0.984093in}{1.025187in}}{\pgfqpoint{0.979703in}{1.035787in}}{\pgfqpoint{0.971889in}{1.043600in}}%
\pgfpathcurveto{\pgfqpoint{0.964075in}{1.051414in}}{\pgfqpoint{0.953476in}{1.055804in}}{\pgfqpoint{0.942426in}{1.055804in}}%
\pgfpathcurveto{\pgfqpoint{0.931376in}{1.055804in}}{\pgfqpoint{0.920777in}{1.051414in}}{\pgfqpoint{0.912963in}{1.043600in}}%
\pgfpathcurveto{\pgfqpoint{0.905150in}{1.035787in}}{\pgfqpoint{0.900759in}{1.025187in}}{\pgfqpoint{0.900759in}{1.014137in}}%
\pgfpathcurveto{\pgfqpoint{0.900759in}{1.003087in}}{\pgfqpoint{0.905150in}{0.992488in}}{\pgfqpoint{0.912963in}{0.984675in}}%
\pgfpathcurveto{\pgfqpoint{0.920777in}{0.976861in}}{\pgfqpoint{0.931376in}{0.972471in}}{\pgfqpoint{0.942426in}{0.972471in}}%
\pgfpathclose%
\pgfusepath{stroke,fill}%
\end{pgfscope}%
\begin{pgfscope}%
\pgfpathrectangle{\pgfqpoint{0.532717in}{0.370679in}}{\pgfqpoint{2.617283in}{1.479321in}}%
\pgfusepath{clip}%
\pgfsetbuttcap%
\pgfsetroundjoin%
\definecolor{currentfill}{rgb}{1.000000,0.411765,0.380392}%
\pgfsetfillcolor{currentfill}%
\pgfsetfillopacity{0.500000}%
\pgfsetlinewidth{1.003750pt}%
\definecolor{currentstroke}{rgb}{1.000000,0.411765,0.380392}%
\pgfsetstrokecolor{currentstroke}%
\pgfsetstrokeopacity{0.500000}%
\pgfsetdash{}{0pt}%
\pgfpathmoveto{\pgfqpoint{0.917746in}{1.523510in}}%
\pgfpathcurveto{\pgfqpoint{0.928796in}{1.523510in}}{\pgfqpoint{0.939395in}{1.527900in}}{\pgfqpoint{0.947209in}{1.535714in}}%
\pgfpathcurveto{\pgfqpoint{0.955023in}{1.543527in}}{\pgfqpoint{0.959413in}{1.554127in}}{\pgfqpoint{0.959413in}{1.565177in}}%
\pgfpathcurveto{\pgfqpoint{0.959413in}{1.576227in}}{\pgfqpoint{0.955023in}{1.586826in}}{\pgfqpoint{0.947209in}{1.594639in}}%
\pgfpathcurveto{\pgfqpoint{0.939395in}{1.602453in}}{\pgfqpoint{0.928796in}{1.606843in}}{\pgfqpoint{0.917746in}{1.606843in}}%
\pgfpathcurveto{\pgfqpoint{0.906696in}{1.606843in}}{\pgfqpoint{0.896097in}{1.602453in}}{\pgfqpoint{0.888283in}{1.594639in}}%
\pgfpathcurveto{\pgfqpoint{0.880470in}{1.586826in}}{\pgfqpoint{0.876080in}{1.576227in}}{\pgfqpoint{0.876080in}{1.565177in}}%
\pgfpathcurveto{\pgfqpoint{0.876080in}{1.554127in}}{\pgfqpoint{0.880470in}{1.543527in}}{\pgfqpoint{0.888283in}{1.535714in}}%
\pgfpathcurveto{\pgfqpoint{0.896097in}{1.527900in}}{\pgfqpoint{0.906696in}{1.523510in}}{\pgfqpoint{0.917746in}{1.523510in}}%
\pgfpathclose%
\pgfusepath{stroke,fill}%
\end{pgfscope}%
\begin{pgfscope}%
\pgfpathrectangle{\pgfqpoint{0.532717in}{0.370679in}}{\pgfqpoint{2.617283in}{1.479321in}}%
\pgfusepath{clip}%
\pgfsetbuttcap%
\pgfsetroundjoin%
\definecolor{currentfill}{rgb}{1.000000,0.411765,0.380392}%
\pgfsetfillcolor{currentfill}%
\pgfsetfillopacity{0.500000}%
\pgfsetlinewidth{1.003750pt}%
\definecolor{currentstroke}{rgb}{1.000000,0.411765,0.380392}%
\pgfsetstrokecolor{currentstroke}%
\pgfsetstrokeopacity{0.500000}%
\pgfsetdash{}{0pt}%
\pgfpathmoveto{\pgfqpoint{0.873817in}{1.249549in}}%
\pgfpathcurveto{\pgfqpoint{0.884867in}{1.249549in}}{\pgfqpoint{0.895466in}{1.253939in}}{\pgfqpoint{0.903280in}{1.261753in}}%
\pgfpathcurveto{\pgfqpoint{0.911093in}{1.269566in}}{\pgfqpoint{0.915483in}{1.280165in}}{\pgfqpoint{0.915483in}{1.291215in}}%
\pgfpathcurveto{\pgfqpoint{0.915483in}{1.302265in}}{\pgfqpoint{0.911093in}{1.312864in}}{\pgfqpoint{0.903280in}{1.320678in}}%
\pgfpathcurveto{\pgfqpoint{0.895466in}{1.328492in}}{\pgfqpoint{0.884867in}{1.332882in}}{\pgfqpoint{0.873817in}{1.332882in}}%
\pgfpathcurveto{\pgfqpoint{0.862767in}{1.332882in}}{\pgfqpoint{0.852168in}{1.328492in}}{\pgfqpoint{0.844354in}{1.320678in}}%
\pgfpathcurveto{\pgfqpoint{0.836540in}{1.312864in}}{\pgfqpoint{0.832150in}{1.302265in}}{\pgfqpoint{0.832150in}{1.291215in}}%
\pgfpathcurveto{\pgfqpoint{0.832150in}{1.280165in}}{\pgfqpoint{0.836540in}{1.269566in}}{\pgfqpoint{0.844354in}{1.261753in}}%
\pgfpathcurveto{\pgfqpoint{0.852168in}{1.253939in}}{\pgfqpoint{0.862767in}{1.249549in}}{\pgfqpoint{0.873817in}{1.249549in}}%
\pgfpathclose%
\pgfusepath{stroke,fill}%
\end{pgfscope}%
\begin{pgfscope}%
\pgfpathrectangle{\pgfqpoint{0.532717in}{0.370679in}}{\pgfqpoint{2.617283in}{1.479321in}}%
\pgfusepath{clip}%
\pgfsetbuttcap%
\pgfsetroundjoin%
\definecolor{currentfill}{rgb}{1.000000,0.411765,0.380392}%
\pgfsetfillcolor{currentfill}%
\pgfsetfillopacity{0.500000}%
\pgfsetlinewidth{1.003750pt}%
\definecolor{currentstroke}{rgb}{1.000000,0.411765,0.380392}%
\pgfsetstrokecolor{currentstroke}%
\pgfsetstrokeopacity{0.500000}%
\pgfsetdash{}{0pt}%
\pgfpathmoveto{\pgfqpoint{1.052073in}{1.006863in}}%
\pgfpathcurveto{\pgfqpoint{1.063123in}{1.006863in}}{\pgfqpoint{1.073722in}{1.011254in}}{\pgfqpoint{1.081535in}{1.019067in}}%
\pgfpathcurveto{\pgfqpoint{1.089349in}{1.026881in}}{\pgfqpoint{1.093739in}{1.037480in}}{\pgfqpoint{1.093739in}{1.048530in}}%
\pgfpathcurveto{\pgfqpoint{1.093739in}{1.059580in}}{\pgfqpoint{1.089349in}{1.070179in}}{\pgfqpoint{1.081535in}{1.077993in}}%
\pgfpathcurveto{\pgfqpoint{1.073722in}{1.085806in}}{\pgfqpoint{1.063123in}{1.090197in}}{\pgfqpoint{1.052073in}{1.090197in}}%
\pgfpathcurveto{\pgfqpoint{1.041023in}{1.090197in}}{\pgfqpoint{1.030423in}{1.085806in}}{\pgfqpoint{1.022610in}{1.077993in}}%
\pgfpathcurveto{\pgfqpoint{1.014796in}{1.070179in}}{\pgfqpoint{1.010406in}{1.059580in}}{\pgfqpoint{1.010406in}{1.048530in}}%
\pgfpathcurveto{\pgfqpoint{1.010406in}{1.037480in}}{\pgfqpoint{1.014796in}{1.026881in}}{\pgfqpoint{1.022610in}{1.019067in}}%
\pgfpathcurveto{\pgfqpoint{1.030423in}{1.011254in}}{\pgfqpoint{1.041023in}{1.006863in}}{\pgfqpoint{1.052073in}{1.006863in}}%
\pgfpathclose%
\pgfusepath{stroke,fill}%
\end{pgfscope}%
\begin{pgfscope}%
\pgfpathrectangle{\pgfqpoint{0.532717in}{0.370679in}}{\pgfqpoint{2.617283in}{1.479321in}}%
\pgfusepath{clip}%
\pgfsetbuttcap%
\pgfsetroundjoin%
\definecolor{currentfill}{rgb}{1.000000,0.411765,0.380392}%
\pgfsetfillcolor{currentfill}%
\pgfsetfillopacity{0.500000}%
\pgfsetlinewidth{1.003750pt}%
\definecolor{currentstroke}{rgb}{1.000000,0.411765,0.380392}%
\pgfsetstrokecolor{currentstroke}%
\pgfsetstrokeopacity{0.500000}%
\pgfsetdash{}{0pt}%
\pgfpathmoveto{\pgfqpoint{1.111217in}{0.893912in}}%
\pgfpathcurveto{\pgfqpoint{1.122267in}{0.893912in}}{\pgfqpoint{1.132866in}{0.898302in}}{\pgfqpoint{1.140679in}{0.906116in}}%
\pgfpathcurveto{\pgfqpoint{1.148493in}{0.913930in}}{\pgfqpoint{1.152883in}{0.924529in}}{\pgfqpoint{1.152883in}{0.935579in}}%
\pgfpathcurveto{\pgfqpoint{1.152883in}{0.946629in}}{\pgfqpoint{1.148493in}{0.957228in}}{\pgfqpoint{1.140679in}{0.965042in}}%
\pgfpathcurveto{\pgfqpoint{1.132866in}{0.972855in}}{\pgfqpoint{1.122267in}{0.977246in}}{\pgfqpoint{1.111217in}{0.977246in}}%
\pgfpathcurveto{\pgfqpoint{1.100167in}{0.977246in}}{\pgfqpoint{1.089568in}{0.972855in}}{\pgfqpoint{1.081754in}{0.965042in}}%
\pgfpathcurveto{\pgfqpoint{1.073940in}{0.957228in}}{\pgfqpoint{1.069550in}{0.946629in}}{\pgfqpoint{1.069550in}{0.935579in}}%
\pgfpathcurveto{\pgfqpoint{1.069550in}{0.924529in}}{\pgfqpoint{1.073940in}{0.913930in}}{\pgfqpoint{1.081754in}{0.906116in}}%
\pgfpathcurveto{\pgfqpoint{1.089568in}{0.898302in}}{\pgfqpoint{1.100167in}{0.893912in}}{\pgfqpoint{1.111217in}{0.893912in}}%
\pgfpathclose%
\pgfusepath{stroke,fill}%
\end{pgfscope}%
\begin{pgfscope}%
\pgfpathrectangle{\pgfqpoint{0.532717in}{0.370679in}}{\pgfqpoint{2.617283in}{1.479321in}}%
\pgfusepath{clip}%
\pgfsetbuttcap%
\pgfsetroundjoin%
\definecolor{currentfill}{rgb}{1.000000,0.411765,0.380392}%
\pgfsetfillcolor{currentfill}%
\pgfsetfillopacity{0.500000}%
\pgfsetlinewidth{1.003750pt}%
\definecolor{currentstroke}{rgb}{1.000000,0.411765,0.380392}%
\pgfsetstrokecolor{currentstroke}%
\pgfsetstrokeopacity{0.500000}%
\pgfsetdash{}{0pt}%
\pgfpathmoveto{\pgfqpoint{0.958807in}{1.207611in}}%
\pgfpathcurveto{\pgfqpoint{0.969857in}{1.207611in}}{\pgfqpoint{0.980456in}{1.212001in}}{\pgfqpoint{0.988270in}{1.219815in}}%
\pgfpathcurveto{\pgfqpoint{0.996083in}{1.227629in}}{\pgfqpoint{1.000474in}{1.238228in}}{\pgfqpoint{1.000474in}{1.249278in}}%
\pgfpathcurveto{\pgfqpoint{1.000474in}{1.260328in}}{\pgfqpoint{0.996083in}{1.270927in}}{\pgfqpoint{0.988270in}{1.278741in}}%
\pgfpathcurveto{\pgfqpoint{0.980456in}{1.286554in}}{\pgfqpoint{0.969857in}{1.290944in}}{\pgfqpoint{0.958807in}{1.290944in}}%
\pgfpathcurveto{\pgfqpoint{0.947757in}{1.290944in}}{\pgfqpoint{0.937158in}{1.286554in}}{\pgfqpoint{0.929344in}{1.278741in}}%
\pgfpathcurveto{\pgfqpoint{0.921531in}{1.270927in}}{\pgfqpoint{0.917140in}{1.260328in}}{\pgfqpoint{0.917140in}{1.249278in}}%
\pgfpathcurveto{\pgfqpoint{0.917140in}{1.238228in}}{\pgfqpoint{0.921531in}{1.227629in}}{\pgfqpoint{0.929344in}{1.219815in}}%
\pgfpathcurveto{\pgfqpoint{0.937158in}{1.212001in}}{\pgfqpoint{0.947757in}{1.207611in}}{\pgfqpoint{0.958807in}{1.207611in}}%
\pgfpathclose%
\pgfusepath{stroke,fill}%
\end{pgfscope}%
\begin{pgfscope}%
\pgfpathrectangle{\pgfqpoint{0.532717in}{0.370679in}}{\pgfqpoint{2.617283in}{1.479321in}}%
\pgfusepath{clip}%
\pgfsetbuttcap%
\pgfsetroundjoin%
\definecolor{currentfill}{rgb}{1.000000,0.411765,0.380392}%
\pgfsetfillcolor{currentfill}%
\pgfsetfillopacity{0.500000}%
\pgfsetlinewidth{1.003750pt}%
\definecolor{currentstroke}{rgb}{1.000000,0.411765,0.380392}%
\pgfsetstrokecolor{currentstroke}%
\pgfsetstrokeopacity{0.500000}%
\pgfsetdash{}{0pt}%
\pgfpathmoveto{\pgfqpoint{1.012361in}{0.862939in}}%
\pgfpathcurveto{\pgfqpoint{1.023411in}{0.862939in}}{\pgfqpoint{1.034010in}{0.867329in}}{\pgfqpoint{1.041824in}{0.875143in}}%
\pgfpathcurveto{\pgfqpoint{1.049637in}{0.882956in}}{\pgfqpoint{1.054028in}{0.893555in}}{\pgfqpoint{1.054028in}{0.904605in}}%
\pgfpathcurveto{\pgfqpoint{1.054028in}{0.915656in}}{\pgfqpoint{1.049637in}{0.926255in}}{\pgfqpoint{1.041824in}{0.934068in}}%
\pgfpathcurveto{\pgfqpoint{1.034010in}{0.941882in}}{\pgfqpoint{1.023411in}{0.946272in}}{\pgfqpoint{1.012361in}{0.946272in}}%
\pgfpathcurveto{\pgfqpoint{1.001311in}{0.946272in}}{\pgfqpoint{0.990712in}{0.941882in}}{\pgfqpoint{0.982898in}{0.934068in}}%
\pgfpathcurveto{\pgfqpoint{0.975085in}{0.926255in}}{\pgfqpoint{0.970694in}{0.915656in}}{\pgfqpoint{0.970694in}{0.904605in}}%
\pgfpathcurveto{\pgfqpoint{0.970694in}{0.893555in}}{\pgfqpoint{0.975085in}{0.882956in}}{\pgfqpoint{0.982898in}{0.875143in}}%
\pgfpathcurveto{\pgfqpoint{0.990712in}{0.867329in}}{\pgfqpoint{1.001311in}{0.862939in}}{\pgfqpoint{1.012361in}{0.862939in}}%
\pgfpathclose%
\pgfusepath{stroke,fill}%
\end{pgfscope}%
\begin{pgfscope}%
\pgfpathrectangle{\pgfqpoint{0.532717in}{0.370679in}}{\pgfqpoint{2.617283in}{1.479321in}}%
\pgfusepath{clip}%
\pgfsetbuttcap%
\pgfsetroundjoin%
\definecolor{currentfill}{rgb}{1.000000,0.411765,0.380392}%
\pgfsetfillcolor{currentfill}%
\pgfsetfillopacity{0.500000}%
\pgfsetlinewidth{1.003750pt}%
\definecolor{currentstroke}{rgb}{1.000000,0.411765,0.380392}%
\pgfsetstrokecolor{currentstroke}%
\pgfsetstrokeopacity{0.500000}%
\pgfsetdash{}{0pt}%
\pgfpathmoveto{\pgfqpoint{0.921322in}{1.197494in}}%
\pgfpathcurveto{\pgfqpoint{0.932372in}{1.197494in}}{\pgfqpoint{0.942971in}{1.201884in}}{\pgfqpoint{0.950784in}{1.209697in}}%
\pgfpathcurveto{\pgfqpoint{0.958598in}{1.217511in}}{\pgfqpoint{0.962988in}{1.228110in}}{\pgfqpoint{0.962988in}{1.239160in}}%
\pgfpathcurveto{\pgfqpoint{0.962988in}{1.250210in}}{\pgfqpoint{0.958598in}{1.260809in}}{\pgfqpoint{0.950784in}{1.268623in}}%
\pgfpathcurveto{\pgfqpoint{0.942971in}{1.276437in}}{\pgfqpoint{0.932372in}{1.280827in}}{\pgfqpoint{0.921322in}{1.280827in}}%
\pgfpathcurveto{\pgfqpoint{0.910271in}{1.280827in}}{\pgfqpoint{0.899672in}{1.276437in}}{\pgfqpoint{0.891859in}{1.268623in}}%
\pgfpathcurveto{\pgfqpoint{0.884045in}{1.260809in}}{\pgfqpoint{0.879655in}{1.250210in}}{\pgfqpoint{0.879655in}{1.239160in}}%
\pgfpathcurveto{\pgfqpoint{0.879655in}{1.228110in}}{\pgfqpoint{0.884045in}{1.217511in}}{\pgfqpoint{0.891859in}{1.209697in}}%
\pgfpathcurveto{\pgfqpoint{0.899672in}{1.201884in}}{\pgfqpoint{0.910271in}{1.197494in}}{\pgfqpoint{0.921322in}{1.197494in}}%
\pgfpathclose%
\pgfusepath{stroke,fill}%
\end{pgfscope}%
\begin{pgfscope}%
\pgfpathrectangle{\pgfqpoint{0.532717in}{0.370679in}}{\pgfqpoint{2.617283in}{1.479321in}}%
\pgfusepath{clip}%
\pgfsetbuttcap%
\pgfsetroundjoin%
\definecolor{currentfill}{rgb}{1.000000,0.411765,0.380392}%
\pgfsetfillcolor{currentfill}%
\pgfsetfillopacity{0.500000}%
\pgfsetlinewidth{1.003750pt}%
\definecolor{currentstroke}{rgb}{1.000000,0.411765,0.380392}%
\pgfsetstrokecolor{currentstroke}%
\pgfsetstrokeopacity{0.500000}%
\pgfsetdash{}{0pt}%
\pgfpathmoveto{\pgfqpoint{1.010824in}{0.828193in}}%
\pgfpathcurveto{\pgfqpoint{1.021874in}{0.828193in}}{\pgfqpoint{1.032473in}{0.832583in}}{\pgfqpoint{1.040286in}{0.840397in}}%
\pgfpathcurveto{\pgfqpoint{1.048100in}{0.848210in}}{\pgfqpoint{1.052490in}{0.858810in}}{\pgfqpoint{1.052490in}{0.869860in}}%
\pgfpathcurveto{\pgfqpoint{1.052490in}{0.880910in}}{\pgfqpoint{1.048100in}{0.891509in}}{\pgfqpoint{1.040286in}{0.899322in}}%
\pgfpathcurveto{\pgfqpoint{1.032473in}{0.907136in}}{\pgfqpoint{1.021874in}{0.911526in}}{\pgfqpoint{1.010824in}{0.911526in}}%
\pgfpathcurveto{\pgfqpoint{0.999774in}{0.911526in}}{\pgfqpoint{0.989175in}{0.907136in}}{\pgfqpoint{0.981361in}{0.899322in}}%
\pgfpathcurveto{\pgfqpoint{0.973547in}{0.891509in}}{\pgfqpoint{0.969157in}{0.880910in}}{\pgfqpoint{0.969157in}{0.869860in}}%
\pgfpathcurveto{\pgfqpoint{0.969157in}{0.858810in}}{\pgfqpoint{0.973547in}{0.848210in}}{\pgfqpoint{0.981361in}{0.840397in}}%
\pgfpathcurveto{\pgfqpoint{0.989175in}{0.832583in}}{\pgfqpoint{0.999774in}{0.828193in}}{\pgfqpoint{1.010824in}{0.828193in}}%
\pgfpathclose%
\pgfusepath{stroke,fill}%
\end{pgfscope}%
\begin{pgfscope}%
\pgfpathrectangle{\pgfqpoint{0.532717in}{0.370679in}}{\pgfqpoint{2.617283in}{1.479321in}}%
\pgfusepath{clip}%
\pgfsetbuttcap%
\pgfsetroundjoin%
\definecolor{currentfill}{rgb}{1.000000,0.411765,0.380392}%
\pgfsetfillcolor{currentfill}%
\pgfsetfillopacity{0.500000}%
\pgfsetlinewidth{1.003750pt}%
\definecolor{currentstroke}{rgb}{1.000000,0.411765,0.380392}%
\pgfsetstrokecolor{currentstroke}%
\pgfsetstrokeopacity{0.500000}%
\pgfsetdash{}{0pt}%
\pgfpathmoveto{\pgfqpoint{0.962341in}{1.043894in}}%
\pgfpathcurveto{\pgfqpoint{0.973391in}{1.043894in}}{\pgfqpoint{0.983990in}{1.048285in}}{\pgfqpoint{0.991804in}{1.056098in}}%
\pgfpathcurveto{\pgfqpoint{0.999618in}{1.063912in}}{\pgfqpoint{1.004008in}{1.074511in}}{\pgfqpoint{1.004008in}{1.085561in}}%
\pgfpathcurveto{\pgfqpoint{1.004008in}{1.096611in}}{\pgfqpoint{0.999618in}{1.107210in}}{\pgfqpoint{0.991804in}{1.115024in}}%
\pgfpathcurveto{\pgfqpoint{0.983990in}{1.122838in}}{\pgfqpoint{0.973391in}{1.127228in}}{\pgfqpoint{0.962341in}{1.127228in}}%
\pgfpathcurveto{\pgfqpoint{0.951291in}{1.127228in}}{\pgfqpoint{0.940692in}{1.122838in}}{\pgfqpoint{0.932878in}{1.115024in}}%
\pgfpathcurveto{\pgfqpoint{0.925065in}{1.107210in}}{\pgfqpoint{0.920675in}{1.096611in}}{\pgfqpoint{0.920675in}{1.085561in}}%
\pgfpathcurveto{\pgfqpoint{0.920675in}{1.074511in}}{\pgfqpoint{0.925065in}{1.063912in}}{\pgfqpoint{0.932878in}{1.056098in}}%
\pgfpathcurveto{\pgfqpoint{0.940692in}{1.048285in}}{\pgfqpoint{0.951291in}{1.043894in}}{\pgfqpoint{0.962341in}{1.043894in}}%
\pgfpathclose%
\pgfusepath{stroke,fill}%
\end{pgfscope}%
\begin{pgfscope}%
\pgfpathrectangle{\pgfqpoint{0.532717in}{0.370679in}}{\pgfqpoint{2.617283in}{1.479321in}}%
\pgfusepath{clip}%
\pgfsetbuttcap%
\pgfsetroundjoin%
\definecolor{currentfill}{rgb}{0.466667,0.866667,0.466667}%
\pgfsetfillcolor{currentfill}%
\pgfsetfillopacity{0.500000}%
\pgfsetlinewidth{1.003750pt}%
\definecolor{currentstroke}{rgb}{0.466667,0.866667,0.466667}%
\pgfsetstrokecolor{currentstroke}%
\pgfsetstrokeopacity{0.500000}%
\pgfsetdash{}{0pt}%
\pgfpathmoveto{\pgfqpoint{2.140954in}{0.782808in}}%
\pgfpathlineto{\pgfqpoint{2.161787in}{0.803641in}}%
\pgfpathlineto{\pgfqpoint{2.182620in}{0.782808in}}%
\pgfpathlineto{\pgfqpoint{2.203454in}{0.803641in}}%
\pgfpathlineto{\pgfqpoint{2.182620in}{0.824475in}}%
\pgfpathlineto{\pgfqpoint{2.203454in}{0.845308in}}%
\pgfpathlineto{\pgfqpoint{2.182620in}{0.866141in}}%
\pgfpathlineto{\pgfqpoint{2.161787in}{0.845308in}}%
\pgfpathlineto{\pgfqpoint{2.140954in}{0.866141in}}%
\pgfpathlineto{\pgfqpoint{2.120120in}{0.845308in}}%
\pgfpathlineto{\pgfqpoint{2.140954in}{0.824475in}}%
\pgfpathlineto{\pgfqpoint{2.120120in}{0.803641in}}%
\pgfpathclose%
\pgfusepath{stroke,fill}%
\end{pgfscope}%
\begin{pgfscope}%
\pgfpathrectangle{\pgfqpoint{0.532717in}{0.370679in}}{\pgfqpoint{2.617283in}{1.479321in}}%
\pgfusepath{clip}%
\pgfsetbuttcap%
\pgfsetroundjoin%
\definecolor{currentfill}{rgb}{0.466667,0.866667,0.466667}%
\pgfsetfillcolor{currentfill}%
\pgfsetfillopacity{0.500000}%
\pgfsetlinewidth{1.003750pt}%
\definecolor{currentstroke}{rgb}{0.466667,0.866667,0.466667}%
\pgfsetstrokecolor{currentstroke}%
\pgfsetstrokeopacity{0.500000}%
\pgfsetdash{}{0pt}%
\pgfpathmoveto{\pgfqpoint{2.034988in}{0.950437in}}%
\pgfpathlineto{\pgfqpoint{2.055821in}{0.971271in}}%
\pgfpathlineto{\pgfqpoint{2.076655in}{0.950437in}}%
\pgfpathlineto{\pgfqpoint{2.097488in}{0.971271in}}%
\pgfpathlineto{\pgfqpoint{2.076655in}{0.992104in}}%
\pgfpathlineto{\pgfqpoint{2.097488in}{1.012937in}}%
\pgfpathlineto{\pgfqpoint{2.076655in}{1.033771in}}%
\pgfpathlineto{\pgfqpoint{2.055821in}{1.012937in}}%
\pgfpathlineto{\pgfqpoint{2.034988in}{1.033771in}}%
\pgfpathlineto{\pgfqpoint{2.014155in}{1.012937in}}%
\pgfpathlineto{\pgfqpoint{2.034988in}{0.992104in}}%
\pgfpathlineto{\pgfqpoint{2.014155in}{0.971271in}}%
\pgfpathclose%
\pgfusepath{stroke,fill}%
\end{pgfscope}%
\begin{pgfscope}%
\pgfpathrectangle{\pgfqpoint{0.532717in}{0.370679in}}{\pgfqpoint{2.617283in}{1.479321in}}%
\pgfusepath{clip}%
\pgfsetbuttcap%
\pgfsetroundjoin%
\definecolor{currentfill}{rgb}{0.466667,0.866667,0.466667}%
\pgfsetfillcolor{currentfill}%
\pgfsetfillopacity{0.500000}%
\pgfsetlinewidth{1.003750pt}%
\definecolor{currentstroke}{rgb}{0.466667,0.866667,0.466667}%
\pgfsetstrokecolor{currentstroke}%
\pgfsetstrokeopacity{0.500000}%
\pgfsetdash{}{0pt}%
\pgfpathmoveto{\pgfqpoint{2.194861in}{0.865766in}}%
\pgfpathlineto{\pgfqpoint{2.215694in}{0.886600in}}%
\pgfpathlineto{\pgfqpoint{2.236528in}{0.865766in}}%
\pgfpathlineto{\pgfqpoint{2.257361in}{0.886600in}}%
\pgfpathlineto{\pgfqpoint{2.236528in}{0.907433in}}%
\pgfpathlineto{\pgfqpoint{2.257361in}{0.928266in}}%
\pgfpathlineto{\pgfqpoint{2.236528in}{0.949100in}}%
\pgfpathlineto{\pgfqpoint{2.215694in}{0.928266in}}%
\pgfpathlineto{\pgfqpoint{2.194861in}{0.949100in}}%
\pgfpathlineto{\pgfqpoint{2.174028in}{0.928266in}}%
\pgfpathlineto{\pgfqpoint{2.194861in}{0.907433in}}%
\pgfpathlineto{\pgfqpoint{2.174028in}{0.886600in}}%
\pgfpathclose%
\pgfusepath{stroke,fill}%
\end{pgfscope}%
\begin{pgfscope}%
\pgfpathrectangle{\pgfqpoint{0.532717in}{0.370679in}}{\pgfqpoint{2.617283in}{1.479321in}}%
\pgfusepath{clip}%
\pgfsetbuttcap%
\pgfsetroundjoin%
\definecolor{currentfill}{rgb}{0.466667,0.866667,0.466667}%
\pgfsetfillcolor{currentfill}%
\pgfsetfillopacity{0.500000}%
\pgfsetlinewidth{1.003750pt}%
\definecolor{currentstroke}{rgb}{0.466667,0.866667,0.466667}%
\pgfsetstrokecolor{currentstroke}%
\pgfsetstrokeopacity{0.500000}%
\pgfsetdash{}{0pt}%
\pgfpathmoveto{\pgfqpoint{1.809021in}{1.474416in}}%
\pgfpathlineto{\pgfqpoint{1.829855in}{1.495249in}}%
\pgfpathlineto{\pgfqpoint{1.850688in}{1.474416in}}%
\pgfpathlineto{\pgfqpoint{1.871521in}{1.495249in}}%
\pgfpathlineto{\pgfqpoint{1.850688in}{1.516083in}}%
\pgfpathlineto{\pgfqpoint{1.871521in}{1.536916in}}%
\pgfpathlineto{\pgfqpoint{1.850688in}{1.557749in}}%
\pgfpathlineto{\pgfqpoint{1.829855in}{1.536916in}}%
\pgfpathlineto{\pgfqpoint{1.809021in}{1.557749in}}%
\pgfpathlineto{\pgfqpoint{1.788188in}{1.536916in}}%
\pgfpathlineto{\pgfqpoint{1.809021in}{1.516083in}}%
\pgfpathlineto{\pgfqpoint{1.788188in}{1.495249in}}%
\pgfpathclose%
\pgfusepath{stroke,fill}%
\end{pgfscope}%
\begin{pgfscope}%
\pgfpathrectangle{\pgfqpoint{0.532717in}{0.370679in}}{\pgfqpoint{2.617283in}{1.479321in}}%
\pgfusepath{clip}%
\pgfsetbuttcap%
\pgfsetroundjoin%
\definecolor{currentfill}{rgb}{0.466667,0.866667,0.466667}%
\pgfsetfillcolor{currentfill}%
\pgfsetfillopacity{0.500000}%
\pgfsetlinewidth{1.003750pt}%
\definecolor{currentstroke}{rgb}{0.466667,0.866667,0.466667}%
\pgfsetstrokecolor{currentstroke}%
\pgfsetstrokeopacity{0.500000}%
\pgfsetdash{}{0pt}%
\pgfpathmoveto{\pgfqpoint{2.081515in}{1.062029in}}%
\pgfpathlineto{\pgfqpoint{2.102348in}{1.082862in}}%
\pgfpathlineto{\pgfqpoint{2.123182in}{1.062029in}}%
\pgfpathlineto{\pgfqpoint{2.144015in}{1.082862in}}%
\pgfpathlineto{\pgfqpoint{2.123182in}{1.103696in}}%
\pgfpathlineto{\pgfqpoint{2.144015in}{1.124529in}}%
\pgfpathlineto{\pgfqpoint{2.123182in}{1.145362in}}%
\pgfpathlineto{\pgfqpoint{2.102348in}{1.124529in}}%
\pgfpathlineto{\pgfqpoint{2.081515in}{1.145362in}}%
\pgfpathlineto{\pgfqpoint{2.060682in}{1.124529in}}%
\pgfpathlineto{\pgfqpoint{2.081515in}{1.103696in}}%
\pgfpathlineto{\pgfqpoint{2.060682in}{1.082862in}}%
\pgfpathclose%
\pgfusepath{stroke,fill}%
\end{pgfscope}%
\begin{pgfscope}%
\pgfpathrectangle{\pgfqpoint{0.532717in}{0.370679in}}{\pgfqpoint{2.617283in}{1.479321in}}%
\pgfusepath{clip}%
\pgfsetbuttcap%
\pgfsetroundjoin%
\definecolor{currentfill}{rgb}{0.466667,0.866667,0.466667}%
\pgfsetfillcolor{currentfill}%
\pgfsetfillopacity{0.500000}%
\pgfsetlinewidth{1.003750pt}%
\definecolor{currentstroke}{rgb}{0.466667,0.866667,0.466667}%
\pgfsetstrokecolor{currentstroke}%
\pgfsetstrokeopacity{0.500000}%
\pgfsetdash{}{0pt}%
\pgfpathmoveto{\pgfqpoint{1.947188in}{1.287545in}}%
\pgfpathlineto{\pgfqpoint{1.968022in}{1.308379in}}%
\pgfpathlineto{\pgfqpoint{1.988855in}{1.287545in}}%
\pgfpathlineto{\pgfqpoint{2.009688in}{1.308379in}}%
\pgfpathlineto{\pgfqpoint{1.988855in}{1.329212in}}%
\pgfpathlineto{\pgfqpoint{2.009688in}{1.350045in}}%
\pgfpathlineto{\pgfqpoint{1.988855in}{1.370879in}}%
\pgfpathlineto{\pgfqpoint{1.968022in}{1.350045in}}%
\pgfpathlineto{\pgfqpoint{1.947188in}{1.370879in}}%
\pgfpathlineto{\pgfqpoint{1.926355in}{1.350045in}}%
\pgfpathlineto{\pgfqpoint{1.947188in}{1.329212in}}%
\pgfpathlineto{\pgfqpoint{1.926355in}{1.308379in}}%
\pgfpathclose%
\pgfusepath{stroke,fill}%
\end{pgfscope}%
\begin{pgfscope}%
\pgfpathrectangle{\pgfqpoint{0.532717in}{0.370679in}}{\pgfqpoint{2.617283in}{1.479321in}}%
\pgfusepath{clip}%
\pgfsetbuttcap%
\pgfsetroundjoin%
\definecolor{currentfill}{rgb}{0.466667,0.866667,0.466667}%
\pgfsetfillcolor{currentfill}%
\pgfsetfillopacity{0.500000}%
\pgfsetlinewidth{1.003750pt}%
\definecolor{currentstroke}{rgb}{0.466667,0.866667,0.466667}%
\pgfsetstrokecolor{currentstroke}%
\pgfsetstrokeopacity{0.500000}%
\pgfsetdash{}{0pt}%
\pgfpathmoveto{\pgfqpoint{2.083948in}{0.966598in}}%
\pgfpathlineto{\pgfqpoint{2.104781in}{0.987431in}}%
\pgfpathlineto{\pgfqpoint{2.125614in}{0.966598in}}%
\pgfpathlineto{\pgfqpoint{2.146448in}{0.987431in}}%
\pgfpathlineto{\pgfqpoint{2.125614in}{1.008264in}}%
\pgfpathlineto{\pgfqpoint{2.146448in}{1.029098in}}%
\pgfpathlineto{\pgfqpoint{2.125614in}{1.049931in}}%
\pgfpathlineto{\pgfqpoint{2.104781in}{1.029098in}}%
\pgfpathlineto{\pgfqpoint{2.083948in}{1.049931in}}%
\pgfpathlineto{\pgfqpoint{2.063114in}{1.029098in}}%
\pgfpathlineto{\pgfqpoint{2.083948in}{1.008264in}}%
\pgfpathlineto{\pgfqpoint{2.063114in}{0.987431in}}%
\pgfpathclose%
\pgfusepath{stroke,fill}%
\end{pgfscope}%
\begin{pgfscope}%
\pgfpathrectangle{\pgfqpoint{0.532717in}{0.370679in}}{\pgfqpoint{2.617283in}{1.479321in}}%
\pgfusepath{clip}%
\pgfsetbuttcap%
\pgfsetroundjoin%
\definecolor{currentfill}{rgb}{0.466667,0.866667,0.466667}%
\pgfsetfillcolor{currentfill}%
\pgfsetfillopacity{0.500000}%
\pgfsetlinewidth{1.003750pt}%
\definecolor{currentstroke}{rgb}{0.466667,0.866667,0.466667}%
\pgfsetstrokecolor{currentstroke}%
\pgfsetstrokeopacity{0.500000}%
\pgfsetdash{}{0pt}%
\pgfpathmoveto{\pgfqpoint{1.528628in}{1.554745in}}%
\pgfpathlineto{\pgfqpoint{1.549462in}{1.575578in}}%
\pgfpathlineto{\pgfqpoint{1.570295in}{1.554745in}}%
\pgfpathlineto{\pgfqpoint{1.591128in}{1.575578in}}%
\pgfpathlineto{\pgfqpoint{1.570295in}{1.596411in}}%
\pgfpathlineto{\pgfqpoint{1.591128in}{1.617245in}}%
\pgfpathlineto{\pgfqpoint{1.570295in}{1.638078in}}%
\pgfpathlineto{\pgfqpoint{1.549462in}{1.617245in}}%
\pgfpathlineto{\pgfqpoint{1.528628in}{1.638078in}}%
\pgfpathlineto{\pgfqpoint{1.507795in}{1.617245in}}%
\pgfpathlineto{\pgfqpoint{1.528628in}{1.596411in}}%
\pgfpathlineto{\pgfqpoint{1.507795in}{1.575578in}}%
\pgfpathclose%
\pgfusepath{stroke,fill}%
\end{pgfscope}%
\begin{pgfscope}%
\pgfpathrectangle{\pgfqpoint{0.532717in}{0.370679in}}{\pgfqpoint{2.617283in}{1.479321in}}%
\pgfusepath{clip}%
\pgfsetbuttcap%
\pgfsetroundjoin%
\definecolor{currentfill}{rgb}{0.466667,0.866667,0.466667}%
\pgfsetfillcolor{currentfill}%
\pgfsetfillopacity{0.500000}%
\pgfsetlinewidth{1.003750pt}%
\definecolor{currentstroke}{rgb}{0.466667,0.866667,0.466667}%
\pgfsetstrokecolor{currentstroke}%
\pgfsetstrokeopacity{0.500000}%
\pgfsetdash{}{0pt}%
\pgfpathmoveto{\pgfqpoint{2.068333in}{0.991741in}}%
\pgfpathlineto{\pgfqpoint{2.089166in}{1.012574in}}%
\pgfpathlineto{\pgfqpoint{2.110000in}{0.991741in}}%
\pgfpathlineto{\pgfqpoint{2.130833in}{1.012574in}}%
\pgfpathlineto{\pgfqpoint{2.110000in}{1.033408in}}%
\pgfpathlineto{\pgfqpoint{2.130833in}{1.054241in}}%
\pgfpathlineto{\pgfqpoint{2.110000in}{1.075074in}}%
\pgfpathlineto{\pgfqpoint{2.089166in}{1.054241in}}%
\pgfpathlineto{\pgfqpoint{2.068333in}{1.075074in}}%
\pgfpathlineto{\pgfqpoint{2.047500in}{1.054241in}}%
\pgfpathlineto{\pgfqpoint{2.068333in}{1.033408in}}%
\pgfpathlineto{\pgfqpoint{2.047500in}{1.012574in}}%
\pgfpathclose%
\pgfusepath{stroke,fill}%
\end{pgfscope}%
\begin{pgfscope}%
\pgfpathrectangle{\pgfqpoint{0.532717in}{0.370679in}}{\pgfqpoint{2.617283in}{1.479321in}}%
\pgfusepath{clip}%
\pgfsetbuttcap%
\pgfsetroundjoin%
\definecolor{currentfill}{rgb}{0.466667,0.866667,0.466667}%
\pgfsetfillcolor{currentfill}%
\pgfsetfillopacity{0.500000}%
\pgfsetlinewidth{1.003750pt}%
\definecolor{currentstroke}{rgb}{0.466667,0.866667,0.466667}%
\pgfsetstrokecolor{currentstroke}%
\pgfsetstrokeopacity{0.500000}%
\pgfsetdash{}{0pt}%
\pgfpathmoveto{\pgfqpoint{1.751538in}{1.426344in}}%
\pgfpathlineto{\pgfqpoint{1.772372in}{1.447177in}}%
\pgfpathlineto{\pgfqpoint{1.793205in}{1.426344in}}%
\pgfpathlineto{\pgfqpoint{1.814038in}{1.447177in}}%
\pgfpathlineto{\pgfqpoint{1.793205in}{1.468011in}}%
\pgfpathlineto{\pgfqpoint{1.814038in}{1.488844in}}%
\pgfpathlineto{\pgfqpoint{1.793205in}{1.509677in}}%
\pgfpathlineto{\pgfqpoint{1.772372in}{1.488844in}}%
\pgfpathlineto{\pgfqpoint{1.751538in}{1.509677in}}%
\pgfpathlineto{\pgfqpoint{1.730705in}{1.488844in}}%
\pgfpathlineto{\pgfqpoint{1.751538in}{1.468011in}}%
\pgfpathlineto{\pgfqpoint{1.730705in}{1.447177in}}%
\pgfpathclose%
\pgfusepath{stroke,fill}%
\end{pgfscope}%
\begin{pgfscope}%
\pgfpathrectangle{\pgfqpoint{0.532717in}{0.370679in}}{\pgfqpoint{2.617283in}{1.479321in}}%
\pgfusepath{clip}%
\pgfsetbuttcap%
\pgfsetroundjoin%
\definecolor{currentfill}{rgb}{0.466667,0.866667,0.466667}%
\pgfsetfillcolor{currentfill}%
\pgfsetfillopacity{0.500000}%
\pgfsetlinewidth{1.003750pt}%
\definecolor{currentstroke}{rgb}{0.466667,0.866667,0.466667}%
\pgfsetstrokecolor{currentstroke}%
\pgfsetstrokeopacity{0.500000}%
\pgfsetdash{}{0pt}%
\pgfpathmoveto{\pgfqpoint{1.600914in}{1.674381in}}%
\pgfpathlineto{\pgfqpoint{1.621747in}{1.695214in}}%
\pgfpathlineto{\pgfqpoint{1.642580in}{1.674381in}}%
\pgfpathlineto{\pgfqpoint{1.663414in}{1.695214in}}%
\pgfpathlineto{\pgfqpoint{1.642580in}{1.716048in}}%
\pgfpathlineto{\pgfqpoint{1.663414in}{1.736881in}}%
\pgfpathlineto{\pgfqpoint{1.642580in}{1.757714in}}%
\pgfpathlineto{\pgfqpoint{1.621747in}{1.736881in}}%
\pgfpathlineto{\pgfqpoint{1.600914in}{1.757714in}}%
\pgfpathlineto{\pgfqpoint{1.580080in}{1.736881in}}%
\pgfpathlineto{\pgfqpoint{1.600914in}{1.716048in}}%
\pgfpathlineto{\pgfqpoint{1.580080in}{1.695214in}}%
\pgfpathclose%
\pgfusepath{stroke,fill}%
\end{pgfscope}%
\begin{pgfscope}%
\pgfpathrectangle{\pgfqpoint{0.532717in}{0.370679in}}{\pgfqpoint{2.617283in}{1.479321in}}%
\pgfusepath{clip}%
\pgfsetbuttcap%
\pgfsetroundjoin%
\definecolor{currentfill}{rgb}{0.466667,0.866667,0.466667}%
\pgfsetfillcolor{currentfill}%
\pgfsetfillopacity{0.500000}%
\pgfsetlinewidth{1.003750pt}%
\definecolor{currentstroke}{rgb}{0.466667,0.866667,0.466667}%
\pgfsetstrokecolor{currentstroke}%
\pgfsetstrokeopacity{0.500000}%
\pgfsetdash{}{0pt}%
\pgfpathmoveto{\pgfqpoint{1.908295in}{1.143351in}}%
\pgfpathlineto{\pgfqpoint{1.929128in}{1.164184in}}%
\pgfpathlineto{\pgfqpoint{1.949962in}{1.143351in}}%
\pgfpathlineto{\pgfqpoint{1.970795in}{1.164184in}}%
\pgfpathlineto{\pgfqpoint{1.949962in}{1.185018in}}%
\pgfpathlineto{\pgfqpoint{1.970795in}{1.205851in}}%
\pgfpathlineto{\pgfqpoint{1.949962in}{1.226684in}}%
\pgfpathlineto{\pgfqpoint{1.929128in}{1.205851in}}%
\pgfpathlineto{\pgfqpoint{1.908295in}{1.226684in}}%
\pgfpathlineto{\pgfqpoint{1.887462in}{1.205851in}}%
\pgfpathlineto{\pgfqpoint{1.908295in}{1.185018in}}%
\pgfpathlineto{\pgfqpoint{1.887462in}{1.164184in}}%
\pgfpathclose%
\pgfusepath{stroke,fill}%
\end{pgfscope}%
\begin{pgfscope}%
\pgfpathrectangle{\pgfqpoint{0.532717in}{0.370679in}}{\pgfqpoint{2.617283in}{1.479321in}}%
\pgfusepath{clip}%
\pgfsetbuttcap%
\pgfsetroundjoin%
\definecolor{currentfill}{rgb}{0.466667,0.866667,0.466667}%
\pgfsetfillcolor{currentfill}%
\pgfsetfillopacity{0.500000}%
\pgfsetlinewidth{1.003750pt}%
\definecolor{currentstroke}{rgb}{0.466667,0.866667,0.466667}%
\pgfsetstrokecolor{currentstroke}%
\pgfsetstrokeopacity{0.500000}%
\pgfsetdash{}{0pt}%
\pgfpathmoveto{\pgfqpoint{1.833490in}{1.347307in}}%
\pgfpathlineto{\pgfqpoint{1.854323in}{1.368141in}}%
\pgfpathlineto{\pgfqpoint{1.875156in}{1.347307in}}%
\pgfpathlineto{\pgfqpoint{1.895990in}{1.368141in}}%
\pgfpathlineto{\pgfqpoint{1.875156in}{1.388974in}}%
\pgfpathlineto{\pgfqpoint{1.895990in}{1.409807in}}%
\pgfpathlineto{\pgfqpoint{1.875156in}{1.430641in}}%
\pgfpathlineto{\pgfqpoint{1.854323in}{1.409807in}}%
\pgfpathlineto{\pgfqpoint{1.833490in}{1.430641in}}%
\pgfpathlineto{\pgfqpoint{1.812656in}{1.409807in}}%
\pgfpathlineto{\pgfqpoint{1.833490in}{1.388974in}}%
\pgfpathlineto{\pgfqpoint{1.812656in}{1.368141in}}%
\pgfpathclose%
\pgfusepath{stroke,fill}%
\end{pgfscope}%
\begin{pgfscope}%
\pgfpathrectangle{\pgfqpoint{0.532717in}{0.370679in}}{\pgfqpoint{2.617283in}{1.479321in}}%
\pgfusepath{clip}%
\pgfsetbuttcap%
\pgfsetroundjoin%
\definecolor{currentfill}{rgb}{0.466667,0.866667,0.466667}%
\pgfsetfillcolor{currentfill}%
\pgfsetfillopacity{0.500000}%
\pgfsetlinewidth{1.003750pt}%
\definecolor{currentstroke}{rgb}{0.466667,0.866667,0.466667}%
\pgfsetstrokecolor{currentstroke}%
\pgfsetstrokeopacity{0.500000}%
\pgfsetdash{}{0pt}%
\pgfpathmoveto{\pgfqpoint{2.050515in}{1.153455in}}%
\pgfpathlineto{\pgfqpoint{2.071348in}{1.174289in}}%
\pgfpathlineto{\pgfqpoint{2.092182in}{1.153455in}}%
\pgfpathlineto{\pgfqpoint{2.113015in}{1.174289in}}%
\pgfpathlineto{\pgfqpoint{2.092182in}{1.195122in}}%
\pgfpathlineto{\pgfqpoint{2.113015in}{1.215955in}}%
\pgfpathlineto{\pgfqpoint{2.092182in}{1.236789in}}%
\pgfpathlineto{\pgfqpoint{2.071348in}{1.215955in}}%
\pgfpathlineto{\pgfqpoint{2.050515in}{1.236789in}}%
\pgfpathlineto{\pgfqpoint{2.029682in}{1.215955in}}%
\pgfpathlineto{\pgfqpoint{2.050515in}{1.195122in}}%
\pgfpathlineto{\pgfqpoint{2.029682in}{1.174289in}}%
\pgfpathclose%
\pgfusepath{stroke,fill}%
\end{pgfscope}%
\begin{pgfscope}%
\pgfpathrectangle{\pgfqpoint{0.532717in}{0.370679in}}{\pgfqpoint{2.617283in}{1.479321in}}%
\pgfusepath{clip}%
\pgfsetbuttcap%
\pgfsetroundjoin%
\definecolor{currentfill}{rgb}{0.466667,0.866667,0.466667}%
\pgfsetfillcolor{currentfill}%
\pgfsetfillopacity{0.500000}%
\pgfsetlinewidth{1.003750pt}%
\definecolor{currentstroke}{rgb}{0.466667,0.866667,0.466667}%
\pgfsetstrokecolor{currentstroke}%
\pgfsetstrokeopacity{0.500000}%
\pgfsetdash{}{0pt}%
\pgfpathmoveto{\pgfqpoint{1.702019in}{1.211792in}}%
\pgfpathlineto{\pgfqpoint{1.722852in}{1.232625in}}%
\pgfpathlineto{\pgfqpoint{1.743686in}{1.211792in}}%
\pgfpathlineto{\pgfqpoint{1.764519in}{1.232625in}}%
\pgfpathlineto{\pgfqpoint{1.743686in}{1.253458in}}%
\pgfpathlineto{\pgfqpoint{1.764519in}{1.274292in}}%
\pgfpathlineto{\pgfqpoint{1.743686in}{1.295125in}}%
\pgfpathlineto{\pgfqpoint{1.722852in}{1.274292in}}%
\pgfpathlineto{\pgfqpoint{1.702019in}{1.295125in}}%
\pgfpathlineto{\pgfqpoint{1.681186in}{1.274292in}}%
\pgfpathlineto{\pgfqpoint{1.702019in}{1.253458in}}%
\pgfpathlineto{\pgfqpoint{1.681186in}{1.232625in}}%
\pgfpathclose%
\pgfusepath{stroke,fill}%
\end{pgfscope}%
\begin{pgfscope}%
\pgfpathrectangle{\pgfqpoint{0.532717in}{0.370679in}}{\pgfqpoint{2.617283in}{1.479321in}}%
\pgfusepath{clip}%
\pgfsetbuttcap%
\pgfsetroundjoin%
\definecolor{currentfill}{rgb}{0.466667,0.866667,0.466667}%
\pgfsetfillcolor{currentfill}%
\pgfsetfillopacity{0.500000}%
\pgfsetlinewidth{1.003750pt}%
\definecolor{currentstroke}{rgb}{0.466667,0.866667,0.466667}%
\pgfsetstrokecolor{currentstroke}%
\pgfsetstrokeopacity{0.500000}%
\pgfsetdash{}{0pt}%
\pgfpathmoveto{\pgfqpoint{2.033533in}{0.882222in}}%
\pgfpathlineto{\pgfqpoint{2.054367in}{0.903056in}}%
\pgfpathlineto{\pgfqpoint{2.075200in}{0.882222in}}%
\pgfpathlineto{\pgfqpoint{2.096033in}{0.903056in}}%
\pgfpathlineto{\pgfqpoint{2.075200in}{0.923889in}}%
\pgfpathlineto{\pgfqpoint{2.096033in}{0.944722in}}%
\pgfpathlineto{\pgfqpoint{2.075200in}{0.965556in}}%
\pgfpathlineto{\pgfqpoint{2.054367in}{0.944722in}}%
\pgfpathlineto{\pgfqpoint{2.033533in}{0.965556in}}%
\pgfpathlineto{\pgfqpoint{2.012700in}{0.944722in}}%
\pgfpathlineto{\pgfqpoint{2.033533in}{0.923889in}}%
\pgfpathlineto{\pgfqpoint{2.012700in}{0.903056in}}%
\pgfpathclose%
\pgfusepath{stroke,fill}%
\end{pgfscope}%
\begin{pgfscope}%
\pgfpathrectangle{\pgfqpoint{0.532717in}{0.370679in}}{\pgfqpoint{2.617283in}{1.479321in}}%
\pgfusepath{clip}%
\pgfsetbuttcap%
\pgfsetroundjoin%
\definecolor{currentfill}{rgb}{0.466667,0.866667,0.466667}%
\pgfsetfillcolor{currentfill}%
\pgfsetfillopacity{0.500000}%
\pgfsetlinewidth{1.003750pt}%
\definecolor{currentstroke}{rgb}{0.466667,0.866667,0.466667}%
\pgfsetstrokecolor{currentstroke}%
\pgfsetstrokeopacity{0.500000}%
\pgfsetdash{}{0pt}%
\pgfpathmoveto{\pgfqpoint{1.952949in}{1.257636in}}%
\pgfpathlineto{\pgfqpoint{1.973782in}{1.278469in}}%
\pgfpathlineto{\pgfqpoint{1.994616in}{1.257636in}}%
\pgfpathlineto{\pgfqpoint{2.015449in}{1.278469in}}%
\pgfpathlineto{\pgfqpoint{1.994616in}{1.299302in}}%
\pgfpathlineto{\pgfqpoint{2.015449in}{1.320136in}}%
\pgfpathlineto{\pgfqpoint{1.994616in}{1.340969in}}%
\pgfpathlineto{\pgfqpoint{1.973782in}{1.320136in}}%
\pgfpathlineto{\pgfqpoint{1.952949in}{1.340969in}}%
\pgfpathlineto{\pgfqpoint{1.932116in}{1.320136in}}%
\pgfpathlineto{\pgfqpoint{1.952949in}{1.299302in}}%
\pgfpathlineto{\pgfqpoint{1.932116in}{1.278469in}}%
\pgfpathclose%
\pgfusepath{stroke,fill}%
\end{pgfscope}%
\begin{pgfscope}%
\pgfpathrectangle{\pgfqpoint{0.532717in}{0.370679in}}{\pgfqpoint{2.617283in}{1.479321in}}%
\pgfusepath{clip}%
\pgfsetbuttcap%
\pgfsetroundjoin%
\definecolor{currentfill}{rgb}{0.466667,0.866667,0.466667}%
\pgfsetfillcolor{currentfill}%
\pgfsetfillopacity{0.500000}%
\pgfsetlinewidth{1.003750pt}%
\definecolor{currentstroke}{rgb}{0.466667,0.866667,0.466667}%
\pgfsetstrokecolor{currentstroke}%
\pgfsetstrokeopacity{0.500000}%
\pgfsetdash{}{0pt}%
\pgfpathmoveto{\pgfqpoint{1.825131in}{1.248457in}}%
\pgfpathlineto{\pgfqpoint{1.845965in}{1.269290in}}%
\pgfpathlineto{\pgfqpoint{1.866798in}{1.248457in}}%
\pgfpathlineto{\pgfqpoint{1.887631in}{1.269290in}}%
\pgfpathlineto{\pgfqpoint{1.866798in}{1.290123in}}%
\pgfpathlineto{\pgfqpoint{1.887631in}{1.310957in}}%
\pgfpathlineto{\pgfqpoint{1.866798in}{1.331790in}}%
\pgfpathlineto{\pgfqpoint{1.845965in}{1.310957in}}%
\pgfpathlineto{\pgfqpoint{1.825131in}{1.331790in}}%
\pgfpathlineto{\pgfqpoint{1.804298in}{1.310957in}}%
\pgfpathlineto{\pgfqpoint{1.825131in}{1.290123in}}%
\pgfpathlineto{\pgfqpoint{1.804298in}{1.269290in}}%
\pgfpathclose%
\pgfusepath{stroke,fill}%
\end{pgfscope}%
\begin{pgfscope}%
\pgfpathrectangle{\pgfqpoint{0.532717in}{0.370679in}}{\pgfqpoint{2.617283in}{1.479321in}}%
\pgfusepath{clip}%
\pgfsetbuttcap%
\pgfsetroundjoin%
\definecolor{currentfill}{rgb}{0.466667,0.866667,0.466667}%
\pgfsetfillcolor{currentfill}%
\pgfsetfillopacity{0.500000}%
\pgfsetlinewidth{1.003750pt}%
\definecolor{currentstroke}{rgb}{0.466667,0.866667,0.466667}%
\pgfsetstrokecolor{currentstroke}%
\pgfsetstrokeopacity{0.500000}%
\pgfsetdash{}{0pt}%
\pgfpathmoveto{\pgfqpoint{2.037980in}{1.344529in}}%
\pgfpathlineto{\pgfqpoint{2.058814in}{1.365362in}}%
\pgfpathlineto{\pgfqpoint{2.079647in}{1.344529in}}%
\pgfpathlineto{\pgfqpoint{2.100480in}{1.365362in}}%
\pgfpathlineto{\pgfqpoint{2.079647in}{1.386195in}}%
\pgfpathlineto{\pgfqpoint{2.100480in}{1.407029in}}%
\pgfpathlineto{\pgfqpoint{2.079647in}{1.427862in}}%
\pgfpathlineto{\pgfqpoint{2.058814in}{1.407029in}}%
\pgfpathlineto{\pgfqpoint{2.037980in}{1.427862in}}%
\pgfpathlineto{\pgfqpoint{2.017147in}{1.407029in}}%
\pgfpathlineto{\pgfqpoint{2.037980in}{1.386195in}}%
\pgfpathlineto{\pgfqpoint{2.017147in}{1.365362in}}%
\pgfpathclose%
\pgfusepath{stroke,fill}%
\end{pgfscope}%
\begin{pgfscope}%
\pgfpathrectangle{\pgfqpoint{0.532717in}{0.370679in}}{\pgfqpoint{2.617283in}{1.479321in}}%
\pgfusepath{clip}%
\pgfsetbuttcap%
\pgfsetroundjoin%
\definecolor{currentfill}{rgb}{0.466667,0.866667,0.466667}%
\pgfsetfillcolor{currentfill}%
\pgfsetfillopacity{0.500000}%
\pgfsetlinewidth{1.003750pt}%
\definecolor{currentstroke}{rgb}{0.466667,0.866667,0.466667}%
\pgfsetstrokecolor{currentstroke}%
\pgfsetstrokeopacity{0.500000}%
\pgfsetdash{}{0pt}%
\pgfpathmoveto{\pgfqpoint{1.767607in}{1.362684in}}%
\pgfpathlineto{\pgfqpoint{1.788440in}{1.383518in}}%
\pgfpathlineto{\pgfqpoint{1.809274in}{1.362684in}}%
\pgfpathlineto{\pgfqpoint{1.830107in}{1.383518in}}%
\pgfpathlineto{\pgfqpoint{1.809274in}{1.404351in}}%
\pgfpathlineto{\pgfqpoint{1.830107in}{1.425184in}}%
\pgfpathlineto{\pgfqpoint{1.809274in}{1.446018in}}%
\pgfpathlineto{\pgfqpoint{1.788440in}{1.425184in}}%
\pgfpathlineto{\pgfqpoint{1.767607in}{1.446018in}}%
\pgfpathlineto{\pgfqpoint{1.746774in}{1.425184in}}%
\pgfpathlineto{\pgfqpoint{1.767607in}{1.404351in}}%
\pgfpathlineto{\pgfqpoint{1.746774in}{1.383518in}}%
\pgfpathclose%
\pgfusepath{stroke,fill}%
\end{pgfscope}%
\begin{pgfscope}%
\pgfpathrectangle{\pgfqpoint{0.532717in}{0.370679in}}{\pgfqpoint{2.617283in}{1.479321in}}%
\pgfusepath{clip}%
\pgfsetbuttcap%
\pgfsetroundjoin%
\definecolor{currentfill}{rgb}{0.466667,0.866667,0.466667}%
\pgfsetfillcolor{currentfill}%
\pgfsetfillopacity{0.500000}%
\pgfsetlinewidth{1.003750pt}%
\definecolor{currentstroke}{rgb}{0.466667,0.866667,0.466667}%
\pgfsetstrokecolor{currentstroke}%
\pgfsetstrokeopacity{0.500000}%
\pgfsetdash{}{0pt}%
\pgfpathmoveto{\pgfqpoint{2.090268in}{1.135080in}}%
\pgfpathlineto{\pgfqpoint{2.111101in}{1.155914in}}%
\pgfpathlineto{\pgfqpoint{2.131934in}{1.135080in}}%
\pgfpathlineto{\pgfqpoint{2.152768in}{1.155914in}}%
\pgfpathlineto{\pgfqpoint{2.131934in}{1.176747in}}%
\pgfpathlineto{\pgfqpoint{2.152768in}{1.197580in}}%
\pgfpathlineto{\pgfqpoint{2.131934in}{1.218414in}}%
\pgfpathlineto{\pgfqpoint{2.111101in}{1.197580in}}%
\pgfpathlineto{\pgfqpoint{2.090268in}{1.218414in}}%
\pgfpathlineto{\pgfqpoint{2.069434in}{1.197580in}}%
\pgfpathlineto{\pgfqpoint{2.090268in}{1.176747in}}%
\pgfpathlineto{\pgfqpoint{2.069434in}{1.155914in}}%
\pgfpathclose%
\pgfusepath{stroke,fill}%
\end{pgfscope}%
\begin{pgfscope}%
\pgfpathrectangle{\pgfqpoint{0.532717in}{0.370679in}}{\pgfqpoint{2.617283in}{1.479321in}}%
\pgfusepath{clip}%
\pgfsetbuttcap%
\pgfsetroundjoin%
\definecolor{currentfill}{rgb}{0.466667,0.866667,0.466667}%
\pgfsetfillcolor{currentfill}%
\pgfsetfillopacity{0.500000}%
\pgfsetlinewidth{1.003750pt}%
\definecolor{currentstroke}{rgb}{0.466667,0.866667,0.466667}%
\pgfsetstrokecolor{currentstroke}%
\pgfsetstrokeopacity{0.500000}%
\pgfsetdash{}{0pt}%
\pgfpathmoveto{\pgfqpoint{1.861892in}{1.127121in}}%
\pgfpathlineto{\pgfqpoint{1.882725in}{1.147954in}}%
\pgfpathlineto{\pgfqpoint{1.903559in}{1.127121in}}%
\pgfpathlineto{\pgfqpoint{1.924392in}{1.147954in}}%
\pgfpathlineto{\pgfqpoint{1.903559in}{1.168787in}}%
\pgfpathlineto{\pgfqpoint{1.924392in}{1.189621in}}%
\pgfpathlineto{\pgfqpoint{1.903559in}{1.210454in}}%
\pgfpathlineto{\pgfqpoint{1.882725in}{1.189621in}}%
\pgfpathlineto{\pgfqpoint{1.861892in}{1.210454in}}%
\pgfpathlineto{\pgfqpoint{1.841059in}{1.189621in}}%
\pgfpathlineto{\pgfqpoint{1.861892in}{1.168787in}}%
\pgfpathlineto{\pgfqpoint{1.841059in}{1.147954in}}%
\pgfpathclose%
\pgfusepath{stroke,fill}%
\end{pgfscope}%
\begin{pgfscope}%
\pgfpathrectangle{\pgfqpoint{0.532717in}{0.370679in}}{\pgfqpoint{2.617283in}{1.479321in}}%
\pgfusepath{clip}%
\pgfsetbuttcap%
\pgfsetroundjoin%
\definecolor{currentfill}{rgb}{0.466667,0.866667,0.466667}%
\pgfsetfillcolor{currentfill}%
\pgfsetfillopacity{0.500000}%
\pgfsetlinewidth{1.003750pt}%
\definecolor{currentstroke}{rgb}{0.466667,0.866667,0.466667}%
\pgfsetstrokecolor{currentstroke}%
\pgfsetstrokeopacity{0.500000}%
\pgfsetdash{}{0pt}%
\pgfpathmoveto{\pgfqpoint{2.144464in}{1.246461in}}%
\pgfpathlineto{\pgfqpoint{2.165298in}{1.267294in}}%
\pgfpathlineto{\pgfqpoint{2.186131in}{1.246461in}}%
\pgfpathlineto{\pgfqpoint{2.206964in}{1.267294in}}%
\pgfpathlineto{\pgfqpoint{2.186131in}{1.288128in}}%
\pgfpathlineto{\pgfqpoint{2.206964in}{1.308961in}}%
\pgfpathlineto{\pgfqpoint{2.186131in}{1.329794in}}%
\pgfpathlineto{\pgfqpoint{2.165298in}{1.308961in}}%
\pgfpathlineto{\pgfqpoint{2.144464in}{1.329794in}}%
\pgfpathlineto{\pgfqpoint{2.123631in}{1.308961in}}%
\pgfpathlineto{\pgfqpoint{2.144464in}{1.288128in}}%
\pgfpathlineto{\pgfqpoint{2.123631in}{1.267294in}}%
\pgfpathclose%
\pgfusepath{stroke,fill}%
\end{pgfscope}%
\begin{pgfscope}%
\pgfpathrectangle{\pgfqpoint{0.532717in}{0.370679in}}{\pgfqpoint{2.617283in}{1.479321in}}%
\pgfusepath{clip}%
\pgfsetbuttcap%
\pgfsetroundjoin%
\definecolor{currentfill}{rgb}{0.466667,0.866667,0.466667}%
\pgfsetfillcolor{currentfill}%
\pgfsetfillopacity{0.500000}%
\pgfsetlinewidth{1.003750pt}%
\definecolor{currentstroke}{rgb}{0.466667,0.866667,0.466667}%
\pgfsetstrokecolor{currentstroke}%
\pgfsetstrokeopacity{0.500000}%
\pgfsetdash{}{0pt}%
\pgfpathmoveto{\pgfqpoint{2.031407in}{1.180016in}}%
\pgfpathlineto{\pgfqpoint{2.052241in}{1.200849in}}%
\pgfpathlineto{\pgfqpoint{2.073074in}{1.180016in}}%
\pgfpathlineto{\pgfqpoint{2.093907in}{1.200849in}}%
\pgfpathlineto{\pgfqpoint{2.073074in}{1.221682in}}%
\pgfpathlineto{\pgfqpoint{2.093907in}{1.242516in}}%
\pgfpathlineto{\pgfqpoint{2.073074in}{1.263349in}}%
\pgfpathlineto{\pgfqpoint{2.052241in}{1.242516in}}%
\pgfpathlineto{\pgfqpoint{2.031407in}{1.263349in}}%
\pgfpathlineto{\pgfqpoint{2.010574in}{1.242516in}}%
\pgfpathlineto{\pgfqpoint{2.031407in}{1.221682in}}%
\pgfpathlineto{\pgfqpoint{2.010574in}{1.200849in}}%
\pgfpathclose%
\pgfusepath{stroke,fill}%
\end{pgfscope}%
\begin{pgfscope}%
\pgfpathrectangle{\pgfqpoint{0.532717in}{0.370679in}}{\pgfqpoint{2.617283in}{1.479321in}}%
\pgfusepath{clip}%
\pgfsetbuttcap%
\pgfsetroundjoin%
\definecolor{currentfill}{rgb}{0.466667,0.866667,0.466667}%
\pgfsetfillcolor{currentfill}%
\pgfsetfillopacity{0.500000}%
\pgfsetlinewidth{1.003750pt}%
\definecolor{currentstroke}{rgb}{0.466667,0.866667,0.466667}%
\pgfsetstrokecolor{currentstroke}%
\pgfsetstrokeopacity{0.500000}%
\pgfsetdash{}{0pt}%
\pgfpathmoveto{\pgfqpoint{1.969312in}{1.027706in}}%
\pgfpathlineto{\pgfqpoint{1.990146in}{1.048540in}}%
\pgfpathlineto{\pgfqpoint{2.010979in}{1.027706in}}%
\pgfpathlineto{\pgfqpoint{2.031812in}{1.048540in}}%
\pgfpathlineto{\pgfqpoint{2.010979in}{1.069373in}}%
\pgfpathlineto{\pgfqpoint{2.031812in}{1.090206in}}%
\pgfpathlineto{\pgfqpoint{2.010979in}{1.111040in}}%
\pgfpathlineto{\pgfqpoint{1.990146in}{1.090206in}}%
\pgfpathlineto{\pgfqpoint{1.969312in}{1.111040in}}%
\pgfpathlineto{\pgfqpoint{1.948479in}{1.090206in}}%
\pgfpathlineto{\pgfqpoint{1.969312in}{1.069373in}}%
\pgfpathlineto{\pgfqpoint{1.948479in}{1.048540in}}%
\pgfpathclose%
\pgfusepath{stroke,fill}%
\end{pgfscope}%
\begin{pgfscope}%
\pgfpathrectangle{\pgfqpoint{0.532717in}{0.370679in}}{\pgfqpoint{2.617283in}{1.479321in}}%
\pgfusepath{clip}%
\pgfsetbuttcap%
\pgfsetroundjoin%
\definecolor{currentfill}{rgb}{0.466667,0.866667,0.466667}%
\pgfsetfillcolor{currentfill}%
\pgfsetfillopacity{0.500000}%
\pgfsetlinewidth{1.003750pt}%
\definecolor{currentstroke}{rgb}{0.466667,0.866667,0.466667}%
\pgfsetstrokecolor{currentstroke}%
\pgfsetstrokeopacity{0.500000}%
\pgfsetdash{}{0pt}%
\pgfpathmoveto{\pgfqpoint{2.025134in}{0.945671in}}%
\pgfpathlineto{\pgfqpoint{2.045967in}{0.966505in}}%
\pgfpathlineto{\pgfqpoint{2.066801in}{0.945671in}}%
\pgfpathlineto{\pgfqpoint{2.087634in}{0.966505in}}%
\pgfpathlineto{\pgfqpoint{2.066801in}{0.987338in}}%
\pgfpathlineto{\pgfqpoint{2.087634in}{1.008171in}}%
\pgfpathlineto{\pgfqpoint{2.066801in}{1.029005in}}%
\pgfpathlineto{\pgfqpoint{2.045967in}{1.008171in}}%
\pgfpathlineto{\pgfqpoint{2.025134in}{1.029005in}}%
\pgfpathlineto{\pgfqpoint{2.004301in}{1.008171in}}%
\pgfpathlineto{\pgfqpoint{2.025134in}{0.987338in}}%
\pgfpathlineto{\pgfqpoint{2.004301in}{0.966505in}}%
\pgfpathclose%
\pgfusepath{stroke,fill}%
\end{pgfscope}%
\begin{pgfscope}%
\pgfpathrectangle{\pgfqpoint{0.532717in}{0.370679in}}{\pgfqpoint{2.617283in}{1.479321in}}%
\pgfusepath{clip}%
\pgfsetbuttcap%
\pgfsetroundjoin%
\definecolor{currentfill}{rgb}{0.466667,0.866667,0.466667}%
\pgfsetfillcolor{currentfill}%
\pgfsetfillopacity{0.500000}%
\pgfsetlinewidth{1.003750pt}%
\definecolor{currentstroke}{rgb}{0.466667,0.866667,0.466667}%
\pgfsetstrokecolor{currentstroke}%
\pgfsetstrokeopacity{0.500000}%
\pgfsetdash{}{0pt}%
\pgfpathmoveto{\pgfqpoint{2.154861in}{0.984549in}}%
\pgfpathlineto{\pgfqpoint{2.175694in}{1.005383in}}%
\pgfpathlineto{\pgfqpoint{2.196527in}{0.984549in}}%
\pgfpathlineto{\pgfqpoint{2.217361in}{1.005383in}}%
\pgfpathlineto{\pgfqpoint{2.196527in}{1.026216in}}%
\pgfpathlineto{\pgfqpoint{2.217361in}{1.047049in}}%
\pgfpathlineto{\pgfqpoint{2.196527in}{1.067883in}}%
\pgfpathlineto{\pgfqpoint{2.175694in}{1.047049in}}%
\pgfpathlineto{\pgfqpoint{2.154861in}{1.067883in}}%
\pgfpathlineto{\pgfqpoint{2.134027in}{1.047049in}}%
\pgfpathlineto{\pgfqpoint{2.154861in}{1.026216in}}%
\pgfpathlineto{\pgfqpoint{2.134027in}{1.005383in}}%
\pgfpathclose%
\pgfusepath{stroke,fill}%
\end{pgfscope}%
\begin{pgfscope}%
\pgfpathrectangle{\pgfqpoint{0.532717in}{0.370679in}}{\pgfqpoint{2.617283in}{1.479321in}}%
\pgfusepath{clip}%
\pgfsetbuttcap%
\pgfsetroundjoin%
\definecolor{currentfill}{rgb}{0.466667,0.866667,0.466667}%
\pgfsetfillcolor{currentfill}%
\pgfsetfillopacity{0.500000}%
\pgfsetlinewidth{1.003750pt}%
\definecolor{currentstroke}{rgb}{0.466667,0.866667,0.466667}%
\pgfsetstrokecolor{currentstroke}%
\pgfsetstrokeopacity{0.500000}%
\pgfsetdash{}{0pt}%
\pgfpathmoveto{\pgfqpoint{2.222928in}{0.974149in}}%
\pgfpathlineto{\pgfqpoint{2.243761in}{0.994982in}}%
\pgfpathlineto{\pgfqpoint{2.264594in}{0.974149in}}%
\pgfpathlineto{\pgfqpoint{2.285428in}{0.994982in}}%
\pgfpathlineto{\pgfqpoint{2.264594in}{1.015816in}}%
\pgfpathlineto{\pgfqpoint{2.285428in}{1.036649in}}%
\pgfpathlineto{\pgfqpoint{2.264594in}{1.057482in}}%
\pgfpathlineto{\pgfqpoint{2.243761in}{1.036649in}}%
\pgfpathlineto{\pgfqpoint{2.222928in}{1.057482in}}%
\pgfpathlineto{\pgfqpoint{2.202094in}{1.036649in}}%
\pgfpathlineto{\pgfqpoint{2.222928in}{1.015816in}}%
\pgfpathlineto{\pgfqpoint{2.202094in}{0.994982in}}%
\pgfpathclose%
\pgfusepath{stroke,fill}%
\end{pgfscope}%
\begin{pgfscope}%
\pgfpathrectangle{\pgfqpoint{0.532717in}{0.370679in}}{\pgfqpoint{2.617283in}{1.479321in}}%
\pgfusepath{clip}%
\pgfsetbuttcap%
\pgfsetroundjoin%
\definecolor{currentfill}{rgb}{0.466667,0.866667,0.466667}%
\pgfsetfillcolor{currentfill}%
\pgfsetfillopacity{0.500000}%
\pgfsetlinewidth{1.003750pt}%
\definecolor{currentstroke}{rgb}{0.466667,0.866667,0.466667}%
\pgfsetstrokecolor{currentstroke}%
\pgfsetstrokeopacity{0.500000}%
\pgfsetdash{}{0pt}%
\pgfpathmoveto{\pgfqpoint{1.998916in}{1.170835in}}%
\pgfpathlineto{\pgfqpoint{2.019750in}{1.191668in}}%
\pgfpathlineto{\pgfqpoint{2.040583in}{1.170835in}}%
\pgfpathlineto{\pgfqpoint{2.061416in}{1.191668in}}%
\pgfpathlineto{\pgfqpoint{2.040583in}{1.212501in}}%
\pgfpathlineto{\pgfqpoint{2.061416in}{1.233335in}}%
\pgfpathlineto{\pgfqpoint{2.040583in}{1.254168in}}%
\pgfpathlineto{\pgfqpoint{2.019750in}{1.233335in}}%
\pgfpathlineto{\pgfqpoint{1.998916in}{1.254168in}}%
\pgfpathlineto{\pgfqpoint{1.978083in}{1.233335in}}%
\pgfpathlineto{\pgfqpoint{1.998916in}{1.212501in}}%
\pgfpathlineto{\pgfqpoint{1.978083in}{1.191668in}}%
\pgfpathclose%
\pgfusepath{stroke,fill}%
\end{pgfscope}%
\begin{pgfscope}%
\pgfpathrectangle{\pgfqpoint{0.532717in}{0.370679in}}{\pgfqpoint{2.617283in}{1.479321in}}%
\pgfusepath{clip}%
\pgfsetbuttcap%
\pgfsetroundjoin%
\definecolor{currentfill}{rgb}{0.466667,0.866667,0.466667}%
\pgfsetfillcolor{currentfill}%
\pgfsetfillopacity{0.500000}%
\pgfsetlinewidth{1.003750pt}%
\definecolor{currentstroke}{rgb}{0.466667,0.866667,0.466667}%
\pgfsetstrokecolor{currentstroke}%
\pgfsetstrokeopacity{0.500000}%
\pgfsetdash{}{0pt}%
\pgfpathmoveto{\pgfqpoint{1.662184in}{1.263636in}}%
\pgfpathlineto{\pgfqpoint{1.683017in}{1.284470in}}%
\pgfpathlineto{\pgfqpoint{1.703850in}{1.263636in}}%
\pgfpathlineto{\pgfqpoint{1.724684in}{1.284470in}}%
\pgfpathlineto{\pgfqpoint{1.703850in}{1.305303in}}%
\pgfpathlineto{\pgfqpoint{1.724684in}{1.326136in}}%
\pgfpathlineto{\pgfqpoint{1.703850in}{1.346970in}}%
\pgfpathlineto{\pgfqpoint{1.683017in}{1.326136in}}%
\pgfpathlineto{\pgfqpoint{1.662184in}{1.346970in}}%
\pgfpathlineto{\pgfqpoint{1.641350in}{1.326136in}}%
\pgfpathlineto{\pgfqpoint{1.662184in}{1.305303in}}%
\pgfpathlineto{\pgfqpoint{1.641350in}{1.284470in}}%
\pgfpathclose%
\pgfusepath{stroke,fill}%
\end{pgfscope}%
\begin{pgfscope}%
\pgfpathrectangle{\pgfqpoint{0.532717in}{0.370679in}}{\pgfqpoint{2.617283in}{1.479321in}}%
\pgfusepath{clip}%
\pgfsetbuttcap%
\pgfsetroundjoin%
\definecolor{currentfill}{rgb}{0.466667,0.866667,0.466667}%
\pgfsetfillcolor{currentfill}%
\pgfsetfillopacity{0.500000}%
\pgfsetlinewidth{1.003750pt}%
\definecolor{currentstroke}{rgb}{0.466667,0.866667,0.466667}%
\pgfsetstrokecolor{currentstroke}%
\pgfsetstrokeopacity{0.500000}%
\pgfsetdash{}{0pt}%
\pgfpathmoveto{\pgfqpoint{1.733450in}{1.418089in}}%
\pgfpathlineto{\pgfqpoint{1.754283in}{1.438922in}}%
\pgfpathlineto{\pgfqpoint{1.775116in}{1.418089in}}%
\pgfpathlineto{\pgfqpoint{1.795950in}{1.438922in}}%
\pgfpathlineto{\pgfqpoint{1.775116in}{1.459755in}}%
\pgfpathlineto{\pgfqpoint{1.795950in}{1.480589in}}%
\pgfpathlineto{\pgfqpoint{1.775116in}{1.501422in}}%
\pgfpathlineto{\pgfqpoint{1.754283in}{1.480589in}}%
\pgfpathlineto{\pgfqpoint{1.733450in}{1.501422in}}%
\pgfpathlineto{\pgfqpoint{1.712616in}{1.480589in}}%
\pgfpathlineto{\pgfqpoint{1.733450in}{1.459755in}}%
\pgfpathlineto{\pgfqpoint{1.712616in}{1.438922in}}%
\pgfpathclose%
\pgfusepath{stroke,fill}%
\end{pgfscope}%
\begin{pgfscope}%
\pgfpathrectangle{\pgfqpoint{0.532717in}{0.370679in}}{\pgfqpoint{2.617283in}{1.479321in}}%
\pgfusepath{clip}%
\pgfsetbuttcap%
\pgfsetroundjoin%
\definecolor{currentfill}{rgb}{0.466667,0.866667,0.466667}%
\pgfsetfillcolor{currentfill}%
\pgfsetfillopacity{0.500000}%
\pgfsetlinewidth{1.003750pt}%
\definecolor{currentstroke}{rgb}{0.466667,0.866667,0.466667}%
\pgfsetstrokecolor{currentstroke}%
\pgfsetstrokeopacity{0.500000}%
\pgfsetdash{}{0pt}%
\pgfpathmoveto{\pgfqpoint{1.696901in}{1.406624in}}%
\pgfpathlineto{\pgfqpoint{1.717734in}{1.427458in}}%
\pgfpathlineto{\pgfqpoint{1.738567in}{1.406624in}}%
\pgfpathlineto{\pgfqpoint{1.759401in}{1.427458in}}%
\pgfpathlineto{\pgfqpoint{1.738567in}{1.448291in}}%
\pgfpathlineto{\pgfqpoint{1.759401in}{1.469124in}}%
\pgfpathlineto{\pgfqpoint{1.738567in}{1.489958in}}%
\pgfpathlineto{\pgfqpoint{1.717734in}{1.469124in}}%
\pgfpathlineto{\pgfqpoint{1.696901in}{1.489958in}}%
\pgfpathlineto{\pgfqpoint{1.676067in}{1.469124in}}%
\pgfpathlineto{\pgfqpoint{1.696901in}{1.448291in}}%
\pgfpathlineto{\pgfqpoint{1.676067in}{1.427458in}}%
\pgfpathclose%
\pgfusepath{stroke,fill}%
\end{pgfscope}%
\begin{pgfscope}%
\pgfpathrectangle{\pgfqpoint{0.532717in}{0.370679in}}{\pgfqpoint{2.617283in}{1.479321in}}%
\pgfusepath{clip}%
\pgfsetbuttcap%
\pgfsetroundjoin%
\definecolor{currentfill}{rgb}{0.466667,0.866667,0.466667}%
\pgfsetfillcolor{currentfill}%
\pgfsetfillopacity{0.500000}%
\pgfsetlinewidth{1.003750pt}%
\definecolor{currentstroke}{rgb}{0.466667,0.866667,0.466667}%
\pgfsetstrokecolor{currentstroke}%
\pgfsetstrokeopacity{0.500000}%
\pgfsetdash{}{0pt}%
\pgfpathmoveto{\pgfqpoint{1.795197in}{1.239205in}}%
\pgfpathlineto{\pgfqpoint{1.816030in}{1.260039in}}%
\pgfpathlineto{\pgfqpoint{1.836864in}{1.239205in}}%
\pgfpathlineto{\pgfqpoint{1.857697in}{1.260039in}}%
\pgfpathlineto{\pgfqpoint{1.836864in}{1.280872in}}%
\pgfpathlineto{\pgfqpoint{1.857697in}{1.301705in}}%
\pgfpathlineto{\pgfqpoint{1.836864in}{1.322539in}}%
\pgfpathlineto{\pgfqpoint{1.816030in}{1.301705in}}%
\pgfpathlineto{\pgfqpoint{1.795197in}{1.322539in}}%
\pgfpathlineto{\pgfqpoint{1.774364in}{1.301705in}}%
\pgfpathlineto{\pgfqpoint{1.795197in}{1.280872in}}%
\pgfpathlineto{\pgfqpoint{1.774364in}{1.260039in}}%
\pgfpathclose%
\pgfusepath{stroke,fill}%
\end{pgfscope}%
\begin{pgfscope}%
\pgfpathrectangle{\pgfqpoint{0.532717in}{0.370679in}}{\pgfqpoint{2.617283in}{1.479321in}}%
\pgfusepath{clip}%
\pgfsetbuttcap%
\pgfsetroundjoin%
\definecolor{currentfill}{rgb}{0.466667,0.866667,0.466667}%
\pgfsetfillcolor{currentfill}%
\pgfsetfillopacity{0.500000}%
\pgfsetlinewidth{1.003750pt}%
\definecolor{currentstroke}{rgb}{0.466667,0.866667,0.466667}%
\pgfsetstrokecolor{currentstroke}%
\pgfsetstrokeopacity{0.500000}%
\pgfsetdash{}{0pt}%
\pgfpathmoveto{\pgfqpoint{2.169203in}{1.289322in}}%
\pgfpathlineto{\pgfqpoint{2.190036in}{1.310155in}}%
\pgfpathlineto{\pgfqpoint{2.210870in}{1.289322in}}%
\pgfpathlineto{\pgfqpoint{2.231703in}{1.310155in}}%
\pgfpathlineto{\pgfqpoint{2.210870in}{1.330989in}}%
\pgfpathlineto{\pgfqpoint{2.231703in}{1.351822in}}%
\pgfpathlineto{\pgfqpoint{2.210870in}{1.372655in}}%
\pgfpathlineto{\pgfqpoint{2.190036in}{1.351822in}}%
\pgfpathlineto{\pgfqpoint{2.169203in}{1.372655in}}%
\pgfpathlineto{\pgfqpoint{2.148370in}{1.351822in}}%
\pgfpathlineto{\pgfqpoint{2.169203in}{1.330989in}}%
\pgfpathlineto{\pgfqpoint{2.148370in}{1.310155in}}%
\pgfpathclose%
\pgfusepath{stroke,fill}%
\end{pgfscope}%
\begin{pgfscope}%
\pgfpathrectangle{\pgfqpoint{0.532717in}{0.370679in}}{\pgfqpoint{2.617283in}{1.479321in}}%
\pgfusepath{clip}%
\pgfsetbuttcap%
\pgfsetroundjoin%
\definecolor{currentfill}{rgb}{0.466667,0.866667,0.466667}%
\pgfsetfillcolor{currentfill}%
\pgfsetfillopacity{0.500000}%
\pgfsetlinewidth{1.003750pt}%
\definecolor{currentstroke}{rgb}{0.466667,0.866667,0.466667}%
\pgfsetstrokecolor{currentstroke}%
\pgfsetstrokeopacity{0.500000}%
\pgfsetdash{}{0pt}%
\pgfpathmoveto{\pgfqpoint{1.931202in}{1.317736in}}%
\pgfpathlineto{\pgfqpoint{1.952035in}{1.338569in}}%
\pgfpathlineto{\pgfqpoint{1.972869in}{1.317736in}}%
\pgfpathlineto{\pgfqpoint{1.993702in}{1.338569in}}%
\pgfpathlineto{\pgfqpoint{1.972869in}{1.359402in}}%
\pgfpathlineto{\pgfqpoint{1.993702in}{1.380236in}}%
\pgfpathlineto{\pgfqpoint{1.972869in}{1.401069in}}%
\pgfpathlineto{\pgfqpoint{1.952035in}{1.380236in}}%
\pgfpathlineto{\pgfqpoint{1.931202in}{1.401069in}}%
\pgfpathlineto{\pgfqpoint{1.910369in}{1.380236in}}%
\pgfpathlineto{\pgfqpoint{1.931202in}{1.359402in}}%
\pgfpathlineto{\pgfqpoint{1.910369in}{1.338569in}}%
\pgfpathclose%
\pgfusepath{stroke,fill}%
\end{pgfscope}%
\begin{pgfscope}%
\pgfpathrectangle{\pgfqpoint{0.532717in}{0.370679in}}{\pgfqpoint{2.617283in}{1.479321in}}%
\pgfusepath{clip}%
\pgfsetbuttcap%
\pgfsetroundjoin%
\definecolor{currentfill}{rgb}{0.466667,0.866667,0.466667}%
\pgfsetfillcolor{currentfill}%
\pgfsetfillopacity{0.500000}%
\pgfsetlinewidth{1.003750pt}%
\definecolor{currentstroke}{rgb}{0.466667,0.866667,0.466667}%
\pgfsetstrokecolor{currentstroke}%
\pgfsetstrokeopacity{0.500000}%
\pgfsetdash{}{0pt}%
\pgfpathmoveto{\pgfqpoint{1.997338in}{1.007259in}}%
\pgfpathlineto{\pgfqpoint{2.018171in}{1.028092in}}%
\pgfpathlineto{\pgfqpoint{2.039004in}{1.007259in}}%
\pgfpathlineto{\pgfqpoint{2.059838in}{1.028092in}}%
\pgfpathlineto{\pgfqpoint{2.039004in}{1.048925in}}%
\pgfpathlineto{\pgfqpoint{2.059838in}{1.069759in}}%
\pgfpathlineto{\pgfqpoint{2.039004in}{1.090592in}}%
\pgfpathlineto{\pgfqpoint{2.018171in}{1.069759in}}%
\pgfpathlineto{\pgfqpoint{1.997338in}{1.090592in}}%
\pgfpathlineto{\pgfqpoint{1.976504in}{1.069759in}}%
\pgfpathlineto{\pgfqpoint{1.997338in}{1.048925in}}%
\pgfpathlineto{\pgfqpoint{1.976504in}{1.028092in}}%
\pgfpathclose%
\pgfusepath{stroke,fill}%
\end{pgfscope}%
\begin{pgfscope}%
\pgfpathrectangle{\pgfqpoint{0.532717in}{0.370679in}}{\pgfqpoint{2.617283in}{1.479321in}}%
\pgfusepath{clip}%
\pgfsetbuttcap%
\pgfsetroundjoin%
\definecolor{currentfill}{rgb}{0.466667,0.866667,0.466667}%
\pgfsetfillcolor{currentfill}%
\pgfsetfillopacity{0.500000}%
\pgfsetlinewidth{1.003750pt}%
\definecolor{currentstroke}{rgb}{0.466667,0.866667,0.466667}%
\pgfsetstrokecolor{currentstroke}%
\pgfsetstrokeopacity{0.500000}%
\pgfsetdash{}{0pt}%
\pgfpathmoveto{\pgfqpoint{2.121598in}{0.909776in}}%
\pgfpathlineto{\pgfqpoint{2.142432in}{0.930610in}}%
\pgfpathlineto{\pgfqpoint{2.163265in}{0.909776in}}%
\pgfpathlineto{\pgfqpoint{2.184098in}{0.930610in}}%
\pgfpathlineto{\pgfqpoint{2.163265in}{0.951443in}}%
\pgfpathlineto{\pgfqpoint{2.184098in}{0.972276in}}%
\pgfpathlineto{\pgfqpoint{2.163265in}{0.993110in}}%
\pgfpathlineto{\pgfqpoint{2.142432in}{0.972276in}}%
\pgfpathlineto{\pgfqpoint{2.121598in}{0.993110in}}%
\pgfpathlineto{\pgfqpoint{2.100765in}{0.972276in}}%
\pgfpathlineto{\pgfqpoint{2.121598in}{0.951443in}}%
\pgfpathlineto{\pgfqpoint{2.100765in}{0.930610in}}%
\pgfpathclose%
\pgfusepath{stroke,fill}%
\end{pgfscope}%
\begin{pgfscope}%
\pgfpathrectangle{\pgfqpoint{0.532717in}{0.370679in}}{\pgfqpoint{2.617283in}{1.479321in}}%
\pgfusepath{clip}%
\pgfsetbuttcap%
\pgfsetroundjoin%
\definecolor{currentfill}{rgb}{0.466667,0.866667,0.466667}%
\pgfsetfillcolor{currentfill}%
\pgfsetfillopacity{0.500000}%
\pgfsetlinewidth{1.003750pt}%
\definecolor{currentstroke}{rgb}{0.466667,0.866667,0.466667}%
\pgfsetstrokecolor{currentstroke}%
\pgfsetstrokeopacity{0.500000}%
\pgfsetdash{}{0pt}%
\pgfpathmoveto{\pgfqpoint{1.999040in}{1.266196in}}%
\pgfpathlineto{\pgfqpoint{2.019874in}{1.287029in}}%
\pgfpathlineto{\pgfqpoint{2.040707in}{1.266196in}}%
\pgfpathlineto{\pgfqpoint{2.061540in}{1.287029in}}%
\pgfpathlineto{\pgfqpoint{2.040707in}{1.307863in}}%
\pgfpathlineto{\pgfqpoint{2.061540in}{1.328696in}}%
\pgfpathlineto{\pgfqpoint{2.040707in}{1.349529in}}%
\pgfpathlineto{\pgfqpoint{2.019874in}{1.328696in}}%
\pgfpathlineto{\pgfqpoint{1.999040in}{1.349529in}}%
\pgfpathlineto{\pgfqpoint{1.978207in}{1.328696in}}%
\pgfpathlineto{\pgfqpoint{1.999040in}{1.307863in}}%
\pgfpathlineto{\pgfqpoint{1.978207in}{1.287029in}}%
\pgfpathclose%
\pgfusepath{stroke,fill}%
\end{pgfscope}%
\begin{pgfscope}%
\pgfpathrectangle{\pgfqpoint{0.532717in}{0.370679in}}{\pgfqpoint{2.617283in}{1.479321in}}%
\pgfusepath{clip}%
\pgfsetbuttcap%
\pgfsetroundjoin%
\definecolor{currentfill}{rgb}{0.466667,0.866667,0.466667}%
\pgfsetfillcolor{currentfill}%
\pgfsetfillopacity{0.500000}%
\pgfsetlinewidth{1.003750pt}%
\definecolor{currentstroke}{rgb}{0.466667,0.866667,0.466667}%
\pgfsetstrokecolor{currentstroke}%
\pgfsetstrokeopacity{0.500000}%
\pgfsetdash{}{0pt}%
\pgfpathmoveto{\pgfqpoint{1.828335in}{1.218617in}}%
\pgfpathlineto{\pgfqpoint{1.849169in}{1.239451in}}%
\pgfpathlineto{\pgfqpoint{1.870002in}{1.218617in}}%
\pgfpathlineto{\pgfqpoint{1.890835in}{1.239451in}}%
\pgfpathlineto{\pgfqpoint{1.870002in}{1.260284in}}%
\pgfpathlineto{\pgfqpoint{1.890835in}{1.281117in}}%
\pgfpathlineto{\pgfqpoint{1.870002in}{1.301951in}}%
\pgfpathlineto{\pgfqpoint{1.849169in}{1.281117in}}%
\pgfpathlineto{\pgfqpoint{1.828335in}{1.301951in}}%
\pgfpathlineto{\pgfqpoint{1.807502in}{1.281117in}}%
\pgfpathlineto{\pgfqpoint{1.828335in}{1.260284in}}%
\pgfpathlineto{\pgfqpoint{1.807502in}{1.239451in}}%
\pgfpathclose%
\pgfusepath{stroke,fill}%
\end{pgfscope}%
\begin{pgfscope}%
\pgfpathrectangle{\pgfqpoint{0.532717in}{0.370679in}}{\pgfqpoint{2.617283in}{1.479321in}}%
\pgfusepath{clip}%
\pgfsetbuttcap%
\pgfsetroundjoin%
\definecolor{currentfill}{rgb}{0.466667,0.866667,0.466667}%
\pgfsetfillcolor{currentfill}%
\pgfsetfillopacity{0.500000}%
\pgfsetlinewidth{1.003750pt}%
\definecolor{currentstroke}{rgb}{0.466667,0.866667,0.466667}%
\pgfsetstrokecolor{currentstroke}%
\pgfsetstrokeopacity{0.500000}%
\pgfsetdash{}{0pt}%
\pgfpathmoveto{\pgfqpoint{1.804073in}{1.407618in}}%
\pgfpathlineto{\pgfqpoint{1.824907in}{1.428451in}}%
\pgfpathlineto{\pgfqpoint{1.845740in}{1.407618in}}%
\pgfpathlineto{\pgfqpoint{1.866573in}{1.428451in}}%
\pgfpathlineto{\pgfqpoint{1.845740in}{1.449285in}}%
\pgfpathlineto{\pgfqpoint{1.866573in}{1.470118in}}%
\pgfpathlineto{\pgfqpoint{1.845740in}{1.490951in}}%
\pgfpathlineto{\pgfqpoint{1.824907in}{1.470118in}}%
\pgfpathlineto{\pgfqpoint{1.804073in}{1.490951in}}%
\pgfpathlineto{\pgfqpoint{1.783240in}{1.470118in}}%
\pgfpathlineto{\pgfqpoint{1.804073in}{1.449285in}}%
\pgfpathlineto{\pgfqpoint{1.783240in}{1.428451in}}%
\pgfpathclose%
\pgfusepath{stroke,fill}%
\end{pgfscope}%
\begin{pgfscope}%
\pgfpathrectangle{\pgfqpoint{0.532717in}{0.370679in}}{\pgfqpoint{2.617283in}{1.479321in}}%
\pgfusepath{clip}%
\pgfsetbuttcap%
\pgfsetroundjoin%
\definecolor{currentfill}{rgb}{0.466667,0.866667,0.466667}%
\pgfsetfillcolor{currentfill}%
\pgfsetfillopacity{0.500000}%
\pgfsetlinewidth{1.003750pt}%
\definecolor{currentstroke}{rgb}{0.466667,0.866667,0.466667}%
\pgfsetstrokecolor{currentstroke}%
\pgfsetstrokeopacity{0.500000}%
\pgfsetdash{}{0pt}%
\pgfpathmoveto{\pgfqpoint{1.893841in}{1.402979in}}%
\pgfpathlineto{\pgfqpoint{1.914674in}{1.423812in}}%
\pgfpathlineto{\pgfqpoint{1.935507in}{1.402979in}}%
\pgfpathlineto{\pgfqpoint{1.956341in}{1.423812in}}%
\pgfpathlineto{\pgfqpoint{1.935507in}{1.444646in}}%
\pgfpathlineto{\pgfqpoint{1.956341in}{1.465479in}}%
\pgfpathlineto{\pgfqpoint{1.935507in}{1.486312in}}%
\pgfpathlineto{\pgfqpoint{1.914674in}{1.465479in}}%
\pgfpathlineto{\pgfqpoint{1.893841in}{1.486312in}}%
\pgfpathlineto{\pgfqpoint{1.873007in}{1.465479in}}%
\pgfpathlineto{\pgfqpoint{1.893841in}{1.444646in}}%
\pgfpathlineto{\pgfqpoint{1.873007in}{1.423812in}}%
\pgfpathclose%
\pgfusepath{stroke,fill}%
\end{pgfscope}%
\begin{pgfscope}%
\pgfpathrectangle{\pgfqpoint{0.532717in}{0.370679in}}{\pgfqpoint{2.617283in}{1.479321in}}%
\pgfusepath{clip}%
\pgfsetbuttcap%
\pgfsetroundjoin%
\definecolor{currentfill}{rgb}{0.466667,0.866667,0.466667}%
\pgfsetfillcolor{currentfill}%
\pgfsetfillopacity{0.500000}%
\pgfsetlinewidth{1.003750pt}%
\definecolor{currentstroke}{rgb}{0.466667,0.866667,0.466667}%
\pgfsetstrokecolor{currentstroke}%
\pgfsetstrokeopacity{0.500000}%
\pgfsetdash{}{0pt}%
\pgfpathmoveto{\pgfqpoint{2.022283in}{1.112011in}}%
\pgfpathlineto{\pgfqpoint{2.043116in}{1.132845in}}%
\pgfpathlineto{\pgfqpoint{2.063950in}{1.112011in}}%
\pgfpathlineto{\pgfqpoint{2.084783in}{1.132845in}}%
\pgfpathlineto{\pgfqpoint{2.063950in}{1.153678in}}%
\pgfpathlineto{\pgfqpoint{2.084783in}{1.174511in}}%
\pgfpathlineto{\pgfqpoint{2.063950in}{1.195345in}}%
\pgfpathlineto{\pgfqpoint{2.043116in}{1.174511in}}%
\pgfpathlineto{\pgfqpoint{2.022283in}{1.195345in}}%
\pgfpathlineto{\pgfqpoint{2.001450in}{1.174511in}}%
\pgfpathlineto{\pgfqpoint{2.022283in}{1.153678in}}%
\pgfpathlineto{\pgfqpoint{2.001450in}{1.132845in}}%
\pgfpathclose%
\pgfusepath{stroke,fill}%
\end{pgfscope}%
\begin{pgfscope}%
\pgfpathrectangle{\pgfqpoint{0.532717in}{0.370679in}}{\pgfqpoint{2.617283in}{1.479321in}}%
\pgfusepath{clip}%
\pgfsetbuttcap%
\pgfsetroundjoin%
\definecolor{currentfill}{rgb}{0.466667,0.866667,0.466667}%
\pgfsetfillcolor{currentfill}%
\pgfsetfillopacity{0.500000}%
\pgfsetlinewidth{1.003750pt}%
\definecolor{currentstroke}{rgb}{0.466667,0.866667,0.466667}%
\pgfsetstrokecolor{currentstroke}%
\pgfsetstrokeopacity{0.500000}%
\pgfsetdash{}{0pt}%
\pgfpathmoveto{\pgfqpoint{1.823429in}{1.280650in}}%
\pgfpathlineto{\pgfqpoint{1.844262in}{1.301483in}}%
\pgfpathlineto{\pgfqpoint{1.865095in}{1.280650in}}%
\pgfpathlineto{\pgfqpoint{1.885929in}{1.301483in}}%
\pgfpathlineto{\pgfqpoint{1.865095in}{1.322316in}}%
\pgfpathlineto{\pgfqpoint{1.885929in}{1.343150in}}%
\pgfpathlineto{\pgfqpoint{1.865095in}{1.363983in}}%
\pgfpathlineto{\pgfqpoint{1.844262in}{1.343150in}}%
\pgfpathlineto{\pgfqpoint{1.823429in}{1.363983in}}%
\pgfpathlineto{\pgfqpoint{1.802595in}{1.343150in}}%
\pgfpathlineto{\pgfqpoint{1.823429in}{1.322316in}}%
\pgfpathlineto{\pgfqpoint{1.802595in}{1.301483in}}%
\pgfpathclose%
\pgfusepath{stroke,fill}%
\end{pgfscope}%
\begin{pgfscope}%
\pgfpathrectangle{\pgfqpoint{0.532717in}{0.370679in}}{\pgfqpoint{2.617283in}{1.479321in}}%
\pgfusepath{clip}%
\pgfsetbuttcap%
\pgfsetroundjoin%
\definecolor{currentfill}{rgb}{0.466667,0.866667,0.466667}%
\pgfsetfillcolor{currentfill}%
\pgfsetfillopacity{0.500000}%
\pgfsetlinewidth{1.003750pt}%
\definecolor{currentstroke}{rgb}{0.466667,0.866667,0.466667}%
\pgfsetstrokecolor{currentstroke}%
\pgfsetstrokeopacity{0.500000}%
\pgfsetdash{}{0pt}%
\pgfpathmoveto{\pgfqpoint{1.541976in}{1.558094in}}%
\pgfpathlineto{\pgfqpoint{1.562809in}{1.578927in}}%
\pgfpathlineto{\pgfqpoint{1.583642in}{1.558094in}}%
\pgfpathlineto{\pgfqpoint{1.604476in}{1.578927in}}%
\pgfpathlineto{\pgfqpoint{1.583642in}{1.599760in}}%
\pgfpathlineto{\pgfqpoint{1.604476in}{1.620594in}}%
\pgfpathlineto{\pgfqpoint{1.583642in}{1.641427in}}%
\pgfpathlineto{\pgfqpoint{1.562809in}{1.620594in}}%
\pgfpathlineto{\pgfqpoint{1.541976in}{1.641427in}}%
\pgfpathlineto{\pgfqpoint{1.521142in}{1.620594in}}%
\pgfpathlineto{\pgfqpoint{1.541976in}{1.599760in}}%
\pgfpathlineto{\pgfqpoint{1.521142in}{1.578927in}}%
\pgfpathclose%
\pgfusepath{stroke,fill}%
\end{pgfscope}%
\begin{pgfscope}%
\pgfpathrectangle{\pgfqpoint{0.532717in}{0.370679in}}{\pgfqpoint{2.617283in}{1.479321in}}%
\pgfusepath{clip}%
\pgfsetbuttcap%
\pgfsetroundjoin%
\definecolor{currentfill}{rgb}{0.466667,0.866667,0.466667}%
\pgfsetfillcolor{currentfill}%
\pgfsetfillopacity{0.500000}%
\pgfsetlinewidth{1.003750pt}%
\definecolor{currentstroke}{rgb}{0.466667,0.866667,0.466667}%
\pgfsetstrokecolor{currentstroke}%
\pgfsetstrokeopacity{0.500000}%
\pgfsetdash{}{0pt}%
\pgfpathmoveto{\pgfqpoint{1.861515in}{1.326860in}}%
\pgfpathlineto{\pgfqpoint{1.882348in}{1.347693in}}%
\pgfpathlineto{\pgfqpoint{1.903182in}{1.326860in}}%
\pgfpathlineto{\pgfqpoint{1.924015in}{1.347693in}}%
\pgfpathlineto{\pgfqpoint{1.903182in}{1.368526in}}%
\pgfpathlineto{\pgfqpoint{1.924015in}{1.389360in}}%
\pgfpathlineto{\pgfqpoint{1.903182in}{1.410193in}}%
\pgfpathlineto{\pgfqpoint{1.882348in}{1.389360in}}%
\pgfpathlineto{\pgfqpoint{1.861515in}{1.410193in}}%
\pgfpathlineto{\pgfqpoint{1.840682in}{1.389360in}}%
\pgfpathlineto{\pgfqpoint{1.861515in}{1.368526in}}%
\pgfpathlineto{\pgfqpoint{1.840682in}{1.347693in}}%
\pgfpathclose%
\pgfusepath{stroke,fill}%
\end{pgfscope}%
\begin{pgfscope}%
\pgfpathrectangle{\pgfqpoint{0.532717in}{0.370679in}}{\pgfqpoint{2.617283in}{1.479321in}}%
\pgfusepath{clip}%
\pgfsetbuttcap%
\pgfsetroundjoin%
\definecolor{currentfill}{rgb}{0.466667,0.866667,0.466667}%
\pgfsetfillcolor{currentfill}%
\pgfsetfillopacity{0.500000}%
\pgfsetlinewidth{1.003750pt}%
\definecolor{currentstroke}{rgb}{0.466667,0.866667,0.466667}%
\pgfsetstrokecolor{currentstroke}%
\pgfsetstrokeopacity{0.500000}%
\pgfsetdash{}{0pt}%
\pgfpathmoveto{\pgfqpoint{1.854176in}{1.193193in}}%
\pgfpathlineto{\pgfqpoint{1.875009in}{1.214026in}}%
\pgfpathlineto{\pgfqpoint{1.895842in}{1.193193in}}%
\pgfpathlineto{\pgfqpoint{1.916676in}{1.214026in}}%
\pgfpathlineto{\pgfqpoint{1.895842in}{1.234860in}}%
\pgfpathlineto{\pgfqpoint{1.916676in}{1.255693in}}%
\pgfpathlineto{\pgfqpoint{1.895842in}{1.276526in}}%
\pgfpathlineto{\pgfqpoint{1.875009in}{1.255693in}}%
\pgfpathlineto{\pgfqpoint{1.854176in}{1.276526in}}%
\pgfpathlineto{\pgfqpoint{1.833342in}{1.255693in}}%
\pgfpathlineto{\pgfqpoint{1.854176in}{1.234860in}}%
\pgfpathlineto{\pgfqpoint{1.833342in}{1.214026in}}%
\pgfpathclose%
\pgfusepath{stroke,fill}%
\end{pgfscope}%
\begin{pgfscope}%
\pgfpathrectangle{\pgfqpoint{0.532717in}{0.370679in}}{\pgfqpoint{2.617283in}{1.479321in}}%
\pgfusepath{clip}%
\pgfsetbuttcap%
\pgfsetroundjoin%
\definecolor{currentfill}{rgb}{0.466667,0.866667,0.466667}%
\pgfsetfillcolor{currentfill}%
\pgfsetfillopacity{0.500000}%
\pgfsetlinewidth{1.003750pt}%
\definecolor{currentstroke}{rgb}{0.466667,0.866667,0.466667}%
\pgfsetstrokecolor{currentstroke}%
\pgfsetstrokeopacity{0.500000}%
\pgfsetdash{}{0pt}%
\pgfpathmoveto{\pgfqpoint{1.867441in}{1.230011in}}%
\pgfpathlineto{\pgfqpoint{1.888274in}{1.250845in}}%
\pgfpathlineto{\pgfqpoint{1.909107in}{1.230011in}}%
\pgfpathlineto{\pgfqpoint{1.929941in}{1.250845in}}%
\pgfpathlineto{\pgfqpoint{1.909107in}{1.271678in}}%
\pgfpathlineto{\pgfqpoint{1.929941in}{1.292511in}}%
\pgfpathlineto{\pgfqpoint{1.909107in}{1.313345in}}%
\pgfpathlineto{\pgfqpoint{1.888274in}{1.292511in}}%
\pgfpathlineto{\pgfqpoint{1.867441in}{1.313345in}}%
\pgfpathlineto{\pgfqpoint{1.846607in}{1.292511in}}%
\pgfpathlineto{\pgfqpoint{1.867441in}{1.271678in}}%
\pgfpathlineto{\pgfqpoint{1.846607in}{1.250845in}}%
\pgfpathclose%
\pgfusepath{stroke,fill}%
\end{pgfscope}%
\begin{pgfscope}%
\pgfpathrectangle{\pgfqpoint{0.532717in}{0.370679in}}{\pgfqpoint{2.617283in}{1.479321in}}%
\pgfusepath{clip}%
\pgfsetbuttcap%
\pgfsetroundjoin%
\definecolor{currentfill}{rgb}{0.466667,0.866667,0.466667}%
\pgfsetfillcolor{currentfill}%
\pgfsetfillopacity{0.500000}%
\pgfsetlinewidth{1.003750pt}%
\definecolor{currentstroke}{rgb}{0.466667,0.866667,0.466667}%
\pgfsetstrokecolor{currentstroke}%
\pgfsetstrokeopacity{0.500000}%
\pgfsetdash{}{0pt}%
\pgfpathmoveto{\pgfqpoint{1.947565in}{1.087806in}}%
\pgfpathlineto{\pgfqpoint{1.968399in}{1.108640in}}%
\pgfpathlineto{\pgfqpoint{1.989232in}{1.087806in}}%
\pgfpathlineto{\pgfqpoint{2.010065in}{1.108640in}}%
\pgfpathlineto{\pgfqpoint{1.989232in}{1.129473in}}%
\pgfpathlineto{\pgfqpoint{2.010065in}{1.150306in}}%
\pgfpathlineto{\pgfqpoint{1.989232in}{1.171140in}}%
\pgfpathlineto{\pgfqpoint{1.968399in}{1.150306in}}%
\pgfpathlineto{\pgfqpoint{1.947565in}{1.171140in}}%
\pgfpathlineto{\pgfqpoint{1.926732in}{1.150306in}}%
\pgfpathlineto{\pgfqpoint{1.947565in}{1.129473in}}%
\pgfpathlineto{\pgfqpoint{1.926732in}{1.108640in}}%
\pgfpathclose%
\pgfusepath{stroke,fill}%
\end{pgfscope}%
\begin{pgfscope}%
\pgfpathrectangle{\pgfqpoint{0.532717in}{0.370679in}}{\pgfqpoint{2.617283in}{1.479321in}}%
\pgfusepath{clip}%
\pgfsetbuttcap%
\pgfsetroundjoin%
\definecolor{currentfill}{rgb}{0.466667,0.866667,0.466667}%
\pgfsetfillcolor{currentfill}%
\pgfsetfillopacity{0.500000}%
\pgfsetlinewidth{1.003750pt}%
\definecolor{currentstroke}{rgb}{0.466667,0.866667,0.466667}%
\pgfsetstrokecolor{currentstroke}%
\pgfsetstrokeopacity{0.500000}%
\pgfsetdash{}{0pt}%
\pgfpathmoveto{\pgfqpoint{1.481418in}{1.440530in}}%
\pgfpathlineto{\pgfqpoint{1.502251in}{1.461363in}}%
\pgfpathlineto{\pgfqpoint{1.523085in}{1.440530in}}%
\pgfpathlineto{\pgfqpoint{1.543918in}{1.461363in}}%
\pgfpathlineto{\pgfqpoint{1.523085in}{1.482197in}}%
\pgfpathlineto{\pgfqpoint{1.543918in}{1.503030in}}%
\pgfpathlineto{\pgfqpoint{1.523085in}{1.523863in}}%
\pgfpathlineto{\pgfqpoint{1.502251in}{1.503030in}}%
\pgfpathlineto{\pgfqpoint{1.481418in}{1.523863in}}%
\pgfpathlineto{\pgfqpoint{1.460585in}{1.503030in}}%
\pgfpathlineto{\pgfqpoint{1.481418in}{1.482197in}}%
\pgfpathlineto{\pgfqpoint{1.460585in}{1.461363in}}%
\pgfpathclose%
\pgfusepath{stroke,fill}%
\end{pgfscope}%
\begin{pgfscope}%
\pgfpathrectangle{\pgfqpoint{0.532717in}{0.370679in}}{\pgfqpoint{2.617283in}{1.479321in}}%
\pgfusepath{clip}%
\pgfsetbuttcap%
\pgfsetroundjoin%
\definecolor{currentfill}{rgb}{0.466667,0.866667,0.466667}%
\pgfsetfillcolor{currentfill}%
\pgfsetfillopacity{0.500000}%
\pgfsetlinewidth{1.003750pt}%
\definecolor{currentstroke}{rgb}{0.466667,0.866667,0.466667}%
\pgfsetstrokecolor{currentstroke}%
\pgfsetstrokeopacity{0.500000}%
\pgfsetdash{}{0pt}%
\pgfpathmoveto{\pgfqpoint{1.844156in}{1.255366in}}%
\pgfpathlineto{\pgfqpoint{1.864990in}{1.276199in}}%
\pgfpathlineto{\pgfqpoint{1.885823in}{1.255366in}}%
\pgfpathlineto{\pgfqpoint{1.906656in}{1.276199in}}%
\pgfpathlineto{\pgfqpoint{1.885823in}{1.297032in}}%
\pgfpathlineto{\pgfqpoint{1.906656in}{1.317866in}}%
\pgfpathlineto{\pgfqpoint{1.885823in}{1.338699in}}%
\pgfpathlineto{\pgfqpoint{1.864990in}{1.317866in}}%
\pgfpathlineto{\pgfqpoint{1.844156in}{1.338699in}}%
\pgfpathlineto{\pgfqpoint{1.823323in}{1.317866in}}%
\pgfpathlineto{\pgfqpoint{1.844156in}{1.297032in}}%
\pgfpathlineto{\pgfqpoint{1.823323in}{1.276199in}}%
\pgfpathclose%
\pgfusepath{stroke,fill}%
\end{pgfscope}%
\begin{pgfscope}%
\pgfpathrectangle{\pgfqpoint{0.532717in}{0.370679in}}{\pgfqpoint{2.617283in}{1.479321in}}%
\pgfusepath{clip}%
\pgfsetbuttcap%
\pgfsetroundjoin%
\definecolor{currentfill}{rgb}{0.423529,0.627451,0.862745}%
\pgfsetfillcolor{currentfill}%
\pgfsetfillopacity{0.500000}%
\pgfsetlinewidth{1.003750pt}%
\definecolor{currentstroke}{rgb}{0.423529,0.627451,0.862745}%
\pgfsetstrokecolor{currentstroke}%
\pgfsetstrokeopacity{0.500000}%
\pgfsetdash{}{0pt}%
\pgfpathmoveto{\pgfqpoint{2.536753in}{1.084697in}}%
\pgfpathlineto{\pgfqpoint{2.595678in}{1.143622in}}%
\pgfpathlineto{\pgfqpoint{2.536753in}{1.202548in}}%
\pgfpathlineto{\pgfqpoint{2.477827in}{1.143622in}}%
\pgfpathclose%
\pgfusepath{stroke,fill}%
\end{pgfscope}%
\begin{pgfscope}%
\pgfpathrectangle{\pgfqpoint{0.532717in}{0.370679in}}{\pgfqpoint{2.617283in}{1.479321in}}%
\pgfusepath{clip}%
\pgfsetbuttcap%
\pgfsetroundjoin%
\definecolor{currentfill}{rgb}{0.423529,0.627451,0.862745}%
\pgfsetfillcolor{currentfill}%
\pgfsetfillopacity{0.500000}%
\pgfsetlinewidth{1.003750pt}%
\definecolor{currentstroke}{rgb}{0.423529,0.627451,0.862745}%
\pgfsetstrokecolor{currentstroke}%
\pgfsetstrokeopacity{0.500000}%
\pgfsetdash{}{0pt}%
\pgfpathmoveto{\pgfqpoint{2.200662in}{1.342421in}}%
\pgfpathlineto{\pgfqpoint{2.259588in}{1.401346in}}%
\pgfpathlineto{\pgfqpoint{2.200662in}{1.460272in}}%
\pgfpathlineto{\pgfqpoint{2.141737in}{1.401346in}}%
\pgfpathclose%
\pgfusepath{stroke,fill}%
\end{pgfscope}%
\begin{pgfscope}%
\pgfpathrectangle{\pgfqpoint{0.532717in}{0.370679in}}{\pgfqpoint{2.617283in}{1.479321in}}%
\pgfusepath{clip}%
\pgfsetbuttcap%
\pgfsetroundjoin%
\definecolor{currentfill}{rgb}{0.423529,0.627451,0.862745}%
\pgfsetfillcolor{currentfill}%
\pgfsetfillopacity{0.500000}%
\pgfsetlinewidth{1.003750pt}%
\definecolor{currentstroke}{rgb}{0.423529,0.627451,0.862745}%
\pgfsetstrokecolor{currentstroke}%
\pgfsetstrokeopacity{0.500000}%
\pgfsetdash{}{0pt}%
\pgfpathmoveto{\pgfqpoint{2.562240in}{0.922772in}}%
\pgfpathlineto{\pgfqpoint{2.621166in}{0.981697in}}%
\pgfpathlineto{\pgfqpoint{2.562240in}{1.040623in}}%
\pgfpathlineto{\pgfqpoint{2.503314in}{0.981697in}}%
\pgfpathclose%
\pgfusepath{stroke,fill}%
\end{pgfscope}%
\begin{pgfscope}%
\pgfpathrectangle{\pgfqpoint{0.532717in}{0.370679in}}{\pgfqpoint{2.617283in}{1.479321in}}%
\pgfusepath{clip}%
\pgfsetbuttcap%
\pgfsetroundjoin%
\definecolor{currentfill}{rgb}{0.423529,0.627451,0.862745}%
\pgfsetfillcolor{currentfill}%
\pgfsetfillopacity{0.500000}%
\pgfsetlinewidth{1.003750pt}%
\definecolor{currentstroke}{rgb}{0.423529,0.627451,0.862745}%
\pgfsetstrokecolor{currentstroke}%
\pgfsetstrokeopacity{0.500000}%
\pgfsetdash{}{0pt}%
\pgfpathmoveto{\pgfqpoint{2.368080in}{1.162178in}}%
\pgfpathlineto{\pgfqpoint{2.427006in}{1.221104in}}%
\pgfpathlineto{\pgfqpoint{2.368080in}{1.280029in}}%
\pgfpathlineto{\pgfqpoint{2.309155in}{1.221104in}}%
\pgfpathclose%
\pgfusepath{stroke,fill}%
\end{pgfscope}%
\begin{pgfscope}%
\pgfpathrectangle{\pgfqpoint{0.532717in}{0.370679in}}{\pgfqpoint{2.617283in}{1.479321in}}%
\pgfusepath{clip}%
\pgfsetbuttcap%
\pgfsetroundjoin%
\definecolor{currentfill}{rgb}{0.423529,0.627451,0.862745}%
\pgfsetfillcolor{currentfill}%
\pgfsetfillopacity{0.500000}%
\pgfsetlinewidth{1.003750pt}%
\definecolor{currentstroke}{rgb}{0.423529,0.627451,0.862745}%
\pgfsetstrokecolor{currentstroke}%
\pgfsetstrokeopacity{0.500000}%
\pgfsetdash{}{0pt}%
\pgfpathmoveto{\pgfqpoint{2.482033in}{1.098446in}}%
\pgfpathlineto{\pgfqpoint{2.540958in}{1.157372in}}%
\pgfpathlineto{\pgfqpoint{2.482033in}{1.216297in}}%
\pgfpathlineto{\pgfqpoint{2.423107in}{1.157372in}}%
\pgfpathclose%
\pgfusepath{stroke,fill}%
\end{pgfscope}%
\begin{pgfscope}%
\pgfpathrectangle{\pgfqpoint{0.532717in}{0.370679in}}{\pgfqpoint{2.617283in}{1.479321in}}%
\pgfusepath{clip}%
\pgfsetbuttcap%
\pgfsetroundjoin%
\definecolor{currentfill}{rgb}{0.423529,0.627451,0.862745}%
\pgfsetfillcolor{currentfill}%
\pgfsetfillopacity{0.500000}%
\pgfsetlinewidth{1.003750pt}%
\definecolor{currentstroke}{rgb}{0.423529,0.627451,0.862745}%
\pgfsetstrokecolor{currentstroke}%
\pgfsetstrokeopacity{0.500000}%
\pgfsetdash{}{0pt}%
\pgfpathmoveto{\pgfqpoint{2.796913in}{0.828836in}}%
\pgfpathlineto{\pgfqpoint{2.855839in}{0.887762in}}%
\pgfpathlineto{\pgfqpoint{2.796913in}{0.946687in}}%
\pgfpathlineto{\pgfqpoint{2.737987in}{0.887762in}}%
\pgfpathclose%
\pgfusepath{stroke,fill}%
\end{pgfscope}%
\begin{pgfscope}%
\pgfpathrectangle{\pgfqpoint{0.532717in}{0.370679in}}{\pgfqpoint{2.617283in}{1.479321in}}%
\pgfusepath{clip}%
\pgfsetbuttcap%
\pgfsetroundjoin%
\definecolor{currentfill}{rgb}{0.423529,0.627451,0.862745}%
\pgfsetfillcolor{currentfill}%
\pgfsetfillopacity{0.500000}%
\pgfsetlinewidth{1.003750pt}%
\definecolor{currentstroke}{rgb}{0.423529,0.627451,0.862745}%
\pgfsetstrokecolor{currentstroke}%
\pgfsetstrokeopacity{0.500000}%
\pgfsetdash{}{0pt}%
\pgfpathmoveto{\pgfqpoint{1.931620in}{1.624561in}}%
\pgfpathlineto{\pgfqpoint{1.990545in}{1.683486in}}%
\pgfpathlineto{\pgfqpoint{1.931620in}{1.742412in}}%
\pgfpathlineto{\pgfqpoint{1.872694in}{1.683486in}}%
\pgfpathclose%
\pgfusepath{stroke,fill}%
\end{pgfscope}%
\begin{pgfscope}%
\pgfpathrectangle{\pgfqpoint{0.532717in}{0.370679in}}{\pgfqpoint{2.617283in}{1.479321in}}%
\pgfusepath{clip}%
\pgfsetbuttcap%
\pgfsetroundjoin%
\definecolor{currentfill}{rgb}{0.423529,0.627451,0.862745}%
\pgfsetfillcolor{currentfill}%
\pgfsetfillopacity{0.500000}%
\pgfsetlinewidth{1.003750pt}%
\definecolor{currentstroke}{rgb}{0.423529,0.627451,0.862745}%
\pgfsetstrokecolor{currentstroke}%
\pgfsetstrokeopacity{0.500000}%
\pgfsetdash{}{0pt}%
\pgfpathmoveto{\pgfqpoint{2.657120in}{0.917993in}}%
\pgfpathlineto{\pgfqpoint{2.716046in}{0.976918in}}%
\pgfpathlineto{\pgfqpoint{2.657120in}{1.035844in}}%
\pgfpathlineto{\pgfqpoint{2.598195in}{0.976918in}}%
\pgfpathclose%
\pgfusepath{stroke,fill}%
\end{pgfscope}%
\begin{pgfscope}%
\pgfpathrectangle{\pgfqpoint{0.532717in}{0.370679in}}{\pgfqpoint{2.617283in}{1.479321in}}%
\pgfusepath{clip}%
\pgfsetbuttcap%
\pgfsetroundjoin%
\definecolor{currentfill}{rgb}{0.423529,0.627451,0.862745}%
\pgfsetfillcolor{currentfill}%
\pgfsetfillopacity{0.500000}%
\pgfsetlinewidth{1.003750pt}%
\definecolor{currentstroke}{rgb}{0.423529,0.627451,0.862745}%
\pgfsetstrokecolor{currentstroke}%
\pgfsetstrokeopacity{0.500000}%
\pgfsetdash{}{0pt}%
\pgfpathmoveto{\pgfqpoint{2.472986in}{1.191664in}}%
\pgfpathlineto{\pgfqpoint{2.531911in}{1.250590in}}%
\pgfpathlineto{\pgfqpoint{2.472986in}{1.309515in}}%
\pgfpathlineto{\pgfqpoint{2.414060in}{1.250590in}}%
\pgfpathclose%
\pgfusepath{stroke,fill}%
\end{pgfscope}%
\begin{pgfscope}%
\pgfpathrectangle{\pgfqpoint{0.532717in}{0.370679in}}{\pgfqpoint{2.617283in}{1.479321in}}%
\pgfusepath{clip}%
\pgfsetbuttcap%
\pgfsetroundjoin%
\definecolor{currentfill}{rgb}{0.423529,0.627451,0.862745}%
\pgfsetfillcolor{currentfill}%
\pgfsetfillopacity{0.500000}%
\pgfsetlinewidth{1.003750pt}%
\definecolor{currentstroke}{rgb}{0.423529,0.627451,0.862745}%
\pgfsetstrokecolor{currentstroke}%
\pgfsetstrokeopacity{0.500000}%
\pgfsetdash{}{0pt}%
\pgfpathmoveto{\pgfqpoint{2.652949in}{0.722094in}}%
\pgfpathlineto{\pgfqpoint{2.711875in}{0.781020in}}%
\pgfpathlineto{\pgfqpoint{2.652949in}{0.839945in}}%
\pgfpathlineto{\pgfqpoint{2.594024in}{0.781020in}}%
\pgfpathclose%
\pgfusepath{stroke,fill}%
\end{pgfscope}%
\begin{pgfscope}%
\pgfpathrectangle{\pgfqpoint{0.532717in}{0.370679in}}{\pgfqpoint{2.617283in}{1.479321in}}%
\pgfusepath{clip}%
\pgfsetbuttcap%
\pgfsetroundjoin%
\definecolor{currentfill}{rgb}{0.423529,0.627451,0.862745}%
\pgfsetfillcolor{currentfill}%
\pgfsetfillopacity{0.500000}%
\pgfsetlinewidth{1.003750pt}%
\definecolor{currentstroke}{rgb}{0.423529,0.627451,0.862745}%
\pgfsetstrokecolor{currentstroke}%
\pgfsetstrokeopacity{0.500000}%
\pgfsetdash{}{0pt}%
\pgfpathmoveto{\pgfqpoint{2.275197in}{0.968495in}}%
\pgfpathlineto{\pgfqpoint{2.334123in}{1.027420in}}%
\pgfpathlineto{\pgfqpoint{2.275197in}{1.086346in}}%
\pgfpathlineto{\pgfqpoint{2.216271in}{1.027420in}}%
\pgfpathclose%
\pgfusepath{stroke,fill}%
\end{pgfscope}%
\begin{pgfscope}%
\pgfpathrectangle{\pgfqpoint{0.532717in}{0.370679in}}{\pgfqpoint{2.617283in}{1.479321in}}%
\pgfusepath{clip}%
\pgfsetbuttcap%
\pgfsetroundjoin%
\definecolor{currentfill}{rgb}{0.423529,0.627451,0.862745}%
\pgfsetfillcolor{currentfill}%
\pgfsetfillopacity{0.500000}%
\pgfsetlinewidth{1.003750pt}%
\definecolor{currentstroke}{rgb}{0.423529,0.627451,0.862745}%
\pgfsetstrokecolor{currentstroke}%
\pgfsetstrokeopacity{0.500000}%
\pgfsetdash{}{0pt}%
\pgfpathmoveto{\pgfqpoint{2.317418in}{1.178211in}}%
\pgfpathlineto{\pgfqpoint{2.376344in}{1.237136in}}%
\pgfpathlineto{\pgfqpoint{2.317418in}{1.296062in}}%
\pgfpathlineto{\pgfqpoint{2.258493in}{1.237136in}}%
\pgfpathclose%
\pgfusepath{stroke,fill}%
\end{pgfscope}%
\begin{pgfscope}%
\pgfpathrectangle{\pgfqpoint{0.532717in}{0.370679in}}{\pgfqpoint{2.617283in}{1.479321in}}%
\pgfusepath{clip}%
\pgfsetbuttcap%
\pgfsetroundjoin%
\definecolor{currentfill}{rgb}{0.423529,0.627451,0.862745}%
\pgfsetfillcolor{currentfill}%
\pgfsetfillopacity{0.500000}%
\pgfsetlinewidth{1.003750pt}%
\definecolor{currentstroke}{rgb}{0.423529,0.627451,0.862745}%
\pgfsetstrokecolor{currentstroke}%
\pgfsetstrokeopacity{0.500000}%
\pgfsetdash{}{0pt}%
\pgfpathmoveto{\pgfqpoint{2.426588in}{0.980742in}}%
\pgfpathlineto{\pgfqpoint{2.485513in}{1.039668in}}%
\pgfpathlineto{\pgfqpoint{2.426588in}{1.098593in}}%
\pgfpathlineto{\pgfqpoint{2.367662in}{1.039668in}}%
\pgfpathclose%
\pgfusepath{stroke,fill}%
\end{pgfscope}%
\begin{pgfscope}%
\pgfpathrectangle{\pgfqpoint{0.532717in}{0.370679in}}{\pgfqpoint{2.617283in}{1.479321in}}%
\pgfusepath{clip}%
\pgfsetbuttcap%
\pgfsetroundjoin%
\definecolor{currentfill}{rgb}{0.423529,0.627451,0.862745}%
\pgfsetfillcolor{currentfill}%
\pgfsetfillopacity{0.500000}%
\pgfsetlinewidth{1.003750pt}%
\definecolor{currentstroke}{rgb}{0.423529,0.627451,0.862745}%
\pgfsetstrokecolor{currentstroke}%
\pgfsetstrokeopacity{0.500000}%
\pgfsetdash{}{0pt}%
\pgfpathmoveto{\pgfqpoint{2.179769in}{1.434644in}}%
\pgfpathlineto{\pgfqpoint{2.238695in}{1.493569in}}%
\pgfpathlineto{\pgfqpoint{2.179769in}{1.552495in}}%
\pgfpathlineto{\pgfqpoint{2.120844in}{1.493569in}}%
\pgfpathclose%
\pgfusepath{stroke,fill}%
\end{pgfscope}%
\begin{pgfscope}%
\pgfpathrectangle{\pgfqpoint{0.532717in}{0.370679in}}{\pgfqpoint{2.617283in}{1.479321in}}%
\pgfusepath{clip}%
\pgfsetbuttcap%
\pgfsetroundjoin%
\definecolor{currentfill}{rgb}{0.423529,0.627451,0.862745}%
\pgfsetfillcolor{currentfill}%
\pgfsetfillopacity{0.500000}%
\pgfsetlinewidth{1.003750pt}%
\definecolor{currentstroke}{rgb}{0.423529,0.627451,0.862745}%
\pgfsetstrokecolor{currentstroke}%
\pgfsetstrokeopacity{0.500000}%
\pgfsetdash{}{0pt}%
\pgfpathmoveto{\pgfqpoint{2.252142in}{1.326118in}}%
\pgfpathlineto{\pgfqpoint{2.311068in}{1.385044in}}%
\pgfpathlineto{\pgfqpoint{2.252142in}{1.443970in}}%
\pgfpathlineto{\pgfqpoint{2.193217in}{1.385044in}}%
\pgfpathclose%
\pgfusepath{stroke,fill}%
\end{pgfscope}%
\begin{pgfscope}%
\pgfpathrectangle{\pgfqpoint{0.532717in}{0.370679in}}{\pgfqpoint{2.617283in}{1.479321in}}%
\pgfusepath{clip}%
\pgfsetbuttcap%
\pgfsetroundjoin%
\definecolor{currentfill}{rgb}{0.423529,0.627451,0.862745}%
\pgfsetfillcolor{currentfill}%
\pgfsetfillopacity{0.500000}%
\pgfsetlinewidth{1.003750pt}%
\definecolor{currentstroke}{rgb}{0.423529,0.627451,0.862745}%
\pgfsetstrokecolor{currentstroke}%
\pgfsetstrokeopacity{0.500000}%
\pgfsetdash{}{0pt}%
\pgfpathmoveto{\pgfqpoint{2.348212in}{1.024893in}}%
\pgfpathlineto{\pgfqpoint{2.407138in}{1.083818in}}%
\pgfpathlineto{\pgfqpoint{2.348212in}{1.142744in}}%
\pgfpathlineto{\pgfqpoint{2.289287in}{1.083818in}}%
\pgfpathclose%
\pgfusepath{stroke,fill}%
\end{pgfscope}%
\begin{pgfscope}%
\pgfpathrectangle{\pgfqpoint{0.532717in}{0.370679in}}{\pgfqpoint{2.617283in}{1.479321in}}%
\pgfusepath{clip}%
\pgfsetbuttcap%
\pgfsetroundjoin%
\definecolor{currentfill}{rgb}{0.423529,0.627451,0.862745}%
\pgfsetfillcolor{currentfill}%
\pgfsetfillopacity{0.500000}%
\pgfsetlinewidth{1.003750pt}%
\definecolor{currentstroke}{rgb}{0.423529,0.627451,0.862745}%
\pgfsetstrokecolor{currentstroke}%
\pgfsetstrokeopacity{0.500000}%
\pgfsetdash{}{0pt}%
\pgfpathmoveto{\pgfqpoint{2.361595in}{1.060634in}}%
\pgfpathlineto{\pgfqpoint{2.420521in}{1.119560in}}%
\pgfpathlineto{\pgfqpoint{2.361595in}{1.178485in}}%
\pgfpathlineto{\pgfqpoint{2.302670in}{1.119560in}}%
\pgfpathclose%
\pgfusepath{stroke,fill}%
\end{pgfscope}%
\begin{pgfscope}%
\pgfpathrectangle{\pgfqpoint{0.532717in}{0.370679in}}{\pgfqpoint{2.617283in}{1.479321in}}%
\pgfusepath{clip}%
\pgfsetbuttcap%
\pgfsetroundjoin%
\definecolor{currentfill}{rgb}{0.423529,0.627451,0.862745}%
\pgfsetfillcolor{currentfill}%
\pgfsetfillopacity{0.500000}%
\pgfsetlinewidth{1.003750pt}%
\definecolor{currentstroke}{rgb}{0.423529,0.627451,0.862745}%
\pgfsetstrokecolor{currentstroke}%
\pgfsetstrokeopacity{0.500000}%
\pgfsetdash{}{0pt}%
\pgfpathmoveto{\pgfqpoint{2.824544in}{0.543058in}}%
\pgfpathlineto{\pgfqpoint{2.883470in}{0.601983in}}%
\pgfpathlineto{\pgfqpoint{2.824544in}{0.660909in}}%
\pgfpathlineto{\pgfqpoint{2.765618in}{0.601983in}}%
\pgfpathclose%
\pgfusepath{stroke,fill}%
\end{pgfscope}%
\begin{pgfscope}%
\pgfpathrectangle{\pgfqpoint{0.532717in}{0.370679in}}{\pgfqpoint{2.617283in}{1.479321in}}%
\pgfusepath{clip}%
\pgfsetbuttcap%
\pgfsetroundjoin%
\definecolor{currentfill}{rgb}{0.423529,0.627451,0.862745}%
\pgfsetfillcolor{currentfill}%
\pgfsetfillopacity{0.500000}%
\pgfsetlinewidth{1.003750pt}%
\definecolor{currentstroke}{rgb}{0.423529,0.627451,0.862745}%
\pgfsetstrokecolor{currentstroke}%
\pgfsetstrokeopacity{0.500000}%
\pgfsetdash{}{0pt}%
\pgfpathmoveto{\pgfqpoint{2.916538in}{0.963356in}}%
\pgfpathlineto{\pgfqpoint{2.975463in}{1.022282in}}%
\pgfpathlineto{\pgfqpoint{2.916538in}{1.081207in}}%
\pgfpathlineto{\pgfqpoint{2.857612in}{1.022282in}}%
\pgfpathclose%
\pgfusepath{stroke,fill}%
\end{pgfscope}%
\begin{pgfscope}%
\pgfpathrectangle{\pgfqpoint{0.532717in}{0.370679in}}{\pgfqpoint{2.617283in}{1.479321in}}%
\pgfusepath{clip}%
\pgfsetbuttcap%
\pgfsetroundjoin%
\definecolor{currentfill}{rgb}{0.423529,0.627451,0.862745}%
\pgfsetfillcolor{currentfill}%
\pgfsetfillopacity{0.500000}%
\pgfsetlinewidth{1.003750pt}%
\definecolor{currentstroke}{rgb}{0.423529,0.627451,0.862745}%
\pgfsetstrokecolor{currentstroke}%
\pgfsetstrokeopacity{0.500000}%
\pgfsetdash{}{0pt}%
\pgfpathmoveto{\pgfqpoint{2.165857in}{1.427594in}}%
\pgfpathlineto{\pgfqpoint{2.224783in}{1.486520in}}%
\pgfpathlineto{\pgfqpoint{2.165857in}{1.545445in}}%
\pgfpathlineto{\pgfqpoint{2.106932in}{1.486520in}}%
\pgfpathclose%
\pgfusepath{stroke,fill}%
\end{pgfscope}%
\begin{pgfscope}%
\pgfpathrectangle{\pgfqpoint{0.532717in}{0.370679in}}{\pgfqpoint{2.617283in}{1.479321in}}%
\pgfusepath{clip}%
\pgfsetbuttcap%
\pgfsetroundjoin%
\definecolor{currentfill}{rgb}{0.423529,0.627451,0.862745}%
\pgfsetfillcolor{currentfill}%
\pgfsetfillopacity{0.500000}%
\pgfsetlinewidth{1.003750pt}%
\definecolor{currentstroke}{rgb}{0.423529,0.627451,0.862745}%
\pgfsetstrokecolor{currentstroke}%
\pgfsetstrokeopacity{0.500000}%
\pgfsetdash{}{0pt}%
\pgfpathmoveto{\pgfqpoint{2.505611in}{0.906822in}}%
\pgfpathlineto{\pgfqpoint{2.564537in}{0.965748in}}%
\pgfpathlineto{\pgfqpoint{2.505611in}{1.024673in}}%
\pgfpathlineto{\pgfqpoint{2.446685in}{0.965748in}}%
\pgfpathclose%
\pgfusepath{stroke,fill}%
\end{pgfscope}%
\begin{pgfscope}%
\pgfpathrectangle{\pgfqpoint{0.532717in}{0.370679in}}{\pgfqpoint{2.617283in}{1.479321in}}%
\pgfusepath{clip}%
\pgfsetbuttcap%
\pgfsetroundjoin%
\definecolor{currentfill}{rgb}{0.423529,0.627451,0.862745}%
\pgfsetfillcolor{currentfill}%
\pgfsetfillopacity{0.500000}%
\pgfsetlinewidth{1.003750pt}%
\definecolor{currentstroke}{rgb}{0.423529,0.627451,0.862745}%
\pgfsetstrokecolor{currentstroke}%
\pgfsetstrokeopacity{0.500000}%
\pgfsetdash{}{0pt}%
\pgfpathmoveto{\pgfqpoint{2.135716in}{1.356451in}}%
\pgfpathlineto{\pgfqpoint{2.194642in}{1.415377in}}%
\pgfpathlineto{\pgfqpoint{2.135716in}{1.474302in}}%
\pgfpathlineto{\pgfqpoint{2.076791in}{1.415377in}}%
\pgfpathclose%
\pgfusepath{stroke,fill}%
\end{pgfscope}%
\begin{pgfscope}%
\pgfpathrectangle{\pgfqpoint{0.532717in}{0.370679in}}{\pgfqpoint{2.617283in}{1.479321in}}%
\pgfusepath{clip}%
\pgfsetbuttcap%
\pgfsetroundjoin%
\definecolor{currentfill}{rgb}{0.423529,0.627451,0.862745}%
\pgfsetfillcolor{currentfill}%
\pgfsetfillopacity{0.500000}%
\pgfsetlinewidth{1.003750pt}%
\definecolor{currentstroke}{rgb}{0.423529,0.627451,0.862745}%
\pgfsetstrokecolor{currentstroke}%
\pgfsetstrokeopacity{0.500000}%
\pgfsetdash{}{0pt}%
\pgfpathmoveto{\pgfqpoint{2.827701in}{0.870210in}}%
\pgfpathlineto{\pgfqpoint{2.886627in}{0.929136in}}%
\pgfpathlineto{\pgfqpoint{2.827701in}{0.988061in}}%
\pgfpathlineto{\pgfqpoint{2.768776in}{0.929136in}}%
\pgfpathclose%
\pgfusepath{stroke,fill}%
\end{pgfscope}%
\begin{pgfscope}%
\pgfpathrectangle{\pgfqpoint{0.532717in}{0.370679in}}{\pgfqpoint{2.617283in}{1.479321in}}%
\pgfusepath{clip}%
\pgfsetbuttcap%
\pgfsetroundjoin%
\definecolor{currentfill}{rgb}{0.423529,0.627451,0.862745}%
\pgfsetfillcolor{currentfill}%
\pgfsetfillopacity{0.500000}%
\pgfsetlinewidth{1.003750pt}%
\definecolor{currentstroke}{rgb}{0.423529,0.627451,0.862745}%
\pgfsetstrokecolor{currentstroke}%
\pgfsetstrokeopacity{0.500000}%
\pgfsetdash{}{0pt}%
\pgfpathmoveto{\pgfqpoint{2.192722in}{1.172662in}}%
\pgfpathlineto{\pgfqpoint{2.251648in}{1.231587in}}%
\pgfpathlineto{\pgfqpoint{2.192722in}{1.290513in}}%
\pgfpathlineto{\pgfqpoint{2.133797in}{1.231587in}}%
\pgfpathclose%
\pgfusepath{stroke,fill}%
\end{pgfscope}%
\begin{pgfscope}%
\pgfpathrectangle{\pgfqpoint{0.532717in}{0.370679in}}{\pgfqpoint{2.617283in}{1.479321in}}%
\pgfusepath{clip}%
\pgfsetbuttcap%
\pgfsetroundjoin%
\definecolor{currentfill}{rgb}{0.423529,0.627451,0.862745}%
\pgfsetfillcolor{currentfill}%
\pgfsetfillopacity{0.500000}%
\pgfsetlinewidth{1.003750pt}%
\definecolor{currentstroke}{rgb}{0.423529,0.627451,0.862745}%
\pgfsetstrokecolor{currentstroke}%
\pgfsetstrokeopacity{0.500000}%
\pgfsetdash{}{0pt}%
\pgfpathmoveto{\pgfqpoint{2.459809in}{0.926685in}}%
\pgfpathlineto{\pgfqpoint{2.518734in}{0.985610in}}%
\pgfpathlineto{\pgfqpoint{2.459809in}{1.044536in}}%
\pgfpathlineto{\pgfqpoint{2.400883in}{0.985610in}}%
\pgfpathclose%
\pgfusepath{stroke,fill}%
\end{pgfscope}%
\begin{pgfscope}%
\pgfpathrectangle{\pgfqpoint{0.532717in}{0.370679in}}{\pgfqpoint{2.617283in}{1.479321in}}%
\pgfusepath{clip}%
\pgfsetbuttcap%
\pgfsetroundjoin%
\definecolor{currentfill}{rgb}{0.423529,0.627451,0.862745}%
\pgfsetfillcolor{currentfill}%
\pgfsetfillopacity{0.500000}%
\pgfsetlinewidth{1.003750pt}%
\definecolor{currentstroke}{rgb}{0.423529,0.627451,0.862745}%
\pgfsetstrokecolor{currentstroke}%
\pgfsetstrokeopacity{0.500000}%
\pgfsetdash{}{0pt}%
\pgfpathmoveto{\pgfqpoint{2.561551in}{0.823710in}}%
\pgfpathlineto{\pgfqpoint{2.620477in}{0.882636in}}%
\pgfpathlineto{\pgfqpoint{2.561551in}{0.941562in}}%
\pgfpathlineto{\pgfqpoint{2.502626in}{0.882636in}}%
\pgfpathclose%
\pgfusepath{stroke,fill}%
\end{pgfscope}%
\begin{pgfscope}%
\pgfpathrectangle{\pgfqpoint{0.532717in}{0.370679in}}{\pgfqpoint{2.617283in}{1.479321in}}%
\pgfusepath{clip}%
\pgfsetbuttcap%
\pgfsetroundjoin%
\definecolor{currentfill}{rgb}{0.423529,0.627451,0.862745}%
\pgfsetfillcolor{currentfill}%
\pgfsetfillopacity{0.500000}%
\pgfsetlinewidth{1.003750pt}%
\definecolor{currentstroke}{rgb}{0.423529,0.627451,0.862745}%
\pgfsetstrokecolor{currentstroke}%
\pgfsetstrokeopacity{0.500000}%
\pgfsetdash{}{0pt}%
\pgfpathmoveto{\pgfqpoint{2.153617in}{1.161268in}}%
\pgfpathlineto{\pgfqpoint{2.212542in}{1.220193in}}%
\pgfpathlineto{\pgfqpoint{2.153617in}{1.279119in}}%
\pgfpathlineto{\pgfqpoint{2.094691in}{1.220193in}}%
\pgfpathclose%
\pgfusepath{stroke,fill}%
\end{pgfscope}%
\begin{pgfscope}%
\pgfpathrectangle{\pgfqpoint{0.532717in}{0.370679in}}{\pgfqpoint{2.617283in}{1.479321in}}%
\pgfusepath{clip}%
\pgfsetbuttcap%
\pgfsetroundjoin%
\definecolor{currentfill}{rgb}{0.423529,0.627451,0.862745}%
\pgfsetfillcolor{currentfill}%
\pgfsetfillopacity{0.500000}%
\pgfsetlinewidth{1.003750pt}%
\definecolor{currentstroke}{rgb}{0.423529,0.627451,0.862745}%
\pgfsetstrokecolor{currentstroke}%
\pgfsetstrokeopacity{0.500000}%
\pgfsetdash{}{0pt}%
\pgfpathmoveto{\pgfqpoint{2.163554in}{1.132564in}}%
\pgfpathlineto{\pgfqpoint{2.222479in}{1.191490in}}%
\pgfpathlineto{\pgfqpoint{2.163554in}{1.250416in}}%
\pgfpathlineto{\pgfqpoint{2.104628in}{1.191490in}}%
\pgfpathclose%
\pgfusepath{stroke,fill}%
\end{pgfscope}%
\begin{pgfscope}%
\pgfpathrectangle{\pgfqpoint{0.532717in}{0.370679in}}{\pgfqpoint{2.617283in}{1.479321in}}%
\pgfusepath{clip}%
\pgfsetbuttcap%
\pgfsetroundjoin%
\definecolor{currentfill}{rgb}{0.423529,0.627451,0.862745}%
\pgfsetfillcolor{currentfill}%
\pgfsetfillopacity{0.500000}%
\pgfsetlinewidth{1.003750pt}%
\definecolor{currentstroke}{rgb}{0.423529,0.627451,0.862745}%
\pgfsetstrokecolor{currentstroke}%
\pgfsetstrokeopacity{0.500000}%
\pgfsetdash{}{0pt}%
\pgfpathmoveto{\pgfqpoint{2.413800in}{1.175785in}}%
\pgfpathlineto{\pgfqpoint{2.472726in}{1.234711in}}%
\pgfpathlineto{\pgfqpoint{2.413800in}{1.293636in}}%
\pgfpathlineto{\pgfqpoint{2.354875in}{1.234711in}}%
\pgfpathclose%
\pgfusepath{stroke,fill}%
\end{pgfscope}%
\begin{pgfscope}%
\pgfpathrectangle{\pgfqpoint{0.532717in}{0.370679in}}{\pgfqpoint{2.617283in}{1.479321in}}%
\pgfusepath{clip}%
\pgfsetbuttcap%
\pgfsetroundjoin%
\definecolor{currentfill}{rgb}{0.423529,0.627451,0.862745}%
\pgfsetfillcolor{currentfill}%
\pgfsetfillopacity{0.500000}%
\pgfsetlinewidth{1.003750pt}%
\definecolor{currentstroke}{rgb}{0.423529,0.627451,0.862745}%
\pgfsetstrokecolor{currentstroke}%
\pgfsetstrokeopacity{0.500000}%
\pgfsetdash{}{0pt}%
\pgfpathmoveto{\pgfqpoint{2.493401in}{0.867580in}}%
\pgfpathlineto{\pgfqpoint{2.552327in}{0.926506in}}%
\pgfpathlineto{\pgfqpoint{2.493401in}{0.985431in}}%
\pgfpathlineto{\pgfqpoint{2.434476in}{0.926506in}}%
\pgfpathclose%
\pgfusepath{stroke,fill}%
\end{pgfscope}%
\begin{pgfscope}%
\pgfpathrectangle{\pgfqpoint{0.532717in}{0.370679in}}{\pgfqpoint{2.617283in}{1.479321in}}%
\pgfusepath{clip}%
\pgfsetbuttcap%
\pgfsetroundjoin%
\definecolor{currentfill}{rgb}{0.423529,0.627451,0.862745}%
\pgfsetfillcolor{currentfill}%
\pgfsetfillopacity{0.500000}%
\pgfsetlinewidth{1.003750pt}%
\definecolor{currentstroke}{rgb}{0.423529,0.627451,0.862745}%
\pgfsetstrokecolor{currentstroke}%
\pgfsetstrokeopacity{0.500000}%
\pgfsetdash{}{0pt}%
\pgfpathmoveto{\pgfqpoint{2.629742in}{0.908671in}}%
\pgfpathlineto{\pgfqpoint{2.688668in}{0.967597in}}%
\pgfpathlineto{\pgfqpoint{2.629742in}{1.026522in}}%
\pgfpathlineto{\pgfqpoint{2.570817in}{0.967597in}}%
\pgfpathclose%
\pgfusepath{stroke,fill}%
\end{pgfscope}%
\begin{pgfscope}%
\pgfpathrectangle{\pgfqpoint{0.532717in}{0.370679in}}{\pgfqpoint{2.617283in}{1.479321in}}%
\pgfusepath{clip}%
\pgfsetbuttcap%
\pgfsetroundjoin%
\definecolor{currentfill}{rgb}{0.423529,0.627451,0.862745}%
\pgfsetfillcolor{currentfill}%
\pgfsetfillopacity{0.500000}%
\pgfsetlinewidth{1.003750pt}%
\definecolor{currentstroke}{rgb}{0.423529,0.627451,0.862745}%
\pgfsetstrokecolor{currentstroke}%
\pgfsetstrokeopacity{0.500000}%
\pgfsetdash{}{0pt}%
\pgfpathmoveto{\pgfqpoint{2.747435in}{0.451984in}}%
\pgfpathlineto{\pgfqpoint{2.806361in}{0.510910in}}%
\pgfpathlineto{\pgfqpoint{2.747435in}{0.569835in}}%
\pgfpathlineto{\pgfqpoint{2.688509in}{0.510910in}}%
\pgfpathclose%
\pgfusepath{stroke,fill}%
\end{pgfscope}%
\begin{pgfscope}%
\pgfpathrectangle{\pgfqpoint{0.532717in}{0.370679in}}{\pgfqpoint{2.617283in}{1.479321in}}%
\pgfusepath{clip}%
\pgfsetbuttcap%
\pgfsetroundjoin%
\definecolor{currentfill}{rgb}{0.423529,0.627451,0.862745}%
\pgfsetfillcolor{currentfill}%
\pgfsetfillopacity{0.500000}%
\pgfsetlinewidth{1.003750pt}%
\definecolor{currentstroke}{rgb}{0.423529,0.627451,0.862745}%
\pgfsetstrokecolor{currentstroke}%
\pgfsetstrokeopacity{0.500000}%
\pgfsetdash{}{0pt}%
\pgfpathmoveto{\pgfqpoint{2.424591in}{1.179204in}}%
\pgfpathlineto{\pgfqpoint{2.483517in}{1.238130in}}%
\pgfpathlineto{\pgfqpoint{2.424591in}{1.297056in}}%
\pgfpathlineto{\pgfqpoint{2.365665in}{1.238130in}}%
\pgfpathclose%
\pgfusepath{stroke,fill}%
\end{pgfscope}%
\begin{pgfscope}%
\pgfpathrectangle{\pgfqpoint{0.532717in}{0.370679in}}{\pgfqpoint{2.617283in}{1.479321in}}%
\pgfusepath{clip}%
\pgfsetbuttcap%
\pgfsetroundjoin%
\definecolor{currentfill}{rgb}{0.423529,0.627451,0.862745}%
\pgfsetfillcolor{currentfill}%
\pgfsetfillopacity{0.500000}%
\pgfsetlinewidth{1.003750pt}%
\definecolor{currentstroke}{rgb}{0.423529,0.627451,0.862745}%
\pgfsetstrokecolor{currentstroke}%
\pgfsetstrokeopacity{0.500000}%
\pgfsetdash{}{0pt}%
\pgfpathmoveto{\pgfqpoint{2.209392in}{1.145095in}}%
\pgfpathlineto{\pgfqpoint{2.268317in}{1.204020in}}%
\pgfpathlineto{\pgfqpoint{2.209392in}{1.262946in}}%
\pgfpathlineto{\pgfqpoint{2.150466in}{1.204020in}}%
\pgfpathclose%
\pgfusepath{stroke,fill}%
\end{pgfscope}%
\begin{pgfscope}%
\pgfpathrectangle{\pgfqpoint{0.532717in}{0.370679in}}{\pgfqpoint{2.617283in}{1.479321in}}%
\pgfusepath{clip}%
\pgfsetbuttcap%
\pgfsetroundjoin%
\definecolor{currentfill}{rgb}{0.423529,0.627451,0.862745}%
\pgfsetfillcolor{currentfill}%
\pgfsetfillopacity{0.500000}%
\pgfsetlinewidth{1.003750pt}%
\definecolor{currentstroke}{rgb}{0.423529,0.627451,0.862745}%
\pgfsetstrokecolor{currentstroke}%
\pgfsetstrokeopacity{0.500000}%
\pgfsetdash{}{0pt}%
\pgfpathmoveto{\pgfqpoint{2.310592in}{1.308798in}}%
\pgfpathlineto{\pgfqpoint{2.369518in}{1.367724in}}%
\pgfpathlineto{\pgfqpoint{2.310592in}{1.426649in}}%
\pgfpathlineto{\pgfqpoint{2.251667in}{1.367724in}}%
\pgfpathclose%
\pgfusepath{stroke,fill}%
\end{pgfscope}%
\begin{pgfscope}%
\pgfpathrectangle{\pgfqpoint{0.532717in}{0.370679in}}{\pgfqpoint{2.617283in}{1.479321in}}%
\pgfusepath{clip}%
\pgfsetbuttcap%
\pgfsetroundjoin%
\definecolor{currentfill}{rgb}{0.423529,0.627451,0.862745}%
\pgfsetfillcolor{currentfill}%
\pgfsetfillopacity{0.500000}%
\pgfsetlinewidth{1.003750pt}%
\definecolor{currentstroke}{rgb}{0.423529,0.627451,0.862745}%
\pgfsetstrokecolor{currentstroke}%
\pgfsetstrokeopacity{0.500000}%
\pgfsetdash{}{0pt}%
\pgfpathmoveto{\pgfqpoint{2.700578in}{0.765400in}}%
\pgfpathlineto{\pgfqpoint{2.759504in}{0.824326in}}%
\pgfpathlineto{\pgfqpoint{2.700578in}{0.883251in}}%
\pgfpathlineto{\pgfqpoint{2.641652in}{0.824326in}}%
\pgfpathclose%
\pgfusepath{stroke,fill}%
\end{pgfscope}%
\begin{pgfscope}%
\pgfpathrectangle{\pgfqpoint{0.532717in}{0.370679in}}{\pgfqpoint{2.617283in}{1.479321in}}%
\pgfusepath{clip}%
\pgfsetbuttcap%
\pgfsetroundjoin%
\definecolor{currentfill}{rgb}{0.423529,0.627451,0.862745}%
\pgfsetfillcolor{currentfill}%
\pgfsetfillopacity{0.500000}%
\pgfsetlinewidth{1.003750pt}%
\definecolor{currentstroke}{rgb}{0.423529,0.627451,0.862745}%
\pgfsetstrokecolor{currentstroke}%
\pgfsetstrokeopacity{0.500000}%
\pgfsetdash{}{0pt}%
\pgfpathmoveto{\pgfqpoint{2.420456in}{1.015698in}}%
\pgfpathlineto{\pgfqpoint{2.479381in}{1.074624in}}%
\pgfpathlineto{\pgfqpoint{2.420456in}{1.133550in}}%
\pgfpathlineto{\pgfqpoint{2.361530in}{1.074624in}}%
\pgfpathclose%
\pgfusepath{stroke,fill}%
\end{pgfscope}%
\begin{pgfscope}%
\pgfpathrectangle{\pgfqpoint{0.532717in}{0.370679in}}{\pgfqpoint{2.617283in}{1.479321in}}%
\pgfusepath{clip}%
\pgfsetbuttcap%
\pgfsetroundjoin%
\definecolor{currentfill}{rgb}{0.423529,0.627451,0.862745}%
\pgfsetfillcolor{currentfill}%
\pgfsetfillopacity{0.500000}%
\pgfsetlinewidth{1.003750pt}%
\definecolor{currentstroke}{rgb}{0.423529,0.627451,0.862745}%
\pgfsetstrokecolor{currentstroke}%
\pgfsetstrokeopacity{0.500000}%
\pgfsetdash{}{0pt}%
\pgfpathmoveto{\pgfqpoint{2.348248in}{1.057285in}}%
\pgfpathlineto{\pgfqpoint{2.407174in}{1.116210in}}%
\pgfpathlineto{\pgfqpoint{2.348248in}{1.175136in}}%
\pgfpathlineto{\pgfqpoint{2.289322in}{1.116210in}}%
\pgfpathclose%
\pgfusepath{stroke,fill}%
\end{pgfscope}%
\begin{pgfscope}%
\pgfpathrectangle{\pgfqpoint{0.532717in}{0.370679in}}{\pgfqpoint{2.617283in}{1.479321in}}%
\pgfusepath{clip}%
\pgfsetbuttcap%
\pgfsetroundjoin%
\definecolor{currentfill}{rgb}{0.423529,0.627451,0.862745}%
\pgfsetfillcolor{currentfill}%
\pgfsetfillopacity{0.500000}%
\pgfsetlinewidth{1.003750pt}%
\definecolor{currentstroke}{rgb}{0.423529,0.627451,0.862745}%
\pgfsetstrokecolor{currentstroke}%
\pgfsetstrokeopacity{0.500000}%
\pgfsetdash{}{0pt}%
\pgfpathmoveto{\pgfqpoint{2.126922in}{1.154570in}}%
\pgfpathlineto{\pgfqpoint{2.185848in}{1.213495in}}%
\pgfpathlineto{\pgfqpoint{2.126922in}{1.272421in}}%
\pgfpathlineto{\pgfqpoint{2.067997in}{1.213495in}}%
\pgfpathclose%
\pgfusepath{stroke,fill}%
\end{pgfscope}%
\begin{pgfscope}%
\pgfpathrectangle{\pgfqpoint{0.532717in}{0.370679in}}{\pgfqpoint{2.617283in}{1.479321in}}%
\pgfusepath{clip}%
\pgfsetbuttcap%
\pgfsetroundjoin%
\definecolor{currentfill}{rgb}{0.423529,0.627451,0.862745}%
\pgfsetfillcolor{currentfill}%
\pgfsetfillopacity{0.500000}%
\pgfsetlinewidth{1.003750pt}%
\definecolor{currentstroke}{rgb}{0.423529,0.627451,0.862745}%
\pgfsetstrokecolor{currentstroke}%
\pgfsetstrokeopacity{0.500000}%
\pgfsetdash{}{0pt}%
\pgfpathmoveto{\pgfqpoint{2.409229in}{0.909248in}}%
\pgfpathlineto{\pgfqpoint{2.468155in}{0.968174in}}%
\pgfpathlineto{\pgfqpoint{2.409229in}{1.027099in}}%
\pgfpathlineto{\pgfqpoint{2.350304in}{0.968174in}}%
\pgfpathclose%
\pgfusepath{stroke,fill}%
\end{pgfscope}%
\begin{pgfscope}%
\pgfpathrectangle{\pgfqpoint{0.532717in}{0.370679in}}{\pgfqpoint{2.617283in}{1.479321in}}%
\pgfusepath{clip}%
\pgfsetbuttcap%
\pgfsetroundjoin%
\definecolor{currentfill}{rgb}{0.423529,0.627451,0.862745}%
\pgfsetfillcolor{currentfill}%
\pgfsetfillopacity{0.500000}%
\pgfsetlinewidth{1.003750pt}%
\definecolor{currentstroke}{rgb}{0.423529,0.627451,0.862745}%
\pgfsetstrokecolor{currentstroke}%
\pgfsetstrokeopacity{0.500000}%
\pgfsetdash{}{0pt}%
\pgfpathmoveto{\pgfqpoint{2.471371in}{0.995696in}}%
\pgfpathlineto{\pgfqpoint{2.530297in}{1.054621in}}%
\pgfpathlineto{\pgfqpoint{2.471371in}{1.113547in}}%
\pgfpathlineto{\pgfqpoint{2.412445in}{1.054621in}}%
\pgfpathclose%
\pgfusepath{stroke,fill}%
\end{pgfscope}%
\begin{pgfscope}%
\pgfpathrectangle{\pgfqpoint{0.532717in}{0.370679in}}{\pgfqpoint{2.617283in}{1.479321in}}%
\pgfusepath{clip}%
\pgfsetbuttcap%
\pgfsetroundjoin%
\definecolor{currentfill}{rgb}{0.423529,0.627451,0.862745}%
\pgfsetfillcolor{currentfill}%
\pgfsetfillopacity{0.500000}%
\pgfsetlinewidth{1.003750pt}%
\definecolor{currentstroke}{rgb}{0.423529,0.627451,0.862745}%
\pgfsetstrokecolor{currentstroke}%
\pgfsetstrokeopacity{0.500000}%
\pgfsetdash{}{0pt}%
\pgfpathmoveto{\pgfqpoint{2.353537in}{0.891952in}}%
\pgfpathlineto{\pgfqpoint{2.412462in}{0.950877in}}%
\pgfpathlineto{\pgfqpoint{2.353537in}{1.009803in}}%
\pgfpathlineto{\pgfqpoint{2.294611in}{0.950877in}}%
\pgfpathclose%
\pgfusepath{stroke,fill}%
\end{pgfscope}%
\begin{pgfscope}%
\pgfpathrectangle{\pgfqpoint{0.532717in}{0.370679in}}{\pgfqpoint{2.617283in}{1.479321in}}%
\pgfusepath{clip}%
\pgfsetbuttcap%
\pgfsetroundjoin%
\definecolor{currentfill}{rgb}{0.423529,0.627451,0.862745}%
\pgfsetfillcolor{currentfill}%
\pgfsetfillopacity{0.500000}%
\pgfsetlinewidth{1.003750pt}%
\definecolor{currentstroke}{rgb}{0.423529,0.627451,0.862745}%
\pgfsetstrokecolor{currentstroke}%
\pgfsetstrokeopacity{0.500000}%
\pgfsetdash{}{0pt}%
\pgfpathmoveto{\pgfqpoint{2.200662in}{1.342421in}}%
\pgfpathlineto{\pgfqpoint{2.259588in}{1.401346in}}%
\pgfpathlineto{\pgfqpoint{2.200662in}{1.460272in}}%
\pgfpathlineto{\pgfqpoint{2.141737in}{1.401346in}}%
\pgfpathclose%
\pgfusepath{stroke,fill}%
\end{pgfscope}%
\begin{pgfscope}%
\pgfpathrectangle{\pgfqpoint{0.532717in}{0.370679in}}{\pgfqpoint{2.617283in}{1.479321in}}%
\pgfusepath{clip}%
\pgfsetbuttcap%
\pgfsetroundjoin%
\definecolor{currentfill}{rgb}{0.423529,0.627451,0.862745}%
\pgfsetfillcolor{currentfill}%
\pgfsetfillopacity{0.500000}%
\pgfsetlinewidth{1.003750pt}%
\definecolor{currentstroke}{rgb}{0.423529,0.627451,0.862745}%
\pgfsetstrokecolor{currentstroke}%
\pgfsetstrokeopacity{0.500000}%
\pgfsetdash{}{0pt}%
\pgfpathmoveto{\pgfqpoint{2.546254in}{0.952962in}}%
\pgfpathlineto{\pgfqpoint{2.605179in}{1.011888in}}%
\pgfpathlineto{\pgfqpoint{2.546254in}{1.070813in}}%
\pgfpathlineto{\pgfqpoint{2.487328in}{1.011888in}}%
\pgfpathclose%
\pgfusepath{stroke,fill}%
\end{pgfscope}%
\begin{pgfscope}%
\pgfpathrectangle{\pgfqpoint{0.532717in}{0.370679in}}{\pgfqpoint{2.617283in}{1.479321in}}%
\pgfusepath{clip}%
\pgfsetbuttcap%
\pgfsetroundjoin%
\definecolor{currentfill}{rgb}{0.423529,0.627451,0.862745}%
\pgfsetfillcolor{currentfill}%
\pgfsetfillopacity{0.500000}%
\pgfsetlinewidth{1.003750pt}%
\definecolor{currentstroke}{rgb}{0.423529,0.627451,0.862745}%
\pgfsetstrokecolor{currentstroke}%
\pgfsetstrokeopacity{0.500000}%
\pgfsetdash{}{0pt}%
\pgfpathmoveto{\pgfqpoint{2.502972in}{0.940362in}}%
\pgfpathlineto{\pgfqpoint{2.561898in}{0.999287in}}%
\pgfpathlineto{\pgfqpoint{2.502972in}{1.058213in}}%
\pgfpathlineto{\pgfqpoint{2.444046in}{0.999287in}}%
\pgfpathclose%
\pgfusepath{stroke,fill}%
\end{pgfscope}%
\begin{pgfscope}%
\pgfpathrectangle{\pgfqpoint{0.532717in}{0.370679in}}{\pgfqpoint{2.617283in}{1.479321in}}%
\pgfusepath{clip}%
\pgfsetbuttcap%
\pgfsetroundjoin%
\definecolor{currentfill}{rgb}{0.423529,0.627451,0.862745}%
\pgfsetfillcolor{currentfill}%
\pgfsetfillopacity{0.500000}%
\pgfsetlinewidth{1.003750pt}%
\definecolor{currentstroke}{rgb}{0.423529,0.627451,0.862745}%
\pgfsetstrokecolor{currentstroke}%
\pgfsetstrokeopacity{0.500000}%
\pgfsetdash{}{0pt}%
\pgfpathmoveto{\pgfqpoint{2.360022in}{0.993496in}}%
\pgfpathlineto{\pgfqpoint{2.418948in}{1.052421in}}%
\pgfpathlineto{\pgfqpoint{2.360022in}{1.111347in}}%
\pgfpathlineto{\pgfqpoint{2.301096in}{1.052421in}}%
\pgfpathclose%
\pgfusepath{stroke,fill}%
\end{pgfscope}%
\begin{pgfscope}%
\pgfpathrectangle{\pgfqpoint{0.532717in}{0.370679in}}{\pgfqpoint{2.617283in}{1.479321in}}%
\pgfusepath{clip}%
\pgfsetbuttcap%
\pgfsetroundjoin%
\definecolor{currentfill}{rgb}{0.423529,0.627451,0.862745}%
\pgfsetfillcolor{currentfill}%
\pgfsetfillopacity{0.500000}%
\pgfsetlinewidth{1.003750pt}%
\definecolor{currentstroke}{rgb}{0.423529,0.627451,0.862745}%
\pgfsetstrokecolor{currentstroke}%
\pgfsetstrokeopacity{0.500000}%
\pgfsetdash{}{0pt}%
\pgfpathmoveto{\pgfqpoint{2.234219in}{1.250924in}}%
\pgfpathlineto{\pgfqpoint{2.293144in}{1.309850in}}%
\pgfpathlineto{\pgfqpoint{2.234219in}{1.368775in}}%
\pgfpathlineto{\pgfqpoint{2.175293in}{1.309850in}}%
\pgfpathclose%
\pgfusepath{stroke,fill}%
\end{pgfscope}%
\begin{pgfscope}%
\pgfpathrectangle{\pgfqpoint{0.532717in}{0.370679in}}{\pgfqpoint{2.617283in}{1.479321in}}%
\pgfusepath{clip}%
\pgfsetbuttcap%
\pgfsetroundjoin%
\definecolor{currentfill}{rgb}{0.423529,0.627451,0.862745}%
\pgfsetfillcolor{currentfill}%
\pgfsetfillopacity{0.500000}%
\pgfsetlinewidth{1.003750pt}%
\definecolor{currentstroke}{rgb}{0.423529,0.627451,0.862745}%
\pgfsetstrokecolor{currentstroke}%
\pgfsetstrokeopacity{0.500000}%
\pgfsetdash{}{0pt}%
\pgfpathmoveto{\pgfqpoint{2.305903in}{1.043338in}}%
\pgfpathlineto{\pgfqpoint{2.364828in}{1.102263in}}%
\pgfpathlineto{\pgfqpoint{2.305903in}{1.161189in}}%
\pgfpathlineto{\pgfqpoint{2.246977in}{1.102263in}}%
\pgfpathclose%
\pgfusepath{stroke,fill}%
\end{pgfscope}%
\begin{pgfscope}%
\pgfpathrectangle{\pgfqpoint{0.532717in}{0.370679in}}{\pgfqpoint{2.617283in}{1.479321in}}%
\pgfusepath{clip}%
\pgfsetbuttcap%
\pgfsetroundjoin%
\definecolor{currentfill}{rgb}{0.423529,0.627451,0.862745}%
\pgfsetfillcolor{currentfill}%
\pgfsetfillopacity{0.500000}%
\pgfsetlinewidth{1.003750pt}%
\definecolor{currentstroke}{rgb}{0.423529,0.627451,0.862745}%
\pgfsetstrokecolor{currentstroke}%
\pgfsetstrokeopacity{0.500000}%
\pgfsetdash{}{0pt}%
\pgfpathmoveto{\pgfqpoint{2.347276in}{1.026239in}}%
\pgfpathlineto{\pgfqpoint{2.406201in}{1.085165in}}%
\pgfpathlineto{\pgfqpoint{2.347276in}{1.144090in}}%
\pgfpathlineto{\pgfqpoint{2.288350in}{1.085165in}}%
\pgfpathclose%
\pgfusepath{stroke,fill}%
\end{pgfscope}%
\begin{pgfscope}%
\pgfpathrectangle{\pgfqpoint{0.532717in}{0.370679in}}{\pgfqpoint{2.617283in}{1.479321in}}%
\pgfusepath{clip}%
\pgfsetbuttcap%
\pgfsetroundjoin%
\definecolor{currentfill}{rgb}{0.423529,0.627451,0.862745}%
\pgfsetfillcolor{currentfill}%
\pgfsetfillopacity{0.500000}%
\pgfsetlinewidth{1.003750pt}%
\definecolor{currentstroke}{rgb}{0.423529,0.627451,0.862745}%
\pgfsetstrokecolor{currentstroke}%
\pgfsetstrokeopacity{0.500000}%
\pgfsetdash{}{0pt}%
\pgfpathmoveto{\pgfqpoint{2.193323in}{1.208754in}}%
\pgfpathlineto{\pgfqpoint{2.252248in}{1.267680in}}%
\pgfpathlineto{\pgfqpoint{2.193323in}{1.326605in}}%
\pgfpathlineto{\pgfqpoint{2.134397in}{1.267680in}}%
\pgfpathclose%
\pgfusepath{stroke,fill}%
\end{pgfscope}%
\begin{pgfscope}%
\pgfsetbuttcap%
\pgfsetroundjoin%
\definecolor{currentfill}{rgb}{0.000000,0.000000,0.000000}%
\pgfsetfillcolor{currentfill}%
\pgfsetlinewidth{0.803000pt}%
\definecolor{currentstroke}{rgb}{0.000000,0.000000,0.000000}%
\pgfsetstrokecolor{currentstroke}%
\pgfsetdash{}{0pt}%
\pgfsys@defobject{currentmarker}{\pgfqpoint{0.000000in}{-0.048611in}}{\pgfqpoint{0.000000in}{0.000000in}}{%
\pgfpathmoveto{\pgfqpoint{0.000000in}{0.000000in}}%
\pgfpathlineto{\pgfqpoint{0.000000in}{-0.048611in}}%
\pgfusepath{stroke,fill}%
}%
\begin{pgfscope}%
\pgfsys@transformshift{1.020957in}{0.370679in}%
\pgfsys@useobject{currentmarker}{}%
\end{pgfscope}%
\end{pgfscope}%
\begin{pgfscope}%
\definecolor{textcolor}{rgb}{0.000000,0.000000,0.000000}%
\pgfsetstrokecolor{textcolor}%
\pgfsetfillcolor{textcolor}%
\pgftext[x=1.020957in,y=0.273457in,,top]{\color{textcolor}\fontsize{10.000000}{12.000000}\selectfont \(\displaystyle -0.1\)}%
\end{pgfscope}%
\begin{pgfscope}%
\pgfsetbuttcap%
\pgfsetroundjoin%
\definecolor{currentfill}{rgb}{0.000000,0.000000,0.000000}%
\pgfsetfillcolor{currentfill}%
\pgfsetlinewidth{0.803000pt}%
\definecolor{currentstroke}{rgb}{0.000000,0.000000,0.000000}%
\pgfsetstrokecolor{currentstroke}%
\pgfsetdash{}{0pt}%
\pgfsys@defobject{currentmarker}{\pgfqpoint{0.000000in}{-0.048611in}}{\pgfqpoint{0.000000in}{0.000000in}}{%
\pgfpathmoveto{\pgfqpoint{0.000000in}{0.000000in}}%
\pgfpathlineto{\pgfqpoint{0.000000in}{-0.048611in}}%
\pgfusepath{stroke,fill}%
}%
\begin{pgfscope}%
\pgfsys@transformshift{1.775436in}{0.370679in}%
\pgfsys@useobject{currentmarker}{}%
\end{pgfscope}%
\end{pgfscope}%
\begin{pgfscope}%
\definecolor{textcolor}{rgb}{0.000000,0.000000,0.000000}%
\pgfsetstrokecolor{textcolor}%
\pgfsetfillcolor{textcolor}%
\pgftext[x=1.775436in,y=0.273457in,,top]{\color{textcolor}\fontsize{10.000000}{12.000000}\selectfont \(\displaystyle 0.0\)}%
\end{pgfscope}%
\begin{pgfscope}%
\pgfsetbuttcap%
\pgfsetroundjoin%
\definecolor{currentfill}{rgb}{0.000000,0.000000,0.000000}%
\pgfsetfillcolor{currentfill}%
\pgfsetlinewidth{0.803000pt}%
\definecolor{currentstroke}{rgb}{0.000000,0.000000,0.000000}%
\pgfsetstrokecolor{currentstroke}%
\pgfsetdash{}{0pt}%
\pgfsys@defobject{currentmarker}{\pgfqpoint{0.000000in}{-0.048611in}}{\pgfqpoint{0.000000in}{0.000000in}}{%
\pgfpathmoveto{\pgfqpoint{0.000000in}{0.000000in}}%
\pgfpathlineto{\pgfqpoint{0.000000in}{-0.048611in}}%
\pgfusepath{stroke,fill}%
}%
\begin{pgfscope}%
\pgfsys@transformshift{2.529914in}{0.370679in}%
\pgfsys@useobject{currentmarker}{}%
\end{pgfscope}%
\end{pgfscope}%
\begin{pgfscope}%
\definecolor{textcolor}{rgb}{0.000000,0.000000,0.000000}%
\pgfsetstrokecolor{textcolor}%
\pgfsetfillcolor{textcolor}%
\pgftext[x=2.529914in,y=0.273457in,,top]{\color{textcolor}\fontsize{10.000000}{12.000000}\selectfont \(\displaystyle 0.1\)}%
\end{pgfscope}%
\begin{pgfscope}%
\pgfsetbuttcap%
\pgfsetroundjoin%
\definecolor{currentfill}{rgb}{0.000000,0.000000,0.000000}%
\pgfsetfillcolor{currentfill}%
\pgfsetlinewidth{0.803000pt}%
\definecolor{currentstroke}{rgb}{0.000000,0.000000,0.000000}%
\pgfsetstrokecolor{currentstroke}%
\pgfsetdash{}{0pt}%
\pgfsys@defobject{currentmarker}{\pgfqpoint{-0.048611in}{0.000000in}}{\pgfqpoint{0.000000in}{0.000000in}}{%
\pgfpathmoveto{\pgfqpoint{0.000000in}{0.000000in}}%
\pgfpathlineto{\pgfqpoint{-0.048611in}{0.000000in}}%
\pgfusepath{stroke,fill}%
}%
\begin{pgfscope}%
\pgfsys@transformshift{0.532717in}{0.588240in}%
\pgfsys@useobject{currentmarker}{}%
\end{pgfscope}%
\end{pgfscope}%
\begin{pgfscope}%
\definecolor{textcolor}{rgb}{0.000000,0.000000,0.000000}%
\pgfsetstrokecolor{textcolor}%
\pgfsetfillcolor{textcolor}%
\pgftext[x=0.150000in,y=0.540015in,left,base]{\color{textcolor}\fontsize{10.000000}{12.000000}\selectfont \(\displaystyle -0.2\)}%
\end{pgfscope}%
\begin{pgfscope}%
\pgfsetbuttcap%
\pgfsetroundjoin%
\definecolor{currentfill}{rgb}{0.000000,0.000000,0.000000}%
\pgfsetfillcolor{currentfill}%
\pgfsetlinewidth{0.803000pt}%
\definecolor{currentstroke}{rgb}{0.000000,0.000000,0.000000}%
\pgfsetstrokecolor{currentstroke}%
\pgfsetdash{}{0pt}%
\pgfsys@defobject{currentmarker}{\pgfqpoint{-0.048611in}{0.000000in}}{\pgfqpoint{0.000000in}{0.000000in}}{%
\pgfpathmoveto{\pgfqpoint{0.000000in}{0.000000in}}%
\pgfpathlineto{\pgfqpoint{-0.048611in}{0.000000in}}%
\pgfusepath{stroke,fill}%
}%
\begin{pgfscope}%
\pgfsys@transformshift{0.532717in}{1.138202in}%
\pgfsys@useobject{currentmarker}{}%
\end{pgfscope}%
\end{pgfscope}%
\begin{pgfscope}%
\definecolor{textcolor}{rgb}{0.000000,0.000000,0.000000}%
\pgfsetstrokecolor{textcolor}%
\pgfsetfillcolor{textcolor}%
\pgftext[x=0.258025in,y=1.089977in,left,base]{\color{textcolor}\fontsize{10.000000}{12.000000}\selectfont \(\displaystyle 0.0\)}%
\end{pgfscope}%
\begin{pgfscope}%
\pgfsetbuttcap%
\pgfsetroundjoin%
\definecolor{currentfill}{rgb}{0.000000,0.000000,0.000000}%
\pgfsetfillcolor{currentfill}%
\pgfsetlinewidth{0.803000pt}%
\definecolor{currentstroke}{rgb}{0.000000,0.000000,0.000000}%
\pgfsetstrokecolor{currentstroke}%
\pgfsetdash{}{0pt}%
\pgfsys@defobject{currentmarker}{\pgfqpoint{-0.048611in}{0.000000in}}{\pgfqpoint{0.000000in}{0.000000in}}{%
\pgfpathmoveto{\pgfqpoint{0.000000in}{0.000000in}}%
\pgfpathlineto{\pgfqpoint{-0.048611in}{0.000000in}}%
\pgfusepath{stroke,fill}%
}%
\begin{pgfscope}%
\pgfsys@transformshift{0.532717in}{1.688164in}%
\pgfsys@useobject{currentmarker}{}%
\end{pgfscope}%
\end{pgfscope}%
\begin{pgfscope}%
\definecolor{textcolor}{rgb}{0.000000,0.000000,0.000000}%
\pgfsetstrokecolor{textcolor}%
\pgfsetfillcolor{textcolor}%
\pgftext[x=0.258025in,y=1.639938in,left,base]{\color{textcolor}\fontsize{10.000000}{12.000000}\selectfont \(\displaystyle 0.2\)}%
\end{pgfscope}%
\begin{pgfscope}%
\pgfsetrectcap%
\pgfsetmiterjoin%
\pgfsetlinewidth{0.803000pt}%
\definecolor{currentstroke}{rgb}{0.000000,0.000000,0.000000}%
\pgfsetstrokecolor{currentstroke}%
\pgfsetdash{}{0pt}%
\pgfpathmoveto{\pgfqpoint{0.532717in}{0.370679in}}%
\pgfpathlineto{\pgfqpoint{0.532717in}{1.850000in}}%
\pgfusepath{stroke}%
\end{pgfscope}%
\begin{pgfscope}%
\pgfsetrectcap%
\pgfsetmiterjoin%
\pgfsetlinewidth{0.803000pt}%
\definecolor{currentstroke}{rgb}{0.000000,0.000000,0.000000}%
\pgfsetstrokecolor{currentstroke}%
\pgfsetdash{}{0pt}%
\pgfpathmoveto{\pgfqpoint{0.532717in}{0.370679in}}%
\pgfpathlineto{\pgfqpoint{3.150000in}{0.370679in}}%
\pgfusepath{stroke}%
\end{pgfscope}%
\end{pgfpicture}%
\makeatother%
\endgroup%
}
    \hspace{1.5mm}
    \subfloat[GD]{%% Creator: Matplotlib, PGF backend
%%
%% To include the figure in your LaTeX document, write
%%   \input{<filename>.pgf}
%%
%% Make sure the required packages are loaded in your preamble
%%   \usepackage{pgf}
%%
%% Figures using additional raster images can only be included by \input if
%% they are in the same directory as the main LaTeX file. For loading figures
%% from other directories you can use the `import` package
%%   \usepackage{import}
%% and then include the figures with
%%   \import{<path to file>}{<filename>.pgf}
%%
%% Matplotlib used the following preamble
%%
\begingroup%
\makeatletter%
\begin{pgfpicture}%
\pgfpathrectangle{\pgfpointorigin}{\pgfqpoint{3.300000in}{2.000000in}}%
\pgfusepath{use as bounding box, clip}%
\begin{pgfscope}%
\pgfsetbuttcap%
\pgfsetmiterjoin%
\definecolor{currentfill}{rgb}{1.000000,1.000000,1.000000}%
\pgfsetfillcolor{currentfill}%
\pgfsetlinewidth{0.000000pt}%
\definecolor{currentstroke}{rgb}{1.000000,1.000000,1.000000}%
\pgfsetstrokecolor{currentstroke}%
\pgfsetdash{}{0pt}%
\pgfpathmoveto{\pgfqpoint{0.000000in}{0.000000in}}%
\pgfpathlineto{\pgfqpoint{3.300000in}{0.000000in}}%
\pgfpathlineto{\pgfqpoint{3.300000in}{2.000000in}}%
\pgfpathlineto{\pgfqpoint{0.000000in}{2.000000in}}%
\pgfpathclose%
\pgfusepath{fill}%
\end{pgfscope}%
\begin{pgfscope}%
\pgfsetbuttcap%
\pgfsetmiterjoin%
\definecolor{currentfill}{rgb}{1.000000,1.000000,1.000000}%
\pgfsetfillcolor{currentfill}%
\pgfsetlinewidth{0.000000pt}%
\definecolor{currentstroke}{rgb}{0.000000,0.000000,0.000000}%
\pgfsetstrokecolor{currentstroke}%
\pgfsetstrokeopacity{0.000000}%
\pgfsetdash{}{0pt}%
\pgfpathmoveto{\pgfqpoint{0.424692in}{0.370679in}}%
\pgfpathlineto{\pgfqpoint{3.150000in}{0.370679in}}%
\pgfpathlineto{\pgfqpoint{3.150000in}{1.850000in}}%
\pgfpathlineto{\pgfqpoint{0.424692in}{1.850000in}}%
\pgfpathclose%
\pgfusepath{fill}%
\end{pgfscope}%
\begin{pgfscope}%
\pgfpathrectangle{\pgfqpoint{0.424692in}{0.370679in}}{\pgfqpoint{2.725308in}{1.479321in}}%
\pgfusepath{clip}%
\pgfsetbuttcap%
\pgfsetroundjoin%
\definecolor{currentfill}{rgb}{1.000000,0.411765,0.380392}%
\pgfsetfillcolor{currentfill}%
\pgfsetfillopacity{0.500000}%
\pgfsetlinewidth{1.003750pt}%
\definecolor{currentstroke}{rgb}{1.000000,0.411765,0.380392}%
\pgfsetstrokecolor{currentstroke}%
\pgfsetstrokeopacity{0.500000}%
\pgfsetdash{}{0pt}%
\pgfpathmoveto{\pgfqpoint{1.961085in}{0.834517in}}%
\pgfpathcurveto{\pgfqpoint{1.972135in}{0.834517in}}{\pgfqpoint{1.982734in}{0.838907in}}{\pgfqpoint{1.990548in}{0.846721in}}%
\pgfpathcurveto{\pgfqpoint{1.998362in}{0.854535in}}{\pgfqpoint{2.002752in}{0.865134in}}{\pgfqpoint{2.002752in}{0.876184in}}%
\pgfpathcurveto{\pgfqpoint{2.002752in}{0.887234in}}{\pgfqpoint{1.998362in}{0.897833in}}{\pgfqpoint{1.990548in}{0.905647in}}%
\pgfpathcurveto{\pgfqpoint{1.982734in}{0.913460in}}{\pgfqpoint{1.972135in}{0.917851in}}{\pgfqpoint{1.961085in}{0.917851in}}%
\pgfpathcurveto{\pgfqpoint{1.950035in}{0.917851in}}{\pgfqpoint{1.939436in}{0.913460in}}{\pgfqpoint{1.931622in}{0.905647in}}%
\pgfpathcurveto{\pgfqpoint{1.923809in}{0.897833in}}{\pgfqpoint{1.919419in}{0.887234in}}{\pgfqpoint{1.919419in}{0.876184in}}%
\pgfpathcurveto{\pgfqpoint{1.919419in}{0.865134in}}{\pgfqpoint{1.923809in}{0.854535in}}{\pgfqpoint{1.931622in}{0.846721in}}%
\pgfpathcurveto{\pgfqpoint{1.939436in}{0.838907in}}{\pgfqpoint{1.950035in}{0.834517in}}{\pgfqpoint{1.961085in}{0.834517in}}%
\pgfpathclose%
\pgfusepath{stroke,fill}%
\end{pgfscope}%
\begin{pgfscope}%
\pgfpathrectangle{\pgfqpoint{0.424692in}{0.370679in}}{\pgfqpoint{2.725308in}{1.479321in}}%
\pgfusepath{clip}%
\pgfsetbuttcap%
\pgfsetroundjoin%
\definecolor{currentfill}{rgb}{1.000000,0.411765,0.380392}%
\pgfsetfillcolor{currentfill}%
\pgfsetfillopacity{0.500000}%
\pgfsetlinewidth{1.003750pt}%
\definecolor{currentstroke}{rgb}{1.000000,0.411765,0.380392}%
\pgfsetstrokecolor{currentstroke}%
\pgfsetstrokeopacity{0.500000}%
\pgfsetdash{}{0pt}%
\pgfpathmoveto{\pgfqpoint{2.320823in}{1.026601in}}%
\pgfpathcurveto{\pgfqpoint{2.331873in}{1.026601in}}{\pgfqpoint{2.342472in}{1.030992in}}{\pgfqpoint{2.350286in}{1.038805in}}%
\pgfpathcurveto{\pgfqpoint{2.358099in}{1.046619in}}{\pgfqpoint{2.362489in}{1.057218in}}{\pgfqpoint{2.362489in}{1.068268in}}%
\pgfpathcurveto{\pgfqpoint{2.362489in}{1.079318in}}{\pgfqpoint{2.358099in}{1.089917in}}{\pgfqpoint{2.350286in}{1.097731in}}%
\pgfpathcurveto{\pgfqpoint{2.342472in}{1.105545in}}{\pgfqpoint{2.331873in}{1.109935in}}{\pgfqpoint{2.320823in}{1.109935in}}%
\pgfpathcurveto{\pgfqpoint{2.309773in}{1.109935in}}{\pgfqpoint{2.299174in}{1.105545in}}{\pgfqpoint{2.291360in}{1.097731in}}%
\pgfpathcurveto{\pgfqpoint{2.283546in}{1.089917in}}{\pgfqpoint{2.279156in}{1.079318in}}{\pgfqpoint{2.279156in}{1.068268in}}%
\pgfpathcurveto{\pgfqpoint{2.279156in}{1.057218in}}{\pgfqpoint{2.283546in}{1.046619in}}{\pgfqpoint{2.291360in}{1.038805in}}%
\pgfpathcurveto{\pgfqpoint{2.299174in}{1.030992in}}{\pgfqpoint{2.309773in}{1.026601in}}{\pgfqpoint{2.320823in}{1.026601in}}%
\pgfpathclose%
\pgfusepath{stroke,fill}%
\end{pgfscope}%
\begin{pgfscope}%
\pgfpathrectangle{\pgfqpoint{0.424692in}{0.370679in}}{\pgfqpoint{2.725308in}{1.479321in}}%
\pgfusepath{clip}%
\pgfsetbuttcap%
\pgfsetroundjoin%
\definecolor{currentfill}{rgb}{1.000000,0.411765,0.380392}%
\pgfsetfillcolor{currentfill}%
\pgfsetfillopacity{0.500000}%
\pgfsetlinewidth{1.003750pt}%
\definecolor{currentstroke}{rgb}{1.000000,0.411765,0.380392}%
\pgfsetstrokecolor{currentstroke}%
\pgfsetstrokeopacity{0.500000}%
\pgfsetdash{}{0pt}%
\pgfpathmoveto{\pgfqpoint{2.314675in}{1.001848in}}%
\pgfpathcurveto{\pgfqpoint{2.325725in}{1.001848in}}{\pgfqpoint{2.336324in}{1.006239in}}{\pgfqpoint{2.344138in}{1.014052in}}%
\pgfpathcurveto{\pgfqpoint{2.351951in}{1.021866in}}{\pgfqpoint{2.356342in}{1.032465in}}{\pgfqpoint{2.356342in}{1.043515in}}%
\pgfpathcurveto{\pgfqpoint{2.356342in}{1.054565in}}{\pgfqpoint{2.351951in}{1.065164in}}{\pgfqpoint{2.344138in}{1.072978in}}%
\pgfpathcurveto{\pgfqpoint{2.336324in}{1.080792in}}{\pgfqpoint{2.325725in}{1.085182in}}{\pgfqpoint{2.314675in}{1.085182in}}%
\pgfpathcurveto{\pgfqpoint{2.303625in}{1.085182in}}{\pgfqpoint{2.293026in}{1.080792in}}{\pgfqpoint{2.285212in}{1.072978in}}%
\pgfpathcurveto{\pgfqpoint{2.277399in}{1.065164in}}{\pgfqpoint{2.273008in}{1.054565in}}{\pgfqpoint{2.273008in}{1.043515in}}%
\pgfpathcurveto{\pgfqpoint{2.273008in}{1.032465in}}{\pgfqpoint{2.277399in}{1.021866in}}{\pgfqpoint{2.285212in}{1.014052in}}%
\pgfpathcurveto{\pgfqpoint{2.293026in}{1.006239in}}{\pgfqpoint{2.303625in}{1.001848in}}{\pgfqpoint{2.314675in}{1.001848in}}%
\pgfpathclose%
\pgfusepath{stroke,fill}%
\end{pgfscope}%
\begin{pgfscope}%
\pgfpathrectangle{\pgfqpoint{0.424692in}{0.370679in}}{\pgfqpoint{2.725308in}{1.479321in}}%
\pgfusepath{clip}%
\pgfsetbuttcap%
\pgfsetroundjoin%
\definecolor{currentfill}{rgb}{1.000000,0.411765,0.380392}%
\pgfsetfillcolor{currentfill}%
\pgfsetfillopacity{0.500000}%
\pgfsetlinewidth{1.003750pt}%
\definecolor{currentstroke}{rgb}{1.000000,0.411765,0.380392}%
\pgfsetstrokecolor{currentstroke}%
\pgfsetstrokeopacity{0.500000}%
\pgfsetdash{}{0pt}%
\pgfpathmoveto{\pgfqpoint{2.424716in}{1.079367in}}%
\pgfpathcurveto{\pgfqpoint{2.435766in}{1.079367in}}{\pgfqpoint{2.446365in}{1.083757in}}{\pgfqpoint{2.454179in}{1.091571in}}%
\pgfpathcurveto{\pgfqpoint{2.461993in}{1.099385in}}{\pgfqpoint{2.466383in}{1.109984in}}{\pgfqpoint{2.466383in}{1.121034in}}%
\pgfpathcurveto{\pgfqpoint{2.466383in}{1.132084in}}{\pgfqpoint{2.461993in}{1.142683in}}{\pgfqpoint{2.454179in}{1.150497in}}%
\pgfpathcurveto{\pgfqpoint{2.446365in}{1.158310in}}{\pgfqpoint{2.435766in}{1.162701in}}{\pgfqpoint{2.424716in}{1.162701in}}%
\pgfpathcurveto{\pgfqpoint{2.413666in}{1.162701in}}{\pgfqpoint{2.403067in}{1.158310in}}{\pgfqpoint{2.395254in}{1.150497in}}%
\pgfpathcurveto{\pgfqpoint{2.387440in}{1.142683in}}{\pgfqpoint{2.383050in}{1.132084in}}{\pgfqpoint{2.383050in}{1.121034in}}%
\pgfpathcurveto{\pgfqpoint{2.383050in}{1.109984in}}{\pgfqpoint{2.387440in}{1.099385in}}{\pgfqpoint{2.395254in}{1.091571in}}%
\pgfpathcurveto{\pgfqpoint{2.403067in}{1.083757in}}{\pgfqpoint{2.413666in}{1.079367in}}{\pgfqpoint{2.424716in}{1.079367in}}%
\pgfpathclose%
\pgfusepath{stroke,fill}%
\end{pgfscope}%
\begin{pgfscope}%
\pgfpathrectangle{\pgfqpoint{0.424692in}{0.370679in}}{\pgfqpoint{2.725308in}{1.479321in}}%
\pgfusepath{clip}%
\pgfsetbuttcap%
\pgfsetroundjoin%
\definecolor{currentfill}{rgb}{1.000000,0.411765,0.380392}%
\pgfsetfillcolor{currentfill}%
\pgfsetfillopacity{0.500000}%
\pgfsetlinewidth{1.003750pt}%
\definecolor{currentstroke}{rgb}{1.000000,0.411765,0.380392}%
\pgfsetstrokecolor{currentstroke}%
\pgfsetstrokeopacity{0.500000}%
\pgfsetdash{}{0pt}%
\pgfpathmoveto{\pgfqpoint{1.958025in}{0.827397in}}%
\pgfpathcurveto{\pgfqpoint{1.969075in}{0.827397in}}{\pgfqpoint{1.979674in}{0.831787in}}{\pgfqpoint{1.987488in}{0.839601in}}%
\pgfpathcurveto{\pgfqpoint{1.995301in}{0.847414in}}{\pgfqpoint{1.999692in}{0.858013in}}{\pgfqpoint{1.999692in}{0.869064in}}%
\pgfpathcurveto{\pgfqpoint{1.999692in}{0.880114in}}{\pgfqpoint{1.995301in}{0.890713in}}{\pgfqpoint{1.987488in}{0.898526in}}%
\pgfpathcurveto{\pgfqpoint{1.979674in}{0.906340in}}{\pgfqpoint{1.969075in}{0.910730in}}{\pgfqpoint{1.958025in}{0.910730in}}%
\pgfpathcurveto{\pgfqpoint{1.946975in}{0.910730in}}{\pgfqpoint{1.936376in}{0.906340in}}{\pgfqpoint{1.928562in}{0.898526in}}%
\pgfpathcurveto{\pgfqpoint{1.920749in}{0.890713in}}{\pgfqpoint{1.916358in}{0.880114in}}{\pgfqpoint{1.916358in}{0.869064in}}%
\pgfpathcurveto{\pgfqpoint{1.916358in}{0.858013in}}{\pgfqpoint{1.920749in}{0.847414in}}{\pgfqpoint{1.928562in}{0.839601in}}%
\pgfpathcurveto{\pgfqpoint{1.936376in}{0.831787in}}{\pgfqpoint{1.946975in}{0.827397in}}{\pgfqpoint{1.958025in}{0.827397in}}%
\pgfpathclose%
\pgfusepath{stroke,fill}%
\end{pgfscope}%
\begin{pgfscope}%
\pgfpathrectangle{\pgfqpoint{0.424692in}{0.370679in}}{\pgfqpoint{2.725308in}{1.479321in}}%
\pgfusepath{clip}%
\pgfsetbuttcap%
\pgfsetroundjoin%
\definecolor{currentfill}{rgb}{1.000000,0.411765,0.380392}%
\pgfsetfillcolor{currentfill}%
\pgfsetfillopacity{0.500000}%
\pgfsetlinewidth{1.003750pt}%
\definecolor{currentstroke}{rgb}{1.000000,0.411765,0.380392}%
\pgfsetstrokecolor{currentstroke}%
\pgfsetstrokeopacity{0.500000}%
\pgfsetdash{}{0pt}%
\pgfpathmoveto{\pgfqpoint{1.617928in}{0.697245in}}%
\pgfpathcurveto{\pgfqpoint{1.628978in}{0.697245in}}{\pgfqpoint{1.639577in}{0.701635in}}{\pgfqpoint{1.647391in}{0.709449in}}%
\pgfpathcurveto{\pgfqpoint{1.655204in}{0.717262in}}{\pgfqpoint{1.659595in}{0.727861in}}{\pgfqpoint{1.659595in}{0.738911in}}%
\pgfpathcurveto{\pgfqpoint{1.659595in}{0.749961in}}{\pgfqpoint{1.655204in}{0.760561in}}{\pgfqpoint{1.647391in}{0.768374in}}%
\pgfpathcurveto{\pgfqpoint{1.639577in}{0.776188in}}{\pgfqpoint{1.628978in}{0.780578in}}{\pgfqpoint{1.617928in}{0.780578in}}%
\pgfpathcurveto{\pgfqpoint{1.606878in}{0.780578in}}{\pgfqpoint{1.596279in}{0.776188in}}{\pgfqpoint{1.588465in}{0.768374in}}%
\pgfpathcurveto{\pgfqpoint{1.580652in}{0.760561in}}{\pgfqpoint{1.576261in}{0.749961in}}{\pgfqpoint{1.576261in}{0.738911in}}%
\pgfpathcurveto{\pgfqpoint{1.576261in}{0.727861in}}{\pgfqpoint{1.580652in}{0.717262in}}{\pgfqpoint{1.588465in}{0.709449in}}%
\pgfpathcurveto{\pgfqpoint{1.596279in}{0.701635in}}{\pgfqpoint{1.606878in}{0.697245in}}{\pgfqpoint{1.617928in}{0.697245in}}%
\pgfpathclose%
\pgfusepath{stroke,fill}%
\end{pgfscope}%
\begin{pgfscope}%
\pgfpathrectangle{\pgfqpoint{0.424692in}{0.370679in}}{\pgfqpoint{2.725308in}{1.479321in}}%
\pgfusepath{clip}%
\pgfsetbuttcap%
\pgfsetroundjoin%
\definecolor{currentfill}{rgb}{1.000000,0.411765,0.380392}%
\pgfsetfillcolor{currentfill}%
\pgfsetfillopacity{0.500000}%
\pgfsetlinewidth{1.003750pt}%
\definecolor{currentstroke}{rgb}{1.000000,0.411765,0.380392}%
\pgfsetstrokecolor{currentstroke}%
\pgfsetstrokeopacity{0.500000}%
\pgfsetdash{}{0pt}%
\pgfpathmoveto{\pgfqpoint{2.268361in}{0.985084in}}%
\pgfpathcurveto{\pgfqpoint{2.279411in}{0.985084in}}{\pgfqpoint{2.290011in}{0.989474in}}{\pgfqpoint{2.297824in}{0.997288in}}%
\pgfpathcurveto{\pgfqpoint{2.305638in}{1.005101in}}{\pgfqpoint{2.310028in}{1.015700in}}{\pgfqpoint{2.310028in}{1.026750in}}%
\pgfpathcurveto{\pgfqpoint{2.310028in}{1.037800in}}{\pgfqpoint{2.305638in}{1.048399in}}{\pgfqpoint{2.297824in}{1.056213in}}%
\pgfpathcurveto{\pgfqpoint{2.290011in}{1.064027in}}{\pgfqpoint{2.279411in}{1.068417in}}{\pgfqpoint{2.268361in}{1.068417in}}%
\pgfpathcurveto{\pgfqpoint{2.257311in}{1.068417in}}{\pgfqpoint{2.246712in}{1.064027in}}{\pgfqpoint{2.238899in}{1.056213in}}%
\pgfpathcurveto{\pgfqpoint{2.231085in}{1.048399in}}{\pgfqpoint{2.226695in}{1.037800in}}{\pgfqpoint{2.226695in}{1.026750in}}%
\pgfpathcurveto{\pgfqpoint{2.226695in}{1.015700in}}{\pgfqpoint{2.231085in}{1.005101in}}{\pgfqpoint{2.238899in}{0.997288in}}%
\pgfpathcurveto{\pgfqpoint{2.246712in}{0.989474in}}{\pgfqpoint{2.257311in}{0.985084in}}{\pgfqpoint{2.268361in}{0.985084in}}%
\pgfpathclose%
\pgfusepath{stroke,fill}%
\end{pgfscope}%
\begin{pgfscope}%
\pgfpathrectangle{\pgfqpoint{0.424692in}{0.370679in}}{\pgfqpoint{2.725308in}{1.479321in}}%
\pgfusepath{clip}%
\pgfsetbuttcap%
\pgfsetroundjoin%
\definecolor{currentfill}{rgb}{1.000000,0.411765,0.380392}%
\pgfsetfillcolor{currentfill}%
\pgfsetfillopacity{0.500000}%
\pgfsetlinewidth{1.003750pt}%
\definecolor{currentstroke}{rgb}{1.000000,0.411765,0.380392}%
\pgfsetstrokecolor{currentstroke}%
\pgfsetstrokeopacity{0.500000}%
\pgfsetdash{}{0pt}%
\pgfpathmoveto{\pgfqpoint{2.066568in}{0.899042in}}%
\pgfpathcurveto{\pgfqpoint{2.077618in}{0.899042in}}{\pgfqpoint{2.088217in}{0.903432in}}{\pgfqpoint{2.096031in}{0.911246in}}%
\pgfpathcurveto{\pgfqpoint{2.103844in}{0.919059in}}{\pgfqpoint{2.108235in}{0.929658in}}{\pgfqpoint{2.108235in}{0.940708in}}%
\pgfpathcurveto{\pgfqpoint{2.108235in}{0.951759in}}{\pgfqpoint{2.103844in}{0.962358in}}{\pgfqpoint{2.096031in}{0.970171in}}%
\pgfpathcurveto{\pgfqpoint{2.088217in}{0.977985in}}{\pgfqpoint{2.077618in}{0.982375in}}{\pgfqpoint{2.066568in}{0.982375in}}%
\pgfpathcurveto{\pgfqpoint{2.055518in}{0.982375in}}{\pgfqpoint{2.044919in}{0.977985in}}{\pgfqpoint{2.037105in}{0.970171in}}%
\pgfpathcurveto{\pgfqpoint{2.029292in}{0.962358in}}{\pgfqpoint{2.024901in}{0.951759in}}{\pgfqpoint{2.024901in}{0.940708in}}%
\pgfpathcurveto{\pgfqpoint{2.024901in}{0.929658in}}{\pgfqpoint{2.029292in}{0.919059in}}{\pgfqpoint{2.037105in}{0.911246in}}%
\pgfpathcurveto{\pgfqpoint{2.044919in}{0.903432in}}{\pgfqpoint{2.055518in}{0.899042in}}{\pgfqpoint{2.066568in}{0.899042in}}%
\pgfpathclose%
\pgfusepath{stroke,fill}%
\end{pgfscope}%
\begin{pgfscope}%
\pgfpathrectangle{\pgfqpoint{0.424692in}{0.370679in}}{\pgfqpoint{2.725308in}{1.479321in}}%
\pgfusepath{clip}%
\pgfsetbuttcap%
\pgfsetroundjoin%
\definecolor{currentfill}{rgb}{1.000000,0.411765,0.380392}%
\pgfsetfillcolor{currentfill}%
\pgfsetfillopacity{0.500000}%
\pgfsetlinewidth{1.003750pt}%
\definecolor{currentstroke}{rgb}{1.000000,0.411765,0.380392}%
\pgfsetstrokecolor{currentstroke}%
\pgfsetstrokeopacity{0.500000}%
\pgfsetdash{}{0pt}%
\pgfpathmoveto{\pgfqpoint{2.625602in}{1.171403in}}%
\pgfpathcurveto{\pgfqpoint{2.636653in}{1.171403in}}{\pgfqpoint{2.647252in}{1.175794in}}{\pgfqpoint{2.655065in}{1.183607in}}%
\pgfpathcurveto{\pgfqpoint{2.662879in}{1.191421in}}{\pgfqpoint{2.667269in}{1.202020in}}{\pgfqpoint{2.667269in}{1.213070in}}%
\pgfpathcurveto{\pgfqpoint{2.667269in}{1.224120in}}{\pgfqpoint{2.662879in}{1.234719in}}{\pgfqpoint{2.655065in}{1.242533in}}%
\pgfpathcurveto{\pgfqpoint{2.647252in}{1.250347in}}{\pgfqpoint{2.636653in}{1.254737in}}{\pgfqpoint{2.625602in}{1.254737in}}%
\pgfpathcurveto{\pgfqpoint{2.614552in}{1.254737in}}{\pgfqpoint{2.603953in}{1.250347in}}{\pgfqpoint{2.596140in}{1.242533in}}%
\pgfpathcurveto{\pgfqpoint{2.588326in}{1.234719in}}{\pgfqpoint{2.583936in}{1.224120in}}{\pgfqpoint{2.583936in}{1.213070in}}%
\pgfpathcurveto{\pgfqpoint{2.583936in}{1.202020in}}{\pgfqpoint{2.588326in}{1.191421in}}{\pgfqpoint{2.596140in}{1.183607in}}%
\pgfpathcurveto{\pgfqpoint{2.603953in}{1.175794in}}{\pgfqpoint{2.614552in}{1.171403in}}{\pgfqpoint{2.625602in}{1.171403in}}%
\pgfpathclose%
\pgfusepath{stroke,fill}%
\end{pgfscope}%
\begin{pgfscope}%
\pgfpathrectangle{\pgfqpoint{0.424692in}{0.370679in}}{\pgfqpoint{2.725308in}{1.479321in}}%
\pgfusepath{clip}%
\pgfsetbuttcap%
\pgfsetroundjoin%
\definecolor{currentfill}{rgb}{1.000000,0.411765,0.380392}%
\pgfsetfillcolor{currentfill}%
\pgfsetfillopacity{0.500000}%
\pgfsetlinewidth{1.003750pt}%
\definecolor{currentstroke}{rgb}{1.000000,0.411765,0.380392}%
\pgfsetstrokecolor{currentstroke}%
\pgfsetstrokeopacity{0.500000}%
\pgfsetdash{}{0pt}%
\pgfpathmoveto{\pgfqpoint{2.271792in}{1.004999in}}%
\pgfpathcurveto{\pgfqpoint{2.282842in}{1.004999in}}{\pgfqpoint{2.293441in}{1.009389in}}{\pgfqpoint{2.301255in}{1.017202in}}%
\pgfpathcurveto{\pgfqpoint{2.309069in}{1.025016in}}{\pgfqpoint{2.313459in}{1.035615in}}{\pgfqpoint{2.313459in}{1.046665in}}%
\pgfpathcurveto{\pgfqpoint{2.313459in}{1.057715in}}{\pgfqpoint{2.309069in}{1.068314in}}{\pgfqpoint{2.301255in}{1.076128in}}%
\pgfpathcurveto{\pgfqpoint{2.293441in}{1.083942in}}{\pgfqpoint{2.282842in}{1.088332in}}{\pgfqpoint{2.271792in}{1.088332in}}%
\pgfpathcurveto{\pgfqpoint{2.260742in}{1.088332in}}{\pgfqpoint{2.250143in}{1.083942in}}{\pgfqpoint{2.242329in}{1.076128in}}%
\pgfpathcurveto{\pgfqpoint{2.234516in}{1.068314in}}{\pgfqpoint{2.230126in}{1.057715in}}{\pgfqpoint{2.230126in}{1.046665in}}%
\pgfpathcurveto{\pgfqpoint{2.230126in}{1.035615in}}{\pgfqpoint{2.234516in}{1.025016in}}{\pgfqpoint{2.242329in}{1.017202in}}%
\pgfpathcurveto{\pgfqpoint{2.250143in}{1.009389in}}{\pgfqpoint{2.260742in}{1.004999in}}{\pgfqpoint{2.271792in}{1.004999in}}%
\pgfpathclose%
\pgfusepath{stroke,fill}%
\end{pgfscope}%
\begin{pgfscope}%
\pgfpathrectangle{\pgfqpoint{0.424692in}{0.370679in}}{\pgfqpoint{2.725308in}{1.479321in}}%
\pgfusepath{clip}%
\pgfsetbuttcap%
\pgfsetroundjoin%
\definecolor{currentfill}{rgb}{1.000000,0.411765,0.380392}%
\pgfsetfillcolor{currentfill}%
\pgfsetfillopacity{0.500000}%
\pgfsetlinewidth{1.003750pt}%
\definecolor{currentstroke}{rgb}{1.000000,0.411765,0.380392}%
\pgfsetstrokecolor{currentstroke}%
\pgfsetstrokeopacity{0.500000}%
\pgfsetdash{}{0pt}%
\pgfpathmoveto{\pgfqpoint{1.707031in}{0.717956in}}%
\pgfpathcurveto{\pgfqpoint{1.718082in}{0.717956in}}{\pgfqpoint{1.728681in}{0.722346in}}{\pgfqpoint{1.736494in}{0.730160in}}%
\pgfpathcurveto{\pgfqpoint{1.744308in}{0.737974in}}{\pgfqpoint{1.748698in}{0.748573in}}{\pgfqpoint{1.748698in}{0.759623in}}%
\pgfpathcurveto{\pgfqpoint{1.748698in}{0.770673in}}{\pgfqpoint{1.744308in}{0.781272in}}{\pgfqpoint{1.736494in}{0.789085in}}%
\pgfpathcurveto{\pgfqpoint{1.728681in}{0.796899in}}{\pgfqpoint{1.718082in}{0.801289in}}{\pgfqpoint{1.707031in}{0.801289in}}%
\pgfpathcurveto{\pgfqpoint{1.695981in}{0.801289in}}{\pgfqpoint{1.685382in}{0.796899in}}{\pgfqpoint{1.677569in}{0.789085in}}%
\pgfpathcurveto{\pgfqpoint{1.669755in}{0.781272in}}{\pgfqpoint{1.665365in}{0.770673in}}{\pgfqpoint{1.665365in}{0.759623in}}%
\pgfpathcurveto{\pgfqpoint{1.665365in}{0.748573in}}{\pgfqpoint{1.669755in}{0.737974in}}{\pgfqpoint{1.677569in}{0.730160in}}%
\pgfpathcurveto{\pgfqpoint{1.685382in}{0.722346in}}{\pgfqpoint{1.695981in}{0.717956in}}{\pgfqpoint{1.707031in}{0.717956in}}%
\pgfpathclose%
\pgfusepath{stroke,fill}%
\end{pgfscope}%
\begin{pgfscope}%
\pgfpathrectangle{\pgfqpoint{0.424692in}{0.370679in}}{\pgfqpoint{2.725308in}{1.479321in}}%
\pgfusepath{clip}%
\pgfsetbuttcap%
\pgfsetroundjoin%
\definecolor{currentfill}{rgb}{1.000000,0.411765,0.380392}%
\pgfsetfillcolor{currentfill}%
\pgfsetfillopacity{0.500000}%
\pgfsetlinewidth{1.003750pt}%
\definecolor{currentstroke}{rgb}{1.000000,0.411765,0.380392}%
\pgfsetstrokecolor{currentstroke}%
\pgfsetstrokeopacity{0.500000}%
\pgfsetdash{}{0pt}%
\pgfpathmoveto{\pgfqpoint{2.172665in}{0.958459in}}%
\pgfpathcurveto{\pgfqpoint{2.183715in}{0.958459in}}{\pgfqpoint{2.194314in}{0.962849in}}{\pgfqpoint{2.202128in}{0.970663in}}%
\pgfpathcurveto{\pgfqpoint{2.209942in}{0.978476in}}{\pgfqpoint{2.214332in}{0.989075in}}{\pgfqpoint{2.214332in}{1.000125in}}%
\pgfpathcurveto{\pgfqpoint{2.214332in}{1.011175in}}{\pgfqpoint{2.209942in}{1.021774in}}{\pgfqpoint{2.202128in}{1.029588in}}%
\pgfpathcurveto{\pgfqpoint{2.194314in}{1.037402in}}{\pgfqpoint{2.183715in}{1.041792in}}{\pgfqpoint{2.172665in}{1.041792in}}%
\pgfpathcurveto{\pgfqpoint{2.161615in}{1.041792in}}{\pgfqpoint{2.151016in}{1.037402in}}{\pgfqpoint{2.143202in}{1.029588in}}%
\pgfpathcurveto{\pgfqpoint{2.135389in}{1.021774in}}{\pgfqpoint{2.130999in}{1.011175in}}{\pgfqpoint{2.130999in}{1.000125in}}%
\pgfpathcurveto{\pgfqpoint{2.130999in}{0.989075in}}{\pgfqpoint{2.135389in}{0.978476in}}{\pgfqpoint{2.143202in}{0.970663in}}%
\pgfpathcurveto{\pgfqpoint{2.151016in}{0.962849in}}{\pgfqpoint{2.161615in}{0.958459in}}{\pgfqpoint{2.172665in}{0.958459in}}%
\pgfpathclose%
\pgfusepath{stroke,fill}%
\end{pgfscope}%
\begin{pgfscope}%
\pgfpathrectangle{\pgfqpoint{0.424692in}{0.370679in}}{\pgfqpoint{2.725308in}{1.479321in}}%
\pgfusepath{clip}%
\pgfsetbuttcap%
\pgfsetroundjoin%
\definecolor{currentfill}{rgb}{1.000000,0.411765,0.380392}%
\pgfsetfillcolor{currentfill}%
\pgfsetfillopacity{0.500000}%
\pgfsetlinewidth{1.003750pt}%
\definecolor{currentstroke}{rgb}{1.000000,0.411765,0.380392}%
\pgfsetstrokecolor{currentstroke}%
\pgfsetstrokeopacity{0.500000}%
\pgfsetdash{}{0pt}%
\pgfpathmoveto{\pgfqpoint{2.367611in}{1.043235in}}%
\pgfpathcurveto{\pgfqpoint{2.378661in}{1.043235in}}{\pgfqpoint{2.389260in}{1.047626in}}{\pgfqpoint{2.397074in}{1.055439in}}%
\pgfpathcurveto{\pgfqpoint{2.404888in}{1.063253in}}{\pgfqpoint{2.409278in}{1.073852in}}{\pgfqpoint{2.409278in}{1.084902in}}%
\pgfpathcurveto{\pgfqpoint{2.409278in}{1.095952in}}{\pgfqpoint{2.404888in}{1.106551in}}{\pgfqpoint{2.397074in}{1.114365in}}%
\pgfpathcurveto{\pgfqpoint{2.389260in}{1.122178in}}{\pgfqpoint{2.378661in}{1.126569in}}{\pgfqpoint{2.367611in}{1.126569in}}%
\pgfpathcurveto{\pgfqpoint{2.356561in}{1.126569in}}{\pgfqpoint{2.345962in}{1.122178in}}{\pgfqpoint{2.338148in}{1.114365in}}%
\pgfpathcurveto{\pgfqpoint{2.330335in}{1.106551in}}{\pgfqpoint{2.325945in}{1.095952in}}{\pgfqpoint{2.325945in}{1.084902in}}%
\pgfpathcurveto{\pgfqpoint{2.325945in}{1.073852in}}{\pgfqpoint{2.330335in}{1.063253in}}{\pgfqpoint{2.338148in}{1.055439in}}%
\pgfpathcurveto{\pgfqpoint{2.345962in}{1.047626in}}{\pgfqpoint{2.356561in}{1.043235in}}{\pgfqpoint{2.367611in}{1.043235in}}%
\pgfpathclose%
\pgfusepath{stroke,fill}%
\end{pgfscope}%
\begin{pgfscope}%
\pgfpathrectangle{\pgfqpoint{0.424692in}{0.370679in}}{\pgfqpoint{2.725308in}{1.479321in}}%
\pgfusepath{clip}%
\pgfsetbuttcap%
\pgfsetroundjoin%
\definecolor{currentfill}{rgb}{1.000000,0.411765,0.380392}%
\pgfsetfillcolor{currentfill}%
\pgfsetfillopacity{0.500000}%
\pgfsetlinewidth{1.003750pt}%
\definecolor{currentstroke}{rgb}{1.000000,0.411765,0.380392}%
\pgfsetstrokecolor{currentstroke}%
\pgfsetstrokeopacity{0.500000}%
\pgfsetdash{}{0pt}%
\pgfpathmoveto{\pgfqpoint{2.609344in}{1.121165in}}%
\pgfpathcurveto{\pgfqpoint{2.620395in}{1.121165in}}{\pgfqpoint{2.630994in}{1.125556in}}{\pgfqpoint{2.638807in}{1.133369in}}%
\pgfpathcurveto{\pgfqpoint{2.646621in}{1.141183in}}{\pgfqpoint{2.651011in}{1.151782in}}{\pgfqpoint{2.651011in}{1.162832in}}%
\pgfpathcurveto{\pgfqpoint{2.651011in}{1.173882in}}{\pgfqpoint{2.646621in}{1.184481in}}{\pgfqpoint{2.638807in}{1.192295in}}%
\pgfpathcurveto{\pgfqpoint{2.630994in}{1.200109in}}{\pgfqpoint{2.620395in}{1.204499in}}{\pgfqpoint{2.609344in}{1.204499in}}%
\pgfpathcurveto{\pgfqpoint{2.598294in}{1.204499in}}{\pgfqpoint{2.587695in}{1.200109in}}{\pgfqpoint{2.579882in}{1.192295in}}%
\pgfpathcurveto{\pgfqpoint{2.572068in}{1.184481in}}{\pgfqpoint{2.567678in}{1.173882in}}{\pgfqpoint{2.567678in}{1.162832in}}%
\pgfpathcurveto{\pgfqpoint{2.567678in}{1.151782in}}{\pgfqpoint{2.572068in}{1.141183in}}{\pgfqpoint{2.579882in}{1.133369in}}%
\pgfpathcurveto{\pgfqpoint{2.587695in}{1.125556in}}{\pgfqpoint{2.598294in}{1.121165in}}{\pgfqpoint{2.609344in}{1.121165in}}%
\pgfpathclose%
\pgfusepath{stroke,fill}%
\end{pgfscope}%
\begin{pgfscope}%
\pgfpathrectangle{\pgfqpoint{0.424692in}{0.370679in}}{\pgfqpoint{2.725308in}{1.479321in}}%
\pgfusepath{clip}%
\pgfsetbuttcap%
\pgfsetroundjoin%
\definecolor{currentfill}{rgb}{1.000000,0.411765,0.380392}%
\pgfsetfillcolor{currentfill}%
\pgfsetfillopacity{0.500000}%
\pgfsetlinewidth{1.003750pt}%
\definecolor{currentstroke}{rgb}{1.000000,0.411765,0.380392}%
\pgfsetstrokecolor{currentstroke}%
\pgfsetstrokeopacity{0.500000}%
\pgfsetdash{}{0pt}%
\pgfpathmoveto{\pgfqpoint{1.333240in}{0.497571in}}%
\pgfpathcurveto{\pgfqpoint{1.344290in}{0.497571in}}{\pgfqpoint{1.354890in}{0.501961in}}{\pgfqpoint{1.362703in}{0.509775in}}%
\pgfpathcurveto{\pgfqpoint{1.370517in}{0.517588in}}{\pgfqpoint{1.374907in}{0.528187in}}{\pgfqpoint{1.374907in}{0.539238in}}%
\pgfpathcurveto{\pgfqpoint{1.374907in}{0.550288in}}{\pgfqpoint{1.370517in}{0.560887in}}{\pgfqpoint{1.362703in}{0.568700in}}%
\pgfpathcurveto{\pgfqpoint{1.354890in}{0.576514in}}{\pgfqpoint{1.344290in}{0.580904in}}{\pgfqpoint{1.333240in}{0.580904in}}%
\pgfpathcurveto{\pgfqpoint{1.322190in}{0.580904in}}{\pgfqpoint{1.311591in}{0.576514in}}{\pgfqpoint{1.303778in}{0.568700in}}%
\pgfpathcurveto{\pgfqpoint{1.295964in}{0.560887in}}{\pgfqpoint{1.291574in}{0.550288in}}{\pgfqpoint{1.291574in}{0.539238in}}%
\pgfpathcurveto{\pgfqpoint{1.291574in}{0.528187in}}{\pgfqpoint{1.295964in}{0.517588in}}{\pgfqpoint{1.303778in}{0.509775in}}%
\pgfpathcurveto{\pgfqpoint{1.311591in}{0.501961in}}{\pgfqpoint{1.322190in}{0.497571in}}{\pgfqpoint{1.333240in}{0.497571in}}%
\pgfpathclose%
\pgfusepath{stroke,fill}%
\end{pgfscope}%
\begin{pgfscope}%
\pgfpathrectangle{\pgfqpoint{0.424692in}{0.370679in}}{\pgfqpoint{2.725308in}{1.479321in}}%
\pgfusepath{clip}%
\pgfsetbuttcap%
\pgfsetroundjoin%
\definecolor{currentfill}{rgb}{1.000000,0.411765,0.380392}%
\pgfsetfillcolor{currentfill}%
\pgfsetfillopacity{0.500000}%
\pgfsetlinewidth{1.003750pt}%
\definecolor{currentstroke}{rgb}{1.000000,0.411765,0.380392}%
\pgfsetstrokecolor{currentstroke}%
\pgfsetstrokeopacity{0.500000}%
\pgfsetdash{}{0pt}%
\pgfpathmoveto{\pgfqpoint{1.198575in}{0.456110in}}%
\pgfpathcurveto{\pgfqpoint{1.209626in}{0.456110in}}{\pgfqpoint{1.220225in}{0.460500in}}{\pgfqpoint{1.228038in}{0.468313in}}%
\pgfpathcurveto{\pgfqpoint{1.235852in}{0.476127in}}{\pgfqpoint{1.240242in}{0.486726in}}{\pgfqpoint{1.240242in}{0.497776in}}%
\pgfpathcurveto{\pgfqpoint{1.240242in}{0.508826in}}{\pgfqpoint{1.235852in}{0.519425in}}{\pgfqpoint{1.228038in}{0.527239in}}%
\pgfpathcurveto{\pgfqpoint{1.220225in}{0.535053in}}{\pgfqpoint{1.209626in}{0.539443in}}{\pgfqpoint{1.198575in}{0.539443in}}%
\pgfpathcurveto{\pgfqpoint{1.187525in}{0.539443in}}{\pgfqpoint{1.176926in}{0.535053in}}{\pgfqpoint{1.169113in}{0.527239in}}%
\pgfpathcurveto{\pgfqpoint{1.161299in}{0.519425in}}{\pgfqpoint{1.156909in}{0.508826in}}{\pgfqpoint{1.156909in}{0.497776in}}%
\pgfpathcurveto{\pgfqpoint{1.156909in}{0.486726in}}{\pgfqpoint{1.161299in}{0.476127in}}{\pgfqpoint{1.169113in}{0.468313in}}%
\pgfpathcurveto{\pgfqpoint{1.176926in}{0.460500in}}{\pgfqpoint{1.187525in}{0.456110in}}{\pgfqpoint{1.198575in}{0.456110in}}%
\pgfpathclose%
\pgfusepath{stroke,fill}%
\end{pgfscope}%
\begin{pgfscope}%
\pgfpathrectangle{\pgfqpoint{0.424692in}{0.370679in}}{\pgfqpoint{2.725308in}{1.479321in}}%
\pgfusepath{clip}%
\pgfsetbuttcap%
\pgfsetroundjoin%
\definecolor{currentfill}{rgb}{1.000000,0.411765,0.380392}%
\pgfsetfillcolor{currentfill}%
\pgfsetfillopacity{0.500000}%
\pgfsetlinewidth{1.003750pt}%
\definecolor{currentstroke}{rgb}{1.000000,0.411765,0.380392}%
\pgfsetstrokecolor{currentstroke}%
\pgfsetstrokeopacity{0.500000}%
\pgfsetdash{}{0pt}%
\pgfpathmoveto{\pgfqpoint{1.597208in}{0.643907in}}%
\pgfpathcurveto{\pgfqpoint{1.608258in}{0.643907in}}{\pgfqpoint{1.618857in}{0.648297in}}{\pgfqpoint{1.626671in}{0.656111in}}%
\pgfpathcurveto{\pgfqpoint{1.634484in}{0.663924in}}{\pgfqpoint{1.638874in}{0.674523in}}{\pgfqpoint{1.638874in}{0.685573in}}%
\pgfpathcurveto{\pgfqpoint{1.638874in}{0.696623in}}{\pgfqpoint{1.634484in}{0.707223in}}{\pgfqpoint{1.626671in}{0.715036in}}%
\pgfpathcurveto{\pgfqpoint{1.618857in}{0.722850in}}{\pgfqpoint{1.608258in}{0.727240in}}{\pgfqpoint{1.597208in}{0.727240in}}%
\pgfpathcurveto{\pgfqpoint{1.586158in}{0.727240in}}{\pgfqpoint{1.575559in}{0.722850in}}{\pgfqpoint{1.567745in}{0.715036in}}%
\pgfpathcurveto{\pgfqpoint{1.559931in}{0.707223in}}{\pgfqpoint{1.555541in}{0.696623in}}{\pgfqpoint{1.555541in}{0.685573in}}%
\pgfpathcurveto{\pgfqpoint{1.555541in}{0.674523in}}{\pgfqpoint{1.559931in}{0.663924in}}{\pgfqpoint{1.567745in}{0.656111in}}%
\pgfpathcurveto{\pgfqpoint{1.575559in}{0.648297in}}{\pgfqpoint{1.586158in}{0.643907in}}{\pgfqpoint{1.597208in}{0.643907in}}%
\pgfpathclose%
\pgfusepath{stroke,fill}%
\end{pgfscope}%
\begin{pgfscope}%
\pgfpathrectangle{\pgfqpoint{0.424692in}{0.370679in}}{\pgfqpoint{2.725308in}{1.479321in}}%
\pgfusepath{clip}%
\pgfsetbuttcap%
\pgfsetroundjoin%
\definecolor{currentfill}{rgb}{1.000000,0.411765,0.380392}%
\pgfsetfillcolor{currentfill}%
\pgfsetfillopacity{0.500000}%
\pgfsetlinewidth{1.003750pt}%
\definecolor{currentstroke}{rgb}{1.000000,0.411765,0.380392}%
\pgfsetstrokecolor{currentstroke}%
\pgfsetstrokeopacity{0.500000}%
\pgfsetdash{}{0pt}%
\pgfpathmoveto{\pgfqpoint{1.962665in}{0.839780in}}%
\pgfpathcurveto{\pgfqpoint{1.973715in}{0.839780in}}{\pgfqpoint{1.984314in}{0.844170in}}{\pgfqpoint{1.992128in}{0.851983in}}%
\pgfpathcurveto{\pgfqpoint{1.999941in}{0.859797in}}{\pgfqpoint{2.004332in}{0.870396in}}{\pgfqpoint{2.004332in}{0.881446in}}%
\pgfpathcurveto{\pgfqpoint{2.004332in}{0.892496in}}{\pgfqpoint{1.999941in}{0.903095in}}{\pgfqpoint{1.992128in}{0.910909in}}%
\pgfpathcurveto{\pgfqpoint{1.984314in}{0.918723in}}{\pgfqpoint{1.973715in}{0.923113in}}{\pgfqpoint{1.962665in}{0.923113in}}%
\pgfpathcurveto{\pgfqpoint{1.951615in}{0.923113in}}{\pgfqpoint{1.941016in}{0.918723in}}{\pgfqpoint{1.933202in}{0.910909in}}%
\pgfpathcurveto{\pgfqpoint{1.925389in}{0.903095in}}{\pgfqpoint{1.920998in}{0.892496in}}{\pgfqpoint{1.920998in}{0.881446in}}%
\pgfpathcurveto{\pgfqpoint{1.920998in}{0.870396in}}{\pgfqpoint{1.925389in}{0.859797in}}{\pgfqpoint{1.933202in}{0.851983in}}%
\pgfpathcurveto{\pgfqpoint{1.941016in}{0.844170in}}{\pgfqpoint{1.951615in}{0.839780in}}{\pgfqpoint{1.962665in}{0.839780in}}%
\pgfpathclose%
\pgfusepath{stroke,fill}%
\end{pgfscope}%
\begin{pgfscope}%
\pgfpathrectangle{\pgfqpoint{0.424692in}{0.370679in}}{\pgfqpoint{2.725308in}{1.479321in}}%
\pgfusepath{clip}%
\pgfsetbuttcap%
\pgfsetroundjoin%
\definecolor{currentfill}{rgb}{1.000000,0.411765,0.380392}%
\pgfsetfillcolor{currentfill}%
\pgfsetfillopacity{0.500000}%
\pgfsetlinewidth{1.003750pt}%
\definecolor{currentstroke}{rgb}{1.000000,0.411765,0.380392}%
\pgfsetstrokecolor{currentstroke}%
\pgfsetstrokeopacity{0.500000}%
\pgfsetdash{}{0pt}%
\pgfpathmoveto{\pgfqpoint{1.515242in}{0.651241in}}%
\pgfpathcurveto{\pgfqpoint{1.526293in}{0.651241in}}{\pgfqpoint{1.536892in}{0.655631in}}{\pgfqpoint{1.544705in}{0.663445in}}%
\pgfpathcurveto{\pgfqpoint{1.552519in}{0.671258in}}{\pgfqpoint{1.556909in}{0.681857in}}{\pgfqpoint{1.556909in}{0.692908in}}%
\pgfpathcurveto{\pgfqpoint{1.556909in}{0.703958in}}{\pgfqpoint{1.552519in}{0.714557in}}{\pgfqpoint{1.544705in}{0.722370in}}%
\pgfpathcurveto{\pgfqpoint{1.536892in}{0.730184in}}{\pgfqpoint{1.526293in}{0.734574in}}{\pgfqpoint{1.515242in}{0.734574in}}%
\pgfpathcurveto{\pgfqpoint{1.504192in}{0.734574in}}{\pgfqpoint{1.493593in}{0.730184in}}{\pgfqpoint{1.485780in}{0.722370in}}%
\pgfpathcurveto{\pgfqpoint{1.477966in}{0.714557in}}{\pgfqpoint{1.473576in}{0.703958in}}{\pgfqpoint{1.473576in}{0.692908in}}%
\pgfpathcurveto{\pgfqpoint{1.473576in}{0.681857in}}{\pgfqpoint{1.477966in}{0.671258in}}{\pgfqpoint{1.485780in}{0.663445in}}%
\pgfpathcurveto{\pgfqpoint{1.493593in}{0.655631in}}{\pgfqpoint{1.504192in}{0.651241in}}{\pgfqpoint{1.515242in}{0.651241in}}%
\pgfpathclose%
\pgfusepath{stroke,fill}%
\end{pgfscope}%
\begin{pgfscope}%
\pgfpathrectangle{\pgfqpoint{0.424692in}{0.370679in}}{\pgfqpoint{2.725308in}{1.479321in}}%
\pgfusepath{clip}%
\pgfsetbuttcap%
\pgfsetroundjoin%
\definecolor{currentfill}{rgb}{1.000000,0.411765,0.380392}%
\pgfsetfillcolor{currentfill}%
\pgfsetfillopacity{0.500000}%
\pgfsetlinewidth{1.003750pt}%
\definecolor{currentstroke}{rgb}{1.000000,0.411765,0.380392}%
\pgfsetstrokecolor{currentstroke}%
\pgfsetstrokeopacity{0.500000}%
\pgfsetdash{}{0pt}%
\pgfpathmoveto{\pgfqpoint{1.811897in}{0.765154in}}%
\pgfpathcurveto{\pgfqpoint{1.822947in}{0.765154in}}{\pgfqpoint{1.833546in}{0.769544in}}{\pgfqpoint{1.841360in}{0.777358in}}%
\pgfpathcurveto{\pgfqpoint{1.849174in}{0.785171in}}{\pgfqpoint{1.853564in}{0.795770in}}{\pgfqpoint{1.853564in}{0.806821in}}%
\pgfpathcurveto{\pgfqpoint{1.853564in}{0.817871in}}{\pgfqpoint{1.849174in}{0.828470in}}{\pgfqpoint{1.841360in}{0.836283in}}%
\pgfpathcurveto{\pgfqpoint{1.833546in}{0.844097in}}{\pgfqpoint{1.822947in}{0.848487in}}{\pgfqpoint{1.811897in}{0.848487in}}%
\pgfpathcurveto{\pgfqpoint{1.800847in}{0.848487in}}{\pgfqpoint{1.790248in}{0.844097in}}{\pgfqpoint{1.782434in}{0.836283in}}%
\pgfpathcurveto{\pgfqpoint{1.774621in}{0.828470in}}{\pgfqpoint{1.770230in}{0.817871in}}{\pgfqpoint{1.770230in}{0.806821in}}%
\pgfpathcurveto{\pgfqpoint{1.770230in}{0.795770in}}{\pgfqpoint{1.774621in}{0.785171in}}{\pgfqpoint{1.782434in}{0.777358in}}%
\pgfpathcurveto{\pgfqpoint{1.790248in}{0.769544in}}{\pgfqpoint{1.800847in}{0.765154in}}{\pgfqpoint{1.811897in}{0.765154in}}%
\pgfpathclose%
\pgfusepath{stroke,fill}%
\end{pgfscope}%
\begin{pgfscope}%
\pgfpathrectangle{\pgfqpoint{0.424692in}{0.370679in}}{\pgfqpoint{2.725308in}{1.479321in}}%
\pgfusepath{clip}%
\pgfsetbuttcap%
\pgfsetroundjoin%
\definecolor{currentfill}{rgb}{1.000000,0.411765,0.380392}%
\pgfsetfillcolor{currentfill}%
\pgfsetfillopacity{0.500000}%
\pgfsetlinewidth{1.003750pt}%
\definecolor{currentstroke}{rgb}{1.000000,0.411765,0.380392}%
\pgfsetstrokecolor{currentstroke}%
\pgfsetstrokeopacity{0.500000}%
\pgfsetdash{}{0pt}%
\pgfpathmoveto{\pgfqpoint{1.870622in}{0.831097in}}%
\pgfpathcurveto{\pgfqpoint{1.881672in}{0.831097in}}{\pgfqpoint{1.892271in}{0.835487in}}{\pgfqpoint{1.900085in}{0.843301in}}%
\pgfpathcurveto{\pgfqpoint{1.907899in}{0.851115in}}{\pgfqpoint{1.912289in}{0.861714in}}{\pgfqpoint{1.912289in}{0.872764in}}%
\pgfpathcurveto{\pgfqpoint{1.912289in}{0.883814in}}{\pgfqpoint{1.907899in}{0.894413in}}{\pgfqpoint{1.900085in}{0.902227in}}%
\pgfpathcurveto{\pgfqpoint{1.892271in}{0.910040in}}{\pgfqpoint{1.881672in}{0.914431in}}{\pgfqpoint{1.870622in}{0.914431in}}%
\pgfpathcurveto{\pgfqpoint{1.859572in}{0.914431in}}{\pgfqpoint{1.848973in}{0.910040in}}{\pgfqpoint{1.841159in}{0.902227in}}%
\pgfpathcurveto{\pgfqpoint{1.833346in}{0.894413in}}{\pgfqpoint{1.828956in}{0.883814in}}{\pgfqpoint{1.828956in}{0.872764in}}%
\pgfpathcurveto{\pgfqpoint{1.828956in}{0.861714in}}{\pgfqpoint{1.833346in}{0.851115in}}{\pgfqpoint{1.841159in}{0.843301in}}%
\pgfpathcurveto{\pgfqpoint{1.848973in}{0.835487in}}{\pgfqpoint{1.859572in}{0.831097in}}{\pgfqpoint{1.870622in}{0.831097in}}%
\pgfpathclose%
\pgfusepath{stroke,fill}%
\end{pgfscope}%
\begin{pgfscope}%
\pgfpathrectangle{\pgfqpoint{0.424692in}{0.370679in}}{\pgfqpoint{2.725308in}{1.479321in}}%
\pgfusepath{clip}%
\pgfsetbuttcap%
\pgfsetroundjoin%
\definecolor{currentfill}{rgb}{1.000000,0.411765,0.380392}%
\pgfsetfillcolor{currentfill}%
\pgfsetfillopacity{0.500000}%
\pgfsetlinewidth{1.003750pt}%
\definecolor{currentstroke}{rgb}{1.000000,0.411765,0.380392}%
\pgfsetstrokecolor{currentstroke}%
\pgfsetstrokeopacity{0.500000}%
\pgfsetdash{}{0pt}%
\pgfpathmoveto{\pgfqpoint{1.865941in}{0.800000in}}%
\pgfpathcurveto{\pgfqpoint{1.876991in}{0.800000in}}{\pgfqpoint{1.887590in}{0.804390in}}{\pgfqpoint{1.895404in}{0.812204in}}%
\pgfpathcurveto{\pgfqpoint{1.903217in}{0.820017in}}{\pgfqpoint{1.907607in}{0.830616in}}{\pgfqpoint{1.907607in}{0.841666in}}%
\pgfpathcurveto{\pgfqpoint{1.907607in}{0.852717in}}{\pgfqpoint{1.903217in}{0.863316in}}{\pgfqpoint{1.895404in}{0.871129in}}%
\pgfpathcurveto{\pgfqpoint{1.887590in}{0.878943in}}{\pgfqpoint{1.876991in}{0.883333in}}{\pgfqpoint{1.865941in}{0.883333in}}%
\pgfpathcurveto{\pgfqpoint{1.854891in}{0.883333in}}{\pgfqpoint{1.844292in}{0.878943in}}{\pgfqpoint{1.836478in}{0.871129in}}%
\pgfpathcurveto{\pgfqpoint{1.828664in}{0.863316in}}{\pgfqpoint{1.824274in}{0.852717in}}{\pgfqpoint{1.824274in}{0.841666in}}%
\pgfpathcurveto{\pgfqpoint{1.824274in}{0.830616in}}{\pgfqpoint{1.828664in}{0.820017in}}{\pgfqpoint{1.836478in}{0.812204in}}%
\pgfpathcurveto{\pgfqpoint{1.844292in}{0.804390in}}{\pgfqpoint{1.854891in}{0.800000in}}{\pgfqpoint{1.865941in}{0.800000in}}%
\pgfpathclose%
\pgfusepath{stroke,fill}%
\end{pgfscope}%
\begin{pgfscope}%
\pgfpathrectangle{\pgfqpoint{0.424692in}{0.370679in}}{\pgfqpoint{2.725308in}{1.479321in}}%
\pgfusepath{clip}%
\pgfsetbuttcap%
\pgfsetroundjoin%
\definecolor{currentfill}{rgb}{1.000000,0.411765,0.380392}%
\pgfsetfillcolor{currentfill}%
\pgfsetfillopacity{0.500000}%
\pgfsetlinewidth{1.003750pt}%
\definecolor{currentstroke}{rgb}{1.000000,0.411765,0.380392}%
\pgfsetstrokecolor{currentstroke}%
\pgfsetstrokeopacity{0.500000}%
\pgfsetdash{}{0pt}%
\pgfpathmoveto{\pgfqpoint{2.142652in}{0.868148in}}%
\pgfpathcurveto{\pgfqpoint{2.153702in}{0.868148in}}{\pgfqpoint{2.164301in}{0.872538in}}{\pgfqpoint{2.172115in}{0.880351in}}%
\pgfpathcurveto{\pgfqpoint{2.179928in}{0.888165in}}{\pgfqpoint{2.184319in}{0.898764in}}{\pgfqpoint{2.184319in}{0.909814in}}%
\pgfpathcurveto{\pgfqpoint{2.184319in}{0.920864in}}{\pgfqpoint{2.179928in}{0.931463in}}{\pgfqpoint{2.172115in}{0.939277in}}%
\pgfpathcurveto{\pgfqpoint{2.164301in}{0.947091in}}{\pgfqpoint{2.153702in}{0.951481in}}{\pgfqpoint{2.142652in}{0.951481in}}%
\pgfpathcurveto{\pgfqpoint{2.131602in}{0.951481in}}{\pgfqpoint{2.121003in}{0.947091in}}{\pgfqpoint{2.113189in}{0.939277in}}%
\pgfpathcurveto{\pgfqpoint{2.105376in}{0.931463in}}{\pgfqpoint{2.100985in}{0.920864in}}{\pgfqpoint{2.100985in}{0.909814in}}%
\pgfpathcurveto{\pgfqpoint{2.100985in}{0.898764in}}{\pgfqpoint{2.105376in}{0.888165in}}{\pgfqpoint{2.113189in}{0.880351in}}%
\pgfpathcurveto{\pgfqpoint{2.121003in}{0.872538in}}{\pgfqpoint{2.131602in}{0.868148in}}{\pgfqpoint{2.142652in}{0.868148in}}%
\pgfpathclose%
\pgfusepath{stroke,fill}%
\end{pgfscope}%
\begin{pgfscope}%
\pgfpathrectangle{\pgfqpoint{0.424692in}{0.370679in}}{\pgfqpoint{2.725308in}{1.479321in}}%
\pgfusepath{clip}%
\pgfsetbuttcap%
\pgfsetroundjoin%
\definecolor{currentfill}{rgb}{1.000000,0.411765,0.380392}%
\pgfsetfillcolor{currentfill}%
\pgfsetfillopacity{0.500000}%
\pgfsetlinewidth{1.003750pt}%
\definecolor{currentstroke}{rgb}{1.000000,0.411765,0.380392}%
\pgfsetstrokecolor{currentstroke}%
\pgfsetstrokeopacity{0.500000}%
\pgfsetdash{}{0pt}%
\pgfpathmoveto{\pgfqpoint{2.084874in}{0.948698in}}%
\pgfpathcurveto{\pgfqpoint{2.095924in}{0.948698in}}{\pgfqpoint{2.106523in}{0.953088in}}{\pgfqpoint{2.114337in}{0.960902in}}%
\pgfpathcurveto{\pgfqpoint{2.122150in}{0.968716in}}{\pgfqpoint{2.126541in}{0.979315in}}{\pgfqpoint{2.126541in}{0.990365in}}%
\pgfpathcurveto{\pgfqpoint{2.126541in}{1.001415in}}{\pgfqpoint{2.122150in}{1.012014in}}{\pgfqpoint{2.114337in}{1.019828in}}%
\pgfpathcurveto{\pgfqpoint{2.106523in}{1.027641in}}{\pgfqpoint{2.095924in}{1.032031in}}{\pgfqpoint{2.084874in}{1.032031in}}%
\pgfpathcurveto{\pgfqpoint{2.073824in}{1.032031in}}{\pgfqpoint{2.063225in}{1.027641in}}{\pgfqpoint{2.055411in}{1.019828in}}%
\pgfpathcurveto{\pgfqpoint{2.047597in}{1.012014in}}{\pgfqpoint{2.043207in}{1.001415in}}{\pgfqpoint{2.043207in}{0.990365in}}%
\pgfpathcurveto{\pgfqpoint{2.043207in}{0.979315in}}{\pgfqpoint{2.047597in}{0.968716in}}{\pgfqpoint{2.055411in}{0.960902in}}%
\pgfpathcurveto{\pgfqpoint{2.063225in}{0.953088in}}{\pgfqpoint{2.073824in}{0.948698in}}{\pgfqpoint{2.084874in}{0.948698in}}%
\pgfpathclose%
\pgfusepath{stroke,fill}%
\end{pgfscope}%
\begin{pgfscope}%
\pgfpathrectangle{\pgfqpoint{0.424692in}{0.370679in}}{\pgfqpoint{2.725308in}{1.479321in}}%
\pgfusepath{clip}%
\pgfsetbuttcap%
\pgfsetroundjoin%
\definecolor{currentfill}{rgb}{1.000000,0.411765,0.380392}%
\pgfsetfillcolor{currentfill}%
\pgfsetfillopacity{0.500000}%
\pgfsetlinewidth{1.003750pt}%
\definecolor{currentstroke}{rgb}{1.000000,0.411765,0.380392}%
\pgfsetstrokecolor{currentstroke}%
\pgfsetstrokeopacity{0.500000}%
\pgfsetdash{}{0pt}%
\pgfpathmoveto{\pgfqpoint{2.191291in}{1.000152in}}%
\pgfpathcurveto{\pgfqpoint{2.202341in}{1.000152in}}{\pgfqpoint{2.212940in}{1.004542in}}{\pgfqpoint{2.220754in}{1.012356in}}%
\pgfpathcurveto{\pgfqpoint{2.228567in}{1.020169in}}{\pgfqpoint{2.232958in}{1.030769in}}{\pgfqpoint{2.232958in}{1.041819in}}%
\pgfpathcurveto{\pgfqpoint{2.232958in}{1.052869in}}{\pgfqpoint{2.228567in}{1.063468in}}{\pgfqpoint{2.220754in}{1.071281in}}%
\pgfpathcurveto{\pgfqpoint{2.212940in}{1.079095in}}{\pgfqpoint{2.202341in}{1.083485in}}{\pgfqpoint{2.191291in}{1.083485in}}%
\pgfpathcurveto{\pgfqpoint{2.180241in}{1.083485in}}{\pgfqpoint{2.169642in}{1.079095in}}{\pgfqpoint{2.161828in}{1.071281in}}%
\pgfpathcurveto{\pgfqpoint{2.154014in}{1.063468in}}{\pgfqpoint{2.149624in}{1.052869in}}{\pgfqpoint{2.149624in}{1.041819in}}%
\pgfpathcurveto{\pgfqpoint{2.149624in}{1.030769in}}{\pgfqpoint{2.154014in}{1.020169in}}{\pgfqpoint{2.161828in}{1.012356in}}%
\pgfpathcurveto{\pgfqpoint{2.169642in}{1.004542in}}{\pgfqpoint{2.180241in}{1.000152in}}{\pgfqpoint{2.191291in}{1.000152in}}%
\pgfpathclose%
\pgfusepath{stroke,fill}%
\end{pgfscope}%
\begin{pgfscope}%
\pgfpathrectangle{\pgfqpoint{0.424692in}{0.370679in}}{\pgfqpoint{2.725308in}{1.479321in}}%
\pgfusepath{clip}%
\pgfsetbuttcap%
\pgfsetroundjoin%
\definecolor{currentfill}{rgb}{1.000000,0.411765,0.380392}%
\pgfsetfillcolor{currentfill}%
\pgfsetfillopacity{0.500000}%
\pgfsetlinewidth{1.003750pt}%
\definecolor{currentstroke}{rgb}{1.000000,0.411765,0.380392}%
\pgfsetstrokecolor{currentstroke}%
\pgfsetstrokeopacity{0.500000}%
\pgfsetdash{}{0pt}%
\pgfpathmoveto{\pgfqpoint{2.278845in}{1.029200in}}%
\pgfpathcurveto{\pgfqpoint{2.289895in}{1.029200in}}{\pgfqpoint{2.300494in}{1.033590in}}{\pgfqpoint{2.308308in}{1.041404in}}%
\pgfpathcurveto{\pgfqpoint{2.316121in}{1.049218in}}{\pgfqpoint{2.320512in}{1.059817in}}{\pgfqpoint{2.320512in}{1.070867in}}%
\pgfpathcurveto{\pgfqpoint{2.320512in}{1.081917in}}{\pgfqpoint{2.316121in}{1.092516in}}{\pgfqpoint{2.308308in}{1.100330in}}%
\pgfpathcurveto{\pgfqpoint{2.300494in}{1.108143in}}{\pgfqpoint{2.289895in}{1.112533in}}{\pgfqpoint{2.278845in}{1.112533in}}%
\pgfpathcurveto{\pgfqpoint{2.267795in}{1.112533in}}{\pgfqpoint{2.257196in}{1.108143in}}{\pgfqpoint{2.249382in}{1.100330in}}%
\pgfpathcurveto{\pgfqpoint{2.241569in}{1.092516in}}{\pgfqpoint{2.237178in}{1.081917in}}{\pgfqpoint{2.237178in}{1.070867in}}%
\pgfpathcurveto{\pgfqpoint{2.237178in}{1.059817in}}{\pgfqpoint{2.241569in}{1.049218in}}{\pgfqpoint{2.249382in}{1.041404in}}%
\pgfpathcurveto{\pgfqpoint{2.257196in}{1.033590in}}{\pgfqpoint{2.267795in}{1.029200in}}{\pgfqpoint{2.278845in}{1.029200in}}%
\pgfpathclose%
\pgfusepath{stroke,fill}%
\end{pgfscope}%
\begin{pgfscope}%
\pgfpathrectangle{\pgfqpoint{0.424692in}{0.370679in}}{\pgfqpoint{2.725308in}{1.479321in}}%
\pgfusepath{clip}%
\pgfsetbuttcap%
\pgfsetroundjoin%
\definecolor{currentfill}{rgb}{1.000000,0.411765,0.380392}%
\pgfsetfillcolor{currentfill}%
\pgfsetfillopacity{0.500000}%
\pgfsetlinewidth{1.003750pt}%
\definecolor{currentstroke}{rgb}{1.000000,0.411765,0.380392}%
\pgfsetstrokecolor{currentstroke}%
\pgfsetstrokeopacity{0.500000}%
\pgfsetdash{}{0pt}%
\pgfpathmoveto{\pgfqpoint{2.077171in}{0.924141in}}%
\pgfpathcurveto{\pgfqpoint{2.088221in}{0.924141in}}{\pgfqpoint{2.098820in}{0.928531in}}{\pgfqpoint{2.106634in}{0.936345in}}%
\pgfpathcurveto{\pgfqpoint{2.114448in}{0.944158in}}{\pgfqpoint{2.118838in}{0.954757in}}{\pgfqpoint{2.118838in}{0.965807in}}%
\pgfpathcurveto{\pgfqpoint{2.118838in}{0.976857in}}{\pgfqpoint{2.114448in}{0.987456in}}{\pgfqpoint{2.106634in}{0.995270in}}%
\pgfpathcurveto{\pgfqpoint{2.098820in}{1.003084in}}{\pgfqpoint{2.088221in}{1.007474in}}{\pgfqpoint{2.077171in}{1.007474in}}%
\pgfpathcurveto{\pgfqpoint{2.066121in}{1.007474in}}{\pgfqpoint{2.055522in}{1.003084in}}{\pgfqpoint{2.047708in}{0.995270in}}%
\pgfpathcurveto{\pgfqpoint{2.039895in}{0.987456in}}{\pgfqpoint{2.035504in}{0.976857in}}{\pgfqpoint{2.035504in}{0.965807in}}%
\pgfpathcurveto{\pgfqpoint{2.035504in}{0.954757in}}{\pgfqpoint{2.039895in}{0.944158in}}{\pgfqpoint{2.047708in}{0.936345in}}%
\pgfpathcurveto{\pgfqpoint{2.055522in}{0.928531in}}{\pgfqpoint{2.066121in}{0.924141in}}{\pgfqpoint{2.077171in}{0.924141in}}%
\pgfpathclose%
\pgfusepath{stroke,fill}%
\end{pgfscope}%
\begin{pgfscope}%
\pgfpathrectangle{\pgfqpoint{0.424692in}{0.370679in}}{\pgfqpoint{2.725308in}{1.479321in}}%
\pgfusepath{clip}%
\pgfsetbuttcap%
\pgfsetroundjoin%
\definecolor{currentfill}{rgb}{1.000000,0.411765,0.380392}%
\pgfsetfillcolor{currentfill}%
\pgfsetfillopacity{0.500000}%
\pgfsetlinewidth{1.003750pt}%
\definecolor{currentstroke}{rgb}{1.000000,0.411765,0.380392}%
\pgfsetstrokecolor{currentstroke}%
\pgfsetstrokeopacity{0.500000}%
\pgfsetdash{}{0pt}%
\pgfpathmoveto{\pgfqpoint{1.913478in}{0.823535in}}%
\pgfpathcurveto{\pgfqpoint{1.924528in}{0.823535in}}{\pgfqpoint{1.935127in}{0.827926in}}{\pgfqpoint{1.942941in}{0.835739in}}%
\pgfpathcurveto{\pgfqpoint{1.950755in}{0.843553in}}{\pgfqpoint{1.955145in}{0.854152in}}{\pgfqpoint{1.955145in}{0.865202in}}%
\pgfpathcurveto{\pgfqpoint{1.955145in}{0.876252in}}{\pgfqpoint{1.950755in}{0.886851in}}{\pgfqpoint{1.942941in}{0.894665in}}%
\pgfpathcurveto{\pgfqpoint{1.935127in}{0.902478in}}{\pgfqpoint{1.924528in}{0.906869in}}{\pgfqpoint{1.913478in}{0.906869in}}%
\pgfpathcurveto{\pgfqpoint{1.902428in}{0.906869in}}{\pgfqpoint{1.891829in}{0.902478in}}{\pgfqpoint{1.884016in}{0.894665in}}%
\pgfpathcurveto{\pgfqpoint{1.876202in}{0.886851in}}{\pgfqpoint{1.871812in}{0.876252in}}{\pgfqpoint{1.871812in}{0.865202in}}%
\pgfpathcurveto{\pgfqpoint{1.871812in}{0.854152in}}{\pgfqpoint{1.876202in}{0.843553in}}{\pgfqpoint{1.884016in}{0.835739in}}%
\pgfpathcurveto{\pgfqpoint{1.891829in}{0.827926in}}{\pgfqpoint{1.902428in}{0.823535in}}{\pgfqpoint{1.913478in}{0.823535in}}%
\pgfpathclose%
\pgfusepath{stroke,fill}%
\end{pgfscope}%
\begin{pgfscope}%
\pgfpathrectangle{\pgfqpoint{0.424692in}{0.370679in}}{\pgfqpoint{2.725308in}{1.479321in}}%
\pgfusepath{clip}%
\pgfsetbuttcap%
\pgfsetroundjoin%
\definecolor{currentfill}{rgb}{1.000000,0.411765,0.380392}%
\pgfsetfillcolor{currentfill}%
\pgfsetfillopacity{0.500000}%
\pgfsetlinewidth{1.003750pt}%
\definecolor{currentstroke}{rgb}{1.000000,0.411765,0.380392}%
\pgfsetstrokecolor{currentstroke}%
\pgfsetstrokeopacity{0.500000}%
\pgfsetdash{}{0pt}%
\pgfpathmoveto{\pgfqpoint{1.960622in}{0.839708in}}%
\pgfpathcurveto{\pgfqpoint{1.971673in}{0.839708in}}{\pgfqpoint{1.982272in}{0.844098in}}{\pgfqpoint{1.990085in}{0.851912in}}%
\pgfpathcurveto{\pgfqpoint{1.997899in}{0.859726in}}{\pgfqpoint{2.002289in}{0.870325in}}{\pgfqpoint{2.002289in}{0.881375in}}%
\pgfpathcurveto{\pgfqpoint{2.002289in}{0.892425in}}{\pgfqpoint{1.997899in}{0.903024in}}{\pgfqpoint{1.990085in}{0.910837in}}%
\pgfpathcurveto{\pgfqpoint{1.982272in}{0.918651in}}{\pgfqpoint{1.971673in}{0.923041in}}{\pgfqpoint{1.960622in}{0.923041in}}%
\pgfpathcurveto{\pgfqpoint{1.949572in}{0.923041in}}{\pgfqpoint{1.938973in}{0.918651in}}{\pgfqpoint{1.931160in}{0.910837in}}%
\pgfpathcurveto{\pgfqpoint{1.923346in}{0.903024in}}{\pgfqpoint{1.918956in}{0.892425in}}{\pgfqpoint{1.918956in}{0.881375in}}%
\pgfpathcurveto{\pgfqpoint{1.918956in}{0.870325in}}{\pgfqpoint{1.923346in}{0.859726in}}{\pgfqpoint{1.931160in}{0.851912in}}%
\pgfpathcurveto{\pgfqpoint{1.938973in}{0.844098in}}{\pgfqpoint{1.949572in}{0.839708in}}{\pgfqpoint{1.960622in}{0.839708in}}%
\pgfpathclose%
\pgfusepath{stroke,fill}%
\end{pgfscope}%
\begin{pgfscope}%
\pgfpathrectangle{\pgfqpoint{0.424692in}{0.370679in}}{\pgfqpoint{2.725308in}{1.479321in}}%
\pgfusepath{clip}%
\pgfsetbuttcap%
\pgfsetroundjoin%
\definecolor{currentfill}{rgb}{1.000000,0.411765,0.380392}%
\pgfsetfillcolor{currentfill}%
\pgfsetfillopacity{0.500000}%
\pgfsetlinewidth{1.003750pt}%
\definecolor{currentstroke}{rgb}{1.000000,0.411765,0.380392}%
\pgfsetstrokecolor{currentstroke}%
\pgfsetstrokeopacity{0.500000}%
\pgfsetdash{}{0pt}%
\pgfpathmoveto{\pgfqpoint{2.327006in}{1.040094in}}%
\pgfpathcurveto{\pgfqpoint{2.338056in}{1.040094in}}{\pgfqpoint{2.348655in}{1.044484in}}{\pgfqpoint{2.356468in}{1.052298in}}%
\pgfpathcurveto{\pgfqpoint{2.364282in}{1.060112in}}{\pgfqpoint{2.368672in}{1.070711in}}{\pgfqpoint{2.368672in}{1.081761in}}%
\pgfpathcurveto{\pgfqpoint{2.368672in}{1.092811in}}{\pgfqpoint{2.364282in}{1.103410in}}{\pgfqpoint{2.356468in}{1.111224in}}%
\pgfpathcurveto{\pgfqpoint{2.348655in}{1.119037in}}{\pgfqpoint{2.338056in}{1.123428in}}{\pgfqpoint{2.327006in}{1.123428in}}%
\pgfpathcurveto{\pgfqpoint{2.315955in}{1.123428in}}{\pgfqpoint{2.305356in}{1.119037in}}{\pgfqpoint{2.297543in}{1.111224in}}%
\pgfpathcurveto{\pgfqpoint{2.289729in}{1.103410in}}{\pgfqpoint{2.285339in}{1.092811in}}{\pgfqpoint{2.285339in}{1.081761in}}%
\pgfpathcurveto{\pgfqpoint{2.285339in}{1.070711in}}{\pgfqpoint{2.289729in}{1.060112in}}{\pgfqpoint{2.297543in}{1.052298in}}%
\pgfpathcurveto{\pgfqpoint{2.305356in}{1.044484in}}{\pgfqpoint{2.315955in}{1.040094in}}{\pgfqpoint{2.327006in}{1.040094in}}%
\pgfpathclose%
\pgfusepath{stroke,fill}%
\end{pgfscope}%
\begin{pgfscope}%
\pgfpathrectangle{\pgfqpoint{0.424692in}{0.370679in}}{\pgfqpoint{2.725308in}{1.479321in}}%
\pgfusepath{clip}%
\pgfsetbuttcap%
\pgfsetroundjoin%
\definecolor{currentfill}{rgb}{1.000000,0.411765,0.380392}%
\pgfsetfillcolor{currentfill}%
\pgfsetfillopacity{0.500000}%
\pgfsetlinewidth{1.003750pt}%
\definecolor{currentstroke}{rgb}{1.000000,0.411765,0.380392}%
\pgfsetstrokecolor{currentstroke}%
\pgfsetstrokeopacity{0.500000}%
\pgfsetdash{}{0pt}%
\pgfpathmoveto{\pgfqpoint{2.328005in}{1.046086in}}%
\pgfpathcurveto{\pgfqpoint{2.339055in}{1.046086in}}{\pgfqpoint{2.349654in}{1.050476in}}{\pgfqpoint{2.357467in}{1.058289in}}%
\pgfpathcurveto{\pgfqpoint{2.365281in}{1.066103in}}{\pgfqpoint{2.369671in}{1.076702in}}{\pgfqpoint{2.369671in}{1.087752in}}%
\pgfpathcurveto{\pgfqpoint{2.369671in}{1.098802in}}{\pgfqpoint{2.365281in}{1.109401in}}{\pgfqpoint{2.357467in}{1.117215in}}%
\pgfpathcurveto{\pgfqpoint{2.349654in}{1.125029in}}{\pgfqpoint{2.339055in}{1.129419in}}{\pgfqpoint{2.328005in}{1.129419in}}%
\pgfpathcurveto{\pgfqpoint{2.316954in}{1.129419in}}{\pgfqpoint{2.306355in}{1.125029in}}{\pgfqpoint{2.298542in}{1.117215in}}%
\pgfpathcurveto{\pgfqpoint{2.290728in}{1.109401in}}{\pgfqpoint{2.286338in}{1.098802in}}{\pgfqpoint{2.286338in}{1.087752in}}%
\pgfpathcurveto{\pgfqpoint{2.286338in}{1.076702in}}{\pgfqpoint{2.290728in}{1.066103in}}{\pgfqpoint{2.298542in}{1.058289in}}%
\pgfpathcurveto{\pgfqpoint{2.306355in}{1.050476in}}{\pgfqpoint{2.316954in}{1.046086in}}{\pgfqpoint{2.328005in}{1.046086in}}%
\pgfpathclose%
\pgfusepath{stroke,fill}%
\end{pgfscope}%
\begin{pgfscope}%
\pgfpathrectangle{\pgfqpoint{0.424692in}{0.370679in}}{\pgfqpoint{2.725308in}{1.479321in}}%
\pgfusepath{clip}%
\pgfsetbuttcap%
\pgfsetroundjoin%
\definecolor{currentfill}{rgb}{1.000000,0.411765,0.380392}%
\pgfsetfillcolor{currentfill}%
\pgfsetfillopacity{0.500000}%
\pgfsetlinewidth{1.003750pt}%
\definecolor{currentstroke}{rgb}{1.000000,0.411765,0.380392}%
\pgfsetstrokecolor{currentstroke}%
\pgfsetstrokeopacity{0.500000}%
\pgfsetdash{}{0pt}%
\pgfpathmoveto{\pgfqpoint{1.868025in}{0.817474in}}%
\pgfpathcurveto{\pgfqpoint{1.879076in}{0.817474in}}{\pgfqpoint{1.889675in}{0.821864in}}{\pgfqpoint{1.897488in}{0.829678in}}%
\pgfpathcurveto{\pgfqpoint{1.905302in}{0.837492in}}{\pgfqpoint{1.909692in}{0.848091in}}{\pgfqpoint{1.909692in}{0.859141in}}%
\pgfpathcurveto{\pgfqpoint{1.909692in}{0.870191in}}{\pgfqpoint{1.905302in}{0.880790in}}{\pgfqpoint{1.897488in}{0.888604in}}%
\pgfpathcurveto{\pgfqpoint{1.889675in}{0.896417in}}{\pgfqpoint{1.879076in}{0.900808in}}{\pgfqpoint{1.868025in}{0.900808in}}%
\pgfpathcurveto{\pgfqpoint{1.856975in}{0.900808in}}{\pgfqpoint{1.846376in}{0.896417in}}{\pgfqpoint{1.838563in}{0.888604in}}%
\pgfpathcurveto{\pgfqpoint{1.830749in}{0.880790in}}{\pgfqpoint{1.826359in}{0.870191in}}{\pgfqpoint{1.826359in}{0.859141in}}%
\pgfpathcurveto{\pgfqpoint{1.826359in}{0.848091in}}{\pgfqpoint{1.830749in}{0.837492in}}{\pgfqpoint{1.838563in}{0.829678in}}%
\pgfpathcurveto{\pgfqpoint{1.846376in}{0.821864in}}{\pgfqpoint{1.856975in}{0.817474in}}{\pgfqpoint{1.868025in}{0.817474in}}%
\pgfpathclose%
\pgfusepath{stroke,fill}%
\end{pgfscope}%
\begin{pgfscope}%
\pgfpathrectangle{\pgfqpoint{0.424692in}{0.370679in}}{\pgfqpoint{2.725308in}{1.479321in}}%
\pgfusepath{clip}%
\pgfsetbuttcap%
\pgfsetroundjoin%
\definecolor{currentfill}{rgb}{1.000000,0.411765,0.380392}%
\pgfsetfillcolor{currentfill}%
\pgfsetfillopacity{0.500000}%
\pgfsetlinewidth{1.003750pt}%
\definecolor{currentstroke}{rgb}{1.000000,0.411765,0.380392}%
\pgfsetstrokecolor{currentstroke}%
\pgfsetstrokeopacity{0.500000}%
\pgfsetdash{}{0pt}%
\pgfpathmoveto{\pgfqpoint{1.599397in}{0.642021in}}%
\pgfpathcurveto{\pgfqpoint{1.610448in}{0.642021in}}{\pgfqpoint{1.621047in}{0.646411in}}{\pgfqpoint{1.628860in}{0.654225in}}%
\pgfpathcurveto{\pgfqpoint{1.636674in}{0.662039in}}{\pgfqpoint{1.641064in}{0.672638in}}{\pgfqpoint{1.641064in}{0.683688in}}%
\pgfpathcurveto{\pgfqpoint{1.641064in}{0.694738in}}{\pgfqpoint{1.636674in}{0.705337in}}{\pgfqpoint{1.628860in}{0.713150in}}%
\pgfpathcurveto{\pgfqpoint{1.621047in}{0.720964in}}{\pgfqpoint{1.610448in}{0.725354in}}{\pgfqpoint{1.599397in}{0.725354in}}%
\pgfpathcurveto{\pgfqpoint{1.588347in}{0.725354in}}{\pgfqpoint{1.577748in}{0.720964in}}{\pgfqpoint{1.569935in}{0.713150in}}%
\pgfpathcurveto{\pgfqpoint{1.562121in}{0.705337in}}{\pgfqpoint{1.557731in}{0.694738in}}{\pgfqpoint{1.557731in}{0.683688in}}%
\pgfpathcurveto{\pgfqpoint{1.557731in}{0.672638in}}{\pgfqpoint{1.562121in}{0.662039in}}{\pgfqpoint{1.569935in}{0.654225in}}%
\pgfpathcurveto{\pgfqpoint{1.577748in}{0.646411in}}{\pgfqpoint{1.588347in}{0.642021in}}{\pgfqpoint{1.599397in}{0.642021in}}%
\pgfpathclose%
\pgfusepath{stroke,fill}%
\end{pgfscope}%
\begin{pgfscope}%
\pgfpathrectangle{\pgfqpoint{0.424692in}{0.370679in}}{\pgfqpoint{2.725308in}{1.479321in}}%
\pgfusepath{clip}%
\pgfsetbuttcap%
\pgfsetroundjoin%
\definecolor{currentfill}{rgb}{1.000000,0.411765,0.380392}%
\pgfsetfillcolor{currentfill}%
\pgfsetfillopacity{0.500000}%
\pgfsetlinewidth{1.003750pt}%
\definecolor{currentstroke}{rgb}{1.000000,0.411765,0.380392}%
\pgfsetstrokecolor{currentstroke}%
\pgfsetstrokeopacity{0.500000}%
\pgfsetdash{}{0pt}%
\pgfpathmoveto{\pgfqpoint{1.393438in}{0.536052in}}%
\pgfpathcurveto{\pgfqpoint{1.404488in}{0.536052in}}{\pgfqpoint{1.415087in}{0.540443in}}{\pgfqpoint{1.422900in}{0.548256in}}%
\pgfpathcurveto{\pgfqpoint{1.430714in}{0.556070in}}{\pgfqpoint{1.435104in}{0.566669in}}{\pgfqpoint{1.435104in}{0.577719in}}%
\pgfpathcurveto{\pgfqpoint{1.435104in}{0.588769in}}{\pgfqpoint{1.430714in}{0.599368in}}{\pgfqpoint{1.422900in}{0.607182in}}%
\pgfpathcurveto{\pgfqpoint{1.415087in}{0.614995in}}{\pgfqpoint{1.404488in}{0.619386in}}{\pgfqpoint{1.393438in}{0.619386in}}%
\pgfpathcurveto{\pgfqpoint{1.382388in}{0.619386in}}{\pgfqpoint{1.371789in}{0.614995in}}{\pgfqpoint{1.363975in}{0.607182in}}%
\pgfpathcurveto{\pgfqpoint{1.356161in}{0.599368in}}{\pgfqpoint{1.351771in}{0.588769in}}{\pgfqpoint{1.351771in}{0.577719in}}%
\pgfpathcurveto{\pgfqpoint{1.351771in}{0.566669in}}{\pgfqpoint{1.356161in}{0.556070in}}{\pgfqpoint{1.363975in}{0.548256in}}%
\pgfpathcurveto{\pgfqpoint{1.371789in}{0.540443in}}{\pgfqpoint{1.382388in}{0.536052in}}{\pgfqpoint{1.393438in}{0.536052in}}%
\pgfpathclose%
\pgfusepath{stroke,fill}%
\end{pgfscope}%
\begin{pgfscope}%
\pgfpathrectangle{\pgfqpoint{0.424692in}{0.370679in}}{\pgfqpoint{2.725308in}{1.479321in}}%
\pgfusepath{clip}%
\pgfsetbuttcap%
\pgfsetroundjoin%
\definecolor{currentfill}{rgb}{1.000000,0.411765,0.380392}%
\pgfsetfillcolor{currentfill}%
\pgfsetfillopacity{0.500000}%
\pgfsetlinewidth{1.003750pt}%
\definecolor{currentstroke}{rgb}{1.000000,0.411765,0.380392}%
\pgfsetstrokecolor{currentstroke}%
\pgfsetstrokeopacity{0.500000}%
\pgfsetdash{}{0pt}%
\pgfpathmoveto{\pgfqpoint{2.270136in}{1.004092in}}%
\pgfpathcurveto{\pgfqpoint{2.281186in}{1.004092in}}{\pgfqpoint{2.291785in}{1.008482in}}{\pgfqpoint{2.299599in}{1.016295in}}%
\pgfpathcurveto{\pgfqpoint{2.307413in}{1.024109in}}{\pgfqpoint{2.311803in}{1.034708in}}{\pgfqpoint{2.311803in}{1.045758in}}%
\pgfpathcurveto{\pgfqpoint{2.311803in}{1.056808in}}{\pgfqpoint{2.307413in}{1.067407in}}{\pgfqpoint{2.299599in}{1.075221in}}%
\pgfpathcurveto{\pgfqpoint{2.291785in}{1.083035in}}{\pgfqpoint{2.281186in}{1.087425in}}{\pgfqpoint{2.270136in}{1.087425in}}%
\pgfpathcurveto{\pgfqpoint{2.259086in}{1.087425in}}{\pgfqpoint{2.248487in}{1.083035in}}{\pgfqpoint{2.240673in}{1.075221in}}%
\pgfpathcurveto{\pgfqpoint{2.232860in}{1.067407in}}{\pgfqpoint{2.228470in}{1.056808in}}{\pgfqpoint{2.228470in}{1.045758in}}%
\pgfpathcurveto{\pgfqpoint{2.228470in}{1.034708in}}{\pgfqpoint{2.232860in}{1.024109in}}{\pgfqpoint{2.240673in}{1.016295in}}%
\pgfpathcurveto{\pgfqpoint{2.248487in}{1.008482in}}{\pgfqpoint{2.259086in}{1.004092in}}{\pgfqpoint{2.270136in}{1.004092in}}%
\pgfpathclose%
\pgfusepath{stroke,fill}%
\end{pgfscope}%
\begin{pgfscope}%
\pgfpathrectangle{\pgfqpoint{0.424692in}{0.370679in}}{\pgfqpoint{2.725308in}{1.479321in}}%
\pgfusepath{clip}%
\pgfsetbuttcap%
\pgfsetroundjoin%
\definecolor{currentfill}{rgb}{1.000000,0.411765,0.380392}%
\pgfsetfillcolor{currentfill}%
\pgfsetfillopacity{0.500000}%
\pgfsetlinewidth{1.003750pt}%
\definecolor{currentstroke}{rgb}{1.000000,0.411765,0.380392}%
\pgfsetstrokecolor{currentstroke}%
\pgfsetstrokeopacity{0.500000}%
\pgfsetdash{}{0pt}%
\pgfpathmoveto{\pgfqpoint{2.157080in}{0.918821in}}%
\pgfpathcurveto{\pgfqpoint{2.168130in}{0.918821in}}{\pgfqpoint{2.178729in}{0.923212in}}{\pgfqpoint{2.186543in}{0.931025in}}%
\pgfpathcurveto{\pgfqpoint{2.194357in}{0.938839in}}{\pgfqpoint{2.198747in}{0.949438in}}{\pgfqpoint{2.198747in}{0.960488in}}%
\pgfpathcurveto{\pgfqpoint{2.198747in}{0.971538in}}{\pgfqpoint{2.194357in}{0.982137in}}{\pgfqpoint{2.186543in}{0.989951in}}%
\pgfpathcurveto{\pgfqpoint{2.178729in}{0.997765in}}{\pgfqpoint{2.168130in}{1.002155in}}{\pgfqpoint{2.157080in}{1.002155in}}%
\pgfpathcurveto{\pgfqpoint{2.146030in}{1.002155in}}{\pgfqpoint{2.135431in}{0.997765in}}{\pgfqpoint{2.127617in}{0.989951in}}%
\pgfpathcurveto{\pgfqpoint{2.119804in}{0.982137in}}{\pgfqpoint{2.115414in}{0.971538in}}{\pgfqpoint{2.115414in}{0.960488in}}%
\pgfpathcurveto{\pgfqpoint{2.115414in}{0.949438in}}{\pgfqpoint{2.119804in}{0.938839in}}{\pgfqpoint{2.127617in}{0.931025in}}%
\pgfpathcurveto{\pgfqpoint{2.135431in}{0.923212in}}{\pgfqpoint{2.146030in}{0.918821in}}{\pgfqpoint{2.157080in}{0.918821in}}%
\pgfpathclose%
\pgfusepath{stroke,fill}%
\end{pgfscope}%
\begin{pgfscope}%
\pgfpathrectangle{\pgfqpoint{0.424692in}{0.370679in}}{\pgfqpoint{2.725308in}{1.479321in}}%
\pgfusepath{clip}%
\pgfsetbuttcap%
\pgfsetroundjoin%
\definecolor{currentfill}{rgb}{1.000000,0.411765,0.380392}%
\pgfsetfillcolor{currentfill}%
\pgfsetfillopacity{0.500000}%
\pgfsetlinewidth{1.003750pt}%
\definecolor{currentstroke}{rgb}{1.000000,0.411765,0.380392}%
\pgfsetstrokecolor{currentstroke}%
\pgfsetstrokeopacity{0.500000}%
\pgfsetdash{}{0pt}%
\pgfpathmoveto{\pgfqpoint{1.750283in}{0.726943in}}%
\pgfpathcurveto{\pgfqpoint{1.761333in}{0.726943in}}{\pgfqpoint{1.771932in}{0.731334in}}{\pgfqpoint{1.779746in}{0.739147in}}%
\pgfpathcurveto{\pgfqpoint{1.787559in}{0.746961in}}{\pgfqpoint{1.791950in}{0.757560in}}{\pgfqpoint{1.791950in}{0.768610in}}%
\pgfpathcurveto{\pgfqpoint{1.791950in}{0.779660in}}{\pgfqpoint{1.787559in}{0.790259in}}{\pgfqpoint{1.779746in}{0.798073in}}%
\pgfpathcurveto{\pgfqpoint{1.771932in}{0.805887in}}{\pgfqpoint{1.761333in}{0.810277in}}{\pgfqpoint{1.750283in}{0.810277in}}%
\pgfpathcurveto{\pgfqpoint{1.739233in}{0.810277in}}{\pgfqpoint{1.728634in}{0.805887in}}{\pgfqpoint{1.720820in}{0.798073in}}%
\pgfpathcurveto{\pgfqpoint{1.713006in}{0.790259in}}{\pgfqpoint{1.708616in}{0.779660in}}{\pgfqpoint{1.708616in}{0.768610in}}%
\pgfpathcurveto{\pgfqpoint{1.708616in}{0.757560in}}{\pgfqpoint{1.713006in}{0.746961in}}{\pgfqpoint{1.720820in}{0.739147in}}%
\pgfpathcurveto{\pgfqpoint{1.728634in}{0.731334in}}{\pgfqpoint{1.739233in}{0.726943in}}{\pgfqpoint{1.750283in}{0.726943in}}%
\pgfpathclose%
\pgfusepath{stroke,fill}%
\end{pgfscope}%
\begin{pgfscope}%
\pgfpathrectangle{\pgfqpoint{0.424692in}{0.370679in}}{\pgfqpoint{2.725308in}{1.479321in}}%
\pgfusepath{clip}%
\pgfsetbuttcap%
\pgfsetroundjoin%
\definecolor{currentfill}{rgb}{1.000000,0.411765,0.380392}%
\pgfsetfillcolor{currentfill}%
\pgfsetfillopacity{0.500000}%
\pgfsetlinewidth{1.003750pt}%
\definecolor{currentstroke}{rgb}{1.000000,0.411765,0.380392}%
\pgfsetstrokecolor{currentstroke}%
\pgfsetstrokeopacity{0.500000}%
\pgfsetdash{}{0pt}%
\pgfpathmoveto{\pgfqpoint{2.268315in}{1.003094in}}%
\pgfpathcurveto{\pgfqpoint{2.279366in}{1.003094in}}{\pgfqpoint{2.289965in}{1.007485in}}{\pgfqpoint{2.297778in}{1.015298in}}%
\pgfpathcurveto{\pgfqpoint{2.305592in}{1.023112in}}{\pgfqpoint{2.309982in}{1.033711in}}{\pgfqpoint{2.309982in}{1.044761in}}%
\pgfpathcurveto{\pgfqpoint{2.309982in}{1.055811in}}{\pgfqpoint{2.305592in}{1.066410in}}{\pgfqpoint{2.297778in}{1.074224in}}%
\pgfpathcurveto{\pgfqpoint{2.289965in}{1.082037in}}{\pgfqpoint{2.279366in}{1.086428in}}{\pgfqpoint{2.268315in}{1.086428in}}%
\pgfpathcurveto{\pgfqpoint{2.257265in}{1.086428in}}{\pgfqpoint{2.246666in}{1.082037in}}{\pgfqpoint{2.238853in}{1.074224in}}%
\pgfpathcurveto{\pgfqpoint{2.231039in}{1.066410in}}{\pgfqpoint{2.226649in}{1.055811in}}{\pgfqpoint{2.226649in}{1.044761in}}%
\pgfpathcurveto{\pgfqpoint{2.226649in}{1.033711in}}{\pgfqpoint{2.231039in}{1.023112in}}{\pgfqpoint{2.238853in}{1.015298in}}%
\pgfpathcurveto{\pgfqpoint{2.246666in}{1.007485in}}{\pgfqpoint{2.257265in}{1.003094in}}{\pgfqpoint{2.268315in}{1.003094in}}%
\pgfpathclose%
\pgfusepath{stroke,fill}%
\end{pgfscope}%
\begin{pgfscope}%
\pgfpathrectangle{\pgfqpoint{0.424692in}{0.370679in}}{\pgfqpoint{2.725308in}{1.479321in}}%
\pgfusepath{clip}%
\pgfsetbuttcap%
\pgfsetroundjoin%
\definecolor{currentfill}{rgb}{1.000000,0.411765,0.380392}%
\pgfsetfillcolor{currentfill}%
\pgfsetfillopacity{0.500000}%
\pgfsetlinewidth{1.003750pt}%
\definecolor{currentstroke}{rgb}{1.000000,0.411765,0.380392}%
\pgfsetstrokecolor{currentstroke}%
\pgfsetstrokeopacity{0.500000}%
\pgfsetdash{}{0pt}%
\pgfpathmoveto{\pgfqpoint{2.568105in}{1.128566in}}%
\pgfpathcurveto{\pgfqpoint{2.579155in}{1.128566in}}{\pgfqpoint{2.589754in}{1.132956in}}{\pgfqpoint{2.597568in}{1.140770in}}%
\pgfpathcurveto{\pgfqpoint{2.605382in}{1.148583in}}{\pgfqpoint{2.609772in}{1.159182in}}{\pgfqpoint{2.609772in}{1.170232in}}%
\pgfpathcurveto{\pgfqpoint{2.609772in}{1.181282in}}{\pgfqpoint{2.605382in}{1.191882in}}{\pgfqpoint{2.597568in}{1.199695in}}%
\pgfpathcurveto{\pgfqpoint{2.589754in}{1.207509in}}{\pgfqpoint{2.579155in}{1.211899in}}{\pgfqpoint{2.568105in}{1.211899in}}%
\pgfpathcurveto{\pgfqpoint{2.557055in}{1.211899in}}{\pgfqpoint{2.546456in}{1.207509in}}{\pgfqpoint{2.538642in}{1.199695in}}%
\pgfpathcurveto{\pgfqpoint{2.530829in}{1.191882in}}{\pgfqpoint{2.526439in}{1.181282in}}{\pgfqpoint{2.526439in}{1.170232in}}%
\pgfpathcurveto{\pgfqpoint{2.526439in}{1.159182in}}{\pgfqpoint{2.530829in}{1.148583in}}{\pgfqpoint{2.538642in}{1.140770in}}%
\pgfpathcurveto{\pgfqpoint{2.546456in}{1.132956in}}{\pgfqpoint{2.557055in}{1.128566in}}{\pgfqpoint{2.568105in}{1.128566in}}%
\pgfpathclose%
\pgfusepath{stroke,fill}%
\end{pgfscope}%
\begin{pgfscope}%
\pgfpathrectangle{\pgfqpoint{0.424692in}{0.370679in}}{\pgfqpoint{2.725308in}{1.479321in}}%
\pgfusepath{clip}%
\pgfsetbuttcap%
\pgfsetroundjoin%
\definecolor{currentfill}{rgb}{1.000000,0.411765,0.380392}%
\pgfsetfillcolor{currentfill}%
\pgfsetfillopacity{0.500000}%
\pgfsetlinewidth{1.003750pt}%
\definecolor{currentstroke}{rgb}{1.000000,0.411765,0.380392}%
\pgfsetstrokecolor{currentstroke}%
\pgfsetstrokeopacity{0.500000}%
\pgfsetdash{}{0pt}%
\pgfpathmoveto{\pgfqpoint{2.014794in}{0.875280in}}%
\pgfpathcurveto{\pgfqpoint{2.025844in}{0.875280in}}{\pgfqpoint{2.036443in}{0.879670in}}{\pgfqpoint{2.044257in}{0.887484in}}%
\pgfpathcurveto{\pgfqpoint{2.052070in}{0.895298in}}{\pgfqpoint{2.056461in}{0.905897in}}{\pgfqpoint{2.056461in}{0.916947in}}%
\pgfpathcurveto{\pgfqpoint{2.056461in}{0.927997in}}{\pgfqpoint{2.052070in}{0.938596in}}{\pgfqpoint{2.044257in}{0.946410in}}%
\pgfpathcurveto{\pgfqpoint{2.036443in}{0.954223in}}{\pgfqpoint{2.025844in}{0.958614in}}{\pgfqpoint{2.014794in}{0.958614in}}%
\pgfpathcurveto{\pgfqpoint{2.003744in}{0.958614in}}{\pgfqpoint{1.993145in}{0.954223in}}{\pgfqpoint{1.985331in}{0.946410in}}%
\pgfpathcurveto{\pgfqpoint{1.977518in}{0.938596in}}{\pgfqpoint{1.973127in}{0.927997in}}{\pgfqpoint{1.973127in}{0.916947in}}%
\pgfpathcurveto{\pgfqpoint{1.973127in}{0.905897in}}{\pgfqpoint{1.977518in}{0.895298in}}{\pgfqpoint{1.985331in}{0.887484in}}%
\pgfpathcurveto{\pgfqpoint{1.993145in}{0.879670in}}{\pgfqpoint{2.003744in}{0.875280in}}{\pgfqpoint{2.014794in}{0.875280in}}%
\pgfpathclose%
\pgfusepath{stroke,fill}%
\end{pgfscope}%
\begin{pgfscope}%
\pgfpathrectangle{\pgfqpoint{0.424692in}{0.370679in}}{\pgfqpoint{2.725308in}{1.479321in}}%
\pgfusepath{clip}%
\pgfsetbuttcap%
\pgfsetroundjoin%
\definecolor{currentfill}{rgb}{1.000000,0.411765,0.380392}%
\pgfsetfillcolor{currentfill}%
\pgfsetfillopacity{0.500000}%
\pgfsetlinewidth{1.003750pt}%
\definecolor{currentstroke}{rgb}{1.000000,0.411765,0.380392}%
\pgfsetstrokecolor{currentstroke}%
\pgfsetstrokeopacity{0.500000}%
\pgfsetdash{}{0pt}%
\pgfpathmoveto{\pgfqpoint{2.006564in}{0.848731in}}%
\pgfpathcurveto{\pgfqpoint{2.017614in}{0.848731in}}{\pgfqpoint{2.028213in}{0.853121in}}{\pgfqpoint{2.036027in}{0.860934in}}%
\pgfpathcurveto{\pgfqpoint{2.043840in}{0.868748in}}{\pgfqpoint{2.048230in}{0.879347in}}{\pgfqpoint{2.048230in}{0.890397in}}%
\pgfpathcurveto{\pgfqpoint{2.048230in}{0.901447in}}{\pgfqpoint{2.043840in}{0.912046in}}{\pgfqpoint{2.036027in}{0.919860in}}%
\pgfpathcurveto{\pgfqpoint{2.028213in}{0.927674in}}{\pgfqpoint{2.017614in}{0.932064in}}{\pgfqpoint{2.006564in}{0.932064in}}%
\pgfpathcurveto{\pgfqpoint{1.995514in}{0.932064in}}{\pgfqpoint{1.984915in}{0.927674in}}{\pgfqpoint{1.977101in}{0.919860in}}%
\pgfpathcurveto{\pgfqpoint{1.969287in}{0.912046in}}{\pgfqpoint{1.964897in}{0.901447in}}{\pgfqpoint{1.964897in}{0.890397in}}%
\pgfpathcurveto{\pgfqpoint{1.964897in}{0.879347in}}{\pgfqpoint{1.969287in}{0.868748in}}{\pgfqpoint{1.977101in}{0.860934in}}%
\pgfpathcurveto{\pgfqpoint{1.984915in}{0.853121in}}{\pgfqpoint{1.995514in}{0.848731in}}{\pgfqpoint{2.006564in}{0.848731in}}%
\pgfpathclose%
\pgfusepath{stroke,fill}%
\end{pgfscope}%
\begin{pgfscope}%
\pgfpathrectangle{\pgfqpoint{0.424692in}{0.370679in}}{\pgfqpoint{2.725308in}{1.479321in}}%
\pgfusepath{clip}%
\pgfsetbuttcap%
\pgfsetroundjoin%
\definecolor{currentfill}{rgb}{1.000000,0.411765,0.380392}%
\pgfsetfillcolor{currentfill}%
\pgfsetfillopacity{0.500000}%
\pgfsetlinewidth{1.003750pt}%
\definecolor{currentstroke}{rgb}{1.000000,0.411765,0.380392}%
\pgfsetstrokecolor{currentstroke}%
\pgfsetstrokeopacity{0.500000}%
\pgfsetdash{}{0pt}%
\pgfpathmoveto{\pgfqpoint{2.882384in}{1.315633in}}%
\pgfpathcurveto{\pgfqpoint{2.893434in}{1.315633in}}{\pgfqpoint{2.904033in}{1.320023in}}{\pgfqpoint{2.911847in}{1.327837in}}%
\pgfpathcurveto{\pgfqpoint{2.919660in}{1.335650in}}{\pgfqpoint{2.924051in}{1.346249in}}{\pgfqpoint{2.924051in}{1.357299in}}%
\pgfpathcurveto{\pgfqpoint{2.924051in}{1.368349in}}{\pgfqpoint{2.919660in}{1.378948in}}{\pgfqpoint{2.911847in}{1.386762in}}%
\pgfpathcurveto{\pgfqpoint{2.904033in}{1.394576in}}{\pgfqpoint{2.893434in}{1.398966in}}{\pgfqpoint{2.882384in}{1.398966in}}%
\pgfpathcurveto{\pgfqpoint{2.871334in}{1.398966in}}{\pgfqpoint{2.860735in}{1.394576in}}{\pgfqpoint{2.852921in}{1.386762in}}%
\pgfpathcurveto{\pgfqpoint{2.845108in}{1.378948in}}{\pgfqpoint{2.840717in}{1.368349in}}{\pgfqpoint{2.840717in}{1.357299in}}%
\pgfpathcurveto{\pgfqpoint{2.840717in}{1.346249in}}{\pgfqpoint{2.845108in}{1.335650in}}{\pgfqpoint{2.852921in}{1.327837in}}%
\pgfpathcurveto{\pgfqpoint{2.860735in}{1.320023in}}{\pgfqpoint{2.871334in}{1.315633in}}{\pgfqpoint{2.882384in}{1.315633in}}%
\pgfpathclose%
\pgfusepath{stroke,fill}%
\end{pgfscope}%
\begin{pgfscope}%
\pgfpathrectangle{\pgfqpoint{0.424692in}{0.370679in}}{\pgfqpoint{2.725308in}{1.479321in}}%
\pgfusepath{clip}%
\pgfsetbuttcap%
\pgfsetroundjoin%
\definecolor{currentfill}{rgb}{1.000000,0.411765,0.380392}%
\pgfsetfillcolor{currentfill}%
\pgfsetfillopacity{0.500000}%
\pgfsetlinewidth{1.003750pt}%
\definecolor{currentstroke}{rgb}{1.000000,0.411765,0.380392}%
\pgfsetstrokecolor{currentstroke}%
\pgfsetstrokeopacity{0.500000}%
\pgfsetdash{}{0pt}%
\pgfpathmoveto{\pgfqpoint{2.466261in}{1.071087in}}%
\pgfpathcurveto{\pgfqpoint{2.477311in}{1.071087in}}{\pgfqpoint{2.487910in}{1.075477in}}{\pgfqpoint{2.495724in}{1.083291in}}%
\pgfpathcurveto{\pgfqpoint{2.503537in}{1.091105in}}{\pgfqpoint{2.507928in}{1.101704in}}{\pgfqpoint{2.507928in}{1.112754in}}%
\pgfpathcurveto{\pgfqpoint{2.507928in}{1.123804in}}{\pgfqpoint{2.503537in}{1.134403in}}{\pgfqpoint{2.495724in}{1.142217in}}%
\pgfpathcurveto{\pgfqpoint{2.487910in}{1.150030in}}{\pgfqpoint{2.477311in}{1.154420in}}{\pgfqpoint{2.466261in}{1.154420in}}%
\pgfpathcurveto{\pgfqpoint{2.455211in}{1.154420in}}{\pgfqpoint{2.444612in}{1.150030in}}{\pgfqpoint{2.436798in}{1.142217in}}%
\pgfpathcurveto{\pgfqpoint{2.428985in}{1.134403in}}{\pgfqpoint{2.424594in}{1.123804in}}{\pgfqpoint{2.424594in}{1.112754in}}%
\pgfpathcurveto{\pgfqpoint{2.424594in}{1.101704in}}{\pgfqpoint{2.428985in}{1.091105in}}{\pgfqpoint{2.436798in}{1.083291in}}%
\pgfpathcurveto{\pgfqpoint{2.444612in}{1.075477in}}{\pgfqpoint{2.455211in}{1.071087in}}{\pgfqpoint{2.466261in}{1.071087in}}%
\pgfpathclose%
\pgfusepath{stroke,fill}%
\end{pgfscope}%
\begin{pgfscope}%
\pgfpathrectangle{\pgfqpoint{0.424692in}{0.370679in}}{\pgfqpoint{2.725308in}{1.479321in}}%
\pgfusepath{clip}%
\pgfsetbuttcap%
\pgfsetroundjoin%
\definecolor{currentfill}{rgb}{1.000000,0.411765,0.380392}%
\pgfsetfillcolor{currentfill}%
\pgfsetfillopacity{0.500000}%
\pgfsetlinewidth{1.003750pt}%
\definecolor{currentstroke}{rgb}{1.000000,0.411765,0.380392}%
\pgfsetstrokecolor{currentstroke}%
\pgfsetstrokeopacity{0.500000}%
\pgfsetdash{}{0pt}%
\pgfpathmoveto{\pgfqpoint{2.030169in}{0.906343in}}%
\pgfpathcurveto{\pgfqpoint{2.041219in}{0.906343in}}{\pgfqpoint{2.051818in}{0.910733in}}{\pgfqpoint{2.059632in}{0.918547in}}%
\pgfpathcurveto{\pgfqpoint{2.067445in}{0.926360in}}{\pgfqpoint{2.071836in}{0.936959in}}{\pgfqpoint{2.071836in}{0.948009in}}%
\pgfpathcurveto{\pgfqpoint{2.071836in}{0.959059in}}{\pgfqpoint{2.067445in}{0.969658in}}{\pgfqpoint{2.059632in}{0.977472in}}%
\pgfpathcurveto{\pgfqpoint{2.051818in}{0.985286in}}{\pgfqpoint{2.041219in}{0.989676in}}{\pgfqpoint{2.030169in}{0.989676in}}%
\pgfpathcurveto{\pgfqpoint{2.019119in}{0.989676in}}{\pgfqpoint{2.008520in}{0.985286in}}{\pgfqpoint{2.000706in}{0.977472in}}%
\pgfpathcurveto{\pgfqpoint{1.992893in}{0.969658in}}{\pgfqpoint{1.988502in}{0.959059in}}{\pgfqpoint{1.988502in}{0.948009in}}%
\pgfpathcurveto{\pgfqpoint{1.988502in}{0.936959in}}{\pgfqpoint{1.992893in}{0.926360in}}{\pgfqpoint{2.000706in}{0.918547in}}%
\pgfpathcurveto{\pgfqpoint{2.008520in}{0.910733in}}{\pgfqpoint{2.019119in}{0.906343in}}{\pgfqpoint{2.030169in}{0.906343in}}%
\pgfpathclose%
\pgfusepath{stroke,fill}%
\end{pgfscope}%
\begin{pgfscope}%
\pgfpathrectangle{\pgfqpoint{0.424692in}{0.370679in}}{\pgfqpoint{2.725308in}{1.479321in}}%
\pgfusepath{clip}%
\pgfsetbuttcap%
\pgfsetroundjoin%
\definecolor{currentfill}{rgb}{1.000000,0.411765,0.380392}%
\pgfsetfillcolor{currentfill}%
\pgfsetfillopacity{0.500000}%
\pgfsetlinewidth{1.003750pt}%
\definecolor{currentstroke}{rgb}{1.000000,0.411765,0.380392}%
\pgfsetstrokecolor{currentstroke}%
\pgfsetstrokeopacity{0.500000}%
\pgfsetdash{}{0pt}%
\pgfpathmoveto{\pgfqpoint{1.833014in}{0.823106in}}%
\pgfpathcurveto{\pgfqpoint{1.844064in}{0.823106in}}{\pgfqpoint{1.854663in}{0.827497in}}{\pgfqpoint{1.862477in}{0.835310in}}%
\pgfpathcurveto{\pgfqpoint{1.870290in}{0.843124in}}{\pgfqpoint{1.874681in}{0.853723in}}{\pgfqpoint{1.874681in}{0.864773in}}%
\pgfpathcurveto{\pgfqpoint{1.874681in}{0.875823in}}{\pgfqpoint{1.870290in}{0.886422in}}{\pgfqpoint{1.862477in}{0.894236in}}%
\pgfpathcurveto{\pgfqpoint{1.854663in}{0.902049in}}{\pgfqpoint{1.844064in}{0.906440in}}{\pgfqpoint{1.833014in}{0.906440in}}%
\pgfpathcurveto{\pgfqpoint{1.821964in}{0.906440in}}{\pgfqpoint{1.811365in}{0.902049in}}{\pgfqpoint{1.803551in}{0.894236in}}%
\pgfpathcurveto{\pgfqpoint{1.795738in}{0.886422in}}{\pgfqpoint{1.791347in}{0.875823in}}{\pgfqpoint{1.791347in}{0.864773in}}%
\pgfpathcurveto{\pgfqpoint{1.791347in}{0.853723in}}{\pgfqpoint{1.795738in}{0.843124in}}{\pgfqpoint{1.803551in}{0.835310in}}%
\pgfpathcurveto{\pgfqpoint{1.811365in}{0.827497in}}{\pgfqpoint{1.821964in}{0.823106in}}{\pgfqpoint{1.833014in}{0.823106in}}%
\pgfpathclose%
\pgfusepath{stroke,fill}%
\end{pgfscope}%
\begin{pgfscope}%
\pgfpathrectangle{\pgfqpoint{0.424692in}{0.370679in}}{\pgfqpoint{2.725308in}{1.479321in}}%
\pgfusepath{clip}%
\pgfsetbuttcap%
\pgfsetroundjoin%
\definecolor{currentfill}{rgb}{1.000000,0.411765,0.380392}%
\pgfsetfillcolor{currentfill}%
\pgfsetfillopacity{0.500000}%
\pgfsetlinewidth{1.003750pt}%
\definecolor{currentstroke}{rgb}{1.000000,0.411765,0.380392}%
\pgfsetstrokecolor{currentstroke}%
\pgfsetstrokeopacity{0.500000}%
\pgfsetdash{}{0pt}%
\pgfpathmoveto{\pgfqpoint{2.374112in}{1.055590in}}%
\pgfpathcurveto{\pgfqpoint{2.385162in}{1.055590in}}{\pgfqpoint{2.395761in}{1.059980in}}{\pgfqpoint{2.403575in}{1.067794in}}%
\pgfpathcurveto{\pgfqpoint{2.411388in}{1.075608in}}{\pgfqpoint{2.415779in}{1.086207in}}{\pgfqpoint{2.415779in}{1.097257in}}%
\pgfpathcurveto{\pgfqpoint{2.415779in}{1.108307in}}{\pgfqpoint{2.411388in}{1.118906in}}{\pgfqpoint{2.403575in}{1.126720in}}%
\pgfpathcurveto{\pgfqpoint{2.395761in}{1.134533in}}{\pgfqpoint{2.385162in}{1.138923in}}{\pgfqpoint{2.374112in}{1.138923in}}%
\pgfpathcurveto{\pgfqpoint{2.363062in}{1.138923in}}{\pgfqpoint{2.352463in}{1.134533in}}{\pgfqpoint{2.344649in}{1.126720in}}%
\pgfpathcurveto{\pgfqpoint{2.336836in}{1.118906in}}{\pgfqpoint{2.332445in}{1.108307in}}{\pgfqpoint{2.332445in}{1.097257in}}%
\pgfpathcurveto{\pgfqpoint{2.332445in}{1.086207in}}{\pgfqpoint{2.336836in}{1.075608in}}{\pgfqpoint{2.344649in}{1.067794in}}%
\pgfpathcurveto{\pgfqpoint{2.352463in}{1.059980in}}{\pgfqpoint{2.363062in}{1.055590in}}{\pgfqpoint{2.374112in}{1.055590in}}%
\pgfpathclose%
\pgfusepath{stroke,fill}%
\end{pgfscope}%
\begin{pgfscope}%
\pgfpathrectangle{\pgfqpoint{0.424692in}{0.370679in}}{\pgfqpoint{2.725308in}{1.479321in}}%
\pgfusepath{clip}%
\pgfsetbuttcap%
\pgfsetroundjoin%
\definecolor{currentfill}{rgb}{1.000000,0.411765,0.380392}%
\pgfsetfillcolor{currentfill}%
\pgfsetfillopacity{0.500000}%
\pgfsetlinewidth{1.003750pt}%
\definecolor{currentstroke}{rgb}{1.000000,0.411765,0.380392}%
\pgfsetstrokecolor{currentstroke}%
\pgfsetstrokeopacity{0.500000}%
\pgfsetdash{}{0pt}%
\pgfpathmoveto{\pgfqpoint{1.813824in}{0.772310in}}%
\pgfpathcurveto{\pgfqpoint{1.824874in}{0.772310in}}{\pgfqpoint{1.835473in}{0.776700in}}{\pgfqpoint{1.843287in}{0.784514in}}%
\pgfpathcurveto{\pgfqpoint{1.851101in}{0.792327in}}{\pgfqpoint{1.855491in}{0.802926in}}{\pgfqpoint{1.855491in}{0.813976in}}%
\pgfpathcurveto{\pgfqpoint{1.855491in}{0.825026in}}{\pgfqpoint{1.851101in}{0.835626in}}{\pgfqpoint{1.843287in}{0.843439in}}%
\pgfpathcurveto{\pgfqpoint{1.835473in}{0.851253in}}{\pgfqpoint{1.824874in}{0.855643in}}{\pgfqpoint{1.813824in}{0.855643in}}%
\pgfpathcurveto{\pgfqpoint{1.802774in}{0.855643in}}{\pgfqpoint{1.792175in}{0.851253in}}{\pgfqpoint{1.784362in}{0.843439in}}%
\pgfpathcurveto{\pgfqpoint{1.776548in}{0.835626in}}{\pgfqpoint{1.772158in}{0.825026in}}{\pgfqpoint{1.772158in}{0.813976in}}%
\pgfpathcurveto{\pgfqpoint{1.772158in}{0.802926in}}{\pgfqpoint{1.776548in}{0.792327in}}{\pgfqpoint{1.784362in}{0.784514in}}%
\pgfpathcurveto{\pgfqpoint{1.792175in}{0.776700in}}{\pgfqpoint{1.802774in}{0.772310in}}{\pgfqpoint{1.813824in}{0.772310in}}%
\pgfpathclose%
\pgfusepath{stroke,fill}%
\end{pgfscope}%
\begin{pgfscope}%
\pgfpathrectangle{\pgfqpoint{0.424692in}{0.370679in}}{\pgfqpoint{2.725308in}{1.479321in}}%
\pgfusepath{clip}%
\pgfsetbuttcap%
\pgfsetroundjoin%
\definecolor{currentfill}{rgb}{1.000000,0.411765,0.380392}%
\pgfsetfillcolor{currentfill}%
\pgfsetfillopacity{0.500000}%
\pgfsetlinewidth{1.003750pt}%
\definecolor{currentstroke}{rgb}{1.000000,0.411765,0.380392}%
\pgfsetstrokecolor{currentstroke}%
\pgfsetstrokeopacity{0.500000}%
\pgfsetdash{}{0pt}%
\pgfpathmoveto{\pgfqpoint{2.369171in}{1.037599in}}%
\pgfpathcurveto{\pgfqpoint{2.380222in}{1.037599in}}{\pgfqpoint{2.390821in}{1.041989in}}{\pgfqpoint{2.398634in}{1.049803in}}%
\pgfpathcurveto{\pgfqpoint{2.406448in}{1.057616in}}{\pgfqpoint{2.410838in}{1.068215in}}{\pgfqpoint{2.410838in}{1.079265in}}%
\pgfpathcurveto{\pgfqpoint{2.410838in}{1.090315in}}{\pgfqpoint{2.406448in}{1.100914in}}{\pgfqpoint{2.398634in}{1.108728in}}%
\pgfpathcurveto{\pgfqpoint{2.390821in}{1.116542in}}{\pgfqpoint{2.380222in}{1.120932in}}{\pgfqpoint{2.369171in}{1.120932in}}%
\pgfpathcurveto{\pgfqpoint{2.358121in}{1.120932in}}{\pgfqpoint{2.347522in}{1.116542in}}{\pgfqpoint{2.339709in}{1.108728in}}%
\pgfpathcurveto{\pgfqpoint{2.331895in}{1.100914in}}{\pgfqpoint{2.327505in}{1.090315in}}{\pgfqpoint{2.327505in}{1.079265in}}%
\pgfpathcurveto{\pgfqpoint{2.327505in}{1.068215in}}{\pgfqpoint{2.331895in}{1.057616in}}{\pgfqpoint{2.339709in}{1.049803in}}%
\pgfpathcurveto{\pgfqpoint{2.347522in}{1.041989in}}{\pgfqpoint{2.358121in}{1.037599in}}{\pgfqpoint{2.369171in}{1.037599in}}%
\pgfpathclose%
\pgfusepath{stroke,fill}%
\end{pgfscope}%
\begin{pgfscope}%
\pgfpathrectangle{\pgfqpoint{0.424692in}{0.370679in}}{\pgfqpoint{2.725308in}{1.479321in}}%
\pgfusepath{clip}%
\pgfsetbuttcap%
\pgfsetroundjoin%
\definecolor{currentfill}{rgb}{1.000000,0.411765,0.380392}%
\pgfsetfillcolor{currentfill}%
\pgfsetfillopacity{0.500000}%
\pgfsetlinewidth{1.003750pt}%
\definecolor{currentstroke}{rgb}{1.000000,0.411765,0.380392}%
\pgfsetstrokecolor{currentstroke}%
\pgfsetstrokeopacity{0.500000}%
\pgfsetdash{}{0pt}%
\pgfpathmoveto{\pgfqpoint{1.756607in}{0.740514in}}%
\pgfpathcurveto{\pgfqpoint{1.767657in}{0.740514in}}{\pgfqpoint{1.778256in}{0.744904in}}{\pgfqpoint{1.786070in}{0.752718in}}%
\pgfpathcurveto{\pgfqpoint{1.793884in}{0.760531in}}{\pgfqpoint{1.798274in}{0.771130in}}{\pgfqpoint{1.798274in}{0.782180in}}%
\pgfpathcurveto{\pgfqpoint{1.798274in}{0.793231in}}{\pgfqpoint{1.793884in}{0.803830in}}{\pgfqpoint{1.786070in}{0.811643in}}%
\pgfpathcurveto{\pgfqpoint{1.778256in}{0.819457in}}{\pgfqpoint{1.767657in}{0.823847in}}{\pgfqpoint{1.756607in}{0.823847in}}%
\pgfpathcurveto{\pgfqpoint{1.745557in}{0.823847in}}{\pgfqpoint{1.734958in}{0.819457in}}{\pgfqpoint{1.727144in}{0.811643in}}%
\pgfpathcurveto{\pgfqpoint{1.719331in}{0.803830in}}{\pgfqpoint{1.714941in}{0.793231in}}{\pgfqpoint{1.714941in}{0.782180in}}%
\pgfpathcurveto{\pgfqpoint{1.714941in}{0.771130in}}{\pgfqpoint{1.719331in}{0.760531in}}{\pgfqpoint{1.727144in}{0.752718in}}%
\pgfpathcurveto{\pgfqpoint{1.734958in}{0.744904in}}{\pgfqpoint{1.745557in}{0.740514in}}{\pgfqpoint{1.756607in}{0.740514in}}%
\pgfpathclose%
\pgfusepath{stroke,fill}%
\end{pgfscope}%
\begin{pgfscope}%
\pgfpathrectangle{\pgfqpoint{0.424692in}{0.370679in}}{\pgfqpoint{2.725308in}{1.479321in}}%
\pgfusepath{clip}%
\pgfsetbuttcap%
\pgfsetroundjoin%
\definecolor{currentfill}{rgb}{1.000000,0.411765,0.380392}%
\pgfsetfillcolor{currentfill}%
\pgfsetfillopacity{0.500000}%
\pgfsetlinewidth{1.003750pt}%
\definecolor{currentstroke}{rgb}{1.000000,0.411765,0.380392}%
\pgfsetstrokecolor{currentstroke}%
\pgfsetstrokeopacity{0.500000}%
\pgfsetdash{}{0pt}%
\pgfpathmoveto{\pgfqpoint{2.114403in}{0.915593in}}%
\pgfpathcurveto{\pgfqpoint{2.125453in}{0.915593in}}{\pgfqpoint{2.136052in}{0.919983in}}{\pgfqpoint{2.143866in}{0.927797in}}%
\pgfpathcurveto{\pgfqpoint{2.151680in}{0.935610in}}{\pgfqpoint{2.156070in}{0.946209in}}{\pgfqpoint{2.156070in}{0.957260in}}%
\pgfpathcurveto{\pgfqpoint{2.156070in}{0.968310in}}{\pgfqpoint{2.151680in}{0.978909in}}{\pgfqpoint{2.143866in}{0.986722in}}%
\pgfpathcurveto{\pgfqpoint{2.136052in}{0.994536in}}{\pgfqpoint{2.125453in}{0.998926in}}{\pgfqpoint{2.114403in}{0.998926in}}%
\pgfpathcurveto{\pgfqpoint{2.103353in}{0.998926in}}{\pgfqpoint{2.092754in}{0.994536in}}{\pgfqpoint{2.084940in}{0.986722in}}%
\pgfpathcurveto{\pgfqpoint{2.077127in}{0.978909in}}{\pgfqpoint{2.072737in}{0.968310in}}{\pgfqpoint{2.072737in}{0.957260in}}%
\pgfpathcurveto{\pgfqpoint{2.072737in}{0.946209in}}{\pgfqpoint{2.077127in}{0.935610in}}{\pgfqpoint{2.084940in}{0.927797in}}%
\pgfpathcurveto{\pgfqpoint{2.092754in}{0.919983in}}{\pgfqpoint{2.103353in}{0.915593in}}{\pgfqpoint{2.114403in}{0.915593in}}%
\pgfpathclose%
\pgfusepath{stroke,fill}%
\end{pgfscope}%
\begin{pgfscope}%
\pgfpathrectangle{\pgfqpoint{0.424692in}{0.370679in}}{\pgfqpoint{2.725308in}{1.479321in}}%
\pgfusepath{clip}%
\pgfsetbuttcap%
\pgfsetroundjoin%
\definecolor{currentfill}{rgb}{0.466667,0.866667,0.466667}%
\pgfsetfillcolor{currentfill}%
\pgfsetfillopacity{0.500000}%
\pgfsetlinewidth{1.003750pt}%
\definecolor{currentstroke}{rgb}{0.466667,0.866667,0.466667}%
\pgfsetstrokecolor{currentstroke}%
\pgfsetstrokeopacity{0.500000}%
\pgfsetdash{}{0pt}%
\pgfpathmoveto{\pgfqpoint{1.318208in}{0.982856in}}%
\pgfpathlineto{\pgfqpoint{1.339041in}{1.003689in}}%
\pgfpathlineto{\pgfqpoint{1.359875in}{0.982856in}}%
\pgfpathlineto{\pgfqpoint{1.380708in}{1.003689in}}%
\pgfpathlineto{\pgfqpoint{1.359875in}{1.024522in}}%
\pgfpathlineto{\pgfqpoint{1.380708in}{1.045356in}}%
\pgfpathlineto{\pgfqpoint{1.359875in}{1.066189in}}%
\pgfpathlineto{\pgfqpoint{1.339041in}{1.045356in}}%
\pgfpathlineto{\pgfqpoint{1.318208in}{1.066189in}}%
\pgfpathlineto{\pgfqpoint{1.297375in}{1.045356in}}%
\pgfpathlineto{\pgfqpoint{1.318208in}{1.024522in}}%
\pgfpathlineto{\pgfqpoint{1.297375in}{1.003689in}}%
\pgfpathclose%
\pgfusepath{stroke,fill}%
\end{pgfscope}%
\begin{pgfscope}%
\pgfpathrectangle{\pgfqpoint{0.424692in}{0.370679in}}{\pgfqpoint{2.725308in}{1.479321in}}%
\pgfusepath{clip}%
\pgfsetbuttcap%
\pgfsetroundjoin%
\definecolor{currentfill}{rgb}{0.466667,0.866667,0.466667}%
\pgfsetfillcolor{currentfill}%
\pgfsetfillopacity{0.500000}%
\pgfsetlinewidth{1.003750pt}%
\definecolor{currentstroke}{rgb}{0.466667,0.866667,0.466667}%
\pgfsetstrokecolor{currentstroke}%
\pgfsetstrokeopacity{0.500000}%
\pgfsetdash{}{0pt}%
\pgfpathmoveto{\pgfqpoint{1.617453in}{1.102584in}}%
\pgfpathlineto{\pgfqpoint{1.638286in}{1.123418in}}%
\pgfpathlineto{\pgfqpoint{1.659119in}{1.102584in}}%
\pgfpathlineto{\pgfqpoint{1.679953in}{1.123418in}}%
\pgfpathlineto{\pgfqpoint{1.659119in}{1.144251in}}%
\pgfpathlineto{\pgfqpoint{1.679953in}{1.165084in}}%
\pgfpathlineto{\pgfqpoint{1.659119in}{1.185918in}}%
\pgfpathlineto{\pgfqpoint{1.638286in}{1.165084in}}%
\pgfpathlineto{\pgfqpoint{1.617453in}{1.185918in}}%
\pgfpathlineto{\pgfqpoint{1.596619in}{1.165084in}}%
\pgfpathlineto{\pgfqpoint{1.617453in}{1.144251in}}%
\pgfpathlineto{\pgfqpoint{1.596619in}{1.123418in}}%
\pgfpathclose%
\pgfusepath{stroke,fill}%
\end{pgfscope}%
\begin{pgfscope}%
\pgfpathrectangle{\pgfqpoint{0.424692in}{0.370679in}}{\pgfqpoint{2.725308in}{1.479321in}}%
\pgfusepath{clip}%
\pgfsetbuttcap%
\pgfsetroundjoin%
\definecolor{currentfill}{rgb}{0.466667,0.866667,0.466667}%
\pgfsetfillcolor{currentfill}%
\pgfsetfillopacity{0.500000}%
\pgfsetlinewidth{1.003750pt}%
\definecolor{currentstroke}{rgb}{0.466667,0.866667,0.466667}%
\pgfsetstrokecolor{currentstroke}%
\pgfsetstrokeopacity{0.500000}%
\pgfsetdash{}{0pt}%
\pgfpathmoveto{\pgfqpoint{1.434279in}{1.068074in}}%
\pgfpathlineto{\pgfqpoint{1.455113in}{1.088907in}}%
\pgfpathlineto{\pgfqpoint{1.475946in}{1.068074in}}%
\pgfpathlineto{\pgfqpoint{1.496779in}{1.088907in}}%
\pgfpathlineto{\pgfqpoint{1.475946in}{1.109741in}}%
\pgfpathlineto{\pgfqpoint{1.496779in}{1.130574in}}%
\pgfpathlineto{\pgfqpoint{1.475946in}{1.151407in}}%
\pgfpathlineto{\pgfqpoint{1.455113in}{1.130574in}}%
\pgfpathlineto{\pgfqpoint{1.434279in}{1.151407in}}%
\pgfpathlineto{\pgfqpoint{1.413446in}{1.130574in}}%
\pgfpathlineto{\pgfqpoint{1.434279in}{1.109741in}}%
\pgfpathlineto{\pgfqpoint{1.413446in}{1.088907in}}%
\pgfpathclose%
\pgfusepath{stroke,fill}%
\end{pgfscope}%
\begin{pgfscope}%
\pgfpathrectangle{\pgfqpoint{0.424692in}{0.370679in}}{\pgfqpoint{2.725308in}{1.479321in}}%
\pgfusepath{clip}%
\pgfsetbuttcap%
\pgfsetroundjoin%
\definecolor{currentfill}{rgb}{0.466667,0.866667,0.466667}%
\pgfsetfillcolor{currentfill}%
\pgfsetfillopacity{0.500000}%
\pgfsetlinewidth{1.003750pt}%
\definecolor{currentstroke}{rgb}{0.466667,0.866667,0.466667}%
\pgfsetstrokecolor{currentstroke}%
\pgfsetstrokeopacity{0.500000}%
\pgfsetdash{}{0pt}%
\pgfpathmoveto{\pgfqpoint{2.513069in}{1.498120in}}%
\pgfpathlineto{\pgfqpoint{2.533902in}{1.518953in}}%
\pgfpathlineto{\pgfqpoint{2.554736in}{1.498120in}}%
\pgfpathlineto{\pgfqpoint{2.575569in}{1.518953in}}%
\pgfpathlineto{\pgfqpoint{2.554736in}{1.539787in}}%
\pgfpathlineto{\pgfqpoint{2.575569in}{1.560620in}}%
\pgfpathlineto{\pgfqpoint{2.554736in}{1.581453in}}%
\pgfpathlineto{\pgfqpoint{2.533902in}{1.560620in}}%
\pgfpathlineto{\pgfqpoint{2.513069in}{1.581453in}}%
\pgfpathlineto{\pgfqpoint{2.492236in}{1.560620in}}%
\pgfpathlineto{\pgfqpoint{2.513069in}{1.539787in}}%
\pgfpathlineto{\pgfqpoint{2.492236in}{1.518953in}}%
\pgfpathclose%
\pgfusepath{stroke,fill}%
\end{pgfscope}%
\begin{pgfscope}%
\pgfpathrectangle{\pgfqpoint{0.424692in}{0.370679in}}{\pgfqpoint{2.725308in}{1.479321in}}%
\pgfusepath{clip}%
\pgfsetbuttcap%
\pgfsetroundjoin%
\definecolor{currentfill}{rgb}{0.466667,0.866667,0.466667}%
\pgfsetfillcolor{currentfill}%
\pgfsetfillopacity{0.500000}%
\pgfsetlinewidth{1.003750pt}%
\definecolor{currentstroke}{rgb}{0.466667,0.866667,0.466667}%
\pgfsetstrokecolor{currentstroke}%
\pgfsetstrokeopacity{0.500000}%
\pgfsetdash{}{0pt}%
\pgfpathmoveto{\pgfqpoint{1.778023in}{1.209018in}}%
\pgfpathlineto{\pgfqpoint{1.798856in}{1.229851in}}%
\pgfpathlineto{\pgfqpoint{1.819690in}{1.209018in}}%
\pgfpathlineto{\pgfqpoint{1.840523in}{1.229851in}}%
\pgfpathlineto{\pgfqpoint{1.819690in}{1.250685in}}%
\pgfpathlineto{\pgfqpoint{1.840523in}{1.271518in}}%
\pgfpathlineto{\pgfqpoint{1.819690in}{1.292351in}}%
\pgfpathlineto{\pgfqpoint{1.798856in}{1.271518in}}%
\pgfpathlineto{\pgfqpoint{1.778023in}{1.292351in}}%
\pgfpathlineto{\pgfqpoint{1.757190in}{1.271518in}}%
\pgfpathlineto{\pgfqpoint{1.778023in}{1.250685in}}%
\pgfpathlineto{\pgfqpoint{1.757190in}{1.229851in}}%
\pgfpathclose%
\pgfusepath{stroke,fill}%
\end{pgfscope}%
\begin{pgfscope}%
\pgfpathrectangle{\pgfqpoint{0.424692in}{0.370679in}}{\pgfqpoint{2.725308in}{1.479321in}}%
\pgfusepath{clip}%
\pgfsetbuttcap%
\pgfsetroundjoin%
\definecolor{currentfill}{rgb}{0.466667,0.866667,0.466667}%
\pgfsetfillcolor{currentfill}%
\pgfsetfillopacity{0.500000}%
\pgfsetlinewidth{1.003750pt}%
\definecolor{currentstroke}{rgb}{0.466667,0.866667,0.466667}%
\pgfsetstrokecolor{currentstroke}%
\pgfsetstrokeopacity{0.500000}%
\pgfsetdash{}{0pt}%
\pgfpathmoveto{\pgfqpoint{2.176359in}{1.371135in}}%
\pgfpathlineto{\pgfqpoint{2.197192in}{1.391968in}}%
\pgfpathlineto{\pgfqpoint{2.218025in}{1.371135in}}%
\pgfpathlineto{\pgfqpoint{2.238859in}{1.391968in}}%
\pgfpathlineto{\pgfqpoint{2.218025in}{1.412801in}}%
\pgfpathlineto{\pgfqpoint{2.238859in}{1.433635in}}%
\pgfpathlineto{\pgfqpoint{2.218025in}{1.454468in}}%
\pgfpathlineto{\pgfqpoint{2.197192in}{1.433635in}}%
\pgfpathlineto{\pgfqpoint{2.176359in}{1.454468in}}%
\pgfpathlineto{\pgfqpoint{2.155525in}{1.433635in}}%
\pgfpathlineto{\pgfqpoint{2.176359in}{1.412801in}}%
\pgfpathlineto{\pgfqpoint{2.155525in}{1.391968in}}%
\pgfpathclose%
\pgfusepath{stroke,fill}%
\end{pgfscope}%
\begin{pgfscope}%
\pgfpathrectangle{\pgfqpoint{0.424692in}{0.370679in}}{\pgfqpoint{2.725308in}{1.479321in}}%
\pgfusepath{clip}%
\pgfsetbuttcap%
\pgfsetroundjoin%
\definecolor{currentfill}{rgb}{0.466667,0.866667,0.466667}%
\pgfsetfillcolor{currentfill}%
\pgfsetfillopacity{0.500000}%
\pgfsetlinewidth{1.003750pt}%
\definecolor{currentstroke}{rgb}{0.466667,0.866667,0.466667}%
\pgfsetstrokecolor{currentstroke}%
\pgfsetstrokeopacity{0.500000}%
\pgfsetdash{}{0pt}%
\pgfpathmoveto{\pgfqpoint{1.627631in}{1.128107in}}%
\pgfpathlineto{\pgfqpoint{1.648465in}{1.148940in}}%
\pgfpathlineto{\pgfqpoint{1.669298in}{1.128107in}}%
\pgfpathlineto{\pgfqpoint{1.690131in}{1.148940in}}%
\pgfpathlineto{\pgfqpoint{1.669298in}{1.169773in}}%
\pgfpathlineto{\pgfqpoint{1.690131in}{1.190607in}}%
\pgfpathlineto{\pgfqpoint{1.669298in}{1.211440in}}%
\pgfpathlineto{\pgfqpoint{1.648465in}{1.190607in}}%
\pgfpathlineto{\pgfqpoint{1.627631in}{1.211440in}}%
\pgfpathlineto{\pgfqpoint{1.606798in}{1.190607in}}%
\pgfpathlineto{\pgfqpoint{1.627631in}{1.169773in}}%
\pgfpathlineto{\pgfqpoint{1.606798in}{1.148940in}}%
\pgfpathclose%
\pgfusepath{stroke,fill}%
\end{pgfscope}%
\begin{pgfscope}%
\pgfpathrectangle{\pgfqpoint{0.424692in}{0.370679in}}{\pgfqpoint{2.725308in}{1.479321in}}%
\pgfusepath{clip}%
\pgfsetbuttcap%
\pgfsetroundjoin%
\definecolor{currentfill}{rgb}{0.466667,0.866667,0.466667}%
\pgfsetfillcolor{currentfill}%
\pgfsetfillopacity{0.500000}%
\pgfsetlinewidth{1.003750pt}%
\definecolor{currentstroke}{rgb}{0.466667,0.866667,0.466667}%
\pgfsetstrokecolor{currentstroke}%
\pgfsetstrokeopacity{0.500000}%
\pgfsetdash{}{0pt}%
\pgfpathmoveto{\pgfqpoint{2.723149in}{1.498088in}}%
\pgfpathlineto{\pgfqpoint{2.743982in}{1.518922in}}%
\pgfpathlineto{\pgfqpoint{2.764816in}{1.498088in}}%
\pgfpathlineto{\pgfqpoint{2.785649in}{1.518922in}}%
\pgfpathlineto{\pgfqpoint{2.764816in}{1.539755in}}%
\pgfpathlineto{\pgfqpoint{2.785649in}{1.560588in}}%
\pgfpathlineto{\pgfqpoint{2.764816in}{1.581422in}}%
\pgfpathlineto{\pgfqpoint{2.743982in}{1.560588in}}%
\pgfpathlineto{\pgfqpoint{2.723149in}{1.581422in}}%
\pgfpathlineto{\pgfqpoint{2.702316in}{1.560588in}}%
\pgfpathlineto{\pgfqpoint{2.723149in}{1.539755in}}%
\pgfpathlineto{\pgfqpoint{2.702316in}{1.518922in}}%
\pgfpathclose%
\pgfusepath{stroke,fill}%
\end{pgfscope}%
\begin{pgfscope}%
\pgfpathrectangle{\pgfqpoint{0.424692in}{0.370679in}}{\pgfqpoint{2.725308in}{1.479321in}}%
\pgfusepath{clip}%
\pgfsetbuttcap%
\pgfsetroundjoin%
\definecolor{currentfill}{rgb}{0.466667,0.866667,0.466667}%
\pgfsetfillcolor{currentfill}%
\pgfsetfillopacity{0.500000}%
\pgfsetlinewidth{1.003750pt}%
\definecolor{currentstroke}{rgb}{0.466667,0.866667,0.466667}%
\pgfsetstrokecolor{currentstroke}%
\pgfsetstrokeopacity{0.500000}%
\pgfsetdash{}{0pt}%
\pgfpathmoveto{\pgfqpoint{1.669637in}{1.144607in}}%
\pgfpathlineto{\pgfqpoint{1.690471in}{1.165440in}}%
\pgfpathlineto{\pgfqpoint{1.711304in}{1.144607in}}%
\pgfpathlineto{\pgfqpoint{1.732137in}{1.165440in}}%
\pgfpathlineto{\pgfqpoint{1.711304in}{1.186273in}}%
\pgfpathlineto{\pgfqpoint{1.732137in}{1.207107in}}%
\pgfpathlineto{\pgfqpoint{1.711304in}{1.227940in}}%
\pgfpathlineto{\pgfqpoint{1.690471in}{1.207107in}}%
\pgfpathlineto{\pgfqpoint{1.669637in}{1.227940in}}%
\pgfpathlineto{\pgfqpoint{1.648804in}{1.207107in}}%
\pgfpathlineto{\pgfqpoint{1.669637in}{1.186273in}}%
\pgfpathlineto{\pgfqpoint{1.648804in}{1.165440in}}%
\pgfpathclose%
\pgfusepath{stroke,fill}%
\end{pgfscope}%
\begin{pgfscope}%
\pgfpathrectangle{\pgfqpoint{0.424692in}{0.370679in}}{\pgfqpoint{2.725308in}{1.479321in}}%
\pgfusepath{clip}%
\pgfsetbuttcap%
\pgfsetroundjoin%
\definecolor{currentfill}{rgb}{0.466667,0.866667,0.466667}%
\pgfsetfillcolor{currentfill}%
\pgfsetfillopacity{0.500000}%
\pgfsetlinewidth{1.003750pt}%
\definecolor{currentstroke}{rgb}{0.466667,0.866667,0.466667}%
\pgfsetstrokecolor{currentstroke}%
\pgfsetstrokeopacity{0.500000}%
\pgfsetdash{}{0pt}%
\pgfpathmoveto{\pgfqpoint{2.455870in}{1.443510in}}%
\pgfpathlineto{\pgfqpoint{2.476703in}{1.464343in}}%
\pgfpathlineto{\pgfqpoint{2.497537in}{1.443510in}}%
\pgfpathlineto{\pgfqpoint{2.518370in}{1.464343in}}%
\pgfpathlineto{\pgfqpoint{2.497537in}{1.485177in}}%
\pgfpathlineto{\pgfqpoint{2.518370in}{1.506010in}}%
\pgfpathlineto{\pgfqpoint{2.497537in}{1.526843in}}%
\pgfpathlineto{\pgfqpoint{2.476703in}{1.506010in}}%
\pgfpathlineto{\pgfqpoint{2.455870in}{1.526843in}}%
\pgfpathlineto{\pgfqpoint{2.435037in}{1.506010in}}%
\pgfpathlineto{\pgfqpoint{2.455870in}{1.485177in}}%
\pgfpathlineto{\pgfqpoint{2.435037in}{1.464343in}}%
\pgfpathclose%
\pgfusepath{stroke,fill}%
\end{pgfscope}%
\begin{pgfscope}%
\pgfpathrectangle{\pgfqpoint{0.424692in}{0.370679in}}{\pgfqpoint{2.725308in}{1.479321in}}%
\pgfusepath{clip}%
\pgfsetbuttcap%
\pgfsetroundjoin%
\definecolor{currentfill}{rgb}{0.466667,0.866667,0.466667}%
\pgfsetfillcolor{currentfill}%
\pgfsetfillopacity{0.500000}%
\pgfsetlinewidth{1.003750pt}%
\definecolor{currentstroke}{rgb}{0.466667,0.866667,0.466667}%
\pgfsetstrokecolor{currentstroke}%
\pgfsetstrokeopacity{0.500000}%
\pgfsetdash{}{0pt}%
\pgfpathmoveto{\pgfqpoint{2.889767in}{1.618332in}}%
\pgfpathlineto{\pgfqpoint{2.910600in}{1.639165in}}%
\pgfpathlineto{\pgfqpoint{2.931433in}{1.618332in}}%
\pgfpathlineto{\pgfqpoint{2.952267in}{1.639165in}}%
\pgfpathlineto{\pgfqpoint{2.931433in}{1.659998in}}%
\pgfpathlineto{\pgfqpoint{2.952267in}{1.680832in}}%
\pgfpathlineto{\pgfqpoint{2.931433in}{1.701665in}}%
\pgfpathlineto{\pgfqpoint{2.910600in}{1.680832in}}%
\pgfpathlineto{\pgfqpoint{2.889767in}{1.701665in}}%
\pgfpathlineto{\pgfqpoint{2.868933in}{1.680832in}}%
\pgfpathlineto{\pgfqpoint{2.889767in}{1.659998in}}%
\pgfpathlineto{\pgfqpoint{2.868933in}{1.639165in}}%
\pgfpathclose%
\pgfusepath{stroke,fill}%
\end{pgfscope}%
\begin{pgfscope}%
\pgfpathrectangle{\pgfqpoint{0.424692in}{0.370679in}}{\pgfqpoint{2.725308in}{1.479321in}}%
\pgfusepath{clip}%
\pgfsetbuttcap%
\pgfsetroundjoin%
\definecolor{currentfill}{rgb}{0.466667,0.866667,0.466667}%
\pgfsetfillcolor{currentfill}%
\pgfsetfillopacity{0.500000}%
\pgfsetlinewidth{1.003750pt}%
\definecolor{currentstroke}{rgb}{0.466667,0.866667,0.466667}%
\pgfsetstrokecolor{currentstroke}%
\pgfsetstrokeopacity{0.500000}%
\pgfsetdash{}{0pt}%
\pgfpathmoveto{\pgfqpoint{1.960173in}{1.237524in}}%
\pgfpathlineto{\pgfqpoint{1.981007in}{1.258357in}}%
\pgfpathlineto{\pgfqpoint{2.001840in}{1.237524in}}%
\pgfpathlineto{\pgfqpoint{2.022673in}{1.258357in}}%
\pgfpathlineto{\pgfqpoint{2.001840in}{1.279190in}}%
\pgfpathlineto{\pgfqpoint{2.022673in}{1.300024in}}%
\pgfpathlineto{\pgfqpoint{2.001840in}{1.320857in}}%
\pgfpathlineto{\pgfqpoint{1.981007in}{1.300024in}}%
\pgfpathlineto{\pgfqpoint{1.960173in}{1.320857in}}%
\pgfpathlineto{\pgfqpoint{1.939340in}{1.300024in}}%
\pgfpathlineto{\pgfqpoint{1.960173in}{1.279190in}}%
\pgfpathlineto{\pgfqpoint{1.939340in}{1.258357in}}%
\pgfpathclose%
\pgfusepath{stroke,fill}%
\end{pgfscope}%
\begin{pgfscope}%
\pgfpathrectangle{\pgfqpoint{0.424692in}{0.370679in}}{\pgfqpoint{2.725308in}{1.479321in}}%
\pgfusepath{clip}%
\pgfsetbuttcap%
\pgfsetroundjoin%
\definecolor{currentfill}{rgb}{0.466667,0.866667,0.466667}%
\pgfsetfillcolor{currentfill}%
\pgfsetfillopacity{0.500000}%
\pgfsetlinewidth{1.003750pt}%
\definecolor{currentstroke}{rgb}{0.466667,0.866667,0.466667}%
\pgfsetstrokecolor{currentstroke}%
\pgfsetstrokeopacity{0.500000}%
\pgfsetdash{}{0pt}%
\pgfpathmoveto{\pgfqpoint{2.300459in}{1.392306in}}%
\pgfpathlineto{\pgfqpoint{2.321292in}{1.413140in}}%
\pgfpathlineto{\pgfqpoint{2.342126in}{1.392306in}}%
\pgfpathlineto{\pgfqpoint{2.362959in}{1.413140in}}%
\pgfpathlineto{\pgfqpoint{2.342126in}{1.433973in}}%
\pgfpathlineto{\pgfqpoint{2.362959in}{1.454806in}}%
\pgfpathlineto{\pgfqpoint{2.342126in}{1.475640in}}%
\pgfpathlineto{\pgfqpoint{2.321292in}{1.454806in}}%
\pgfpathlineto{\pgfqpoint{2.300459in}{1.475640in}}%
\pgfpathlineto{\pgfqpoint{2.279626in}{1.454806in}}%
\pgfpathlineto{\pgfqpoint{2.300459in}{1.433973in}}%
\pgfpathlineto{\pgfqpoint{2.279626in}{1.413140in}}%
\pgfpathclose%
\pgfusepath{stroke,fill}%
\end{pgfscope}%
\begin{pgfscope}%
\pgfpathrectangle{\pgfqpoint{0.424692in}{0.370679in}}{\pgfqpoint{2.725308in}{1.479321in}}%
\pgfusepath{clip}%
\pgfsetbuttcap%
\pgfsetroundjoin%
\definecolor{currentfill}{rgb}{0.466667,0.866667,0.466667}%
\pgfsetfillcolor{currentfill}%
\pgfsetfillopacity{0.500000}%
\pgfsetlinewidth{1.003750pt}%
\definecolor{currentstroke}{rgb}{0.466667,0.866667,0.466667}%
\pgfsetstrokecolor{currentstroke}%
\pgfsetstrokeopacity{0.500000}%
\pgfsetdash{}{0pt}%
\pgfpathmoveto{\pgfqpoint{1.932180in}{1.280321in}}%
\pgfpathlineto{\pgfqpoint{1.953014in}{1.301154in}}%
\pgfpathlineto{\pgfqpoint{1.973847in}{1.280321in}}%
\pgfpathlineto{\pgfqpoint{1.994680in}{1.301154in}}%
\pgfpathlineto{\pgfqpoint{1.973847in}{1.321988in}}%
\pgfpathlineto{\pgfqpoint{1.994680in}{1.342821in}}%
\pgfpathlineto{\pgfqpoint{1.973847in}{1.363654in}}%
\pgfpathlineto{\pgfqpoint{1.953014in}{1.342821in}}%
\pgfpathlineto{\pgfqpoint{1.932180in}{1.363654in}}%
\pgfpathlineto{\pgfqpoint{1.911347in}{1.342821in}}%
\pgfpathlineto{\pgfqpoint{1.932180in}{1.321988in}}%
\pgfpathlineto{\pgfqpoint{1.911347in}{1.301154in}}%
\pgfpathclose%
\pgfusepath{stroke,fill}%
\end{pgfscope}%
\begin{pgfscope}%
\pgfpathrectangle{\pgfqpoint{0.424692in}{0.370679in}}{\pgfqpoint{2.725308in}{1.479321in}}%
\pgfusepath{clip}%
\pgfsetbuttcap%
\pgfsetroundjoin%
\definecolor{currentfill}{rgb}{0.466667,0.866667,0.466667}%
\pgfsetfillcolor{currentfill}%
\pgfsetfillopacity{0.500000}%
\pgfsetlinewidth{1.003750pt}%
\definecolor{currentstroke}{rgb}{0.466667,0.866667,0.466667}%
\pgfsetstrokecolor{currentstroke}%
\pgfsetstrokeopacity{0.500000}%
\pgfsetdash{}{0pt}%
\pgfpathmoveto{\pgfqpoint{2.127874in}{1.244659in}}%
\pgfpathlineto{\pgfqpoint{2.148707in}{1.265492in}}%
\pgfpathlineto{\pgfqpoint{2.169540in}{1.244659in}}%
\pgfpathlineto{\pgfqpoint{2.190374in}{1.265492in}}%
\pgfpathlineto{\pgfqpoint{2.169540in}{1.286325in}}%
\pgfpathlineto{\pgfqpoint{2.190374in}{1.307159in}}%
\pgfpathlineto{\pgfqpoint{2.169540in}{1.327992in}}%
\pgfpathlineto{\pgfqpoint{2.148707in}{1.307159in}}%
\pgfpathlineto{\pgfqpoint{2.127874in}{1.327992in}}%
\pgfpathlineto{\pgfqpoint{2.107040in}{1.307159in}}%
\pgfpathlineto{\pgfqpoint{2.127874in}{1.286325in}}%
\pgfpathlineto{\pgfqpoint{2.107040in}{1.265492in}}%
\pgfpathclose%
\pgfusepath{stroke,fill}%
\end{pgfscope}%
\begin{pgfscope}%
\pgfpathrectangle{\pgfqpoint{0.424692in}{0.370679in}}{\pgfqpoint{2.725308in}{1.479321in}}%
\pgfusepath{clip}%
\pgfsetbuttcap%
\pgfsetroundjoin%
\definecolor{currentfill}{rgb}{0.466667,0.866667,0.466667}%
\pgfsetfillcolor{currentfill}%
\pgfsetfillopacity{0.500000}%
\pgfsetlinewidth{1.003750pt}%
\definecolor{currentstroke}{rgb}{0.466667,0.866667,0.466667}%
\pgfsetstrokecolor{currentstroke}%
\pgfsetstrokeopacity{0.500000}%
\pgfsetdash{}{0pt}%
\pgfpathmoveto{\pgfqpoint{1.507445in}{1.042073in}}%
\pgfpathlineto{\pgfqpoint{1.528278in}{1.062906in}}%
\pgfpathlineto{\pgfqpoint{1.549111in}{1.042073in}}%
\pgfpathlineto{\pgfqpoint{1.569945in}{1.062906in}}%
\pgfpathlineto{\pgfqpoint{1.549111in}{1.083739in}}%
\pgfpathlineto{\pgfqpoint{1.569945in}{1.104573in}}%
\pgfpathlineto{\pgfqpoint{1.549111in}{1.125406in}}%
\pgfpathlineto{\pgfqpoint{1.528278in}{1.104573in}}%
\pgfpathlineto{\pgfqpoint{1.507445in}{1.125406in}}%
\pgfpathlineto{\pgfqpoint{1.486611in}{1.104573in}}%
\pgfpathlineto{\pgfqpoint{1.507445in}{1.083739in}}%
\pgfpathlineto{\pgfqpoint{1.486611in}{1.062906in}}%
\pgfpathclose%
\pgfusepath{stroke,fill}%
\end{pgfscope}%
\begin{pgfscope}%
\pgfpathrectangle{\pgfqpoint{0.424692in}{0.370679in}}{\pgfqpoint{2.725308in}{1.479321in}}%
\pgfusepath{clip}%
\pgfsetbuttcap%
\pgfsetroundjoin%
\definecolor{currentfill}{rgb}{0.466667,0.866667,0.466667}%
\pgfsetfillcolor{currentfill}%
\pgfsetfillopacity{0.500000}%
\pgfsetlinewidth{1.003750pt}%
\definecolor{currentstroke}{rgb}{0.466667,0.866667,0.466667}%
\pgfsetstrokecolor{currentstroke}%
\pgfsetstrokeopacity{0.500000}%
\pgfsetdash{}{0pt}%
\pgfpathmoveto{\pgfqpoint{2.129353in}{1.347890in}}%
\pgfpathlineto{\pgfqpoint{2.150186in}{1.368724in}}%
\pgfpathlineto{\pgfqpoint{2.171020in}{1.347890in}}%
\pgfpathlineto{\pgfqpoint{2.191853in}{1.368724in}}%
\pgfpathlineto{\pgfqpoint{2.171020in}{1.389557in}}%
\pgfpathlineto{\pgfqpoint{2.191853in}{1.410390in}}%
\pgfpathlineto{\pgfqpoint{2.171020in}{1.431224in}}%
\pgfpathlineto{\pgfqpoint{2.150186in}{1.410390in}}%
\pgfpathlineto{\pgfqpoint{2.129353in}{1.431224in}}%
\pgfpathlineto{\pgfqpoint{2.108520in}{1.410390in}}%
\pgfpathlineto{\pgfqpoint{2.129353in}{1.389557in}}%
\pgfpathlineto{\pgfqpoint{2.108520in}{1.368724in}}%
\pgfpathclose%
\pgfusepath{stroke,fill}%
\end{pgfscope}%
\begin{pgfscope}%
\pgfpathrectangle{\pgfqpoint{0.424692in}{0.370679in}}{\pgfqpoint{2.725308in}{1.479321in}}%
\pgfusepath{clip}%
\pgfsetbuttcap%
\pgfsetroundjoin%
\definecolor{currentfill}{rgb}{0.466667,0.866667,0.466667}%
\pgfsetfillcolor{currentfill}%
\pgfsetfillopacity{0.500000}%
\pgfsetlinewidth{1.003750pt}%
\definecolor{currentstroke}{rgb}{0.466667,0.866667,0.466667}%
\pgfsetstrokecolor{currentstroke}%
\pgfsetstrokeopacity{0.500000}%
\pgfsetdash{}{0pt}%
\pgfpathmoveto{\pgfqpoint{2.147384in}{1.305528in}}%
\pgfpathlineto{\pgfqpoint{2.168217in}{1.326362in}}%
\pgfpathlineto{\pgfqpoint{2.189050in}{1.305528in}}%
\pgfpathlineto{\pgfqpoint{2.209884in}{1.326362in}}%
\pgfpathlineto{\pgfqpoint{2.189050in}{1.347195in}}%
\pgfpathlineto{\pgfqpoint{2.209884in}{1.368028in}}%
\pgfpathlineto{\pgfqpoint{2.189050in}{1.388862in}}%
\pgfpathlineto{\pgfqpoint{2.168217in}{1.368028in}}%
\pgfpathlineto{\pgfqpoint{2.147384in}{1.388862in}}%
\pgfpathlineto{\pgfqpoint{2.126550in}{1.368028in}}%
\pgfpathlineto{\pgfqpoint{2.147384in}{1.347195in}}%
\pgfpathlineto{\pgfqpoint{2.126550in}{1.326362in}}%
\pgfpathclose%
\pgfusepath{stroke,fill}%
\end{pgfscope}%
\begin{pgfscope}%
\pgfpathrectangle{\pgfqpoint{0.424692in}{0.370679in}}{\pgfqpoint{2.725308in}{1.479321in}}%
\pgfusepath{clip}%
\pgfsetbuttcap%
\pgfsetroundjoin%
\definecolor{currentfill}{rgb}{0.466667,0.866667,0.466667}%
\pgfsetfillcolor{currentfill}%
\pgfsetfillopacity{0.500000}%
\pgfsetlinewidth{1.003750pt}%
\definecolor{currentstroke}{rgb}{0.466667,0.866667,0.466667}%
\pgfsetstrokecolor{currentstroke}%
\pgfsetstrokeopacity{0.500000}%
\pgfsetdash{}{0pt}%
\pgfpathmoveto{\pgfqpoint{2.235976in}{1.440651in}}%
\pgfpathlineto{\pgfqpoint{2.256809in}{1.461485in}}%
\pgfpathlineto{\pgfqpoint{2.277643in}{1.440651in}}%
\pgfpathlineto{\pgfqpoint{2.298476in}{1.461485in}}%
\pgfpathlineto{\pgfqpoint{2.277643in}{1.482318in}}%
\pgfpathlineto{\pgfqpoint{2.298476in}{1.503151in}}%
\pgfpathlineto{\pgfqpoint{2.277643in}{1.523985in}}%
\pgfpathlineto{\pgfqpoint{2.256809in}{1.503151in}}%
\pgfpathlineto{\pgfqpoint{2.235976in}{1.523985in}}%
\pgfpathlineto{\pgfqpoint{2.215143in}{1.503151in}}%
\pgfpathlineto{\pgfqpoint{2.235976in}{1.482318in}}%
\pgfpathlineto{\pgfqpoint{2.215143in}{1.461485in}}%
\pgfpathclose%
\pgfusepath{stroke,fill}%
\end{pgfscope}%
\begin{pgfscope}%
\pgfpathrectangle{\pgfqpoint{0.424692in}{0.370679in}}{\pgfqpoint{2.725308in}{1.479321in}}%
\pgfusepath{clip}%
\pgfsetbuttcap%
\pgfsetroundjoin%
\definecolor{currentfill}{rgb}{0.466667,0.866667,0.466667}%
\pgfsetfillcolor{currentfill}%
\pgfsetfillopacity{0.500000}%
\pgfsetlinewidth{1.003750pt}%
\definecolor{currentstroke}{rgb}{0.466667,0.866667,0.466667}%
\pgfsetstrokecolor{currentstroke}%
\pgfsetstrokeopacity{0.500000}%
\pgfsetdash{}{0pt}%
\pgfpathmoveto{\pgfqpoint{2.347398in}{1.390987in}}%
\pgfpathlineto{\pgfqpoint{2.368232in}{1.411820in}}%
\pgfpathlineto{\pgfqpoint{2.389065in}{1.390987in}}%
\pgfpathlineto{\pgfqpoint{2.409898in}{1.411820in}}%
\pgfpathlineto{\pgfqpoint{2.389065in}{1.432654in}}%
\pgfpathlineto{\pgfqpoint{2.409898in}{1.453487in}}%
\pgfpathlineto{\pgfqpoint{2.389065in}{1.474320in}}%
\pgfpathlineto{\pgfqpoint{2.368232in}{1.453487in}}%
\pgfpathlineto{\pgfqpoint{2.347398in}{1.474320in}}%
\pgfpathlineto{\pgfqpoint{2.326565in}{1.453487in}}%
\pgfpathlineto{\pgfqpoint{2.347398in}{1.432654in}}%
\pgfpathlineto{\pgfqpoint{2.326565in}{1.411820in}}%
\pgfpathclose%
\pgfusepath{stroke,fill}%
\end{pgfscope}%
\begin{pgfscope}%
\pgfpathrectangle{\pgfqpoint{0.424692in}{0.370679in}}{\pgfqpoint{2.725308in}{1.479321in}}%
\pgfusepath{clip}%
\pgfsetbuttcap%
\pgfsetroundjoin%
\definecolor{currentfill}{rgb}{0.466667,0.866667,0.466667}%
\pgfsetfillcolor{currentfill}%
\pgfsetfillopacity{0.500000}%
\pgfsetlinewidth{1.003750pt}%
\definecolor{currentstroke}{rgb}{0.466667,0.866667,0.466667}%
\pgfsetstrokecolor{currentstroke}%
\pgfsetstrokeopacity{0.500000}%
\pgfsetdash{}{0pt}%
\pgfpathmoveto{\pgfqpoint{1.893933in}{1.275721in}}%
\pgfpathlineto{\pgfqpoint{1.914766in}{1.296554in}}%
\pgfpathlineto{\pgfqpoint{1.935600in}{1.275721in}}%
\pgfpathlineto{\pgfqpoint{1.956433in}{1.296554in}}%
\pgfpathlineto{\pgfqpoint{1.935600in}{1.317388in}}%
\pgfpathlineto{\pgfqpoint{1.956433in}{1.338221in}}%
\pgfpathlineto{\pgfqpoint{1.935600in}{1.359054in}}%
\pgfpathlineto{\pgfqpoint{1.914766in}{1.338221in}}%
\pgfpathlineto{\pgfqpoint{1.893933in}{1.359054in}}%
\pgfpathlineto{\pgfqpoint{1.873100in}{1.338221in}}%
\pgfpathlineto{\pgfqpoint{1.893933in}{1.317388in}}%
\pgfpathlineto{\pgfqpoint{1.873100in}{1.296554in}}%
\pgfpathclose%
\pgfusepath{stroke,fill}%
\end{pgfscope}%
\begin{pgfscope}%
\pgfpathrectangle{\pgfqpoint{0.424692in}{0.370679in}}{\pgfqpoint{2.725308in}{1.479321in}}%
\pgfusepath{clip}%
\pgfsetbuttcap%
\pgfsetroundjoin%
\definecolor{currentfill}{rgb}{0.466667,0.866667,0.466667}%
\pgfsetfillcolor{currentfill}%
\pgfsetfillopacity{0.500000}%
\pgfsetlinewidth{1.003750pt}%
\definecolor{currentstroke}{rgb}{0.466667,0.866667,0.466667}%
\pgfsetstrokecolor{currentstroke}%
\pgfsetstrokeopacity{0.500000}%
\pgfsetdash{}{0pt}%
\pgfpathmoveto{\pgfqpoint{1.947806in}{1.211849in}}%
\pgfpathlineto{\pgfqpoint{1.968639in}{1.232682in}}%
\pgfpathlineto{\pgfqpoint{1.989472in}{1.211849in}}%
\pgfpathlineto{\pgfqpoint{2.010306in}{1.232682in}}%
\pgfpathlineto{\pgfqpoint{1.989472in}{1.253516in}}%
\pgfpathlineto{\pgfqpoint{2.010306in}{1.274349in}}%
\pgfpathlineto{\pgfqpoint{1.989472in}{1.295182in}}%
\pgfpathlineto{\pgfqpoint{1.968639in}{1.274349in}}%
\pgfpathlineto{\pgfqpoint{1.947806in}{1.295182in}}%
\pgfpathlineto{\pgfqpoint{1.926972in}{1.274349in}}%
\pgfpathlineto{\pgfqpoint{1.947806in}{1.253516in}}%
\pgfpathlineto{\pgfqpoint{1.926972in}{1.232682in}}%
\pgfpathclose%
\pgfusepath{stroke,fill}%
\end{pgfscope}%
\begin{pgfscope}%
\pgfpathrectangle{\pgfqpoint{0.424692in}{0.370679in}}{\pgfqpoint{2.725308in}{1.479321in}}%
\pgfusepath{clip}%
\pgfsetbuttcap%
\pgfsetroundjoin%
\definecolor{currentfill}{rgb}{0.466667,0.866667,0.466667}%
\pgfsetfillcolor{currentfill}%
\pgfsetfillopacity{0.500000}%
\pgfsetlinewidth{1.003750pt}%
\definecolor{currentstroke}{rgb}{0.466667,0.866667,0.466667}%
\pgfsetstrokecolor{currentstroke}%
\pgfsetstrokeopacity{0.500000}%
\pgfsetdash{}{0pt}%
\pgfpathmoveto{\pgfqpoint{2.048654in}{1.382092in}}%
\pgfpathlineto{\pgfqpoint{2.069487in}{1.402926in}}%
\pgfpathlineto{\pgfqpoint{2.090321in}{1.382092in}}%
\pgfpathlineto{\pgfqpoint{2.111154in}{1.402926in}}%
\pgfpathlineto{\pgfqpoint{2.090321in}{1.423759in}}%
\pgfpathlineto{\pgfqpoint{2.111154in}{1.444592in}}%
\pgfpathlineto{\pgfqpoint{2.090321in}{1.465426in}}%
\pgfpathlineto{\pgfqpoint{2.069487in}{1.444592in}}%
\pgfpathlineto{\pgfqpoint{2.048654in}{1.465426in}}%
\pgfpathlineto{\pgfqpoint{2.027821in}{1.444592in}}%
\pgfpathlineto{\pgfqpoint{2.048654in}{1.423759in}}%
\pgfpathlineto{\pgfqpoint{2.027821in}{1.402926in}}%
\pgfpathclose%
\pgfusepath{stroke,fill}%
\end{pgfscope}%
\begin{pgfscope}%
\pgfpathrectangle{\pgfqpoint{0.424692in}{0.370679in}}{\pgfqpoint{2.725308in}{1.479321in}}%
\pgfusepath{clip}%
\pgfsetbuttcap%
\pgfsetroundjoin%
\definecolor{currentfill}{rgb}{0.466667,0.866667,0.466667}%
\pgfsetfillcolor{currentfill}%
\pgfsetfillopacity{0.500000}%
\pgfsetlinewidth{1.003750pt}%
\definecolor{currentstroke}{rgb}{0.466667,0.866667,0.466667}%
\pgfsetstrokecolor{currentstroke}%
\pgfsetstrokeopacity{0.500000}%
\pgfsetdash{}{0pt}%
\pgfpathmoveto{\pgfqpoint{1.977294in}{1.297085in}}%
\pgfpathlineto{\pgfqpoint{1.998128in}{1.317918in}}%
\pgfpathlineto{\pgfqpoint{2.018961in}{1.297085in}}%
\pgfpathlineto{\pgfqpoint{2.039794in}{1.317918in}}%
\pgfpathlineto{\pgfqpoint{2.018961in}{1.338752in}}%
\pgfpathlineto{\pgfqpoint{2.039794in}{1.359585in}}%
\pgfpathlineto{\pgfqpoint{2.018961in}{1.380418in}}%
\pgfpathlineto{\pgfqpoint{1.998128in}{1.359585in}}%
\pgfpathlineto{\pgfqpoint{1.977294in}{1.380418in}}%
\pgfpathlineto{\pgfqpoint{1.956461in}{1.359585in}}%
\pgfpathlineto{\pgfqpoint{1.977294in}{1.338752in}}%
\pgfpathlineto{\pgfqpoint{1.956461in}{1.317918in}}%
\pgfpathclose%
\pgfusepath{stroke,fill}%
\end{pgfscope}%
\begin{pgfscope}%
\pgfpathrectangle{\pgfqpoint{0.424692in}{0.370679in}}{\pgfqpoint{2.725308in}{1.479321in}}%
\pgfusepath{clip}%
\pgfsetbuttcap%
\pgfsetroundjoin%
\definecolor{currentfill}{rgb}{0.466667,0.866667,0.466667}%
\pgfsetfillcolor{currentfill}%
\pgfsetfillopacity{0.500000}%
\pgfsetlinewidth{1.003750pt}%
\definecolor{currentstroke}{rgb}{0.466667,0.866667,0.466667}%
\pgfsetstrokecolor{currentstroke}%
\pgfsetstrokeopacity{0.500000}%
\pgfsetdash{}{0pt}%
\pgfpathmoveto{\pgfqpoint{1.758165in}{1.152411in}}%
\pgfpathlineto{\pgfqpoint{1.778999in}{1.173245in}}%
\pgfpathlineto{\pgfqpoint{1.799832in}{1.152411in}}%
\pgfpathlineto{\pgfqpoint{1.820665in}{1.173245in}}%
\pgfpathlineto{\pgfqpoint{1.799832in}{1.194078in}}%
\pgfpathlineto{\pgfqpoint{1.820665in}{1.214911in}}%
\pgfpathlineto{\pgfqpoint{1.799832in}{1.235745in}}%
\pgfpathlineto{\pgfqpoint{1.778999in}{1.214911in}}%
\pgfpathlineto{\pgfqpoint{1.758165in}{1.235745in}}%
\pgfpathlineto{\pgfqpoint{1.737332in}{1.214911in}}%
\pgfpathlineto{\pgfqpoint{1.758165in}{1.194078in}}%
\pgfpathlineto{\pgfqpoint{1.737332in}{1.173245in}}%
\pgfpathclose%
\pgfusepath{stroke,fill}%
\end{pgfscope}%
\begin{pgfscope}%
\pgfpathrectangle{\pgfqpoint{0.424692in}{0.370679in}}{\pgfqpoint{2.725308in}{1.479321in}}%
\pgfusepath{clip}%
\pgfsetbuttcap%
\pgfsetroundjoin%
\definecolor{currentfill}{rgb}{0.466667,0.866667,0.466667}%
\pgfsetfillcolor{currentfill}%
\pgfsetfillopacity{0.500000}%
\pgfsetlinewidth{1.003750pt}%
\definecolor{currentstroke}{rgb}{0.466667,0.866667,0.466667}%
\pgfsetstrokecolor{currentstroke}%
\pgfsetstrokeopacity{0.500000}%
\pgfsetdash{}{0pt}%
\pgfpathmoveto{\pgfqpoint{1.610465in}{1.094751in}}%
\pgfpathlineto{\pgfqpoint{1.631298in}{1.115584in}}%
\pgfpathlineto{\pgfqpoint{1.652131in}{1.094751in}}%
\pgfpathlineto{\pgfqpoint{1.672965in}{1.115584in}}%
\pgfpathlineto{\pgfqpoint{1.652131in}{1.136418in}}%
\pgfpathlineto{\pgfqpoint{1.672965in}{1.157251in}}%
\pgfpathlineto{\pgfqpoint{1.652131in}{1.178084in}}%
\pgfpathlineto{\pgfqpoint{1.631298in}{1.157251in}}%
\pgfpathlineto{\pgfqpoint{1.610465in}{1.178084in}}%
\pgfpathlineto{\pgfqpoint{1.589631in}{1.157251in}}%
\pgfpathlineto{\pgfqpoint{1.610465in}{1.136418in}}%
\pgfpathlineto{\pgfqpoint{1.589631in}{1.115584in}}%
\pgfpathclose%
\pgfusepath{stroke,fill}%
\end{pgfscope}%
\begin{pgfscope}%
\pgfpathrectangle{\pgfqpoint{0.424692in}{0.370679in}}{\pgfqpoint{2.725308in}{1.479321in}}%
\pgfusepath{clip}%
\pgfsetbuttcap%
\pgfsetroundjoin%
\definecolor{currentfill}{rgb}{0.466667,0.866667,0.466667}%
\pgfsetfillcolor{currentfill}%
\pgfsetfillopacity{0.500000}%
\pgfsetlinewidth{1.003750pt}%
\definecolor{currentstroke}{rgb}{0.466667,0.866667,0.466667}%
\pgfsetstrokecolor{currentstroke}%
\pgfsetstrokeopacity{0.500000}%
\pgfsetdash{}{0pt}%
\pgfpathmoveto{\pgfqpoint{1.633252in}{1.160110in}}%
\pgfpathlineto{\pgfqpoint{1.654086in}{1.180943in}}%
\pgfpathlineto{\pgfqpoint{1.674919in}{1.160110in}}%
\pgfpathlineto{\pgfqpoint{1.695752in}{1.180943in}}%
\pgfpathlineto{\pgfqpoint{1.674919in}{1.201776in}}%
\pgfpathlineto{\pgfqpoint{1.695752in}{1.222610in}}%
\pgfpathlineto{\pgfqpoint{1.674919in}{1.243443in}}%
\pgfpathlineto{\pgfqpoint{1.654086in}{1.222610in}}%
\pgfpathlineto{\pgfqpoint{1.633252in}{1.243443in}}%
\pgfpathlineto{\pgfqpoint{1.612419in}{1.222610in}}%
\pgfpathlineto{\pgfqpoint{1.633252in}{1.201776in}}%
\pgfpathlineto{\pgfqpoint{1.612419in}{1.180943in}}%
\pgfpathclose%
\pgfusepath{stroke,fill}%
\end{pgfscope}%
\begin{pgfscope}%
\pgfpathrectangle{\pgfqpoint{0.424692in}{0.370679in}}{\pgfqpoint{2.725308in}{1.479321in}}%
\pgfusepath{clip}%
\pgfsetbuttcap%
\pgfsetroundjoin%
\definecolor{currentfill}{rgb}{0.466667,0.866667,0.466667}%
\pgfsetfillcolor{currentfill}%
\pgfsetfillopacity{0.500000}%
\pgfsetlinewidth{1.003750pt}%
\definecolor{currentstroke}{rgb}{0.466667,0.866667,0.466667}%
\pgfsetstrokecolor{currentstroke}%
\pgfsetstrokeopacity{0.500000}%
\pgfsetdash{}{0pt}%
\pgfpathmoveto{\pgfqpoint{1.597841in}{1.168608in}}%
\pgfpathlineto{\pgfqpoint{1.618674in}{1.189441in}}%
\pgfpathlineto{\pgfqpoint{1.639508in}{1.168608in}}%
\pgfpathlineto{\pgfqpoint{1.660341in}{1.189441in}}%
\pgfpathlineto{\pgfqpoint{1.639508in}{1.210274in}}%
\pgfpathlineto{\pgfqpoint{1.660341in}{1.231108in}}%
\pgfpathlineto{\pgfqpoint{1.639508in}{1.251941in}}%
\pgfpathlineto{\pgfqpoint{1.618674in}{1.231108in}}%
\pgfpathlineto{\pgfqpoint{1.597841in}{1.251941in}}%
\pgfpathlineto{\pgfqpoint{1.577008in}{1.231108in}}%
\pgfpathlineto{\pgfqpoint{1.597841in}{1.210274in}}%
\pgfpathlineto{\pgfqpoint{1.577008in}{1.189441in}}%
\pgfpathclose%
\pgfusepath{stroke,fill}%
\end{pgfscope}%
\begin{pgfscope}%
\pgfpathrectangle{\pgfqpoint{0.424692in}{0.370679in}}{\pgfqpoint{2.725308in}{1.479321in}}%
\pgfusepath{clip}%
\pgfsetbuttcap%
\pgfsetroundjoin%
\definecolor{currentfill}{rgb}{0.466667,0.866667,0.466667}%
\pgfsetfillcolor{currentfill}%
\pgfsetfillopacity{0.500000}%
\pgfsetlinewidth{1.003750pt}%
\definecolor{currentstroke}{rgb}{0.466667,0.866667,0.466667}%
\pgfsetstrokecolor{currentstroke}%
\pgfsetstrokeopacity{0.500000}%
\pgfsetdash{}{0pt}%
\pgfpathmoveto{\pgfqpoint{1.975773in}{1.283004in}}%
\pgfpathlineto{\pgfqpoint{1.996607in}{1.303837in}}%
\pgfpathlineto{\pgfqpoint{2.017440in}{1.283004in}}%
\pgfpathlineto{\pgfqpoint{2.038273in}{1.303837in}}%
\pgfpathlineto{\pgfqpoint{2.017440in}{1.324671in}}%
\pgfpathlineto{\pgfqpoint{2.038273in}{1.345504in}}%
\pgfpathlineto{\pgfqpoint{2.017440in}{1.366337in}}%
\pgfpathlineto{\pgfqpoint{1.996607in}{1.345504in}}%
\pgfpathlineto{\pgfqpoint{1.975773in}{1.366337in}}%
\pgfpathlineto{\pgfqpoint{1.954940in}{1.345504in}}%
\pgfpathlineto{\pgfqpoint{1.975773in}{1.324671in}}%
\pgfpathlineto{\pgfqpoint{1.954940in}{1.303837in}}%
\pgfpathclose%
\pgfusepath{stroke,fill}%
\end{pgfscope}%
\begin{pgfscope}%
\pgfpathrectangle{\pgfqpoint{0.424692in}{0.370679in}}{\pgfqpoint{2.725308in}{1.479321in}}%
\pgfusepath{clip}%
\pgfsetbuttcap%
\pgfsetroundjoin%
\definecolor{currentfill}{rgb}{0.466667,0.866667,0.466667}%
\pgfsetfillcolor{currentfill}%
\pgfsetfillopacity{0.500000}%
\pgfsetlinewidth{1.003750pt}%
\definecolor{currentstroke}{rgb}{0.466667,0.866667,0.466667}%
\pgfsetstrokecolor{currentstroke}%
\pgfsetstrokeopacity{0.500000}%
\pgfsetdash{}{0pt}%
\pgfpathmoveto{\pgfqpoint{2.223803in}{1.280653in}}%
\pgfpathlineto{\pgfqpoint{2.244637in}{1.301487in}}%
\pgfpathlineto{\pgfqpoint{2.265470in}{1.280653in}}%
\pgfpathlineto{\pgfqpoint{2.286303in}{1.301487in}}%
\pgfpathlineto{\pgfqpoint{2.265470in}{1.322320in}}%
\pgfpathlineto{\pgfqpoint{2.286303in}{1.343153in}}%
\pgfpathlineto{\pgfqpoint{2.265470in}{1.363987in}}%
\pgfpathlineto{\pgfqpoint{2.244637in}{1.343153in}}%
\pgfpathlineto{\pgfqpoint{2.223803in}{1.363987in}}%
\pgfpathlineto{\pgfqpoint{2.202970in}{1.343153in}}%
\pgfpathlineto{\pgfqpoint{2.223803in}{1.322320in}}%
\pgfpathlineto{\pgfqpoint{2.202970in}{1.301487in}}%
\pgfpathclose%
\pgfusepath{stroke,fill}%
\end{pgfscope}%
\begin{pgfscope}%
\pgfpathrectangle{\pgfqpoint{0.424692in}{0.370679in}}{\pgfqpoint{2.725308in}{1.479321in}}%
\pgfusepath{clip}%
\pgfsetbuttcap%
\pgfsetroundjoin%
\definecolor{currentfill}{rgb}{0.466667,0.866667,0.466667}%
\pgfsetfillcolor{currentfill}%
\pgfsetfillopacity{0.500000}%
\pgfsetlinewidth{1.003750pt}%
\definecolor{currentstroke}{rgb}{0.466667,0.866667,0.466667}%
\pgfsetstrokecolor{currentstroke}%
\pgfsetstrokeopacity{0.500000}%
\pgfsetdash{}{0pt}%
\pgfpathmoveto{\pgfqpoint{2.445512in}{1.430481in}}%
\pgfpathlineto{\pgfqpoint{2.466345in}{1.451314in}}%
\pgfpathlineto{\pgfqpoint{2.487179in}{1.430481in}}%
\pgfpathlineto{\pgfqpoint{2.508012in}{1.451314in}}%
\pgfpathlineto{\pgfqpoint{2.487179in}{1.472147in}}%
\pgfpathlineto{\pgfqpoint{2.508012in}{1.492981in}}%
\pgfpathlineto{\pgfqpoint{2.487179in}{1.513814in}}%
\pgfpathlineto{\pgfqpoint{2.466345in}{1.492981in}}%
\pgfpathlineto{\pgfqpoint{2.445512in}{1.513814in}}%
\pgfpathlineto{\pgfqpoint{2.424679in}{1.492981in}}%
\pgfpathlineto{\pgfqpoint{2.445512in}{1.472147in}}%
\pgfpathlineto{\pgfqpoint{2.424679in}{1.451314in}}%
\pgfpathclose%
\pgfusepath{stroke,fill}%
\end{pgfscope}%
\begin{pgfscope}%
\pgfpathrectangle{\pgfqpoint{0.424692in}{0.370679in}}{\pgfqpoint{2.725308in}{1.479321in}}%
\pgfusepath{clip}%
\pgfsetbuttcap%
\pgfsetroundjoin%
\definecolor{currentfill}{rgb}{0.466667,0.866667,0.466667}%
\pgfsetfillcolor{currentfill}%
\pgfsetfillopacity{0.500000}%
\pgfsetlinewidth{1.003750pt}%
\definecolor{currentstroke}{rgb}{0.466667,0.866667,0.466667}%
\pgfsetstrokecolor{currentstroke}%
\pgfsetstrokeopacity{0.500000}%
\pgfsetdash{}{0pt}%
\pgfpathmoveto{\pgfqpoint{2.437999in}{1.411471in}}%
\pgfpathlineto{\pgfqpoint{2.458832in}{1.432304in}}%
\pgfpathlineto{\pgfqpoint{2.479665in}{1.411471in}}%
\pgfpathlineto{\pgfqpoint{2.500499in}{1.432304in}}%
\pgfpathlineto{\pgfqpoint{2.479665in}{1.453138in}}%
\pgfpathlineto{\pgfqpoint{2.500499in}{1.473971in}}%
\pgfpathlineto{\pgfqpoint{2.479665in}{1.494804in}}%
\pgfpathlineto{\pgfqpoint{2.458832in}{1.473971in}}%
\pgfpathlineto{\pgfqpoint{2.437999in}{1.494804in}}%
\pgfpathlineto{\pgfqpoint{2.417165in}{1.473971in}}%
\pgfpathlineto{\pgfqpoint{2.437999in}{1.453138in}}%
\pgfpathlineto{\pgfqpoint{2.417165in}{1.432304in}}%
\pgfpathclose%
\pgfusepath{stroke,fill}%
\end{pgfscope}%
\begin{pgfscope}%
\pgfpathrectangle{\pgfqpoint{0.424692in}{0.370679in}}{\pgfqpoint{2.725308in}{1.479321in}}%
\pgfusepath{clip}%
\pgfsetbuttcap%
\pgfsetroundjoin%
\definecolor{currentfill}{rgb}{0.466667,0.866667,0.466667}%
\pgfsetfillcolor{currentfill}%
\pgfsetfillopacity{0.500000}%
\pgfsetlinewidth{1.003750pt}%
\definecolor{currentstroke}{rgb}{0.466667,0.866667,0.466667}%
\pgfsetstrokecolor{currentstroke}%
\pgfsetstrokeopacity{0.500000}%
\pgfsetdash{}{0pt}%
\pgfpathmoveto{\pgfqpoint{2.145894in}{1.292512in}}%
\pgfpathlineto{\pgfqpoint{2.166727in}{1.313345in}}%
\pgfpathlineto{\pgfqpoint{2.187561in}{1.292512in}}%
\pgfpathlineto{\pgfqpoint{2.208394in}{1.313345in}}%
\pgfpathlineto{\pgfqpoint{2.187561in}{1.334178in}}%
\pgfpathlineto{\pgfqpoint{2.208394in}{1.355012in}}%
\pgfpathlineto{\pgfqpoint{2.187561in}{1.375845in}}%
\pgfpathlineto{\pgfqpoint{2.166727in}{1.355012in}}%
\pgfpathlineto{\pgfqpoint{2.145894in}{1.375845in}}%
\pgfpathlineto{\pgfqpoint{2.125061in}{1.355012in}}%
\pgfpathlineto{\pgfqpoint{2.145894in}{1.334178in}}%
\pgfpathlineto{\pgfqpoint{2.125061in}{1.313345in}}%
\pgfpathclose%
\pgfusepath{stroke,fill}%
\end{pgfscope}%
\begin{pgfscope}%
\pgfpathrectangle{\pgfqpoint{0.424692in}{0.370679in}}{\pgfqpoint{2.725308in}{1.479321in}}%
\pgfusepath{clip}%
\pgfsetbuttcap%
\pgfsetroundjoin%
\definecolor{currentfill}{rgb}{0.466667,0.866667,0.466667}%
\pgfsetfillcolor{currentfill}%
\pgfsetfillopacity{0.500000}%
\pgfsetlinewidth{1.003750pt}%
\definecolor{currentstroke}{rgb}{0.466667,0.866667,0.466667}%
\pgfsetstrokecolor{currentstroke}%
\pgfsetstrokeopacity{0.500000}%
\pgfsetdash{}{0pt}%
\pgfpathmoveto{\pgfqpoint{2.112650in}{1.427051in}}%
\pgfpathlineto{\pgfqpoint{2.133484in}{1.447885in}}%
\pgfpathlineto{\pgfqpoint{2.154317in}{1.427051in}}%
\pgfpathlineto{\pgfqpoint{2.175150in}{1.447885in}}%
\pgfpathlineto{\pgfqpoint{2.154317in}{1.468718in}}%
\pgfpathlineto{\pgfqpoint{2.175150in}{1.489551in}}%
\pgfpathlineto{\pgfqpoint{2.154317in}{1.510385in}}%
\pgfpathlineto{\pgfqpoint{2.133484in}{1.489551in}}%
\pgfpathlineto{\pgfqpoint{2.112650in}{1.510385in}}%
\pgfpathlineto{\pgfqpoint{2.091817in}{1.489551in}}%
\pgfpathlineto{\pgfqpoint{2.112650in}{1.468718in}}%
\pgfpathlineto{\pgfqpoint{2.091817in}{1.447885in}}%
\pgfpathclose%
\pgfusepath{stroke,fill}%
\end{pgfscope}%
\begin{pgfscope}%
\pgfpathrectangle{\pgfqpoint{0.424692in}{0.370679in}}{\pgfqpoint{2.725308in}{1.479321in}}%
\pgfusepath{clip}%
\pgfsetbuttcap%
\pgfsetroundjoin%
\definecolor{currentfill}{rgb}{0.466667,0.866667,0.466667}%
\pgfsetfillcolor{currentfill}%
\pgfsetfillopacity{0.500000}%
\pgfsetlinewidth{1.003750pt}%
\definecolor{currentstroke}{rgb}{0.466667,0.866667,0.466667}%
\pgfsetstrokecolor{currentstroke}%
\pgfsetstrokeopacity{0.500000}%
\pgfsetdash{}{0pt}%
\pgfpathmoveto{\pgfqpoint{2.229027in}{1.393292in}}%
\pgfpathlineto{\pgfqpoint{2.249861in}{1.414125in}}%
\pgfpathlineto{\pgfqpoint{2.270694in}{1.393292in}}%
\pgfpathlineto{\pgfqpoint{2.291527in}{1.414125in}}%
\pgfpathlineto{\pgfqpoint{2.270694in}{1.434959in}}%
\pgfpathlineto{\pgfqpoint{2.291527in}{1.455792in}}%
\pgfpathlineto{\pgfqpoint{2.270694in}{1.476625in}}%
\pgfpathlineto{\pgfqpoint{2.249861in}{1.455792in}}%
\pgfpathlineto{\pgfqpoint{2.229027in}{1.476625in}}%
\pgfpathlineto{\pgfqpoint{2.208194in}{1.455792in}}%
\pgfpathlineto{\pgfqpoint{2.229027in}{1.434959in}}%
\pgfpathlineto{\pgfqpoint{2.208194in}{1.414125in}}%
\pgfpathclose%
\pgfusepath{stroke,fill}%
\end{pgfscope}%
\begin{pgfscope}%
\pgfpathrectangle{\pgfqpoint{0.424692in}{0.370679in}}{\pgfqpoint{2.725308in}{1.479321in}}%
\pgfusepath{clip}%
\pgfsetbuttcap%
\pgfsetroundjoin%
\definecolor{currentfill}{rgb}{0.466667,0.866667,0.466667}%
\pgfsetfillcolor{currentfill}%
\pgfsetfillopacity{0.500000}%
\pgfsetlinewidth{1.003750pt}%
\definecolor{currentstroke}{rgb}{0.466667,0.866667,0.466667}%
\pgfsetstrokecolor{currentstroke}%
\pgfsetstrokeopacity{0.500000}%
\pgfsetdash{}{0pt}%
\pgfpathmoveto{\pgfqpoint{1.718027in}{1.141988in}}%
\pgfpathlineto{\pgfqpoint{1.738860in}{1.162821in}}%
\pgfpathlineto{\pgfqpoint{1.759694in}{1.141988in}}%
\pgfpathlineto{\pgfqpoint{1.780527in}{1.162821in}}%
\pgfpathlineto{\pgfqpoint{1.759694in}{1.183654in}}%
\pgfpathlineto{\pgfqpoint{1.780527in}{1.204488in}}%
\pgfpathlineto{\pgfqpoint{1.759694in}{1.225321in}}%
\pgfpathlineto{\pgfqpoint{1.738860in}{1.204488in}}%
\pgfpathlineto{\pgfqpoint{1.718027in}{1.225321in}}%
\pgfpathlineto{\pgfqpoint{1.697194in}{1.204488in}}%
\pgfpathlineto{\pgfqpoint{1.718027in}{1.183654in}}%
\pgfpathlineto{\pgfqpoint{1.697194in}{1.162821in}}%
\pgfpathclose%
\pgfusepath{stroke,fill}%
\end{pgfscope}%
\begin{pgfscope}%
\pgfpathrectangle{\pgfqpoint{0.424692in}{0.370679in}}{\pgfqpoint{2.725308in}{1.479321in}}%
\pgfusepath{clip}%
\pgfsetbuttcap%
\pgfsetroundjoin%
\definecolor{currentfill}{rgb}{0.466667,0.866667,0.466667}%
\pgfsetfillcolor{currentfill}%
\pgfsetfillopacity{0.500000}%
\pgfsetlinewidth{1.003750pt}%
\definecolor{currentstroke}{rgb}{0.466667,0.866667,0.466667}%
\pgfsetstrokecolor{currentstroke}%
\pgfsetstrokeopacity{0.500000}%
\pgfsetdash{}{0pt}%
\pgfpathmoveto{\pgfqpoint{1.526769in}{1.088546in}}%
\pgfpathlineto{\pgfqpoint{1.547603in}{1.109379in}}%
\pgfpathlineto{\pgfqpoint{1.568436in}{1.088546in}}%
\pgfpathlineto{\pgfqpoint{1.589269in}{1.109379in}}%
\pgfpathlineto{\pgfqpoint{1.568436in}{1.130213in}}%
\pgfpathlineto{\pgfqpoint{1.589269in}{1.151046in}}%
\pgfpathlineto{\pgfqpoint{1.568436in}{1.171879in}}%
\pgfpathlineto{\pgfqpoint{1.547603in}{1.151046in}}%
\pgfpathlineto{\pgfqpoint{1.526769in}{1.171879in}}%
\pgfpathlineto{\pgfqpoint{1.505936in}{1.151046in}}%
\pgfpathlineto{\pgfqpoint{1.526769in}{1.130213in}}%
\pgfpathlineto{\pgfqpoint{1.505936in}{1.109379in}}%
\pgfpathclose%
\pgfusepath{stroke,fill}%
\end{pgfscope}%
\begin{pgfscope}%
\pgfpathrectangle{\pgfqpoint{0.424692in}{0.370679in}}{\pgfqpoint{2.725308in}{1.479321in}}%
\pgfusepath{clip}%
\pgfsetbuttcap%
\pgfsetroundjoin%
\definecolor{currentfill}{rgb}{0.466667,0.866667,0.466667}%
\pgfsetfillcolor{currentfill}%
\pgfsetfillopacity{0.500000}%
\pgfsetlinewidth{1.003750pt}%
\definecolor{currentstroke}{rgb}{0.466667,0.866667,0.466667}%
\pgfsetstrokecolor{currentstroke}%
\pgfsetstrokeopacity{0.500000}%
\pgfsetdash{}{0pt}%
\pgfpathmoveto{\pgfqpoint{2.123615in}{1.363566in}}%
\pgfpathlineto{\pgfqpoint{2.144448in}{1.384399in}}%
\pgfpathlineto{\pgfqpoint{2.165282in}{1.363566in}}%
\pgfpathlineto{\pgfqpoint{2.186115in}{1.384399in}}%
\pgfpathlineto{\pgfqpoint{2.165282in}{1.405232in}}%
\pgfpathlineto{\pgfqpoint{2.186115in}{1.426066in}}%
\pgfpathlineto{\pgfqpoint{2.165282in}{1.446899in}}%
\pgfpathlineto{\pgfqpoint{2.144448in}{1.426066in}}%
\pgfpathlineto{\pgfqpoint{2.123615in}{1.446899in}}%
\pgfpathlineto{\pgfqpoint{2.102782in}{1.426066in}}%
\pgfpathlineto{\pgfqpoint{2.123615in}{1.405232in}}%
\pgfpathlineto{\pgfqpoint{2.102782in}{1.384399in}}%
\pgfpathclose%
\pgfusepath{stroke,fill}%
\end{pgfscope}%
\begin{pgfscope}%
\pgfpathrectangle{\pgfqpoint{0.424692in}{0.370679in}}{\pgfqpoint{2.725308in}{1.479321in}}%
\pgfusepath{clip}%
\pgfsetbuttcap%
\pgfsetroundjoin%
\definecolor{currentfill}{rgb}{0.466667,0.866667,0.466667}%
\pgfsetfillcolor{currentfill}%
\pgfsetfillopacity{0.500000}%
\pgfsetlinewidth{1.003750pt}%
\definecolor{currentstroke}{rgb}{0.466667,0.866667,0.466667}%
\pgfsetstrokecolor{currentstroke}%
\pgfsetstrokeopacity{0.500000}%
\pgfsetdash{}{0pt}%
\pgfpathmoveto{\pgfqpoint{2.103444in}{1.282917in}}%
\pgfpathlineto{\pgfqpoint{2.124278in}{1.303750in}}%
\pgfpathlineto{\pgfqpoint{2.145111in}{1.282917in}}%
\pgfpathlineto{\pgfqpoint{2.165944in}{1.303750in}}%
\pgfpathlineto{\pgfqpoint{2.145111in}{1.324583in}}%
\pgfpathlineto{\pgfqpoint{2.165944in}{1.345417in}}%
\pgfpathlineto{\pgfqpoint{2.145111in}{1.366250in}}%
\pgfpathlineto{\pgfqpoint{2.124278in}{1.345417in}}%
\pgfpathlineto{\pgfqpoint{2.103444in}{1.366250in}}%
\pgfpathlineto{\pgfqpoint{2.082611in}{1.345417in}}%
\pgfpathlineto{\pgfqpoint{2.103444in}{1.324583in}}%
\pgfpathlineto{\pgfqpoint{2.082611in}{1.303750in}}%
\pgfpathclose%
\pgfusepath{stroke,fill}%
\end{pgfscope}%
\begin{pgfscope}%
\pgfpathrectangle{\pgfqpoint{0.424692in}{0.370679in}}{\pgfqpoint{2.725308in}{1.479321in}}%
\pgfusepath{clip}%
\pgfsetbuttcap%
\pgfsetroundjoin%
\definecolor{currentfill}{rgb}{0.466667,0.866667,0.466667}%
\pgfsetfillcolor{currentfill}%
\pgfsetfillopacity{0.500000}%
\pgfsetlinewidth{1.003750pt}%
\definecolor{currentstroke}{rgb}{0.466667,0.866667,0.466667}%
\pgfsetstrokecolor{currentstroke}%
\pgfsetstrokeopacity{0.500000}%
\pgfsetdash{}{0pt}%
\pgfpathmoveto{\pgfqpoint{2.407936in}{1.438840in}}%
\pgfpathlineto{\pgfqpoint{2.428769in}{1.459673in}}%
\pgfpathlineto{\pgfqpoint{2.449602in}{1.438840in}}%
\pgfpathlineto{\pgfqpoint{2.470436in}{1.459673in}}%
\pgfpathlineto{\pgfqpoint{2.449602in}{1.480507in}}%
\pgfpathlineto{\pgfqpoint{2.470436in}{1.501340in}}%
\pgfpathlineto{\pgfqpoint{2.449602in}{1.522173in}}%
\pgfpathlineto{\pgfqpoint{2.428769in}{1.501340in}}%
\pgfpathlineto{\pgfqpoint{2.407936in}{1.522173in}}%
\pgfpathlineto{\pgfqpoint{2.387102in}{1.501340in}}%
\pgfpathlineto{\pgfqpoint{2.407936in}{1.480507in}}%
\pgfpathlineto{\pgfqpoint{2.387102in}{1.459673in}}%
\pgfpathclose%
\pgfusepath{stroke,fill}%
\end{pgfscope}%
\begin{pgfscope}%
\pgfpathrectangle{\pgfqpoint{0.424692in}{0.370679in}}{\pgfqpoint{2.725308in}{1.479321in}}%
\pgfusepath{clip}%
\pgfsetbuttcap%
\pgfsetroundjoin%
\definecolor{currentfill}{rgb}{0.466667,0.866667,0.466667}%
\pgfsetfillcolor{currentfill}%
\pgfsetfillopacity{0.500000}%
\pgfsetlinewidth{1.003750pt}%
\definecolor{currentstroke}{rgb}{0.466667,0.866667,0.466667}%
\pgfsetstrokecolor{currentstroke}%
\pgfsetstrokeopacity{0.500000}%
\pgfsetdash{}{0pt}%
\pgfpathmoveto{\pgfqpoint{2.374038in}{1.457018in}}%
\pgfpathlineto{\pgfqpoint{2.394871in}{1.477851in}}%
\pgfpathlineto{\pgfqpoint{2.415705in}{1.457018in}}%
\pgfpathlineto{\pgfqpoint{2.436538in}{1.477851in}}%
\pgfpathlineto{\pgfqpoint{2.415705in}{1.498684in}}%
\pgfpathlineto{\pgfqpoint{2.436538in}{1.519518in}}%
\pgfpathlineto{\pgfqpoint{2.415705in}{1.540351in}}%
\pgfpathlineto{\pgfqpoint{2.394871in}{1.519518in}}%
\pgfpathlineto{\pgfqpoint{2.374038in}{1.540351in}}%
\pgfpathlineto{\pgfqpoint{2.353205in}{1.519518in}}%
\pgfpathlineto{\pgfqpoint{2.374038in}{1.498684in}}%
\pgfpathlineto{\pgfqpoint{2.353205in}{1.477851in}}%
\pgfpathclose%
\pgfusepath{stroke,fill}%
\end{pgfscope}%
\begin{pgfscope}%
\pgfpathrectangle{\pgfqpoint{0.424692in}{0.370679in}}{\pgfqpoint{2.725308in}{1.479321in}}%
\pgfusepath{clip}%
\pgfsetbuttcap%
\pgfsetroundjoin%
\definecolor{currentfill}{rgb}{0.466667,0.866667,0.466667}%
\pgfsetfillcolor{currentfill}%
\pgfsetfillopacity{0.500000}%
\pgfsetlinewidth{1.003750pt}%
\definecolor{currentstroke}{rgb}{0.466667,0.866667,0.466667}%
\pgfsetstrokecolor{currentstroke}%
\pgfsetstrokeopacity{0.500000}%
\pgfsetdash{}{0pt}%
\pgfpathmoveto{\pgfqpoint{1.873884in}{1.237045in}}%
\pgfpathlineto{\pgfqpoint{1.894717in}{1.257879in}}%
\pgfpathlineto{\pgfqpoint{1.915550in}{1.237045in}}%
\pgfpathlineto{\pgfqpoint{1.936384in}{1.257879in}}%
\pgfpathlineto{\pgfqpoint{1.915550in}{1.278712in}}%
\pgfpathlineto{\pgfqpoint{1.936384in}{1.299545in}}%
\pgfpathlineto{\pgfqpoint{1.915550in}{1.320379in}}%
\pgfpathlineto{\pgfqpoint{1.894717in}{1.299545in}}%
\pgfpathlineto{\pgfqpoint{1.873884in}{1.320379in}}%
\pgfpathlineto{\pgfqpoint{1.853050in}{1.299545in}}%
\pgfpathlineto{\pgfqpoint{1.873884in}{1.278712in}}%
\pgfpathlineto{\pgfqpoint{1.853050in}{1.257879in}}%
\pgfpathclose%
\pgfusepath{stroke,fill}%
\end{pgfscope}%
\begin{pgfscope}%
\pgfpathrectangle{\pgfqpoint{0.424692in}{0.370679in}}{\pgfqpoint{2.725308in}{1.479321in}}%
\pgfusepath{clip}%
\pgfsetbuttcap%
\pgfsetroundjoin%
\definecolor{currentfill}{rgb}{0.466667,0.866667,0.466667}%
\pgfsetfillcolor{currentfill}%
\pgfsetfillopacity{0.500000}%
\pgfsetlinewidth{1.003750pt}%
\definecolor{currentstroke}{rgb}{0.466667,0.866667,0.466667}%
\pgfsetstrokecolor{currentstroke}%
\pgfsetstrokeopacity{0.500000}%
\pgfsetdash{}{0pt}%
\pgfpathmoveto{\pgfqpoint{2.199960in}{1.333470in}}%
\pgfpathlineto{\pgfqpoint{2.220793in}{1.354304in}}%
\pgfpathlineto{\pgfqpoint{2.241627in}{1.333470in}}%
\pgfpathlineto{\pgfqpoint{2.262460in}{1.354304in}}%
\pgfpathlineto{\pgfqpoint{2.241627in}{1.375137in}}%
\pgfpathlineto{\pgfqpoint{2.262460in}{1.395970in}}%
\pgfpathlineto{\pgfqpoint{2.241627in}{1.416804in}}%
\pgfpathlineto{\pgfqpoint{2.220793in}{1.395970in}}%
\pgfpathlineto{\pgfqpoint{2.199960in}{1.416804in}}%
\pgfpathlineto{\pgfqpoint{2.179127in}{1.395970in}}%
\pgfpathlineto{\pgfqpoint{2.199960in}{1.375137in}}%
\pgfpathlineto{\pgfqpoint{2.179127in}{1.354304in}}%
\pgfpathclose%
\pgfusepath{stroke,fill}%
\end{pgfscope}%
\begin{pgfscope}%
\pgfpathrectangle{\pgfqpoint{0.424692in}{0.370679in}}{\pgfqpoint{2.725308in}{1.479321in}}%
\pgfusepath{clip}%
\pgfsetbuttcap%
\pgfsetroundjoin%
\definecolor{currentfill}{rgb}{0.466667,0.866667,0.466667}%
\pgfsetfillcolor{currentfill}%
\pgfsetfillopacity{0.500000}%
\pgfsetlinewidth{1.003750pt}%
\definecolor{currentstroke}{rgb}{0.466667,0.866667,0.466667}%
\pgfsetstrokecolor{currentstroke}%
\pgfsetstrokeopacity{0.500000}%
\pgfsetdash{}{0pt}%
\pgfpathmoveto{\pgfqpoint{2.725094in}{1.504598in}}%
\pgfpathlineto{\pgfqpoint{2.745927in}{1.525431in}}%
\pgfpathlineto{\pgfqpoint{2.766760in}{1.504598in}}%
\pgfpathlineto{\pgfqpoint{2.787594in}{1.525431in}}%
\pgfpathlineto{\pgfqpoint{2.766760in}{1.546264in}}%
\pgfpathlineto{\pgfqpoint{2.787594in}{1.567098in}}%
\pgfpathlineto{\pgfqpoint{2.766760in}{1.587931in}}%
\pgfpathlineto{\pgfqpoint{2.745927in}{1.567098in}}%
\pgfpathlineto{\pgfqpoint{2.725094in}{1.587931in}}%
\pgfpathlineto{\pgfqpoint{2.704260in}{1.567098in}}%
\pgfpathlineto{\pgfqpoint{2.725094in}{1.546264in}}%
\pgfpathlineto{\pgfqpoint{2.704260in}{1.525431in}}%
\pgfpathclose%
\pgfusepath{stroke,fill}%
\end{pgfscope}%
\begin{pgfscope}%
\pgfpathrectangle{\pgfqpoint{0.424692in}{0.370679in}}{\pgfqpoint{2.725308in}{1.479321in}}%
\pgfusepath{clip}%
\pgfsetbuttcap%
\pgfsetroundjoin%
\definecolor{currentfill}{rgb}{0.466667,0.866667,0.466667}%
\pgfsetfillcolor{currentfill}%
\pgfsetfillopacity{0.500000}%
\pgfsetlinewidth{1.003750pt}%
\definecolor{currentstroke}{rgb}{0.466667,0.866667,0.466667}%
\pgfsetstrokecolor{currentstroke}%
\pgfsetstrokeopacity{0.500000}%
\pgfsetdash{}{0pt}%
\pgfpathmoveto{\pgfqpoint{2.264064in}{1.383933in}}%
\pgfpathlineto{\pgfqpoint{2.284897in}{1.404766in}}%
\pgfpathlineto{\pgfqpoint{2.305731in}{1.383933in}}%
\pgfpathlineto{\pgfqpoint{2.326564in}{1.404766in}}%
\pgfpathlineto{\pgfqpoint{2.305731in}{1.425600in}}%
\pgfpathlineto{\pgfqpoint{2.326564in}{1.446433in}}%
\pgfpathlineto{\pgfqpoint{2.305731in}{1.467266in}}%
\pgfpathlineto{\pgfqpoint{2.284897in}{1.446433in}}%
\pgfpathlineto{\pgfqpoint{2.264064in}{1.467266in}}%
\pgfpathlineto{\pgfqpoint{2.243231in}{1.446433in}}%
\pgfpathlineto{\pgfqpoint{2.264064in}{1.425600in}}%
\pgfpathlineto{\pgfqpoint{2.243231in}{1.404766in}}%
\pgfpathclose%
\pgfusepath{stroke,fill}%
\end{pgfscope}%
\begin{pgfscope}%
\pgfpathrectangle{\pgfqpoint{0.424692in}{0.370679in}}{\pgfqpoint{2.725308in}{1.479321in}}%
\pgfusepath{clip}%
\pgfsetbuttcap%
\pgfsetroundjoin%
\definecolor{currentfill}{rgb}{0.466667,0.866667,0.466667}%
\pgfsetfillcolor{currentfill}%
\pgfsetfillopacity{0.500000}%
\pgfsetlinewidth{1.003750pt}%
\definecolor{currentstroke}{rgb}{0.466667,0.866667,0.466667}%
\pgfsetstrokecolor{currentstroke}%
\pgfsetstrokeopacity{0.500000}%
\pgfsetdash{}{0pt}%
\pgfpathmoveto{\pgfqpoint{2.055845in}{1.267542in}}%
\pgfpathlineto{\pgfqpoint{2.076678in}{1.288375in}}%
\pgfpathlineto{\pgfqpoint{2.097512in}{1.267542in}}%
\pgfpathlineto{\pgfqpoint{2.118345in}{1.288375in}}%
\pgfpathlineto{\pgfqpoint{2.097512in}{1.309208in}}%
\pgfpathlineto{\pgfqpoint{2.118345in}{1.330042in}}%
\pgfpathlineto{\pgfqpoint{2.097512in}{1.350875in}}%
\pgfpathlineto{\pgfqpoint{2.076678in}{1.330042in}}%
\pgfpathlineto{\pgfqpoint{2.055845in}{1.350875in}}%
\pgfpathlineto{\pgfqpoint{2.035012in}{1.330042in}}%
\pgfpathlineto{\pgfqpoint{2.055845in}{1.309208in}}%
\pgfpathlineto{\pgfqpoint{2.035012in}{1.288375in}}%
\pgfpathclose%
\pgfusepath{stroke,fill}%
\end{pgfscope}%
\begin{pgfscope}%
\pgfpathrectangle{\pgfqpoint{0.424692in}{0.370679in}}{\pgfqpoint{2.725308in}{1.479321in}}%
\pgfusepath{clip}%
\pgfsetbuttcap%
\pgfsetroundjoin%
\definecolor{currentfill}{rgb}{0.466667,0.866667,0.466667}%
\pgfsetfillcolor{currentfill}%
\pgfsetfillopacity{0.500000}%
\pgfsetlinewidth{1.003750pt}%
\definecolor{currentstroke}{rgb}{0.466667,0.866667,0.466667}%
\pgfsetstrokecolor{currentstroke}%
\pgfsetstrokeopacity{0.500000}%
\pgfsetdash{}{0pt}%
\pgfpathmoveto{\pgfqpoint{2.107507in}{1.301084in}}%
\pgfpathlineto{\pgfqpoint{2.128341in}{1.321917in}}%
\pgfpathlineto{\pgfqpoint{2.149174in}{1.301084in}}%
\pgfpathlineto{\pgfqpoint{2.170007in}{1.321917in}}%
\pgfpathlineto{\pgfqpoint{2.149174in}{1.342750in}}%
\pgfpathlineto{\pgfqpoint{2.170007in}{1.363584in}}%
\pgfpathlineto{\pgfqpoint{2.149174in}{1.384417in}}%
\pgfpathlineto{\pgfqpoint{2.128341in}{1.363584in}}%
\pgfpathlineto{\pgfqpoint{2.107507in}{1.384417in}}%
\pgfpathlineto{\pgfqpoint{2.086674in}{1.363584in}}%
\pgfpathlineto{\pgfqpoint{2.107507in}{1.342750in}}%
\pgfpathlineto{\pgfqpoint{2.086674in}{1.321917in}}%
\pgfpathclose%
\pgfusepath{stroke,fill}%
\end{pgfscope}%
\begin{pgfscope}%
\pgfpathrectangle{\pgfqpoint{0.424692in}{0.370679in}}{\pgfqpoint{2.725308in}{1.479321in}}%
\pgfusepath{clip}%
\pgfsetbuttcap%
\pgfsetroundjoin%
\definecolor{currentfill}{rgb}{0.466667,0.866667,0.466667}%
\pgfsetfillcolor{currentfill}%
\pgfsetfillopacity{0.500000}%
\pgfsetlinewidth{1.003750pt}%
\definecolor{currentstroke}{rgb}{0.466667,0.866667,0.466667}%
\pgfsetstrokecolor{currentstroke}%
\pgfsetstrokeopacity{0.500000}%
\pgfsetdash{}{0pt}%
\pgfpathmoveto{\pgfqpoint{1.859223in}{1.198571in}}%
\pgfpathlineto{\pgfqpoint{1.880057in}{1.219404in}}%
\pgfpathlineto{\pgfqpoint{1.900890in}{1.198571in}}%
\pgfpathlineto{\pgfqpoint{1.921723in}{1.219404in}}%
\pgfpathlineto{\pgfqpoint{1.900890in}{1.240237in}}%
\pgfpathlineto{\pgfqpoint{1.921723in}{1.261071in}}%
\pgfpathlineto{\pgfqpoint{1.900890in}{1.281904in}}%
\pgfpathlineto{\pgfqpoint{1.880057in}{1.261071in}}%
\pgfpathlineto{\pgfqpoint{1.859223in}{1.281904in}}%
\pgfpathlineto{\pgfqpoint{1.838390in}{1.261071in}}%
\pgfpathlineto{\pgfqpoint{1.859223in}{1.240237in}}%
\pgfpathlineto{\pgfqpoint{1.838390in}{1.219404in}}%
\pgfpathclose%
\pgfusepath{stroke,fill}%
\end{pgfscope}%
\begin{pgfscope}%
\pgfpathrectangle{\pgfqpoint{0.424692in}{0.370679in}}{\pgfqpoint{2.725308in}{1.479321in}}%
\pgfusepath{clip}%
\pgfsetbuttcap%
\pgfsetroundjoin%
\definecolor{currentfill}{rgb}{0.466667,0.866667,0.466667}%
\pgfsetfillcolor{currentfill}%
\pgfsetfillopacity{0.500000}%
\pgfsetlinewidth{1.003750pt}%
\definecolor{currentstroke}{rgb}{0.466667,0.866667,0.466667}%
\pgfsetstrokecolor{currentstroke}%
\pgfsetstrokeopacity{0.500000}%
\pgfsetdash{}{0pt}%
\pgfpathmoveto{\pgfqpoint{2.559657in}{1.389789in}}%
\pgfpathlineto{\pgfqpoint{2.580490in}{1.410623in}}%
\pgfpathlineto{\pgfqpoint{2.601323in}{1.389789in}}%
\pgfpathlineto{\pgfqpoint{2.622157in}{1.410623in}}%
\pgfpathlineto{\pgfqpoint{2.601323in}{1.431456in}}%
\pgfpathlineto{\pgfqpoint{2.622157in}{1.452289in}}%
\pgfpathlineto{\pgfqpoint{2.601323in}{1.473123in}}%
\pgfpathlineto{\pgfqpoint{2.580490in}{1.452289in}}%
\pgfpathlineto{\pgfqpoint{2.559657in}{1.473123in}}%
\pgfpathlineto{\pgfqpoint{2.538823in}{1.452289in}}%
\pgfpathlineto{\pgfqpoint{2.559657in}{1.431456in}}%
\pgfpathlineto{\pgfqpoint{2.538823in}{1.410623in}}%
\pgfpathclose%
\pgfusepath{stroke,fill}%
\end{pgfscope}%
\begin{pgfscope}%
\pgfpathrectangle{\pgfqpoint{0.424692in}{0.370679in}}{\pgfqpoint{2.725308in}{1.479321in}}%
\pgfusepath{clip}%
\pgfsetbuttcap%
\pgfsetroundjoin%
\definecolor{currentfill}{rgb}{0.466667,0.866667,0.466667}%
\pgfsetfillcolor{currentfill}%
\pgfsetfillopacity{0.500000}%
\pgfsetlinewidth{1.003750pt}%
\definecolor{currentstroke}{rgb}{0.466667,0.866667,0.466667}%
\pgfsetstrokecolor{currentstroke}%
\pgfsetstrokeopacity{0.500000}%
\pgfsetdash{}{0pt}%
\pgfpathmoveto{\pgfqpoint{2.157568in}{1.318854in}}%
\pgfpathlineto{\pgfqpoint{2.178402in}{1.339687in}}%
\pgfpathlineto{\pgfqpoint{2.199235in}{1.318854in}}%
\pgfpathlineto{\pgfqpoint{2.220068in}{1.339687in}}%
\pgfpathlineto{\pgfqpoint{2.199235in}{1.360520in}}%
\pgfpathlineto{\pgfqpoint{2.220068in}{1.381354in}}%
\pgfpathlineto{\pgfqpoint{2.199235in}{1.402187in}}%
\pgfpathlineto{\pgfqpoint{2.178402in}{1.381354in}}%
\pgfpathlineto{\pgfqpoint{2.157568in}{1.402187in}}%
\pgfpathlineto{\pgfqpoint{2.136735in}{1.381354in}}%
\pgfpathlineto{\pgfqpoint{2.157568in}{1.360520in}}%
\pgfpathlineto{\pgfqpoint{2.136735in}{1.339687in}}%
\pgfpathclose%
\pgfusepath{stroke,fill}%
\end{pgfscope}%
\begin{pgfscope}%
\pgfpathrectangle{\pgfqpoint{0.424692in}{0.370679in}}{\pgfqpoint{2.725308in}{1.479321in}}%
\pgfusepath{clip}%
\pgfsetbuttcap%
\pgfsetroundjoin%
\definecolor{currentfill}{rgb}{0.423529,0.627451,0.862745}%
\pgfsetfillcolor{currentfill}%
\pgfsetfillopacity{0.500000}%
\pgfsetlinewidth{1.003750pt}%
\definecolor{currentstroke}{rgb}{0.423529,0.627451,0.862745}%
\pgfsetstrokecolor{currentstroke}%
\pgfsetstrokeopacity{0.500000}%
\pgfsetdash{}{0pt}%
\pgfpathmoveto{\pgfqpoint{1.735701in}{1.334667in}}%
\pgfpathlineto{\pgfqpoint{1.794626in}{1.393593in}}%
\pgfpathlineto{\pgfqpoint{1.735701in}{1.452518in}}%
\pgfpathlineto{\pgfqpoint{1.676775in}{1.393593in}}%
\pgfpathclose%
\pgfusepath{stroke,fill}%
\end{pgfscope}%
\begin{pgfscope}%
\pgfpathrectangle{\pgfqpoint{0.424692in}{0.370679in}}{\pgfqpoint{2.725308in}{1.479321in}}%
\pgfusepath{clip}%
\pgfsetbuttcap%
\pgfsetroundjoin%
\definecolor{currentfill}{rgb}{0.423529,0.627451,0.862745}%
\pgfsetfillcolor{currentfill}%
\pgfsetfillopacity{0.500000}%
\pgfsetlinewidth{1.003750pt}%
\definecolor{currentstroke}{rgb}{0.423529,0.627451,0.862745}%
\pgfsetstrokecolor{currentstroke}%
\pgfsetstrokeopacity{0.500000}%
\pgfsetdash{}{0pt}%
\pgfpathmoveto{\pgfqpoint{2.243647in}{1.474130in}}%
\pgfpathlineto{\pgfqpoint{2.302572in}{1.533056in}}%
\pgfpathlineto{\pgfqpoint{2.243647in}{1.591981in}}%
\pgfpathlineto{\pgfqpoint{2.184721in}{1.533056in}}%
\pgfpathclose%
\pgfusepath{stroke,fill}%
\end{pgfscope}%
\begin{pgfscope}%
\pgfpathrectangle{\pgfqpoint{0.424692in}{0.370679in}}{\pgfqpoint{2.725308in}{1.479321in}}%
\pgfusepath{clip}%
\pgfsetbuttcap%
\pgfsetroundjoin%
\definecolor{currentfill}{rgb}{0.423529,0.627451,0.862745}%
\pgfsetfillcolor{currentfill}%
\pgfsetfillopacity{0.500000}%
\pgfsetlinewidth{1.003750pt}%
\definecolor{currentstroke}{rgb}{0.423529,0.627451,0.862745}%
\pgfsetstrokecolor{currentstroke}%
\pgfsetstrokeopacity{0.500000}%
\pgfsetdash{}{0pt}%
\pgfpathmoveto{\pgfqpoint{1.469214in}{1.199934in}}%
\pgfpathlineto{\pgfqpoint{1.528140in}{1.258860in}}%
\pgfpathlineto{\pgfqpoint{1.469214in}{1.317785in}}%
\pgfpathlineto{\pgfqpoint{1.410289in}{1.258860in}}%
\pgfpathclose%
\pgfusepath{stroke,fill}%
\end{pgfscope}%
\begin{pgfscope}%
\pgfpathrectangle{\pgfqpoint{0.424692in}{0.370679in}}{\pgfqpoint{2.725308in}{1.479321in}}%
\pgfusepath{clip}%
\pgfsetbuttcap%
\pgfsetroundjoin%
\definecolor{currentfill}{rgb}{0.423529,0.627451,0.862745}%
\pgfsetfillcolor{currentfill}%
\pgfsetfillopacity{0.500000}%
\pgfsetlinewidth{1.003750pt}%
\definecolor{currentstroke}{rgb}{0.423529,0.627451,0.862745}%
\pgfsetstrokecolor{currentstroke}%
\pgfsetstrokeopacity{0.500000}%
\pgfsetdash{}{0pt}%
\pgfpathmoveto{\pgfqpoint{1.904764in}{1.357890in}}%
\pgfpathlineto{\pgfqpoint{1.963689in}{1.416816in}}%
\pgfpathlineto{\pgfqpoint{1.904764in}{1.475742in}}%
\pgfpathlineto{\pgfqpoint{1.845838in}{1.416816in}}%
\pgfpathclose%
\pgfusepath{stroke,fill}%
\end{pgfscope}%
\begin{pgfscope}%
\pgfpathrectangle{\pgfqpoint{0.424692in}{0.370679in}}{\pgfqpoint{2.725308in}{1.479321in}}%
\pgfusepath{clip}%
\pgfsetbuttcap%
\pgfsetroundjoin%
\definecolor{currentfill}{rgb}{0.423529,0.627451,0.862745}%
\pgfsetfillcolor{currentfill}%
\pgfsetfillopacity{0.500000}%
\pgfsetlinewidth{1.003750pt}%
\definecolor{currentstroke}{rgb}{0.423529,0.627451,0.862745}%
\pgfsetstrokecolor{currentstroke}%
\pgfsetstrokeopacity{0.500000}%
\pgfsetdash{}{0pt}%
\pgfpathmoveto{\pgfqpoint{1.773187in}{1.332750in}}%
\pgfpathlineto{\pgfqpoint{1.832113in}{1.391675in}}%
\pgfpathlineto{\pgfqpoint{1.773187in}{1.450601in}}%
\pgfpathlineto{\pgfqpoint{1.714261in}{1.391675in}}%
\pgfpathclose%
\pgfusepath{stroke,fill}%
\end{pgfscope}%
\begin{pgfscope}%
\pgfpathrectangle{\pgfqpoint{0.424692in}{0.370679in}}{\pgfqpoint{2.725308in}{1.479321in}}%
\pgfusepath{clip}%
\pgfsetbuttcap%
\pgfsetroundjoin%
\definecolor{currentfill}{rgb}{0.423529,0.627451,0.862745}%
\pgfsetfillcolor{currentfill}%
\pgfsetfillopacity{0.500000}%
\pgfsetlinewidth{1.003750pt}%
\definecolor{currentstroke}{rgb}{0.423529,0.627451,0.862745}%
\pgfsetstrokecolor{currentstroke}%
\pgfsetstrokeopacity{0.500000}%
\pgfsetdash{}{0pt}%
\pgfpathmoveto{\pgfqpoint{1.246351in}{1.174329in}}%
\pgfpathlineto{\pgfqpoint{1.305276in}{1.233255in}}%
\pgfpathlineto{\pgfqpoint{1.246351in}{1.292180in}}%
\pgfpathlineto{\pgfqpoint{1.187425in}{1.233255in}}%
\pgfpathclose%
\pgfusepath{stroke,fill}%
\end{pgfscope}%
\begin{pgfscope}%
\pgfpathrectangle{\pgfqpoint{0.424692in}{0.370679in}}{\pgfqpoint{2.725308in}{1.479321in}}%
\pgfusepath{clip}%
\pgfsetbuttcap%
\pgfsetroundjoin%
\definecolor{currentfill}{rgb}{0.423529,0.627451,0.862745}%
\pgfsetfillcolor{currentfill}%
\pgfsetfillopacity{0.500000}%
\pgfsetlinewidth{1.003750pt}%
\definecolor{currentstroke}{rgb}{0.423529,0.627451,0.862745}%
\pgfsetstrokecolor{currentstroke}%
\pgfsetstrokeopacity{0.500000}%
\pgfsetdash{}{0pt}%
\pgfpathmoveto{\pgfqpoint{2.771170in}{1.651619in}}%
\pgfpathlineto{\pgfqpoint{2.830095in}{1.710545in}}%
\pgfpathlineto{\pgfqpoint{2.771170in}{1.769470in}}%
\pgfpathlineto{\pgfqpoint{2.712244in}{1.710545in}}%
\pgfpathclose%
\pgfusepath{stroke,fill}%
\end{pgfscope}%
\begin{pgfscope}%
\pgfpathrectangle{\pgfqpoint{0.424692in}{0.370679in}}{\pgfqpoint{2.725308in}{1.479321in}}%
\pgfusepath{clip}%
\pgfsetbuttcap%
\pgfsetroundjoin%
\definecolor{currentfill}{rgb}{0.423529,0.627451,0.862745}%
\pgfsetfillcolor{currentfill}%
\pgfsetfillopacity{0.500000}%
\pgfsetlinewidth{1.003750pt}%
\definecolor{currentstroke}{rgb}{0.423529,0.627451,0.862745}%
\pgfsetstrokecolor{currentstroke}%
\pgfsetstrokeopacity{0.500000}%
\pgfsetdash{}{0pt}%
\pgfpathmoveto{\pgfqpoint{1.428834in}{1.216656in}}%
\pgfpathlineto{\pgfqpoint{1.487760in}{1.275581in}}%
\pgfpathlineto{\pgfqpoint{1.428834in}{1.334507in}}%
\pgfpathlineto{\pgfqpoint{1.369908in}{1.275581in}}%
\pgfpathclose%
\pgfusepath{stroke,fill}%
\end{pgfscope}%
\begin{pgfscope}%
\pgfpathrectangle{\pgfqpoint{0.424692in}{0.370679in}}{\pgfqpoint{2.725308in}{1.479321in}}%
\pgfusepath{clip}%
\pgfsetbuttcap%
\pgfsetroundjoin%
\definecolor{currentfill}{rgb}{0.423529,0.627451,0.862745}%
\pgfsetfillcolor{currentfill}%
\pgfsetfillopacity{0.500000}%
\pgfsetlinewidth{1.003750pt}%
\definecolor{currentstroke}{rgb}{0.423529,0.627451,0.862745}%
\pgfsetstrokecolor{currentstroke}%
\pgfsetstrokeopacity{0.500000}%
\pgfsetdash{}{0pt}%
\pgfpathmoveto{\pgfqpoint{1.917947in}{1.407883in}}%
\pgfpathlineto{\pgfqpoint{1.976873in}{1.466808in}}%
\pgfpathlineto{\pgfqpoint{1.917947in}{1.525734in}}%
\pgfpathlineto{\pgfqpoint{1.859022in}{1.466808in}}%
\pgfpathclose%
\pgfusepath{stroke,fill}%
\end{pgfscope}%
\begin{pgfscope}%
\pgfpathrectangle{\pgfqpoint{0.424692in}{0.370679in}}{\pgfqpoint{2.725308in}{1.479321in}}%
\pgfusepath{clip}%
\pgfsetbuttcap%
\pgfsetroundjoin%
\definecolor{currentfill}{rgb}{0.423529,0.627451,0.862745}%
\pgfsetfillcolor{currentfill}%
\pgfsetfillopacity{0.500000}%
\pgfsetlinewidth{1.003750pt}%
\definecolor{currentstroke}{rgb}{0.423529,0.627451,0.862745}%
\pgfsetstrokecolor{currentstroke}%
\pgfsetstrokeopacity{0.500000}%
\pgfsetdash{}{0pt}%
\pgfpathmoveto{\pgfqpoint{1.129542in}{1.051979in}}%
\pgfpathlineto{\pgfqpoint{1.188468in}{1.110904in}}%
\pgfpathlineto{\pgfqpoint{1.129542in}{1.169830in}}%
\pgfpathlineto{\pgfqpoint{1.070617in}{1.110904in}}%
\pgfpathclose%
\pgfusepath{stroke,fill}%
\end{pgfscope}%
\begin{pgfscope}%
\pgfpathrectangle{\pgfqpoint{0.424692in}{0.370679in}}{\pgfqpoint{2.725308in}{1.479321in}}%
\pgfusepath{clip}%
\pgfsetbuttcap%
\pgfsetroundjoin%
\definecolor{currentfill}{rgb}{0.423529,0.627451,0.862745}%
\pgfsetfillcolor{currentfill}%
\pgfsetfillopacity{0.500000}%
\pgfsetlinewidth{1.003750pt}%
\definecolor{currentstroke}{rgb}{0.423529,0.627451,0.862745}%
\pgfsetstrokecolor{currentstroke}%
\pgfsetstrokeopacity{0.500000}%
\pgfsetdash{}{0pt}%
\pgfpathmoveto{\pgfqpoint{1.627022in}{1.168721in}}%
\pgfpathlineto{\pgfqpoint{1.685948in}{1.227647in}}%
\pgfpathlineto{\pgfqpoint{1.627022in}{1.286572in}}%
\pgfpathlineto{\pgfqpoint{1.568096in}{1.227647in}}%
\pgfpathclose%
\pgfusepath{stroke,fill}%
\end{pgfscope}%
\begin{pgfscope}%
\pgfpathrectangle{\pgfqpoint{0.424692in}{0.370679in}}{\pgfqpoint{2.725308in}{1.479321in}}%
\pgfusepath{clip}%
\pgfsetbuttcap%
\pgfsetroundjoin%
\definecolor{currentfill}{rgb}{0.423529,0.627451,0.862745}%
\pgfsetfillcolor{currentfill}%
\pgfsetfillopacity{0.500000}%
\pgfsetlinewidth{1.003750pt}%
\definecolor{currentstroke}{rgb}{0.423529,0.627451,0.862745}%
\pgfsetstrokecolor{currentstroke}%
\pgfsetstrokeopacity{0.500000}%
\pgfsetdash{}{0pt}%
\pgfpathmoveto{\pgfqpoint{1.944897in}{1.359070in}}%
\pgfpathlineto{\pgfqpoint{2.003823in}{1.417995in}}%
\pgfpathlineto{\pgfqpoint{1.944897in}{1.476921in}}%
\pgfpathlineto{\pgfqpoint{1.885971in}{1.417995in}}%
\pgfpathclose%
\pgfusepath{stroke,fill}%
\end{pgfscope}%
\begin{pgfscope}%
\pgfpathrectangle{\pgfqpoint{0.424692in}{0.370679in}}{\pgfqpoint{2.725308in}{1.479321in}}%
\pgfusepath{clip}%
\pgfsetbuttcap%
\pgfsetroundjoin%
\definecolor{currentfill}{rgb}{0.423529,0.627451,0.862745}%
\pgfsetfillcolor{currentfill}%
\pgfsetfillopacity{0.500000}%
\pgfsetlinewidth{1.003750pt}%
\definecolor{currentstroke}{rgb}{0.423529,0.627451,0.862745}%
\pgfsetstrokecolor{currentstroke}%
\pgfsetstrokeopacity{0.500000}%
\pgfsetdash{}{0pt}%
\pgfpathmoveto{\pgfqpoint{1.600216in}{1.215910in}}%
\pgfpathlineto{\pgfqpoint{1.659142in}{1.274835in}}%
\pgfpathlineto{\pgfqpoint{1.600216in}{1.333761in}}%
\pgfpathlineto{\pgfqpoint{1.541291in}{1.274835in}}%
\pgfpathclose%
\pgfusepath{stroke,fill}%
\end{pgfscope}%
\begin{pgfscope}%
\pgfpathrectangle{\pgfqpoint{0.424692in}{0.370679in}}{\pgfqpoint{2.725308in}{1.479321in}}%
\pgfusepath{clip}%
\pgfsetbuttcap%
\pgfsetroundjoin%
\definecolor{currentfill}{rgb}{0.423529,0.627451,0.862745}%
\pgfsetfillcolor{currentfill}%
\pgfsetfillopacity{0.500000}%
\pgfsetlinewidth{1.003750pt}%
\definecolor{currentstroke}{rgb}{0.423529,0.627451,0.862745}%
\pgfsetstrokecolor{currentstroke}%
\pgfsetstrokeopacity{0.500000}%
\pgfsetdash{}{0pt}%
\pgfpathmoveto{\pgfqpoint{2.395491in}{1.548298in}}%
\pgfpathlineto{\pgfqpoint{2.454417in}{1.607224in}}%
\pgfpathlineto{\pgfqpoint{2.395491in}{1.666150in}}%
\pgfpathlineto{\pgfqpoint{2.336566in}{1.607224in}}%
\pgfpathclose%
\pgfusepath{stroke,fill}%
\end{pgfscope}%
\begin{pgfscope}%
\pgfpathrectangle{\pgfqpoint{0.424692in}{0.370679in}}{\pgfqpoint{2.725308in}{1.479321in}}%
\pgfusepath{clip}%
\pgfsetbuttcap%
\pgfsetroundjoin%
\definecolor{currentfill}{rgb}{0.423529,0.627451,0.862745}%
\pgfsetfillcolor{currentfill}%
\pgfsetfillopacity{0.500000}%
\pgfsetlinewidth{1.003750pt}%
\definecolor{currentstroke}{rgb}{0.423529,0.627451,0.862745}%
\pgfsetstrokecolor{currentstroke}%
\pgfsetstrokeopacity{0.500000}%
\pgfsetdash{}{0pt}%
\pgfpathmoveto{\pgfqpoint{2.203137in}{1.473079in}}%
\pgfpathlineto{\pgfqpoint{2.262063in}{1.532005in}}%
\pgfpathlineto{\pgfqpoint{2.203137in}{1.590930in}}%
\pgfpathlineto{\pgfqpoint{2.144212in}{1.532005in}}%
\pgfpathclose%
\pgfusepath{stroke,fill}%
\end{pgfscope}%
\begin{pgfscope}%
\pgfpathrectangle{\pgfqpoint{0.424692in}{0.370679in}}{\pgfqpoint{2.725308in}{1.479321in}}%
\pgfusepath{clip}%
\pgfsetbuttcap%
\pgfsetroundjoin%
\definecolor{currentfill}{rgb}{0.423529,0.627451,0.862745}%
\pgfsetfillcolor{currentfill}%
\pgfsetfillopacity{0.500000}%
\pgfsetlinewidth{1.003750pt}%
\definecolor{currentstroke}{rgb}{0.423529,0.627451,0.862745}%
\pgfsetstrokecolor{currentstroke}%
\pgfsetstrokeopacity{0.500000}%
\pgfsetdash{}{0pt}%
\pgfpathmoveto{\pgfqpoint{1.697149in}{1.236721in}}%
\pgfpathlineto{\pgfqpoint{1.756075in}{1.295646in}}%
\pgfpathlineto{\pgfqpoint{1.697149in}{1.354572in}}%
\pgfpathlineto{\pgfqpoint{1.638224in}{1.295646in}}%
\pgfpathclose%
\pgfusepath{stroke,fill}%
\end{pgfscope}%
\begin{pgfscope}%
\pgfpathrectangle{\pgfqpoint{0.424692in}{0.370679in}}{\pgfqpoint{2.725308in}{1.479321in}}%
\pgfusepath{clip}%
\pgfsetbuttcap%
\pgfsetroundjoin%
\definecolor{currentfill}{rgb}{0.423529,0.627451,0.862745}%
\pgfsetfillcolor{currentfill}%
\pgfsetfillopacity{0.500000}%
\pgfsetlinewidth{1.003750pt}%
\definecolor{currentstroke}{rgb}{0.423529,0.627451,0.862745}%
\pgfsetstrokecolor{currentstroke}%
\pgfsetstrokeopacity{0.500000}%
\pgfsetdash{}{0pt}%
\pgfpathmoveto{\pgfqpoint{1.746677in}{1.269150in}}%
\pgfpathlineto{\pgfqpoint{1.805603in}{1.328076in}}%
\pgfpathlineto{\pgfqpoint{1.746677in}{1.387001in}}%
\pgfpathlineto{\pgfqpoint{1.687752in}{1.328076in}}%
\pgfpathclose%
\pgfusepath{stroke,fill}%
\end{pgfscope}%
\begin{pgfscope}%
\pgfpathrectangle{\pgfqpoint{0.424692in}{0.370679in}}{\pgfqpoint{2.725308in}{1.479321in}}%
\pgfusepath{clip}%
\pgfsetbuttcap%
\pgfsetroundjoin%
\definecolor{currentfill}{rgb}{0.423529,0.627451,0.862745}%
\pgfsetfillcolor{currentfill}%
\pgfsetfillopacity{0.500000}%
\pgfsetlinewidth{1.003750pt}%
\definecolor{currentstroke}{rgb}{0.423529,0.627451,0.862745}%
\pgfsetstrokecolor{currentstroke}%
\pgfsetstrokeopacity{0.500000}%
\pgfsetdash{}{0pt}%
\pgfpathmoveto{\pgfqpoint{0.792557in}{0.938482in}}%
\pgfpathlineto{\pgfqpoint{0.851482in}{0.997407in}}%
\pgfpathlineto{\pgfqpoint{0.792557in}{1.056333in}}%
\pgfpathlineto{\pgfqpoint{0.733631in}{0.997407in}}%
\pgfpathclose%
\pgfusepath{stroke,fill}%
\end{pgfscope}%
\begin{pgfscope}%
\pgfpathrectangle{\pgfqpoint{0.424692in}{0.370679in}}{\pgfqpoint{2.725308in}{1.479321in}}%
\pgfusepath{clip}%
\pgfsetbuttcap%
\pgfsetroundjoin%
\definecolor{currentfill}{rgb}{0.423529,0.627451,0.862745}%
\pgfsetfillcolor{currentfill}%
\pgfsetfillopacity{0.500000}%
\pgfsetlinewidth{1.003750pt}%
\definecolor{currentstroke}{rgb}{0.423529,0.627451,0.862745}%
\pgfsetstrokecolor{currentstroke}%
\pgfsetstrokeopacity{0.500000}%
\pgfsetdash{}{0pt}%
\pgfpathmoveto{\pgfqpoint{1.423014in}{1.319366in}}%
\pgfpathlineto{\pgfqpoint{1.481940in}{1.378291in}}%
\pgfpathlineto{\pgfqpoint{1.423014in}{1.437217in}}%
\pgfpathlineto{\pgfqpoint{1.364089in}{1.378291in}}%
\pgfpathclose%
\pgfusepath{stroke,fill}%
\end{pgfscope}%
\begin{pgfscope}%
\pgfpathrectangle{\pgfqpoint{0.424692in}{0.370679in}}{\pgfqpoint{2.725308in}{1.479321in}}%
\pgfusepath{clip}%
\pgfsetbuttcap%
\pgfsetroundjoin%
\definecolor{currentfill}{rgb}{0.423529,0.627451,0.862745}%
\pgfsetfillcolor{currentfill}%
\pgfsetfillopacity{0.500000}%
\pgfsetlinewidth{1.003750pt}%
\definecolor{currentstroke}{rgb}{0.423529,0.627451,0.862745}%
\pgfsetstrokecolor{currentstroke}%
\pgfsetstrokeopacity{0.500000}%
\pgfsetdash{}{0pt}%
\pgfpathmoveto{\pgfqpoint{2.384772in}{1.536775in}}%
\pgfpathlineto{\pgfqpoint{2.443698in}{1.595700in}}%
\pgfpathlineto{\pgfqpoint{2.384772in}{1.654626in}}%
\pgfpathlineto{\pgfqpoint{2.325847in}{1.595700in}}%
\pgfpathclose%
\pgfusepath{stroke,fill}%
\end{pgfscope}%
\begin{pgfscope}%
\pgfpathrectangle{\pgfqpoint{0.424692in}{0.370679in}}{\pgfqpoint{2.725308in}{1.479321in}}%
\pgfusepath{clip}%
\pgfsetbuttcap%
\pgfsetroundjoin%
\definecolor{currentfill}{rgb}{0.423529,0.627451,0.862745}%
\pgfsetfillcolor{currentfill}%
\pgfsetfillopacity{0.500000}%
\pgfsetlinewidth{1.003750pt}%
\definecolor{currentstroke}{rgb}{0.423529,0.627451,0.862745}%
\pgfsetstrokecolor{currentstroke}%
\pgfsetstrokeopacity{0.500000}%
\pgfsetdash{}{0pt}%
\pgfpathmoveto{\pgfqpoint{1.461734in}{1.172749in}}%
\pgfpathlineto{\pgfqpoint{1.520660in}{1.231674in}}%
\pgfpathlineto{\pgfqpoint{1.461734in}{1.290600in}}%
\pgfpathlineto{\pgfqpoint{1.402809in}{1.231674in}}%
\pgfpathclose%
\pgfusepath{stroke,fill}%
\end{pgfscope}%
\begin{pgfscope}%
\pgfpathrectangle{\pgfqpoint{0.424692in}{0.370679in}}{\pgfqpoint{2.725308in}{1.479321in}}%
\pgfusepath{clip}%
\pgfsetbuttcap%
\pgfsetroundjoin%
\definecolor{currentfill}{rgb}{0.423529,0.627451,0.862745}%
\pgfsetfillcolor{currentfill}%
\pgfsetfillopacity{0.500000}%
\pgfsetlinewidth{1.003750pt}%
\definecolor{currentstroke}{rgb}{0.423529,0.627451,0.862745}%
\pgfsetstrokecolor{currentstroke}%
\pgfsetstrokeopacity{0.500000}%
\pgfsetdash{}{0pt}%
\pgfpathmoveto{\pgfqpoint{2.285475in}{1.470403in}}%
\pgfpathlineto{\pgfqpoint{2.344400in}{1.529328in}}%
\pgfpathlineto{\pgfqpoint{2.285475in}{1.588254in}}%
\pgfpathlineto{\pgfqpoint{2.226549in}{1.529328in}}%
\pgfpathclose%
\pgfusepath{stroke,fill}%
\end{pgfscope}%
\begin{pgfscope}%
\pgfpathrectangle{\pgfqpoint{0.424692in}{0.370679in}}{\pgfqpoint{2.725308in}{1.479321in}}%
\pgfusepath{clip}%
\pgfsetbuttcap%
\pgfsetroundjoin%
\definecolor{currentfill}{rgb}{0.423529,0.627451,0.862745}%
\pgfsetfillcolor{currentfill}%
\pgfsetfillopacity{0.500000}%
\pgfsetlinewidth{1.003750pt}%
\definecolor{currentstroke}{rgb}{0.423529,0.627451,0.862745}%
\pgfsetstrokecolor{currentstroke}%
\pgfsetstrokeopacity{0.500000}%
\pgfsetdash{}{0pt}%
\pgfpathmoveto{\pgfqpoint{1.300757in}{1.216521in}}%
\pgfpathlineto{\pgfqpoint{1.359683in}{1.275447in}}%
\pgfpathlineto{\pgfqpoint{1.300757in}{1.334372in}}%
\pgfpathlineto{\pgfqpoint{1.241831in}{1.275447in}}%
\pgfpathclose%
\pgfusepath{stroke,fill}%
\end{pgfscope}%
\begin{pgfscope}%
\pgfpathrectangle{\pgfqpoint{0.424692in}{0.370679in}}{\pgfqpoint{2.725308in}{1.479321in}}%
\pgfusepath{clip}%
\pgfsetbuttcap%
\pgfsetroundjoin%
\definecolor{currentfill}{rgb}{0.423529,0.627451,0.862745}%
\pgfsetfillcolor{currentfill}%
\pgfsetfillopacity{0.500000}%
\pgfsetlinewidth{1.003750pt}%
\definecolor{currentstroke}{rgb}{0.423529,0.627451,0.862745}%
\pgfsetstrokecolor{currentstroke}%
\pgfsetstrokeopacity{0.500000}%
\pgfsetdash{}{0pt}%
\pgfpathmoveto{\pgfqpoint{1.975823in}{1.325026in}}%
\pgfpathlineto{\pgfqpoint{2.034748in}{1.383952in}}%
\pgfpathlineto{\pgfqpoint{1.975823in}{1.442877in}}%
\pgfpathlineto{\pgfqpoint{1.916897in}{1.383952in}}%
\pgfpathclose%
\pgfusepath{stroke,fill}%
\end{pgfscope}%
\begin{pgfscope}%
\pgfpathrectangle{\pgfqpoint{0.424692in}{0.370679in}}{\pgfqpoint{2.725308in}{1.479321in}}%
\pgfusepath{clip}%
\pgfsetbuttcap%
\pgfsetroundjoin%
\definecolor{currentfill}{rgb}{0.423529,0.627451,0.862745}%
\pgfsetfillcolor{currentfill}%
\pgfsetfillopacity{0.500000}%
\pgfsetlinewidth{1.003750pt}%
\definecolor{currentstroke}{rgb}{0.423529,0.627451,0.862745}%
\pgfsetstrokecolor{currentstroke}%
\pgfsetstrokeopacity{0.500000}%
\pgfsetdash{}{0pt}%
\pgfpathmoveto{\pgfqpoint{1.508024in}{1.179268in}}%
\pgfpathlineto{\pgfqpoint{1.566950in}{1.238194in}}%
\pgfpathlineto{\pgfqpoint{1.508024in}{1.297119in}}%
\pgfpathlineto{\pgfqpoint{1.449099in}{1.238194in}}%
\pgfpathclose%
\pgfusepath{stroke,fill}%
\end{pgfscope}%
\begin{pgfscope}%
\pgfpathrectangle{\pgfqpoint{0.424692in}{0.370679in}}{\pgfqpoint{2.725308in}{1.479321in}}%
\pgfusepath{clip}%
\pgfsetbuttcap%
\pgfsetroundjoin%
\definecolor{currentfill}{rgb}{0.423529,0.627451,0.862745}%
\pgfsetfillcolor{currentfill}%
\pgfsetfillopacity{0.500000}%
\pgfsetlinewidth{1.003750pt}%
\definecolor{currentstroke}{rgb}{0.423529,0.627451,0.862745}%
\pgfsetstrokecolor{currentstroke}%
\pgfsetstrokeopacity{0.500000}%
\pgfsetdash{}{0pt}%
\pgfpathmoveto{\pgfqpoint{1.309349in}{1.112579in}}%
\pgfpathlineto{\pgfqpoint{1.368275in}{1.171505in}}%
\pgfpathlineto{\pgfqpoint{1.309349in}{1.230430in}}%
\pgfpathlineto{\pgfqpoint{1.250423in}{1.171505in}}%
\pgfpathclose%
\pgfusepath{stroke,fill}%
\end{pgfscope}%
\begin{pgfscope}%
\pgfpathrectangle{\pgfqpoint{0.424692in}{0.370679in}}{\pgfqpoint{2.725308in}{1.479321in}}%
\pgfusepath{clip}%
\pgfsetbuttcap%
\pgfsetroundjoin%
\definecolor{currentfill}{rgb}{0.423529,0.627451,0.862745}%
\pgfsetfillcolor{currentfill}%
\pgfsetfillopacity{0.500000}%
\pgfsetlinewidth{1.003750pt}%
\definecolor{currentstroke}{rgb}{0.423529,0.627451,0.862745}%
\pgfsetstrokecolor{currentstroke}%
\pgfsetstrokeopacity{0.500000}%
\pgfsetdash{}{0pt}%
\pgfpathmoveto{\pgfqpoint{1.968042in}{1.304823in}}%
\pgfpathlineto{\pgfqpoint{2.026967in}{1.363748in}}%
\pgfpathlineto{\pgfqpoint{1.968042in}{1.422674in}}%
\pgfpathlineto{\pgfqpoint{1.909116in}{1.363748in}}%
\pgfpathclose%
\pgfusepath{stroke,fill}%
\end{pgfscope}%
\begin{pgfscope}%
\pgfpathrectangle{\pgfqpoint{0.424692in}{0.370679in}}{\pgfqpoint{2.725308in}{1.479321in}}%
\pgfusepath{clip}%
\pgfsetbuttcap%
\pgfsetroundjoin%
\definecolor{currentfill}{rgb}{0.423529,0.627451,0.862745}%
\pgfsetfillcolor{currentfill}%
\pgfsetfillopacity{0.500000}%
\pgfsetlinewidth{1.003750pt}%
\definecolor{currentstroke}{rgb}{0.423529,0.627451,0.862745}%
\pgfsetstrokecolor{currentstroke}%
\pgfsetstrokeopacity{0.500000}%
\pgfsetdash{}{0pt}%
\pgfpathmoveto{\pgfqpoint{1.923512in}{1.284638in}}%
\pgfpathlineto{\pgfqpoint{1.982437in}{1.343563in}}%
\pgfpathlineto{\pgfqpoint{1.923512in}{1.402489in}}%
\pgfpathlineto{\pgfqpoint{1.864586in}{1.343563in}}%
\pgfpathclose%
\pgfusepath{stroke,fill}%
\end{pgfscope}%
\begin{pgfscope}%
\pgfpathrectangle{\pgfqpoint{0.424692in}{0.370679in}}{\pgfqpoint{2.725308in}{1.479321in}}%
\pgfusepath{clip}%
\pgfsetbuttcap%
\pgfsetroundjoin%
\definecolor{currentfill}{rgb}{0.423529,0.627451,0.862745}%
\pgfsetfillcolor{currentfill}%
\pgfsetfillopacity{0.500000}%
\pgfsetlinewidth{1.003750pt}%
\definecolor{currentstroke}{rgb}{0.423529,0.627451,0.862745}%
\pgfsetstrokecolor{currentstroke}%
\pgfsetstrokeopacity{0.500000}%
\pgfsetdash{}{0pt}%
\pgfpathmoveto{\pgfqpoint{1.914135in}{1.381659in}}%
\pgfpathlineto{\pgfqpoint{1.973061in}{1.440584in}}%
\pgfpathlineto{\pgfqpoint{1.914135in}{1.499510in}}%
\pgfpathlineto{\pgfqpoint{1.855209in}{1.440584in}}%
\pgfpathclose%
\pgfusepath{stroke,fill}%
\end{pgfscope}%
\begin{pgfscope}%
\pgfpathrectangle{\pgfqpoint{0.424692in}{0.370679in}}{\pgfqpoint{2.725308in}{1.479321in}}%
\pgfusepath{clip}%
\pgfsetbuttcap%
\pgfsetroundjoin%
\definecolor{currentfill}{rgb}{0.423529,0.627451,0.862745}%
\pgfsetfillcolor{currentfill}%
\pgfsetfillopacity{0.500000}%
\pgfsetlinewidth{1.003750pt}%
\definecolor{currentstroke}{rgb}{0.423529,0.627451,0.862745}%
\pgfsetstrokecolor{currentstroke}%
\pgfsetstrokeopacity{0.500000}%
\pgfsetdash{}{0pt}%
\pgfpathmoveto{\pgfqpoint{1.397922in}{1.133000in}}%
\pgfpathlineto{\pgfqpoint{1.456848in}{1.191925in}}%
\pgfpathlineto{\pgfqpoint{1.397922in}{1.250851in}}%
\pgfpathlineto{\pgfqpoint{1.338996in}{1.191925in}}%
\pgfpathclose%
\pgfusepath{stroke,fill}%
\end{pgfscope}%
\begin{pgfscope}%
\pgfpathrectangle{\pgfqpoint{0.424692in}{0.370679in}}{\pgfqpoint{2.725308in}{1.479321in}}%
\pgfusepath{clip}%
\pgfsetbuttcap%
\pgfsetroundjoin%
\definecolor{currentfill}{rgb}{0.423529,0.627451,0.862745}%
\pgfsetfillcolor{currentfill}%
\pgfsetfillopacity{0.500000}%
\pgfsetlinewidth{1.003750pt}%
\definecolor{currentstroke}{rgb}{0.423529,0.627451,0.862745}%
\pgfsetstrokecolor{currentstroke}%
\pgfsetstrokeopacity{0.500000}%
\pgfsetdash{}{0pt}%
\pgfpathmoveto{\pgfqpoint{1.422564in}{1.202068in}}%
\pgfpathlineto{\pgfqpoint{1.481489in}{1.260994in}}%
\pgfpathlineto{\pgfqpoint{1.422564in}{1.319919in}}%
\pgfpathlineto{\pgfqpoint{1.363638in}{1.260994in}}%
\pgfpathclose%
\pgfusepath{stroke,fill}%
\end{pgfscope}%
\begin{pgfscope}%
\pgfpathrectangle{\pgfqpoint{0.424692in}{0.370679in}}{\pgfqpoint{2.725308in}{1.479321in}}%
\pgfusepath{clip}%
\pgfsetbuttcap%
\pgfsetroundjoin%
\definecolor{currentfill}{rgb}{0.423529,0.627451,0.862745}%
\pgfsetfillcolor{currentfill}%
\pgfsetfillopacity{0.500000}%
\pgfsetlinewidth{1.003750pt}%
\definecolor{currentstroke}{rgb}{0.423529,0.627451,0.862745}%
\pgfsetstrokecolor{currentstroke}%
\pgfsetstrokeopacity{0.500000}%
\pgfsetdash{}{0pt}%
\pgfpathmoveto{\pgfqpoint{0.669549in}{0.840014in}}%
\pgfpathlineto{\pgfqpoint{0.728475in}{0.898940in}}%
\pgfpathlineto{\pgfqpoint{0.669549in}{0.957865in}}%
\pgfpathlineto{\pgfqpoint{0.610624in}{0.898940in}}%
\pgfpathclose%
\pgfusepath{stroke,fill}%
\end{pgfscope}%
\begin{pgfscope}%
\pgfpathrectangle{\pgfqpoint{0.424692in}{0.370679in}}{\pgfqpoint{2.725308in}{1.479321in}}%
\pgfusepath{clip}%
\pgfsetbuttcap%
\pgfsetroundjoin%
\definecolor{currentfill}{rgb}{0.423529,0.627451,0.862745}%
\pgfsetfillcolor{currentfill}%
\pgfsetfillopacity{0.500000}%
\pgfsetlinewidth{1.003750pt}%
\definecolor{currentstroke}{rgb}{0.423529,0.627451,0.862745}%
\pgfsetstrokecolor{currentstroke}%
\pgfsetstrokeopacity{0.500000}%
\pgfsetdash{}{0pt}%
\pgfpathmoveto{\pgfqpoint{1.919665in}{1.389085in}}%
\pgfpathlineto{\pgfqpoint{1.978591in}{1.448010in}}%
\pgfpathlineto{\pgfqpoint{1.919665in}{1.506936in}}%
\pgfpathlineto{\pgfqpoint{1.860740in}{1.448010in}}%
\pgfpathclose%
\pgfusepath{stroke,fill}%
\end{pgfscope}%
\begin{pgfscope}%
\pgfpathrectangle{\pgfqpoint{0.424692in}{0.370679in}}{\pgfqpoint{2.725308in}{1.479321in}}%
\pgfusepath{clip}%
\pgfsetbuttcap%
\pgfsetroundjoin%
\definecolor{currentfill}{rgb}{0.423529,0.627451,0.862745}%
\pgfsetfillcolor{currentfill}%
\pgfsetfillopacity{0.500000}%
\pgfsetlinewidth{1.003750pt}%
\definecolor{currentstroke}{rgb}{0.423529,0.627451,0.862745}%
\pgfsetstrokecolor{currentstroke}%
\pgfsetstrokeopacity{0.500000}%
\pgfsetdash{}{0pt}%
\pgfpathmoveto{\pgfqpoint{1.928106in}{1.305846in}}%
\pgfpathlineto{\pgfqpoint{1.987031in}{1.364771in}}%
\pgfpathlineto{\pgfqpoint{1.928106in}{1.423697in}}%
\pgfpathlineto{\pgfqpoint{1.869180in}{1.364771in}}%
\pgfpathclose%
\pgfusepath{stroke,fill}%
\end{pgfscope}%
\begin{pgfscope}%
\pgfpathrectangle{\pgfqpoint{0.424692in}{0.370679in}}{\pgfqpoint{2.725308in}{1.479321in}}%
\pgfusepath{clip}%
\pgfsetbuttcap%
\pgfsetroundjoin%
\definecolor{currentfill}{rgb}{0.423529,0.627451,0.862745}%
\pgfsetfillcolor{currentfill}%
\pgfsetfillopacity{0.500000}%
\pgfsetlinewidth{1.003750pt}%
\definecolor{currentstroke}{rgb}{0.423529,0.627451,0.862745}%
\pgfsetstrokecolor{currentstroke}%
\pgfsetstrokeopacity{0.500000}%
\pgfsetdash{}{0pt}%
\pgfpathmoveto{\pgfqpoint{2.154824in}{1.470618in}}%
\pgfpathlineto{\pgfqpoint{2.213750in}{1.529543in}}%
\pgfpathlineto{\pgfqpoint{2.154824in}{1.588469in}}%
\pgfpathlineto{\pgfqpoint{2.095899in}{1.529543in}}%
\pgfpathclose%
\pgfusepath{stroke,fill}%
\end{pgfscope}%
\begin{pgfscope}%
\pgfpathrectangle{\pgfqpoint{0.424692in}{0.370679in}}{\pgfqpoint{2.725308in}{1.479321in}}%
\pgfusepath{clip}%
\pgfsetbuttcap%
\pgfsetroundjoin%
\definecolor{currentfill}{rgb}{0.423529,0.627451,0.862745}%
\pgfsetfillcolor{currentfill}%
\pgfsetfillopacity{0.500000}%
\pgfsetlinewidth{1.003750pt}%
\definecolor{currentstroke}{rgb}{0.423529,0.627451,0.862745}%
\pgfsetstrokecolor{currentstroke}%
\pgfsetstrokeopacity{0.500000}%
\pgfsetdash{}{0pt}%
\pgfpathmoveto{\pgfqpoint{1.177384in}{1.097458in}}%
\pgfpathlineto{\pgfqpoint{1.236310in}{1.156384in}}%
\pgfpathlineto{\pgfqpoint{1.177384in}{1.215309in}}%
\pgfpathlineto{\pgfqpoint{1.118459in}{1.156384in}}%
\pgfpathclose%
\pgfusepath{stroke,fill}%
\end{pgfscope}%
\begin{pgfscope}%
\pgfpathrectangle{\pgfqpoint{0.424692in}{0.370679in}}{\pgfqpoint{2.725308in}{1.479321in}}%
\pgfusepath{clip}%
\pgfsetbuttcap%
\pgfsetroundjoin%
\definecolor{currentfill}{rgb}{0.423529,0.627451,0.862745}%
\pgfsetfillcolor{currentfill}%
\pgfsetfillopacity{0.500000}%
\pgfsetlinewidth{1.003750pt}%
\definecolor{currentstroke}{rgb}{0.423529,0.627451,0.862745}%
\pgfsetstrokecolor{currentstroke}%
\pgfsetstrokeopacity{0.500000}%
\pgfsetdash{}{0pt}%
\pgfpathmoveto{\pgfqpoint{1.662061in}{1.247099in}}%
\pgfpathlineto{\pgfqpoint{1.720986in}{1.306024in}}%
\pgfpathlineto{\pgfqpoint{1.662061in}{1.364950in}}%
\pgfpathlineto{\pgfqpoint{1.603135in}{1.306024in}}%
\pgfpathclose%
\pgfusepath{stroke,fill}%
\end{pgfscope}%
\begin{pgfscope}%
\pgfpathrectangle{\pgfqpoint{0.424692in}{0.370679in}}{\pgfqpoint{2.725308in}{1.479321in}}%
\pgfusepath{clip}%
\pgfsetbuttcap%
\pgfsetroundjoin%
\definecolor{currentfill}{rgb}{0.423529,0.627451,0.862745}%
\pgfsetfillcolor{currentfill}%
\pgfsetfillopacity{0.500000}%
\pgfsetlinewidth{1.003750pt}%
\definecolor{currentstroke}{rgb}{0.423529,0.627451,0.862745}%
\pgfsetstrokecolor{currentstroke}%
\pgfsetstrokeopacity{0.500000}%
\pgfsetdash{}{0pt}%
\pgfpathmoveto{\pgfqpoint{1.745788in}{1.263219in}}%
\pgfpathlineto{\pgfqpoint{1.804714in}{1.322144in}}%
\pgfpathlineto{\pgfqpoint{1.745788in}{1.381070in}}%
\pgfpathlineto{\pgfqpoint{1.686862in}{1.322144in}}%
\pgfpathclose%
\pgfusepath{stroke,fill}%
\end{pgfscope}%
\begin{pgfscope}%
\pgfpathrectangle{\pgfqpoint{0.424692in}{0.370679in}}{\pgfqpoint{2.725308in}{1.479321in}}%
\pgfusepath{clip}%
\pgfsetbuttcap%
\pgfsetroundjoin%
\definecolor{currentfill}{rgb}{0.423529,0.627451,0.862745}%
\pgfsetfillcolor{currentfill}%
\pgfsetfillopacity{0.500000}%
\pgfsetlinewidth{1.003750pt}%
\definecolor{currentstroke}{rgb}{0.423529,0.627451,0.862745}%
\pgfsetstrokecolor{currentstroke}%
\pgfsetstrokeopacity{0.500000}%
\pgfsetdash{}{0pt}%
\pgfpathmoveto{\pgfqpoint{1.969740in}{1.294864in}}%
\pgfpathlineto{\pgfqpoint{2.028665in}{1.353790in}}%
\pgfpathlineto{\pgfqpoint{1.969740in}{1.412716in}}%
\pgfpathlineto{\pgfqpoint{1.910814in}{1.353790in}}%
\pgfpathclose%
\pgfusepath{stroke,fill}%
\end{pgfscope}%
\begin{pgfscope}%
\pgfpathrectangle{\pgfqpoint{0.424692in}{0.370679in}}{\pgfqpoint{2.725308in}{1.479321in}}%
\pgfusepath{clip}%
\pgfsetbuttcap%
\pgfsetroundjoin%
\definecolor{currentfill}{rgb}{0.423529,0.627451,0.862745}%
\pgfsetfillcolor{currentfill}%
\pgfsetfillopacity{0.500000}%
\pgfsetlinewidth{1.003750pt}%
\definecolor{currentstroke}{rgb}{0.423529,0.627451,0.862745}%
\pgfsetstrokecolor{currentstroke}%
\pgfsetstrokeopacity{0.500000}%
\pgfsetdash{}{0pt}%
\pgfpathmoveto{\pgfqpoint{1.493020in}{1.150447in}}%
\pgfpathlineto{\pgfqpoint{1.551946in}{1.209372in}}%
\pgfpathlineto{\pgfqpoint{1.493020in}{1.268298in}}%
\pgfpathlineto{\pgfqpoint{1.434095in}{1.209372in}}%
\pgfpathclose%
\pgfusepath{stroke,fill}%
\end{pgfscope}%
\begin{pgfscope}%
\pgfpathrectangle{\pgfqpoint{0.424692in}{0.370679in}}{\pgfqpoint{2.725308in}{1.479321in}}%
\pgfusepath{clip}%
\pgfsetbuttcap%
\pgfsetroundjoin%
\definecolor{currentfill}{rgb}{0.423529,0.627451,0.862745}%
\pgfsetfillcolor{currentfill}%
\pgfsetfillopacity{0.500000}%
\pgfsetlinewidth{1.003750pt}%
\definecolor{currentstroke}{rgb}{0.423529,0.627451,0.862745}%
\pgfsetstrokecolor{currentstroke}%
\pgfsetstrokeopacity{0.500000}%
\pgfsetdash{}{0pt}%
\pgfpathmoveto{\pgfqpoint{1.612690in}{1.240986in}}%
\pgfpathlineto{\pgfqpoint{1.671615in}{1.299911in}}%
\pgfpathlineto{\pgfqpoint{1.612690in}{1.358837in}}%
\pgfpathlineto{\pgfqpoint{1.553764in}{1.299911in}}%
\pgfpathclose%
\pgfusepath{stroke,fill}%
\end{pgfscope}%
\begin{pgfscope}%
\pgfpathrectangle{\pgfqpoint{0.424692in}{0.370679in}}{\pgfqpoint{2.725308in}{1.479321in}}%
\pgfusepath{clip}%
\pgfsetbuttcap%
\pgfsetroundjoin%
\definecolor{currentfill}{rgb}{0.423529,0.627451,0.862745}%
\pgfsetfillcolor{currentfill}%
\pgfsetfillopacity{0.500000}%
\pgfsetlinewidth{1.003750pt}%
\definecolor{currentstroke}{rgb}{0.423529,0.627451,0.862745}%
\pgfsetstrokecolor{currentstroke}%
\pgfsetstrokeopacity{0.500000}%
\pgfsetdash{}{0pt}%
\pgfpathmoveto{\pgfqpoint{1.482790in}{1.122146in}}%
\pgfpathlineto{\pgfqpoint{1.541716in}{1.181071in}}%
\pgfpathlineto{\pgfqpoint{1.482790in}{1.239997in}}%
\pgfpathlineto{\pgfqpoint{1.423865in}{1.181071in}}%
\pgfpathclose%
\pgfusepath{stroke,fill}%
\end{pgfscope}%
\begin{pgfscope}%
\pgfpathrectangle{\pgfqpoint{0.424692in}{0.370679in}}{\pgfqpoint{2.725308in}{1.479321in}}%
\pgfusepath{clip}%
\pgfsetbuttcap%
\pgfsetroundjoin%
\definecolor{currentfill}{rgb}{0.423529,0.627451,0.862745}%
\pgfsetfillcolor{currentfill}%
\pgfsetfillopacity{0.500000}%
\pgfsetlinewidth{1.003750pt}%
\definecolor{currentstroke}{rgb}{0.423529,0.627451,0.862745}%
\pgfsetstrokecolor{currentstroke}%
\pgfsetstrokeopacity{0.500000}%
\pgfsetdash{}{0pt}%
\pgfpathmoveto{\pgfqpoint{2.240894in}{1.472623in}}%
\pgfpathlineto{\pgfqpoint{2.299820in}{1.531548in}}%
\pgfpathlineto{\pgfqpoint{2.240894in}{1.590474in}}%
\pgfpathlineto{\pgfqpoint{2.181969in}{1.531548in}}%
\pgfpathclose%
\pgfusepath{stroke,fill}%
\end{pgfscope}%
\begin{pgfscope}%
\pgfpathrectangle{\pgfqpoint{0.424692in}{0.370679in}}{\pgfqpoint{2.725308in}{1.479321in}}%
\pgfusepath{clip}%
\pgfsetbuttcap%
\pgfsetroundjoin%
\definecolor{currentfill}{rgb}{0.423529,0.627451,0.862745}%
\pgfsetfillcolor{currentfill}%
\pgfsetfillopacity{0.500000}%
\pgfsetlinewidth{1.003750pt}%
\definecolor{currentstroke}{rgb}{0.423529,0.627451,0.862745}%
\pgfsetstrokecolor{currentstroke}%
\pgfsetstrokeopacity{0.500000}%
\pgfsetdash{}{0pt}%
\pgfpathmoveto{\pgfqpoint{1.524585in}{1.223572in}}%
\pgfpathlineto{\pgfqpoint{1.583511in}{1.282497in}}%
\pgfpathlineto{\pgfqpoint{1.524585in}{1.341423in}}%
\pgfpathlineto{\pgfqpoint{1.465660in}{1.282497in}}%
\pgfpathclose%
\pgfusepath{stroke,fill}%
\end{pgfscope}%
\begin{pgfscope}%
\pgfpathrectangle{\pgfqpoint{0.424692in}{0.370679in}}{\pgfqpoint{2.725308in}{1.479321in}}%
\pgfusepath{clip}%
\pgfsetbuttcap%
\pgfsetroundjoin%
\definecolor{currentfill}{rgb}{0.423529,0.627451,0.862745}%
\pgfsetfillcolor{currentfill}%
\pgfsetfillopacity{0.500000}%
\pgfsetlinewidth{1.003750pt}%
\definecolor{currentstroke}{rgb}{0.423529,0.627451,0.862745}%
\pgfsetstrokecolor{currentstroke}%
\pgfsetstrokeopacity{0.500000}%
\pgfsetdash{}{0pt}%
\pgfpathmoveto{\pgfqpoint{1.518435in}{1.202558in}}%
\pgfpathlineto{\pgfqpoint{1.577360in}{1.261484in}}%
\pgfpathlineto{\pgfqpoint{1.518435in}{1.320409in}}%
\pgfpathlineto{\pgfqpoint{1.459509in}{1.261484in}}%
\pgfpathclose%
\pgfusepath{stroke,fill}%
\end{pgfscope}%
\begin{pgfscope}%
\pgfpathrectangle{\pgfqpoint{0.424692in}{0.370679in}}{\pgfqpoint{2.725308in}{1.479321in}}%
\pgfusepath{clip}%
\pgfsetbuttcap%
\pgfsetroundjoin%
\definecolor{currentfill}{rgb}{0.423529,0.627451,0.862745}%
\pgfsetfillcolor{currentfill}%
\pgfsetfillopacity{0.500000}%
\pgfsetlinewidth{1.003750pt}%
\definecolor{currentstroke}{rgb}{0.423529,0.627451,0.862745}%
\pgfsetstrokecolor{currentstroke}%
\pgfsetstrokeopacity{0.500000}%
\pgfsetdash{}{0pt}%
\pgfpathmoveto{\pgfqpoint{1.643672in}{1.212417in}}%
\pgfpathlineto{\pgfqpoint{1.702598in}{1.271343in}}%
\pgfpathlineto{\pgfqpoint{1.643672in}{1.330268in}}%
\pgfpathlineto{\pgfqpoint{1.584747in}{1.271343in}}%
\pgfpathclose%
\pgfusepath{stroke,fill}%
\end{pgfscope}%
\begin{pgfscope}%
\pgfpathrectangle{\pgfqpoint{0.424692in}{0.370679in}}{\pgfqpoint{2.725308in}{1.479321in}}%
\pgfusepath{clip}%
\pgfsetbuttcap%
\pgfsetroundjoin%
\definecolor{currentfill}{rgb}{0.423529,0.627451,0.862745}%
\pgfsetfillcolor{currentfill}%
\pgfsetfillopacity{0.500000}%
\pgfsetlinewidth{1.003750pt}%
\definecolor{currentstroke}{rgb}{0.423529,0.627451,0.862745}%
\pgfsetstrokecolor{currentstroke}%
\pgfsetstrokeopacity{0.500000}%
\pgfsetdash{}{0pt}%
\pgfpathmoveto{\pgfqpoint{2.087130in}{1.402582in}}%
\pgfpathlineto{\pgfqpoint{2.146056in}{1.461508in}}%
\pgfpathlineto{\pgfqpoint{2.087130in}{1.520433in}}%
\pgfpathlineto{\pgfqpoint{2.028205in}{1.461508in}}%
\pgfpathclose%
\pgfusepath{stroke,fill}%
\end{pgfscope}%
\begin{pgfscope}%
\pgfpathrectangle{\pgfqpoint{0.424692in}{0.370679in}}{\pgfqpoint{2.725308in}{1.479321in}}%
\pgfusepath{clip}%
\pgfsetbuttcap%
\pgfsetroundjoin%
\definecolor{currentfill}{rgb}{0.423529,0.627451,0.862745}%
\pgfsetfillcolor{currentfill}%
\pgfsetfillopacity{0.500000}%
\pgfsetlinewidth{1.003750pt}%
\definecolor{currentstroke}{rgb}{0.423529,0.627451,0.862745}%
\pgfsetstrokecolor{currentstroke}%
\pgfsetstrokeopacity{0.500000}%
\pgfsetdash{}{0pt}%
\pgfpathmoveto{\pgfqpoint{1.736121in}{1.240671in}}%
\pgfpathlineto{\pgfqpoint{1.795047in}{1.299596in}}%
\pgfpathlineto{\pgfqpoint{1.736121in}{1.358522in}}%
\pgfpathlineto{\pgfqpoint{1.677196in}{1.299596in}}%
\pgfpathclose%
\pgfusepath{stroke,fill}%
\end{pgfscope}%
\begin{pgfscope}%
\pgfpathrectangle{\pgfqpoint{0.424692in}{0.370679in}}{\pgfqpoint{2.725308in}{1.479321in}}%
\pgfusepath{clip}%
\pgfsetbuttcap%
\pgfsetroundjoin%
\definecolor{currentfill}{rgb}{0.423529,0.627451,0.862745}%
\pgfsetfillcolor{currentfill}%
\pgfsetfillopacity{0.500000}%
\pgfsetlinewidth{1.003750pt}%
\definecolor{currentstroke}{rgb}{0.423529,0.627451,0.862745}%
\pgfsetstrokecolor{currentstroke}%
\pgfsetstrokeopacity{0.500000}%
\pgfsetdash{}{0pt}%
\pgfpathmoveto{\pgfqpoint{1.701420in}{1.238669in}}%
\pgfpathlineto{\pgfqpoint{1.760346in}{1.297595in}}%
\pgfpathlineto{\pgfqpoint{1.701420in}{1.356520in}}%
\pgfpathlineto{\pgfqpoint{1.642495in}{1.297595in}}%
\pgfpathclose%
\pgfusepath{stroke,fill}%
\end{pgfscope}%
\begin{pgfscope}%
\pgfpathrectangle{\pgfqpoint{0.424692in}{0.370679in}}{\pgfqpoint{2.725308in}{1.479321in}}%
\pgfusepath{clip}%
\pgfsetbuttcap%
\pgfsetroundjoin%
\definecolor{currentfill}{rgb}{0.423529,0.627451,0.862745}%
\pgfsetfillcolor{currentfill}%
\pgfsetfillopacity{0.500000}%
\pgfsetlinewidth{1.003750pt}%
\definecolor{currentstroke}{rgb}{0.423529,0.627451,0.862745}%
\pgfsetstrokecolor{currentstroke}%
\pgfsetstrokeopacity{0.500000}%
\pgfsetdash{}{0pt}%
\pgfpathmoveto{\pgfqpoint{2.034472in}{1.357216in}}%
\pgfpathlineto{\pgfqpoint{2.093398in}{1.416141in}}%
\pgfpathlineto{\pgfqpoint{2.034472in}{1.475067in}}%
\pgfpathlineto{\pgfqpoint{1.975547in}{1.416141in}}%
\pgfpathclose%
\pgfusepath{stroke,fill}%
\end{pgfscope}%
\begin{pgfscope}%
\pgfsetbuttcap%
\pgfsetroundjoin%
\definecolor{currentfill}{rgb}{0.000000,0.000000,0.000000}%
\pgfsetfillcolor{currentfill}%
\pgfsetlinewidth{0.803000pt}%
\definecolor{currentstroke}{rgb}{0.000000,0.000000,0.000000}%
\pgfsetstrokecolor{currentstroke}%
\pgfsetdash{}{0pt}%
\pgfsys@defobject{currentmarker}{\pgfqpoint{0.000000in}{-0.048611in}}{\pgfqpoint{0.000000in}{0.000000in}}{%
\pgfpathmoveto{\pgfqpoint{0.000000in}{0.000000in}}%
\pgfpathlineto{\pgfqpoint{0.000000in}{-0.048611in}}%
\pgfusepath{stroke,fill}%
}%
\begin{pgfscope}%
\pgfsys@transformshift{0.627278in}{0.370679in}%
\pgfsys@useobject{currentmarker}{}%
\end{pgfscope}%
\end{pgfscope}%
\begin{pgfscope}%
\definecolor{textcolor}{rgb}{0.000000,0.000000,0.000000}%
\pgfsetstrokecolor{textcolor}%
\pgfsetfillcolor{textcolor}%
\pgftext[x=0.627278in,y=0.273457in,,top]{\color{textcolor}\fontsize{10.000000}{12.000000}\selectfont \(\displaystyle -2\)}%
\end{pgfscope}%
\begin{pgfscope}%
\pgfsetbuttcap%
\pgfsetroundjoin%
\definecolor{currentfill}{rgb}{0.000000,0.000000,0.000000}%
\pgfsetfillcolor{currentfill}%
\pgfsetlinewidth{0.803000pt}%
\definecolor{currentstroke}{rgb}{0.000000,0.000000,0.000000}%
\pgfsetstrokecolor{currentstroke}%
\pgfsetdash{}{0pt}%
\pgfsys@defobject{currentmarker}{\pgfqpoint{0.000000in}{-0.048611in}}{\pgfqpoint{0.000000in}{0.000000in}}{%
\pgfpathmoveto{\pgfqpoint{0.000000in}{0.000000in}}%
\pgfpathlineto{\pgfqpoint{0.000000in}{-0.048611in}}%
\pgfusepath{stroke,fill}%
}%
\begin{pgfscope}%
\pgfsys@transformshift{1.288561in}{0.370679in}%
\pgfsys@useobject{currentmarker}{}%
\end{pgfscope}%
\end{pgfscope}%
\begin{pgfscope}%
\definecolor{textcolor}{rgb}{0.000000,0.000000,0.000000}%
\pgfsetstrokecolor{textcolor}%
\pgfsetfillcolor{textcolor}%
\pgftext[x=1.288561in,y=0.273457in,,top]{\color{textcolor}\fontsize{10.000000}{12.000000}\selectfont \(\displaystyle -1\)}%
\end{pgfscope}%
\begin{pgfscope}%
\pgfsetbuttcap%
\pgfsetroundjoin%
\definecolor{currentfill}{rgb}{0.000000,0.000000,0.000000}%
\pgfsetfillcolor{currentfill}%
\pgfsetlinewidth{0.803000pt}%
\definecolor{currentstroke}{rgb}{0.000000,0.000000,0.000000}%
\pgfsetstrokecolor{currentstroke}%
\pgfsetdash{}{0pt}%
\pgfsys@defobject{currentmarker}{\pgfqpoint{0.000000in}{-0.048611in}}{\pgfqpoint{0.000000in}{0.000000in}}{%
\pgfpathmoveto{\pgfqpoint{0.000000in}{0.000000in}}%
\pgfpathlineto{\pgfqpoint{0.000000in}{-0.048611in}}%
\pgfusepath{stroke,fill}%
}%
\begin{pgfscope}%
\pgfsys@transformshift{1.949844in}{0.370679in}%
\pgfsys@useobject{currentmarker}{}%
\end{pgfscope}%
\end{pgfscope}%
\begin{pgfscope}%
\definecolor{textcolor}{rgb}{0.000000,0.000000,0.000000}%
\pgfsetstrokecolor{textcolor}%
\pgfsetfillcolor{textcolor}%
\pgftext[x=1.949844in,y=0.273457in,,top]{\color{textcolor}\fontsize{10.000000}{12.000000}\selectfont \(\displaystyle 0\)}%
\end{pgfscope}%
\begin{pgfscope}%
\pgfsetbuttcap%
\pgfsetroundjoin%
\definecolor{currentfill}{rgb}{0.000000,0.000000,0.000000}%
\pgfsetfillcolor{currentfill}%
\pgfsetlinewidth{0.803000pt}%
\definecolor{currentstroke}{rgb}{0.000000,0.000000,0.000000}%
\pgfsetstrokecolor{currentstroke}%
\pgfsetdash{}{0pt}%
\pgfsys@defobject{currentmarker}{\pgfqpoint{0.000000in}{-0.048611in}}{\pgfqpoint{0.000000in}{0.000000in}}{%
\pgfpathmoveto{\pgfqpoint{0.000000in}{0.000000in}}%
\pgfpathlineto{\pgfqpoint{0.000000in}{-0.048611in}}%
\pgfusepath{stroke,fill}%
}%
\begin{pgfscope}%
\pgfsys@transformshift{2.611128in}{0.370679in}%
\pgfsys@useobject{currentmarker}{}%
\end{pgfscope}%
\end{pgfscope}%
\begin{pgfscope}%
\definecolor{textcolor}{rgb}{0.000000,0.000000,0.000000}%
\pgfsetstrokecolor{textcolor}%
\pgfsetfillcolor{textcolor}%
\pgftext[x=2.611128in,y=0.273457in,,top]{\color{textcolor}\fontsize{10.000000}{12.000000}\selectfont \(\displaystyle 1\)}%
\end{pgfscope}%
\begin{pgfscope}%
\pgfsetbuttcap%
\pgfsetroundjoin%
\definecolor{currentfill}{rgb}{0.000000,0.000000,0.000000}%
\pgfsetfillcolor{currentfill}%
\pgfsetlinewidth{0.803000pt}%
\definecolor{currentstroke}{rgb}{0.000000,0.000000,0.000000}%
\pgfsetstrokecolor{currentstroke}%
\pgfsetdash{}{0pt}%
\pgfsys@defobject{currentmarker}{\pgfqpoint{-0.048611in}{0.000000in}}{\pgfqpoint{0.000000in}{0.000000in}}{%
\pgfpathmoveto{\pgfqpoint{0.000000in}{0.000000in}}%
\pgfpathlineto{\pgfqpoint{-0.048611in}{0.000000in}}%
\pgfusepath{stroke,fill}%
}%
\begin{pgfscope}%
\pgfsys@transformshift{0.424692in}{0.415001in}%
\pgfsys@useobject{currentmarker}{}%
\end{pgfscope}%
\end{pgfscope}%
\begin{pgfscope}%
\definecolor{textcolor}{rgb}{0.000000,0.000000,0.000000}%
\pgfsetstrokecolor{textcolor}%
\pgfsetfillcolor{textcolor}%
\pgftext[x=0.150000in,y=0.366775in,left,base]{\color{textcolor}\fontsize{10.000000}{12.000000}\selectfont \(\displaystyle -2\)}%
\end{pgfscope}%
\begin{pgfscope}%
\pgfsetbuttcap%
\pgfsetroundjoin%
\definecolor{currentfill}{rgb}{0.000000,0.000000,0.000000}%
\pgfsetfillcolor{currentfill}%
\pgfsetlinewidth{0.803000pt}%
\definecolor{currentstroke}{rgb}{0.000000,0.000000,0.000000}%
\pgfsetstrokecolor{currentstroke}%
\pgfsetdash{}{0pt}%
\pgfsys@defobject{currentmarker}{\pgfqpoint{-0.048611in}{0.000000in}}{\pgfqpoint{0.000000in}{0.000000in}}{%
\pgfpathmoveto{\pgfqpoint{0.000000in}{0.000000in}}%
\pgfpathlineto{\pgfqpoint{-0.048611in}{0.000000in}}%
\pgfusepath{stroke,fill}%
}%
\begin{pgfscope}%
\pgfsys@transformshift{0.424692in}{0.807241in}%
\pgfsys@useobject{currentmarker}{}%
\end{pgfscope}%
\end{pgfscope}%
\begin{pgfscope}%
\definecolor{textcolor}{rgb}{0.000000,0.000000,0.000000}%
\pgfsetstrokecolor{textcolor}%
\pgfsetfillcolor{textcolor}%
\pgftext[x=0.150000in,y=0.759016in,left,base]{\color{textcolor}\fontsize{10.000000}{12.000000}\selectfont \(\displaystyle -1\)}%
\end{pgfscope}%
\begin{pgfscope}%
\pgfsetbuttcap%
\pgfsetroundjoin%
\definecolor{currentfill}{rgb}{0.000000,0.000000,0.000000}%
\pgfsetfillcolor{currentfill}%
\pgfsetlinewidth{0.803000pt}%
\definecolor{currentstroke}{rgb}{0.000000,0.000000,0.000000}%
\pgfsetstrokecolor{currentstroke}%
\pgfsetdash{}{0pt}%
\pgfsys@defobject{currentmarker}{\pgfqpoint{-0.048611in}{0.000000in}}{\pgfqpoint{0.000000in}{0.000000in}}{%
\pgfpathmoveto{\pgfqpoint{0.000000in}{0.000000in}}%
\pgfpathlineto{\pgfqpoint{-0.048611in}{0.000000in}}%
\pgfusepath{stroke,fill}%
}%
\begin{pgfscope}%
\pgfsys@transformshift{0.424692in}{1.199482in}%
\pgfsys@useobject{currentmarker}{}%
\end{pgfscope}%
\end{pgfscope}%
\begin{pgfscope}%
\definecolor{textcolor}{rgb}{0.000000,0.000000,0.000000}%
\pgfsetstrokecolor{textcolor}%
\pgfsetfillcolor{textcolor}%
\pgftext[x=0.258025in,y=1.151257in,left,base]{\color{textcolor}\fontsize{10.000000}{12.000000}\selectfont \(\displaystyle 0\)}%
\end{pgfscope}%
\begin{pgfscope}%
\pgfsetbuttcap%
\pgfsetroundjoin%
\definecolor{currentfill}{rgb}{0.000000,0.000000,0.000000}%
\pgfsetfillcolor{currentfill}%
\pgfsetlinewidth{0.803000pt}%
\definecolor{currentstroke}{rgb}{0.000000,0.000000,0.000000}%
\pgfsetstrokecolor{currentstroke}%
\pgfsetdash{}{0pt}%
\pgfsys@defobject{currentmarker}{\pgfqpoint{-0.048611in}{0.000000in}}{\pgfqpoint{0.000000in}{0.000000in}}{%
\pgfpathmoveto{\pgfqpoint{0.000000in}{0.000000in}}%
\pgfpathlineto{\pgfqpoint{-0.048611in}{0.000000in}}%
\pgfusepath{stroke,fill}%
}%
\begin{pgfscope}%
\pgfsys@transformshift{0.424692in}{1.591723in}%
\pgfsys@useobject{currentmarker}{}%
\end{pgfscope}%
\end{pgfscope}%
\begin{pgfscope}%
\definecolor{textcolor}{rgb}{0.000000,0.000000,0.000000}%
\pgfsetstrokecolor{textcolor}%
\pgfsetfillcolor{textcolor}%
\pgftext[x=0.258025in,y=1.543497in,left,base]{\color{textcolor}\fontsize{10.000000}{12.000000}\selectfont \(\displaystyle 1\)}%
\end{pgfscope}%
\begin{pgfscope}%
\pgfsetrectcap%
\pgfsetmiterjoin%
\pgfsetlinewidth{0.803000pt}%
\definecolor{currentstroke}{rgb}{0.000000,0.000000,0.000000}%
\pgfsetstrokecolor{currentstroke}%
\pgfsetdash{}{0pt}%
\pgfpathmoveto{\pgfqpoint{0.424692in}{0.370679in}}%
\pgfpathlineto{\pgfqpoint{0.424692in}{1.850000in}}%
\pgfusepath{stroke}%
\end{pgfscope}%
\begin{pgfscope}%
\pgfsetrectcap%
\pgfsetmiterjoin%
\pgfsetlinewidth{0.803000pt}%
\definecolor{currentstroke}{rgb}{0.000000,0.000000,0.000000}%
\pgfsetstrokecolor{currentstroke}%
\pgfsetdash{}{0pt}%
\pgfpathmoveto{\pgfqpoint{0.424692in}{0.370679in}}%
\pgfpathlineto{\pgfqpoint{3.150000in}{0.370679in}}%
\pgfusepath{stroke}%
\end{pgfscope}%
\begin{pgfscope}%
\pgfsetbuttcap%
\pgfsetmiterjoin%
\definecolor{currentfill}{rgb}{1.000000,1.000000,1.000000}%
\pgfsetfillcolor{currentfill}%
\pgfsetfillopacity{0.800000}%
\pgfsetlinewidth{1.003750pt}%
\definecolor{currentstroke}{rgb}{0.800000,0.800000,0.800000}%
\pgfsetstrokecolor{currentstroke}%
\pgfsetstrokeopacity{0.800000}%
\pgfsetdash{}{0pt}%
\pgfpathmoveto{\pgfqpoint{2.023235in}{0.426234in}}%
\pgfpathlineto{\pgfqpoint{3.072222in}{0.426234in}}%
\pgfpathquadraticcurveto{\pgfqpoint{3.094444in}{0.426234in}}{\pgfqpoint{3.094444in}{0.448457in}}%
\pgfpathlineto{\pgfqpoint{3.094444in}{0.902161in}}%
\pgfpathquadraticcurveto{\pgfqpoint{3.094444in}{0.924383in}}{\pgfqpoint{3.072222in}{0.924383in}}%
\pgfpathlineto{\pgfqpoint{2.023235in}{0.924383in}}%
\pgfpathquadraticcurveto{\pgfqpoint{2.001012in}{0.924383in}}{\pgfqpoint{2.001012in}{0.902161in}}%
\pgfpathlineto{\pgfqpoint{2.001012in}{0.448457in}}%
\pgfpathquadraticcurveto{\pgfqpoint{2.001012in}{0.426234in}}{\pgfqpoint{2.023235in}{0.426234in}}%
\pgfpathclose%
\pgfusepath{stroke,fill}%
\end{pgfscope}%
\begin{pgfscope}%
\pgfsetbuttcap%
\pgfsetroundjoin%
\definecolor{currentfill}{rgb}{1.000000,0.411765,0.380392}%
\pgfsetfillcolor{currentfill}%
\pgfsetfillopacity{0.500000}%
\pgfsetlinewidth{1.003750pt}%
\definecolor{currentstroke}{rgb}{1.000000,0.411765,0.380392}%
\pgfsetstrokecolor{currentstroke}%
\pgfsetstrokeopacity{0.500000}%
\pgfsetdash{}{0pt}%
\pgfpathmoveto{\pgfqpoint{2.156568in}{0.789661in}}%
\pgfpathcurveto{\pgfqpoint{2.167618in}{0.789661in}}{\pgfqpoint{2.178217in}{0.794051in}}{\pgfqpoint{2.186031in}{0.801864in}}%
\pgfpathcurveto{\pgfqpoint{2.193844in}{0.809678in}}{\pgfqpoint{2.198235in}{0.820277in}}{\pgfqpoint{2.198235in}{0.831327in}}%
\pgfpathcurveto{\pgfqpoint{2.198235in}{0.842377in}}{\pgfqpoint{2.193844in}{0.852976in}}{\pgfqpoint{2.186031in}{0.860790in}}%
\pgfpathcurveto{\pgfqpoint{2.178217in}{0.868604in}}{\pgfqpoint{2.167618in}{0.872994in}}{\pgfqpoint{2.156568in}{0.872994in}}%
\pgfpathcurveto{\pgfqpoint{2.145518in}{0.872994in}}{\pgfqpoint{2.134919in}{0.868604in}}{\pgfqpoint{2.127105in}{0.860790in}}%
\pgfpathcurveto{\pgfqpoint{2.119292in}{0.852976in}}{\pgfqpoint{2.114901in}{0.842377in}}{\pgfqpoint{2.114901in}{0.831327in}}%
\pgfpathcurveto{\pgfqpoint{2.114901in}{0.820277in}}{\pgfqpoint{2.119292in}{0.809678in}}{\pgfqpoint{2.127105in}{0.801864in}}%
\pgfpathcurveto{\pgfqpoint{2.134919in}{0.794051in}}{\pgfqpoint{2.145518in}{0.789661in}}{\pgfqpoint{2.156568in}{0.789661in}}%
\pgfpathclose%
\pgfusepath{stroke,fill}%
\end{pgfscope}%
\begin{pgfscope}%
\definecolor{textcolor}{rgb}{0.000000,0.000000,0.000000}%
\pgfsetstrokecolor{textcolor}%
\pgfsetfillcolor{textcolor}%
\pgftext[x=2.356568in,y=0.802161in,left,base]{\color{textcolor}\fontsize{8.000000}{9.600000}\selectfont Iris-setosa}%
\end{pgfscope}%
\begin{pgfscope}%
\pgfsetbuttcap%
\pgfsetroundjoin%
\definecolor{currentfill}{rgb}{0.466667,0.866667,0.466667}%
\pgfsetfillcolor{currentfill}%
\pgfsetfillopacity{0.500000}%
\pgfsetlinewidth{1.003750pt}%
\definecolor{currentstroke}{rgb}{0.466667,0.866667,0.466667}%
\pgfsetstrokecolor{currentstroke}%
\pgfsetstrokeopacity{0.500000}%
\pgfsetdash{}{0pt}%
\pgfpathmoveto{\pgfqpoint{2.135735in}{0.634722in}}%
\pgfpathlineto{\pgfqpoint{2.156568in}{0.655556in}}%
\pgfpathlineto{\pgfqpoint{2.177401in}{0.634722in}}%
\pgfpathlineto{\pgfqpoint{2.198235in}{0.655556in}}%
\pgfpathlineto{\pgfqpoint{2.177401in}{0.676389in}}%
\pgfpathlineto{\pgfqpoint{2.198235in}{0.697222in}}%
\pgfpathlineto{\pgfqpoint{2.177401in}{0.718056in}}%
\pgfpathlineto{\pgfqpoint{2.156568in}{0.697222in}}%
\pgfpathlineto{\pgfqpoint{2.135735in}{0.718056in}}%
\pgfpathlineto{\pgfqpoint{2.114901in}{0.697222in}}%
\pgfpathlineto{\pgfqpoint{2.135735in}{0.676389in}}%
\pgfpathlineto{\pgfqpoint{2.114901in}{0.655556in}}%
\pgfpathclose%
\pgfusepath{stroke,fill}%
\end{pgfscope}%
\begin{pgfscope}%
\definecolor{textcolor}{rgb}{0.000000,0.000000,0.000000}%
\pgfsetstrokecolor{textcolor}%
\pgfsetfillcolor{textcolor}%
\pgftext[x=2.356568in,y=0.647222in,left,base]{\color{textcolor}\fontsize{8.000000}{9.600000}\selectfont Iris-versicolor}%
\end{pgfscope}%
\begin{pgfscope}%
\pgfsetbuttcap%
\pgfsetroundjoin%
\definecolor{currentfill}{rgb}{0.423529,0.627451,0.862745}%
\pgfsetfillcolor{currentfill}%
\pgfsetfillopacity{0.500000}%
\pgfsetlinewidth{1.003750pt}%
\definecolor{currentstroke}{rgb}{0.423529,0.627451,0.862745}%
\pgfsetstrokecolor{currentstroke}%
\pgfsetstrokeopacity{0.500000}%
\pgfsetdash{}{0pt}%
\pgfpathmoveto{\pgfqpoint{2.156568in}{0.462525in}}%
\pgfpathlineto{\pgfqpoint{2.215493in}{0.521451in}}%
\pgfpathlineto{\pgfqpoint{2.156568in}{0.580376in}}%
\pgfpathlineto{\pgfqpoint{2.097642in}{0.521451in}}%
\pgfpathclose%
\pgfusepath{stroke,fill}%
\end{pgfscope}%
\begin{pgfscope}%
\definecolor{textcolor}{rgb}{0.000000,0.000000,0.000000}%
\pgfsetstrokecolor{textcolor}%
\pgfsetfillcolor{textcolor}%
\pgftext[x=2.356568in,y=0.492284in,left,base]{\color{textcolor}\fontsize{8.000000}{9.600000}\selectfont Iris-virginica}%
\end{pgfscope}%
\end{pgfpicture}%
\makeatother%
\endgroup%
}
    \end{tabular}
    \caption{Rank 2 reduction on the Iris Dataset}
    \label{fig:pca_sg}
\end{figure}
\noindent The groupings of the data factorised by PCA and GD are very similar. It appears that a linear rotation and scaling transformation would map one to the other. A consequence of PCA maximising the variance is that the reconstruction error will be minimised -- this is the loss function being used by the GD. Although the mapped values are not identical, GD is not being applied to a multi-layer perceptron so it can at most learn a linear mapping of PCA.

\section{Exercise 2}
\noindent\textbf{Exercise 2.1:} Multi-Layer Perceptron (MLP).\\
\begin{listing}[H]
    \pythoncode{code/mlp.py}
    \caption{PyTorch MLP Classifier for Iris Dataset}
    \label{lst:mlp}
    \end{listing}
\noindent\textbf{Exercise 2.2:} MLP classification on Iris Dataset.\\
\noindent Below are the median train and validation results for classifying the Iris Dataset across 100 independent trainings. The order of class results are $\begin{bmatrix}\text{Iris setosa}, \text{Iris versicolor}, \text{Iris virginica}\end{bmatrix}$.
\begin{alignat*}{1}
\text{Acc}_\text{train} &= \begin{bmatrix}1.000 & 0.806 & 0.969\end{bmatrix}\quad |\quad \text{Overall} = 0.92\\
\text{Acc}_\text{validation} &= \begin{bmatrix}1.000 & 0.786 & 0.889\end{bmatrix}\quad |\quad \text{Overall} = 0.90
\end{alignat*}
This lines up with the rank reduction which shows Iris setosa is highly separable from the other classes. The results show that differentiating between Iris versicolor and Iris virginica is still not always possible even with the higher rank information. Validation accuracy is less for all groups, indicating either a lack of required diversity in the training set or overfitting.

\end{document}

